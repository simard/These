\documentclass[english,french,dvips,10pt]{mybeamer}
\usepackage{my}
\usepackage{mydate}
\usepackage{mycolor}
\usepackage{myfloat}
\usepackage[biblatex]{mybib}
\usepackage{myps}
\usepackage{myuml}

\makeatletter
% Modify the bibliography style
\newcounter{mymaxcitenames}
\AtBeginDocument{%
	\setcounter{mymaxcitenames}{\value{maxnames}}%
}
\renewbibmacro{begentry}{%
	\printtext[brackets]{%
		\defcounter{maxnames}{\value{mymaxcitenames}}%
		\printnames{labelname}~\usebibmacro{cite:labelyear+extrayear}%
	}%
	\newline%
}
% AAA
\newcommand{\myACER}{\textsc{acer}\xspace}
\newcommand{\myAlanine}{Alanine\xspace}
\newcommand{\myanalysis}[1]{\input{files/#1}\%}
\newcommand{\myangstrom}{\AA ngström\xspace}
\newcommand{\myanova}[1]{\input{files/#1}}
\newcommand{\myatom}[2][]{%
	{%
		\ifstrempty{#1}%
			{\makefirstuc{\textsf{#2}}}%
			{\textcolor{#1}{\makefirstuc{\textsf{#2}}}}%
		\xspace%
	}%
}
\newcommand{\myAudacity}{\textsc{audacity}\myregistered}% No '\xspace' because of already one in '\myregistered'
% CCC
\newcommand{\mycarbon}{\myatom[mycarboncolor]{C}}
\newcommand{\myCasioXJ}{\textsc{Casio xj}\xspace}
\newcommand{\myCHARMM}{\textsc{charmm}\xspace}
\newcommand{\myChimera}{\textsc{chimera}\xspace}
\newcommand{\myClayWorks}{\textsc{Clayworks}\xspace}
\newcommand{\mycondition}[1]{$\left(\mathcal{C}_{#1}\right)$\xspace}
\newcommand{\myCPK}{\textsc{cpk}\xspace}
% DDD
\newcommand{\myDesktop}{\myPHANToM Desktop\myregistered}% No '\xspace' because of already one in '\myregistered'
% FFF
\newcommand{\myfeuillet}{feuillet-$\beta$\xspace}
\WithSuffix\newcommand\myfeuillet*{feuillets-$\beta$\xspace}
\newcommand{\myform}[1]{\textbf{\sffamily\MakeUppercase{#1}}}
% GGG
\newcommand{\myGhost}{\textsc{Ghost}\xspace}
\newcommand{\myGromacs}{\textsc{Gromacs}\xspace}
\newcommand{\mygroup}[1]{$\left(\mathcal{G}_{#1}\right)$\xspace}
% HHH
\newcommand{\myHaption}{\textsc{Haption}\xspace}
\newcommand{\myHawthorne}{\myemph{Hawthorne Works}\xspace}
\newcommand{\myHBonds}{\textit{HBonds}\xspace}
\newcommand{\myhelice}{hélice-$\alpha$\xspace}
\WithSuffix\newcommand\myhelice*{hélices-$\alpha$\xspace}
\newcommand{\myhypothesis}[1]{$\left(\mathcal{H}_{#1}\right)$\xspace}
% III
\newcommand{\myIntelCore}{Intel\myregistered Core\mytrademark~2 \textsc{q9450} (\mynum[GHz]{2.66})\xspace}
% JJJ
\newcommand{\myJmol}{\textsc{Jmol}\xspace}
% LLL
\newcommand{\myLCD}{\textsc{lcd}\xspace}
\newcommand{\myLicorice}{\textit{Licorice}\xspace}
\newcommand{\myLinux}{\textsc{Linux}\xspace}
% MMM
\newcommand{\myMacOS}{Mac~\textsc{OS}\xspace}
\newcommand{\myMDDriver}{\textsc{MDDriver}\xspace}
% NNN
\newcommand{\myNewRibbon}{\textit{NewRibbon}\xspace}
\def\mynode{%
	\@ifnextchar[{\mynode@i}{\mynode@i[style=nodestyle]}%
}
\def\mynode@i[#1](#2,#3)[#4]#5{%
	\rput(#2,#3){\Rnode{#4}{\psframebox[style=nodestyle,#1]{\vphantom{pÉ}#5}}}%
}
\newcommand{\mynytrogen}{\myatom[mynytrogencolor]{A}}
\newcommand{\myNusE}{\textsc{NusE}\xspace}
\newcommand{\myNusENusG}{\textsc{NusE:NusG}\xspace}
\newcommand{\myNusG}{\textsc{NusG}\xspace}
% OOO
\newcommand{\myOmni}{\myPHANToM Omni\myregistered}% No '\xspace' because of already one in '\myregistered'
\newcommand{\myOpenHaptics}{\textsc{OpenHaptics}\mytrademark}% No '\xspace' because of already one in '\mytrademark'
\newcommand{\myoxygen}{\myatom[myoxygencolor]{O}}
% PPP
\newcommand{\myPC}{\textsc{pc}\xspace}
\newcommand{\myPDB}{\textsc{pdb}\xspace}
\newcommand{\myPDBbase}{\emph{Protein~DataBase}\xspace}
\newcommand{\myPDBlink}[2]{\href{#1}{\textsc{\MakeLowercase{#2}}}}
\newcommand{\myPHANToM}{\textsc{phant}o\textsc{m}\xspace}
\newcommand{\myPremium}{\myPHANToM Premium\myregistered}% No '\xspace' because of already one in '\myregistered'
\newcommand{\myPrion}{Prion\xspace}
\newcommand{\myPSF}{\textsc{psf}\xspace}
\newcommand{\mypvalue}{$p$-value\xspace}
\newcommand{\myPyMOL}{\textsc{p}y\textsc{mol}\xspace}
% RRR
\newcommand{\myRAM}[2][Go]{\mynum[#1]{#2} de \textsc{ram}}
\newcommand{\myRasmol}{\textsc{RasMol}\xspace}
\newcommand{\myresidue}[1]{$\left(\mathcal{R}_{#1}\right)$\xspace}
% SSS
\newcommand{\myscenario}[1]{\textsc{#1}}
\newcommand{\mySensAble}{\textsc{SensAble}\xspace}
\newcommand{\myShaddock}{\textsc{Shaddock}\xspace}
\newcommand{\mySony}{\textsc{sony}\myregistered}% No '\xspace' because of already one in '\myregistered'
\newcommand{\mySpaceNavigator}{SpaceNavigator\myregistered}% No '\xspace' because of already one in '\myregistered'
\newcommand{\mysubject}[1]{$\mathcal{S}_#1$}
\newcommand{\mysulfur}{\myatom[mysulfurcolor]{S}}
\newcommand{\mysummary}[1]{\input{files/#1}}
% TTT
\newcommand{\myTCPIP}{\textsc{tcp/ip}\xspace}
\newcommand{\myThreeD}{\textsc{3d}\xspace}
\newcommand{\mytool}[1]{\myemph{#1}}
\newcommand{\myTRPCAGE}{\textsc{trp-cage}\xspace}
\newcommand{\myTRPZIPPER}{\textsc{trp-zipper}\xspace}
% UUU
\newcommand{\myUbiquitin}{Ubiquitin\xspace}
\newcommand{\myUbuntu}{\textsc{Ubuntu}~v$10.04$\xspace}
\newcommand{\myUSB}{\textsc{usb}\xspace}
\newcommand{\myuser}[1]{$\mathcal{#1}$}
% VVV
\newcommand{\myvar}[2]{$\left(\mathcal{V}_{\mathrm{#1}#2}\right)$\xspace}
\newcommand{\myvard}[1]{\myvar{d}{#1}}
\newcommand{\myvari}[1]{\myvar{i}{#1}}
\newcommand{\myVGA}{\textsc{vga}\xspace}
\newcommand{\myVirtuose}{\textsc{Virtuose}\mytrademark~\textsc{6d}\mynum{35}--\mynum{45}\xspace}
% WWW
\newcommand{\myWindows}{\textsc{Windows}\xspace}

% Needed lengths
\newlength{\mywidth}
\newlength{\myheight}

% PSTricks style
\newpsstyle{nodestyle}{framearc=0.25,shadow=true,shadowcolor=myblue,blur=true}
\makeatother

%\includeonlyframes{current}

\hypersetup{%
	pdftitle={Collaboration haptique étroitement couplée pour la déformation moléculaire interactive},%
	pdfauthor={Jean SIMARD},%
	pdfkeywords={collaboration,haptique,environnement virtuel,simulation moléculaire},%
	pdflang={FR-fr},%
	pdfsubject={Soutenance de thèse en informatique}%
}

\newcommand{\schemafactor}{1}
\newlength{\schemaunit}

\newenvironment<>{myblock}[1]{%
	\begin{block}{#1}%
		\setbeamercolor{local structure}{fg=myblue!75}
}{%
	\end{block}%
}
\newenvironment<>{myplusblock}[1]{%
	\begin{exampleblock}{#1}%
		\setbeamercolor{local structure}{fg=mygreen!75}
}{%
	\end{exampleblock}%
}
\newenvironment<>{myminusblock}[1]{%
	\begin{alertblock}{#1}%
		\setbeamercolor{local structure}{fg=myred!75}
}{%
	\end{alertblock}%
}

\addglobalbib[datatype=bibtex]{biblio.bib}

\title{Collaboration haptique étroitement couplée pour la déformation moléculaire interactive}
\author[J. \myname{Simard}]{Jean \myname{Simard}}
\institute{\textsc{cnrs-limsi}}
\school{Université de \myname{Paris}-Sud}
\logo{%
	\myimage[height=1cm]{logo-ups}%
	\hspace{1em}%
	\myimage[height=1cm]{logo-limsi}%
}
\date{\mydate[datestyle=long]{12/03/2012}}
\newenvironment<>{myframe}[2][environment=myframe]{%
	\iflanguage{french}{%
		\shorthandoff{!}%
	}{}%
	\begin{frame}#3[#1]{#2}%
}{%
	\end{frame}%
	\iflanguage{french}{%
		\shorthandon{!}%
	}{}%
}

\begin{document}
	\transfade[duration=0.1]
	\begin{myframe}
		\titlepage
	\end{myframe}
	\logo{}
	\begin{myframe}{Sommaire}
		\tableofcontents[hideallsubsections]
	\end{myframe}
	\section{Introduction}
	\begin{myframe}{Sommaire}
		\tableofcontents[sectionstyle=show/shaded,subsectionstyle=show/shaded/hide,subsubsectionstyle=show/show/hide]
	\end{myframe}
	\subsection{\protect\textit{Docking} moléculaire}
	\begin{myframe}{\protect\textit{Docking} moléculaire}
		\begin{myplusblock}{Définition}%
			ou \myemph{amarrage moléculaire}, consiste à trouver l'orientation et la conformation optimale permettant d'assembler \mynum{2}~molécules.%
		\end{myplusblock}%

		\begin{columns}[T]%
			\begin{column}{0.6\textwidth}%
				\begin{myfigure}%
					\psset{xunit=0.045\textwidth,yunit=0.4cm}%
					\begin{myps}(-10,-5)(10,5)%
						\only<1-3>{%
							\rput(-5,3){\myimage[width=0.15\paperwidth]{sota-docking-deformed-A}}%
							\only<1-2>{%
								\rput(-5,-3){\myimage[width=0.15\paperwidth,angle=180]{sota-docking-deformed-B}}%
							}%
							\only<3>{%
								\rput(-5,-3){\myimage[width=0.15\paperwidth]{sota-docking-deformed-B}}%
							}%
						}%
						\only<2-3>{%
							\rput(5,2.5){\mymolAd}%
							\only<2>{%
								\rput{180}(5,-2.5){\mymolBd}%
								\psarcn[linewidth=2pt,linecolor=mygreen,arrowsize=2pt 4,arrowlength=0.75]{->}(5,-2.5){0.3}{90}{135}%
								\uput{2ex}[-90](5,-2.5){\textcolor{mygreen}{réorientation}}%
							}%
							\only<3>{%
								\rput(5,-2.5){\mymolBd}%
								\rput(5,0){\rnode{deformation}{\vphantom{pÉ}\textcolor{mygreen}{déformation}}}%
								\pnode(6,2.5){A}
								\pnode(4,-2.5){B}
								\pnode(7.5,-2){C}
								\ncline[linecolor=mygreen]{->}{deformation}{A}
								\ncline[linecolor=mygreen]{->}{deformation}{B}
								\ncline[linecolor=mygreen]{->}{deformation}{C}
							}%
						}%
						\only<4-5>{%
							\rput(-5,3){\myimage[width=0.15\paperwidth]{sota-docking-molecule-A}}%
							\rput(-5,-3){\myimage[width=0.15\paperwidth]{sota-docking-molecule-B}}%
							\rput(5,2.5){%
								\mymolA%
								\only<4>{%
									\rput(0,2.5){\textcolor{mygreen}{propriétés chimiques}}%
								}
								\only<5>{%
									\rput(-3,0.5){\textbf{\textcolor{white}{+}}}%
									\rput(-0.5,0.75){\textbf{\textcolor{white}{--}}}%
									\rput(2,1.5){\textbf{\textcolor{white}{+}}}%
									\psline[linewidth=4pt,linecolor=mygreen]{->}(0,-0.25)(0,-2)%
								}%
							}%
							\rput(5,-2.5){%
								\mymolB%
								\only<4>{%
									\rput(0,-2.5){\textcolor{mygreen}{propriétés chimiques}}%
								}
								\only<5>{%
									\rput(-3,-1){\textbf{\textcolor{white}{--}}}%
									\rput(-0.5,-0.5){\textbf{\textcolor{white}{+}}}%
									\rput(2,-0.25){\textbf{\textcolor{white}{--}}}%
									\psline[linewidth=4pt,linecolor=mygreen]{->}(0,0.25)(0,2)%
								}%
							}%
						}%
						\only<5>{%
							\rput(5,0){\vphantom{pÉ}\textcolor{mygreen}{amarrage}}%
						}%
						\only<6>{%
							\rput(-5,0){\myimage[width=0.175\paperwidth]{sota-docking-complex-AB}}%
							\rput(5,0){%
								\mymolA%
								\rput(-3,0.5){\textbf{\textcolor{white}{+}}}%
								\rput(-0.5,0.75){\textbf{\textcolor{white}{--}}}%
								\rput(2,1.5){\textbf{\textcolor{white}{+}}}%
							}%
							\rput(5,0){%
								\mymolB%
								\rput(-3,-1){\textbf{\textcolor{white}{--}}}%
								\rput(-0.5,-0.5){\textbf{\textcolor{white}{+}}}%
								\rput(2,-0.25){\textbf{\textcolor{white}{--}}}%
							}%
						}%
						\mycircleletter[fillcolor=myblue](-9.5,2.5){A}%
						\mycircleletter(-9.5,-2.5){B}%
						\only<6>{%
							\rput(-9.5,0){\Huge\bfseries +}%
						}
					\end{myps}%
					\mycaption[fig-DockingMoleculaire]{\textit{Docking} moléculaire}%
				\end{myfigure}%
			\end{column}%
			\begin{column}{0.35\textwidth}%
				\begin{myblock}{Facteurs de complexité}%
					\begin{itemize}%
						\item<2-> Nombreux atomes%
						\item<3-> Orientation%
						\item<4-> Flexibilité%
						\item<5-> Facteurs chimiques
						\item<6-> Complémentarité%
							\begin{itemize}%
								\item géométrique%
								\item électrostatique%
							\end{itemize}%
					\end{itemize}%
				\end{myblock}%
			\end{column}%
		\end{columns}%
	\end{myframe}
	\subsection{Distribution des charges de travail}
	\begin{myframe}{Distribution des charges de travail}
		\begin{myplusblock}{Définition}
			Étendre la capacité cognitive d'analyse d'un individu pour inclure le matériel et l'environnement social comme composant d'un système cognitif plus étendu.
		\end{myplusblock}

		\begin{columns}[T]%
			\begin{column}{0.55\textwidth}%
				\begin{myfigure}%
					\psset{unit=0.105\textwidth}%
					\definecolor{mylightestredblue}{rgb}{0.875 0.705 0.79328125}
					\begin{myps}(-5,-2.5)(5,4)
						\footnotesize%
						\pspolygon*[linearc=0.5,linecolor=mylightestred](-0.5,4)(-4.5,4)(-4.5,0)(-0.5,-1.7)(1.15,-1.6)%
						\pspolygon*[linearc=0.5,linecolor=mylightestblue](0.5,4)(4.5,4)(4.5,0)(0.5,-1.7)(-1.15,-1.6)%
						\psclip{%
							\pscustom[linestyle=none]{%
								\pspolygon*[linearc=0.5](-0.5,4)(-4.5,4)(-4.5,0)(-0.5,-1.7)(1.15,-1.6)%
							}%
							\pscustom[linestyle=none]{%
								\pspolygon*[linearc=0.5](0.5,4)(4.5,4)(4.5,0)(0.5,-1.7)(-1.15,-1.6)%
							}%
						}%
						\psframe*[linecolor=mylightestredblue](-5,-2.5)(5,4)%
						\endpsclip%
						\psset{fillcolor=white}%
						\psframe[fillstyle=solid,linecolor=black!70,linestyle=dashed,framearc=0.25](-4.5,0)(-0.5,4)%
						\rput(-2.5,3.75){\textcolor{black!70}{Espace interne}}%
						\psframe[fillstyle=solid,linecolor=black!70,linestyle=dashed,framearc=0.25](4.5,0)(0.5,4)%
						\rput(2.5,3.75){\textcolor{black!70}{Espace externe}}%
						\psframe[linecolor=black!70,linestyle=dashed,framearc=0.25](-4.5,-2.5)(4.5,-0.5)%
						\mynode(-2,3.15)[individu1]{\textcolor{myblue}{Individu~\mynum{1}}}%
						\mynode(-3,2)[individu2]{\textcolor{myred}{Individu~\mynum{2}}}%
						\mynode(-2,0.85)[individu3]{\textcolor{mygreen}{Individu~\mynum{3}}}%
						\mynode(2.5,2.75)[artefact1]{Artefact~\mynum{1}}%
						\mynode(2.5,1.25)[artefact2]{Artefact~\mynum{2}}%
						\psellipse[fillstyle=solid,fillcolor=white](0,-1.25)(1,0.5)%
						\rput(0,-1.25){Tâche}%
						\rput(0,-2.25){\textcolor{black!70}{Abstraction de l'espace de travail}}%
						\psset{linewidth=0.5pt}
						\psset{arrows={<->}}%
						\ncline{individu1}{individu2}%
						\ncline{individu2}{individu3}%
						\ncline[offset=8pt]{individu1}{individu3}%
						\psset{arrows={-}}%
						\ncline{artefact1}{artefact2}%
						\psset{arrows={<->}}%
						\psset{linecolor=myblue}%
						\ncline{individu1}{artefact1}%
						\ncline{individu1}{artefact2}%
						\psset{linecolor=myred}%
						\ncline{individu2}{artefact1}%
						\ncline{individu2}{artefact2}%
						\psset{linecolor=mygreen}%
						\ncline{individu3}{artefact1}%
						\ncline{individu3}{artefact2}%
					\end{myps}
					\mycaption[fig-sota-SystemeCognitifDistribue]{Système cognitif distribué}
				\end{myfigure}
			\end{column}
			\begin{column}{0.4\textwidth}
				\begin{myblock}{Travaux sur la collaboration}
					\begin{itemize}
						\item<2-> manipulation colocalisée \mycite{Kriz-2003}
						\item<3-> un manipulateur guidé par deux partenaires \mycite{Park-2006}
						\item<4-> inter-référencement \mycite{Chastine-2007}
						\item<5-> gestion des droits \mycite{Ma-2007}
					\end{itemize}
				\end{myblock}
			\end{column}
		\end{columns}
	\end{myframe}
	\subsection{Objectifs de la thèse}
	\begin{myframe}{Objectifs de la thèse}
		\begin{myminusblock}{Problématique}
			\begin{itemize}
				\item Quels sont les avantages du travail en collaboration ?
				\item Quelles problématiques supplémentaires la collaboration apporte-t-elle ?
				\item Comment améliorer la collaboration dans un environnement de travail complexe ?
			\end{itemize}
		\end{myminusblock}
		\begin{myplusblock}{Objectifs}
			Analyser le travail collaboratif dans le contexte du \textit{docking} moléculaire pour proposer des outils haptique adaptés.
			\begin{enumerate}
				\item Étudier le travail collaboratif dans les tâches de manipulation moléculaire
				\item Identifier les faiblesses de cette configuration de travail
				\item Proposer des solutions appropriées pour assister le travail
				\item Évaluer ces solutions en situation réelle
			\end{enumerate}
		\end{myplusblock}
	\end{myframe}
	\section{Plateforme de manipulation moléculaire \myShaddock}
	\begin{myframe}{Sommaire}
		\tableofcontents[sectionstyle=show/shaded,subsectionstyle=show/shaded/hide,subsubsectionstyle=show/show/hide]
	\end{myframe}
	\subsection{Cahier des charges}
	\begin{myframe}{Cahier des charges}
		\begin{myblock}{Objectif}
			Élaborer une plateforme pour étudier le travail collaboratif dans le contexte de la manipulation interactive de molécules.
		\end{myblock}
		\begin{columns}[T]
			\begin{column}{0.475\textwidth}
				\begin{myminusblock}{Contraintes}
					\begin{itemize}
						\item Travail en collaboration
						\item Interaction temps-réel avec des molécules
						\item Manipulation à l'aide d'interface haptique
						\item Simulation temps-réel de la dynamique des molécules
					\end{itemize}
				\end{myminusblock}
			\end{column}
			\begin{column}{0.475\textwidth}
				\begin{myplusblock}{Solutions}
					\begin{itemize}
						\item Modularité
						\item Composant logiciels existants en biologie
						\item Modules dédiés à la réalité virtuelle
						\item Développement de nouveaux outils
					\end{itemize}
				\end{myplusblock}
			\end{column}
		\end{columns}
	\end{myframe}
	\subsection{Organisation logicielle de la plateforme \myShaddock}
	\begin{myframe}{Organisation logicielle de la plateforme \myShaddock}
		\begin{myfigure}
			\psset{xunit=0.0666667\textwidth,yunit=0.06\textheight}
			\psset{framearc=.1,shadow=true,blur=true}
			\begin{myps}(-7.5,-5)(7.5,6)
				\psframe*[linecolor=mygreen!5,shadow=false](-7.4,-5)(-2.6,6)
				\psframe*[linecolor=myblue!5,shadow=false](-2.4,-5)(2.4,6)
				\psframe*[linecolor=myred!5,shadow=false](2.6,-5)(7.4,6)
				\uput[-90](-5,6){\large\textcolor{mygreen!25}{Simulation}}
				\uput[-90](0,6){\large\textcolor{myblue!25}{Visualisation}}
				\uput[-90](5,6){\large\textcolor{myred!25}{Interaction}}
				\uput[-90](0,5){%
					\myumlnode*<PCUtilisateur>{\vphantom{pÉ}\scriptsize Nœud principal}{%
						\myumlcomponent<VMD>[\tiny application]{\scriptsize\myVMD}%
					}%
				}
				\uput[-90](-5,5){%
					\myumlnode*<ServeurNAMD>{\vphantom{pÉ}\scriptsize Nœud \myNAMD}{%
						\begin{psmatrix}[rowsep=1]%
							\myumlcomponent<NAMD>[\tiny programme]{{\scriptsize\texttt{namd2}}} \\%
							\myumlcomponent<FichierSimulation>[\tiny fichier]{%
								\\[-1ex]%
								\begin{psmatrix}[rowsep=0]%
									\scriptsize Données de\\\scriptsize simulation%
								\end{psmatrix}%
							}
						\end{psmatrix}%
					}%
				}
				\uput[-90](5,5){%
					\myumlnode*<ServeurVRPN1>{\vphantom{pÉ}\scriptsize Nœud \myVRPN}{%
						\myumlcomponent<VRPN1>[\tiny programme]{{\scriptsize\texttt{vrpn\_server}}}%
					}%
				}
				\uput[-90](5,0){%
					\myumlnode<PHANToM1>[\tiny\myOmni]{\vphantom{pÉ}\scriptsize Interface}%
				}
				\rput(5,-4){\Large$\vdots$}
				\uput[-90](0,0){%
					\myumlnode<VideoProjecteur>[\tiny vue partagée]{\scriptsize Vidéoprojecteur}
				}
				\psset{shadow=false}

				\myumlrealization[angleA=-90,angleB=90]{NAMD}{FichierSimulation}[nccurve]%
				\myumlrealization[angleA=-90,angleB=90]{VRPN1}{PHANToM1}[nccurve]%
				\myumlinterface[angleA=-90,angleB=90,ArrowInsidePos=0.5]{VMD}{VideoProjecteur}[nccurve]
				\myumlinterface[angleA=0,angleB=180,offsetB=-8pt,ArrowInsidePos=0.4]{NAMD}{VMD}[nccurve]
				\myumlinterface[angleA=180,angleB=0,ncurvA=1.5,offsetA=8pt,ArrowInsidePos=0.6]{VRPN1}{VMD}[nccurve]
			\end{myps}
			\mycaption[fig-Shaddock-DiagrammeDeDeploiementUMLDeLaPlateformeShaddock]{Diagramme de déploiement \myUML de la plateforme \myShaddock}
		\end{myfigure}
	\end{myframe}
	\subsection{Organisation matérielle de la plateforme \myShaddock}
	\begin{myframe}{Organisation matérielle de la plateforme \myShaddock}
		\psset{xunit=0.666666666667\paperwidth,yunit=0.4\paperwidth}
		\begin{myfigure}
			\begin{myps}(0,0)(1,1)%
				\only<1-6>{%
				\rput(0.5,0.5){\myimage[width=0.666666666667\paperwidth,height=0.4\paperwidth]{exp1-schema}}%
				\rput(0.2,0.8){\myimage[width=0.266666666667\paperwidth,height=0.16\paperwidth]{exp1-photo}}%
				\pnode(0.52,0.66){tug1}%
				\pnode(0.62,0.76){tug1-from}%
				\pnode(0.62,0.60){tug2}%
				\pnode(0.72,0.70){tug2-from}%
				\pnode(0.57,0.63){grab}%
				\pnode(0.67,0.73){grab-from}%
			}%
				\only<3>{%
					\ncline[linewidth=2.5pt,linecolor=myred,nodesepB=2.5pt]{c->}{tug1-from}{tug1}%
					\ncline[linewidth=2.5pt,linecolor=mygreen,nodesepB=2.5pt]{c->}{tug2-from}{tug2}%
				}%
				\only<4>{%
					\ncline[linewidth=2.5pt,linecolor=myblue,nodesepB=2.5pt]{c->}{grab-from}{grab}%
				}%
				\only<5>{%
					\pspolygon[linewidth=4pt,linecolor=myred](0.60,0.68)(0.61,0.93)(0.86,0.88)(0.83,0.57)%
				}%
				\only<6>{%
					\pspolygon[linewidth=2pt,linecolor=myred](0.64,0.63)(0.65,0.69)(0.71,0.665)(0.70,0.60)%
				}%
				\only<7>{
					\rput(0.5,0.5){\myimage[width=0.666666666667\paperwidth,height=0.4\paperwidth]{exp2-schema}}
					\rput(0.2,0.8){\myimage[width=0.266666666667\paperwidth,height=0.16\paperwidth]{exp2-photo}}
					\pnode(0.59,0.6){grab}
					\pnode(0.69,0.7){grab-from}
					\ncline[linewidth=2.5pt,linecolor=blue,nodesepB=2.5pt]{c->}{grab-from}{grab}
				}
				\only<8>{
					\rput(0.5,0.5){\myimage[width=0.666666666667\paperwidth,height=0.4\paperwidth]{exp3-schema}}
					\rput(0.2,0.8){\myimage[width=0.266666666667\paperwidth,height=0.16\paperwidth]{exp3-photo}}
					\pnode(0.52,0.675){tug1}
					\pnode(0.62,0.775){tug1-from}
					\pnode(0.55,0.65){tug2}
					\pnode(0.65,0.75){tug2-from}
					\pnode(0.58,0.63){tug3}
					\pnode(0.68,0.73){tug3-from}
					\pnode(0.62,0.61){tug4}
					\pnode(0.72,0.71){tug4-from}
					\ncline[linewidth=2.5pt,linecolor=myred,nodesepB=2.5pt]{c->}{tug1-from}{tug1}
					\ncline[linewidth=2.5pt,linecolor=myblue,nodesepB=2.5pt]{c->}{tug2-from}{tug2}
					\ncline[linewidth=2.5pt,linecolor=magenta,nodesepB=2.5pt]{c->}{tug3-from}{tug3}
					\ncline[linewidth=2.5pt,linecolor=mygreen,nodesepB=2.5pt]{c->}{tug4-from}{tug4}
				}
				\only<9>{
					\rput(0.5,0.5){\myimage[width=0.666666666667\paperwidth,height=0.4\paperwidth]{exp4-schema}}
					\rput(0.2,0.8){\myimage[width=0.266666666667\paperwidth,height=0.16\paperwidth]{exp4-photo}}
					\pnode(0.51,0.67){tug1a}
					\pnode(0.54,0.65){tug1b}
					\pnode(0.61,0.775){tug1-from}
					\pnode(0.58,0.63){tug3}
					\pnode(0.68,0.73){tug3-from}
					\pnode(0.63,0.61){tug4}
					\pnode(0.73,0.71){tug4-from}
					\ncline[linewidth=2.5pt,linecolor=myred,nodesepB=2.5pt]{c->}{tug1-from}{tug1a}
					\ncline[linewidth=2.5pt,linecolor=myred,nodesepB=2.5pt]{c->}{tug1-from}{tug1b}
					\ncline[linewidth=2.5pt,linecolor=myblue,nodesepB=2.5pt]{c->}{tug3-from}{tug3}
					\ncline[linewidth=2.5pt,linecolor=mygreen,nodesepB=2.5pt]{c->}{tug4-from}{tug4}
				}
			\end{myps}
			\mycaption[fig-PlateFormeExperimentale1]{Plate-forme expérimentale}
		\end{myfigure}
		\only<1>{
			\begin{itemize}
				\item Configuration \alert{colocalisée} et \alert{synchrone}
			\end{itemize}
		}
		\only<2>{
			\begin{itemize}
				\item \alert{Communication} orale et gestuelle autorisée
			\end{itemize}
		}
		\only<3>{
			\begin{itemize}
				\item Outil de \alert{déformation} de la molécule (Omni de SensAble\myregistered)
			\end{itemize}
		}
		\only<4>{
			\begin{itemize}
				\item Outil pour \alert{déplacer} la molécule (Omni de SensAble\myregistered)
			\end{itemize}
		}
		\only<5>{
			\begin{itemize}
				\item Vue monoscopique, unique, \alert{publique} et vidéoprojetée
			\end{itemize}
		}
		\only<6>{
			\begin{itemize}
				\item Affichage \alert{déporté} des objectifs
			\end{itemize}
		}
		\only<7>{
			\begin{itemize}
				\item Outil pour \alert{orienter} la molécule (SpaceTraveler de 3\textsc{d}connexion\myregistered{})
			\end{itemize}
		}
		\only<8>{
			\begin{itemize}
				\item Nombre d'outils quasiment \alert{illimité}
			\end{itemize}
		}
		\only<9>{
			\begin{itemize}
				\item Collaboration \alert{asymétrique} entre les participants
			\end{itemize}
		}
	\end{myframe}
	\subsection{Outils supplémentaires proposés}
	\begin{myframe}{Outils supplémentaires proposés}
		\begin{myblock}{Objectif}
			Faciliter le processus de sélection dans l'application \myVMD
		\end{myblock}

		\begin{myfigure}
			\psset{xunit=0.6\textwidth,yunit=0.365838509\textwidth}
			\begin{myps}(0,0)(1,1)%
				\rput(0.5,0.5){%
					\only<1-2>{%
						\myimage[width=0.6\textwidth]{select-molecule}%
					}%
					\only<3>{%
						\myimage[width=0.6\textwidth]{select-atom-target}%
					}%
					\only<4>{%
						\myimage[width=0.6\textwidth]{select-residue-target}%
					}%
					\only<5>{%
						\myimage[width=0.6\textwidth]{select-residue-select}%
					}%
				}%
				\only<2>{%
					\psarc{c-c}(0.88,0.86){0.13}{-45}{45}%
					\psarc{c-c}(0.88,0.86){0.20}{-45}{45}%
					\psarc{c-c}(0.88,0.86){0.30}{-45}{45}%
					\psarc{c-c}(0.88,0.86){0.45}{-45}{45}%
					\uput{0pt}[15](1.1,0.92){%
						\myimage[width=40pt]{red-cursor}%
					}%
					\psline[linewidth=2pt,linecolor=mygreen]{c->}(1.075,0.86)(0.95,0.86)
				}%
				\only<3->{%
					\uput{0pt}[15](0.91,0.92){%
						\myimage[width=40pt]{red-cursor}%
					}%
				}%
			\end{myps}
			\mycaption[fig-OutilDeSelectionAmeliore]{Outil de sélection amélioré}
		\end{myfigure}
		\only<1>{
			\begin{itemize}
				\item Pointage d'une cible difficile
			\end{itemize}
		}
		\only<2>{
			\begin{itemize}
				\item Modèle haptique d'attraction sur les atomes
			\end{itemize}
		}
		\only<3>{
			\begin{itemize}
				\item Possibilité de pointer un atome\dots{}
			\end{itemize}
		}
		\only<4>{
			\begin{itemize}
				\item \dots{} ou un résidue (ou d'autres structures moléculaires)
			\end{itemize}
		}
		\only<5>{
			\begin{itemize}
				\item Pour enfin le sélectionner
			\end{itemize}
		}
	\end{myframe}
	\section{Étude du travail collaboratif}
	\begin{myframe}<1-7>[label={fra-sota-LesEtapesDeLEtude}]{Les étapes de l'étude}
		\begin{columns}[T]
			\begin{column}{0.45\textwidth}
				\begin{myfigure}
					\psset{xunit=1cm,yunit=1.1cm}
					\begin{myps}(-2.5,-0.5)(2.5,5)%
						\only<1,3-7,9>{%
							\mynode(0,4)[Search]{Recherche}%
						}%
						\only<2,8>{%
							\mynode[fillstyle=solid,fillcolor=myred!25](0,4)[Search]{Recherche}%
							\mycirclenumber(0,4){1}%
						}%
						\only<1-2,4-7,9>{%
							\mynode(0,3)[Selection]{Sélection}%
						}%
						\only<3,8>{%
							\mynode[fillstyle=solid,fillcolor=myred!25](0,3)[Selection]{Sélection}%
							\mycirclenumber(0,3){2}%
						}%
						\only<1-3,5-8>{%
							\mynode(0,2)[Manipulation]{Manipulation}%
						}%
						\only<4,9>{%
							\mynode[fillstyle=solid,fillcolor=myred!25](0,2)[Manipulation]{Manipulation}%
							\mycirclenumber(0,2){3}%
						}%
						\only<1-4,6-8>{%
							\mynode(0,1)[Evaluation]{Évaluation}%
						}%
						\only<5,9>{%
							\mynode[fillstyle=solid,fillcolor=myred!25](0,1)[Evaluation]{Évaluation}%
							\mycirclenumber(0,1){4}%
						}%
						\ncline{->}{Search}{Selection}%
						\ncline{->}{Selection}{Manipulation}%
						\ncline{->}{Manipulation}{Evaluation}%
						\only<6->{%
							\ncloop[loopsize=4em,angleA=-90,angleB=90,linearc=0.05,armA=0.2]{->}{Evaluation}{Search}%
						}%
						\only<6>{%
							\mycirclenumber[180](0,2.5){5}%
						}%
						\only<7>{%
							\mynode[fillstyle=solid,fillcolor=myred!25](0,0)[Objective]{Objectif atteint}%
							\mycirclenumber(0,0){6}%
						}%
						\only<8->{%
							\mynode(0,0)[Objective]{Objectif atteint}%
						}%
						\only<7->{%
							\ncline{->}{Evaluation}{Objective}%
						}%
					\end{myps}
					\mycaption[fig-ManipulationMoleculaire]{Manipulation moléculaire}
				\end{myfigure}
			\end{column}
			\only<1-7>{%
				\begin{column}{0.5\textwidth}%
					\begin{myblock}{Description}%
						Basé sur les PCV de \mycite[author]{Fuchs-2006}%
						\begin{description}%
							\item<2->[Recherche] Identifier une tâche élémentaire%
							\item<3->[Sélection] Sélectionner une structure moléculaire (atome, résidue, \dots)%
							\item<4->[Manipulation] Déplacer ou déformer la molécule%
							\item<5->[Évaluation] Évaluer l'équilibre physico-chimique de la molécule%
							\item<6->[Recommencer] Si l'évaluation n'est pas satisfaisante%
						\end{description}%
					\end{myblock}%
				\end{column}%
			}
		\end{columns}
	\end{myframe}
	\subsection{Étude~\mynum{1} -- Recherche collaborative de résidus}
	\begin{myframe}{Sommaire}
		\tableofcontents[sectionstyle=show/shaded,subsectionstyle=show/shaded/hide,subsubsectionstyle=show/show/hide]
	\end{myframe}
	\againframe<8>{fra-sota-LesEtapesDeLEtude}
	\subsubsection{Objectifs}
	\begin{myframe}{Objectifs}
		\begin{myblock}{Objectif principal}
			Étudier la contribution et les contraintes de la collaboration dans une tâche de recherche de structures moléculaires dans un environnement complexe
		\end{myblock}
		\begin{myplusblock}{Hypothèses}
			\begin{enumerate}
				\item Amélioration des performances (individuel $\rightarrow$ collaboratif)
					\begin{itemize}
						\item Comparer les performances en collaboration et seul
						\item Valider le contexte de travail (tâche complexe)
					\end{itemize}
				\item Identifier les stratégies de travail
					\begin{itemize}
						\item Identifier et caractériser les stratégies de travail
						\item Identifier les conflits de coordination et de communication
					\end{itemize}
				\item Utilisabilité de la plate-forme
					\begin{itemize}
						\item Évaluer les outils proposés
						\item Identifier les faiblesses
					\end{itemize}
			\end{enumerate}
		\end{myplusblock}
	\end{myframe}
	\subsubsection{Protocole expérimental}
	\begin{myframe}{La tâche}
		\renewcommand{\schemafactor}{0.1125}
		\setlength{\schemaunit}{\schemafactor\paperwidth}
		\psset{unit=\schemaunit}
		\begin{myfigure}
			\begin{myps}(-4,-2.3)(4,2.3)
				\rput(-1.75,0){%
					\myimage[height=2\schemaunit]{trp-cage}}
				\rput(1.25,0){%
					\myimage[height=2\schemaunit]{prion}}
				\rput(-3.5,0){%
					\myimage[height=\schemaunit]{pattern1}}
				\rput(-1,1.5){%
					\myimage[width=\schemaunit]{pattern3-8}}
				\rput(1,1.5){%
					\myimage[width=\schemaunit]{pattern2-7}}
				\rput(-1,-1.5){%
					\myimage[width=\schemaunit]{pattern4-9}}
				\rput(1,-1.5){%
					\myimage[width=\schemaunit]{pattern5-10}}
				\rput(3.5,0){%
					\myimage[height=\schemaunit]{pattern6}}

				\psset{framesize=1 1}
				\fnode(-3.5,0){P1}
				\uput[90](-3.5,0.5){Residue~1}
				\fnode(-1,1.5){P38}
				\uput[90](-1,2){Residue~3 and 8}
				\fnode(1,1.5){P27}
				\uput[90](1,2){Residue~2 and 7}
				\fnode(-1,-1.5){P49}
				\uput[-90](-1,-2){Residue~4 and 9}
				\fnode(1,-1.5){P510}
				\uput[-90](1,-2){Residue~5 and 10}
				\fnode(3.5,0){P6}
				\uput[90](3.5,0.5){Residue~6}

				\psset{linecolor=myred}
				\cnode(-1.5,0.3){0.2}{TRPP1}
				\cnode(-2,0.15){0.2}{TRPP38}
				\cnode(-1.25,-0.1){0.2}{TRPP27}
				\cnode(-2.2,-0.5){0.2}{TRPP49}
				\cnode(-1.25,-0.65){0.2}{TRPP510}
				\ncline{-}{P1}{TRPP1}
				\ncline{-}{P38}{TRPP38}
				\ncline{-}{P27}{TRPP27}
				\ncline{-}{P49}{TRPP49}
				\ncline{-}{P510}{TRPP510}

				\psset{linecolor=myblue}
				\cnode(-0.2,0.4){0.2}{PrionP38}
				\cnode(2.8,0.6){0.2}{PrionP27}
				\cnode(0.8,0.2){0.2}{PrionP49}
				\cnode(1.7,-0.7){0.2}{PrionP510}
				\cnode(1.4,0.0){0.2}{PrionP6}
				\ncline{-}{P38}{PrionP38}
				\ncline{-}{P27}{PrionP27}
				\ncline{-}{P49}{PrionP49}
				\ncline{-}{P510}{PrionP510}
				\ncline{-}{P6}{PrionP6}
			\end{myps}
			\mycaption[fig-RepartitionDesResiduesSurLesMoleculesTRPCageEtPrion]{Répartitions des \emph{residues} sur les molécules (TRP-Cage et Prion)}
		\end{myfigure}
	\end{myframe}
	\begin{myframe}{Protocole}
		\begin{myblock}{Sujets}
			\begin{itemize}
				\item 24~participants
				\item Différents niveaux d'expertise
				\item Étude intra-population
			\end{itemize}
		\end{myblock}
		\begin{myblock}{Variables}
			\begin{description}
				\item[Nombre de participants] un (24~sujets) ou deux (12~couples)
				\item[Taille de la molécule] une petite (TRP-Cage) et une grande (Prion)
				\item[Caractéristiques du \emph{residue}] Forme, nature, position, similarités\dots{}
			\end{description}
		\end{myblock}
	\end{myframe}
	\subsubsection{Résultats}
	\begin{myframe}{Amélioration des performances en collaboration}
		\begin{myfigure}
			\mylegend{%
				\myleg{monôme}{myblue}%
				\myleg{binôme}{myblue!70}%
			}
			\begin{myboxgraph}[llx=-3em,yAxisLabelPos={-3em,c}](10,0.75\textwidth)[100]{résidu}(500,2cm){temps~(s)}
				\myboxplot{exp1-time-residue-group.csv}
				\only<2>{%
					\psframe[linecolor=myred,linewidth=2pt,framearc=0.25,dimen=inner](0,0)(5,200)
					\psframe[linecolor=myred,linewidth=2pt,framearc=0.25,dimen=inner](6,0)(8,200)
				}
				\only<3>{%
					\psframe[linecolor=mygreen,linewidth=2pt,framearc=0.25,dimen=inner](5,0)(6,300)
					\psframe[linecolor=mygreen,linewidth=2pt,framearc=0.25,dimen=inner](8,0)(10,450)
				}
			\end{myboxgraph}
			\mycaption[fig-TempsDeRealisationDeLaTache]{Temps de réalisation de la tâche}
		\end{myfigure}
		\begin{myfigure}
			\mylegend{%
				\myleg{recherche}{myblue}%
				\myleg{sélection}{myblue!70}%
			}
			\begin{myboxgraph}[llx=-3em,yAxisLabelPos={-3em,c}](10,0.75\textwidth)[100]{résidu}(500,2cm){temps~(s)}
				\myboxplot{exp1-timeaudio-residue-searchselection.csv}
				\only<2>{%
					\psframe[linecolor=myred,linewidth=2pt,framearc=0.25,dimen=inner](0,0)(5,100)
					\psframe[linecolor=myred,linewidth=2pt,framearc=0.25,dimen=inner](6,0)(8,100)
				}
				\only<3>{%
					\psframe[linecolor=mygreen,linewidth=2pt,framearc=0.25,dimen=inner](5,0)(6,250)
					\psframe[linecolor=mygreen,linewidth=2pt,framearc=0.25,dimen=inner](8,0)(10,450)
				}
			\end{myboxgraph}
			\mycaption[fig-exp1-TempsDeRechercheEtDeSelectionCompares]{Temps de recherche et de sélection comparés}
		\end{myfigure}
	\end{myframe}
	\begin{myframe}{Stratégies de travail}
		\begin{myfigure}
			\begin{myboxgraph}[lly=-5ex,xAxisLabelPos={c,-5ex}](12,0.75\textwidth)[4]{binôme}(24,2cm){distance~(mm)}
				\psframe*[linecolor=myred!25,dimen=inner](0,0)(12,8)
				\psframe*[linecolor=mygreen!25](0,8)(12,14)
				\psframe*[linecolor=myblue!25](0,14)(12,20)
				\psline[linewidth=0.5pt,linestyle=dashed,arrows=-](0,4)(12,4)
				\psline[linewidth=0.5pt,linestyle=dashed,arrows=-](0,8)(12,8)
				\psline[linewidth=0.5pt,linestyle=dashed,arrows=-](0,12)(12,12)
				\psline[linewidth=0.5pt,linestyle=dashed,arrows=-](0,16)(12,16)
				\psline[linewidth=0.5pt,linestyle=dashed,arrows=-](0,20)(12,20)
				% Once header are readed, they are defined for other barplot
				% That's why barplots without headers are in first position
				\mybarplot[header=false,barstyle=third-barstyle]{exp1-diff-groups3.csv}
				\mybarplot[header=false,barstyle=second-barstyle]{exp1-diff-groups2.csv}
				\mybarplot[header=true,barstyle=first-barstyle]{exp1-diff-groups1.csv}
				\psset{linewidth=0.1pt,linecolor=white,fillstyle=solid,fillcolor=myred}
				\uput[180](12,5){\pscharpath{\Large\bf\sffamily Champ proche}}
				\psset{fillcolor=mygreen}
				\uput[180](12,11){\pscharpath{\Large\bf\sffamily Champ voisin}}
				\psset{fillcolor=myblue}
				\uput[180](12,17){\pscharpath{\Large\bf\sffamily Champ distant}}
			\end{myboxgraph}
			\mycaption[fig-DistanceMoyenneEntreLeCurseurDesSujets]{Distance moyenne entre le curseur des sujets}
		\end{myfigure}
		\begin{myfigure}
			\begin{myboxgraph}[lly=-5ex,xAxisLabelPos={c,-5ex}](12,0.75\textwidth){binôme}(6,2cm){affinité~(\mynum{1}--\mynum{5})}
				\mybarplot{exp1-affinity-groups.csv}
				\psframe[linecolor=myblue,linewidth=2pt,framearc=0.25](0,0)(1,-1.75)
				\psframe[linecolor=mygreen,linewidth=2pt,framearc=0.25](1,0)(4,-1.75)
				\psframe[linecolor=myred,linewidth=2pt,framearc=0.25](4,0)(12,-1.75)
			\end{myboxgraph}
			\mycaption[fig-exp1-AffiniteEntreLesSujetsPourChaqueBinome]{Affinité entre les sujets pour chaque binômes}
		\end{myfigure}
	\end{myframe}
	\subsubsection{Synthèse}
	\subsection{Étude~\mynum{2} -- Déformation collaborative de molécule}
	\begin{myframe}{Sommaire}
		\tableofcontents[sectionstyle=show/shaded,subsectionstyle=show/shaded/hide,subsubsectionstyle=show/show/hide]
	\end{myframe}
	\againframe<9>{fra-sota-LesEtapesDeLEtude}
	\subsubsection{Objectifs}
	\begin{myframe}{Objectifs}
		\begin{myblock}{Objectif principal}
			Quantifier et qualifier les conflits de coordination en fonction de la complexité de la tâche
		\end{myblock}
		\begin{myplusblock}{Hypothèses}
			\begin{enumerate}
				\item Amélioration des performances (individuel $\rightarrow$ collaboration)
					\begin{itemize}
						\item Coordination étroitement couplée
					\end{itemize}
				\item La complexité de la tâche influence différemment les performances individuelles et collaboratives
					\begin{itemize}
						\item Tâches de difficulté variable
						\item Identifier les tâches nécessitant une collaboration
					\end{itemize}
				\item Évaluation du travail collaboratif par les sujets
					\begin{itemize}
						\item Questionnaire pour valider les améliorations de la plate-forme
						\item Évaluation de la configuration de travail collaboratif
					\end{itemize}
			\end{enumerate}
		\end{myplusblock}
	\end{myframe}
	\subsubsection{Protocole expérimental}
	\begin{myframe}<1>[label={fra-exp2-LaTache}]{La tâche}
		\renewcommand{\schemafactor}{0.055}
		\setlength{\schemaunit}{\schemafactor\paperwidth}
		\psset{unit=\schemaunit}
		\begin{myfigure}
			\begin{myps}(-1,0)(12,9)
				\rput[bl](1,0){\myimage[width=10\schemaunit]{TRP-ZIPPER}}
				\pnode(6.8,3.6){deformed}
				\rput(9.2,2.6){\rnode{deformed-label}{\textcolor{myred}{molécule courante}}}
				\pnode(1.8,4){ghost}
				\rput(0.8,5.5){\rnode{ghost-label}{\textcolor{myred}{molécule cible}}}
				\psset{linecolor=myblue}
				\cnode(6.2,4.9){1.0}{deformed-residue}
				\rput(7.1,7.0){\rnode{deformed-residue-label}{\textcolor{myblue}{résidue courant}}}
				\cnode(2.3,1.6){0.8}{ghost-residue}
				\rput(1,2.75){\rnode{ghost-residue-label}{\textcolor{myblue}{résidue cible}}}
				\psset{linecolor=gray}
				\cnode(2.0,6.6){0.8}{fixed-residue}
				\rput(4.0,7.75){\rnode{fixed-residue-label}{\textcolor{gray}{résidu fixe}}}
				\psset{linewidth=1pt,linecolor=myred,linearc=.1,arrowsize=0.5pt 3,arrowinset=.2,nodesepA=3pt}
				\ncangle[angleA=90,angleB=0]{c->}{deformed-label}{deformed}
				\psset{nodesepB=0pt}
				\ncdiagg[angleA=-90,angleB=135]{c->}{ghost-label}{ghost}
				\psset{linecolor=myblue}
				\ncdiagg[angleA=-90]{c->}{deformed-residue-label}{deformed-residue}
				\ncdiagg[angleA=-90]{c->}{ghost-residue-label}{ghost-residue}
				\ncdiagg[angleA=180,linecolor=gray]{c->}{fixed-residue-label}{fixed-residue}
				\ncline[linewidth=10pt,linecolor=mygreen]{C->}{deformed-residue}{ghost-residue}
				\only<1>{%
					\psframe*[linecolor=green](-1,8.5)(5,9)
					\psframe*[linecolor=red](5,8.5)(12,9)
					\rput(5.5,8.75){\textcolor{white}{\bf score \textsc{rmsd}}}
					\psframe[linewidth=1pt,linecolor=black](-1,0)(12,9)
				}
				\only<2>{%
					\psframe*[linecolor=blue](11.25,0)(12,2)
					\psline[linecolor=blue](11,2)(11.25,2)
					\uput{1pt}[180](11,2){\tiny\textcolor{blue}{RMSD courant}}
					\psframe*[linecolor=orange](11.45,0)(11.8,1)
					\psline[linecolor=orange](11,1)(11.45,1)
					\uput{1pt}[180](11,1){\tiny\textcolor{orange}{RMSD minimum}}
					\psframe[linewidth=1pt,linecolor=black](-1,0)(12,9)
				}
			\end{myps}
			\mycaption[fig-TacheDeDeformation]{Tâche de déformation}
		\end{myfigure}
	\end{myframe}
	\begin{myframe}{Protocole}
		\begin{myblock}{Sujets}
			\begin{itemize}
				\item 36~participants (12~couples et 12~sujets seuls)
				\item Sujets avec différents niveaux d'expertise
				\item Couples choisis pour leurs affinités
				\item Étude inter-population
			\end{itemize}
		\end{myblock}
		\begin{myblock}{Variables}
			\begin{description}
				\item[Complexité de la molécule] 2~molécules (1~petite et 1~grande)
				\item[Outil de déformation] 2~configuration de déformation (\emph{atom} et \emph{residue})
			\end{description}
		\end{myblock}
	\end{myframe}
	\subsubsection{Résultats}
	\begin{myframe}{Amélioration des performances}
		\begin{myfigure}
				\mylegend{%
					\myleg{monôme}{myblue}%
					\myleg{binôme}{myblue!70}%
				}
			\begin{myboxgraph}(2,0.75\textwidth){distance}(3.25,2cm){distance~(mm)}
				\myboxplot{exp2-diff-activepassive-group.csv}
			\end{myboxgraph}
			\mycaption[fig-exp2-DistancePassiveEtActive]{Distances passive et active}
		\end{myfigure}
		\begin{myfigure}
			\mylegend{%
				\myleg{main dominante}{myblue}%
				\myleg{main dominée}{myblue!70}%
			}
			\begin{myboxgraph}(2,0.75\textwidth)[10]{nombre de sujets}(50,2cm){sélections~(nb)}
				\myboxplot{exp2-numsel-group-dominant.csv}
			\end{myboxgraph}
			\mycaption[fig-exp2-NombreDeSelectionsParMainDominanteDominee]{Nombre de sélections par main dominante/dominée}
		\end{myfigure}
	\end{myframe}
	\begin{myframe}{Influence de la complexité de la tâche}
		\begin{myfigure}
			\mylegend{%
				\myleg{monôme}{myblue}%
				\myleg{binôme}{myblue!70}%
			}
			\begin{myboxgraph}[llx=-3em,yAxisLabelPos={-3em,c}](4,0.75\textwidth)[100]{scénario}(350,1.75cm){temps~(s)}
				\myboxplot{exp2-time-task-group.csv}
			\end{myboxgraph}
			\mycaption[fig-exp2-TempsDeRealisationDesScenariosEnFonctionDuNombreDeSujets]{Temps de réalisation des scénarios}
		\end{myfigure}
		\begin{mytable}
			\begin{tabular}{cp{7cm}c}
				\hline
				Difficulté & Description & Exemple \\
				\hline
				\hline
				\multirow{2}*{Simple} & -- \mynum{1}~outil est nécessaire & \multirow{2}*{Tâche~1a} \\
				& -- \mynum{1}~manipulation \\
				\hline
				\multirow{2}*{Avancé} & -- \mynum{1}~outil est suffisant mais \mynum{2} sont préférables & \multirow{2}*{Tâche~2a, 2b} \\
				& -- \mynum{2}~manipulations peuvent être coordonnées \\
				\hline
				\multirow{2}*{Expert} 	& -- \mynum{2}~outils sont nécessaires & \multirow{2}*{Tâche~1b} \\
				& -- \mynum{2}~manipulations \alert{doivent} être coordonnées \\
				\hline
			\end{tabular}
			\mycaption[tab-ClassificationDesTaches]{Classification des tâches}
		\end{mytable}
	\end{myframe}
	\subsubsection{Synthèse}
	\subsection{Étude~\mynum{3} -- Dynamique de groupe}
	\begin{myframe}{Sommaire}
		\tableofcontents[sectionstyle=show/shaded,subsectionstyle=show/shaded/hide,subsubsectionstyle=show/show/hide]
	\end{myframe}
	\subsubsection{Travaux existants}
	\begin{myframe}{Travaux existants}
		\begin{myplusblock}{Dynamique de groupe}
			\begin{itemize}
				\item facilitation sociale \mycite{Ringelmann-1913}
				\item paresse sociale \mycite{Roethlisberger-1939}
				\item brainstorming \mycite{Osborn-1963,Tuckman-1965}
			\end{itemize}
		\end{myplusblock}
		\begin{myminusblock}{Problématique}
			\begin{itemize}
				\item Aucune étude de dynamique de groupe sur des tâches avec une interaction étroitement couplée
			\end{itemize}
		\end{myminusblock}
	\end{myframe}
	\subsubsection{Objectifs}
	\begin{myframe}{Objectifs}
		\begin{myblock}{Objectif principal}
			Observer la dynamique de groupe lors d'une coordination étroitement couplée
		\end{myblock}
		\begin{myplusblock}{Hypothèses}
			\begin{enumerate}
				\item Amélioration des performances en quadrinôme
					\begin{itemize}
						\item Variation de la taille d'un groupe
						\item Quantification des conflits dans des groupes
					\end{itemize}
				\item Émergence d'un meneur
					\begin{itemize}
						\item Observer la dynamique des groupes
						\item Caractériser les différents rôles
					\end{itemize}
				\item Le \textit{brainstorming} améliore les performances
					\begin{itemize}
						\item Période pour organiser le travail
						\item Limiter les conflits \textit{a priori}
					\end{itemize}
			\end{enumerate}
		\end{myplusblock}
	\end{myframe}
	\subsubsection{Protocole expérimental}
	\againframe<2>{fra-exp2-LaTache}
	\begin{myframe}{Protocole}
		\begin{myblock}{Sujets}
			\begin{itemize}
				\item 16~participants
				\item Sujets avec expérience sur la plate-forme
				\item Étude intra-population
			\end{itemize}
		\end{myblock}
		\begin{myblock}{Variables}
			\begin{description}
				\item[Nombre de participants] 8~couples et 4~groupes
				\item[Tâche différente] 2~molécules (1~faiblement et 1~fortement couplée)
				\item[Stratégie] Possibilité ou non d'établir une stratégie
			\end{description}
		\end{myblock}
	\end{myframe}
	\subsubsection{Résultats}
	\begin{myframe}{Analyse}
		\begin{myfigure}
			\mylegend{%
				\myleg{binome}{myblue}%
				\myleg{quadrinôme}{myblue!70}%
			}
			\begin{myboxgraph}(2,0.6\textwidth)[200]{scénario}(900,3cm){temps~(s)}
				\myboxplot{exp3-time-molecule-group.csv}
			\end{myboxgraph}
			\mycaption[fig-exp3-TempsDeRealisationDesScenariosEnFonctionDuNombreDeParticipants]{Temps de réalisation des scénarios en fonction du nombre de participants}
		\end{myfigure}
		\begin{myblock}{Travail collaboratif}
			\begin{itemize}
				\item Pas de différences entre couples et groupes
				\item Conflits très importants dans les groupes
			\end{itemize}
		\end{myblock}
	\end{myframe}
	\begin{myframe}{Analyse}
		\begin{myfigure}
			\mylegend{%
				\myleg{binôme}{myblue}%
				\myleg{quadrinôme}{myblue!70}%
			}
			\begin{myboxgraph}(2,0.6\textwidth)[200]{\textit{brainstorming}}(900,3cm){temps~(s)}
				\myboxplot{exp3-time-brainstorm-group.csv}
			\end{myboxgraph}
			\mycaption[fig-exp3-TempsDeRealisationDesScenariosEnFonctionDesGroupesAvecOuSansBrainstorming]{Temps de réalisation des scénarios en fonction des groupes avec ou sans \textit{brainstorming}}
		\end{myfigure}
		\begin{myblock}{Pré-élaboration d'une stratégie}
			\begin{itemize}
				\item La pré-élaboration d'une stratégie est nécessaire pour un groupe
				\item L'organisation dans un couple n'apporte rien
				\item Sans stratégie, la perte d'efficacité est due aux conflits
			\end{itemize}
		\end{myblock}
	\end{myframe}
	\subsubsection{Synthèse}
	\section{Aide au travail collaboratif}
	\subsection{Étude~\mynum{4} -- Assistance haptique et stratégie de travail}
	\begin{myframe}{Sommaire}
		\tableofcontents[sectionstyle=show/shaded,subsectionstyle=show/shaded/hide,subsubsectionstyle=show/show/hide]
	\end{myframe}
	\subsubsection{Synthèse des études effectuées}
	\begin{myframe}{Synthèse des études effectuées et solutions}
		\begin{myfigure}
			\psset{xunit=0.95cm,yunit=0.8cm}
			\begin{myps}(-6,-3)(6,4)
				\uput[-90](-2.5,4){\textcolor{black!50}{\Large Problématiques}}
				\uput[-90](2.5,4){\textcolor{black!50}{\Large Solutions}}
				\psset{arrowlength=1}
				\onslide<1->{%
					\myunode[0][shadowcolor=myred](-5,2.5)[complexe-problem]{\onslide<2->{\onslide<4-9>{\color{black!25}}Complexité de la tâche}}[4cm]
					\myunode[0][shadowcolor=myred](-5,1.5)[strategy-problem]{\onslide<3->{\onslide<4-9>{\color{black!25}}Stratégie de travail}}[4cm]
					\psbrace*[linecolor=myred,ref=lC,nodesepA=-1pt](-5,3)(-5,1){\rotateright{\textcolor{myred}{\tiny Étude~1}}}%
				}
				\onslide<4->{%
					\myunode[0][shadowcolor=mygreen](-5,0.5)[distribution-problem]{\onslide<5->{\onslide<7-9>{\color{black!25}}Charge de travail}}[4cm]
					\myunode[0][shadowcolor=mygreen](-5,-0.5)[conflict-problem]{\onslide<6->{\onslide<7-9>{\color{black!25}}Conflits de coordination}}[4cm]
					\psbrace*[linecolor=mygreen,ref=lC,nodesepA=-1pt](-5,1)(-5,-1){\rotateright{\textcolor{mygreen}{\tiny Étude~2}}}%
				}
				\onslide<7->{%
					\myunode[0][shadowcolor=myblue](-5,-1.5)[identification-problem]{\onslide<8->{Paresse sociale}}[4cm]
					\myunode[0][shadowcolor=myblue](-5,-2.5)[brainstorming-problem]{\onslide<9->{\textit{Brainstorming}}}[4cm]
					\psbrace*[linecolor=myblue,ref=lC,nodesepA=-1pt](-5,-1)(-5,-3){\rotateright{\textcolor{myblue}{\tiny Étude~3}}}%
				}
				\onslide<10->{%
					\psbrace*[linecolor=black!50,ref=lC,nodesepA=1pt](5,-3)(5,3){\rotateright{\textcolor{black!50}{\tiny Étude~4}}}
				}
				\onslide<11->{%
					\myunode[180][shadowcolor=black!50](5,2.5)[complexe-solution]{Tâche de \textit{docking}}[4cm]%
					\ncline[linewidth=4pt,linecolor=myred!25]{->}{complexe-problem}{complexe-solution}
				}
				\onslide<12->{%
					\myunode[180][shadowcolor=black!50](5,1.5)[strategy-solution]{Manipulation de résidu}[4cm]%
					\ncline[linewidth=4pt,linecolor=myred!25]{->}{strategy-problem}{strategy-solution}
				}
				\onslide<13->{%
					\myunode[180][shadowcolor=black!50](5,0.5)[distribution-solution]{Distribuer les rôles}[4cm]%
					\ncline[linewidth=4pt,linecolor=mygreen!25]{->}{distribution-problem}{distribution-solution}
				}
				\onslide<14->{%
					\myunode[180][shadowcolor=black!50](5,-0.5)[conflict-solution]{Solutions haptiques}[4cm]%
					\ncline[linewidth=4pt,linecolor=mygreen!25]{->}{conflict-problem}{conflict-solution}
				}
				\onslide<15->{%
					\myunode[180][shadowcolor=black!50](5,-1.5)[identification-solution]{Identifier les rôles}[4cm]%
					\ncline[linewidth=4pt,linecolor=myblue!25]{->}{identification-problem}{identification-solution}
				}
				\onslide<16->{%
					\myunode[180][shadowcolor=black!50](5,-2.5)[brainstorming-solution]{Phase exploratoire}[4cm]%
					\ncline[linewidth=4pt,linecolor=myblue!25]{->}{brainstorming-problem}{brainstorming-solution}
				}
			\end{myps}
			\mycaption[fig-SyntheseDesProblematiques]{Synthèse des problématiques}
		\end{myfigure}
	\end{myframe}
	\subsubsection{Objectifs}
	\begin{myframe}{Objectifs}
		\begin{myblock}{Objectif principal}
			Proposer et évaluer des outils haptiques pour assister la coordination
		\end{myblock}
		\begin{myblock}{Hypothèses}
			\begin{enumerate}
				\item Performances améliorées par l'assistance haptique
					\begin{itemize}
						\item Rapidité d'exécution
						\item Qualité de la solution atteinte
					\end{itemize}
				\item L'assistance haptique améliore la communication
					\begin{itemize}
						\item Temps de réaction réduits
						\item Meilleure compréhension des intentions de chacun
					\end{itemize}
				\item Les experts sont satisfaits des outils proposés
					\begin{itemize}
						\item Évaluer les outils proposés
						\item Identifier les faiblesses
					\end{itemize}
			\end{enumerate}
		\end{myblock}
	\end{myframe}
	\subsubsection{Protocole expérimental}
	\begin{myframe}{Protocole}
		\begin{myblock}{Sujets}
			\begin{itemize}
				\item 24~participants
				\item Sujets avec expérience sur la plate-forme
				\item Étude intra-population
			\end{itemize}
		\end{myblock}
		\begin{myblock}{Variables}
			\begin{description}
				\item[Nombre de participants] 8~trinômes
				\item[Tâche différente] 2~molécules (1~déformation et 1~docking moléculaire)
				\item[Assistance] Avec ou sans assistance haptique
			\end{description}
		\end{myblock}
	\end{myframe}
	\subsubsection{Résultats}
	\begin{myframe}{Analyse}
		\begin{myfigure}
			\mylegend{%
				\myleg{sans assistance}{myblue}%
				\myleg{avec assistance}{myblue!70}%
			}
			\begin{myboxgraph}(2,0.9\textwidth)[100]{scénario}(575,3cm){temps~(s)}
				\myboxplot{exp4-rmsd-time-molecule-haptic.csv}
			\end{myboxgraph}
			\mycaption[fig-exp4-TempsPourAtteindreLeScoreRMSDMinimumAvecEtSansHaptiquePourChaqueScenario]{Temps pour atteindre le score \myRMSD minimum avec et sans haptique pour chaque scénario}
		\end{myfigure}
		\begin{myblock}{Assistance haptique}
			\begin{itemize}
				\item Pas de différences sur les tâches simples
				\item Apport important sur les tâches complexes
			\end{itemize}
		\end{myblock}
	\end{myframe}
	\begin{myframe}{Analyse}
		\begin{myfigure}
			\begin{myboxgraph}(2,0.6\textwidth)[2]{haptique}(12,3cm){temps~(s)}
				\myboxplot{exp4-shake-time-haptic.csv}
			\end{myboxgraph}
			\mycaption[fig-exp4-TempsMoyenDAcceptationDUneDesignationAvecEtSansHaptique]{Temps moyen d'acceptation d'une désignation avec et sans haptique}
		\end{myfigure}
		\begin{myblock}{Communication haptique}
			\begin{itemize}
				\item Amélioration du temps de réaction
				\item Communication haptique et non verbale
			\end{itemize}
		\end{myblock}
	\end{myframe}
	\subsubsection{Synthèse}
	\section{Conclusion et perspectives}
	\begin{myframe}{Conclusion}
		\begin{myblock}{Travail collaboratif}
			\begin{itemize}
				\item Adapté pour l'appréhension de tâches très complexes
				\item Nécessité d'améliorer les canaux de communication
			\end{itemize}
		\end{myblock}
		\begin{myblock}{Communication haptique}
			\begin{itemize}
				\item Remplace la communication verbale dans certains cas
				\item Plus efficace et plus rapide
			\end{itemize}
		\end{myblock}
		\begin{myblock}{Plateforme \myShaddock}
			\begin{itemize}
				\item Plateforme validée
				\item Des améliorations sont encore nécessaires
			\end{itemize}
		\end{myblock}
	\end{myframe}
	\begin{myframe}{Perspectives}
		\begin{myblock}{Travail collaboratif}
			\begin{itemize}
				\item Collaboration distante
				\item Collaboration multi-experts
				\item Apprentissage en collaboration
			\end{itemize}
		\end{myblock}
		\begin{myblock}{Expérimenter le travail collaboratif}
			\begin{itemize}
				\item Comment mesurer les conflits de coordination et de communication ?
				\item Comment définir un protocole expérimental pour le collaboratif ?
			\end{itemize}
		\end{myblock}
	\end{myframe}
	\subsection{Synthèse}
	\subsection{Perspectives}
	\begin{myframe}{Questions}
		Merci pour votre attention
	\end{myframe}
	\begin{myframe}{Références}
		\defbibfilter{all}{\keyword{Simard}}
		\printbibliography[filter=all]
	\end{myframe}
\end{document}
