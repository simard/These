\documentclass[english,french,dvips,10pt]{mybeamer}
\usepackage{my}
\usepackage{mydate}
\usepackage{mycolor}
\usepackage{myfloat}
\usepackage[biblatex]{mybib}
\usepackage{myps}
\usepackage{myuml}
\usepackage{verbatim}

\makeatletter
% Modify the bibliography style
\newcounter{mymaxcitenames}
\AtBeginDocument{%
	\setcounter{mymaxcitenames}{\value{maxnames}}%
}
\renewbibmacro{begentry}{%
	\printtext[brackets]{%
		\defcounter{maxnames}{\value{mymaxcitenames}}%
		\printnames{labelname}~\usebibmacro{cite:labelyear+extrayear}%
	}%
	\newline%
}
% AAA
\newcommand{\myACER}{\textsc{acer}\xspace}
\newcommand{\myAlanine}{Alanine\xspace}
\newcommand{\myanalysis}[1]{\input{files/#1}\%}
\newcommand{\myangstrom}{\AA ngström\xspace}
\newcommand{\myanova}[1]{\input{files/#1}}
\newcommand{\myatom}[2][]{%
	{%
		\ifstrempty{#1}%
			{\makefirstuc{\textsf{#2}}}%
			{\textcolor{#1}{\makefirstuc{\textsf{#2}}}}%
		\xspace%
	}%
}
\newcommand{\myAudacity}{\textsc{audacity}\myregistered}% No '\xspace' because of already one in '\myregistered'
% CCC
\newcommand{\mycarbon}{\myatom[mycarboncolor]{C}}
\newcommand{\myCasioXJ}{\textsc{Casio xj}\xspace}
\newcommand{\myCHARMM}{\textsc{charmm}\xspace}
\newcommand{\myChimera}{\textsc{chimera}\xspace}
\newcommand{\myClayWorks}{\textsc{Clayworks}\xspace}
\newcommand{\mycondition}[1]{$\left(\mathcal{C}_{#1}\right)$\xspace}
\newcommand{\myCPK}{\textsc{cpk}\xspace}
% DDD
\newcommand{\myDesktop}{\myPHANToM Desktop\myregistered}% No '\xspace' because of already one in '\myregistered'
% FFF
\newcommand{\myfeuillet}{feuillet-$\beta$\xspace}
\WithSuffix\newcommand\myfeuillet*{feuillets-$\beta$\xspace}
\newcommand{\myform}[1]{\textbf{\sffamily\MakeUppercase{#1}}}
% GGG
\newcommand{\myGhost}{\textsc{Ghost}\xspace}
\newcommand{\myGromacs}{\textsc{Gromacs}\xspace}
\newcommand{\mygroup}[1]{$\left(\mathcal{G}_{#1}\right)$\xspace}
% HHH
\newcommand{\myHaption}{\textsc{Haption}\xspace}
\newcommand{\myHawthorne}{\myemph{Hawthorne Works}\xspace}
\newcommand{\myHBonds}{\textit{HBonds}\xspace}
\newcommand{\myhelice}{hélice-$\alpha$\xspace}
\WithSuffix\newcommand\myhelice*{hélices-$\alpha$\xspace}
\newcommand{\myhypothesis}[1]{$\left(\mathcal{H}_{#1}\right)$\xspace}
% III
\newcommand{\myIntelCore}{Intel\myregistered Core\mytrademark~2 \textsc{q9450} (\mynum[GHz]{2.66})\xspace}
% JJJ
\newcommand{\myJmol}{\textsc{Jmol}\xspace}
% LLL
\newcommand{\myLCD}{\textsc{lcd}\xspace}
\newcommand{\myLicorice}{\textit{Licorice}\xspace}
\newcommand{\myLinux}{\textsc{Linux}\xspace}
% MMM
\newcommand{\myMacOS}{Mac~\textsc{OS}\xspace}
\newcommand{\myMDDriver}{\textsc{MDDriver}\xspace}
% NNN
\newcommand{\myNewRibbon}{\textit{NewRibbon}\xspace}
\def\mynode{%
	\@ifnextchar[{\mynode@i}{\mynode@i[style=nodestyle]}%
}
\def\mynode@i[#1](#2,#3)[#4]#5{%
	\rput(#2,#3){\Rnode{#4}{\psframebox[style=nodestyle,#1]{\vphantom{pÉ}#5}}}%
}
\newcommand{\mynytrogen}{\myatom[mynytrogencolor]{A}}
\newcommand{\myNusE}{\textsc{NusE}\xspace}
\newcommand{\myNusENusG}{\textsc{NusE:NusG}\xspace}
\newcommand{\myNusG}{\textsc{NusG}\xspace}
% OOO
\newcommand{\myOmni}{\myPHANToM Omni\myregistered}% No '\xspace' because of already one in '\myregistered'
\newcommand{\myOpenHaptics}{\textsc{OpenHaptics}\mytrademark}% No '\xspace' because of already one in '\mytrademark'
\newcommand{\myoxygen}{\myatom[myoxygencolor]{O}}
% PPP
\newcommand{\myPC}{\textsc{pc}\xspace}
\newcommand{\myPDB}{\textsc{pdb}\xspace}
\newcommand{\myPDBbase}{\emph{Protein~DataBase}\xspace}
\newcommand{\myPDBlink}[2]{\href{#1}{\textsc{\MakeLowercase{#2}}}}
\newcommand{\myPHANToM}{\textsc{phant}o\textsc{m}\xspace}
\newcommand{\myPremium}{\myPHANToM Premium\myregistered}% No '\xspace' because of already one in '\myregistered'
\newcommand{\myPrion}{Prion\xspace}
\newcommand{\myPSF}{\textsc{psf}\xspace}
\newcommand{\mypvalue}{$p$-value\xspace}
\newcommand{\myPyMOL}{\textsc{p}y\textsc{mol}\xspace}
% RRR
\newcommand{\myRAM}[2][Go]{\mynum[#1]{#2} de \textsc{ram}}
\newcommand{\myRasmol}{\textsc{RasMol}\xspace}
\newcommand{\myresidue}[1]{$\left(\mathcal{R}_{#1}\right)$\xspace}
% SSS
\newcommand{\myscenario}[1]{\textsc{#1}}
\newcommand{\mySensAble}{\textsc{SensAble}\xspace}
\newcommand{\myShaddock}{\textsc{Shaddock}\xspace}
\newcommand{\mySony}{\textsc{sony}\myregistered}% No '\xspace' because of already one in '\myregistered'
\newcommand{\mySpaceNavigator}{SpaceNavigator\myregistered}% No '\xspace' because of already one in '\myregistered'
\newcommand{\mysubject}[1]{$\mathcal{S}_#1$}
\newcommand{\mysulfur}{\myatom[mysulfurcolor]{S}}
\newcommand{\mysummary}[1]{\input{files/#1}}
% TTT
\newcommand{\myTCPIP}{\textsc{tcp/ip}\xspace}
\newcommand{\myThreeD}{\textsc{3d}\xspace}
\newcommand{\mytool}[1]{\myemph{#1}}
\newcommand{\myTRPCAGE}{\textsc{trp-cage}\xspace}
\newcommand{\myTRPZIPPER}{\textsc{trp-zipper}\xspace}
% UUU
\newcommand{\myUbiquitin}{Ubiquitin\xspace}
\newcommand{\myUbuntu}{\textsc{Ubuntu}~v$10.04$\xspace}
\newcommand{\myUSB}{\textsc{usb}\xspace}
\newcommand{\myuser}[1]{$\mathcal{#1}$}
% VVV
\newcommand{\myvar}[2]{$\left(\mathcal{V}_{\mathrm{#1}#2}\right)$\xspace}
\newcommand{\myvard}[1]{\myvar{d}{#1}}
\newcommand{\myvari}[1]{\myvar{i}{#1}}
\newcommand{\myVGA}{\textsc{vga}\xspace}
\newcommand{\myVirtuose}{\textsc{Virtuose}\mytrademark~\textsc{6d}\mynum{35}--\mynum{45}\xspace}
% WWW
\newcommand{\myWindows}{\textsc{Windows}\xspace}

% Needed lengths
\newlength{\mywidth}
\newlength{\myheight}

% PSTricks style
\newpsstyle{nodestyle}{framearc=0.25,shadow=true,shadowcolor=myblue,blur=true}
\makeatother


\hypersetup{%
	pdftitle={Collaboration haptique étroitement couplée pour la déformation moléculaire interactive},%
	pdfauthor={Jean SIMARD},%
	pdfkeywords={collaboration,haptique,environnement virtuel,simulation moléculaire},%
	pdflang={FR-fr},%
	pdfsubject={Soutenance de thèse en informatique}%
}

\newcommand{\schemafactor}{1}
\newlength{\schemaunit}

\setbeamercolor*{progressbar primary}{fg=myred}
\setbeamercolor*{structure}{fg=myred}
\setbeamercolor*{example text}{fg=white,bg=myred}
\setbeamercolor*{section in head/foot}{parent=palette primary,fg=black}
\setbeamercolor*{subsection in head/foot}{parent=palette primary,fg=myred}
\setbeamercolor*{alerted text}{fg=myred}
\setbeamercolor*{block title}{fg=white,bg=mygreen}
\setbeamercolor*{block body}{bg=black!20}
\setbeamercolor*{block title alerted}{fg=white,bg=myred}
\setbeamercolor*{block body alerted}{bg=black!20}
\setbeamercolor*{block title example}{fg=white,bg=mygreen}
\setbeamercolor*{block body example}{bg=black!20}
\newenvironment<>{myblock}[1]{%
	\begin{alertblock}{#1}%
}{%
	\end{alertblock}%
}
\newenvironment<>{myplusblock}[1]{%
	\setbeamercolor*{structure}{fg=mygreen}
	\begin{block}{#1}%
}{%
	\end{block}%
	\setbeamercolor*{structure}{fg=myred}
}
\newenvironment<>{myminusblock}[1]{%
	\begin{alertblock}{#1}%
}{%
	\end{alertblock}%
}

\addglobalbib[datatype=bibtex]{biblio.bib}

\title{Collaboration haptique étroitement couplée pour la déformation moléculaire interactive}
\author[J. \myname{Simard}]{Jean \myname{Simard}}
\institute{\textsc{cnrs-limsi}}
\school{Université de \myname{Paris}-Sud}
\logo{%
	\myimage[height=1cm]{logo-ups}%
	\hspace{1em}%
	\myimage[height=1cm]{logo-limsi}%
}
\date{\mydate[datestyle=long]{01/02/2012}}

\begin{document}
	\transfade[duration=0.1]
	\begin{frame}
		\titlepage
	\end{frame}
	\logo{}
	\begin{frame}{Sommaire}
		\tableofcontents[hideallsubsections]
	\end{frame}
	\section{Introduction}
	\subsection{\protect\textit{Docking} moléculaire}
	\begin{frame}{Définition}
		\begin{myblock}{\textit{Docking} moléculaire}
			ou \myemph{amarrage moléculaire}, consiste à trouver l'orientation et la conformation optimale permettant d'assembler \mynum{2}~molécules.
		\end{myblock}
		\begin{myfigure}
			\psset{unit=1cm}
			\begin{myps}(-5,-2)(6,2)
				\rput(-3,1){\myimage[width=0.120518359\paperwidth]{sota-docking-molecule-A}}
				\mycircleletter(-4.5,1){A}
				\rput(-3,-1){\myimage[width=0.1\paperwidth,angle=90]{sota-docking-molecule-B}}
				\mycircleletter(-4.5,-1){B}
				\rput(-3,0){\Huge\bfseries +}
				\rput(3,0){\myimage[width=0.186177106\paperwidth]{sota-docking-complex-AB}}
				\psline[linewidth=10pt,linecolor=myblue!70]{->}(-1,0)(1,0)
				\mycircleletter(5,0.75){A}
				\rput(5,0){\Large\bfseries +}
				\mycircleletter(5,-0.75){B}
			\end{myps}
			\mycaption[fig-IllustrationDeLAmarrageMoleculaire]{Illustration de l'amarrage moléculaire}
		\end{myfigure}
	\end{frame}
	\subsection{Distribution des charges de travail}
	\subsection{Collaboration en environnement virtuel}
	\subsection{Synthèse}
	\begin{frame}{Déroulement}
		\begin{myfigure}
			\psset{xunit=1cm,yunit=1.1cm}
			\begin{myps}(-2.5,-0.5)(2.5,5)
				\only<1>{
					\mynode[fillstyle=solid,fillcolor=myred!25](0,4)[Search]{Recherche}
					\mycirclenumber(0,4){1}
				}
				\only<2>{
					\mynode[fillstyle=solid,fillcolor=myred!25](0,3)[Selection]{Sélection}
					\mycirclenumber(0,3){2}
				}
				\only<3>{
					\mynode[fillstyle=solid,fillcolor=myred!25](0,2)[Manipulation]{Déformation}
					\mycirclenumber(0,2){3}
				}
				\only<4>{
					\mynode[fillstyle=solid,fillcolor=myred!25](0,1)[Evaluation]{Évaluation}
					\mycirclenumber(0,1){4}
				}
				\only<5>{
					\mycirclenumber[180](0,2.5){5}
				}
				\only<5->{
					\mynode(0,1)[Evaluation]{Évaluation}
				}
				\only<4->{
					\mynode(0,2)[Manipulation]{Déformation}
					\ncline{->}{Manipulation}{Evaluation}
				}
				\only<3->{
					\mynode(0,3)[Selection]{Sélection}
					\ncline{->}{Selection}{Manipulation}
				}
				\only<2->{
					\mynode(0,4)[Search]{Recherche}
					\ncline{->}{Search}{Selection}
				}
				\only<5->{
					\ncloop[loopsize=4em,angleA=-90,angleB=90,linearc=0.05,armA=0.2]{->}{Evaluation}{Search}
				}
				\only<6>{
					\mynode[fillstyle=solid,fillcolor=myred!25](0,0)[Objective]{Objectif atteint}
					\mycirclenumber(0,0){6}
					\ncline{->}{Evaluation}{Objective}
				}
			\end{myps}
			\mycaption[fig-ProcessusDeDeformationMoleculaire]{Processus de déformation moléculaire}
		\end{myfigure}
	\end{frame}
	\section{\myShaddock}
	\begin{frame}{Sommaire}
		\tableofcontents[sectionstyle=show/shaded,subsectionstyle=show/show/hide,subsubsectionstyle=show/show/hide]
	\end{frame}
	\subsection{Structure globale}
	\begin{frame}{\myShaddock}
		\begin{myfigure}
			\psset{xunit=0.0666667\textwidth,yunit=0.06\textheight}
			\psset{framearc=.1,shadow=true,blur=true}
			\begin{myps}(-7.5,-5)(7.5,6)
				\psframe*[linecolor=mygreen!5,shadow=false](-7.4,-5)(-2.6,6)
				\psframe*[linecolor=myblue!5,shadow=false](-2.4,-5)(2.4,6)
				\psframe*[linecolor=myred!5,shadow=false](2.6,-5)(7.4,6)
				\uput[-90](-5,6){\large\textcolor{mygreen!25}{Simulation}}
				\uput[-90](0,6){\large\textcolor{myblue!25}{Visualisation}}
				\uput[-90](5,6){\large\textcolor{myred!25}{Interaction}}
				\uput[-90](0,5){%
					\myumlnode*<PCUtilisateur>{\vphantom{pÉ}\scriptsize Nœud principal}{%
						\myumlcomponent<VMD>[\tiny application]{\scriptsize\myVMD}%
					}%
				}
				\uput[-90](-5,5){%
					\myumlnode*<ServeurNAMD>{\vphantom{pÉ}\scriptsize Nœud \myNAMD}{%
						\begin{psmatrix}[rowsep=1]%
							\myumlcomponent<NAMD>[\tiny programme]{{\scriptsize\texttt{namd2}}} \\%
							\myumlcomponent<FichierSimulation>[\tiny fichier]{%
								\\[-1ex]%
								\begin{psmatrix}[rowsep=0]%
									\scriptsize Données de\\\scriptsize simulation%
								\end{psmatrix}%
							}
						\end{psmatrix}%
					}%
				}
				\uput[-90](5,5){%
					\myumlnode*<ServeurVRPN1>{\vphantom{pÉ}\scriptsize Nœud \myVRPN}{%
						\myumlcomponent<VRPN1>[\tiny programme]{{\scriptsize\texttt{vrpn\_server}}}%
					}%
				}
				\uput[-90](5,0){%
					\myumlnode<PHANToM1>[\tiny\myOmni]{\vphantom{pÉ}\scriptsize Interface}%
				}
				\rput(5,-4){\Large$\vdots$}
				\uput[-90](0,0){%
					\myumlnode<VideoProjecteur>[\tiny vue partagée]{\scriptsize Vidéoprojecteur}
				}
				\psset{shadow=false}

				\myumlrealization[angleA=-90,angleB=90]{NAMD}{FichierSimulation}[nccurve]%
				\myumlrealization[angleA=-90,angleB=90]{VRPN1}{PHANToM1}[nccurve]%
				\myumlinterface[angleA=-90,angleB=90,ArrowInsidePos=0.5]{VMD}{VideoProjecteur}[nccurve]
				\myumlinterface[angleA=0,angleB=180,offsetB=-8pt,ArrowInsidePos=0.4]{NAMD}{VMD}[nccurve]
				\myumlinterface[angleA=180,angleB=0,ncurvA=1.5,offsetA=8pt,ArrowInsidePos=0.6]{VRPN1}{VMD}[nccurve]
			\end{myps}
			\mycaption[fig-Shaddock-DiagrammeDeDeploiementUMLDeLaPlateformeShaddock]{Diagramme de déploiement \myUML de la plateforme \myShaddock}
		\end{myfigure}
	\end{frame}
	\subsection{Structure du module principal}
	\begin{frame}{\myVMD}
		\begin{myfigure}
			\psset{unit=0.05\textwidth}
			\begin{myps}(-10,-5)(8,5)
				\rput(0,0){%
					\myumlcomponent*[framesep=10pt,framearc=0,shadow=false]<VMD>[application]{\myVMD}{%
						\psset{framesep=5pt,framearc=.1,shadow=true,blur=true}%
						\psframebox[linestyle=none,fillstyle=none,shadow=false]{%
							\begin{psmatrix}[rowsep=1]%
								\myumlcomponent<IMD>[extension]{\myIMD}%
								\hspace{3em}%
								\myumlcomponent<VRPNclient>[fonction]{Client \myVRPN} \\%
								\myumlcomponent<Renderer>[fonction]{Moteur de rendus}%
							\end{psmatrix}%
						}%
					}%
				}

				\psset{fillstyle=none,shadow=false}
				\myumlrealization[angleA=90,angleB=-90]{Renderer}{IMD}[nccurve]%
				\myumlrelation[angleA=180,angleB=0,offsetA=8pt]{VRPNclient}[-135]<*>{IMD}[-45]<*>[nccurve]%
				\myumlinterface[angleA=180,angleB=180,outAngleB=0,offsetA=8pt,ArrowInside={}]{IMD}{VMD}[nccurve]
				\myumlinterface[angleA=0,angleB=0,outAngleB=180,ArrowInside={}]{VRPNclient}{VMD}[nccurve]
				\myumlinterface[angleA=-90,angleB=-90,outAngleB=90,offsetA=8pt,ArrowInside={}]{Renderer}{VMD}[nccurve]
			\end{myps}
			\mycaption[fig-Shaddock-DiagrammeDeComposantUMLDuNoeudVMD]{Diagramme de composant \myUML du nœud \myVMD}
		\end{myfigure}
	\end{frame}
	\section{Étude du travail collaboratif}
	\subsection{Étude~\mynum{1} -- Recherche collaborative}
	\begin{frame}{Sommaire}
		\tableofcontents[sectionstyle=show/shaded,subsectionstyle=show/show/hide,subsubsectionstyle=show/show/hide]
	\end{frame}
	\subsubsection{Travaux existants}
	\begin{frame}{Travaux existants}
		\begin{columns}
			\begin{column}{0.6\paperwidth}
				\begin{myplusblock}{Sélection moléculaire}
					\begin{itemize}
						\item Logiciels de visualisation \mycite{Humphrey-1996,Delano-2002}
						\item Solution de réalité virtuelle \mycite{Polys-2004}
						\item Manipulation moléculaire interactive \mycite{Delalande-2010}
					\end{itemize}
				\end{myplusblock}
				\begin{myminusblock}{Problématique}
					\begin{itemize}
						\item Pas d'exploration collaborative dans un environnement moléculaire temps-réel
					\end{itemize}
				\end{myminusblock}
			\end{column}
			\begin{column}{0.3\paperwidth}
				\begin{myfigure}
					\myimage[width=0.3\paperwidth]{exp1-delalande-2010}
					\mycaption[fig-PlateformeFVNANO]{Plateforme \textsc{fvnano}}
				\end{myfigure}
			\end{column}
		\end{columns}
	\end{frame}
	\subsubsection{Objectifs}
	\begin{frame}{Objectifs}
		\begin{myblock}{Objectif principal}
			Observer les contraintes liées au travail collaboratif et souligner les avantages
		\end{myblock}
		\begin{myblock}{Hypothèses}
			\begin{enumerate}
				\item Amélioration des performances en binôme
					\begin{itemize}
						\item Comparer les performances en collaboration et seul
						\item Valider le contexte de travail (tâche complexe)
					\end{itemize}
				\item Stratégies de travail dépendantes de la personnalité
					\begin{itemize}
						\item Identifier et caractériser les stratégies de travail
						\item Identifier les conflits de coordination et de communication
					\end{itemize}
				\item Bonne utilisabilité de la plate-forme
					\begin{itemize}
						\item Évaluer les outils proposés
						\item Identifier les faiblesses
					\end{itemize}
			\end{enumerate}
		\end{myblock}
	\end{frame}
	\subsubsection{Protocole expérimental}
	\begin{frame}{La plate-forme}
		\psset{xunit=0.666666666667\paperwidth,yunit=0.4\paperwidth}
		\begin{myfigure}
			\begin{myps}(0,0)(1,1)
				\rput(0.5,0.5){\myimage[width=0.666666666667\paperwidth,height=0.4\paperwidth]{exp1-schema}}
				\rput(0.2,0.8){\myimage[width=0.266666666667\paperwidth,height=0.16\paperwidth]{exp1-illustration}}
				\pnode(0.54,0.65){tug1}
				\pnode(0.64,0.75){tug1-from}
				\pnode(0.64,0.59){tug2}
				\pnode(0.74,0.69){tug2-from}
				\pnode(0.59,0.62){grab}
				\pnode(0.69,0.72){grab-from}
				\only<2>{
					\ncline[linewidth=5pt,linecolor=myred,nodesepB=2.5pt]{c->}{tug1-from}{tug1}
					\ncline[linewidth=5pt,linecolor=mygreen,nodesepB=2.5pt]{c->}{tug2-from}{tug2}
				}
				\only<3>{
					\ncline[linewidth=5pt,linecolor=myblue,nodesepB=2.5pt]{c->}{grab-from}{grab}
				}
				\only<4>{
					\pspolygon[linewidth=4pt,linecolor=myred](0.6,0.69)(0.61,0.99)(0.87,0.93)(0.83,0.57)
				}
			\end{myps}
			\mycaption[fig-IllustrationDeLaPlateFormeExperimentale1]{Illustration de la plate-forme expérimentale}
		\end{myfigure}
		\only<1>{
			\begin{itemize}
				\item Configuration \alert{colocalisée} et \alert{synchrone}
			\end{itemize}
		}
		\only<2>{
			\begin{itemize}
				\item Outil de déformation de la molécule (Omni de SensAble\myregistered)
			\end{itemize}
		}
		\only<3>{
			\begin{itemize}
				\item Outil pour déplacer la molécule (Omni de SensAble\myregistered)
			\end{itemize}
		}
		\only<4>{
			\begin{itemize}
				\item Vue monoscopique, unique, publique et vidéoprojetée
			\end{itemize}
		}
		\only<5>{
			\begin{itemize}
				\item Communication orale et gestuelle autorisée
			\end{itemize}
		}
		\only<6>{
			\begin{itemize}
				\item Stratégie de travail libre
			\end{itemize}
		}
	\end{frame}
	\begin{frame}{La tâche}
		\renewcommand{\schemafactor}{0.1125}
		\setlength{\schemaunit}{\schemafactor\paperwidth}
		\psset{unit=\schemaunit}
		\begin{myfigure}
			\begin{myps}(-4,-2.3)(4,2.3)
				\rput(-1.75,0){%
					\myimage[height=2\schemaunit]{trp-cage}}
				\rput(1.25,0){%
					\myimage[height=2\schemaunit]{prion}}
				\rput(-3.5,0){%
					\myimage[height=\schemaunit]{pattern1}}
				\rput(-1,1.5){%
					\myimage[width=\schemaunit]{pattern3-8}}
				\rput(1,1.5){%
					\myimage[width=\schemaunit]{pattern2-7}}
				\rput(-1,-1.5){%
					\myimage[width=\schemaunit]{pattern4-9}}
				\rput(1,-1.5){%
					\myimage[width=\schemaunit]{pattern5-10}}
				\rput(3.5,0){%
					\myimage[height=\schemaunit]{pattern6}}

				\psset{framesize=1 1}
				\fnode(-3.5,0){P1}
				\uput[90](-3.5,0.5){Residue~1}
				\fnode(-1,1.5){P38}
				\uput[90](-1,2){Residue~3 and 8}
				\fnode(1,1.5){P27}
				\uput[90](1,2){Residue~2 and 7}
				\fnode(-1,-1.5){P49}
				\uput[-90](-1,-2){Residue~4 and 9}
				\fnode(1,-1.5){P510}
				\uput[-90](1,-2){Residue~5 and 10}
				\fnode(3.5,0){P6}
				\uput[90](3.5,0.5){Residue~6}

				\psset{linecolor=myred}
				\cnode(-1.5,0.3){0.2}{TRPP1}
				\cnode(-2,0.15){0.2}{TRPP38}
				\cnode(-1.25,-0.1){0.2}{TRPP27}
				\cnode(-2.2,-0.5){0.2}{TRPP49}
				\cnode(-1.25,-0.65){0.2}{TRPP510}
				\ncline{-}{P1}{TRPP1}
				\ncline{-}{P38}{TRPP38}
				\ncline{-}{P27}{TRPP27}
				\ncline{-}{P49}{TRPP49}
				\ncline{-}{P510}{TRPP510}

				\psset{linecolor=myblue}
				\cnode(-0.2,0.4){0.2}{PrionP38}
				\cnode(2.8,0.6){0.2}{PrionP27}
				\cnode(0.8,0.2){0.2}{PrionP49}
				\cnode(1.7,-0.7){0.2}{PrionP510}
				\cnode(1.4,0.0){0.2}{PrionP6}
				\ncline{-}{P38}{PrionP38}
				\ncline{-}{P27}{PrionP27}
				\ncline{-}{P49}{PrionP49}
				\ncline{-}{P510}{PrionP510}
				\ncline{-}{P6}{PrionP6}
			\end{myps}
			\mycaption[fig-RepartitionDesResiduesSurLesMoleculesTRPCageEtPrion]{Répartitions des \emph{residues} sur les molécules (TRP-Cage et Prion)}
		\end{myfigure}
	\end{frame}
	\begin{frame}{Protocole}
		\begin{myblock}{Sujets}
			\begin{itemize}
				\item 24~participants
				\item Différents niveaux d'expertise
				\item Étude intra-population
			\end{itemize}
		\end{myblock}
		\begin{myblock}{Variables}
			\begin{description}
				\item[Nombre de participants] un (24~sujets) ou deux (12~couples)
				\item[Taille de la molécule] une petite (TRP-Cage) et une grande (Prion)
				\item[Caractéristiques du \emph{residue}] Forme, nature, position, similarités\dots{}
			\end{description}
		\end{myblock}
	\end{frame}
	\subsubsection{Résultats}
	\begin{frame}{Analyse}
		\begin{myfigure}
			\mylegend{%
				\myleg{Monôme}{myblue}%
				\myleg{Binôme}{myblue!70}%
			}
			\begin{myboxgraph}(10,0.9\textwidth)[100]{résidu}(500,3cm){temps~(s)}
				\myboxplot{exp1-time-residue-group.csv}
			\end{myboxgraph}
			\mycaption[fig-TempsDExecutionEntreLesSujetsSeulsEtLesCouples]{Temps d'exécution entre les sujets seuls et les couples}
		\end{myfigure}
		\begin{myblock}{Travail collaboratif}
			\begin{itemize}
				\item Le travail collaboratif confirme son intérêt pour les tâches complexes
			\end{itemize}
		\end{myblock}
	\end{frame}
	\begin{frame}{Analyse}
		\begin{myfigure}
			\psset{xunit=0.074\textwidth,yunit=0.12cm}
			\begin{myboxgraph}(12,0.9\textwidth)[4]{binôme}(22,3cm){distance~(mm)}
				% Once header are readed, they are defined for other barplot
				% That's why barplots without headers are in first position
				\mybarplot[header=false,barstyle=third-barstyle]{exp1-diff-groups3.csv}
				\mybarplot[header=false,barstyle=second-barstyle]{exp1-diff-groups2.csv}
				\mybarplot[header=true,barstyle=first-barstyle]{exp1-diff-groups1.csv}
				\psset{linecolor=myred,linewidth=1pt,linestyle=solid}
				\psline(0,14)(12,14)
				\psline(0,8)(12,8)
				\psset{linewidth=0.1pt,linecolor=white,fillstyle=solid,fillcolor=myred}
				\uput[180](12,5){\pscharpath{\LARGE\bf\sffamily Champ proche}}
				\uput[180](12,11){\pscharpath{\LARGE\bf\sffamily Champ voisin}}
				\uput[180](12,17){\pscharpath{\LARGE\bf\sffamily Champ distant}}
			\end{myboxgraph}
			\mycaption[fig-DistanceMoyenneEntreLeCurseurDesSujets]{Distance moyenne entre le curseur des sujets}
		\end{myfigure}
		\begin{myblock}{Les groupes}
			\begin{itemize}
				\item Stratégies de travail différentes
				\item Dépendance de la personnalité
				\item Des conflits d'interaction (dans les \emph{mêmes régions})
			\end{itemize}
		\end{myblock}
	\end{frame}
	\begin{frame}{Analyse}
		\begin{myblock}{La communication verbale}
			\begin{itemize}
				\item<2-> Séparation entre temps de recherche et de sélection
				\item<3-> Mesure du temps de communication verbale
			\end{itemize}
		\end{myblock}
		\begin{myfigure}
			\begin{myps}(0,-0.5)(10,2)
				\psset{linewidth=1pt,linecolor=black}%
				\psframe(0,-0.5)(10,0.5)%
				\only<1>{%
					\psbrace[ref=lC,rot=-90,nodesepA=-5,nodesepB=-1ex](10,0.5)(0,0.5){%
						\parbox{10\psxunit}{\centering\textcolor{black}{Temps de la tâche}}%
					}%
				}%
				\psset{fillstyle=solid}%
				\only<2->{%
					\psframe[fillcolor=myblue!70](0,-0.5)(6,0.5)%
					\psframe[fillcolor=myred!70](6,-0.5)(10,0.5)%
				}%
				\only<2>{%
					\psset{ref=lC,rot=-90}%
					\psbrace[nodesepA=-3](6,0.5)(0,0.5){%
						\parbox{6\psxunit}{\centering\textcolor{myblue}{Temps de recherche}}%
					}%
					\psbrace[nodesepA=-2](10,0.5)(6,0.5){%
						\parbox{4\psxunit}{\centering\textcolor{myred}{Temps de sélection}}%
					}%
				}%
				\only<3->{
					\psframe[fillcolor=myblue](1,-0.5)(1.5,0.5)
					\psframe[fillcolor=myblue](3,-0.5)(4.5,0.5)
					\psframe[fillcolor=myblue](4.8,-0.5)(5,0.5)
					\psframe[fillcolor=myred](6.5,-0.5)(7.5,0.5)
					\psframe[fillcolor=myred](8,-0.5)(8.25,0.5)
				}
				\only<3>{
					\pnode(1.25,0.5){verbal1}
					\pnode(3.75,0.5){verbal2}
					\pnode(4.9,0.5){verbal3}
					\pnode(7,0.5){verbal4}
					\pnode(8.125,0.5){verbal5}
					\rput(5,1.5){\Rnode{verbal}{\psframebox[linestyle=none]{\centering Communication verbale}}}
					\psset{linearc=0.1,angleA=90}
					\ncdiagg{<-}{verbal1}{verbal}
					\ncdiagg{<-}{verbal2}{verbal}
					\ncdiagg{<-}{verbal3}{verbal}
					\ncdiagg{<-}{verbal4}{verbal}
					\ncdiagg{<-}{verbal5}{verbal}
				}
			\end{myps}
			\mycaption[fig-SchemaDeLaCommunicationVerbale]{Schéma de la communication verbale}
		\end{myfigure}
	\end{frame}
	\begin{frame}{Analyse}
		\begin{myfigure}
			\mylegend{%
				\myleg{Recherche}{myblue}%
				\myleg{Sélection}{myblue!70}%
			}
			\begin{myboxgraph}(12,0.9\textwidth)[25]{binôme}(100,3cm){temps~(\%)}
				\mybarplot{exp1-timeaudio-groups-searchselection.csv}
			\end{myboxgraph}
			\mycaption[fig-TempsDeCommunication]{Temps de communication}
		\end{myfigure}
		\begin{myblock}{Travail collaboratif}
			\begin{itemize}
				\item La communication prend une place importante
				\item Des conflits de communication (incompréhension, prise de parole\dots{})
			\end{itemize}
		\end{myblock}
	\end{frame}
	\subsubsection{Synthèse}
	\subsection{Étude~\mynum{2} -- Déformation collaborative}
	\begin{frame}{Sommaire}
		\tableofcontents[sectionstyle=show/shaded,subsectionstyle=show/show/hide,subsubsectionstyle=show/show/hide]
	\end{frame}
	\subsubsection{Travaux existants}
	\begin{frame}{Travaux existants}
		\begin{columns}
			\begin{column}{0.6\paperwidth}
				\begin{myplusblock}{Déformation moléculaire}
					\begin{itemize}
						\item Tissus cellulaires \mycite{Peterlik-2009}
						\item Sculpture sur glaise \mycite{Muller-2006,Gorlatch-2009}
					\end{itemize}
				\end{myplusblock}
				\begin{myminusblock}{Problématique}
					\begin{itemize}
						\item Pas de déformation collaborative dans un environnement moléculaire temps-réel
					\end{itemize}
				\end{myminusblock}
			\end{column}
			\begin{column}{0.3\paperwidth}
				\begin{myfigure}
					\myimage[width=0.3\paperwidth]{exp2-peterlik-2009}
					\mycaption[fig-DeformationDeTissusCellulaires]{Déformation de tissus cellulaires}
				\end{myfigure}
			\end{column}
		\end{columns}
	\end{frame}
	\subsubsection{Objectifs}
	\begin{frame}{Objectifs}
		\begin{myblock}{Objectif principal}
			Proposer une tâche suffisamment complexe pour quantifier et qualifier les conflits de coordination
		\end{myblock}
		\begin{myblock}{Hypothèses}
			\begin{enumerate}
				\item Amélioration des performances en binôme pour la déformation
					\begin{itemize}
						\item Coordination étroitement couplée
					\end{itemize}
				\item Binômes plus performants sur les tâches complexes
					\begin{itemize}
						\item Tâches de difficulté variable
						\item Identifier les tâches nécessitant une collaboration
					\end{itemize}
				\item Évaluation du travail collaboratif par les sujets
					\begin{itemize}
						\item Questionnaire pour valider les améliorations de la plate-forme
						\item Évaluation de la configuration de travail collaboratif
					\end{itemize}
			\end{enumerate}
		\end{myblock}
	\end{frame}
	\subsubsection{Protocole expérimental}
	\begin{frame}{La plate-forme}
		\psset{xunit=0.666666666667\paperwidth,yunit=0.4\paperwidth}
		\begin{myfigure}
			\begin{myps}(0,0)(1,1)
				\only<1>{
					\rput(0.5,0.5){\myimage[width=0.666666666667\paperwidth,height=0.4\paperwidth]{exp1-schema}}
					\rput(0.2,0.8){\myimage[width=0.266666666667\paperwidth,height=0.16\paperwidth]{exp1-illustration}}
				}
				\only<2->{
					\rput(0.5,0.5){\myimage[width=0.666666666667\paperwidth,height=0.4\paperwidth]{exp2-schema}}
					\rput(0.2,0.8){\myimage[width=0.266666666667\paperwidth,height=0.16\paperwidth]{exp2-illustration}}
					\pnode(0.59,0.6){grab}
					\pnode(0.69,0.7){grab-from}
					\ncline[linewidth=5pt,linecolor=blue,nodesepB=2.5pt]{c->}{grab-from}{grab}
				}
			\end{myps}
			\mycaption[fig-IllustrationDeLaPlateFormeExperimentale2]{Illustration de la plate-forme expérimentale}
		\end{myfigure}
		\only<1>{
			\begin{itemize}
				\item Utilisation de la première plate-forme modifiée
			\end{itemize}
		}
		\only<2>{
			\begin{itemize}
				\item Outil pour tourner la molécule (SpaceTraveler de 3dconnexion\myregistered{})
			\end{itemize}
		}
	\end{frame}
	\begin{frame}[label={fra-exp2-LaTache}]{La tâche}
		\renewcommand{\schemafactor}{0.055}
		\setlength{\schemaunit}{\schemafactor\paperwidth}
		\psset{unit=\schemaunit}
		\begin{myfigure}
			\begin{myps}(-1,0)(12,9)
				\rput[bl](1,0){\myimage[width=10\schemaunit]{TRP-ZIPPER}}
				\pnode(6.8,3.6){deformed}
				\rput(8.8,2.6){\rnode{deformed-label}{\textcolor{myred}{Deformed molecule}}}
				\pnode(1.8,4){ghost}
				\rput(1.3,5.5){\rnode{ghost-label}{\textcolor{myred}{Ghost molecule}}}
				\psset{linecolor=myblue}
				\cnode(6.2,4.9){1.0}{deformed-residue}
				\rput(7.1,7.0){\rnode{deformed-residue-label}{\textcolor{myblue}{Deformed residue}}}
				\cnode(2.3,1.6){0.8}{ghost-residue}
				\rput(1.2,2.75){\rnode{ghost-residue-label}{\textcolor{myblue}{Ghost residue}}}
				\psset{linecolor=gray}
				\cnode(2.0,6.6){0.8}{fixed-residue}
				\rput(4.0,7.75){\rnode{fixed-residue-label}{\textcolor{gray}{fixed residue}}}
				\psset{linewidth=1pt,linecolor=myred,linearc=.1,arrowsize=0.5pt 3,arrowinset=.2,nodesepA=3pt}
				\ncangle[angleA=90,angleB=0]{c->}{deformed-label}{deformed}
				\psset{nodesepB=0pt}
				\ncdiagg[angleA=-90,angleB=135]{c->}{ghost-label}{ghost}
				\psset{linecolor=myblue}
				\ncdiagg[angleA=-90]{c->}{deformed-residue-label}{deformed-residue}
				\ncdiagg[angleA=-90]{c->}{ghost-residue-label}{ghost-residue}
				\ncdiagg[angleA=180,linecolor=gray]{c->}{fixed-residue-label}{fixed-residue}
				\ncline[linewidth=10pt,linecolor=mygreen]{c->}{deformed-residue}{ghost-residue}
				\psframe*[linecolor=green](-1,8.5)(5,9)
				\psframe*[linecolor=red](5,8.5)(12,9)
				\rput(5.5,8.75){\textcolor{white}{\bf\textsc{rmsd} score indicator}}
				\psframe[linewidth=1pt,linecolor=black](-1,0)(12,9)
			\end{myps}
			\mycaption[fig-LaMoleculeTRPZipperDeformee]{La molécule TRP-Zipper déformée}
		\end{myfigure}
	\end{frame}
	\begin{frame}{Protocole}
		\begin{myblock}{Sujets}
			\begin{itemize}
				\item 36~participants (12~couples et 12~sujets seuls)
				\item Sujets avec différents niveaux d'expertise
				\item Couples choisis pour leurs affinités
				\item Étude inter-population
			\end{itemize}
		\end{myblock}
		\begin{myblock}{Variables}
			\begin{description}
				\item[Complexité de la molécule] 2~molécules (1~petite et 1~grande)
				\item[Outil de déformation] 2~configuration de déformation (\emph{atom} et \emph{residue})
			\end{description}
		\end{myblock}
	\end{frame}
	\subsubsection{Résultats}
	\begin{frame}{Analyse}
		\begin{mytable}
			\begin{tabular}{cp{7cm}c}
				\hline
				Difficulté & Description & Exemple \\
				\hline
				\hline
				\multirow{2}*{Simple} & -- 1~outil est nécessaire & \multirow{2}*{Tâche~1a} \\
				& -- 1~manipulation \\
				\hline
				\multirow{2}*{Avancé} & -- 1~outil est nécessaire mais 2~outils est mieux & \multirow{2}*{Tâche~2a, 2b} \\
				& -- 2~manipulations peuvent être simultanées \\
				\hline
				\multirow{2}*{Expert} 	& -- 2~outils sont nécessaires & \multirow{2}*{Tâche~1b} \\
				& -- 2~manipulations \alert{doivent} être simultanées \\
				\hline
			\end{tabular}
			\mycaption[tab-ClassificationDesTaches]{Classification des tâches}
		\end{mytable}
	\end{frame}
	\begin{frame}{Analyse}
		\begin{myfigure}
			\mylegend{%
				\myleg{Monôme}{myblue}%
				\myleg{Binôme}{myblue!70}%
			}
			\begin{myboxgraph}(4,0.9\textwidth)[50]{scénario}(300,3cm){temps~(s)}
				\myboxplot{exp2-time-task-group.csv}
			\end{myboxgraph}
			\mycaption[fig-exp2-TempsDeRealisationDesScenariosEnFonctionDuNombreDeSujets]{Temps de réalisation des scénarios en fonction du nombre de sujets}
		\end{myfigure}
		\begin{myblock}{Travail collaboratif}
			\begin{itemize}
				\item Les conflits détériorent l'efficacité de la collaboration
				\item Tâches fortement couplées exclusives pour le travail collaboratif
			\end{itemize}
		\end{myblock}
	\end{frame}
	\begin{frame}{Analyse}
		\begin{myfigure}
			\mylegend{%
				\myleg{Monôme}{myblue}%
				\myleg{Binôme}{myblue!70}%
			}
			\begin{myboxgraph}(3,0.9\textwidth)[50]{essai}(200,3cm){temps~(s)}
				\myboxplot{exp2-time-try-group.csv}
			\end{myboxgraph}
			\mycaption[fig-exp2-TempsDeRealisationDeChaqueEssaiEnFonctionDuNombreDeSujets]{Temps de réalisation de chaque essai en fonction du nombre de sujets}
		\end{myfigure}
		\begin{myblock}{Apprentissage}
			\begin{itemize}
				\item Apprentissage plus rapide en couple
				\item Manque de donnée pour une conclusion définitive
			\end{itemize}
		\end{myblock}
	\end{frame}
	\subsubsection{Synthèse}
	\subsection{Étude~\mynum{3} -- Dynamique de groupe}
	\begin{frame}{Sommaire}
		\tableofcontents[sectionstyle=show/shaded,subsectionstyle=show/show/hide,subsubsectionstyle=show/show/hide]
	\end{frame}
	\subsubsection{Travaux existants}
	\begin{frame}{Travaux existants}
		\begin{myplusblock}{Dynamique de groupe}
			\begin{itemize}
				\item facilitation sociale \mycite{Ringelmann-1913}
				\item paresse sociale \mycite{Roethlisberger-1939}
				\item brainstorming \mycite{Osborn-1963,Tuckman-1965}
			\end{itemize}
		\end{myplusblock}
		\begin{myminusblock}{Problématique}
			\begin{itemize}
				\item Aucune étude de dynamique de groupe sur des tâches avec une interaction étroitement couplée
			\end{itemize}
		\end{myminusblock}
	\end{frame}
	\subsubsection{Objectifs}
	\begin{frame}{Objectifs}
		\begin{myblock}{Objectif principal}
			Observer la dynamique de groupe lors d'une coordination étroitement couplée
		\end{myblock}
		\begin{myblock}{Hypothèses}
			\begin{enumerate}
				\item Amélioration des performances en quadrinôme
					\begin{itemize}
						\item Variation de la taille d'un groupe
						\item Quantification des conflits dans des groupes
					\end{itemize}
				\item Émergence d'un meneur
					\begin{itemize}
						\item Observer la dynamique des groupes
						\item Caractériser les différents rôles
					\end{itemize}
				\item Le \textit{brainstorming} améliore les performances
					\begin{itemize}
						\item Période pour organiser le travail
						\item Limiter les conflits \textit{a priori}
					\end{itemize}
			\end{enumerate}
		\end{myblock}
	\end{frame}
	\subsubsection{Protocole expérimental}
	\begin{frame}{La plate-forme}
		\psset{xunit=0.666666666667\paperwidth,yunit=0.4\paperwidth}
		\begin{myfigure}
			\begin{myps}(0,0)(1,1)
				\only<1>{
					\rput(0.5,0.5){\myimage[width=0.666666666667\paperwidth,height=0.4\paperwidth]{exp2-schema}}
					\rput(0.2,0.8){\myimage[width=0.266666666667\paperwidth,height=0.16\paperwidth]{exp2-illustration}}
				}
				\only<2->{
					\rput(0.5,0.5){\myimage[width=0.666666666667\paperwidth,height=0.4\paperwidth]{exp3-schema}}
					\rput(0.2,0.8){\myimage[width=0.266666666667\paperwidth,height=0.16\paperwidth]{exp3-illustration}}
					\pnode(0.52,0.675){tug1}
					\pnode(0.62,0.775){tug1-from}
					\pnode(0.55,0.65){tug2}
					\pnode(0.65,0.75){tug2-from}
					\pnode(0.58,0.63){tug3}
					\pnode(0.68,0.73){tug3-from}
					\pnode(0.62,0.61){tug4}
					\pnode(0.72,0.71){tug4-from}
					\ncline[linewidth=5pt,linecolor=myred,nodesepB=2.5pt]{c->}{tug1-from}{tug1}
					\ncline[linewidth=5pt,linecolor=myblue,nodesepB=2.5pt]{c->}{tug2-from}{tug2}
					\ncline[linewidth=5pt,linecolor=magenta,nodesepB=2.5pt]{c->}{tug3-from}{tug3}
					\ncline[linewidth=5pt,linecolor=mygreen,nodesepB=2.5pt]{c->}{tug4-from}{tug4}
				}
			\end{myps}
			\mycaption[fig-IllustrationDeLaPlateFormeExperimentale3]{Illustration de la plate-forme expérimentale}
		\end{myfigure}
		\only<1>{
			\begin{itemize}
				\item Utilisation de la seconde plate-forme modifiée
			\end{itemize}
		}
		\only<2>{
			\begin{itemize}
				\item Outil de déformation de la molécule (Omni de SensAble\myregistered{})
			\end{itemize}
		}
		\only<3>{
			\begin{itemize}
				\item Pas d'outil de déplacement ou de rotation de la molécule
			\end{itemize}
		}
		\only<4>{
			\begin{itemize}
				\item Utilisation de la troisième plate-forme
			\end{itemize}
		}
	\end{frame}
	\againframe{fra-exp2-LaTache}
	\begin{frame}{Protocole}
		\begin{myblock}{Sujets}
			\begin{itemize}
				\item 16~participants
				\item Sujets avec expérience sur la plate-forme
				\item Étude intra-population
			\end{itemize}
		\end{myblock}
		\begin{myblock}{Variables}
			\begin{description}
				\item[Nombre de participants] 8~couples et 4~groupes
				\item[Tâche différente] 2~molécules (1~faiblement et 1~fortement couplée)
				\item[Stratégie] Possibilité ou non d'établir une stratégie
			\end{description}
		\end{myblock}
	\end{frame}
	\subsubsection{Résultats}
	\begin{frame}{Analyse}
		\begin{myfigure}
			\mylegend{%
				\myleg{Binome}{myblue}%
				\myleg{Quadrinôme}{myblue!70}%
			}
			\begin{myboxgraph}(2,0.6\textwidth)[200]{scénario}(900,3cm){temps~(s)}
				\myboxplot{exp3-time-molecule-group.csv}
			\end{myboxgraph}
			\mycaption[fig-exp3-TempsDeRealisationDesScenariosEnFonctionDuNombreDeParticipants]{Temps de réalisation des scénarios en fonction du nombre de participants}
		\end{myfigure}
		\begin{myblock}{Travail collaboratif}
			\begin{itemize}
				\item Pas de différences entre couples et groupes
				\item Conflits très importants dans les groupes
			\end{itemize}
		\end{myblock}
	\end{frame}
	\begin{frame}{Analyse}
		\begin{myfigure}
			\mylegend{%
				\myleg{Binôme}{myblue}%
				\myleg{Quadrinôme}{myblue!70}%
			}
			\begin{myboxgraph}(2,0.6\textwidth)[200]{\textit{brainstorming}}(900,3cm){temps~(s)}
				\myboxplot{exp3-time-brainstorm-group.csv}
			\end{myboxgraph}
			\mycaption[fig-exp3-TempsDeRealisationDesScenariosEnFonctionDesGroupesAvecOuSansBrainstorming]{Temps de réalisation des scénarios en fonction des groupes avec ou sans \textit{brainstorming}}
		\end{myfigure}
		\begin{myblock}{Pré-élaboration d'une stratégie}
			\begin{itemize}
				\item La pré-élaboration d'une stratégie est nécessaire pour un groupe
				\item L'organisation dans un couple n'apporte rien
				\item Sans stratégie, la perte d'efficacité est due aux conflits
			\end{itemize}
		\end{myblock}
	\end{frame}
	\subsubsection{Synthèse}
	\section{Aide au travail collaboratif}
	\subsection{Étude~\mynum{4} -- Assistance haptique et stratégie de travail}
	\begin{frame}{Sommaire}
		\tableofcontents[sectionstyle=show/shaded,subsectionstyle=show/show/hide,subsubsectionstyle=show/show/hide]
	\end{frame}
	\subsubsection{Synthèse des études effectuées}
	\begin{frame}{Synthèse des études effectuées et solutions}
		\begin{myfigure}
			\psset{xunit=0.95cm,yunit=0.8cm}
			\begin{myps}(-6,-3)(6,4)
				\uput[-90](-2.5,4){\textcolor{black!50}{\Large Problématiques}}
				\uput[-90](2.5,4){\textcolor{black!50}{\Large Solutions}}
				\psset{arrowlength=1}
				\onslide<1->{%
					\myunode[0][shadowcolor=myred](-5,2.5)[complexe-problem]{\onslide<2->{\onslide<4-9>{\color{black!25}}Complexité de la tâche}}[4cm]
					\myunode[0][shadowcolor=myred](-5,1.5)[strategy-problem]{\onslide<3->{\onslide<4-9>{\color{black!25}}Stratégie de travail}}[4cm]
					\psbrace*[linecolor=myred,ref=lC,nodesepA=-1pt](-5,3)(-5,1){\rotateright{\textcolor{myred}{\tiny Étude~1}}}%
				}
				\onslide<4->{%
					\myunode[0][shadowcolor=mygreen](-5,0.5)[distribution-problem]{\onslide<5->{\onslide<7-9>{\color{black!25}}Charge de travail}}[4cm]
					\myunode[0][shadowcolor=mygreen](-5,-0.5)[conflict-problem]{\onslide<6->{\onslide<7-9>{\color{black!25}}Conflits de coordination}}[4cm]
					\psbrace*[linecolor=mygreen,ref=lC,nodesepA=-1pt](-5,1)(-5,-1){\rotateright{\textcolor{mygreen}{\tiny Étude~2}}}%
				}
				\onslide<7->{%
					\myunode[0][shadowcolor=myblue](-5,-1.5)[identification-problem]{\onslide<8->{Paresse sociale}}[4cm]
					\myunode[0][shadowcolor=myblue](-5,-2.5)[brainstorming-problem]{\onslide<9->{\textit{Brainstorming}}}[4cm]
					\psbrace*[linecolor=myblue,ref=lC,nodesepA=-1pt](-5,-1)(-5,-3){\rotateright{\textcolor{myblue}{\tiny Étude~3}}}%
				}
				\onslide<10->{%
					\psbrace*[linecolor=black!50,ref=lC,nodesepA=1pt](5,-3)(5,3){\rotateright{\textcolor{black!50}{\tiny Étude~4}}}
				}
				\onslide<11->{%
					\myunode[180][shadowcolor=black!50](5,2.5)[complexe-solution]{Tâche de \textit{docking}}[4cm]%
					\ncline[linewidth=4pt,linecolor=myred!25]{->}{complexe-problem}{complexe-solution}
				}
				\onslide<12->{%
					\myunode[180][shadowcolor=black!50](5,1.5)[strategy-solution]{Manipulation de résidu}[4cm]%
					\ncline[linewidth=4pt,linecolor=myred!25]{->}{strategy-problem}{strategy-solution}
				}
				\onslide<13->{%
					\myunode[180][shadowcolor=black!50](5,0.5)[distribution-solution]{Distribuer les rôles}[4cm]%
					\ncline[linewidth=4pt,linecolor=mygreen!25]{->}{distribution-problem}{distribution-solution}
				}
				\onslide<14->{%
					\myunode[180][shadowcolor=black!50](5,-0.5)[conflict-solution]{Solutions haptiques}[4cm]%
					\ncline[linewidth=4pt,linecolor=mygreen!25]{->}{conflict-problem}{conflict-solution}
				}
				\onslide<15->{%
					\myunode[180][shadowcolor=black!50](5,-1.5)[identification-solution]{Identifier les rôles}[4cm]%
					\ncline[linewidth=4pt,linecolor=myblue!25]{->}{identification-problem}{identification-solution}
				}
				\onslide<16->{%
					\myunode[180][shadowcolor=black!50](5,-2.5)[brainstorming-solution]{Phase exploratoire}[4cm]%
					\ncline[linewidth=4pt,linecolor=myblue!25]{->}{brainstorming-problem}{brainstorming-solution}
				}
			\end{myps}
			\mycaption[fig-SyntheseDesProblematiques]{Synthèse des problématiques}
		\end{myfigure}
	\end{frame}
	\subsubsection{Objectifs}
	\begin{frame}{Objectifs}
		\begin{myblock}{Objectif principal}
			Proposer et évaluer des outils haptiques pour assister la coordination
		\end{myblock}
		\begin{myblock}{Hypothèses}
			\begin{enumerate}
				\item Performances améliorées par l'assistance haptique
					\begin{itemize}
						\item Rapidité d'exécution
						\item Qualité de la solution atteinte
					\end{itemize}
				\item L'assistance haptique améliore la communication
					\begin{itemize}
						\item Temps de réaction réduits
						\item Meilleure compréhension des intentions de chacun
					\end{itemize}
				\item Les experts sont satisfaits des outils proposés
					\begin{itemize}
						\item Évaluer les outils proposés
						\item Identifier les faiblesses
					\end{itemize}
			\end{enumerate}
		\end{myblock}
	\end{frame}
	\subsubsection{Protocole expérimental}
	\begin{frame}{La plate-forme}
		\psset{xunit=0.666666666667\paperwidth,yunit=0.4\paperwidth}
		\begin{myfigure}
			\begin{myps}(0,0)(1,1)
				\rput(0.5,0.5){\myimage[width=0.666666666667\paperwidth,height=0.4\paperwidth]{exp4-schema}}
				\rput(0.2,0.8){\myimage[width=0.266666666667\paperwidth,height=0.16\paperwidth]{exp4-illustration}}
			\end{myps}
			\mycaption[fig-IllustrationDeLaPlateFormeExperimentale4]{Illustration de la plate-forme expérimentale}
		\end{myfigure}
		\only<1>{
			\begin{itemize}
				\item Outil de déformation de la molécule (Omni de SensAble\myregistered{})
			\end{itemize}
		}
		\only<2>{
			\begin{itemize}
				\item Outil de désignation
			\end{itemize}
		}
		\only<3>{
			\begin{itemize}
				\item Outil de déplacement ou de rotation de la molécule
			\end{itemize}
		}
	\end{frame}
	\begin{frame}{Protocole}
		\begin{myblock}{Sujets}
			\begin{itemize}
				\item 24~participants
				\item Sujets avec expérience sur la plate-forme
				\item Étude intra-population
			\end{itemize}
		\end{myblock}
		\begin{myblock}{Variables}
			\begin{description}
				\item[Nombre de participants] 8~trinômes
				\item[Tâche différente] 2~molécules (1~déformation et 1~docking moléculaire)
				\item[Assistance] Avec ou sans assistance haptique
			\end{description}
		\end{myblock}
	\end{frame}
	\subsubsection{Résultats}
	\begin{frame}{Analyse}
		\begin{myfigure}
			\mylegend{%
				\myleg{Sans assistance}{myblue}%
				\myleg{Avec assistance}{myblue!70}%
			}
			\begin{myboxgraph}(2,0.9\textwidth)[100]{scénario}(575,3cm){temps~(s)}
				\myboxplot{exp4-rmsd-time-molecule-haptic.csv}
			\end{myboxgraph}
			\mycaption[fig-exp4-TempsPourAtteindreLeScoreRMSDMinimumAvecEtSansHaptiquePourChaqueScenario]{Temps pour atteindre le score \myRMSD minimum avec et sans haptique pour chaque scénario}
		\end{myfigure}
		\begin{myblock}{Assistance haptique}
			\begin{itemize}
				\item Pas de différences sur les tâches simples
				\item Apport important sur les tâches complexes
			\end{itemize}
		\end{myblock}
	\end{frame}
	\begin{frame}{Analyse}
		\begin{myfigure}
			\begin{myboxgraph}(2,0.6\textwidth)[2]{haptique}(12,3cm){temps~(s)}
				\myboxplot{exp4-shake-time-haptic.csv}
			\end{myboxgraph}
			\mycaption[fig-exp4-TempsMoyenDAcceptationDUneDesignationAvecEtSansHaptique]{Temps moyen d'acceptation d'une désignation avec et sans haptique}
		\end{myfigure}
		\begin{myblock}{Communication haptique}
			\begin{itemize}
				\item Amélioration du temps de réaction
				\item Communication haptique et non verbale
			\end{itemize}
		\end{myblock}
	\end{frame}
	\subsubsection{Synthèse}
	\section{Conclusion et perspectives}
	\begin{frame}{Conclusion}
		\begin{myblock}{Travail collaboratif}
			\begin{itemize}
				\item Adapté pour l'appréhension de tâches très complexes
				\item Nécessité d'améliorer les canaux de communication
			\end{itemize}
		\end{myblock}
		\begin{myblock}{Communication haptique}
			\begin{itemize}
				\item Remplace la communication verbale dans certains cas
				\item Plus efficace et plus rapide
			\end{itemize}
		\end{myblock}
		\begin{myblock}{Plateforme \myShaddock}
			\begin{itemize}
				\item Plateforme validée
				\item Des améliorations sont encore nécessaires
			\end{itemize}
		\end{myblock}
	\end{frame}
	\begin{frame}{Perspectives}
		\begin{myblock}{Travail collaboratif}
			\begin{itemize}
				\item Collaboration distante
				\item Collaboration multi-experts
				\item Apprentissage en collaboration
			\end{itemize}
		\end{myblock}
		\begin{myblock}{Expérimenter le travail collaboratif}
			\begin{itemize}
				\item Comment mesurer les conflits de coordination et de communication ?
				\item Comment définir un protocole expérimental pour le collaboratif ?
			\end{itemize}
		\end{myblock}
	\end{frame}
	\subsection{Synthèse}
	\subsection{Perspectives}
	\begin{frame}{Questions}
		Merci pour votre attention
	\end{frame}
	\begin{frame}{Références}
		\defbibfilter{all}{\keyword{Simard}}
		\printbibliography[filter=all]
	\end{frame}
\end{document}
