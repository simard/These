\documentclass[english,french,dvips,trans]{mybeamer}
\usepackage{my}
\usepackage{mydate}
\usepackage{mycolor}
\usepackage{myfloat}
\usepackage[biblatex]{mybib}
\usepackage{myps}
\makeatletter
\newcommand{\mylastname}{Simard}
\newcommand{\myadrienname}{Girard}
\newcommand{\mypierrename}{Albert}
\DeclareBibliographyCategory{myrefs}
\DeclareIndexNameFormat{myrefs}{% Test could be refined
	\ifboolexpr{%
		test {\ifdefstring{\mylastname}{#1}} or test {\ifdefstring{\myadrienname}{#1}} or test {\ifdefstring{\mypierrename}{#1}}
	}{%
		\addtocategory{myrefs}{\thefield{entrykey}}%
	}{}%
}
\AtDataInput{%
	\indexnames[myrefs]{author}%
}
\newcommand*{\mknamesignature}[5]{\def#1{#2|#3|#4|#5}} 
\mknamesignature{\highlightname}{Simard}{Jean}{}{} 
\DeclareNameFormat{sortname}{% 
	\begingroup%
	\mknamesignature{\currentsignature}{#1}{#3}{#5}{#7}%
	\ifdefequal{\highlightname}{\currentsignature}%
	{\let\mkbibnamefirst=\textbf%
		\let\mkbibnamelast=\textbf%
		\let\mkbibnameprefix=\textbf%
		\let\mkbibnameaffix=\textbf%
	}{}%
	\ifnum\value{listcount}=1\relax%
	\usebibmacro{name:last-first}{#1}{#3}{#5}{#7}%
	\ifblank{#3#5}%
	{}%
	{\usebibmacro{name:revsdelim}}%
	\else%
	\usebibmacro{name:first-last}{#1}{#3}{#5}{#7}%
	\fi%
	\endgroup%
	\usebibmacro{name:andothers}%
}
\newcommand{\myACER}{\textsc{acer}\xspace}
\newcommand{\myAMBER}{\textsc{amber}\xspace}
\newcommand{\myAMMPVis}{\textsc{ammp}-Vis\xspace}
\newcommand{\myAMMPEXTN}{\textsc{ammp-extn}\xspace}
\newcommand{\myanalysis}[1]{\input{files/#1}\%}
\newcommand{\myangstrom}{\AA ngström\xspace}
\newcommand{\myanova}[1]{\input{files/#1}}
\newcommand{\myatom}[2][]{%
	{%
		\ifstrempty{#1}%
			{\makefirstuc{\textsf{#2}}}%
			{\textcolor{#1}{\makefirstuc{\textsf{#2}}}}%
		\xspace%
	}%
}
\newcommand{\myAudacity}{Audacity\myregistered}% No '\xspace' because of already one in '\myregistered'
\newcommand{\myAutoDock}{AutoDock\xspace}
\newcommand{\myawareness}{\myenglish{awareness}\xspace}
\newcommand{\mybrainstorming}{\myenglish{brainstorming}\xspace}
\newcommand{\myBoxSearch}{BoxSearch\xspace}
\newcommand{\myCAO}{\textsc{cao}\xspace}
\newcommand{\mycarbon}{\myatom[mycarboncolor]{C}}
\newcommand{\myCasioXJ}{Casio \textsc{xj}\xspace}
\newcommand{\myCAVE}{\textsc{cave}\xspace}
\newcommand{\myCHARMM}{\textsc{charmm}\xspace}
\newcommand{\myChimera}{Chimera\xspace}
\newcommand{\myClayWorks}{Clayworks\xspace}
\newcommand{\myCNRSLIMSI}{\textsc{cnrs-limsi}\xspace}
\newcommand{\mycomputer}[1]{$\mathcal{#1}$}
\newcommand{\mycondition}[1]{$\mathcal{C}_{#1}$\xspace}
\newcommand{\myconditions}[2]{Conditions \mycondition{#1}--\mycondition{#2}}%
\newcommand{\myCPK}{\textsc{cpk}\xspace}
\newcommand{\myCUDA}{\textsc{cuda}\xspace}
\newcommand{\myDesktop}{\myPHANToM Desktop\myregistered}% No '\xspace' because of already one in '\myregistered'
\newcommand{\myDIVERSE}{\textsc{diverse}\xspace}
\newcommand{\myDOCK}{\textsc{dock}\xspace}
\newcommand{\myEDIPS}{\textsc{edips}\xspace}
\newcommand{\myeMinerals}{\textit{e}Minerals\xspace}
\def\myexptwolabel(#1,#2)[#3]#4#5{\rput(#1,#2){\rnode{#3}{\textcolor{#4}{\sffamily #5}}}}
\newcommand{\myEyeChem}{EyeChem\xspace}
\newcommand{\myfeedthrough}{\myenglish{feedthrough}\xspace}
\newcommand{\myFeedthrough}{\myenglish{Feedthrough}\xspace}
\newcommand{\myfeuillet}{feuillet-$\beta$\xspace}
\WithSuffix\newcommand\myfeuillet*{feuillets-$\beta$\xspace}
\newcommand{\myFireWire}{FireWire\xspace}
\newcommand{\myFlexX}{FlexX\xspace}
\newcommand{\myform}[1]{\textbf{\sffamily\MakeUppercase{#1}}}
\newcommand{\myGlide}{Glide\xspace}
\newcommand{\myGOLD}{\textsc{gold}\xspace}
\newcommand{\myGPU}{\textsc{gpu}\xspace}
\newcommand{\myGromacs}{Gromacs\xspace}
\newcommand{\myGROPE}{\textsc{grope}\xspace}
\newcommand{\myGROPEHaptic}{\textsc{grope}Haptic\xspace}
\newcommand{\mygrounding}{\myenglish{grounding}\xspace}
\newcommand{\myGrounding}{\myenglish{Grounding}\xspace}
\newcommand{\mygroup}[1]{$\mathcal{G}_{#1}$\xspace}
\newcommand{\myHawthorne}{\myenglish{Hawthorne Works}\xspace}
\newcommand{\myHBonds}{\textit{HBonds}\xspace}
\newcommand{\myhelice}{hélice-$\alpha$\xspace}
\WithSuffix\newcommand\myhelice*{hélices-$\alpha$\xspace}
\newcommand{\myHeuDiaSyC}{Heudiasyc\xspace}
\newcommand{\myHIMM}{\textsc{himm}\xspace}
\newcommand{\myhypothesis}[1]{$\mathcal{H}_{#1}$\xspace}
\newcommand{\myICMDocking}{\textsc{icm}Docking\xspace}
\newcommand{\myIFSTTAR}{\textsc{ifsttar}\xspace}
\newcommand{\myiMovie}{iMovie\xspace}
\newcommand{\myIntelCore}{Intel\myregistered Core\mytrademark~2 \textsc{q9450} (\mynum[GHz]{2.66})\xspace}
\newcommand{\myInternet}{internet\xspace}
\newcommand{\myJmol}{\textsc{Jmol}\xspace}
\newcommand{\myLBT}{\textsc{lbt}\xspace}
\newcommand{\myLCD}{\textsc{lcd}\xspace}
\newcommand{\myLicorice}{\textit{Licorice}\xspace}
\newcommand{\myLinux}{Linux\xspace}
\newcommand{\myLISA}{\textsc{lisa}\xspace}
\newcommand{\myMacOS}{Mac~\textsc{os}\xspace}
\newcommand{\myMARE}{\textsc{mare}\xspace}
\newcommand{\myMCRLab}{\textsc{mcr}Lab\xspace}
\newcommand{\myMDDriver}{\textsc{md}Driver\xspace}
\newcommand{\myMICE}{\textsc{mice}\xspace}
\newcommand{\myNewRibbon}{\textit{NewRibbon}\xspace}
\def\mynode{%
	\@ifnextchar[{\mynode@i}{\mynode@i[style=nodestyle]}%
}
\def\mynode@i[#1](#2,#3)[#4]#5{%
	\rput(#2,#3){\Rnode{#4}{\psframebox[style=nodestyle,#1]{\vphantom{pÉ}#5}}}%
}
\newcommand{\mynytrogen}{\myatom[mynytrogencolor]{A}}
\newcommand{\myNusE}{\textsc{NusE}\xspace}
\newcommand{\myNusENusG}{\textsc{NusE:NusG}\xspace}
\newcommand{\myNusG}{\textsc{NusG}\xspace}
\newcommand{\myOmni}{\myPHANToM Omni\myregistered}% No '\xspace' because of already one in '\myregistered'
\newcommand{\myoxygen}{\myatom[myoxygencolor]{O}}
\newcommand{\myPaulingWorld}{PaulingWorld\xspace}
\newcommand{\myPCRI}{\textsc{pcri}\xspace}
\newcommand{\myPDB}{\textsc{pdb}\xspace}
\newcommand{\myPDBbase}{\textit{Protein~DataBase}\xspace}
\newcommand{\myPDBlink}[2]{\href{#1}{\textsc{\MakeLowercase{#2}}}}
\newcommand{\myPHANToM}{\textsc{phant}o\textsc{m}\xspace}
\newcommand{\myPrion}{Prion\xspace}
\newcommand{\myPSF}{\textsc{psf}\xspace}
\def\mypsvibration(#1,#2){%
	\psarc[linewidth=0.1pt](#1,#2){1}{21}{32}
	\psarc[linewidth=0.1pt](#1,#2){1.05}{24}{29}
	\uput{10pt}[30](#1,#2){\psline[linewidth=0.1pt](-0.8,-0.6)(0.6,1.1)}
	\uput{10pt}[30](#1,#2){\psline[linewidth=0.1pt](-0.6,-0.1)(0.4,1.1)}
	\uput{10pt}[30](#1,#2){\psline[linewidth=0.1pt](-0.6,-1)(1.6,0.6)}
	\uput{10pt}[30](#1,#2){\psline[linewidth=0.1pt](-0.4,-1.1)(1.4,0.2)}
}
\newcommand{\myPtwoP}{\textsc{p2p}\xspace}
\newcommand{\mypvalue}{$p$-value\xspace}
\newcommand{\myPyMOL}{\textsc{p}y\textsc{mol}\xspace}
\newcommand{\myRAM}[2][Go]{\mynum[#1]{#2} de \textsc{ram}}
\newcommand{\myRasmol}{RasMol\xspace}
\newcommand{\myratio}[1]{\input{files/#1}}
\newcommand{\myresidue}[1]{$\mathcal{R}_{#1}$\xspace}
\newcommand{\myscenario}[1]{\textsc{#1}}
\newcommand{\mySensAble}{SensAble\xspace}
\newcommand{\myShaddock}{Shaddock\xspace}
\newcommand{\mySony}{\textsc{sony}\myregistered}% No '\xspace' because of already one in '\myregistered'
\newcommand{\mySpaceNavigator}{SpaceNavigator\myregistered}% No '\xspace' because of already one in '\myregistered'
\newcommand{\mySTALK}{\textsc{stalk}\xspace}
\newcommand{\mysubject}[1]{$\mathcal{S}_#1$}
\newcommand{\mysulfur}{\myatom[mysulfurcolor]{S}}
\newcommand{\mysummary}[1]{\input{files/#1}}
\newcommand{\myTCPIP}{\textsc{tcp/ip}\xspace}
\newcommand{\myThreeD}{\textsc{3d}\xspace}
\newcommand{\mytool}[1]{\myenglish{#1}}
\newcommand{\myTRPCAGE}{\textsc{trp-cage}\xspace}
\newcommand{\myTRPZIPPER}{\textsc{trp-zipper}\xspace}
\newcommand{\myTwoD}{\textsc{2d}\xspace}
\newcommand{\myUbiquitin}{Ubiquitin\xspace}
\newcommand{\myUbuntu}{Ubuntu~v$10.04$\xspace}
\newcommand{\myUSB}{\textsc{usb}\xspace}
\newcommand{\myuser}[1]{$\mathcal{#1}$}
\newcommand{\myvar}[2]{$\mathcal{V}_{\mathrm{#1}#2}$\xspace}
\newcommand{\myvard}[1]{\myvar{d}{#1}}
\newcommand{\myvari}[1]{\myvar{i}{#1}}
\newcommand{\myVGA}{\textsc{vga}\xspace}
\newcommand{\myVIEW}{\textsc{view}\xspace}
\newcommand{\myVirtuose}{Virtuose\mytrademark~\textsc{6d}\mynum{35}--\mynum{45}\xspace}
\newcommand{\myWindows}{Windows\xspace}
\newcommand{\myWorkbench}{\myenglish{Workbench}\xspace}

% Needed lengths
\newlength{\mywidth}
\newlength{\myheight}

% PSTricks style
\newpsstyle{nodestyle}{framearc=0.25,shadow=true,shadowcolor=myblue,blur=true,linewidth=1pt,linecolor=black,linestyle=solid,fillstyle=solid,fillcolor=white}
\makeatother

\title{Collaboration haptique étroitement couplée pour la déformation moléculaire interactive}
\author[J. \myname{Simard}]{Jean \myname{Simard}}
\institute{\mytextsc{cnrs-limsi}}
\school{Université de \myname{Paris}-Sud}
\logo{%
	\myimage[height=1cm]{logo-ups}%
	\hspace{1em}%
	\myimage[height=1cm]{logo-limsi}%
}
\date{\mydate[datestyle=long]{01/02/2012}}
\begin{document}
	\transfade[duration=0.1]
	\begin{frame}
		\titlepage
	\end{frame}
	\logo{}
	\begin{frame}{Sommaire}
		\tableofcontents[hideallsubsections]
	\end{frame}
	\section{Introduction}
	\subsection{\protect\textit{Docking} moléculaire}
	\begin{frame}{Définition}
		\begin{block}{\textit{Docking} moléculaire}
			ou \myemph{amarrage moléculaire}, consiste à trouver l'orientation et la conformation optimale permettant d'assembler \mynum{2}~molécules.
		\end{block}
		\begin{myfigure}
			\psset{unit=1cm}
			\begin{myps}(-5,-2)(6,2)
				\rput(-3,1){\myimage[width=0.120518359\paperwidth]{sota-docking-molecule-A}}
				\mycircleletter(-4.5,1){A}
				\rput(-3,-1){\myimage[width=0.1\paperwidth,angle=90]{sota-docking-molecule-B}}
				\mycircleletter(-4.5,-1){B}
				\rput(-3,0){\Huge\bfseries +}
				\rput(3,0){\myimage[width=0.186177106\paperwidth]{sota-docking-complex-AB}}
				\psline[linewidth=10pt,linecolor=myblue!70]{->}(-1,0)(1,0)
				\mycircleletter(5,0.75){A}
				\rput(5,0){\Large\bfseries +}
				\mycircleletter(5,-0.75){B}
			\end{myps}
			\mycaption[fig-IllustrationDeLAmarrageMoleculaire]{Illustration de l'amarrage moléculaire}
		\end{myfigure}
	\end{frame}
	\subsection{Distribution des charges de travail}
	\subsection{Collaboration en environnement virtuel}
	\subsection{Synthèse}
	\begin{frame}{Déroulement}
		\begin{myfigure}
			\psset{xunit=1cm,yunit=1.1cm}
			\begin{myps}(-2.5,-0.5)(2.5,5)
				\only<1>{
					\mynode[fillstyle=solid,fillcolor=myred!25](0,4)[Search]{Recherche}
					\mycirclenumber(0,4){1}
				}
				\only<2>{
					\mynode[fillstyle=solid,fillcolor=myred!25](0,3)[Selection]{Sélection}
					\mycirclenumber(0,3){2}
				}
				\only<3>{
					\mynode[fillstyle=solid,fillcolor=myred!25](0,2)[Manipulation]{Déformation}
					\mycirclenumber(0,2){3}
				}
				\only<4>{
					\mynode[fillstyle=solid,fillcolor=myred!25](0,1)[Evaluation]{Évaluation}
					\mycirclenumber(0,1){4}
				}
				\only<5>{
					\mycirclenumber[180](0,2.5){5}
				}
				\only<5->{
					\mynode(0,1)[Evaluation]{Évaluation}
				}
				\only<4->{
					\mynode(0,2)[Manipulation]{Déformation}
					\ncline{->}{Manipulation}{Evaluation}
				}
				\only<3->{
					\mynode(0,3)[Selection]{Sélection}
					\ncline{->}{Selection}{Manipulation}
				}
				\only<2->{
					\mynode(0,4)[Search]{Recherche}
					\ncline{->}{Search}{Selection}
				}
				\only<5->{
					\ncloop[loopsize=4em,angleA=-90,angleB=90,linearc=0.05,armA=0.2]{->}{Evaluation}{Search}
				}
				\only<6>{
					\mynode[fillstyle=solid,fillcolor=myred!25](0,0)[Objective]{Objectif atteint}
					\mycirclenumber(0,0){6}
					\ncline{->}{Evaluation}{Objective}
				}
			\end{myps}
			\mycaption[fig-ProcessusDeDeformationMoleculaire]{Processus de déformation moléculaire}
		\end{myfigure}
	\end{frame}
	\section{\myShaddock}
	\section{Étude du travail collaboratif}
	\subsection{Étude~\mynum{1} -- Recherche collaborative}
	\subsubsection{Objectifs}
	\begin{frame}{Objectifs}
		\begin{block}{Objectif principal}
			Observer les contraintes liées au travail collaboratif et souligner les avantages
		\end{block}
		\begin{block}{Hypothèses}
			\begin{enumerate}
				\item Amélioration des performances en binôme
					\begin{itemize}
						\item Comparer les performances en collaboration et seul
						\item Valider le contexte de travail (tâche complexe)
					\end{itemize}
				\item Stratégies de travail dépendantes de la personnalité
					\begin{itemize}
						\item Identifier et caractériser les stratégies de travail
						\item Identifier les conflits de coordination et de communication
					\end{itemize}
				\item Bonne utilisabilité de la plate-forme
					\begin{itemize}
						\item Évaluer les outils proposés
						\item Identifier les faiblesses
					\end{itemize}
			\end{enumerate}
		\end{block}
	\end{frame}
	\subsubsection{Synthèse}
	\subsection{Étude~\mynum{2} -- Déformation collaborative}
	\subsubsection{Objectifs}
	\begin{frame}{Objectifs}
		\begin{block}{Objectif principal}
			Proposer une tâche suffisamment complexe pour quantifier et qualifier les conflits de coordination
		\end{block}
		\begin{block}{Hypothèses}
			\begin{enumerate}
				\item Amélioration des performances en binôme pour la déformation
					\begin{itemize}
						\item Coordination étroitement couplée
					\end{itemize}
				\item Binômes plus performants sur les tâches complexes
					\begin{itemize}
						\item Tâches de difficulté variable
						\item Identifier les tâches nécessitant une collaboration
					\end{itemize}
				\item Évaluation du travail collaboratif par les sujets
					\begin{itemize}
						\item Questionnaire pour valider les améliorations de la plate-forme
						\item Évaluation de la configuration de travail collaboratif
					\end{itemize}
			\end{enumerate}
		\end{block}
	\end{frame}
	\subsubsection{Synthèse}
	\subsection{Étude~\mynum{3} -- Dynamique de groupe}
	\subsubsection{Objectifs}
	\begin{frame}{Objectifs}
		\begin{block}{Objectif principal}
			Observer la dynamique de groupe lors d'une coordination étroitement couplée
		\end{block}
		\begin{block}{Hypothèses}
			\begin{enumerate}
				\item Amélioration des performances en quadrinôme
					\begin{itemize}
						\item Variation de la taille d'un groupe
						\item Quantification des conflits dans des groupes
					\end{itemize}
				\item Émergence d'un meneur
					\begin{itemize}
						\item Observer la dynamique des groupes
						\item Caractériser les différents rôles
					\end{itemize}
				\item Le \textit{brainstorming} améliore les performances
					\begin{itemize}
						\item Période pour organiser le travail
						\item Limiter les conflits \textit{a priori}
					\end{itemize}
			\end{enumerate}
		\end{block}
	\end{frame}
	\subsubsection{Synthèse}
	\section{Aide au travail collaboratif}
	\subsection{Étude~\mynum{4} -- Assistance haptique et stratégie de travail}
	\subsubsection{Synthèse des études effectuées}
	\begin{frame}{Synthèse des études effectuées et solutions}
		\begin{myfigure}
			\psset{xunit=0.95cm,yunit=0.8cm}
			\begin{myps}(-6,-3)(6,4)
				\uput[-90](-2.5,4){\textcolor{black!50}{\Large Problématiques}}
				\uput[-90](2.5,4){\textcolor{black!50}{\Large Solutions}}
				\psset{arrowlength=1}
				\onslide<1->{%
					\myunode[0][shadowcolor=myred](-5,2.5)[complexe-problem]{\onslide<2->{\onslide<4-9>{\color{black!25}}Complexité de la tâche}}[4cm]
					\myunode[0][shadowcolor=myred](-5,1.5)[strategy-problem]{\onslide<3->{\onslide<4-9>{\color{black!25}}Stratégie de travail}}[4cm]
					\psbrace*[linecolor=myred,ref=lC,nodesepA=-1pt](-5,3)(-5,1){\rotateright{\textcolor{myred}{\tiny Étude~1}}}%
				}
				\onslide<4->{%
					\myunode[0][shadowcolor=mygreen](-5,0.5)[distribution-problem]{\onslide<5->{\onslide<7-9>{\color{black!25}}Charge de travail}}[4cm]
					\myunode[0][shadowcolor=mygreen](-5,-0.5)[conflict-problem]{\onslide<6->{\onslide<7-9>{\color{black!25}}Conflits de coordination}}[4cm]
					\psbrace*[linecolor=mygreen,ref=lC,nodesepA=-1pt](-5,1)(-5,-1){\rotateright{\textcolor{mygreen}{\tiny Étude~2}}}%
				}
				\onslide<7->{%
					\myunode[0][shadowcolor=myblue](-5,-1.5)[identification-problem]{\onslide<8->{Paresse sociale}}[4cm]
					\myunode[0][shadowcolor=myblue](-5,-2.5)[brainstorming-problem]{\onslide<9->{\textit{Brainstorming}}}[4cm]
					\psbrace*[linecolor=myblue,ref=lC,nodesepA=-1pt](-5,-1)(-5,-3){\rotateright{\textcolor{myblue}{\tiny Étude~3}}}%
				}
				\onslide<10->{%
					\psbrace*[linecolor=black!50,ref=lC,nodesepA=1pt](5,-3)(5,3){\rotateright{\textcolor{black!50}{\tiny Étude~4}}}
				}
				\onslide<11->{%
					\myunode[180][shadowcolor=black!50](5,2.5)[complexe-solution]{Tâche de \textit{docking}}[4cm]%
					\ncline[linewidth=4pt,linecolor=myred!25]{->}{complexe-problem}{complexe-solution}
				}
				\onslide<12->{%
					\myunode[180][shadowcolor=black!50](5,1.5)[strategy-solution]{Manipulation de résidu}[4cm]%
					\ncline[linewidth=4pt,linecolor=myred!25]{->}{strategy-problem}{strategy-solution}
				}
				\onslide<13->{%
					\myunode[180][shadowcolor=black!50](5,0.5)[distribution-solution]{Distribuer les rôles}[4cm]%
					\ncline[linewidth=4pt,linecolor=mygreen!25]{->}{distribution-problem}{distribution-solution}
				}
				\onslide<14->{%
					\myunode[180][shadowcolor=black!50](5,-0.5)[conflict-solution]{Solutions haptiques}[4cm]%
					\ncline[linewidth=4pt,linecolor=mygreen!25]{->}{conflict-problem}{conflict-solution}
				}
				\onslide<15->{%
					\myunode[180][shadowcolor=black!50](5,-1.5)[identification-solution]{Identifier les rôles}[4cm]%
					\ncline[linewidth=4pt,linecolor=myblue!25]{->}{identification-problem}{identification-solution}
				}
				\onslide<16->{%
					\myunode[180][shadowcolor=black!50](5,-2.5)[brainstorming-solution]{Phase exploratoire}[4cm]%
					\ncline[linewidth=4pt,linecolor=myblue!25]{->}{brainstorming-problem}{brainstorming-solution}
				}
			\end{myps}
			\mycaption[fig-SyntheseDesProblematiques]{Synthèse des problématiques}
		\end{myfigure}
	\end{frame}
	\subsubsection{Objectifs}
	\begin{frame}{Objectifs}
		\begin{block}{Objectif principal}
			Proposer et évaluer des outils haptiques pour assister la coordination
		\end{block}
		\begin{block}{Hypothèses}
			\begin{enumerate}
				\item Performances améliorées par l'assistance haptique
					\begin{itemize}
						\item Rapidité d'exécution
						\item Qualité de la solution atteinte
					\end{itemize}
				\item L'assistance haptique améliore la communication
					\begin{itemize}
						\item Temps de réaction réduits
						\item Meilleure compréhension des intentions de chacun
					\end{itemize}
				\item Les experts sont satisfaits des outils proposés
					\begin{itemize}
						\item Évaluer les outils proposés
						\item Identifier les faiblesses
					\end{itemize}
			\end{enumerate}
		\end{block}
	\end{frame}
	\subsubsection{Synthèse}
	\begin{frame}{Conclusion}
	\end{frame}
	\section{Conclusion}
	\subsection{Synthèse}
	\subsection{Perspectives}
	\begin{frame}{Questions}
		Merci pour votre attention
	\end{frame}
\end{document}
