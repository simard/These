\documentclass[english,french,dvips,10pt,notes]{mybeamer}
\usepackage{my}
\usepackage{mydate}
\usepackage{mycolor}
\usepackage{myfloat}
\usepackage{myps}
\usepackage{myuml}
\usepackage{pgfpages}
\usepackage{csquotes}
\usepackage[%
	sorting=ydnt,%
	hyperref=true,%
	backend=biber,%
	style=authoryear,%
	citestyle=mycitestyle,%
	bibstyle=mybibliographystyle,%
	dashed=false,
	abbreviate=false,%
	date=short,%
	urldate=short,%
	block=space,%
	maxcitenames=1,%
	maxbibnames=99]{biblatex}

	\captionsetup{figurename={},font={bf,color=myblue}}

\makeatletter
% Modify the bibliography style
\newcounter{mymaxcitenames}
\AtBeginDocument{%
	\setcounter{mymaxcitenames}{\value{maxnames}}%
}
\renewbibmacro{begentry}{%
	\printtext[brackets]{%
		\defcounter{maxnames}{\value{mymaxcitenames}}%
		\printnames{labelname}~\usebibmacro{cite:labelyear+extrayear}%
	}%
	\newline%
}
% AAA
\newcommand{\myACER}{\textsc{acer}\xspace}
\newcommand{\myAlanine}{Alanine\xspace}
\newcommand{\myanalysis}[1]{\input{files/#1}\%}
\newcommand{\myangstrom}{\AA ngström\xspace}
\newcommand{\myanova}[1]{\input{files/#1}}
\newcommand{\myatom}[2][]{%
	{%
		\ifstrempty{#1}%
			{\makefirstuc{\textsf{#2}}}%
			{\textcolor{#1}{\makefirstuc{\textsf{#2}}}}%
		\xspace%
	}%
}
\newcommand{\myAudacity}{\textsc{audacity}\myregistered}% No '\xspace' because of already one in '\myregistered'
% CCC
\newcommand{\mycarbon}{\myatom[mycarboncolor]{C}}
\newcommand{\myCasioXJ}{\textsc{Casio xj}\xspace}
\newcommand{\myCHARMM}{\textsc{charmm}\xspace}
\newcommand{\myChimera}{\textsc{chimera}\xspace}
\newcommand{\myClayWorks}{\textsc{Clayworks}\xspace}
\newcommand{\mycondition}[1]{$\left(\mathcal{C}_{#1}\right)$\xspace}
\newcommand{\myCPK}{\textsc{cpk}\xspace}
% DDD
\newcommand{\myDesktop}{\myPHANToM Desktop\myregistered}% No '\xspace' because of already one in '\myregistered'
% FFF
\newcommand{\myfeuillet}{feuillet-$\beta$\xspace}
\WithSuffix\newcommand\myfeuillet*{feuillets-$\beta$\xspace}
\newcommand{\myform}[1]{\textbf{\sffamily\MakeUppercase{#1}}}
% GGG
\newcommand{\myGhost}{\textsc{Ghost}\xspace}
\newcommand{\myGromacs}{\textsc{Gromacs}\xspace}
\newcommand{\mygroup}[1]{$\left(\mathcal{G}_{#1}\right)$\xspace}
% HHH
\newcommand{\myHaption}{\textsc{Haption}\xspace}
\newcommand{\myHawthorne}{\myemph{Hawthorne Works}\xspace}
\newcommand{\myHBonds}{\textit{HBonds}\xspace}
\newcommand{\myhelice}{hélice-$\alpha$\xspace}
\WithSuffix\newcommand\myhelice*{hélices-$\alpha$\xspace}
\newcommand{\myhypothesis}[1]{$\left(\mathcal{H}_{#1}\right)$\xspace}
% III
\newcommand{\myIntelCore}{Intel\myregistered Core\mytrademark~2 \textsc{q9450} (\mynum[GHz]{2.66})\xspace}
% JJJ
\newcommand{\myJmol}{\textsc{Jmol}\xspace}
% LLL
\newcommand{\myLCD}{\textsc{lcd}\xspace}
\newcommand{\myLicorice}{\textit{Licorice}\xspace}
\newcommand{\myLinux}{\textsc{Linux}\xspace}
% MMM
\newcommand{\myMacOS}{Mac~\textsc{OS}\xspace}
\newcommand{\myMDDriver}{\textsc{MDDriver}\xspace}
% NNN
\newcommand{\myNewRibbon}{\textit{NewRibbon}\xspace}
\def\mynode{%
	\@ifnextchar[{\mynode@i}{\mynode@i[style=nodestyle]}%
}
\def\mynode@i[#1](#2,#3)[#4]#5{%
	\rput(#2,#3){\Rnode{#4}{\psframebox[style=nodestyle,#1]{\vphantom{pÉ}#5}}}%
}
\newcommand{\mynytrogen}{\myatom[mynytrogencolor]{A}}
\newcommand{\myNusE}{\textsc{NusE}\xspace}
\newcommand{\myNusENusG}{\textsc{NusE:NusG}\xspace}
\newcommand{\myNusG}{\textsc{NusG}\xspace}
% OOO
\newcommand{\myOmni}{\myPHANToM Omni\myregistered}% No '\xspace' because of already one in '\myregistered'
\newcommand{\myOpenHaptics}{\textsc{OpenHaptics}\mytrademark}% No '\xspace' because of already one in '\mytrademark'
\newcommand{\myoxygen}{\myatom[myoxygencolor]{O}}
% PPP
\newcommand{\myPC}{\textsc{pc}\xspace}
\newcommand{\myPDB}{\textsc{pdb}\xspace}
\newcommand{\myPDBbase}{\emph{Protein~DataBase}\xspace}
\newcommand{\myPDBlink}[2]{\href{#1}{\textsc{\MakeLowercase{#2}}}}
\newcommand{\myPHANToM}{\textsc{phant}o\textsc{m}\xspace}
\newcommand{\myPremium}{\myPHANToM Premium\myregistered}% No '\xspace' because of already one in '\myregistered'
\newcommand{\myPrion}{Prion\xspace}
\newcommand{\myPSF}{\textsc{psf}\xspace}
\newcommand{\mypvalue}{$p$-value\xspace}
\newcommand{\myPyMOL}{\textsc{p}y\textsc{mol}\xspace}
% RRR
\newcommand{\myRAM}[2][Go]{\mynum[#1]{#2} de \textsc{ram}}
\newcommand{\myRasmol}{\textsc{RasMol}\xspace}
\newcommand{\myresidue}[1]{$\left(\mathcal{R}_{#1}\right)$\xspace}
% SSS
\newcommand{\myscenario}[1]{\textsc{#1}}
\newcommand{\mySensAble}{\textsc{SensAble}\xspace}
\newcommand{\myShaddock}{\textsc{Shaddock}\xspace}
\newcommand{\mySony}{\textsc{sony}\myregistered}% No '\xspace' because of already one in '\myregistered'
\newcommand{\mySpaceNavigator}{SpaceNavigator\myregistered}% No '\xspace' because of already one in '\myregistered'
\newcommand{\mysubject}[1]{$\mathcal{S}_#1$}
\newcommand{\mysulfur}{\myatom[mysulfurcolor]{S}}
\newcommand{\mysummary}[1]{\input{files/#1}}
% TTT
\newcommand{\myTCPIP}{\textsc{tcp/ip}\xspace}
\newcommand{\myThreeD}{\textsc{3d}\xspace}
\newcommand{\mytool}[1]{\myemph{#1}}
\newcommand{\myTRPCAGE}{\textsc{trp-cage}\xspace}
\newcommand{\myTRPZIPPER}{\textsc{trp-zipper}\xspace}
% UUU
\newcommand{\myUbiquitin}{Ubiquitin\xspace}
\newcommand{\myUbuntu}{\textsc{Ubuntu}~v$10.04$\xspace}
\newcommand{\myUSB}{\textsc{usb}\xspace}
\newcommand{\myuser}[1]{$\mathcal{#1}$}
% VVV
\newcommand{\myvar}[2]{$\left(\mathcal{V}_{\mathrm{#1}#2}\right)$\xspace}
\newcommand{\myvard}[1]{\myvar{d}{#1}}
\newcommand{\myvari}[1]{\myvar{i}{#1}}
\newcommand{\myVGA}{\textsc{vga}\xspace}
\newcommand{\myVirtuose}{\textsc{Virtuose}\mytrademark~\textsc{6d}\mynum{35}--\mynum{45}\xspace}
% WWW
\newcommand{\myWindows}{\textsc{Windows}\xspace}

% Needed lengths
\newlength{\mywidth}
\newlength{\myheight}

% PSTricks style
\newpsstyle{nodestyle}{framearc=0.25,shadow=true,shadowcolor=myblue,blur=true}
\makeatother


%\setbeameroption{show notes on second screen}
\includeonlyframes{current}

\makeatletter
\makeatother

\title{Collaboration haptique étroitement couplée pour la manipulation moléculaire interactive}
\author[J. \myname{Simard}]{%
	{\normalsize Jean \myname{Simard}}\\[1ex]
	sous la direction de Philippe \myname{Tarroux}\\
	et l'encadrement scientifique de Mehdi \myname{Ammi}
}
\institute{\textsc{cnrs-limsi}}
\school{Université de \myname{Paris}-Sud}
\logo{%
	\myimage[height=1cm]{logo-ups}%
	\hspace{1em}%
	\myimage[height=1cm]{logo-limsi}%
}
\date{\mydate[datestyle=long]{12/03/2012}}

\bibliography{biblio.bib}

\begin{document}
	\transfade[duration=0.1]
	\begin{myframe}{}
		\titlepage
	\end{myframe}
	\logo{}
	\begin{myframe}{Sommaire}
		\tableofcontents[hideallsubsections]
	\end{myframe}
	\section{Introduction}
	\begin{myframe}{Sommaire}
		\tableofcontents[sectionstyle=show/shaded,subsectionstyle=show/show/hide,subsubsectionstyle=show/show/hide]
	\end{myframe}
	\subsection{\myDocking moléculaire}
	\begin{myframe}{\myDocking moléculaire}
		\begin{myplusblock}{Définition}%
			ou \myemph{amarrage moléculaire}, consiste à trouver l'orientation et la conformation optimale permettant d'assembler \mynum{2}~molécules.%
		\end{myplusblock}%

		\begin{columns}[T]%
			\begin{column}{0.6\textwidth}%
				\begin{myfigure}%
					\psset{xunit=0.045\textwidth,yunit=0.4cm}%
					\begin{myps}(-10,-5)(10,5)%
						\only<1-3>{%
							\rput(-5,3){\myimage[width=0.15\paperwidth]{sota-docking-deformed-A}}%
							\only<1-2>{%
								\rput(-5,-3){\myimage[width=0.15\paperwidth,angle=180]{sota-docking-deformed-B}}%
							}%
							\only<3>{%
								\rput(-5,-3){\myimage[width=0.15\paperwidth]{sota-docking-deformed-B}}%
							}%
						}%
						\only<2-3>{%
							\rput(5,2.5){\mymolAd}%
							\only<2>{%
								\rput{180}(5,-2.5){\mymolBd}%
								\psarcn[linewidth=2pt,linecolor=mygreen,arrowsize=2pt 4,arrowlength=0.75]{->}(5,-2.5){0.3}{90}{135}%
								\uput{2ex}[-90](5,-2.5){\textcolor{mygreen}{réorientation}}%
							}%
							\only<3>{%
								\rput(5,-2.5){\mymolBd}%
								\rput(5,0){\rnode{deformation}{\vphantom{pÉ}\textcolor{mygreen}{déformation}}}%
								\pnode(6,2.5){A}
								\pnode(4,-2.5){B}
								\pnode(7.5,-2){C}
								\ncline[linecolor=mygreen]{->}{deformation}{A}
								\ncline[linecolor=mygreen]{->}{deformation}{B}
								\ncline[linecolor=mygreen]{->}{deformation}{C}
							}%
						}%
						\only<4-5>{%
							\rput(-5,3){\myimage[width=0.15\paperwidth]{sota-docking-molecule-A}}%
							\rput(-5,-3){\myimage[width=0.15\paperwidth]{sota-docking-molecule-B}}%
							\rput(5,2.5){%
								\mymolA%
								\only<4>{%
									\rput(0,2.5){\textcolor{mygreen}{propriétés chimiques}}%
									\rput(-3,0.5){\textbf{\textcolor{white}{+}}}%
									\rput(-0.5,0.75){\textbf{\textcolor{white}{--}}}%
									\rput(2,1.5){\textbf{\textcolor{white}{+}}}%
								}
								\only<5>{%
									\psline[linewidth=4pt,linecolor=mygreen]{->}(0,-0.25)(0,-2)%
								}%
							}%
							\rput(5,-2.5){%
								\mymolB%
								\only<4>{%
									\rput(0,-2.5){\textcolor{mygreen}{propriétés chimiques}}%
								}
								\only<5>{%
									\rput(-3,-1){\textbf{\textcolor{white}{--}}}%
									\rput(-0.5,-0.5){\textbf{\textcolor{white}{+}}}%
									\rput(2,-0.25){\textbf{\textcolor{white}{--}}}%
									\psline[linewidth=4pt,linecolor=mygreen]{->}(0,0.25)(0,2)%
								}%
							}%
						}%
						\only<5>{%
							\rput(5,0){\vphantom{pÉ}\textcolor{mygreen}{amarrage}}%
						}%
						\only<6>{%
							\rput(-5,0){\myimage[width=0.175\paperwidth]{sota-docking-complex-AB}}%
							\rput(5,0){%
								\mymolA%
								\rput(-3,0.5){\textbf{\textcolor{white}{+}}}%
								\rput(-0.5,0.75){\textbf{\textcolor{white}{--}}}%
								\rput(2,1.5){\textbf{\textcolor{white}{+}}}%
							}%
							\rput(5,0){%
								\mymolB%
								\rput(-3,-1){\textbf{\textcolor{white}{--}}}%
								\rput(-0.5,-0.5){\textbf{\textcolor{white}{+}}}%
								\rput(2,-0.25){\textbf{\textcolor{white}{--}}}%
							}%
						}%
						\mycircleletter[fillcolor=myblue](-9.5,2.5){A}%
						\mycircleletter(-9.5,-2.5){B}%
						\only<6>{%
							\rput(-9.5,0){\Huge\bfseries +}%
						}
					\end{myps}%
					\mycaption[fig-DockingMoleculaire]{\myDocking moléculaire}%
				\end{myfigure}%
			\end{column}%
			\begin{column}{0.35\textwidth}%
				\begin{myblock}{Facteurs de complexité}%
					\begin{itemize}%
						\item<1-> Nombreux atomes%
						\item<2-> Orientation et déplacement%
						\item<3-> Flexibilité%
						\item<4-> Physico-chimie
						\item<5-> Complémentarité%
							\begin{itemize}%
								\item géométrique%
								\item physico-chimie
							\end{itemize}%
					\end{itemize}%
				\end{myblock}%
			\end{column}%
		\end{columns}%
	\end{myframe}
	\subsection{Approches centrées sur l'humain}
	\begin{myframe}{Approches centrées sur l'humain}
		\begin{columns}
			\begin{column}{0.475\textwidth}
				\begin{myfigure}
					\myimage[width=0.45\textwidth]{sota-Davies-2005}
					\mycaption[fig-sota-VisualisationMultimodaleDavies2005]{Visualisation multimodale \mycite{Davies-2005}}
				\end{myfigure}
			\end{column}
			\begin{column}{0.475\textwidth}
				\begin{myfigure}
					\myimage[width=0.45\textwidth]{sota-Lai-Yuen-2006}
					\mycaption[fig-sota-InterfaceHaptiqueA5DDLLeDockingLaiYuen2006]{Interface haptique à \mynum[\textsc{ddl}]{5} \mycite{Lai-Yuen-2006}}
				\end{myfigure}
			\end{column}
		\end{columns}
		\begin{columns}
			\begin{column}{0.475\textwidth}
				\begin{myfigure}
					\myimage[width=0.45\textwidth]{sota-Daunay-2009}
					\mycaption[fig-sota-DockingMoleculaireRigide]{\myDocking moléculaire rigide \mycite{Daunay-2009}}
				\end{myfigure}
			\end{column}
			\begin{column}{0.475\textwidth}
				\begin{myfigure}
					\myimage[width=0.45\textwidth]{sota-Hou-2010}
					\mycaption[fig-sota-ModeleHaptiqueMoleculaireHou2010]{Modèle haptique moléculaire \mycite{Hou-2010}}
				\end{myfigure}
			\end{column}
		\end{columns}
		\onslide<2>{%
			\hspace{\stretch{1}}L'outil \alert{haptique} s'impose comme outil de manipulation moléculaire\hspace{\stretch{1}}
		}
	\end{myframe}
	\subsection{Distribution des charges de travail}
	\begin{myframe}{Distribution des charges de travail}
		\begin{myplusblock}{Définition \mycite{Conein-2004}}
			Étendre la capacité cognitive d'analyse d'un individu pour inclure le matériel et l'environnement social comme composant d'un système cognitif plus étendu.
		\end{myplusblock}

		\begin{myfigure}%
			\psset{unit=0.06\textwidth}%
			\definecolor{mylightestredblue}{rgb}{0.875 0.705 0.79328125}
			\begin{myps}(-5,-2.5)(5,4)
				\footnotesize%
				\pspolygon*[linearc=0.5,linecolor=mylightestred](-0.5,4)(-4.5,4)(-4.5,0)(-0.5,-1.7)(1.15,-1.6)%
				\pspolygon*[linearc=0.5,linecolor=mylightestblue](0.5,4)(4.5,4)(4.5,0)(0.5,-1.7)(-1.15,-1.6)%
				\psclip{%
					\pscustom[linestyle=none]{%
						\pspolygon*[linearc=0.5](-0.5,4)(-4.5,4)(-4.5,0)(-0.5,-1.7)(1.15,-1.6)%
					}%
					\pscustom[linestyle=none]{%
						\pspolygon*[linearc=0.5](0.5,4)(4.5,4)(4.5,0)(0.5,-1.7)(-1.15,-1.6)%
					}%
				}%
				\psframe*[linecolor=mylightestredblue](-5,-2.5)(5,4)%
				\endpsclip%
				\psset{fillcolor=white}%
				\psframe[fillstyle=solid,linecolor=black!70,linestyle=dashed,framearc=0.25](-4.5,0)(-0.5,4)%
				\rput(-2.5,3.75){\textcolor{black!70}{Espace interne}}%
				\psframe[fillstyle=solid,linecolor=black!70,linestyle=dashed,framearc=0.25](4.5,0)(0.5,4)%
				\rput(2.5,3.75){\textcolor{black!70}{Espace externe}}%
				\psframe[linecolor=black!70,linestyle=dashed,framearc=0.25](-4.5,-2.5)(4.5,-0.5)%
				\mynode(-2,3.15)[individu1]{\textcolor{myblue}{Individu~\mynum{1}}}%
				\mynode(-3,2)[individu2]{\textcolor{myred}{Individu~\mynum{2}}}%
				\mynode(-2,0.85)[individu3]{\textcolor{mygreen}{Individu~\mynum{3}}}%
				\mynode(2.5,2.75)[artefact1]{Artefact~\mynum{1}}%
				\mynode(2.5,1.25)[artefact2]{Artefact~\mynum{2}}%
				\psellipse[fillstyle=solid,fillcolor=white](0,-1.25)(1,0.5)%
				\rput(0,-1.25){Tâche}%
				\rput(0,-2.25){\textcolor{black!70}{Abstraction de l'espace de travail}}%
				\psset{linewidth=0.5pt}
				\psset{arrows={<->}}%
				\ncline{individu1}{individu2}%
				\ncline{individu2}{individu3}%
				\ncline[offset=8pt]{individu1}{individu3}%
				\psset{arrows={-}}%
				\ncline{artefact1}{artefact2}%
				\psset{arrows={<->}}%
				\psset{linecolor=myblue}%
				\ncline{individu1}{artefact1}%
				\ncline{individu1}{artefact2}%
				\psset{linecolor=myred}%
				\ncline{individu2}{artefact1}%
				\ncline{individu2}{artefact2}%
				\psset{linecolor=mygreen}%
				\ncline{individu3}{artefact1}%
				\ncline{individu3}{artefact2}%
			\end{myps}
			\mycaption[fig-sota-SystemeCognitifDistribue]{Système cognitif distribué}
		\end{myfigure}
	\end{myframe}
	\subsection{Approches collaboratives}
	\begin{myframe}{Approches collaboratives}
		\begin{columns}
			\begin{column}{0.475\textwidth}
				\begin{myfigure}
					\myimage[width=0.6\textwidth]{sota-Kriz-2003}
					\mycaption[fig-sota-ManipulationSynchroneKriz2003]{Manipulation synchrone \mycite{Kriz-2003}}
				\end{myfigure}
			\end{column}
			\begin{column}{0.475\textwidth}
				\begin{myfigure}
					\myimage[width=0.6\textwidth]{sota-Park-2006}
					\mycaption[fig-sota-ManipulationGuideeParDesExpertsPark2006]{Manipulation guidée par des experts \mycite{Park-2006}}
				\end{myfigure}
			\end{column}
		\end{columns}
		\begin{columns}
			\begin{column}{0.475\textwidth}
				\begin{myfigure}
					\myimage[width=0.6\textwidth]{sota-Chastine-2007}
					\mycaption[fig-sota-InterReferencementChastine2007]{Inter-référencement \mycite{Chastine-2007}}
				\end{myfigure}
			\end{column}
			\begin{column}{0.475\textwidth}
				\begin{myfigure}
					\myimage[width=0.6\textwidth]{sota-Ma-2007}
					\mycaption[fig-sota-GestionDesDroitsDAcces]{Gestion des droits d'accès \mycite{Ma-2007}}
				\end{myfigure}
			\end{column}
		\end{columns}
	\end{myframe}
	\subsection{Objectifs de la thèse}
	\begin{myframe}{Objectifs de la thèse}
		\begin{myminusblock}{Problématiques}
			\begin{itemize}
				\item Quels sont les avantages du travail en collaboration étroitement couplée ?
					\note[item]{Quelles sont les influences d'une collaboration nécessitant une forte coordination entre les collaborateurs ?}
				\item Quelles problématiques la collaboration apporte-t-elle ?
				\item Comment améliorer la collaboration dans un environnement de travail complexe ?
			\end{itemize}
		\end{myminusblock}
		\begin{myplusblock}{Démarche}
			\begin{enumerate}
				\item Étudier et analyser le travail collaboratif étroitement couplé
				\item Identifier les limites et les contraintes
				\item Proposer des solutions haptiques pour améliorer cette configuration
				\item Évaluer ces solutions dans une tâche de \mydocking moléculaire
			\end{enumerate}
		\end{myplusblock}
	\end{myframe}
	\section{Plateforme de manipulation moléculaire \myShaddock}
	\begin{myframe}{Sommaire}
		\tableofcontents[sectionstyle=show/shaded,subsectionstyle=show/show/hide,subsubsectionstyle=show/show/hide]
	\end{myframe}
	\subsection{Cahier des charges}
	\begin{myframe}{Cahier des charges}
		\note{Il n'existe pas de plateforme appropriée qui répondent à nos besoins.}
		\note[item]{Pas de manipulation colocalisée}
		\note[item]{Manipulation de molécules simples}
		\begin{myblock}{Objectif}
			Élaborer une plateforme permettant la collaboration étroitement couplée pour la manipulation moléculaire
		\end{myblock}
		\begin{columns}[T]
			\begin{column}{0.475\textwidth}
				\begin{myminusblock}{Contraintes à respecter}
					\begin{itemize}
						\item Interaction synchrone sur des molécules
						\item Simulation de la dynamique moléculaire
						\item Manipulation à l'aide d'interface haptique
						\item Plusieurs outils différents
					\end{itemize}
				\end{myminusblock}
			\end{column}
			\begin{column}{0.475\textwidth}
				\begin{myplusblock}{Solutions proposées}
					\begin{itemize}
						\item Modularité logicielle
						\item Modularité matérielle
						\item Développement basé sur des logiciels de biologie existants
						\item Utilisation de modules dédiés à la réalité virtuelle
						\item Développement de nouveaux outils
					\end{itemize}
				\end{myplusblock}
			\end{column}
		\end{columns}
	\end{myframe}
	\subsection{Organisation logicielle}
	\begin{myframe}{Organisation logicielle}
		\begin{myfigure}
			\psset{xunit=0.0666667\textwidth,yunit=0.06\textheight}
			\psset{framearc=.1,shadow=true,blur=true}
			\begin{myps}(-7.5,-5)(7.5,6)
				\psframe*[linecolor=mygreen!5,shadow=false](-7.4,-5)(-2.6,6)
				\psframe*[linecolor=myblue!5,shadow=false](-2.4,-5)(2.4,6)
				\psframe*[linecolor=myred!5,shadow=false](2.6,-5)(7.4,6)
				\uput[-90](-5,6){\large\textcolor{mygreen!25}{Simulation}}
				\uput[-90](0,6){\large\textcolor{myblue!25}{Visualisation}}
				\uput[-90](5,6){\large\textcolor{myred!25}{Interaction}}
				\uput[-90](0,5){%
					\myumlnode*<PCUtilisateur>{\vphantom{pÉ}\footnotesize Nœud principal}{%
						\myumlcomponent<VMD>[\textcolor{myblue!50}{\scriptsize application}]{\footnotesize\myVMD}%
					}%
				}
				\uput[-90](-5,5){%
					\myumlnode*<ServeurNAMD>{\vphantom{pÉ}\footnotesize Nœud \myNAMD}{%
						\begin{psmatrix}[rowsep=1]%
							\myumlcomponent<NAMD>[\textcolor{myblue!50}{\scriptsize programme}]{{\footnotesize\texttt{namd2}}} \\%
							\myumlcomponent<FichierSimulation>[\textcolor{myblue!50}{\scriptsize fichier}]{%
								\\[-1ex]%
								\begin{psmatrix}[rowsep=0]%
									\footnotesize Données de\\\footnotesize simulation%
								\end{psmatrix}%
							}
						\end{psmatrix}%
					}%
				}
				\uput[-90](5,5){%
					\myumlnode*<ServeurVRPN1>{\vphantom{pÉ}\footnotesize Nœud \myVRPN}{%
						\myumlcomponent<VRPN1>[\textcolor{myblue!50}{\scriptsize programme}]{{\footnotesize\texttt{vrpn\_server}}}%
					}%
				}
				\uput[-90](5,0){%
					\myumlnode<PHANToM1>[\textcolor{myblue!50}{\scriptsize\myOmni}]{\vphantom{pÉ}\footnotesize Interface}%
				}
				\rput(5,-4){\Large$\vdots$}
				\uput[-90](0,0){%
					\myumlnode<VideoProjecteur>[\textcolor{myblue!50}{\scriptsize vue partagée}]{\footnotesize Dispositif d'affichage}
				}
				\psset{shadow=false}

				\myumlrealization[angleA=-90,angleB=90]{NAMD}{FichierSimulation}[nccurve]%
				\myumlrealization[angleA=-90,angleB=90]{VRPN1}{PHANToM1}[nccurve]%
				\myumlinterface[angleA=-90,angleB=90,ArrowInsidePos=0.5]{VMD}{VideoProjecteur}[nccurve]
				\myumlinterface[angleA=0,angleB=180,offsetB=-8pt,ArrowInsidePos=0.4]{NAMD}{VMD}[nccurve]
				\myumlinterface[angleA=180,angleB=0,ncurvA=1.5,offsetA=8pt,ArrowInsidePos=0.6]{VRPN1}{VMD}[nccurve]
			\end{myps}
			\mycaption[fig-Shaddock-DiagrammeDeDeploiementUMLDeLaPlateformeShaddock]{Diagramme de déploiement \myUML de la plateforme \myShaddock}
		\end{myfigure}
		\note{Cette plateforme a donné lieu à un travail d'intégration important.}
		\note[item]{Le nœud \myNAMD nécessite une configuration importante pour la simulation}
		\note[item]{Le nœud \myVRPN a nécessité une modification qui a été soumise et intégrée par les développeurs dans les nouvelles versions}
		\note[item]{C'est au niveau de \myVMD que le plus gros développement a été effectué avec l'ajout de nouveaux outils dédiés pour améliorer la manipulation interactive}
	\end{myframe}
	\subsection{Organisation matérielle}
	\begin{myframe}{Organisation matérielle}
		\only<1>{
			\begin{itemize}
				\item Configuration \alert{colocalisée} et \alert{synchrone}
			\end{itemize}
		}
		\only<2>{
			\begin{itemize}
				\item Communication \alert{orale} et \alert{gestuelle} autorisée
			\end{itemize}
		}
		\only<3>{
			\begin{itemize}
				\item Vue \alert{partagée}
			\end{itemize}
		}
		\only<4>{
			\begin{itemize}
				\item Outil pour la \alert{déformation} ou le \alert{déplacement} de molécule
			\end{itemize}
		}
		\only<5>{
			\begin{itemize}
				\item Outil pour l'\alert{orientation} de la molécule
			\end{itemize}
		}
		\only<6>{
			\begin{itemize}
				\item Nombre d'outils quasiment \alert{illimité}
			\end{itemize}
		}
		\psset{xunit=0.666666666667\paperwidth,yunit=0.4\paperwidth}
		\begin{myfigure}
			\begin{myps}(0,0)(1,1)%
				\only<1-6>{%
					\rput(0.5,0.5){\myimage[width=0.666666666667\paperwidth,height=0.4\paperwidth]{exp1-schema}}%
					\rput(0.2,0.8){\myimage[width=0.266666666667\paperwidth,height=0.16\paperwidth]{exp1-photo}}%
					\pnode(0.52,0.66){tug1}%
					\pnode(0.62,0.76){tug1-from}%
					\pnode(0.62,0.60){tug2}%
					\pnode(0.72,0.70){tug2-from}%
					\pnode(0.57,0.63){grab}%
					\pnode(0.67,0.73){grab-from}%
				}%
				\only<3>{%
					\pspolygon[linewidth=4pt,linecolor=myred](0.60,0.68)(0.61,0.93)(0.86,0.88)(0.83,0.57)%
				}%
				\only<4>{%
					\ncline[linewidth=2.5pt,linecolor=myred,nodesepB=2.5pt]{c->}{tug1-from}{tug1}%
					\ncline[linewidth=2.5pt,linecolor=mygreen,nodesepB=2.5pt]{c->}{tug2-from}{tug2}%
					\ncline[linewidth=2.5pt,linecolor=myblue,nodesepB=2.5pt]{c->}{grab-from}{grab}%
				}%
				\only<5>{%
					\rput(0.5,0.5){\myimage[width=0.666666666667\paperwidth,height=0.4\paperwidth]{exp2-schema}}%
					\rput(0.2,0.8){\myimage[width=0.266666666667\paperwidth,height=0.16\paperwidth]{exp2-photo}}%
					\pnode(0.59,0.6){grab}%
					\pnode(0.69,0.7){grab-from}%
					\ncline[linewidth=2.5pt,linecolor=blue,nodesepB=2.5pt]{c->}{grab-from}{grab}%
				}%
				\only<6>{%
					\rput(0.5,0.5){\myimage[width=0.666666666667\paperwidth,height=0.4\paperwidth]{exp3-schema}}%
					\rput(0.2,0.8){\myimage[width=0.266666666667\paperwidth,height=0.16\paperwidth]{exp3-photo}}%
					\pnode(0.52,0.675){tug1}%
					\pnode(0.62,0.775){tug1-from}%
					\pnode(0.55,0.65){tug2}%
					\pnode(0.65,0.75){tug2-from}%
					\pnode(0.58,0.63){tug3}%
					\pnode(0.68,0.73){tug3-from}%
					\pnode(0.62,0.61){tug4}%
					\pnode(0.72,0.71){tug4-from}%
					\ncline[linewidth=2.5pt,linecolor=myred,nodesepB=2.5pt]{c->}{tug1-from}{tug1}%
					\ncline[linewidth=2.5pt,linecolor=myblue,nodesepB=2.5pt]{c->}{tug2-from}{tug2}%
					\ncline[linewidth=2.5pt,linecolor=magenta,nodesepB=2.5pt]{c->}{tug3-from}{tug3}%
					\ncline[linewidth=2.5pt,linecolor=mygreen,nodesepB=2.5pt]{c->}{tug4-from}{tug4}%
				}%
			\end{myps}
			\mycaption[fig-PlateFormeExperimentale1]{Plate-forme expérimentale}
		\end{myfigure}
	\end{myframe}
	\subsection{Outils supplémentaires proposés}
	\begin{myframe}{Outils supplémentaires proposés}
		\begin{myblock}{Objectif}
			Faciliter le processus de sélection d'une structure moléculaire dans \myVMD
		\end{myblock}

		\only<1>{
			\begin{itemize}
				\item Sélection difficile (nombre d'atomes important, cibles en mouvement, \dots)
			\end{itemize}
		}
		\only<2-3>{
			\begin{itemize}
				\item Attraction haptique sur les structures (champ de potentiel \mycite{Simard-2009})
			\end{itemize}
		}
		\only<4>{
			\begin{itemize}
				\item Possibilité de pointer un atome\dots{}
			\end{itemize}
		}
		\only<5>{
			\begin{itemize}
				\item \dots{} ou un résidue (ou d'autres structures moléculaires)
			\end{itemize}
		}
		\only<6>{
			\begin{itemize}
				\item Pour enfin le sélectionner
			\end{itemize}
		}
		\begin{myfigure}
			\psset{xunit=0.6\textwidth,yunit=0.365838509\textwidth}
			\begin{myps}(0,0)(1,1)%
				\rput(0.5,0.5){%
					\only<1-3>{%
						\myimage[width=0.6\textwidth]{select-molecule}%
					}%
					\only<4>{%
						\myimage[width=0.6\textwidth]{select-atom-target}%
					}%
					\only<5>{%
						\myimage[width=0.6\textwidth]{select-residue-target}%
					}%
					\only<6>{%
						\myimage[width=0.6\textwidth]{select-residue-select}%
					}%
				}%
				\only<2-3>{%
					\psarc{c-c}(0.88,0.86){0.13}{-45}{45}%
					\psarc{c-c}(0.88,0.86){0.20}{-45}{45}%
					\psarc{c-c}(0.88,0.86){0.30}{-45}{45}%
					\psarc{c-c}(0.88,0.86){0.45}{-45}{45}%
					\uput{0pt}[15](1.1,0.92){%
						\myimage[width=40pt]{red-cursor}%
					}%
				}%
				\only<3>{%
					\psline[linewidth=2pt,linecolor=mygreen]{c->}(1.075,0.86)(0.95,0.86)
					\uput{0pt}[-45](0.8,0.75){%
						\psframebox*[fillcolor=black!5,framearc=0.25,boxsep=true,framesep=5pt]{%
							\begin{psmatrix}[colsep=0pt,rowsep=2ex]
								\textbf{Champ de force}\\%
								{\small$\displaystyle F(d)=\phi\frac{d}{\sigma}\exp\left[\frac{\sigma^2-d^2}{2\sigma^2}\right]$}\\%
								\begin{mygraph}[labels=none,ticks=none,xAxisLabel={distance},xAxisLabelPos={c,-1ex},llx={-1ex},yAxisLabel={force},yAxisLabelPos={-1ex,c},lly={-1ex}]{->}(4.5,2.5){3cm}{1cm}%
									\newcommand{\E}{2.828182845904523536028747135266249 }%
									\psplot[linecolor=red,linewidth=0.5pt,plotpoints=500,arrows=c-c]{0}{4}{2 \E 1 2 div exp x \E 0 x 2 exp 2 div sub exp mul mul mul}%
								\end{mygraph}%
							\end{psmatrix}
						}%
					}%
				}%
				\only<4->{%
					\uput{0pt}[15](0.91,0.92){%
						\myimage[width=40pt]{red-cursor}%
					}%
				}%
			\end{myps}
			\mycaption[fig-OutilDeSelectionAmeliore]{Outil de sélection amélioré}
		\end{myfigure}
	\end{myframe}
	\section{Étude du travail collaboratif}
	\begin{myframe}<1-7>[label={fra-sota-PrimitivesComportementales}]{Primitives Comportementales (PC)}
		\begin{columns}[T]
			\begin{column}{0.45\textwidth}
				\begin{myfigure}
					\psset{xunit=1cm,yunit=1.1cm}
					\begin{myps}(-2.5,-0.5)(2.5,5)%
						\only<1,3-7,9>{%
							\mynode(0,4)[Search]{Recherche}%
						}%
						\only<2,8>{%
							\mynode[fillstyle=solid,fillcolor=myred!25](0,4)[Search]{Recherche}%
							\mycirclenumber(0,4){1}%
						}%
						\only<1-2,4-7,9>{%
							\mynode(0,3)[Selection]{Sélection}%
						}%
						\only<3,8>{%
							\mynode[fillstyle=solid,fillcolor=myred!25](0,3)[Selection]{Sélection}%
							\mycirclenumber(0,3){2}%
						}%
						\only<1-3,5-8>{%
							\mynode(0,2)[Manipulation]{Manipulation}%
						}%
						\only<4,9>{%
							\mynode[fillstyle=solid,fillcolor=myred!25](0,2)[Manipulation]{Manipulation}%
							\mycirclenumber(0,2){3}%
						}%
						\only<1-4,6-8>{%
							\mynode(0,1)[Evaluation]{Évaluation}%
						}%
						\only<5,9>{%
							\mynode[fillstyle=solid,fillcolor=myred!25](0,1)[Evaluation]{Évaluation}%
							\mycirclenumber(0,1){4}%
						}%
						\ncline{->}{Search}{Selection}%
						\ncline{->}{Selection}{Manipulation}%
						\ncline{->}{Manipulation}{Evaluation}%
						\only<6->{%
							\ncloop[loopsize=4em,angleA=-90,angleB=90,linearc=0.05,armA=0.2]{->}{Evaluation}{Search}%
						}%
						\only<6>{%
							\mycirclenumber[180](0,2.5){5}%
						}%
						\only<7>{%
							\mynode[fillstyle=solid,fillcolor=myred!25](0,0)[Objective]{Objectif atteint}%
							\mycirclenumber(0,0){6}%
						}%
						\only<8->{%
							\mynode(0,0)[Objective]{Objectif atteint}%
						}%
						\only<7->{%
							\ncline{->}{Evaluation}{Objective}%
						}%
					\end{myps}
					\mycaption[fig-ManipulationMoleculaire]{Manipulation moléculaire}
				\end{myfigure}
			\end{column}
			\only<1-7>{%
				\begin{column}{0.5\textwidth}%
					\begin{myblock}{Description}%
						Basé sur les PC Virtuelles \mycite{Fuchs-2006a}%
						\begin{description}%
							\item<2->[Recherche] Identifier une cible (atome, résidue, \dots)%
							\item<3->[Sélection] Sélectionner la structure moléculaire identifiée%
							\item<4->[Manipulation] Déplacer ou orienter la structure moléculaire%
								\note[item]<4->{Ceci a pour effet de déformer la molécule qui peut être considéré comme un corps articulé.}%
							\item<5->[Évaluation] Évaluer l'équilibre physico-chimique%
							\item<6->[Recommencer] Si l'évaluation n'est pas satisfaisante%
						\end{description}%
					\end{myblock}%
				\end{column}%
			}
		\end{columns}
	\end{myframe}
	\subsection{Étude~\mynum{1} -- Recherche et sélection collaborative de résidus}
	\begin{myframe}{Sommaire}
		\tableofcontents[sectionstyle=show/shaded,subsectionstyle=show/shaded/hide,subsubsectionstyle=show/show/hide]
	\end{myframe}
	\againframe<8>{fra-sota-PrimitivesComportementales}
	\subsubsection{Objectifs}
	\begin{myframe}{Objectifs}
		\begin{myblock}{Objectif principal}
			Étudier la contribution et les contraintes de la collaboration dans une tâche de recherche et de sélection de structures moléculaires
		\end{myblock}
		\begin{myplusblock}{Hypothèses}
			\begin{enumerate}
				\item Amélioration des performances (individuel \myvs collaboratif)
				\item Identifier les stratégies de travail
				\item Utilisabilité de la plate-forme
			\end{enumerate}
		\end{myplusblock}
		\begin{myblock}{Variables}
			\begin{description}
				\item[Nombre de sujets] monôme (\mynum{24}~sujets) ou binôme (\mynum{12}~couples)
				\item[Complexité de la tâche] Forme, nature, position, similarités\dots{}
			\end{description}
		\end{myblock}
	\end{myframe}
	\subsubsection{Présentation de la tâche proposée}
	\begin{myframe}{Présentation de la tâche proposée}
		\renewcommand{\schemafactor}{0.11}
		\setlength{\schemaunit}{\schemafactor\paperwidth}
		\psset{unit=\schemaunit}
		\begin{myfigure}
			\begin{myps}(-4,-2.3)(4,2.3)
				\rput(-1.75,0){%
					\myimage[height=2\schemaunit]{trp-cage}}
				\rput(1.25,0){%
					\myimage[height=2\schemaunit]{prion}}
				\rput(-3.5,0){%
					\myimage[height=\schemaunit]{pattern1}}
				\rput(-1,1.5){%
					\myimage[width=\schemaunit]{pattern3-8}}
				\rput(1,1.5){%
					\myimage[width=\schemaunit]{pattern2-7}}
				\rput(-1,-1.5){%
					\myimage[width=\schemaunit]{pattern4-9}}
				\rput(1,-1.5){%
					\myimage[width=\schemaunit]{pattern5-10}}
				\rput(3.5,0){%
					\myimage[height=\schemaunit]{pattern6}}

				\psset{framesize=1 1}
				\fnode(-3.5,0){P1}
				\uput[90](-3.5,0.5){Residue~1}
				\fnode(-1,1.5){P38}
				\uput[90](-1,2){Residue~3 and 8}
				\fnode(1,1.5){P27}
				\uput[90](1,2){Residue~2 and 7}
				\fnode(-1,-1.5){P49}
				\uput[-90](-1,-2){Residue~4 and 9}
				\fnode(1,-1.5){P510}
				\uput[-90](1,-2){Residue~5 and 10}
				\fnode(3.5,0){P6}
				\uput[90](3.5,0.5){Residue~6}

				\psset{linecolor=myred}
				\cnode(-1.5,0.3){0.2}{TRPP1}
				\cnode(-2,0.15){0.2}{TRPP38}
				\cnode(-1.25,-0.1){0.2}{TRPP27}
				\cnode(-2.2,-0.5){0.2}{TRPP49}
				\cnode(-1.25,-0.65){0.2}{TRPP510}
				\ncline{-}{P1}{TRPP1}
				\ncline{-}{P38}{TRPP38}
				\ncline{-}{P27}{TRPP27}
				\ncline{-}{P49}{TRPP49}
				\ncline{-}{P510}{TRPP510}

				\psset{linecolor=myblue}
				\cnode(-0.2,0.4){0.2}{PrionP38}
				\cnode(2.8,0.6){0.2}{PrionP27}
				\cnode(0.8,0.2){0.2}{PrionP49}
				\cnode(1.7,-0.7){0.2}{PrionP510}
				\cnode(1.4,0.0){0.2}{PrionP6}
				\ncline{-}{P38}{PrionP38}
				\ncline{-}{P27}{PrionP27}
				\ncline{-}{P49}{PrionP49}
				\ncline{-}{P510}{PrionP510}
				\ncline{-}{P6}{PrionP6}
			\end{myps}
			\mycaption[fig-RepartitionDesResiduesSurLesMoleculesTRPCageEtPrion]{Répartitions des \emph{residues} sur les molécules (TRP-Cage et Prion)}
		\end{myfigure}
	\end{myframe}
	\subsubsection{Résultats}
	\begin{myframe}{Amélioration des performances en collaboration}
		\begin{myfigure}
			\mylegend{%
				\myleg{monôme}{myblue}%
				\myleg{binôme}{myblue!70}%
			}
			\begin{myboxgraph}[llx=-3em,yAxisLabelPos={-3em,c}](10,0.75\textwidth)[100]{résidu}(500,3cm){temps~(s)}
				\myboxplot{exp1-time-residue-group.csv}%
				\only<2>{%
					\psframe[linecolor=myred,linewidth=2pt,framearc=0.25,dimen=inner](0,0)(5,200)%
					\psframe[linecolor=myred,linewidth=2pt,framearc=0.25,dimen=inner](6,0)(8,200)%
				}%
				\only<3,5>{%
					\psframe[linecolor=mygreen,linewidth=2pt,framearc=0.25,dimen=inner](5,0)(6,300)%
				}%
				\only<3-4>{%
					\psframe[linecolor=mygreen,linewidth=2pt,framearc=0.25,dimen=inner](8,0)(10,450)%
				}%
				\only<4>{%
					\uput[90](9,450){\textcolor{mygreen}{Recherche}}%
				}%
				\only<5>{%
					\uput[90](5.5,300){\textcolor{mygreen}{Sélection}}%
					\note[item]<5>{C'est cette constatation qui nous a amené à améliorer les outils de sélection.}
				}%
			\end{myboxgraph}
			\mycaption[fig-TempsDeRealisationDeLaTache]{Temps de réalisation de la tâche}
		\end{myfigure}
		\begin{myblock}{Synthèse}
			\begin{itemize}
				\item<2-> Pas d'évolution sur les tâches simples
				\item<3-> Une amélioration significative de la collaboration sur les tâches complexes
			\end{itemize}
		\end{myblock}
%		\begin{myfigure}
%			\mylegend{%
%				\myleg{recherche}{myblue}%
%				\myleg{sélection}{myblue!70}%
%			}
%			\begin{myboxgraph}[llx=-3em,yAxisLabelPos={-3em,c}](10,0.75\textwidth)[100]{résidu}(500,2cm){temps~(s)}
%				\myboxplot{exp1-timeaudio-residue-searchselection.csv}
%				\only<2>{%
%					\psframe[linecolor=myred,linewidth=2pt,framearc=0.25,dimen=inner](0,0)(5,100)
%					\psframe[linecolor=myred,linewidth=2pt,framearc=0.25,dimen=inner](6,0)(8,100)
%				}
%				\only<3>{%
%					\psframe[linecolor=mygreen,linewidth=2pt,framearc=0.25,dimen=inner](5,0)(6,250)
%					\psframe[linecolor=mygreen,linewidth=2pt,framearc=0.25,dimen=inner](8,0)(10,450)
%				}
%			\end{myboxgraph}
%			\mycaption[fig-exp1-TempsDeRechercheEtDeSelectionCompares]{Temps de recherche et de sélection comparés}
%		\end{myfigure}
	\end{myframe}
	\begin{myframe}{Stratégies de travail}
		\begin{myfigure}
			\begin{myboxgraph}[lly=-5ex,xAxisLabelPos={c,-5ex}](12,0.75\textwidth)[4]{binôme}(24,2cm){distance~(mm)}
				\psframe*[linecolor=myred!25,dimen=inner](0,0)(12,8)
				\psframe*[linecolor=mygreen!25](0,8)(12,14)
				\psframe*[linecolor=myblue!25](0,14)(12,20)
				\psline[linewidth=0.5pt,linestyle=dashed,arrows=-](0,4)(12,4)
				\psline[linewidth=0.5pt,linestyle=dashed,arrows=-](0,8)(12,8)
				\psline[linewidth=0.5pt,linestyle=dashed,arrows=-](0,12)(12,12)
				\psline[linewidth=0.5pt,linestyle=dashed,arrows=-](0,16)(12,16)
				\psline[linewidth=0.5pt,linestyle=dashed,arrows=-](0,20)(12,20)
				% Once header are readed, they are defined for other barplot
				% That's why barplots without headers are in first position
				\mybarplot[header=false,barstyle=third-barstyle]{exp1-diff-groups3.csv}
				\mybarplot[header=false,barstyle=second-barstyle]{exp1-diff-groups2.csv}
				\mybarplot[header=true,barstyle=first-barstyle]{exp1-diff-groups1.csv}
				\psaxes[linecolor=myred,xAxis=false,Dy=4,labels=none,yAxisLabelPos={2em,c},yticksize=0pt 3pt]{->}(12,0)(0,24)
				\uput[0](12,0){\textcolor{myred}{0}}
				\uput[0](12,4){\textcolor{myred}{1}}
				\uput[0](12,8){\textcolor{myred}{2}}
				\uput[0](12,12){\textcolor{myred}{3}}
				\uput[0](12,16){\textcolor{myred}{4}}
				\uput[0](12,20){\textcolor{myred}{5}}
				\uput{1.5em}[0](12,10){\rotateright{\textcolor{myred}{affinité~(1--5)}}}
				\pstScalePoints(1,4){}{}
				\readdata{\affinitygroupsdata}{files/exp1-affinity-groups.csv}
				\listplot[arrows={-},linecolor=myred,linewidth=2pt,showpoints=true,shadow=false]{\affinitygroupsdata}
				\psset{linewidth=0.1pt,linecolor=white,fillstyle=solid,fillcolor=myred}
				\uput[180](12,5){\pscharpath{\Large\bf\sffamily Champ proche}}
				\psset{fillcolor=mygreen}
				\uput[180](12,11){\pscharpath{\Large\bf\sffamily Champ voisin}}
				\psset{fillcolor=myblue}
				\uput[180](12,17){\pscharpath{\Large\bf\sffamily Champ distant}}
			\end{myboxgraph}
			\mycaption[fig-DistanceMoyenneEntreLeCurseurDesSujets]{Distance moyenne entre le curseur des sujets}
		\end{myfigure}
		\begin{myblock}{Synthèse}
			Trois stratégies liée à l'affinité entre les collaborateurs
			\begin{description}
				\item[Champs distants] Peu de collaboration avec peu de conflits de coordination
				\item[Champs voisins] Bonne collaboration avec conflits de coordination
				\item[Champs proches] Forte collaboration mais conflits de coordination importants
			\end{description}
		\end{myblock}
	\end{myframe}
	\subsubsection{Synthèse}
	\begin{myframe}{Synthèse}
		\begin{columns}[t]
			\begin{column}{0.475\textwidth}
				\centering
				\myunode[90][shadowcolor=myred](0,0)[complexe-problem]{Complexité de la tâche}[4cm]
				\begin{myplusblock}{Résultats}
					\begin{itemize}
						\item Amélioration des performances sur les tâches complexes
						\item Distribution des charges de travail dépendante de la nature de la tâche
					\end{itemize}
				\end{myplusblock}
				\vfill
				\begin{myminusblock}{Limites}
					\begin{itemize}
						\item Comment définir une tâche \myemph{complexe} ?
						\item La complexité de la tâche influe-t-elle sur les performances ?
							\note[item]{L'analyse de l'influence de la complexité de la tâche sur les performances sera l'un des objets de la prochaine étude.}
					\end{itemize}
				\end{myminusblock}
			\end{column}
			\begin{column}{0.475\textwidth}
				\centering
				\myunode[90][shadowcolor=myred](0,0)[strategy-problem]{Stratégie de travail}[4cm]
				\begin{myplusblock}{Résultats}
					\begin{itemize}
						\item Trois stratégies différentes
						\item Meilleurs résultats avec une stratégie en champs voisins
					\end{itemize}
				\end{myplusblock}
				\vfill
				\begin{myminusblock}{Limites}
					\begin{itemize}
						\item Modification du comportement naturel des groupes
						\item Conflits de coordination en champs voisins
							\note[item]{Dans la prochaine étude, nous allons également tenter de mettre en évidence les conflits de coordination.}
					\end{itemize}
				\end{myminusblock}
			\end{column}
		\end{columns}
	\end{myframe}
	\subsection{Étude~\mynum{2} -- Déformation collaborative de molécules}
	\begin{myframe}{Sommaire}
		\tableofcontents[sectionstyle=show/shaded,subsectionstyle=show/shaded/hide,subsubsectionstyle=show/show/hide]
	\end{myframe}
	\againframe<9>{fra-sota-PrimitivesComportementales}
	\subsubsection{Objectifs}
	\begin{myframe}{Objectifs}
		\begin{myblock}{Objectif principal}
			Quantifier et qualifier les conflits de coordination en fonction de la complexité de la tâche
		\end{myblock}
		\begin{myplusblock}{Hypothèses}
			\begin{enumerate}
				\item Amélioration des performances (bimanuel \myvs collaboratif)
				\item La complexité de la tâche influence différemment les performances individuelles et collaboratives
			\end{enumerate}
		\end{myplusblock}
		\begin{myblock}{Variables}
			\begin{description}
				\item[Nombre de sujets] monôme (\mynum{12}~sujets) ou binôme (\mynum{12}~couples)
				\item[Complexité de la molécule] \mynum{2}~molécules (\myTRPZIPPER et \myTRPCAGE)
				\item[Outil de déformation] 2~configuration de déformation (\emph{atom} et \emph{residue})
					\note[item]{L'objectif des différents outil de déformation est d'étudier différents niveaux de coordination pour qualifier et quantifier les conflits de coordination en fonction de la situation.}
			\end{description}
		\end{myblock}
	\end{myframe}
	\subsubsection{Présentation de la tâche proposée}
	\begin{myframe}<1>[label={fra-exp2-PresentationDeLaTacheProposee}]{Présentation de la tâche proposée}
		\begin{columns}[t]
			\begin{column}{0.3\textwidth}
				\only<1>{%
					\begin{myblock}{Scénarios}
						\begin{itemize}
							\item \mynum{2}~niveaux de manipulation
								\note[item]{Les deux niveaux de manipulation vont nous permettre d'observer différentes situation de conflits de coordination afin de mieux les qualifier et de les quantifier.}
								\begin{itemize}
									\item Résiduel
									\item Atomique
								\end{itemize}
							\item \mynum{4}~niveaux de complexité
								\note[item]{Afin d'observer les variations de performances en fonction de la complexité, nous proposons deux niveaux différents de complexité.
									Le seul facteur actuel de complexité que nous avons constaté concerne le nombre d'atomes.}
								\begin{itemize}
									\item Nombre d'atomes
									\item Cassures
									\item Champ de force
								\end{itemize}
						\end{itemize}
					\end{myblock}
				}
				\only<2>{%
					\begin{myblock}{Scénarios}
						\mynum{2}~niveaux de complexité
						\begin{itemize}
							\item faiblement couplé
							\item fortement couplé
						\end{itemize}
					\end{myblock}
				}
			\end{column}
			\begin{column}{0.65\textwidth}
				\renewcommand{\schemafactor}{0.045}
				\setlength{\schemaunit}{\schemafactor\paperwidth}
				\psset{unit=\schemaunit}
				\begin{myfigure}
					\begin{myps}(-1,0)(12,9)
						\rput[bl](1,0){\myimage[width=10\schemaunit]{TRP-ZIPPER}}
						\pnode(6.8,3.6){deformed}
						\rput(9.5,2.45){\rnode{deformed-label}{\textcolor{myred}{\mymultiline{molécule\\courante}}}}
						\pnode(1.8,4){ghost}
						\rput(0.5,5.25){\rnode{ghost-label}{\textcolor{myred}{\mymultiline{molécule\\cible}}}}
						\psset{linecolor=myblue}
						\cnode(6.2,4.9){1.0}{deformed-residue}
						\rput(7.7,7.4){\rnode{deformed-residue-label}{\textcolor{myblue}{résidue courant}}}
						\cnode(2.3,1.6){0.8}{ghost-residue}
						\rput(0.5,2.75){\rnode{ghost-residue-label}{\textcolor{myblue}{\mymultiline{résidue\\cible}}}}
						\psset{linecolor=gray}
						\cnode(2.0,6.6){0.8}{fixed-residue}
						\rput(4.0,8){\rnode{fixed-residue-label}{\textcolor{gray}{résidu fixe}}}
						\psset{linewidth=1pt,linecolor=myred,linearc=.1,arrowsize=0.5pt 3,arrowinset=.2,nodesepA=3pt}
						\ncangle[angleA=90,angleB=0]{c->}{deformed-label}{deformed}
						\psset{nodesepB=0pt}
						\ncdiagg[angleA=-90,angleB=135]{c->}{ghost-label}{ghost}
						\psset{linecolor=myblue}
						\ncdiagg[angleA=-90]{c->}{deformed-residue-label}{deformed-residue}
						\ncdiagg[angleA=-90]{c->}{ghost-residue-label}{ghost-residue}
						\ncdiagg[angleA=180,linecolor=gray]{c->}{fixed-residue-label}{fixed-residue}
						\ncline[linewidth=8pt,linecolor=mygreen]{C->}{deformed-residue}{ghost-residue}
						\only<1>{%
							\psframe*[linecolor=green](-1,8.5)(5,9)
							\psframe*[linecolor=red](5,8.5)(12,9)
							\rput(5.5,8.75){\textcolor{white}{\bf score \textsc{rmsd}}}
							\psframe[linewidth=1pt,linecolor=black](-1,0)(12,9)
						}
						\only<2>{%
							\psframe*[linecolor=blue](11.25,0)(12,2)
							\psline[linecolor=blue](11,2)(11.25,2)
							\uput{1pt}[180](11,2){\tiny\textcolor{blue}{RMSD courant}}
							\psframe*[linecolor=orange](11.45,0)(11.8,1)
							\psline[linecolor=orange](11,1)(11.45,1)
							\uput{1pt}[180](11,1){\tiny\textcolor{orange}{RMSD minimum}}
							\psframe[linewidth=1pt,linecolor=black](-1,0)(12,9)
						}
					\end{myps}
					\mycaption[fig-TacheDeDeformation]{Tâche de déformation}
				\end{myfigure}
			\end{column}
		\end{columns}
	\end{myframe}
	\subsubsection{Résultats}
	\begin{myframe}{Amélioration des performances}
		\begin{columns}[T]
			\begin{column}{0.65\textwidth}
				\begin{myfigure}
					\mylegend{%
						\myleg{monôme}{myblue}%
						\myleg{binôme}{myblue!70}%
					}
					\begin{myboxgraph}[llx=-2em,yAxisLabelPos={-2em,c}](2,0.8\textwidth){distance}(3.5,1.9cm){distance~(mm)}
						\myboxplot{exp2-diff-activepassive-group.csv}
					\end{myboxgraph}
					\mycaption[fig-exp2-DistancePassiveEtActive]{Distances passive et active}
				\end{myfigure}
			\end{column}
			\begin{column}{0.3\textwidth}
				\note{%
					Nous avons constaté avec surprise que la manipulation bimanuelle permettait d'obtenir des espaces de travail plus important qu'en monomanuel : en d'autres termes, une personne seule couvre une plus grande surface qu'un binôme.
					En corrélant les observations durant l'expérimentation avec ce résultat, nous avons compris que la distance ne représente pas une mesure fiable.
					En effet, les manipulateurs seuls n'utilisent pas toujours les deux outils à leur disposition; dans le cas où ils n'en utilisent qu'un seul, le second est laissé sur le côté ce qui a pour effet de gonfler la mesure de distance bien qu'un des outils soit passif (d'où le nom de distance passive).
					Nous avons donc introduit la mesure de distance active qui ne mesure la distance que lorsque les deux outils sont en action (ils ont sélectionné une structure moléculaire).
				}
				\begin{myblock}{Synthèse}
					Manipulation plus efficace en monomanuel
				\end{myblock}
			\end{column}
		\end{columns}
		\begin{columns}[T]
			\begin{column}{0.65\textwidth}
				\begin{myfigure}
					\mylegend{%
						\myleg{main dominante}{myblue}%
						\myleg{main dominée}{myblue!70}%
					}
					\begin{myboxgraph}(2,0.8\textwidth)[10]{nombre de sujets}(50,1.9cm){sélections~(nb)}
						\myboxplot{exp2-numsel-group-dominant.csv}
					\end{myboxgraph}
					\mycaption[fig-exp2-NombreDeSelections]{Nombre de sélections}
				\end{myfigure}
			\end{column}
			\begin{column}{0.3\textwidth}
				\begin{myblock}{Synthèse}
					Meilleure utilisation des ressources disponibles
				\end{myblock}
			\end{column}
		\end{columns}
	\end{myframe}
	\begin{myframe}{Influence de la complexité de la tâche}
		\begin{myfigure}
			\mylegend{%
				\myleg{monôme}{myblue}%
				\myleg{binôme}{myblue!70}%
			}
			\begin{myboxgraph}[llx=-3em,yAxisLabelPos={-3em,c}](4,0.75\textwidth)[100]{scénario}(350,1.75cm){temps~(s)}
				\myboxplot{exp2-time-task-group.csv}
			\end{myboxgraph}
			\mycaption[fig-exp2-TempsDeRealisationDesScenariosEnFonctionDuNombreDeSujets]{Temps de réalisation des scénarios}
		\end{myfigure}
		\begin{mytable}
			\begin{tabular}{cp{7cm}c}
				\hline
				Difficulté & Description & Exemple \\
				\hline
				\hline
				\multirow{2}*{Simple} & -- \mynum{1}~outil est nécessaire & \multirow{2}*{Tâche~1a} \\
				& -- \mynum{1}~manipulation \\
				\hline
				\multirow{2}*{Avancé} & -- \mynum{1}~outil est suffisant mais \mynum{2} sont préférables & \multirow{2}*{Tâche~2a, 2b} \\
				& -- \mynum{2}~manipulations peuvent être coordonnées \\
				\hline
				\multirow{2}*{Expert} 	& -- \mynum{2}~outils sont nécessaires & \multirow{2}*{Tâche~1b} \\
				& -- \mynum{2}~manipulations \alert{doivent} être coordonnées \\
				\hline
			\end{tabular}
			\mycaption[tab-ClassificationDesTaches]{Classification des tâches}
		\end{mytable}
		\note[item]{La distinction entre les scénario \myemph{simple} et les scénarios \myemph{avancés} réside dans les cassures à réaliser qui s'effectuent de manière plus efficaces avec \mynum{2}~outils de manipulation.}
		\note[item]{Les scénarios de type expert possèdent des champs de force trop important pour être appréhendés à l'aide d'un seul outil.}
	\end{myframe}
	\subsubsection{Synthèse}
	\begin{myframe}{Synthèse}
		\begin{columns}[t]
			\begin{column}{0.475\textwidth}
				\centering
				\myunode[90][shadowcolor=mygreen](0,0)[distribution-problem]{Charge de travail}[4cm]
				\begin{myplusblock}{Résultats}
					\begin{itemize}
						\item Gestion d'un espace de travail plus grand
						\item Meilleur rendement des ressources disponibles
					\end{itemize}
				\end{myplusblock}
				\begin{myminusblock}{Limites}
					\begin{itemize}
						\item Comment répartir équitablement la charge de travail ?
					\end{itemize}
				\end{myminusblock}
			\end{column}
			\begin{column}{0.475\textwidth}
				\centering
				\myunode[90][shadowcolor=mygreen](0,0)[conflict-problem]{Conflits de coordination}[4cm]
				\begin{myplusblock}{Résultats}
					\begin{itemize}
						\item Certaines manipulations nécessitent une coordination
					\end{itemize}
				\end{myplusblock}
				\begin{myminusblock}{Limites}
					\begin{itemize}
						\item La coordination est plus efficace en individuel mais\dots
							\begin{itemize}
								\item \dots{}espace de travail restreint
									\note[item]{Dans un environnement complexe, nous souhaitons pouvoir effectuer des manipulations avec plus de \mynum{2}~tâches élémentaires ce qui implique aussi de pouvoir couvrir un espace de travail plus grand.}
								\item \dots{}nombre réduit de tâches élémentaires en parallèle
									\note[item]{Le contexte proposé dans cette étude est encore de taille et de complexité relativement réduite comparé aux besoins réels; on peut aisément imaginer des tâches plus complexe impliquant de nombreuses tâches élémentaires à réaliser.}
							\end{itemize}
					\end{itemize}
				\end{myminusblock}
			\end{column}
		\end{columns}
		\note[item]{La prochaine étude nous permet d'analyser des environnements plus complexes en introduisant plus de participants dans la collaboration; notre intuition est une augmentation des conflits de coordination liés à la difficulté de communication dans un groupe de plus de \mynum{2}~personnes.}
	\end{myframe}
	\subsection{Étude~\mynum{3} -- Dynamique de groupe}
	\begin{myframe}{Sommaire}
		\tableofcontents[sectionstyle=show/shaded,subsectionstyle=show/shaded/hide,subsubsectionstyle=show/show/hide]
	\end{myframe}
	\subsubsection{Notions importantes sur la dynamique de groupe}
	\begin{myframe}[label=current]{Notions importantes sur la dynamique de groupe}
		\begin{columns}[T]
			\begin{column}{0.475\textwidth}
				\begin{myblock}{Facilitation sociale \mycite{Triplett-1898}}
					Une action collaborative préparée ou en progression possède une réponse; la stimulation sociale provoque une augmentation de cette réponse par la perception de collaborateurs effectuant les mêmes mouvements.
				\end{myblock}
				\begin{myfigure}
					\begin{myboxgraph}(3,0.8\textwidth)[10]{condition}(60,2cm){vitesse~(km/h)}
						\mybarplot{exp3-triplett.csv}
						\mybarlabel(0.5,38.650022682){\mynum[km/h]{39}}
						\mybarlabel(1.5,50.161371429){\mynum[km/h]{50}}
						\mybarlabel(2.5,52.502386951){\mynum[km/h]{52}}
					\end{myboxgraph}
					\mycaption[fig-sota-PerformancesDeCyclistes]{Performances de cyclistes}
				\end{myfigure}
			\end{column}
			\begin{column}{0.475\textwidth}
				\begin{myblock}{Paresse sociale \mycite{Ringelmann-1913}}
					Tendance à fournir un effort moindre lorsqu'une tâche est effectuée en groupe plutôt que de manière individuelle.
				\end{myblock}
				\begin{myfigure}
					\begin{myboxgraph}(8,0.8\textwidth)[0.2]{taille du groupe}(1,2cm){performance}
						\mybarplot{exp3-ringelmann.csv}
						\mybarlabel(0.5,1.00){\rotateleft{\mynum[\%]{100}}}
						\mybarlabel(1.5,0.93){\rotateleft{\mynum[\%]{93}}}
						\mybarlabel(2.5,0.85){\rotateleft{\mynum[\%]{85}}}
						\mybarlabel(3.5,0.77){\rotateleft{\mynum[\%]{77}}}
						\mybarlabel(4.5,0.70){\rotateleft{\mynum[\%]{70}}}
						\mybarlabel(5.5,0.63){\rotateleft{\mynum[\%]{63}}}
						\mybarlabel(6.5,0.56){\rotateleft{\mynum[\%]{56}}}
						\mybarlabel(7.5,0.49){\rotateleft{\mynum[\%]{49}}}
					\end{myboxgraph}
					\mycaption[fig-sota-PerformanceAuTirALaCorde]{Performances au tir à la corde}
				\end{myfigure}
			\end{column}
		\end{columns}
	\end{myframe}
	\subsubsection{Objectifs}
	\begin{myframe}{Objectifs}
		\begin{myblock}{Objectif principal}
			Observer la dynamique de groupe lors d'une coordination étroitement couplée
		\end{myblock}
		\begin{myplusblock}{Hypothèses}
			\begin{enumerate}
				\item Amélioration des performances en fonction du nombre de sujets
				\item Analyse des rôles dans le groupe
				\item Influence d'une étape de \mybrainstorming sur les performances
			\end{enumerate}
		\end{myplusblock}
		\begin{myblock}{Variables}
			\begin{description}
				\item[Nombre de participants] \mynum{8}~couples et \mynum{4}~groupes
				\item[Tâches différentes] \mynum{2}~molécules (tâche faiblement et fortement couplées)
				\item[Stratégie de travail] Étape de \mybrainstorming
			\end{description}
		\end{myblock}
	\end{myframe}
	\subsubsection{Protocole expérimental}
	\againframe<2>{fra-exp2-PresentationDeLaTacheProposee}
	\subsubsection{Résultats}
	\begin{myframe}{Amélioration des performances}
		\begin{columns}[T]
			\begin{column}{0.65\textwidth}
				\begin{myfigure}
					\mylegend{%
						\myleg{binôme}{myblue}%
						\myleg{quadrinôme}{myblue!70}%
					}
					\begin{myboxgraph}(2,0.8\textwidth)[0.5]{scénario}(1.75,1.9cm){vitesse~(mm/s)}
						\myboxplot{exp3-speed-molecule-group.csv}
					\end{myboxgraph}
					\mycaption[fig-exp3-VitesseMoyenne]{Vitesse moyenne}
				\end{myfigure}
			\end{column}
			\begin{column}{0.3\textwidth}
				\begin{myblock}{Synthèse}
					Une vitesse moyenne de travail supérieur : phénomène de facilitation sociale
				\end{myblock}
			\end{column}
		\end{columns}
		\begin{columns}[T]
			\begin{column}{0.65\textwidth}
				\begin{myfigure}
					\mylegend{%
						\myleg{binôme}{myblue}%
						\myleg{quadrinôme}{myblue!70}%
					}
					\begin{myboxgraph}(2,0.8\textwidth)[10]{scénario}(40,1.9cm){échanges~(nb)}
						\myboxplot{exp3-talk-molecule-group.csv}
					\end{myboxgraph}
					\mycaption[fig-exp3-NombreDEchangeVerbaux]{Nombre d'échanges verbaux}
				\end{myfigure}
			\end{column}
			\begin{column}{0.3\textwidth}
				\begin{myblock}{Synthèse}
					Paresse sociale
					\begin{itemize}
						\item Spécialisation
						\item Personnalité
						\item Paresse
					\end{itemize}
				\end{myblock}
			\end{column}
		\end{columns}
	\end{myframe}
	\begin{myframe}{Influence du \mybrainstorming}
		\begin{columns}[T]
			\begin{column}{0.65\textwidth}
				\begin{myfigure}
					\mylegend{%
						\myleg{binôme}{myblue}%
						\myleg{quadrinôme}{myblue!70}%
					}
					\begin{myboxgraph}[llx=-3em,yAxisLabelPos={-3em,c}](2,0.8\textwidth)[300]{\mybrainstorming}(1100,1.9cm){temps~(s)}
						\myboxplot{exp3-time-brainstorm-group.csv}
					\end{myboxgraph}
					\mycaption[fig-exp3-TempsDeRealisation]{Temps de réalisation}
				\end{myfigure}
			\end{column}
			\begin{column}{0.3\textwidth}
				\begin{myblock}{Synthèse}
					Le \mybrainstorming permet l'élaboration d'une stratégie : gain en performances
				\end{myblock}
			\end{column}
		\end{columns}
		\begin{columns}[T]
			\begin{column}{0.65\textwidth}
				\begin{myfigure}
					\mylegend{%
						\myleg{binôme}{myblue}%
						\myleg{quadrinôme}{myblue!70}%
					}
					\begin{myboxgraph}[llx=-3em,yAxisLabelPos={-3em,c}](2,0.8\textwidth)[0.1]{\mybrainstorming}(0.45,1.9cm){sélections~(nb/s)}
						\myboxplot{exp3-freqsel-brainstorm-group.csv}
					\end{myboxgraph}
					\mycaption[fig-exp3-FrequenceDesSelections]{Fréquence des sélections}
				\end{myfigure}
			\end{column}
			\begin{column}{0.3\textwidth}
				\begin{myblock}{Synthèse}
					Meilleur rendement des actions effectuées
				\end{myblock}
			\end{column}
		\end{columns}
	\end{myframe}
	\subsubsection{Synthèse}
	\begin{myframe}{Synthèse}
		\begin{columns}[t]
			\begin{column}{0.475\textwidth}
				\centering
				\myunode[90][shadowcolor=myblue](0,0)[identification-problem]{Paresse sociale}[4cm]
				\begin{myplusblock}{Résultats}
					\begin{itemize}
						\item Déséquilibre important dans la répartition des charges de travail
						\item Potentiel collaboratif non-exploité au maximum
					\end{itemize}
				\end{myplusblock}
				\begin{myminusblock}{Limites}
					\begin{itemize}
						\item Comment redonner de l'importance à chaque membre du groupe ?
					\end{itemize}
				\end{myminusblock}
			\end{column}
			\begin{column}{0.475\textwidth}
				\centering
				\myunode[90][shadowcolor=myblue](0,0)[brainstorming-problem]{\myBrainstorming}[4cm]
				\begin{myplusblock}{Résultats}
					\begin{itemize}
						\item Amélioration importante des performances
						\item Conflits de communication pendant le \mybrainstorming
						\item Réduit les conflits de coordination
					\end{itemize}
				\end{myplusblock}
				\begin{myminusblock}{Limites}
					\begin{itemize}
						\item Comment optimiser cette étape ?
					\end{itemize}
				\end{myminusblock}
			\end{column}
		\end{columns}
	\end{myframe}
	\section{Aide au travail collaboratif}
	\subsection{Étude~\mynum{4} -- Assistance haptique et stratégie de travail}
	\begin{myframe}{Sommaire}
		\tableofcontents[sectionstyle=show/shaded,subsectionstyle=show/shaded/hide,subsubsectionstyle=show/show/hide]
	\end{myframe}
	\subsubsection{Synthèse des études effectuées}
	\begin{myframe}{Synthèse des études effectuées et solutions}
		\begin{myfigure}
			\psset{xunit=0.95cm,yunit=0.8cm}
			\begin{myps}(-6,-3)(6,4)
				\uput[-90](-2.5,4){\textcolor{black!50}{\Large Problématiques}}
				\uput[-90](2.5,4){\textcolor{black!50}{\Large Solutions}}
				\psset{arrowlength=1}
				\onslide<1->{%
					\myunode[0][shadowcolor=myred](-5,2.5)[complexe-problem]{\onslide<2-3>{\color{black!25}}Complexité de la tâche}[4cm]
					\myunode[0][shadowcolor=myred](-5,1.5)[strategy-problem]{\onslide<2-3>{\color{black!25}}Stratégie de travail}[4cm]
					\psbrace*[linecolor=myred,ref=lC,nodesepA=-1pt](-5,3)(-5,1){\rotateright{\textcolor{myred}{\tiny Étude~1}}}%
				}
				\onslide<2->{%
					\myunode[0][shadowcolor=mygreen](-5,0.5)[distribution-problem]{\onslide<3>{\color{black!25}}Charge de travail}[4cm]
					\myunode[0][shadowcolor=mygreen](-5,-0.5)[conflict-problem]{\onslide<3>{\color{black!25}}Conflits de coordination}[4cm]
					\psbrace*[linecolor=mygreen,ref=lC,nodesepA=-1pt](-5,1)(-5,-1){\rotateright{\textcolor{mygreen}{\tiny Étude~2}}}%
				}
				\onslide<3->{%
					\myunode[0][shadowcolor=myblue](-5,-1.5)[identification-problem]{Paresse sociale}[4cm]
					\myunode[0][shadowcolor=myblue](-5,-2.5)[brainstorming-problem]{\myBrainstorming}[4cm]
					\psbrace*[linecolor=myblue,ref=lC,nodesepA=-1pt](-5,-1)(-5,-3){\rotateright{\textcolor{myblue}{\tiny Étude~3}}}%
				}
				\onslide<4->{%
					\psbrace*[linecolor=black!50,ref=lC,nodesepA=1pt](5,-3)(5,3){\rotateright{\textcolor{black!50}{\tiny Étude~4}}}
					\myunode[180][shadowcolor=black!50](5,2.5)[complexe-solution]{Tâche de \mydocking}[4cm]%
					\ncline[linewidth=4pt,linecolor=myred!25]{->}{complexe-problem}{complexe-solution}
				}
				\onslide<5->{%
					\myunode[180][shadowcolor=black!50](5,1.5)[strategy-solution]{Manipulation de résidu}[4cm]%
					\ncline[linewidth=4pt,linecolor=myred!25]{->}{strategy-problem}{strategy-solution}
				}
				\onslide<6->{%
					\myunode[180][shadowcolor=black!50](5,0.5)[distribution-solution]{Identifier les rôles}[4cm]%
					\ncline[linewidth=4pt,linecolor=mygreen!25]{->}{distribution-problem}{distribution-solution}
				}
				\onslide<8->{%
					\myunode[180][shadowcolor=black!50](5,-1.5)[identification-solution]{Identifier les rôles}[4cm]%
					\ncline[linewidth=4pt,linecolor=myblue!25]{->}{identification-problem}{identification-solution}
				}
				\onslide<9->{%
					\myunode[180][shadowcolor=black!50](5,-2.5)[brainstorming-solution]{Phase exploratoire}[4cm]%
					\ncline[linewidth=4pt,linecolor=myblue!25]{->}{brainstorming-problem}{brainstorming-solution}
				}
				\onslide<7-9>{%
					\myunode[180][shadowcolor=black!50](5,-0.5)[conflict-solution]{Solutions haptiques}[4cm]%
					\ncline[linewidth=4pt,linecolor=mygreen!25]{->}{conflict-problem}{conflict-solution}
				}
				\onslide<10>{%
					\myunode[180][fillstyle=solid,fillcolor=myred!25,shadowcolor=black!50](5,-0.5)[conflict-solution]{Solutions haptiques}[4cm]%
					\ncline[linewidth=4pt,linecolor=mygreen!25]{->}{conflict-problem}{conflict-solution}
				}
			\end{myps}
			\mycaption[fig-SyntheseDesProblematiques]{Synthèse des problématiques}
		\end{myfigure}
	\end{myframe}
	\subsubsection{Présentation des solutions proposées}
	\begin{myframe}{Présentation des solutions proposées}
		\begin{columns}
			\begin{column}{0.6\textwidth}
				\begin{myfigure}
					\renewcommand{\schemafactor}{0.025}
					\setlength{\schemaunit}{\schemafactor\paperwidth}
					\psset{xunit=\schemaunit,yunit=0.531966204\schemaunit}
					\begin{myps}(-10,-10)(10,10)
						\only<1-2>{%
							\rput(0,0){\myimage[width=20\schemaunit]{designation-normal}}
							\uput{0pt}[45](-7,6){\myimage[width=2.75\schemaunit]{designation-yellow-cursor}}
						}
						\only<1>{%
							\uput{0pt}[45](2,7){\myimage[width=2.75\schemaunit]{designation-red-cursor}}
						}
						\only<3>{%
							\rput(0,0){\myimage[width=20\schemaunit]{designation-called}}
						}
						\only<2-3>{%
							\uput{0pt}[45](7.8,4){\myimage[width=2.75\schemaunit]{designation-red-cursor}}
							\uput{0pt}[45](-7,6){\myimage[width=2.75\schemaunit]{designation-yellow-cursor}}
						}
						\only<3>{%
							\mypsvibration(7.8,4)
							\mypsvibration(-7,6)
						}
						\only<4-5>{%
							\rput(0,0){\myimage[width=20\schemaunit]{designation-accepted}}
							\pscurve[linewidth=2pt,linecolor=mygreen]{->}(-7,6)(-1,10)(6,10)(7.5,8.9)
							\uput{0pt}[45](5,-8){\myimage[width=2.75\schemaunit]{designation-red-cursor}}
						}
						\only<4>{%
							\uput{0pt}[45](-7,6){\myimage[width=2.75\schemaunit]{designation-yellow-cursor}}
						}
						\only<6>{%
							\rput(0,0){\myimage[width=20\schemaunit]{designation-selected}}
						}
						\only<5-6>{%
							\uput{0pt}[45](8.25,8.1){\myimage[width=2.75\schemaunit]{designation-yellow-cursor}}
						}
					\end{myps}
					\mycaption[fig-OutilDeDesignation]{Outil de désignation}
				\end{myfigure}
			\end{column}
			\begin{column}{0.35\textwidth}
				\begin{myblock}{Étapes de la désignation}
					\begin{enumerate}
						\item<2-> Recherche d'une structure à manipuler (coordinateur)
						\item<3-> Désignation de la structure (coordinateur)
						\item<4-> Acceptation par le manipulateur
						\item<6-> Sélection par le manipulateur
					\end{enumerate}
				\end{myblock}
			\end{column}
		\end{columns}
	\end{myframe}
	\subsubsection{Objectifs}
	\begin{myframe}{Objectifs}
		\begin{myblock}{Objectif principal}
			Proposer et évaluer des outils haptiques pour assister la coordination
		\end{myblock}
		\begin{myplusblock}{Hypothèses}
			\begin{enumerate}
				\item Influence de l'outil proposé associé à la configuration
				\item Influence des propositions sur la communication
				\item Évaluations des propositions par des bio-informaticiens
			\end{enumerate}
		\end{myplusblock}
		\begin{myblock}{Variables}
			\begin{description}
				\item[Nombre de participants] \mynum{8}~trinômes
				\item[Tâches différentes] \mynum{2}~molécules (tâche faiblement et fortement couplée)
				\item[Métaphore haptique] Avec ou sans assistance
			\end{description}
		\end{myblock}
	\end{myframe}
	\subsubsection{Résultats}
	\begin{myframe}{Efficacité de la collaboration}
		\begin{columns}[T]
			\begin{column}{0.65\textwidth}
				\begin{myfigure}
					\mylegend{%
						\myleg{sans assistance}{myblue}%
						\myleg{avec assistance}{myblue!70}%
					}
					\begin{myboxgraph}[llx=-3em,yAxisLabelPos={-3em,c}](2,0.8\textwidth)[200]{scénario}(700,1.9cm){temps~(s)}
						\myboxplot{exp4-rmsd-time-molecule-haptic.csv}
					\end{myboxgraph}
					\mycaption[fig-exp4-TempsPourAtteindreLeScoreRMSDMinimum]{Temps pour atteindre le score \myRMSD minimum}
				\end{myfigure}
			\end{column}
			\begin{column}{0.3\textwidth}
				\begin{myblock}{Synthèse}
					Manipulation plus efficace sur le scénario le plus complexe
				\end{myblock}
			\end{column}
		\end{columns}
		\begin{columns}[T]
			\begin{column}{0.65\textwidth}
				\begin{myfigure}
					\mylegend{%
						\myleg{sans assistance}{myblue}%
						\myleg{avec assistance}{myblue!70}%
					}
					\begin{myboxgraph}[llx=-3.5em,yAxisLabelPos={-3.5em,c}](2,0.8\textwidth)[0.05]{scénario}(0.125,1.9cm){fréquence~(nb/s)}
						\myboxplot{exp4-freq-molecule-haptic.csv}
					\end{myboxgraph}
					\mycaption[fig-exp4-NombreDeSelectionsParSecondeEffectueesParUnOperateur]{Nombre de sélections par seconde effectuées par un opérateur}
				\end{myfigure}
			\end{column}
			\begin{column}{0.3\textwidth}
				\begin{myblock}{Synthèse}
					Meilleur rendement pour l'utilisation des ressources
				\end{myblock}
			\end{column}
		\end{columns}
	\end{myframe}
	\begin{myframe}{Amélioration de la communication}
		\begin{columns}[T]
			\begin{column}{0.65\textwidth}
				\begin{myfigure}
					\begin{myboxgraph}[llx=-2em,yAxisLabelPos={-2em,c}](2,0.6\textwidth)[2]{haptique}(10,1.9cm){temps~(s)}
						\myboxplot{exp4-shake-time-haptic.csv}
					\end{myboxgraph}
					\mycaption[fig-exp4-TempsDAcceptationDUneDesignation]{Temps d'acceptation d'une désignation}
				\end{myfigure}
			\end{column}
			\begin{column}{0.3\textwidth}
				\begin{myblock}{Synthèse}
					Communication haptique plus rapide que la communication verbale
				\end{myblock}
			\end{column}
		\end{columns}
		\begin{columns}[T]
			\begin{column}{0.65\textwidth}
				\begin{myfigure}
					\begin{myboxgraph}(2,0.6\textwidth)[5]{haptique}(17.5,1.9cm){sélections~(nb)}
						\myboxplot{exp4-accept-haptic.csv}
					\end{myboxgraph}
					\mycaption[fig-exp4-NombreDeDesignationsAcceptees]{Nombre de désignations acceptées}
				\end{myfigure}
			\end{column}
			\begin{column}{0.3\textwidth}
				\begin{myblock}{Synthèse}
					Meilleur taux d'acceptation pour les désignations du coordinateur
				\end{myblock}
			\end{column}
		\end{columns}
	\end{myframe}
	\section{Conclusion et perspectives}
	\subsection{Conclusions}
	\begin{myframe}{Conclusion}
		\begin{myblock}{Plateforme \myShaddock}
			\begin{itemize}
				\item Plateforme validée
				\item Des améliorations sont encore nécessaires
			\end{itemize}
		\end{myblock}
		\begin{myblock}{Travail collaboratif}
			\begin{itemize}
				\item Adapté pour l'appréhension de tâches très complexes
				\item Nécessité d'améliorer les canaux de communication
			\end{itemize}
		\end{myblock}
		\begin{myblock}{Communication haptique}
			\begin{itemize}
				\item Remplace la communication verbale dans certains cas
				\item Plus efficace et plus rapide
			\end{itemize}
		\end{myblock}
	\end{myframe}
	\subsection{Perspectives}
	\begin{myframe}{Perspectives}
		\begin{myblock}{Plus loin dans l'étude du travail collaboratif\dots}
			\begin{itemize}
				\item Collaboration distante
				\item Collaboration multi-experts
				\item Apprentissage en collaboration
			\end{itemize}
		\end{myblock}
		\begin{myblock}{Comment expérimenter le travail collaboratif ?}
			\begin{itemize}
				\item Comment mesurer les conflits de coordination et de communication ?
				\item Comment définir un protocole expérimental pour le collaboratif ?
				\item Comment mesure la charge de travail de chaque collaborateur ?
			\end{itemize}
		\end{myblock}
	\end{myframe}
	\subsection{Références}
	\nocite{Simard-2009}
	\nocite{Simard-2010a}
	\nocite{Simard-2010b}
	\nocite{Simard-2010c}
	\nocite{Simard-2011a}
	\nocite{Simard-2011b}
	\nocite{Simard-2012a}
	\nocite{Simard-2012b}
	\nocite{Simard-2012c}
	\nocite{Simard-2012d}
	\nocite{Girard-2012a}
	\nocite{Girard-2012b}
	\begin{myframe}{Publications internationales}
		\defbibheading{bibliography}[Journaux internationaux avec comité de relecture]{Journaux internationaux avec comité de relecture\par}
		\printbibliography[type=article,category=myrefs]
		\defbibheading{bibliography}[Conférences internationales avec comité de relecture]{Journaux internationaux avec comité de relecture\par}
		\printbibliography[type=inproceedings,category=myrefs]
	\end{myframe}
	\subsection{Questions}
	\begin{myframe}{Questions}
		Merci pour votre attention
	\end{myframe}
\end{document}
