\documentclass[english,french,dvips,10pt]{mybeamer}
\usepackage{my}
\usepackage{mydate}
\usepackage{mycolor}
\usepackage[smarttab]{myfloat}
\usepackage{myps}
\usepackage{pst-solides3d}
\usepackage{myuml}
\usepackage{pgfpages}
\usepackage{csquotes}
\usepackage[%
	sorting=mybibsort,%
	hyperref=true,%
	backend=biber,%
	style=authoryear,%
	citestyle=mycitestyle,%
	bibstyle=mybibliographystyle,%
	dashed=false,
	abbreviate=false,%
	date=short,%
	urldate=short,%
	block=space,%
	maxcitenames=1,%
	maxbibnames=99]{biblatex}

\makeatletter
\renewenvironment{alertenv}{%
	\begin{altenv}%
		{%
			\usebeamertemplate{alerted text begin}%
			\usebeamercolor[fg]{alerted text}%
			\usebeamerfont{alerted text}%
			\setbeamercolor{local structure}{fg=myred}%
			\setbeamerfont{description item}{series=\bfseries}%
		}{%
			\usebeamertemplate{alerted text end}%
		}{%
			\color{.}%
		}{}%
	}{%
	\end{altenv}%
}
\newenvironment{myshadeenv}{%
	\only{%
		\color{black!25}%
		\setbeamercolor{local structure}{fg=black!25}%
	}%
}{}
\newenvironment{mydescredenv}{%
	\only{%
		\setbeamercolor{description item}{fg=myred}%
	}%
}{}
\newenvironment{mydescgreenenv}{%
	\only{%
		\setbeamercolor{description item}{fg=mygreen}%
	}%
}{}
\newenvironment{mydescblueenv}{%
	\only{%
		\setbeamercolor{description item}{fg=myblue}%
	}%
}{}
\makeatother

% hyperref setup
\hypersetup{%
	pdftitle={Collaboration haptique étroitement couplée pour la manipulation moléculaire interactive},%
	pdfauthor={Jean SIMARD},%
	pdfkeywords={collaboration,haptique,environnement virtuel,simulation moléculaire},%
	pdflang={FR-fr},%
	pdfsubject={Soutenance de thèse en informatique}%
}
% caption setup
\captionsetup{figurename={},tablename={},justification=centering,font={bf,color=black!75}}

\makeatletter
% Modify the bibliography style
\newcounter{mymaxcitenames}
\AtBeginDocument{%
	\setcounter{mymaxcitenames}{\value{maxnames}}%
}
\renewbibmacro{begentry}{%
	\printtext[brackets]{%
		\defcounter{maxnames}{\value{mymaxcitenames}}%
		\printnames{labelname}~\usebibmacro{cite:labelyear+extrayear}%
	}%
	\newline%
}
% AAA
\newcommand{\myACER}{\textsc{acer}\xspace}
\newcommand{\myAlanine}{Alanine\xspace}
\newcommand{\myanalysis}[1]{\input{files/#1}\%}
\newcommand{\myangstrom}{\AA ngström\xspace}
\newcommand{\myanova}[1]{\input{files/#1}}
\newcommand{\myatom}[2][]{%
	{%
		\ifstrempty{#1}%
			{\makefirstuc{\textsf{#2}}}%
			{\textcolor{#1}{\makefirstuc{\textsf{#2}}}}%
		\xspace%
	}%
}
\newcommand{\myAudacity}{\textsc{audacity}\myregistered}% No '\xspace' because of already one in '\myregistered'
% CCC
\newcommand{\mycarbon}{\myatom[mycarboncolor]{C}}
\newcommand{\myCasioXJ}{\textsc{Casio xj}\xspace}
\newcommand{\myCHARMM}{\textsc{charmm}\xspace}
\newcommand{\myChimera}{\textsc{chimera}\xspace}
\newcommand{\myClayWorks}{\textsc{Clayworks}\xspace}
\newcommand{\mycondition}[1]{$\left(\mathcal{C}_{#1}\right)$\xspace}
\newcommand{\myCPK}{\textsc{cpk}\xspace}
% DDD
\newcommand{\myDesktop}{\myPHANToM Desktop\myregistered}% No '\xspace' because of already one in '\myregistered'
% FFF
\newcommand{\myfeuillet}{feuillet-$\beta$\xspace}
\WithSuffix\newcommand\myfeuillet*{feuillets-$\beta$\xspace}
\newcommand{\myform}[1]{\textbf{\sffamily\MakeUppercase{#1}}}
% GGG
\newcommand{\myGhost}{\textsc{Ghost}\xspace}
\newcommand{\myGromacs}{\textsc{Gromacs}\xspace}
\newcommand{\mygroup}[1]{$\left(\mathcal{G}_{#1}\right)$\xspace}
% HHH
\newcommand{\myHaption}{\textsc{Haption}\xspace}
\newcommand{\myHawthorne}{\myemph{Hawthorne Works}\xspace}
\newcommand{\myHBonds}{\textit{HBonds}\xspace}
\newcommand{\myhelice}{hélice-$\alpha$\xspace}
\WithSuffix\newcommand\myhelice*{hélices-$\alpha$\xspace}
\newcommand{\myhypothesis}[1]{$\left(\mathcal{H}_{#1}\right)$\xspace}
% III
\newcommand{\myIntelCore}{Intel\myregistered Core\mytrademark~2 \textsc{q9450} (\mynum[GHz]{2.66})\xspace}
% JJJ
\newcommand{\myJmol}{\textsc{Jmol}\xspace}
% LLL
\newcommand{\myLCD}{\textsc{lcd}\xspace}
\newcommand{\myLicorice}{\textit{Licorice}\xspace}
\newcommand{\myLinux}{\textsc{Linux}\xspace}
% MMM
\newcommand{\myMacOS}{Mac~\textsc{OS}\xspace}
\newcommand{\myMDDriver}{\textsc{MDDriver}\xspace}
% NNN
\newcommand{\myNewRibbon}{\textit{NewRibbon}\xspace}
\def\mynode{%
	\@ifnextchar[{\mynode@i}{\mynode@i[style=nodestyle]}%
}
\def\mynode@i[#1](#2,#3)[#4]#5{%
	\rput(#2,#3){\Rnode{#4}{\psframebox[style=nodestyle,#1]{\vphantom{pÉ}#5}}}%
}
\newcommand{\mynytrogen}{\myatom[mynytrogencolor]{A}}
\newcommand{\myNusE}{\textsc{NusE}\xspace}
\newcommand{\myNusENusG}{\textsc{NusE:NusG}\xspace}
\newcommand{\myNusG}{\textsc{NusG}\xspace}
% OOO
\newcommand{\myOmni}{\myPHANToM Omni\myregistered}% No '\xspace' because of already one in '\myregistered'
\newcommand{\myOpenHaptics}{\textsc{OpenHaptics}\mytrademark}% No '\xspace' because of already one in '\mytrademark'
\newcommand{\myoxygen}{\myatom[myoxygencolor]{O}}
% PPP
\newcommand{\myPC}{\textsc{pc}\xspace}
\newcommand{\myPDB}{\textsc{pdb}\xspace}
\newcommand{\myPDBbase}{\emph{Protein~DataBase}\xspace}
\newcommand{\myPDBlink}[2]{\href{#1}{\textsc{\MakeLowercase{#2}}}}
\newcommand{\myPHANToM}{\textsc{phant}o\textsc{m}\xspace}
\newcommand{\myPremium}{\myPHANToM Premium\myregistered}% No '\xspace' because of already one in '\myregistered'
\newcommand{\myPrion}{Prion\xspace}
\newcommand{\myPSF}{\textsc{psf}\xspace}
\newcommand{\mypvalue}{$p$-value\xspace}
\newcommand{\myPyMOL}{\textsc{p}y\textsc{mol}\xspace}
% RRR
\newcommand{\myRAM}[2][Go]{\mynum[#1]{#2} de \textsc{ram}}
\newcommand{\myRasmol}{\textsc{RasMol}\xspace}
\newcommand{\myresidue}[1]{$\left(\mathcal{R}_{#1}\right)$\xspace}
% SSS
\newcommand{\myscenario}[1]{\textsc{#1}}
\newcommand{\mySensAble}{\textsc{SensAble}\xspace}
\newcommand{\myShaddock}{\textsc{Shaddock}\xspace}
\newcommand{\mySony}{\textsc{sony}\myregistered}% No '\xspace' because of already one in '\myregistered'
\newcommand{\mySpaceNavigator}{SpaceNavigator\myregistered}% No '\xspace' because of already one in '\myregistered'
\newcommand{\mysubject}[1]{$\mathcal{S}_#1$}
\newcommand{\mysulfur}{\myatom[mysulfurcolor]{S}}
\newcommand{\mysummary}[1]{\input{files/#1}}
% TTT
\newcommand{\myTCPIP}{\textsc{tcp/ip}\xspace}
\newcommand{\myThreeD}{\textsc{3d}\xspace}
\newcommand{\mytool}[1]{\myemph{#1}}
\newcommand{\myTRPCAGE}{\textsc{trp-cage}\xspace}
\newcommand{\myTRPZIPPER}{\textsc{trp-zipper}\xspace}
% UUU
\newcommand{\myUbiquitin}{Ubiquitin\xspace}
\newcommand{\myUbuntu}{\textsc{Ubuntu}~v$10.04$\xspace}
\newcommand{\myUSB}{\textsc{usb}\xspace}
\newcommand{\myuser}[1]{$\mathcal{#1}$}
% VVV
\newcommand{\myvar}[2]{$\left(\mathcal{V}_{\mathrm{#1}#2}\right)$\xspace}
\newcommand{\myvard}[1]{\myvar{d}{#1}}
\newcommand{\myvari}[1]{\myvar{i}{#1}}
\newcommand{\myVGA}{\textsc{vga}\xspace}
\newcommand{\myVirtuose}{\textsc{Virtuose}\mytrademark~\textsc{6d}\mynum{35}--\mynum{45}\xspace}
% WWW
\newcommand{\myWindows}{\textsc{Windows}\xspace}

% Needed lengths
\newlength{\mywidth}
\newlength{\myheight}

% PSTricks style
\newpsstyle{nodestyle}{framearc=0.25,shadow=true,shadowcolor=myblue,blur=true}
\makeatother


%\setbeameroption{show notes on second screen=right}
%\setbeameroption{show notes}
%\includeonlyframes{current}

\title{Collaboration haptique étroitement couplée pour la manipulation moléculaire interactive}
\author[J. \myname{Simard}]{%
	{\normalsize Jean \myname{Simard}}\\[1ex]
	sous la direction de Philippe \myname{Tarroux}\\
	et l'encadrement scientifique de Mehdi \myname{Ammi}
}
\institute{\textsc{cnrs-limsi}}
\school{Université de \myname{Paris}-Sud}
\logo{%
	\myimage[height=1cm]{logo-ups}%
	\hspace{1em}%
	\myimage[height=1cm]{logo-limsi}%
}
\date{\mydate[datestyle=long]{12/03/2012}}

\bibliography{biblio.bib}

\begin{document}
	\transfade[duration=0.1]
	\begin{myframe}{}
		\titlepage
	\end{myframe}
	\logo{}
	\begin{myframe}{Sommaire}
		\tableofcontents[hideallsubsections]
	\end{myframe}
	\section{Introduction}
	\begin{myframe}{Sommaire}
		\tableofcontents[sectionstyle=show/shaded,subsectionstyle=show/show/hide,subsubsectionstyle=show/show/hide]
	\end{myframe}
	\subsection{Le \mydocking moléculaire}
	\begin{myframe}{Le \mydocking moléculaire}
		\begin{myfigure}
			\only<1>{%
				\myimage[width=0.9\paperwidth]{sota-HIV}
			}%
			\only<2>{%
				\myimage[width=0.9\paperwidth]{sota-HIV-inhibitor}
			}%
			\mycaption[fig-InhibiteurDeLaProteineDuVIH]{Protéase du \textsc{vih}\onslide<2>{ avec un inhibiteur}}
		\end{myfigure}
	\end{myframe}
	\subsection{Définition du \mydocking moléculaire}
	\begin{myframe}{Définition du \mydocking moléculaire}
		\begin{myplusblock}{Définition}%
			Consiste à prédire la configuration d'un complexe formés d'un ensemble de molécules
		\end{myplusblock}%
		\vspace{-1ex}
		\begin{columns}[T]%
			\begin{column}{0.6\textwidth}%
				\begin{myfigure}%
					\psset{xunit=0.045\textwidth,yunit=0.4cm}%
					\begin{myps}(-10,-5)(10,5)%
						\only<1-3>{%
							\rput(-5,3){\myimage[width=0.15\paperwidth]{sota-docking-deformed-A}}%
							\only<1-2>{%
								\rput(-5,-3){\myimage[width=0.15\paperwidth,angle=180]{sota-docking-deformed-B}}%
							}%
							\only<3>{%
								\rput(-5,-3){\myimage[width=0.15\paperwidth]{sota-docking-deformed-B}}%
							}%
						}%
						\only<2>{%
							\rput(5,2.5){\mymolAd}%
							\rput{180}(5,-2.5){\mymolBd}%
							\psline[linewidth=2pt,linecolor=mygreen,arrowsize=2pt 4,arrowlength=0.75]{<->}(3.5,3)(6.5,3)
							\psline[linewidth=2pt,linecolor=mygreen,arrowsize=2pt 4,arrowlength=0.75]{<->}(5,4)(5,2)
							\uput{3ex}[90](5,3){\textcolor{mygreen}{déplacement}}%
							\psarcn[linewidth=2pt,linecolor=mygreen,arrowsize=2pt 4,arrowlength=0.75]{->}(5,-2.5){0.3}{90}{135}%
							\uput{3ex}[-90](5,-2.5){\textcolor{mygreen}{orientation}}%
						}%
						\only<3>{%
							\rput(5,2.5){%
								\mymolAdt{myblue!20}{dashed}{0.1pt}%
								\mymolA%
							}%
							\rput(5,-2.5){%
								\mymolBdt{myred!20}{dashed}{0.1pt}%
								\mymolB%
							}%
							\rput(5,0){\rnode{deformation}{\vphantom{pÉ}\textcolor{mygreen}{déformation}}}%
						}%
						\only<4-5>{%
							\rput(-5,3){\myimage[width=0.15\paperwidth]{sota-docking-molecule-A}}%
							\rput(-5,-3){\myimage[width=0.15\paperwidth]{sota-docking-molecule-B}}%
							\rput(5,2.5){%
								\mymolA%
								\rput(-3,0.5){\textbf{\textcolor{white}{+}}}%
								\rput(-0.5,0.75){\textbf{\textcolor{white}{--}}}%
								\rput(2,1.5){\textbf{\textcolor{white}{+}}}%
								\only<4>{%
									\rput(0,2.5){\textcolor{mygreen}{facteurs physico-chimiques}}%
								}
								\only<5>{%
									\psline[linewidth=4pt,linecolor=mygreen]{->}(0,-0.25)(0,-2)%
								}%
							}%
							\rput(5,-2.5){%
								\mymolB%
								\rput(-3,-1){\textbf{\textcolor{white}{--}}}%
								\rput(-0.5,-0.5){\textbf{\textcolor{white}{+}}}%
								\rput(2,-0.25){\textbf{\textcolor{white}{--}}}%
								\only<4>{%
									\rput(0,-2.5){\textcolor{mygreen}{facteurs physico-chimiques}}%
								}
								\only<5>{%
									\psline[linewidth=4pt,linecolor=mygreen]{->}(0,0.25)(0,2)%
								}%
							}%
						}%
						\only<5>{%
							\rput(5,0){\vphantom{pÉ}\textcolor{mygreen}{amarrage}}%
						}%
						\only<6->{%
							\rput(-5,0){\myimage[width=0.175\paperwidth]{sota-docking-complex-AB}}%
							\rput(5,0){%
								\mymolA%
								\rput(-3,0.5){\textbf{\textcolor{white}{+}}}%
								\rput(-0.5,0.75){\textbf{\textcolor{white}{--}}}%
								\rput(2,1.5){\textbf{\textcolor{white}{+}}}%
							}%
							\rput(5,0){%
								\mymolB%
								\rput(-3,-1){\textbf{\textcolor{white}{--}}}%
								\rput(-0.5,-0.5){\textbf{\textcolor{white}{+}}}%
								\rput(2,-0.25){\textbf{\textcolor{white}{--}}}%
							}%
						}%
						\mycircleletter[fillcolor=myblue](-9.5,2.5){A}%
						\mycircleletter(-9.5,-2.5){B}%
						\only<6->{%
							\rput(-9.5,0){\Huge\bfseries +}%
						}
					\end{myps}%
					\mycaption[fig-DockingMoleculaire]{\myDocking moléculaire}%
				\end{myfigure}%
			\end{column}%
			\begin{column}{0.35\textwidth}%
				\begin{myblock}{Facteurs de complexité}%
					\begin{itemize}[<beamer:alert@+| beamer:myshade@1-.>]%
						\item Nombreux atomes%
						\item Déplacement et orientation%
						\item Flexibilité%
						\item Physico-chimie%
						\item Complémentarité%
							\begin{itemize}[<beamer:alert@.| beamer:myshade@1-.>]%
								\item géométrique%
								\item physico-chimique%
							\end{itemize}%
					\end{itemize}%
				\end{myblock}%
			\end{column}%
		\end{columns}%
		\vspace{-1ex}
		\onslide<7>{\centering\alert{$\Rightarrow$ Résolution du \mydocking couteux en temps de calcul}\par}
	\end{myframe}
	\subsection{Manipulation moléculaire \og centrée utilisateur \fg}
	\begin{myframe}{Manipulation moléculaire \og centrée utilisateur \fg}
		\deflength{\myfiguresize}{0.275\textwidth}
		\begin{columns}
			\begin{column}{0.475\textwidth}
				\begin{myfigure}
					\myimage[width=\myfiguresize]{sota-Weghorst-2003}
					\mycaption[fig-sota-InterfaceTangibleWeghorst2003]{Interface tangible \mycite{Weghorst-2003}}
				\end{myfigure}
			\end{column}
			\begin{column}{0.475\textwidth}
				\begin{myfigure}
					\myimage[width=\myfiguresize]{sota-Lai-Yuen-2006}
					\mycaption[fig-sota-InterfaceHaptiqueA5DDLLeDockingLaiYuen2006]{Interface haptique à \mynum{5}~degrés de liberté \mycite{Lai-Yuen-2006}}
				\end{myfigure}
			\end{column}
		\end{columns}
		\vspace{-3ex}
		\begin{columns}
			\begin{column}{0.475\textwidth}
				\begin{myfigure}
					\myimage[width=\myfiguresize]{sota-Ferey-2009}
					\mycaption[fig-sota-ManipulationMoleculaireMultimodaleFerey2009]{Guidage multimodal \mycite{Ferey-2009}}
				\end{myfigure}
			\end{column}
			\begin{column}{0.475\textwidth}
				\begin{myblock}{Synthèse des approches existantes}
					\begin{itemize}
						\item Essentiellement de la perception
							\begin{itemize}
								\item données
								\item contraintes
							\end{itemize}
						\item \mydocking simple et/ou rigide
					\end{itemize}
					\onslide<2>{\alert{$\Rightarrow$ Charge de travail restreinte}}
				\end{myblock}
			\end{column}
		\end{columns}
	\end{myframe}
	\subsection{Distribution de la charge de travail}
	\begin{myframe}{Distribution de la charge de travail}
		\begin{myplusblock}{Définition \mycite{Conein-2004}}
			Étendre la capacité cognitive d'analyse d'un individu pour inclure le matériel et l'environnement social comme composant d'un système cognitif plus étendu.
		\end{myplusblock}
		\begin{myfigure}%
			\psset{unit=0.06\textwidth}%
			\definecolor{mylightestredblue}{rgb}{0.875 0.705 0.79328125}
			\begin{myps}(-5,-2.5)(5,4)
				\footnotesize%
				\pspolygon*[linearc=0.5,linecolor=mylightestred](-0.5,4)(-4.5,4)(-4.5,0)(-0.5,-1.7)(1.15,-1.6)%
				\pspolygon*[linearc=0.5,linecolor=mylightestblue](0.5,4)(4.5,4)(4.5,0)(0.5,-1.7)(-1.15,-1.6)%
				\psclip{%
					\pscustom[linestyle=none]{%
						\pspolygon*[linearc=0.5](-0.5,4)(-4.5,4)(-4.5,0)(-0.5,-1.7)(1.15,-1.6)%
					}%
					\pscustom[linestyle=none]{%
						\pspolygon*[linearc=0.5](0.5,4)(4.5,4)(4.5,0)(0.5,-1.7)(-1.15,-1.6)%
					}%
				}%
				\psframe*[linecolor=mylightestredblue](-5,-2.5)(5,4)%
				\endpsclip%
				\psset{fillcolor=white}%
				\psframe[fillstyle=solid,linecolor=black!70,linestyle=dashed,framearc=0.25](-4.5,0)(-0.5,4)%
				\rput(-2.5,3.75){\textcolor{black!70}{Espace interne}}%
				\psframe[fillstyle=solid,linecolor=black!70,linestyle=dashed,framearc=0.25](4.5,0)(0.5,4)%
				\rput(2.5,3.75){\textcolor{black!70}{Espace externe}}%
				\psframe[linecolor=black!70,linestyle=dashed,framearc=0.25](-4.5,-2.5)(4.5,-0.5)%
				\mynode(-2,3.15)[individu1]{\textcolor{myblue}{Individu~\mynum{1}}}%
				\mynode(-3,2)[individu2]{\textcolor{myred}{Individu~\mynum{2}}}%
				\mynode(-2,0.85)[individu3]{\textcolor{mygreen}{Individu~\mynum{3}}}%
				\mynode(2.5,2.75)[artefact1]{Artefact~\mynum{1}}%
				\mynode(2.5,1.25)[artefact2]{Artefact~\mynum{2}}%
				\psellipse[fillstyle=solid,fillcolor=white](0,-1.25)(1,0.5)%
				\rput(0,-1.25){Tâche}%
				\rput(0,-2.25){\textcolor{black!70}{Abstraction de l'espace de travail}}%
				\psset{linewidth=0.5pt}
				\psset{arrows={<->}}%
				\ncline{individu1}{individu2}%
				\ncline{individu2}{individu3}%
				\ncline[offset=8pt]{individu1}{individu3}%
				\psset{arrows={-}}%
				\ncline{artefact1}{artefact2}%
				\psset{arrows={<->}}%
				\psset{linecolor=myblue}%
				\ncline{individu1}{artefact1}%
				\ncline{individu1}{artefact2}%
				\psset{linecolor=myred}%
				\ncline{individu2}{artefact1}%
				\ncline{individu2}{artefact2}%
				\psset{linecolor=mygreen}%
				\ncline{individu3}{artefact1}%
				\ncline{individu3}{artefact2}%
			\end{myps}
			\mycaption[fig-sota-SystemeCognitifDistribueZhang2006]{Système cognitif distribué \mycite{Zhang-2006}}
		\end{myfigure}
	\end{myframe}
	\subsection{Approches collaboratives pour la manipulation moléculaire}
	\begin{myframe}{Approches collaboratives pour la manipulation moléculaire}
		\deflength{\myfiguresize}{0.35\textwidth}
		\vspace{-3ex}
		\begin{columns}
			\begin{column}{0.475\textwidth}
				\begin{myfigure}
					\myimage[width=\myfiguresize]{sota-Kriz-2003}
					\mycaption[fig-sota-ExplorationMoleculaireSynchroneKriz2003]{Exploration moléculaire synchrone \mycite{Kriz-2003}}
				\end{myfigure}
			\end{column}
			\begin{column}{0.475\textwidth}
				\begin{myfigure}
					\myimage[width=\myfiguresize]{sota-Park-2006}
					\mycaption[fig-sota-ManipulationGuideeParDesExpertsPark2006]{Manipulation guidée par des experts \mycite{Park-2006}}
				\end{myfigure}
			\end{column}
		\end{columns}
		\vspace{-3ex}
		\begin{columns}
			\begin{column}{0.475\textwidth}
				\begin{myfigure}
					\myimage[width=\myfiguresize]{sota-Chastine-2007}
					\mycaption[fig-sota-DesignationEnEnvironnementVirtuelChastine2007]{Désignation en environnement virtuel \mycite{Chastine-2007}}
				\end{myfigure}
			\end{column}
			\begin{column}{0.475\textwidth}
				\begin{myblock}{Synthèse des approches existantes}
					\begin{itemize}
						\item Approches techno-centrées
						\item Approches coopératives
					\end{itemize}
					\onslide<2>{\alert{$\Rightarrow$ Apport du collaboratif ?}}
				\end{myblock}
			\end{column}
		\end{columns}
		\note[item]{Il faut que la discussion aborde le sujet de la distribution des charges de travail (en rapport avec la \myemph{frame} précédente)}
	\end{myframe}
	\subsection{Objectifs et démarche de la thèse}
	\begin{myframe}{Objectifs et démarche de la thèse}
		\begin{myblock}{Contexte de travail}
			Manipulation interactive de structures moléculaires pour le \mydocking
		\end{myblock}
		\vfill
		\begin{enumerate}
			\item Étudier et analyser la contribution des approches collaboratives
				\vfill
			\item Identifier et caractériser les limites et les contraintes
				\vfill
			\item Proposer de nouvelles solutions pour améliorer les approches collaboratives
				\vfill
			\item Évaluer les solutions proposées dans un scénario de \mydocking moléculaire
		\end{enumerate}
		\vfill
	\end{myframe}
	\begin{myframe}<1-7>[label={fra-sota-DemarchePourLEtudeDeLaManipulationMoleculaire}]{Démarche pour l'étude de la manipulation moléculaire}
		\begin{columns}[b]
			\begin{column}{0.45\textwidth}
				\vfill
				\begin{myfigure}
					\psset{xunit=1cm,yunit=1.1cm}%
					\begin{myps}(-2.5,-0.5)(2.5,5)%
						\only<1,3-7,9>{%
							\mynode(0,4)[Search]{Recherche}%
						}%
						\only<2,8>{%
							\mynode[fillstyle=solid,fillcolor=myred!25](0,4)[Search]{Recherche}%
							\mycirclenumber(0,4){1}%
						}%
						\only<1-2,4-7,9>{%
							\mynode(0,3)[Selection]{Sélection}%
						}%
						\only<3,8>{%
							\mynode[fillstyle=solid,fillcolor=myred!25](0,3)[Selection]{Sélection}%
							\mycirclenumber(0,3){2}%
						}%
						\only<1-3,5-8>{%
							\mynode(0,2)[Manipulation]{Manipulation}%
						}%
						\only<4,9>{%
							\mynode[fillstyle=solid,fillcolor=myred!25](0,2)[Manipulation]{Manipulation}%
							\mycirclenumber(0,2){3}%
						}%
						\only<1-4,6-8>{%
							\mynode(0,1)[Evaluation]{Évaluation}%
						}%
						\only<5,9>{%
							\mynode[fillstyle=solid,fillcolor=myred!25](0,1)[Evaluation]{Évaluation}%
							\mycirclenumber(0,1){4}%
						}%
						\ncline{->}{Search}{Selection}%
						\ncline{->}{Selection}{Manipulation}%
						\ncline{->}{Manipulation}{Evaluation}%
						\only<6->{%
							\ncloop[loopsize=4em,angleA=-90,outAngleA=-90,angleB=90,outAngleB=90,linearc=0.05,armA=0.2]{->}{Evaluation}{Search}%
						}%
						\only<6>{%
							\mycirclenumber[180](0,2.5){5}%
						}%
						\only<7>{%
							\mynode[fillstyle=solid,fillcolor=myred!25](0,0)[Objective]{Objectif atteint}%
							\mycirclenumber(0,0){6}%
						}%
						\only<8->{%
							\mynode(0,0)[Objective]{Objectif atteint}%
						}%
						\only<7->{%
							\ncline{->}{Evaluation}{Objective}%
						}%
					\end{myps}
					\mycaption[fig-PrimitivesComportementales]{Primitives Comportementales}%
				\end{myfigure}
			\end{column}
				\only<1-7>{%
			\begin{column}{0.5\textwidth}%
					\begin{myblock}{Description}%
						Selon les Primitives Comportementales Virtuelles \mycite{Bowman-1999,Fuchs-2006a}%
					\end{myblock}%
					\vfill
					\begin{myfigure}%
						\psset{unit=0.55cm}%
						\begin{myps}(-4,-3.5)(4,3.5)%
							\psset{lightsrc=50 50 10,viewpoint=100 50 20 rtp2xyz,Decran=30}%
							\psset{Alpha=45,Beta=30}%
							\psset{grid=false}%
							\only<1>{%
								\myPCVneutral%
							}
							\only<3>{%
								\myPCVselection%
								\rput[bl](2.5,1.25){%
									\myimage[width=40pt]{red-cursor}%
								}%
							}%
							\only<4>{%
								\rput[bl](3,-0.75){%
									\myimage[width=40pt]{red-cursor}%
								}%
								\myPCVmanipulation%
							}%
							\only<5>{%
								\myPCVevaluation%
							}%
							\only<6>{%
								\myPCViteration%
							}%
							\only<7>{%
								\myPCVfinal%
							}%
							\only<2,5-7>{%
								\rput[bl](3,2.5){%
									\myimage[width=40pt]{red-cursor}%
								}%
							}%
							\only<2>{%
								\myPCVsearch%
								\uput{3pt}[90](3,2.5){\Large\textbf{\textsf{?}}}%
								\uput{3pt}[170](3,2.5){\Large\textbf{\textsf{?}}}%
								\uput{3pt}[260](3,2.5){\Large\textbf{\textsf{?}}}%
								\uput{3pt}[330](3,2.5){\Large\textbf{\textsf{?}}}%
							}%
							\only<1-7>{%
								\psframe*[linecolor=myred](-4,-3.5)(4,-3)%
								\small%
								\only<1-3>{%
									\psframe*[linecolor=mygreen](-4,-3.5)(-2.4,-3)%
									\rput(0,-3.25){\textcolor{white}{\textbf{\mynum[\%]{20}}}}%
								}%
								\only<4-6>{%
									\psframe*[linecolor=mygreen](-4,-3.5)(-0.8,-3)%
									\rput(0,-3.25){\textcolor{white}{\textbf{\mynum[\%]{40}}}}%
								}%
								\only<5>{%
									\psframe[linewidth=2pt,linecolor=magenta,dimen=inner](-4,-3.5)(4,-3)%
								}%
								\only<7>{%
									\psframe*[linecolor=mygreen](-4,-3.5)(4,-3)%
									\rput(0,-3.25){\textcolor{white}{\textbf{\mynum[\%]{100}}}}%
								}%
							}%
						\end{myps}%
						\only<1-7>{%
							\mycaption[fig-ManipulationMoleculaire]{Manipulation moléculaire}
						}%
					\end{myfigure}%
			\end{column}%
				}%
		\end{columns}
	\end{myframe}
	\section{Plateforme de manipulation moléculaire \myShaddock}
	\begin{myframe}{Sommaire}
		\tableofcontents[sectionstyle=show/shaded,subsectionstyle=show/show/hide,subsubsectionstyle=show/show/hide]
	\end{myframe}
	\subsection{Cahier des charges}
	\begin{myframe}{Cahier des charges}
		\note{Il n'existe pas de plateforme appropriée qui répondent à nos besoins.}
		\note[item]{Pas de manipulation colocalisée}
		\note[item]{Manipulation de molécules simples}
		\begin{myblock}{Objectif}
			Élaborer une plateforme permettant la collaboration étroitement couplée pour la manipulation moléculaire
		\end{myblock}
		\begin{columns}[T]
			\begin{column}{0.475\textwidth}
				\begin{myminusblock}{Contraintes à respecter}
					\begin{itemize}
						\item Collaboration interactive synchrone avec des molécules
						\item Simulation de la dynamique moléculaire
						\item Manipulation à l'aide de plusieurs interfaces haptiques
						\item Différents outils pour la manipulation moléculaire
					\end{itemize}
				\end{myminusblock}
			\end{column}
			\begin{column}{0.475\textwidth}
				\begin{myplusblock}{Solutions proposées}
					\begin{itemize}
						\item Modularité logicielle
						\item Modularité matérielle
						\item Plateforme basée sur des logiciels de biologie
						\item Utilisation de modules dédiés à la réalité virtuelle
						\item Développement de nouveaux outils d'interaction
					\end{itemize}
				\end{myplusblock}
			\end{column}
		\end{columns}
	\end{myframe}
	\subsection{Organisation matérielle}
	\begin{myframe}{Organisation matérielle}
		\begin{columns}
			\begin{column}{0.35\textwidth}
				\begin{myplusblock}{Fonctionnalités}
					\begin{itemize}
						\item Colocalisée, synchrone
						\item Vue partagée
						\item Communication orale et gestuelle
						\item Différents outils
							\begin{itemize}
								\item déplacement
								\item orientation
								\item déformation
							\end{itemize}
						\item Multiples interfaces
					\end{itemize}
				\end{myplusblock}
			\end{column}
			\begin{column}{0.6\textwidth}
				\begin{myfigure}
					\psset{xunit=0.666666666667\paperwidth,yunit=0.4\paperwidth}
					\begin{myps}(0,0)(1,1)%
						\only<1-3>{%
							\rput(0.5,0.5){\myimage[width=0.666666666667\paperwidth,height=0.4\paperwidth]{exp1-schema}}%
							\rput(0.2,0.8){\myimage[width=0.266666666667\paperwidth,height=0.16\paperwidth]{exp1-photo}}%
							\pnode(0.52,0.66){tug1}%
							\pnode(0.62,0.76){tug1-from}%
							\pnode(0.62,0.60){tug2}%
							\pnode(0.72,0.70){tug2-from}%
							\pnode(0.57,0.63){grab}%
							\pnode(0.67,0.73){grab-from}%
						}%
						%\only<2>{%
							%\pspolygon[linewidth=4pt,linecolor=myred](0.60,0.68)(0.61,0.93)(0.86,0.88)(0.83,0.57)%
						%}%
						\only<2>{%
							\ncline[linewidth=2.5pt,linecolor=myred,nodesepB=2.5pt]{c->}{tug1-from}{tug1}%
							\ncline[linewidth=2.5pt,linecolor=mygreen,nodesepB=2.5pt]{c->}{tug2-from}{tug2}%
							\ncline[linewidth=2.5pt,linecolor=myblue,nodesepB=2.5pt]{c->}{grab-from}{grab}%
						}%
						\only<3>{%
							\rput(0.5,0.5){\myimage[width=0.666666666667\paperwidth,height=0.4\paperwidth]{exp2-schema}}%
							\rput(0.2,0.8){\myimage[width=0.266666666667\paperwidth,height=0.16\paperwidth]{exp2-photo}}%
							\pnode(0.59,0.6){grab}%
							\pnode(0.69,0.7){grab-from}%
							\ncline[linewidth=2.5pt,linecolor=blue,nodesepB=2.5pt]{c->}{grab-from}{grab}%
						}%
						\only<4>{%
							\rput(0.5,0.5){\myimage[width=0.666666666667\paperwidth,height=0.4\paperwidth]{exp3-schema}}%
							\rput(0.2,0.8){\myimage[width=0.266666666667\paperwidth,height=0.16\paperwidth]{exp3-photo}}%
							\pnode(0.52,0.675){tug1}%
							\pnode(0.62,0.775){tug1-from}%
							\pnode(0.55,0.65){tug2}%
							\pnode(0.65,0.75){tug2-from}%
							\pnode(0.58,0.63){tug3}%
							\pnode(0.68,0.73){tug3-from}%
							\pnode(0.62,0.61){tug4}%
							\pnode(0.72,0.71){tug4-from}%
							\ncline[linewidth=2.5pt,linecolor=myred,nodesepB=2.5pt]{c->}{tug1-from}{tug1}%
							\ncline[linewidth=2.5pt,linecolor=myblue,nodesepB=2.5pt]{c->}{tug2-from}{tug2}%
							\ncline[linewidth=2.5pt,linecolor=magenta,nodesepB=2.5pt]{c->}{tug3-from}{tug3}%
							\ncline[linewidth=2.5pt,linecolor=mygreen,nodesepB=2.5pt]{c->}{tug4-from}{tug4}%
						}%
						\rput(0.5,0.5){%
							\psscalebox{0.75 1.25}{%
								\psset{viewpoint=-10 -15 4,Decran=16}%
								\psset{solidmemory}%
								\psSolid[object=plan,definition=normalpoint,args={0 0 -0.33 [0 0 1]},base=-6 5 -3 3,name=myplane,action=none]%
								\psSolid[object=plan,definition=normalpoint,args={5 0 -0.33 [-1 0 0 2.5]},base=-3 0.75 0.25 1.9,name=myscreenplane,action=none]%
								\psProjection[object=texte,pos=dc,plan=myplane,fontsize=16,text={Expérimentateur},phi=-90,visibility=false](-5.75,0)%
								\psProjection[object=texte,pos=uc,plan=myscreenplane,fontsize=16,text={Vue partagée},visibility=false](-1,1.8)%
								\codejps{%
									/oldarrowpointe {xunit 4 div} def%
									/oldarrowplume {xunit 8 div} def %
								}%
								\psProjection[object=texte,pos=cr,plan=myplane,fontsize=14,text={Sujets},phi=-90,visibility=false](0.25,-3.35)%
								\psProjection[object=vecteur,plan=myplane,arrowlength=2.5,args=0 0.5,visibility=false](0.25,-3.35)%
								\psProjection[object=texte,pos=cr,plan=myplane,fontsize=14,text={Outils},phi=-90,visibility=false](1.85,-3.4)%
								\psProjection[object=vecteur,plan=myplane,arrowlength=2.5,args=0 0.5,visibility=false](1.85,-3.4)%
								\psProjection[object=texte,pos=dc,plan=myplane,linecolor=myyellowcolor,fontsize=12,text={VMD},visibility=false](-4.33,-2.75)%
								\psProjection[object=texte,pos=dc,plan=myplane,linecolor=myyellowcolor,fontsize=12,text={NAMD},visibility=false](-4.33,-3.25)%
								\only<1-3>{%
									\psProjection[object=texte,pos=dc,plan=myplane,linecolor=myred,fontsize=12,text={VRPN},visibility=false](-4.33,-3.75)%
									\psProjection[object=texte,pos=dc,plan=myplane,linecolor=myblue,fontsize=12,text={VRPN},visibility=false](-4.33,-4.25)%
									\psProjection[object=texte,pos=dc,plan=myplane,linecolor=mygreen,fontsize=12,text={VRPN},visibility=false](-2.33,-2.8)%
								}%
								\only<4>{%
									\psProjection[object=texte,pos=dc,plan=myplane,linecolor=myred,fontsize=12,text={VRPN},visibility=false](-4.33,-3.75)%
									\psProjection[object=texte,pos=dc,plan=myplane,linecolor=myblue,fontsize=12,text={VRPN},visibility=false](-2.33,-2.8)%
									\psProjection[object=texte,pos=dc,plan=myplane,linecolor=magenta,fontsize=12,text={VRPN},visibility=false](-0.66,-2.85)%
									\psProjection[object=texte,pos=dc,plan=myplane,linecolor=mygreen,fontsize=12,text={VRPN},visibility=false](1,-2.95)%
								}%
								\composeSolid%
							}%
						}%
					\end{myps}
					\mycaption[fig-PlateFormeExperimentale1]{Plate-forme expérimentale}
				\end{myfigure}
			\end{column}
		\end{columns}
	\end{myframe}
	\subsection{Outils d'interaction proposés}
	\begin{myframe}{Outils supplémentaires proposés}
		\begin{myblock}{Objectif}
			Faciliter le processus de sélection d'une structure moléculaire dans \myVMD
		\end{myblock}
		\begin{columns}
			\begin{column}{0.35\textwidth}
				\begin{myminusblock}{Problème}
					Sélection difficile
					\begin{itemize}
						\item Atomes nombreux
						\item Cibles en mouvement
					\end{itemize}
				\end{myminusblock}
				\begin{myplusblock}{Fonctionnalités}
					\begin{itemize}
						\item<beamer:alert@2-6| beamer:myshade@1> Champs de potentiel \mycite{Simard-2009}
						\item<beamer:alert@7| beamer:myshade@1-6> Pointage visuel
						\item<beamer:alert@8| beamer:myshade@1-7> Différents niveaux de sélection
					\end{itemize}
				\end{myplusblock}
			\end{column}
			\begin{column}{0.6\textwidth}
				\begin{myfigure}
					\psset{xunit=0.6\textwidth,yunit=0.365838509\textwidth}
					\begin{myps}(0,0)(1.25,1)%
						\rput(0.5,0.5){%
							\only<1-6>{%
								\myimage[width=0.6\textwidth]{select-molecule}%
							}%
							\only<7>{%
								\myimage[width=0.6\textwidth]{select-atom-target}%
							}%
							\only<8>{%
								\myimage[width=0.6\textwidth]{select-residue-target}%
							}%
						}%
						\only<2-6>{%
							\psarc{c-c}(0.88,0.86){0.13}{-45}{45}%
							\psarc{c-c}(0.88,0.86){0.20}{-45}{45}%
							\psarc{c-c}(0.88,0.86){0.30}{-45}{45}%
							\psarc{c-c}(0.88,0.86){0.45}{-45}{45}%
							\uput{0pt}[15](1.1,0.95){%
								\myimage[width=40pt]{red-cursor}%
							}%
							\psline[linewidth=2pt,linecolor=mygreen]{c->}(1.075,0.86)(0.95,0.86)
						}%
						\only<3-5>{%
							\uput{0pt}[90](0.3,0){%
								\psframebox*[fillcolor=black!5,framearc=0.25,boxsep=true,framesep=5pt]{%
									\begin{psmatrix}[colsep=0pt,rowsep=2ex]
										\textbf{Champs de potentiel}\\%
										{\small$\displaystyle U(\vec{x})=\phi\cdot\sigma\exp\left[\frac{\sigma^2-\vec{x}^2}{2\sigma^2}\right]$}\\%
										\psset{unit=0.025\textwidth,Alpha=60}
										\begin{myps}(-6,-3)(4.5,2.5)
											\psset{drawStyle=yLines,xPlotpoints=25,yPlotpoints=100,linewidth=0.1pt}
											\only<3>{%
												\psplotThreeD[linecolor=myblue](-3,3)(-6,3){\CHAMPP(0,0){1}{2}}
											}%
											\only<4>{%
												\psplotThreeD[linecolor=mygreen](-3,3)(-6,3){\CHAMPP(0,-3){1.125}{2.5}}
											}%
											\only<5>{%
												\psplotThreeD[linecolor=myred](-3,3)(-6,3){\CHAMPP(0,0){1}{2} \CHAMPP(0,-3){1.125}{2.5} min}
											}%
											\psset{viewpoint=100 50 50,Decran=100,lightsrc=4 4 50}
											\psset{solidmemory}
											\psSolid[%
												object=cylindre,%
												h=4,%
												r=0.1,%
												RotX=90,%
												ngrid=\rescylinder,%
												fillcolor=black,%
												action=none,%
												name=B1%
												](0,0,0)
											\psSolid[%
												object=sphere,%
												r=1,%
												ngrid=\ressphere,%
												RotZ=-90,%
												fillcolor=mygreen!33,%
												action=none,%
												name=A1%
											](0,-4,0)
											\psSolid[%
												object=sphere,%
												r=0.75,%
												ngrid=\ressphere,%
												RotZ=-90,%
												fillcolor=myblue!33,%
												action=none,%
												name=A2%
											](0,0,0)
											\psSolid[%
												object=fusion,%
												base=B1 A1 A2,%
												grid,%
												action=draw**%
												](0,0,0)
											\composeSolid
										\end{myps}
									\end{psmatrix}
								}%
							}%
						}%
						\only<6>{%
							\uput{0pt}[90](0.3,0){%
								\psframebox*[fillcolor=black!5,framearc=0.25,boxsep=true,framesep=5pt]{%
									\begin{psmatrix}[colsep=0pt,rowsep=2ex]
										\textbf{Champs de force}\\%
										{\small$\displaystyle F(d)=\phi\frac{d}{\sigma}\exp\left[\frac{\sigma^2-d^2}{2\sigma^2}\right]$}\\%
										\begin{mygraph}[labels=none,ticks=none,xAxisLabel={distance},xAxisLabelPos={c,-1ex},llx={-1ex},yAxisLabel={force},yAxisLabelPos={-1ex,c},lly={-1ex}]{->}(4.5,2.5){3cm}{1cm}%
											\psline[linecolor=black!50,linewidth=0.5pt,linestyle=dashed,arrows=-](1,0)(1,2)(0,2)%
											\psplot[linecolor=red,linewidth=0.5pt,plotpoints=500,arrows=c-c]{0}{4}{2 \E 1 2 div exp x \E 0 x 2 exp 2 div sub exp mul mul mul}%
											\psdot[linecolor=black!50](1,2)%
											\uput{2pt}[45](1,2){\small\color{black!50}$\left(\sigma,\phi\right)$}%
										\end{mygraph}%
									\end{psmatrix}
								}%
							}%
						}%
						\only<7->{%
							\uput{0pt}[15](0.91,0.95){%
								\myimage[width=40pt]{red-cursor}%
							}%
						}%
						\only<7>{%
							\pnode(0.88,0.95){atom}%
							\rput(0.88,1.25){\rnode{atom-label}{\textcolor{mygreen}{Atome\vphantom{p}}}}%
							\ncline[linecolor=mygreen]{->}{atom-label}{atom}%
						}%
						\only<8>{%
							\pnode(0.88,0.95){residue}%
							\rput(0.88,1.25){\rnode{residue-label}{\textcolor{mygreen}{Résidu\vphantom{p}}}}%
							\ncline[linecolor=mygreen]{->}{residue-label}{residue}%
						}%
					\end{myps}
					\mycaption[fig-OutilDeSelectionAmeliore]{Outil de sélection amélioré}
				\end{myfigure}
			\end{column}
		\end{columns}
	\end{myframe}
	\section{Caractérisation des approches collaboratives en environnement moléculaire}
	\subsection{Étude~\mynum{1} -- Recherche et sélection collaborative de motifs}
	\begin{myframe}{Sommaire}
		\tableofcontents[sectionstyle=show/shaded,subsectionstyle=show/shaded/hide,subsubsectionstyle=show/show/hide]
	\end{myframe}
	\againframe<8>{fra-sota-DemarchePourLEtudeDeLaManipulationMoleculaire}
	\subsubsection{Objectifs}
	\begin{myframe}{Objectifs}
		\begin{myblock}{Objectif principal}
			Étudier la contribution de la collaboration pendant une tâche de recherche et de sélection de structures moléculaires
		\end{myblock}
		\begin{myplusblock}{Hypothèses}
			\begin{enumerate}
				\item Amélioration des performances (individuelles \myvs collaboratives)
				\item Des stratégies de travail émergent dans les binômes
			\end{enumerate}
		\end{myplusblock}
		\begin{myblock}{Variables}
			\begin{description}
				\item[Nombre de sujets] monôme (\mynum{24}~sujets) ou binôme (\mynum{12}~couples)
				\item[Complexité de la tâche] Forme, nature, position, similarité\dots{}
			\end{description}
		\end{myblock}
	\end{myframe}
	\subsubsection{Tâche proposée}
	\begin{myframe}{Tâche proposée}
		\begin{myblock}{Objectif}
			Recherche d'un motif dans une molécule
		\end{myblock}
		\renewcommand{\schemafactor}{0.0925}
		\setlength{\schemaunit}{\schemafactor\paperwidth}
		\psset{unit=\schemaunit}
		\begin{myfigure}
			\begin{myps}(-4,-2.3)(4,2.3)
				\rput(-1.75,0){%
					\myimage[height=2\schemaunit]{trp-cage}}
				\rput(1.25,0){%
					\myimage[height=2\schemaunit]{prion}}
				\rput(-3.5,0){%
					\myimage[height=\schemaunit]{pattern1}}
				\rput(-1,1.5){%
					\myimage[width=\schemaunit]{pattern3-8}}
				\rput(1,1.5){%
					\myimage[width=\schemaunit]{pattern2-7}}
				\rput(-1,-1.5){%
					\myimage[width=\schemaunit]{pattern4-9}}
				\rput(1,-1.5){%
					\myimage[width=\schemaunit]{pattern5-10}}
				\rput(3.5,0){%
					\myimage[height=\schemaunit]{pattern6}}

				\psset{framesize=1 1}
				\fnode(-3.5,0){P1}
				\uput[90](-3.5,0.5){Motif~1}
				\fnode(-1,1.5){P38}
				\uput[90](-1,2){Motifs~3 et 8}
				\fnode(1,1.5){P27}
				\uput[90](1,2){Motifs~2 et 7}
				\fnode(-1,-1.5){P49}
				\uput[-90](-1,-2){Motifs~4 et 9}
				\fnode(1,-1.5){P510}
				\uput[-90](1,-2){Motifs~5 et 10}
				\fnode(3.5,0){P6}
				\uput[90](3.5,0.5){Motif~6}

				\psset{linecolor=myred}
				\cnode(-1.5,0.3){0.2}{TRPP1}
				\cnode(-2,0.15){0.2}{TRPP38}
				\cnode(-1.25,-0.1){0.2}{TRPP27}
				\cnode(-2.2,-0.5){0.2}{TRPP49}
				\cnode(-1.25,-0.65){0.2}{TRPP510}
				\ncline{-}{P1}{TRPP1}
				\ncline{-}{P38}{TRPP38}
				\ncline{-}{P27}{TRPP27}
				\ncline{-}{P49}{TRPP49}
				\ncline{-}{P510}{TRPP510}

				\psset{linecolor=myblue}
				\cnode(-0.2,0.4){0.2}{PrionP38}
				\cnode(2.8,0.6){0.2}{PrionP27}
				\cnode(0.8,0.2){0.2}{PrionP49}
				\cnode(1.7,-0.7){0.2}{PrionP510}
				\cnode(1.4,0.0){0.2}{PrionP6}
				\ncline{-}{P38}{PrionP38}
				\ncline{-}{P27}{PrionP27}
				\ncline{-}{P49}{PrionP49}
				\ncline{-}{P510}{PrionP510}
				\ncline{-}{P6}{PrionP6}
			\end{myps}
			\mycaption[fig-RepartitionDesMotifsSurLesMoleculesTRPCageEtPrion]{Répartition des motifs sur les molécules (\textsc{trp-cage} et Prion)}
		\end{myfigure}
	\end{myframe}
	\subsubsection{Résultats}
	\begin{myframe}{Amélioration des performances en collaboration}
		\begin{myfigure}
			\only<1-3>{%
				\mylegend{%
					\myleg{monôme}{myblue}%
					\myleg{binôme}{myblue!70}%
				}
			}%
			\only<4->{%
				\mylegend{%
					\myleg{recherche}{myblue}%
					\myleg{sélection}{myblue!70}%
				}
			}%
			\begin{myboxgraph}[llx=-3em,yAxisLabelPos={-3em,c}](10,0.75\textwidth)[100]{motifs}(500,3cm){temps~(s)}
				\only<1-3>{%
					\myboxplot{exp1-time-residue-group.csv}%
				}%
				\only<4->{%
					\myboxplot{exp1-timeaudio-residue-searchselection.csv}%
				}%
				\only<2>{%
					\psframe[linecolor=myred,linewidth=2pt,framearc=0.25,dimen=inner](0,0)(5,200)%
					\psframe[linecolor=myred,linewidth=2pt,framearc=0.25,dimen=inner](6,0)(8,200)%
				}%
				\only<3-4,6>{%
					\psframe[linecolor=mygreen,linewidth=2pt,framearc=0.25,dimen=inner](5,0)(6,300)%
				}%
				\only<3-5>{%
					\psframe[linecolor=mygreen,linewidth=2pt,framearc=0.25,dimen=inner](8,0)(10,450)%
				}%
				\only<5>{%
					\uput[90](9,450){\textcolor{mygreen}{Recherche}}%
				}%
				\only<6>{%
					\uput[90](5.5,300){\textcolor{mygreen}{Sélection}}%
					\note[item]<5>{C'est cette constatation qui nous a amené à améliorer les outils de sélection.}
				}%
			\end{myboxgraph}
			\mycaption[fig-TempsDeRealisationDeLaTache]{Temps de réalisation de la tâche}
		\end{myfigure}
		\begin{myblock}{Résultat}
			\begin{itemize}
				\item<beamer:alert@2| beamer:myshade@1> Pas d'évolution sur les tâches simples
				\item<beamer:alert@3| beamer:myshade@1-2> Une amélioration significative de la collaboration sur les tâches complexes
				\item<beamer:alert@5-| beamer:myshade@1-4> Répartition différente du temps selon les tâches
			\end{itemize}
		\end{myblock}
	\end{myframe}
	\begin{myframe}{Stratégies de travail}
		\begin{myfigure}
			\def\mymaxy{24}%
			\only<1-5>{%
				\def\mymaxy{24}%
				\def\mydy{4}%
				\def\myaxislabely{\color{myblue}{distance~(mm)}}%
			}%
			\only<6>{%
				\def\mymaxy{350}%
				\def\mydy{100}%
				\def\myaxislabely{\color{myblue}{temps~(s)}}%
			}%
			\begin{myboxgraph}[llx=-3em,lly=-5ex,xAxisLabelPos={c,-5ex},yAxisLabelPos={-3em,c},labels=none](12,0.75\textwidth)[\mydy]{binômes}(\mymaxy,2cm){\myaxislabely}
				\only<2-5>{%
					\psframe*[linecolor=myred!25,dimen=outer](0,0)(12,8)%
					\psframe*[linecolor=mygreen!25](0,8)(12,14)%
					\psframe*[linecolor=myblue!25](0,14)(12,20)%
					\psline[linewidth=0.5pt,linestyle=dashed,arrows=-](0,4)(12,4)%
					\psline[linewidth=0.5pt,linestyle=dashed,arrows=-](0,8)(12,8)%
					\psline[linewidth=0.5pt,linestyle=dashed,arrows=-](0,12)(12,12)%
					\psline[linewidth=0.5pt,linestyle=dashed,arrows=-](0,16)(12,16)%
					\psline[linewidth=0.5pt,linestyle=dashed,arrows=-](0,20)(12,20)%
				}%
				% Once header are readed, they are defined for other barplot
				% That's why barplots without headers are in first position
				\only<1-5>{%
					\mybarplot[header=false,barstyle=third-barstyle]{exp1-diff-groups3.csv}%
					\mybarplot[header=false,barstyle=second-barstyle]{exp1-diff-groups2.csv}%
					\mybarplot[header=true,barstyle=first-barstyle]{exp1-diff-groups1.csv}%
				}%
				\only<1-5>{%
					\psaxes[linecolor=myred,xAxis=false,Dy=4,labels=none,yAxisLabelPos={2em,c},yticksize=0pt 3pt]{->}(12,0)(0,24)%
					\uput[180](0,0){\textcolor{myblue}{0}}%
					\uput[180](0,4){\textcolor{myblue}{4}}%
					\uput[180](0,8){\textcolor{myblue}{8}}%
					\uput[180](0,12){\textcolor{myblue}{12}}%
					\uput[180](0,16){\textcolor{myblue}{16}}%
					\uput[180](0,20){\textcolor{myblue}{20}}%
					\uput[0](12,4){\textcolor{myred}{1}}%
					\uput[0](12,8){\textcolor{myred}{2}}%
					\uput[0](12,12){\textcolor{myred}{3}}%
					\uput[0](12,16){\textcolor{myred}{4}}%
					\uput[0](12,20){\textcolor{myred}{5}}%
					\uput{1.5em}[0](12,10){\rotateright{\textcolor{myred}{affinité~(1--5)}}}%
					\pstScalePoints(1,4){}{}%
					\readdata{\affinitygroupsdata}{files/exp1-affinity-groups.csv}%
					\listplot[arrows={-},linecolor=myred,linewidth=2pt,showpoints=true,shadow=false]{\affinitygroupsdata}%
				}%
				\only<2-5>{%
					\psset{linewidth=0.1pt,linecolor=white,fillstyle=solid,fillcolor=myred}%
					\uput[180](12,5){\pscharpath{\Large\bf\sffamily Champ proche}}%
					\psset{fillcolor=mygreen}%
					\uput[180](12,11){\pscharpath{\Large\bf\sffamily Champ voisin}}%
					\psset{fillcolor=myblue}%
					\uput[180](12,17){\pscharpath{\Large\bf\sffamily Champ distant}}%
				}%
				\only<2-5>{%
					\def\mybottomborder{-7.5}%
				}%
				\only<6>{%
					\def\mybottomborder{-109.375}%
				}%
				\only<2->{%
					\psframe[dimen=outer,fillstyle=none,linewidth=2pt,framearc=0.25,linecolor=myblue](0,\mybottomborder)(1,0)%
					\psframe[dimen=outer,fillstyle=none,linewidth=2pt,framearc=0.25,linecolor=mygreen](1,\mybottomborder)(4,0)%
					\psframe[dimen=outer,fillstyle=none,linewidth=2pt,framearc=0.25,linecolor=myred](4,\mybottomborder)(12,0)%
				}%
				\only<6>{%
					\uput[180](0,0){\textcolor{myblue}{0}}%
					\uput[180](0,100){\textcolor{myblue}{100}}%
					\uput[180](0,200){\textcolor{myblue}{200}}%
					\uput[180](0,300){\textcolor{myblue}{300}}%
					\mybarplot[header=false,barstyle=third-barstyle]{exp1-time-groups3.csv}%
					\mybarplot[header=false,barstyle=second-barstyle]{exp1-time-groups2.csv}%
					\mybarplot[header=true,barstyle=first-barstyle]{exp1-time-groups1.csv}%
				}%
			\end{myboxgraph}
			\only<1-5>{%
				\mycaption[fig-DistanceMoyenneEntreLesCurseursDesSujets]{Distance moyenne entre les curseurs des sujets}
			}
			\only<6>{%
				\mycaption[fig-TempsDeRealisationDesBinomes]{Temps de réalisation de la tâche des binômes}
			}
		\end{myfigure}
		\begin{myblock}{Résultat}
			{\only<1>{\color{black!25}}Trois stratégies liées à l'affinité entre les collaborateurs}
			\begin{description}
				\item<beamer:alert@3| beamer:myshade@1-2| beamer:mydescblue@4->[Champs distants] Peu de collaboration avec peu de conflits de coordination
				\item<beamer:alert@5| beamer:myshade@1-4| beamer:mydescgreen@6->[Champs voisins] Bonne collaboration avec conflits de coordination
				\item<beamer:alert@4| beamer:myshade@1-3| beamer:mydescred@5->[Champs proches] Forte collaboration mais conflits de coordination importants
			\end{description}
		\end{myblock}
	\end{myframe}
	\subsubsection{Synthèse}
	\begin{myframe}{Synthèse}
		\begin{columns}[t]
			\begin{column}{0.475\textwidth}
				\centering
				\myunode[90][shadowcolor=myred](0,0)[complexe-problem]{Complexité de la tâche}[4cm]
				\begin{myplusblock}{Résultats \mycite{Simard-2010b}}
					\begin{itemize}
						\item Amélioration des performances sur les tâches complexes
					\end{itemize}
				\end{myplusblock}
				\vfill%
				\begin{myminusblock}{Limites}
					\begin{itemize}
						\item Quel est le lien entre la complexité et la collaboration ?
							\note[item]{L'analyse de l'influence de la complexité de la tâche sur les performances sera l'un des objets de la prochaine étude.}
						\item Comment définir une tâche \myemph{complexe} ?
					\end{itemize}
				\end{myminusblock}
			\end{column}
			\begin{column}{0.475\textwidth}
				\centering
				\myunode[90][shadowcolor=myred](0,0)[strategy-problem]{Stratégie de travail}[4cm]
				\begin{myplusblock}{Résultats \mycite{Simard-2010c}}
					\begin{itemize}
						\item Trois stratégies différentes
						\item Meilleurs résultats avec une stratégie en champs voisins
					\end{itemize}
				\end{myplusblock}
				\vfill
				\begin{myminusblock}{Limites}
					\begin{itemize}
						\item Modification du comportement naturel des groupes
						\item Conflits de coordination en champs voisins
							\note[item]{Dans la prochaine étude, nous allons également tenter de mettre en évidence les conflits de coordination.}
					\end{itemize}
				\end{myminusblock}
			\end{column}
		\end{columns}
	\end{myframe}
	\subsection{Étude~\mynum{2} -- Déformation collaborative de molécules}
	\begin{myframe}{Sommaire}
		\tableofcontents[sectionstyle=show/shaded,subsectionstyle=show/shaded/hide,subsubsectionstyle=show/show/hide]
	\end{myframe}
	\againframe<9>{fra-sota-DemarchePourLEtudeDeLaManipulationMoleculaire}
	\subsubsection{Objectifs}
	\begin{myframe}{Objectifs}
		\begin{myblock}{Objectif principal}
			Caractériser les scénarios qui nécessitent de la coordination
		\end{myblock}
		\begin{myplusblock}{Hypothèses}
			\begin{enumerate}
				\item Le besoin en coordination influence la complexité et les performances
				\item Amélioration de la répartition des ressources (bimanuelle \myvs collaborative)
			\end{enumerate}
		\end{myplusblock}
		\begin{myblock}{Variables}
			\begin{description}
				\item[Nombre de sujets] monôme (\mynum{12}~sujets) ou binôme (\mynum{12}~couples)
				\item[Complexité de la molécule] \mynum{2}~molécules
				\item[Outil de déformation] 2~configurations de déformation
					\note[item]{L'objectif des différents outil de déformation est d'étudier différents niveaux de coordination pour qualifier et quantifier les conflits de coordination en fonction de la situation.}
			\end{description}
		\end{myblock}
	\end{myframe}
	\subsubsection{Tâche proposée}
	\begin{myframe}<1>[label={fra-exp2-TacheProposee}]{Tâche proposée}
		\begin{columns}[t]
			\begin{column}{0.3\textwidth}
				\only<1>{%
					\begin{myblock}{Scénarios}
						\begin{itemize}
							\item \mynum{2}~niveaux de manipulation
								\note[item]{Les deux niveaux de manipulation vont nous permettre d'observer différentes situation de conflits de coordination afin de mieux les qualifier et de les quantifier.}
								\begin{itemize}
									\item Résidu
									\item Atome
								\end{itemize}
							\item \mynum{4}~scénarios
							\item \mynum{3}~critères de complexité
								\note[item]{Afin d'observer les variations de performances en fonction de la complexité, nous proposons deux niveaux différents de complexité.
									Le seul facteur actuel de complexité que nous avons constaté concerne le nombre d'atomes.}
								\begin{itemize}
									\item Nombre d'atomes
									\item Cassures
									\item Champs de force
								\end{itemize}
						\end{itemize}
					\end{myblock}
				}
				\only<2>{%
					\begin{myblock}{Scénarios}
						\mynum{2}~niveaux de complexité
						\begin{itemize}
							\item faiblement couplé
							\item fortement couplé
						\end{itemize}
					\end{myblock}
				}
			\end{column}
			\begin{column}{0.65\textwidth}
				\renewcommand{\schemafactor}{0.045}
				\setlength{\schemaunit}{\schemafactor\paperwidth}
				\psset{unit=\schemaunit}
				\begin{myfigure}
					\begin{myps}(-1,0)(12,9)
						\rput[bl](1,0){\myimage[width=10\schemaunit]{TRP-ZIPPER}}
						\pnode(6.8,3.6){deformed}
						\rput(9.5,4.55){\rnode{deformed-label}{\textcolor{myred}{\mymultiline{molécule\\(à déformer)}}}}
						\pnode(1.8,4){ghost}
						\rput(0.5,5.25){\rnode{ghost-label}{\textcolor{myred}{\mymultiline{molécule\\(objectif)}}}}
						\psset{linecolor=myblue}
						\cnode(6.2,4.9){1.0}{deformed-residue}
						\rput(7.7,7.4){\rnode{deformed-residue-label}{\textcolor{myblue}{résidu (à déformer)}}}
						\cnode(2.3,1.6){0.8}{ghost-residue}
						\rput(0.5,2.75){\rnode{ghost-residue-label}{\textcolor{myblue}{\mymultiline{résidu\\(objectif)}}}}
						\psset{linecolor=gray}
						\cnode(2.0,6.6){0.8}{fixed-residue}
						\rput(4.0,8){\rnode{fixed-residue-label}{\textcolor{gray}{résidu fixe}}}
						\psset{linewidth=1pt,linecolor=myred,linearc=.1,arrowsize=0.5pt 3,arrowinset=.2,nodesepA=3pt}
						\ncangle[angleA=-90,outAngleA=-90,angleB=0,outAngleB=0]{c->}{deformed-label}{deformed}
						\psset{nodesepB=0pt}
						\ncdiagg[angleA=-90,outAngleA=-90,angleB=135,outAngleB=135]{c->}{ghost-label}{ghost}
						\psset{linecolor=myblue}
						\ncdiagg[angleA=-90,outAngleA=-90]{c->}{deformed-residue-label}{deformed-residue}
						\ncdiagg[angleA=-90,outAngleA=-90]{c->}{ghost-residue-label}{ghost-residue}
						\ncdiagg[angleA=180,outAngleA=180,linecolor=gray]{c->}{fixed-residue-label}{fixed-residue}
						\ncline[linewidth=8pt,linecolor=mygreen]{C->}{deformed-residue}{ghost-residue}
						\only<1>{%
							\psframe*[linecolor=green](-1,8.5)(5,9)
							\psframe*[linecolor=red](5,8.5)(12,9)
							\rput(5.5,8.75){\textcolor{white}{\bf score \textsc{rmsd}}}
							\psframe[linewidth=1pt,linecolor=black](-1,0)(12,9)
						}
						\only<2>{%
							\psframe*[linecolor=blue](11.25,0)(12,2)
							\psline[linecolor=blue](11,2)(11.25,2)
							\uput{1pt}[180](11,2){\tiny\textcolor{blue}{RMSD courant}}
							\psframe*[linecolor=orange](11.45,0)(11.8,1)
							\psline[linecolor=orange](11,1)(11.45,1)
							\uput{1pt}[180](11,1){\tiny\textcolor{orange}{RMSD minimum}}
							\psframe[linewidth=1pt,linecolor=black](-1,0)(12,9)
						}
					\end{myps}
					\mycaption[fig-TacheDeDeformation]{Tâche de déformation}
				\end{myfigure}
			\end{column}
		\end{columns}
	\end{myframe}
	\subsubsection{Résultats}
	\begin{myframe}{Amélioration de la répartition des ressources}
		\begin{columns}[T]
			\begin{column}{0.65\textwidth}
				\begin{myfigure}
					\mylegend{%
						\myleg{monôme}{myblue}%
						\myleg{binôme}{myblue!70}%
					}
					\begin{myboxgraph}[llx=-2em,yAxisLabelPos={-2em,c}](2,0.8\textwidth){distance}(3.5,1.9cm){distance~(mm)}
						\myboxplot{exp2-diff-activepassive-group.csv}
						\only<2>{%
							\psframe[linecolor=myred,linewidth=2pt,framearc=0.25,dimen=inner](0,0)(1,2)%
						}
						\only<3>{%
							\psframe[linecolor=mygreen,linewidth=2pt,framearc=0.25,dimen=inner](1,0)(2,2)%
						}
					\end{myboxgraph}
					\mycaption[fig-exp2-DistancePassiveEtActive]{Distances passive et active}
				\end{myfigure}
			\end{column}
			\begin{column}{0.3\textwidth}
				\note{%
					Nous avons constaté avec surprise que la manipulation bimanuelle permettait d'obtenir des espaces de travail plus important qu'en monomanuel : en d'autres termes, une personne seule couvre une plus grande surface qu'un binôme.
					En corrélant les observations durant l'expérimentation avec ce résultat, nous avons compris que la distance ne représente pas une mesure fiable.
					En effet, les manipulateurs seuls n'utilisent pas toujours les deux outils à leur disposition; dans le cas où ils n'en utilisent qu'un seul, le second est laissé sur le côté ce qui a pour effet de gonfler la mesure de distance bien qu'un des outils soit passif (d'où le nom de distance passive).
					Nous avons donc introduit la mesure de distance active qui ne mesure la distance que lorsque les deux outils sont en action (ils ont sélectionné une structure moléculaire).
				}
				\begin{myblock}{Résultat}
					Espace de travail plus important en binôme
				\end{myblock}
			\end{column}
		\end{columns}
		\begin{columns}[T]
			\begin{column}{0.65\textwidth}
				\begin{myfigure}
					\mylegend{%
						\myleg{main dominante}{myblue}%
						\myleg{main dominée}{myblue!70}%
					}
					\begin{myboxgraph}(2,0.8\textwidth)[10]{}(50,1.9cm){sélections~(nb)}
						\myboxplot{exp2-numsel-group-dominant.csv}
					\end{myboxgraph}
					\mycaption[fig-exp2-NombreDeSelections]{Nombre de sélections}
				\end{myfigure}
			\end{column}
			\begin{column}{0.3\textwidth}
				\begin{myblock}{Résultat}
					Taux d'utilisation des ressources en binôme supérieur de \mynum[\%]{40}
				\end{myblock}
			\end{column}
		\end{columns}
	\end{myframe}
	\begin{myframe}{Influence de la complexité de la tâche}
		\begin{mytable}
			\begin{mytabular}{Cp{6cm}L}
				\mytoprule
				Difficulté & Description & Exemple \\
				\mymiddlerule[\heavyrulewidth]
				1 & \mynum{1}~manipulation & Tâche~\myscenario{1a} \\
				\mymiddlerule
				2 & \mynum{1}~manipulation et \mynum{1}~fixation & Tâches~\myscenario{1r} et \myscenario{2r} \\
				\mymiddlerule
				3 & \mynum{2}~manipulations \alert{coordonnées} & Tâche~\myscenario{2a} \\
				\mybottomrule
			\end{mytabular}
			\mycaption[tab-ClassificationDesTaches]{Classification des tâches}
		\end{mytable}
		\begin{myfigure}
			\mylegend{%
				\myleg{monôme}{myblue}%
				\myleg{binôme}{myblue!70}%
			}
			\begin{myboxgraph}[llx=-3em,yAxisLabelPos={-3em,c},lly=-3.5ex,xAxisLabelPos={c,-3.5ex}](4,0.75\textwidth)[100]{scénarios}(350,1.75cm){temps~(s)}
				\myboxplot{exp2-time-task-group.csv}
			\end{myboxgraph}
			\mycaption[fig-exp2-TempsDeRealisationDesScenariosEnFonctionDuNombreDeSujets]{Temps de réalisation des scénarios}
		\end{myfigure}
		\note[item]{La distinction entre les scénarios \myemph{simple} et les scénarios \myemph{avancés} réside dans les cassures à réaliser qui s'effectuent de manière plus efficaces avec \mynum{2}~outils de manipulation.}
		\note[item]{Les scénarios de type expert possèdent des champs de force trop important pour être appréhendés à l'aide d'un seul outil.}
	\end{myframe}
	\subsubsection{Synthèse}
	\begin{myframe}{Synthèse}
		\begin{columns}[t]
			\begin{column}{0.475\textwidth}
				\centering
				\myunode[90][shadowcolor=mygreen](0,0)[distribution-problem]{Charge de travail}[4cm]
				\begin{myplusblock}{Résultats \mycite{Simard-2012b}}
					\begin{itemize}
						\item Gestion d'un plus grand espace de travail en binôme
						\item Meilleur taux d'utilisation des ressources disponibles en binôme
					\end{itemize}
				\end{myplusblock}
				\begin{myminusblock}{Limites}
					\begin{itemize}
						\item Comment répartir équitablement la charge de travail ?
					\end{itemize}
				\end{myminusblock}
			\end{column}
			\begin{column}{0.475\textwidth}
				\centering
				\myunode[90][shadowcolor=mygreen](0,0)[conflict-problem]{Conflits de coordination}[4cm]
				\begin{myplusblock}{Résultats \mycite{Simard-2011a}}
					\begin{itemize}
						\item Certaines manipulations nécessitent une coordination
					\end{itemize}
				\end{myplusblock}
				\begin{myminusblock}{Limites}
					\begin{itemize}
						\item La coordination est plus intuitive en individuel mais\dots
							\begin{itemize}
								\item \dots{}espace de travail restreint
									\note[item]{Dans un environnement complexe, nous souhaitons pouvoir effectuer des manipulations avec plus de \mynum{2}~tâches élémentaires ce qui implique aussi de pouvoir couvrir un espace de travail plus grand.}
								\item \dots{}coordination limitée à deux outils
									\note[item]{Le contexte proposé dans cette étude est encore de taille et de complexité relativement réduite comparé aux besoins réels; on peut aisément imaginer des tâches plus complexe impliquant de nombreuses tâches élémentaires à réaliser.}
							\end{itemize}
					\end{itemize}
				\end{myminusblock}
			\end{column}
		\end{columns}
		\note[item]{La prochaine étude nous permet d'analyser des environnements plus complexes en introduisant plus de participants dans la collaboration; notre intuition est une augmentation des conflits de coordination liés à la difficulté de communication dans un groupe de plus de \mynum{2}~personnes.}
	\end{myframe}
	\subsection{Étude~\mynum{3} -- Dynamique de groupe}
	\begin{myframe}{Sommaire}
		\tableofcontents[sectionstyle=show/shaded,subsectionstyle=show/shaded/hide,subsubsectionstyle=show/show/hide]
	\end{myframe}
	\subsubsection{Notions importantes sur la dynamique de groupe}
	\begin{myframe}{Notions importantes sur la dynamique de groupe}
		\begin{columns}[T]
			\begin{column}{0.475\textwidth}
				\begin{myblock}{Facilitation sociale \mycite{Triplett-1898}}
					Une action collaborative préparée ou en progression possède une réponse; la stimulation sociale provoque une augmentation de cette réponse par la perception de collaborateurs effectuant les mêmes mouvements.
				\end{myblock}
			\end{column}
			\begin{column}{0.475\textwidth}
				\begin{myblock}{Paresse sociale \mycite{Ringelmann-1913}}
					Tendance à fournir un effort moindre lorsqu'une tâche est effectuée en groupe plutôt que de manière individuelle.
				\end{myblock}
			\end{column}
		\end{columns}
		\begin{columns}[T]
			\begin{column}{0.475\textwidth}
				\begin{myfigure}
					\begin{myboxgraph}(3,0.8\textwidth)[10]{}(60,2cm){vitesse~(km/h)}
						\mybarplot{exp3-triplett.csv}
						\mybarlabel(0.5,38.650022682){\mynum[km/h]{39}}
						\mybarlabel(1.5,50.161371429){\mynum[km/h]{50}}
						\mybarlabel(2.5,52.502386951){\mynum[km/h]{52}}
					\end{myboxgraph}
					\mycaption[fig-sota-PerformancesDeCyclistes]{Performances d'un cycliste}
				\end{myfigure}
			\end{column}
			\begin{column}{0.475\textwidth}
				\begin{myfigure}
					\begin{myboxgraph}[llx=-3em,yAxisLabelPos={-3em,c}](8,0.8\textwidth)[0.2]{taille des groupes}(1,2cm){performance}
						\mybarplot{exp3-ringelmann.csv}
						\mybarlabel(0.5,1.00){\rotateleft{\mynum[\%]{100}}}
						\mybarlabel(1.5,0.93){\rotateleft{\mynum[\%]{93}}}
						\mybarlabel(2.5,0.85){\rotateleft{\mynum[\%]{85}}}
						\mybarlabel(3.5,0.77){\rotateleft{\mynum[\%]{77}}}
						\mybarlabel(4.5,0.70){\rotateleft{\mynum[\%]{70}}}
						\mybarlabel(5.5,0.63){\rotateleft{\mynum[\%]{63}}}
						\mybarlabel(6.5,0.56){\rotateleft{\mynum[\%]{56}}}
						\mybarlabel(7.5,0.49){\rotateleft{\mynum[\%]{49}}}
					\end{myboxgraph}
					\mycaption[fig-sota-PerformanceAuTirALaCorde]{Performances au tir à la corde}
				\end{myfigure}
			\end{column}
		\end{columns}
	\end{myframe}
	\subsubsection{Objectifs}
	\begin{myframe}{Objectifs}
		\begin{myblock}{Objectif principal}
			Observer la dynamique de groupe lors d'une tâche nécessitant une coordination étroitement couplée
		\end{myblock}
		\begin{myplusblock}{Hypothèses}
			\begin{enumerate}
				\item Amélioration des performances (binôme \myvs quadrinôme)
				\item Amélioration des performances par une étape de \mybrainstorming \mycite{Osborn-1963}
			\end{enumerate}
		\end{myplusblock}
		\begin{myblock}{Variables}
			\begin{description}
				\item[Nombre de participants] \mynum{8}~binômes et \mynum{4}~quadrinômes
				\item[Tâches différentes] \mynum{2}~molécules
				\item[Stratégie de travail] Étape de \mybrainstorming
			\end{description}
		\end{myblock}
	\end{myframe}
	\subsubsection{Protocole expérimental}
	\againframe<2>{fra-exp2-TacheProposee}
	\subsubsection{Résultats}
	\begin{myframe}{Amélioration des performances par la facilitation sociale}
		\begin{columns}[T]
			\begin{column}{0.65\textwidth}
				\begin{myfigure}
					\mylegend{%
						\myleg{binôme}{myblue}%
						\myleg{quadrinôme}{myblue!70}%
					}
					\begin{myboxgraph}[llx=-3em,yAxisLabelPos={-3em,c}](2,0.8\textwidth)[0.5]{scénarios}(1.75,1.9cm){vitesse~(mm/s)}
						\myboxplot{exp3-speed-molecule-group.csv}
					\end{myboxgraph}
					\mycaption[fig-exp3-VitesseMoyenne]{Vitesse moyenne}
				\end{myfigure}
			\end{column}
			\begin{column}{0.3\textwidth}
				\begin{myblock}{Résultat}
					Une vitesse de travail pour les quadrinômes supérieure de \mynum[\%]{50} : \alert{facilitation sociale}
				\end{myblock}
			\end{column}
		\end{columns}
		\begin{columns}[T]
			\begin{column}{0.65\textwidth}
				\begin{myfigure}
					\mylegend{%
						\myleg{binôme}{myblue}%
						\myleg{quadrinôme}{myblue!70}%
					}
					\begin{myboxgraph}(2,0.8\textwidth)[10]{scénarios}(40,1.9cm){échanges~(nb)}
						\myboxplot{exp3-talk-molecule-group.csv}
					\end{myboxgraph}
					\mycaption[fig-exp3-NombreDEchangeVerbaux]{Nombre d'échanges verbaux}
				\end{myfigure}
			\end{column}
			\begin{column}{0.3\textwidth}
				\begin{myblock}{Résultat}
					Moins d'échanges : \alert{paresse sociale}
					\begin{itemize}
						\item Spécialisation
						\item Personnalité
						\item Paresse
					\end{itemize}
				\end{myblock}
			\end{column}
		\end{columns}
	\end{myframe}
	\begin{myframe}{Influence du \mybrainstorming}
		\begin{columns}[T]
			\begin{column}{0.65\textwidth}
				\begin{myfigure}
					\mylegend{%
						\myleg{binôme}{myblue}%
						\myleg{quadrinôme}{myblue!70}%
					}
					\begin{myboxgraph}[llx=-3em,yAxisLabelPos={-3em,c}](2,0.8\textwidth)[300]{\mybrainstorming}(1100,1.9cm){temps~(s)}
						\myboxplot{exp3-time-brainstorm-group.csv}
					\end{myboxgraph}
					\mycaption[fig-exp3-TempsDeRealisation]{Temps de réalisation}
				\end{myfigure}
			\end{column}
			\begin{column}{0.3\textwidth}
				\begin{myblock}{Résultat}
					Le \mybrainstorming permet l'élaboration d'une stratégie : gain en performances
				\end{myblock}
			\end{column}
		\end{columns}
		\begin{columns}[T]
			\begin{column}{0.65\textwidth}
				\begin{myfigure}
					\mylegend{%
						\myleg{binôme}{myblue}%
						\myleg{quadrinôme}{myblue!70}%
					}
					\begin{myboxgraph}[llx=-3em,yAxisLabelPos={-3em,c}](2,0.8\textwidth)[0.1]{\mybrainstorming}(0.45,1.9cm){sélections~(nb/s)}
						\myboxplot{exp3-freqsel-brainstorm-group.csv}
					\end{myboxgraph}
					\mycaption[fig-exp3-FrequenceDesSelections]{Fréquence des sélections}
				\end{myfigure}
			\end{column}
			\begin{column}{0.3\textwidth}
				\begin{myblock}{Résultat}
					Meilleur rendement des actions effectuées
				\end{myblock}
			\end{column}
		\end{columns}
	\end{myframe}
	\subsubsection{Synthèse}
	\begin{myframe}{Synthèse}
		\begin{columns}[t]
			\begin{column}{0.475\textwidth}
				\centering
				\myunode[90][shadowcolor=myblue](0,0)[identification-problem]{Paresse sociale}[4cm]
				\begin{myplusblock}{Résultats \mycite{Simard-2012b}}
					\begin{itemize}
						\item Déséquilibre important dans la répartition des charges de travail
						\item Potentiel collaboratif non-exploité au maximum
					\end{itemize}
				\end{myplusblock}
				\begin{myminusblock}{Limites}
					\begin{itemize}
						\item Comment redonner de l'importance à chaque membre du groupe ?
					\end{itemize}
				\end{myminusblock}
			\end{column}
			\begin{column}{0.475\textwidth}
				\centering
				\myunode[90][shadowcolor=myblue](0,0)[brainstorming-problem]{\myBrainstorming}[4cm]
				\begin{myplusblock}{Résultats \mycite{Simard-2011b}}
					\begin{itemize}
						\item Amélioration importante des performances
						\item Réduit les conflits de coordination
						\item Conflits de communication pendant le \mybrainstorming
					\end{itemize}
				\end{myplusblock}
				\begin{myminusblock}{Limites}
					\begin{itemize}
						\item Comment optimiser cette étape ?
					\end{itemize}
				\end{myminusblock}
			\end{column}
		\end{columns}
	\end{myframe}
	\section{Communication haptique pour améliorer la collaboration}
	\subsection{Étude~\mynum{4} -- Métaphore haptique et stratégie de travail}
	\begin{myframe}{Sommaire}
		\tableofcontents[sectionstyle=show/shaded,subsectionstyle=show/shaded/hide,subsubsectionstyle=show/show/hide]
	\end{myframe}
	\subsubsection{Synthèse des résultats et solutions proposées}
	\begin{myframe}{Synthèse des résultats et solutions proposées}
		\begin{myfigure}
			\psset{xunit=0.95cm,yunit=0.8cm}
			\begin{myps}(-6,-3)(6,4)%
				\uput[-90](-2.5,4){\textcolor{black!50}{\Large Synthèse}}%
				\uput[-90](2.5,4){\textcolor{black!50}{\Large Proposition}}%
				\psset{arrowlength=1}%
				\onslide<1->{%
					\myunode[0][shadowcolor=myred](-5,2.5)[complexe-problem]{\onslide<2-3>{\color{black!25}}Complexité de la tâche}[4cm]%
					\myunode[0][shadowcolor=myred](-5,1.5)[strategy-problem]{\onslide<2-3>{\color{black!25}}Stratégie de travail}[4cm]%
					\psbrace*[linecolor=myred,ref=lC,nodesepA=-1pt](-5,3)(-5,1){\rotateright{\textcolor{myred}{\footnotesize Étude~1}}}%
				}%
				\onslide<2->{%
					\myunode[0][shadowcolor=mygreen](-5,0.5)[conflict-problem]{\onslide<3>{\color{black!25}}Conflits de coordination}[4cm]%
					\myunode[0][shadowcolor=mygreen](-5,-0.5)[distribution-problem]{\onslide<3>{\color{black!25}}Charge de travail}[4cm]%
					\psbrace*[linecolor=mygreen,ref=lC,nodesepA=-1pt](-5,1)(-5,-1){\rotateright{\textcolor{mygreen}{\footnotesize Étude~2}}}%
				}%
				\onslide<3->{%
					\myunode[0][shadowcolor=myblue](-5,-1.5)[identification-problem]{Paresse sociale}[4cm]%
					\myunode[0][shadowcolor=myblue](-5,-2.5)[brainstorming-problem]{\myBrainstorming}[4cm]%
					\psbrace*[linecolor=myblue,ref=lC,nodesepA=-1pt](-5,-1)(-5,-3){\rotateright{\textcolor{myblue}{\footnotesize Étude~3}}}%
				}%
				\onslide<4->{%
					\psbrace*[linecolor=black!50,ref=lC,nodesepA=1pt](5,-3)(5,3){\rotateright{\textcolor{black!50}{\footnotesize Étude~4}}}%
					\myunode[180][shadowcolor=black!50](5,2.5)[complexe-solution]{Étapes de \mydocking}[4cm]%
					\ncline[linewidth=4pt,linecolor=myred!25]{->}{complexe-problem}{complexe-solution}%
					\note[item]{Insister sur la spécificité de la tâche de \mydocking moléculaire : en quoi répond-elle à une complexité suffisante ?}%
				}%
				\onslide<5->{%
					\myunode[180][shadowcolor=black!50](5,1.5)[strategy-solution]{Manipulation de résidu}[4cm]%
					\ncline[linewidth=4pt,linecolor=myred!25]{->}{strategy-problem}{strategy-solution}%
				}%
				\onslide<7->{%
					\myunode[180][shadowcolor=black!50](5,-0.5)[distribution-solution]{Identifier les rôles}[4cm]%
					\ncline[linewidth=4pt,linecolor=mygreen!25]{->}{distribution-problem}{distribution-solution}%
					\note[item]{Identifier les rôles doit être l'occasion de définir les différents rôles présents afin de pouvoir expliquer la \myemph{frame} suivante}%
				}%
				\onslide<8->{%
					\ncangle[linewidth=4pt,linecolor=myblue!25,angleA=0,outAngleA=0,angleB=-90,outAngleB=-90]{->}{identification-problem}{distribution-solution}%
				}%
				\onslide<9->{%
					\myunode[180][shadowcolor=black!50](5,-2.5)[brainstorming-solution]{Rôle de coordinateur}[4cm]%
					\ncline[linewidth=4pt,linecolor=myblue!25]{->}{brainstorming-problem}{brainstorming-solution}%
				}%
				\onslide<6-9>{%
					\myunode[180][shadowcolor=black!50](5,0.5)[conflict-solution]{Métaphore haptique}[4cm]%
					\ncline[linewidth=4pt,linecolor=mygreen!25]{->}{conflict-problem}{conflict-solution}%
				}%
				\onslide<10>{%
					\myunode[180][fillstyle=solid,fillcolor=myred!25,shadowcolor=black!50](5,0.5)[conflict-solution]{Métaphore haptique}[4cm]%
					\ncline[linewidth=4pt,linecolor=mygreen!25]{->}{conflict-problem}{conflict-solution}%
				}%
			\end{myps}
			\mycaption[fig-SyntheseEtSolutions]{Synthèse et solutions}
		\end{myfigure}
	\end{myframe}
	\subsubsection{Présentation de la métaphore haptique de communication}
	\begin{myframe}{Présentation de la métaphore haptique de communication}
		\begin{columns}
			\begin{column}{0.6\textwidth}
				\begin{myfigure}
					\renewcommand{\schemafactor}{0.025}
					\setlength{\schemaunit}{\schemafactor\paperwidth}
					\psset{xunit=\schemaunit,yunit=0.531966204\schemaunit}
					\begin{myps}(-10,-10)(10,10)
						\only<1-2>{%
							\rput(0,0){\myimage[width=20\schemaunit]{designation-normal}}
							\uput{0pt}[45](-7,6){\myimage[width=2.75\schemaunit]{designation-yellow-cursor}}
						}
						\only<1>{%
							\uput{0pt}[45](2,7){\myimage[width=2.75\schemaunit]{designation-red-cursor}}
						}
						\only<3>{%
							\rput(0,0){\myimage[width=20\schemaunit]{designation-called}}
						}
						\only<2-3>{%
							\uput{0pt}[45](8.15,8.1){\myimage[width=2.75\schemaunit]{designation-red-cursor}}
							\uput{0pt}[45](-7,6){\myimage[width=2.75\schemaunit]{designation-yellow-cursor}}
						}
						\only<4-5>{%
							\rput(0,0){\myimage[width=20\schemaunit]{designation-accepted}}
							\pscurve[linewidth=2pt,linecolor=mygreen]{->}(-7,6)(-1,10)(6,10)(7.5,8.9)
							\uput[90](0,10){$\vec{F}$}
						}
						\only<4>{%
							\uput{0pt}[45](5,-8){\myimage[width=2.75\schemaunit]{designation-red-cursor}}
							\uput{0pt}[45](-7,6){\myimage[width=2.75\schemaunit]{designation-yellow-cursor}}
						}
						\only<6>{%
							\rput(0,0){\myimage[width=20\schemaunit]{designation-selected}}
						}
						\only<5-6>{%
							\uput{0pt}[45](5,-8){\myimage[width=2.75\schemaunit]{designation-red-cursor}}
							\uput{0pt}[45](8.15,8.1){\myimage[width=2.75\schemaunit]{designation-yellow-cursor}}
						}
						% Vibrations
						\uput{0pt}[45](-8,-8.5){\myimage[width=2.75\schemaunit]{designation-green-cursor}}
						\only<3>{%
							\mypsvibration(8.15,8.1)
							\mypsvibration(-7,6)
							\mypsvibration(-8,-9)
						}
						% About labels
						\only<1>{%
							\pnode(2,7){red-cursor}
							\rput(2,11){\rnode{red-cursor-label}{\textcolor{myred}{coordinateur\vphantom{p}}}}
							\ncangle[linecolor=myred,angleA=-90,outAngleA=-90,angleB=90,outAngleB=90]{c->}{red-cursor-label}{red-cursor}
							\pnode(-7,6){yellow-cursor}
							\rput(-7,10){\rnode{yellow-cursor-label}{\textcolor{myyellowcolor}{opérateur~\textsc{a}}}}
							\ncangle[linecolor=myyellowcolor,angleA=-90,outAngleA=-90,angleB=90,outAngleB=90]{c->}{yellow-cursor-label}{yellow-cursor}
							\pnode(-8,-8.5){green-cursor}
							\rput(-8,-10.75){\rnode{green-cursor-label}{\textcolor{mygreen}{opérateur~\textsc{b}\vphantom{É}}}}
							\ncangle[linecolor=mygreen,angleA=90,outAngleA=90,armA=0pt,angleB=-90,outAngleB=-90,armB=1pt]{c->}{green-cursor-label}{green-cursor}
						}
					\end{myps}
					\mycaption[fig-OutilDeDesignation]{Outil de désignation haptique}
				\end{myfigure}
			\end{column}
			\begin{column}{0.35\textwidth}
				\begin{myblock}{Étapes de la désignation}
					\begin{enumerate}
						\item<beamer:alert@2| beamer:myshade@1> Recherche d'une structure (coordinateur)
						\item<beamer:alert@3| beamer:myshade@1-2> Désignation de la structure (coordinateur)
						\item<beamer:alert@4-5| beamer:myshade@1-3> Acceptation par l'opérateur~\textsc{a}
						\item<beamer:alert@6| beamer:myshade@1-5> Sélection par l'opérateur~\textsc{a}
					\end{enumerate}
				\end{myblock}
			\end{column}
		\end{columns}
		\vfill
		\begin{myplusblock}{Métaphore haptique}%
			\begin{itemize}
				\item<beamer:alert@3| beamer:myshade@1-2> Notification visuo-haptique de la désignation aux opérateurs
				\item<beamer:alert@4-5| beamer:myshade@1-3> Guidage haptique $\displaystyle%
					\vec{F}(\vec{x}) = \left\{%
					\begin{array}{l@{$\quad$\textrm{if }}l}%
						\displaystyle k \left(t - t_0 \right) \left(\vec{x} - \vec{x_t}\right) - b \frac{\partial \vec{x}}{\partial t} & t \geq t_0 \\%
						0 & t < t_0%
					\end{array}%
				\right.$%
			\end{itemize}
		\end{myplusblock}
		\vfill
	\end{myframe}
	\subsubsection{Objectifs}
	\begin{myframe}{Objectifs}
		\begin{myblock}{Objectif principal}
			Évaluer la métaphore haptique pour améliorer la coordination
		\end{myblock}
		\begin{myplusblock}{Hypothèses}
			\begin{itemize}
				\item La métaphore haptique améliore
					\begin{enumerate}
						\item les performances
						\item la communication
					\end{enumerate}
			\end{itemize}
		\end{myplusblock}
		\begin{myblock}{Variables}
			\begin{description}
				\item[Nombre de participants] \mynum{8}~trinômes (composé de bio-informaticiens)
				\item[Tâches différentes] \mynum{2}~molécules
				\item[Métaphore haptique] Avec ou sans métaphore
			\end{description}
		\end{myblock}
	\end{myframe}
	\subsubsection{Résultats}
	\begin{myframe}{Efficacité de la collaboration}
		\begin{columns}[T]
			\begin{column}{0.65\textwidth}
				\begin{myfigure}
					\mylegend{%
						\myleg{sans métaphore}{myblue}%
						\myleg{avec métaphore}{myblue!70}%
					}
					\begin{myboxgraph}[llx=-3em,yAxisLabelPos={-3em,c}](2,0.8\textwidth)[200]{scénarios}(700,1.9cm){temps~(s)}
						\myboxplot{exp4-rmsd-time-molecule-haptic.csv}
					\end{myboxgraph}
					\mycaption[fig-exp4-TempsDeRealisationDeLaTache]{Temps de réalisation de la tâche}
				\end{myfigure}
			\end{column}
			\begin{column}{0.3\textwidth}
				\begin{myblock}{Résultat}
					Manipulation plus efficace sur le scénario le plus complexe
				\end{myblock}
			\end{column}
		\end{columns}
		\begin{columns}[T]
			\begin{column}{0.65\textwidth}
				\begin{myfigure}
					\mylegend{%
						\myleg{sans métaphore}{myblue}%
						\myleg{avec métaphore}{myblue!70}%
					}
					\begin{myboxgraph}[llx=-3.5em,yAxisLabelPos={-3.5em,c}](2,0.8\textwidth)[0.05]{scénarios}(0.125,1.9cm){fréquence~(nb/s)}
						\myboxplot{exp4-freq-molecule-haptic.csv}
					\end{myboxgraph}
					\mycaption[fig-exp4-FrequenceDeSelectionDUnOperateur]{Fréquence de sélection d'un opérateur}
				\end{myfigure}
			\end{column}
			\begin{column}{0.3\textwidth}
				\begin{myblock}{Résultat}
					Meilleur rendement pour l'utilisation des ressources
				\end{myblock}
			\end{column}
		\end{columns}
	\end{myframe}
	\begin{myframe}{Amélioration de la communication}
		\begin{columns}[T]
			\begin{column}{0.65\textwidth}
				\begin{myfigure}
					\begin{myboxgraph}[llx=-2em,yAxisLabelPos={-2em,c}](2,0.6\textwidth)[2]{haptique}(10,1.9cm){temps~(s)}
						\myboxplot{exp4-shake-time-haptic.csv}
					\end{myboxgraph}
					\mycaption[fig-exp4-TempsDAcceptationDUneDesignation]{Temps d'acceptation d'une désignation}
				\end{myfigure}
			\end{column}
			\begin{column}{0.3\textwidth}
				\begin{myblock}{Résultat}
					Communication haptique plus efficace que la communication verbale
				\end{myblock}
			\end{column}
		\end{columns}
		\begin{columns}[T]
			\begin{column}{0.65\textwidth}
				\begin{myfigure}
					\begin{myboxgraph}(2,0.6\textwidth)[5]{haptique}(17.5,1.9cm){sélections~(nb)}
						\myboxplot{exp4-accept-haptic.csv}
					\end{myboxgraph}
					\mycaption[fig-exp4-TauxDeDesignationsAcceptees]{Taux de désignations acceptées}
				\end{myfigure}
			\end{column}
			\begin{column}{0.3\textwidth}
				\begin{myblock}{Résultat}
					Meilleur taux d'acceptation pour les désignations du coordinateur
				\end{myblock}
			\end{column}
		\end{columns}
	\end{myframe}
	\section{Conclusion et perspectives}
	\subsection{Conclusions}
	\begin{myframe}{Conclusion}
		\begin{itemize}
				\vfill
			\item La plateforme \myShaddock
				\begin{itemize}
					\item Intégration pour la manipulation en environnement virtuel
					\item Plateforme pertinente pour la déformation moléculaire, validée par des bio-informaticiens
				\end{itemize}
				\vfill
			\item Étude des approches collaboratives
				\begin{itemize}
					\item Mesure d'une amélioration pour les tâches étroitement couplées
					\item Constatation de différentes stratégies de travail
					\item Caractérisation des conflits de communication et de coordination
				\end{itemize}
				\vfill
			\item Amélioration des approches collaboratives : la communication haptique
				\begin{itemize}
					\item Configuration de travail adaptée
					\item Amélioration de la tâche de désignation
				\end{itemize}
				\vfill
		\end{itemize}
	\end{myframe}
	\subsection{Perspectives}
	\begin{myframe}{Perspectives}
		\begin{itemize}
				\vfill
			\item Problématiques dans l'expérimentation des approches collaboratives
				\begin{itemize}
					\item protocole expérimental pour l'évaluation des approches collaboratives
					\item mesure des conflits de coordination et de communication
					\item mesure de la charge de travail
				\end{itemize}
				\vfill
			\item Étendre l'étude des approches étroitement couplées
				\begin{itemize}
					\item apprentissage non-supervisé en collaboration
					\item collaboration distante
					\item multi-expertise dans la collaboration
					\item domaines d'application (conception, assemblage, \dots)
				\end{itemize}
				\vfill
		\end{itemize}
	\end{myframe}
	\subsection{Références}
	\nocite{Simard-2009}
	\nocite{Simard-2010a}
	\nocite{Simard-2010b}
	\nocite{Simard-2010c}
	\nocite{Simard-2011a}
	\nocite{Simard-2011b}
	\nocite{Simard-2012a}
	\nocite{Simard-2012b}
	\nocite{Simard-2012c}
	\nocite{Girard-2012a}
	\nocite{Girard-2012b}
	\begin{myframe}{Publications}
		\defbibheading{bibliography}{\footnotesize\textcolor{myblue}{Journaux internationaux avec comité de relecture}\par}
		\printbibliography[type=article,category=myrefs]
		\defbibheading{bibliography}{\footnotesize\textcolor{myblue}{Conférences internationales avec comité de relecture}\par}
		\printbibliography[type=inproceedings,category=myrefs]
	\end{myframe}
	\appendix
	\section{Organisation logicielle}
	\begin{myframe}{Organisation logicielle}
		\begin{myfigure}
			\psset{xunit=0.0666667\textwidth,yunit=0.06\textheight}
			\psset{framearc=.1,shadow=true,blur=true}
			\begin{myps}(-7.5,-7)(7.5,6)
				\let\oldll\ll
				\renewcommand{\ll}{\scriptscriptstyle\oldll}
				\let\oldgg\gg
				\renewcommand{\gg}{\scriptscriptstyle\oldgg}
				\psframe*[linecolor=mygreen!10,shadow=false](-7.4,-7)(-2.6,6)
				\psframe*[linecolor=myblue!10,shadow=false](-2.4,-7)(2.4,6)
				\psframe*[linecolor=myred!10,shadow=false](2.6,-7)(7.4,6)
				\uput[-90](-5,6){\large\textcolor{mygreen!25}{Simulation}}
				\uput[-90](0,6){\large\textcolor{myblue!25}{Visualisation}}
				\uput[-90](5,6){\large\textcolor{myred!25}{Interaction}}
				\uput[-90](0,5){%
					\myumlnode*<PCUtilisateur>{\vphantom{pÉ}\footnotesize Nœud principal}{%
						\myumlcomponent<VMD>[\textcolor{myblue!50}{\scriptsize application}]{\footnotesize\myVMD}%
					}%
				}
				\uput[-90](-5,5){%
					\myumlnode*<ServeurNAMD>{\vphantom{pÉ}\footnotesize Nœud \myNAMD}{%
						\begin{psmatrix}[rowsep=1]%
							\myumlcomponent<NAMD>[\textcolor{myblue!50}{\scriptsize programme}]{{\footnotesize\texttt{namd2}}} \\%
							\myumlcomponent<FichierSimulation>[\textcolor{myblue!50}{\scriptsize fichier}]{%
								\\[-1ex]%
								\begin{psmatrix}[rowsep=0]%
									\footnotesize Données de\\\footnotesize simulation%
								\end{psmatrix}%
							}
						\end{psmatrix}%
					}%
				}
				\uput[-90](5,5){%
					\myumlnode*<ServeurVRPN1>{\vphantom{pÉ}\footnotesize Nœud \myVRPN~\mynum{1}}{%
						\myumlcomponent<VRPN1>[\textcolor{myblue!50}{\scriptsize programme}]{{\footnotesize\texttt{vrpn\_server}}}%
					}%
				}
				\uput[-90](5,0.75){%
					\myumlnode<PHANToM1>[\textcolor{myblue!50}{\scriptsize\myOmni}]{\vphantom{pÉ}\footnotesize Interface haptique}%
				}
				\uput[-90](5,-2){%
					\myumlnode*<ServeurVRPN2>{\vphantom{pÉ}\footnotesize Nœud \myVRPN~\mynum{2}}{%
						\myumlcomponent<VRPN2>[\textcolor{myblue!50}{\scriptsize programme}]{{\footnotesize\texttt{vrpn\_server}}}%
					}%
				}
				\uput[-90](0,0){%
					\myumlnode<VideoProjecteur>[\textcolor{myblue!50}{\scriptsize vue partagée}]{\footnotesize Dispositif d'affichage}
				}
				\psset{shadow=false}

				\myumlrealization[angleA=-90,outAngleA=-90,angleB=90,outAngleB=90]{NAMD}{FichierSimulation}[nccurve]%
				\myumlrealization[angleA=-90,outAngleA=-90,angleB=90,outAngleB=90]{VRPN1}{PHANToM1}[nccurve]%
				\myumlinterface[angleA=-90,outAngleA=-90,angleB=90,outAngleB=90,ArrowInsidePos=0.5]{VMD}{VideoProjecteur}[nccurve]
				\myumlinterface[angleA=0,outAngleA=0,angleB=180,outAngleB=180,offsetB=-14pt,ArrowInsidePos=0.4]{NAMD}{VMD}[nccurve]
				\myumlinterface[angleA=180,outAngleA=180,angleB=0,outAngleB=0,ncurvA=1.5,offsetA=14pt,ArrowInsidePos=0.6]{VRPN1}{VMD}[nccurve]
				\myumlinterface[angleA=180,outAngleA=180,angleB=0,outAngleB=0,ncurvA=0.67,ncurvB=0.33,offsetA=14pt,ArrowInsidePos=0.4]{VRPN2}{VMD}[nccurve]
				\psframe*[linecolor=myred!10,shadow=false](2.6,-5.85)(7.4,-6.2)
				\psframe*[linecolor=white,shadow=false](2.6,-6.2)(7.4,-6.5)
				\psframe*[linecolor=white,shadow=false](2.6,-6.7)(7.4,-6.9)
				\rput(5,-6.5){\textcolor{black!50}{Serveurs multiples}}
			\end{myps}
			\mycaption[fig-Shaddock-DiagrammeDeDeploiementUMLDeLaPlateformeShaddock]{Diagramme de déploiement \myUML de la plateforme \myShaddock}
		\end{myfigure}
		\note{Cette plateforme a donné lieu à un travail d'intégration important.}
		\note[item]{Le nœud \myNAMD a nécessité une configuration importante pour la simulation, effectuée avec des biologistes}
		\note[item]{Le nœud \myVRPN a nécessité une modification qui a été soumise et intégrée par les développeurs dans les nouvelles versions}
		\note[item]{C'est au niveau de \myVMD que le plus gros développement a été effectué avec l'ajout de nouveaux outils dédiés pour améliorer la manipulation interactive}
	\end{myframe}
	\section{Apprentissage en collaboration}
	\begin{myframe}{Apprentissage en collaboration}
		\begin{columns}[T]
			\begin{column}{0.65\textwidth}
				\begin{myfigure}
					\mylegend{%
						\myleg{monôme}{myblue}%
						\myleg{binôme}{myblue!70}%
					}
					\begin{myboxgraph}[llx=-3em,yAxisLabelPos={-3em,c}](3,0.8\textwidth)[100]{essais}(250,1.9cm){temps~(s)}
						\myboxplot{exp2-time-try-group.csv}
					\end{myboxgraph}
					\mycaption[fig-exp2-EvolutionDuTempsDeRealisationDeLaTache]{Évolution du temps de réalisation de la tâche}
				\end{myfigure}
			\end{column}
			\begin{column}{0.3\textwidth}
				\begin{myblock}{Résultat}
					Apprentissage plus rapide avec une amélioration des performances
				\end{myblock}
			\end{column}
		\end{columns}
		\begin{columns}[T]
			\begin{column}{0.65\textwidth}
				\begin{myfigure}
					\mylegend{%
						\myleg{monôme}{myblue}%
						\myleg{binôme}{myblue!70}%
					}
					\begin{myboxgraph}[llx=-3em,yAxisLabelPos={-3em,c}](3,0.8\textwidth)[1]{essais}(3,1.9cm){distance~(mm)}
						\myboxplot{exp2-active-try-group.csv}
					\end{myboxgraph}
					\mycaption[fig-exp4-EvolutionDeLEspaceDeTravail]{Évolution de l'espace de travail}
				\end{myfigure}
			\end{column}
			\begin{column}{0.3\textwidth}
				\begin{myblock}{Résultat}
					Un grand espace de travail est rapidement couvert par les binômes
				\end{myblock}
			\end{column}
		\end{columns}
	\end{myframe}
\end{document}
