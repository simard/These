\documentclass[english,french,dvips]{mybeamer}
\usepackage{my}
\usepackage{mydate}
\usepackage{mycolor}
\usepackage{myfloat}
\usepackage[biblatex]{mybib}
\usepackage{myps}
\makeatletter
% Modify the bibliography style
\newcounter{mymaxcitenames}
\AtBeginDocument{%
	\setcounter{mymaxcitenames}{\value{maxnames}}%
}
\renewbibmacro{begentry}{%
	\printtext[brackets]{%
		\defcounter{maxnames}{\value{mymaxcitenames}}%
		\printnames{labelname}~\usebibmacro{cite:labelyear+extrayear}%
	}%
	\newline%
}
% AAA
\newcommand{\myACER}{\textsc{acer}\xspace}
\newcommand{\myAlanine}{Alanine\xspace}
\newcommand{\myanalysis}[1]{\input{files/#1}\%}
\newcommand{\myangstrom}{\AA ngström\xspace}
\newcommand{\myanova}[1]{\input{files/#1}}
\newcommand{\myatom}[2][]{%
	{%
		\ifstrempty{#1}%
			{\makefirstuc{\textsf{#2}}}%
			{\textcolor{#1}{\makefirstuc{\textsf{#2}}}}%
		\xspace%
	}%
}
\newcommand{\myAudacity}{\textsc{audacity}\myregistered}% No '\xspace' because of already one in '\myregistered'
% CCC
\newcommand{\mycarbon}{\myatom[mycarboncolor]{C}}
\newcommand{\myCasioXJ}{\textsc{Casio xj}\xspace}
\newcommand{\myCHARMM}{\textsc{charmm}\xspace}
\newcommand{\myChimera}{\textsc{chimera}\xspace}
\newcommand{\myClayWorks}{\textsc{Clayworks}\xspace}
\newcommand{\mycondition}[1]{$\left(\mathcal{C}_{#1}\right)$\xspace}
\newcommand{\myCPK}{\textsc{cpk}\xspace}
% DDD
\newcommand{\myDesktop}{\myPHANToM Desktop\myregistered}% No '\xspace' because of already one in '\myregistered'
% FFF
\newcommand{\myfeuillet}{feuillet-$\beta$\xspace}
\WithSuffix\newcommand\myfeuillet*{feuillets-$\beta$\xspace}
\newcommand{\myform}[1]{\textbf{\sffamily\MakeUppercase{#1}}}
% GGG
\newcommand{\myGhost}{\textsc{Ghost}\xspace}
\newcommand{\myGromacs}{\textsc{Gromacs}\xspace}
\newcommand{\mygroup}[1]{$\left(\mathcal{G}_{#1}\right)$\xspace}
% HHH
\newcommand{\myHaption}{\textsc{Haption}\xspace}
\newcommand{\myHawthorne}{\myemph{Hawthorne Works}\xspace}
\newcommand{\myHBonds}{\textit{HBonds}\xspace}
\newcommand{\myhelice}{hélice-$\alpha$\xspace}
\WithSuffix\newcommand\myhelice*{hélices-$\alpha$\xspace}
\newcommand{\myhypothesis}[1]{$\left(\mathcal{H}_{#1}\right)$\xspace}
% III
\newcommand{\myIntelCore}{Intel\myregistered Core\mytrademark~2 \textsc{q9450} (\mynum[GHz]{2.66})\xspace}
% JJJ
\newcommand{\myJmol}{\textsc{Jmol}\xspace}
% LLL
\newcommand{\myLCD}{\textsc{lcd}\xspace}
\newcommand{\myLicorice}{\textit{Licorice}\xspace}
\newcommand{\myLinux}{\textsc{Linux}\xspace}
% MMM
\newcommand{\myMacOS}{Mac~\textsc{OS}\xspace}
\newcommand{\myMDDriver}{\textsc{MDDriver}\xspace}
% NNN
\newcommand{\myNewRibbon}{\textit{NewRibbon}\xspace}
\def\mynode{%
	\@ifnextchar[{\mynode@i}{\mynode@i[style=nodestyle]}%
}
\def\mynode@i[#1](#2,#3)[#4]#5{%
	\rput(#2,#3){\Rnode{#4}{\psframebox[style=nodestyle,#1]{\vphantom{pÉ}#5}}}%
}
\newcommand{\mynytrogen}{\myatom[mynytrogencolor]{A}}
\newcommand{\myNusE}{\textsc{NusE}\xspace}
\newcommand{\myNusENusG}{\textsc{NusE:NusG}\xspace}
\newcommand{\myNusG}{\textsc{NusG}\xspace}
% OOO
\newcommand{\myOmni}{\myPHANToM Omni\myregistered}% No '\xspace' because of already one in '\myregistered'
\newcommand{\myOpenHaptics}{\textsc{OpenHaptics}\mytrademark}% No '\xspace' because of already one in '\mytrademark'
\newcommand{\myoxygen}{\myatom[myoxygencolor]{O}}
% PPP
\newcommand{\myPC}{\textsc{pc}\xspace}
\newcommand{\myPDB}{\textsc{pdb}\xspace}
\newcommand{\myPDBbase}{\emph{Protein~DataBase}\xspace}
\newcommand{\myPDBlink}[2]{\href{#1}{\textsc{\MakeLowercase{#2}}}}
\newcommand{\myPHANToM}{\textsc{phant}o\textsc{m}\xspace}
\newcommand{\myPremium}{\myPHANToM Premium\myregistered}% No '\xspace' because of already one in '\myregistered'
\newcommand{\myPrion}{Prion\xspace}
\newcommand{\myPSF}{\textsc{psf}\xspace}
\newcommand{\mypvalue}{$p$-value\xspace}
\newcommand{\myPyMOL}{\textsc{p}y\textsc{mol}\xspace}
% RRR
\newcommand{\myRAM}[2][Go]{\mynum[#1]{#2} de \textsc{ram}}
\newcommand{\myRasmol}{\textsc{RasMol}\xspace}
\newcommand{\myresidue}[1]{$\left(\mathcal{R}_{#1}\right)$\xspace}
% SSS
\newcommand{\myscenario}[1]{\textsc{#1}}
\newcommand{\mySensAble}{\textsc{SensAble}\xspace}
\newcommand{\myShaddock}{\textsc{Shaddock}\xspace}
\newcommand{\mySony}{\textsc{sony}\myregistered}% No '\xspace' because of already one in '\myregistered'
\newcommand{\mySpaceNavigator}{SpaceNavigator\myregistered}% No '\xspace' because of already one in '\myregistered'
\newcommand{\mysubject}[1]{$\mathcal{S}_#1$}
\newcommand{\mysulfur}{\myatom[mysulfurcolor]{S}}
\newcommand{\mysummary}[1]{\input{files/#1}}
% TTT
\newcommand{\myTCPIP}{\textsc{tcp/ip}\xspace}
\newcommand{\myThreeD}{\textsc{3d}\xspace}
\newcommand{\mytool}[1]{\myemph{#1}}
\newcommand{\myTRPCAGE}{\textsc{trp-cage}\xspace}
\newcommand{\myTRPZIPPER}{\textsc{trp-zipper}\xspace}
% UUU
\newcommand{\myUbiquitin}{Ubiquitin\xspace}
\newcommand{\myUbuntu}{\textsc{Ubuntu}~v$10.04$\xspace}
\newcommand{\myUSB}{\textsc{usb}\xspace}
\newcommand{\myuser}[1]{$\mathcal{#1}$}
% VVV
\newcommand{\myvar}[2]{$\left(\mathcal{V}_{\mathrm{#1}#2}\right)$\xspace}
\newcommand{\myvard}[1]{\myvar{d}{#1}}
\newcommand{\myvari}[1]{\myvar{i}{#1}}
\newcommand{\myVGA}{\textsc{vga}\xspace}
\newcommand{\myVirtuose}{\textsc{Virtuose}\mytrademark~\textsc{6d}\mynum{35}--\mynum{45}\xspace}
% WWW
\newcommand{\myWindows}{\textsc{Windows}\xspace}

% Needed lengths
\newlength{\mywidth}
\newlength{\myheight}

% PSTricks style
\newpsstyle{nodestyle}{framearc=0.25,shadow=true,shadowcolor=myblue,blur=true}
\makeatother

\title{Collaboration haptique étroitement couplée pour la déformation moléculaire interactive}
\author{Jean \myname{Simard}}
\date{\mydate[datestyle=long]{01/02/2012}}
\begin{document}
	\transfade[duration=0.1]
	\begin{frame}
		\titlepage
	\end{frame}
	\begin{frame}{Sommaire}
		\tableofcontents[hideallsubsections]
	\end{frame}
	\section{Introduction}
	\subsection{\protect\textit{Docking} moléculaire}
	\begin{frame}{Définition}
		\begin{block}{\textit{Docking} moléculaire}
			ou \myemph{amarrage moléculaire}, consiste à trouver l'orientation et la conformation optimale permettant d'assembler \mynum{2}~molécules.
		\end{block}
		\begin{myfigure}
			\psset{unit=1cm}
			\begin{myps}(-5,-2)(6,2)
				\rput(-3,1){\myimage[width=0.120518359\paperwidth]{sota-docking-molecule-A}}
				\mycircleletter(-4.5,1){A}
				\rput(-3,-1){\myimage[width=0.1\paperwidth,angle=90]{sota-docking-molecule-B}}
				\mycircleletter(-4.5,-1){B}
				\rput(-3,0){\Huge\bfseries +}
				\rput(3,0){\myimage[width=0.186177106\paperwidth]{sota-docking-complex-AB}}
				\psline[linewidth=10pt,linecolor=myblue!70]{->}(-1,0)(1,0)
				\mycircleletter(5,0.75){A}
				\rput(5,0){\Large\bfseries +}
				\mycircleletter(5,-0.75){B}
			\end{myps}
			\mycaption[fig-IllustrationDeLAmarrageMoleculaire]{Illustration de l'amarrage moléculaire}
		\end{myfigure}
	\end{frame}
	\subsection{Distribution des charges de travail}
	\subsection{Collaboration en environnement virtuel}
	\subsection{Synthèse}
	\begin{frame}{Déroulement}
		\begin{myfigure}
			\psset{xunit=1cm,yunit=1.1cm}
			\begin{myps}(-2.5,-0.5)(2.5,5)
				\only<1>{
					\mynode[fillstyle=solid,fillcolor=myred!25](0,4)[Search]{Recherche}
					\mycirclenumber(0,4){1}
				}
				\only<2>{
					\mynode[fillstyle=solid,fillcolor=myred!25](0,3)[Selection]{Sélection}
					\mycirclenumber(0,3){2}
				}
				\only<3>{
					\mynode[fillstyle=solid,fillcolor=myred!25](0,2)[Manipulation]{Déformation}
					\mycirclenumber(0,2){3}
				}
				\only<4>{
					\mynode[fillstyle=solid,fillcolor=myred!25](0,1)[Evaluation]{Évaluation}
					\mycirclenumber(0,1){4}
				}
				\only<5>{
					\mycirclenumber[180](0,2.5){5}
				}
				\only<5->{
					\mynode(0,1)[Evaluation]{Évaluation}
				}
				\only<4->{
					\mynode(0,2)[Manipulation]{Déformation}
					\ncline{->}{Manipulation}{Evaluation}
				}
				\only<3->{
					\mynode(0,3)[Selection]{Sélection}
					\ncline{->}{Selection}{Manipulation}
				}
				\only<2->{
					\mynode(0,4)[Search]{Recherche}
					\ncline{->}{Search}{Selection}
				}
				\only<5->{
					\ncloop[loopsize=4em,angleA=-90,angleB=90,linearc=0.05,armA=0.2]{->}{Evaluation}{Search}
				}
				\only<6>{
					\mynode[fillstyle=solid,fillcolor=myred!25](0,0)[Objective]{Objectif atteint}
					\mycirclenumber(0,0){6}
					\ncline{->}{Evaluation}{Objective}
				}
			\end{myps}
			\mycaption[fig-ProcessusDeDeformationMoleculaire]{Processus de déformation moléculaire}
		\end{myfigure}
	\end{frame}
	\section{\myShaddock}
	\section{Étude du travail collaboratif}
	\subsection{Recherche collaborative}
	\subsubsection{Synthèse}
	\begin{frame}<1-3>[label=fra-synthesis]{Synthèse}
		\begin{myfigure}
			\psset{xunit=0.95cm,yunit=0.8cm}
			\begin{myps}(-6,-3)(6,4)
				\uput[-90](-2.5,4){\textcolor{black!50}{\Large Problématiques}}
				\uput[-90](2.5,4){\textcolor{black!50}{\Large Solutions}}
				\psset{arrowlength=1}
				\onslide<1->{%
					\myunode[0][shadowcolor=myred](-5,2.5)[complexe-problem]{\onslide<2->{\onslide<4-9>{\color{black!25}}Complexité de la tâche}}[4cm]
					\myunode[0][shadowcolor=myred](-5,1.5)[strategy-problem]{\onslide<3->{\onslide<4-9>{\color{black!25}}Stratégie de travail}}[4cm]
					\psbrace*[linecolor=myred,ref=lC,nodesepA=-1pt](-5,3)(-5,1){\rotateright{\textcolor{myred}{\tiny Étude~1}}}%
				}
				\onslide<4->{%
					\myunode[0][shadowcolor=mygreen](-5,0.5)[distribution-problem]{\onslide<5->{\onslide<7-9>{\color{black!25}}Charge de travail}}[4cm]
					\myunode[0][shadowcolor=mygreen](-5,-0.5)[conflict-problem]{\onslide<6->{\onslide<7-9>{\color{black!25}}Conflits de coordination}}[4cm]
					\psbrace*[linecolor=mygreen,ref=lC,nodesepA=-1pt](-5,1)(-5,-1){\rotateright{\textcolor{mygreen}{\tiny Étude~2}}}%
				}
				\onslide<7->{%
					\myunode[0][shadowcolor=myblue](-5,-1.5)[identification-problem]{\onslide<8->{Paresse sociale}}[4cm]
					\myunode[0][shadowcolor=myblue](-5,-2.5)[brainstorming-problem]{\onslide<9->{\textit{Brainstorming}}}[4cm]
					\psbrace*[linecolor=myblue,ref=lC,nodesepA=-1pt](-5,-1)(-5,-3){\rotateright{\textcolor{myblue}{\tiny Étude~3}}}%
				}
				\onslide<10->{%
					\psbrace*[linecolor=black!50,ref=lC,nodesepA=1pt](5,-3)(5,3){\rotateright{\textcolor{black!50}{\tiny Étude~4}}}
				}
				\onslide<11->{%
					\myunode[180][shadowcolor=black!50](5,2.5)[complexe-solution]{Tâche de \textit{docking}}[4cm]%
					\ncline[linewidth=4pt,linecolor=myred!25]{->}{complexe-problem}{complexe-solution}
				}
				\onslide<12->{%
					\myunode[180][shadowcolor=black!50](5,1.5)[strategy-solution]{Manipulation de résidu}[4cm]%
					\ncline[linewidth=4pt,linecolor=myred!25]{->}{strategy-problem}{strategy-solution}
				}
				\onslide<13->{%
					\myunode[180][shadowcolor=black!50](5,0.5)[distribution-solution]{Distribuer les rôles}[4cm]%
					\ncline[linewidth=4pt,linecolor=mygreen!25]{->}{distribution-problem}{distribution-solution}
				}
				\onslide<14->{%
					\myunode[180][shadowcolor=black!50](5,-0.5)[conflict-solution]{Solutions haptiques}[4cm]%
					\ncline[linewidth=4pt,linecolor=mygreen!25]{->}{conflict-problem}{conflict-solution}
				}
				\onslide<15->{%
					\myunode[180][shadowcolor=black!50](5,-1.5)[identification-solution]{Identifier les rôles}[4cm]%
					\ncline[linewidth=4pt,linecolor=myblue!25]{->}{identification-problem}{identification-solution}
				}
				\onslide<16->{%
					\myunode[180][shadowcolor=black!50](5,-2.5)[brainstorming-solution]{Phase exploratoire}[4cm]%
					\ncline[linewidth=4pt,linecolor=myblue!25]{->}{brainstorming-problem}{brainstorming-solution}
				}
			\end{myps}
			\mycaption[fig-SyntheseDesProblematiques]{Synthèse des problématiques}
		\end{myfigure}
	\end{frame}
	\subsection{Déformation collaborative}
	\subsubsection{Synthèse}
	\againframe<4-6>{fra-synthesis}
	\subsection{Dynamique de groupe}
	\subsubsection{Synthèse}
	\againframe<7-9>{fra-synthesis}
	\section{Aide au travail collaboratif}
	\subsection{Assistance haptique}
	\subsubsection{Synthèse}
	\againframe<10->{fra-synthesis}
	\begin{frame}{Conclusion}
	\end{frame}
	\section{Conclusion}
	\subsection{Synthèse}
	\subsection{Perspectives}
	\begin{frame}{Questions}
		Merci pour votre attention
	\end{frame}
\end{document}
