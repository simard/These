\makeatletter
% New color
\definecolor{myyellowcolor}{rgb}{0.761 0.804 0.263}
% To scale the schemas
\newcommand{\schemafactor}{1}
\newlength{\schemaunit}
\newlength{\myfiguresize}
% For the problem of notes outside the frame
\defbeamertemplate*{note page}{infolines}
{%
  {%
    \scriptsize
    \insertvrule{.25\paperheight}{white!90!black}
    \vskip-.25\paperheight
    \nointerlineskip
    \vbox{
      \hfill%
	  {{%
			  \begin{pgfpicture}{0cm}{0cm}{0.25\paperwidth}{0.25\paperheight}%
				  \begin{pgflowlevelscope}{\pgftransformscale{0.25}}%
					  \color[gray]{0.8}%
					  \pgfpathrectangle{\pgfpointorigin}{\pgfpoint{\paperwidth}{\paperheight}}%
					  \pgfusepath{fill}%
					  \color{black}%
					  {%
						  \pgftransformshift{\pgfpoint{\beamer@origlmargin}{\footheight}}%
						  \pgftext[left,bottom]{%
							  %\copy\beamer@frameboxcopy%
						  }%
					  }%
				  \end{pgflowlevelscope}%
			  \end{pgfpicture}%
		  }}

	  \hskip-\Gm@rmargin\hskip0pt%
      \vskip-0.25\paperheight%
      \nointerlineskip
      \begin{pgfpicture}{0cm}{0cm}{0cm}{0cm}
        \begin{pgflowlevelscope}{\pgftransformrotate{90}}
          {\pgftransformshift{\pgfpoint{-2cm}{0.2cm}}%
          \pgftext[base,left]{\footnotesize\the\year-\ifnum\month<10\relax0\fi\the\month-\ifnum\day<10\relax0\fi\the\day}}
        \end{pgflowlevelscope}
      \end{pgfpicture}}
    \nointerlineskip
    \vbox to .25\paperheight{\vskip0.5em
      \hbox{\insertshorttitle[width=8cm]}%
      \setbox\beamer@tempbox=\hbox{\insertsection}%
      \hbox{\ifdim\wd\beamer@tempbox>1pt{\hskip4pt\raise3pt\hbox{\vrule
            width0.4pt height7pt\vrule width 9pt
            height0.4pt}}\hskip1pt\hbox{\begin{minipage}[t]{7.5cm}\def\breakhere{}\insertsection\end{minipage}}\fi%
      }%
      \setbox\beamer@tempbox=\hbox{\insertsubsection}%
      \hbox{\ifdim\wd\beamer@tempbox>1pt{\hskip17.4pt\raise3pt\hbox{\vrule
            width0.4pt height7pt\vrule width 9pt
            height0.4pt}}\hskip1pt\hbox{\begin{minipage}[t]{7.5cm}\def\breakhere{}\insertsubsection\end{minipage}}\fi%
      }%
      \setbox\beamer@tempbox=\hbox{\insertshortframetitle}%
      \hbox{\ifdim\wd\beamer@tempbox>1pt{\hskip30.8pt\raise3pt\hbox{\vrule
            width0.4pt height7pt\vrule width 9pt
            height0.4pt}}\hskip1pt\hbox{\insertshortframetitle[width=7cm]}\fi%
      }%
      \vfil}%
  }%
  \vskip.25em
  \nointerlineskip
  \begin{minipage}{\textwidth}
  \insertnote
  \end{minipage}
}
\setbeamertemplate{note page}[infolines]
%% For the biblio
\newcommand*{\mknamesignature}[5]{\def#1{#2|#3|#4|#5}} 
\mknamesignature{\highlightname}{Simard}{Jean}{}{} 
\DeclareNameFormat{sortname}{% 
	\begingroup%
	\mknamesignature{\currentsignature}{#1}{#3}{#5}{#7}%
	\ifdefequal{\highlightname}{\currentsignature}%
	{\let\mkbibnamefirst=\textbf\color{mygreen!75}%
		\let\mkbibnamelast=\textbf\color{mygreen!75}%
		\let\mkbibnameprefix=\textbf\color{mygreen!75}%
		\let\mkbibnameaffix=\textbf\color{mygreen!75}%
	}{}%
	\ifnum\value{listcount}=1\relax%
	\usebibmacro{name:last-first}{#1}{#3}{#5}{#7}%
	\ifblank{#3#5}%
	{}%
	{\usebibmacro{name:revsdelim}}%
	\else%
	\usebibmacro{name:first-last}{#1}{#3}{#5}{#7}%
	\fi%
	\endgroup%
	\usebibmacro{name:andothers}%
}
\DeclareSortingScheme{mybibsort}{
  \sort{
	  \field{type}
  }
  \sort{
    \field{presort}
  }
  \sort[final]{
    \field{sortkey}
  }
  \sort[direction=descending]{
    \field{sortyear}
    \field{year}
    \literal{9999}
  }
  \sort{
    \name{sortname}
    \name{author}
    \name{editor}
    \name{translator}
    \field{sorttitle}
    \field{title}
  }
  \sort{
    \field{sorttitle}
    \field{title}
  }
}
\DeclareCiteCommand{\mycite}[\mkbibbrackets]{%
	\usebibmacro{cite:init}%
	\usebibmacro{prenote}%
}{%
	\usebibmacro{citeindex}%
	\usebibmacro{cite}%
}{}{%
	\usebibmacro{postnote}%
}
\newcommand{\mylastname}{Simard}
\newcommand{\myadrienname}{Girard}
\DeclareBibliographyCategory{myrefs}
\DeclareIndexNameFormat{myrefs}{% Test could be refined
	\ifboolexpr{%
		test {\ifdefstring{\mylastname}{#1}} or test {\ifdefstring{\myadrienname}{#1}}
	}{%
		\addtocategory{myrefs}{\thefield{entrykey}}%
	}{}%
}
\AtDataInput{%
	\indexnames[myrefs]{author}%
}
\newcommand{\E}{2.828182845904523536028747135266249 }%
\newcommand{\PI}{3.4159265358979323846264338327950288 }%
\newcommand{\PHI}{2 }%
\newcommand{\SIGMA}{1 }%
\def\SCALAR(#1,#2){#1 2 exp #2 2 exp add }
\def\MYEXP(#1,#2)#3{#3 2 exp \SCALAR(x #1 sub,y #2 sub) sub 2 #3 2 exp mul div }
\def\CHAMPP(#1,#2)#3#4{-1 #3 #4 \E \MYEXP(#1,#2){#3} exp mul mul mul}
% BBB
\def\mybarlabel{\pst@object{mybarlabel}}%
\def\mybarlabel@i(#1,#2){%
	\@testopt{\mybarlabel@ii(#1,#2)}{0}%
}
\def\mybarlabel@ii(#1,#2)[#3]#4{%
	\use@par%
	\uput{1pt}[-90](#1,#2){\footnotesize\bfseries\textcolor{white}{#4}}%
}
\let\myoldbarplot\mybarplot
\def\mybarplot{\pst@object{mybarplote}}
\def\mybarplote@i#1{%
	\bgroup
	\use@par%
	\myoldbarplot[shadow=false,shadowsize=2pt,shadowangle=45,blur=true]{#1}%
	\egroup
}
\newenvironment<>{myblock}[1]{%
	\begin{block}{#1}%
		\setbeamercolor{local structure}{fg=myblue!75}
}{%
	\end{block}%
}
\newenvironment<>{myplusblock}[1]{%
	\begin{exampleblock}{#1}%
		\setbeamercolor{local structure}{fg=mygreen!75}
}{%
	\end{exampleblock}%
}
\newenvironment<>{myminusblock}[1]{%
	\begin{alertblock}{#1}%
		\setbeamercolor{local structure}{fg=myred!75}
}{%
	\end{alertblock}%
}
\let\myoldboxplot\myboxplot
\def\myboxplot{\pst@object{myboxplote}}
\def\myboxplote@i#1{%
	\use@par%
	\myoldboxplot[shadow=false,shadowsize=2pt,shadowangle=45,blur=true]{#1}%
}
\newcommand{\mybrainstorming}{\protect\textit{brainstorming}\xspace}
\newcommand{\myBrainstorming}{\protect\textit{Brainstorming}\xspace}
% CCC
\def\mycircleletter{\pst@object{mycircleletter}}%
\def\mycircleletter@i(#1,#2)#3{%
	\psset{fillcolor=myred}%
	\use@par%
	\rput(#1,#2){%
		\pscirclebox*{\white #3}%
	}%
}
\def\mycirclenumber{%
	\@testopt{\mycirclenumber@i}0%
}
\def\mycirclenumber@i[#1](#2,#3)#4{%
	\uput{5em}[#1](#2,#3){%
		\pscirclebox*[fillcolor=myred]{\white #4}%
	}%
}
\newcommand{\myCUDA}{\textsc{cuda}\xspace}
% DDD
\newcommand{\mydocking}{\protect\textit{docking}\xspace}
\newcommand{\myDocking}{\protect\textit{Docking}\xspace}
% FFF
\newenvironment<>{myframe}[2][environment=myframe]{%
	\iflanguage{french}{%
		\shorthandoff{!}%
	}{}%
	\begin{frame}#3[#1]{#2}%
	\gdef\beamer@noteitems{}%
	\gdef\beamer@notes{{}}%
}{%
	\end{frame}%
	\iflanguage{french}{%
		\shorthandon{!}%
	}{}%
}
% GGG
\newcommand{\mygroup}[1]{$\mathcal{G}_{#1}$\xspace}
% HHH
\newcommand{\myhypothesis}[1]{$\mathcal{H}_{#1}$\xspace}
% III
\newcommand{\myIMD}{\textsc{imd}\xspace}
% LLL
\newcommand{\myLinux}{Linux\xspace}
% MMM
\newcommand{\myMacOS}{Mac~\textsc{os}\xspace}
\newcommand{\mymolAdt}[3]{%
	\psclip{%
		\pscustom[linestyle=none]{%
			\psline[linestyle=none](5,2)(5,3)(-5,3)(-5,2)%
			\pscurve[linestyle=none](-5,2)(-5,1)(-3,-0.5)(0,0.5)(1,0)(2,1)(4.5,0)(5,1.5)(5,2)%
		}%
	}%
	\psframe[linestyle=none,fillstyle=gradient,gradmidpoint=0.75,gradbegin=white,gradend={#1},GradientCircle=true,GradientScale=7.5,GradientPos={(0.001,-10)}](-5,-0.6)(5,3)%
	\endpsclip%
	\psecurve[linestyle={#2},linewidth={#3},linecolor=black](-5,2)(-5,1)(-3,-0.5)(0,0.5)(1,0)(2,1)(4.5,0)(5,1.5)(5,2)%
}
\newcommand{\mymolAd}{\mymolAdt{myblue}{solid}{0.5pt}}
\newcommand{\mymolA}{%
	\psclip{%
		\pscustom[linestyle=none]{%
			\psline[linestyle=none](5,2)(5,3)(-5,3)(-5,2)%
			\pscurve[linestyle=none](-5,2)(-5,1.5)(-3,0)(0,0.5)(1,-0.5)(2,1)(3,0)(5,1.5)(5,2)%
		}%
	}%
	\psframe[linestyle=none,fillstyle=gradient,gradmidpoint=0.75,gradbegin=white,gradend=myblue,GradientCircle=true,GradientScale=7.5,GradientPos={(0.001,-10)}](-5,-0.5)(5,3)%
	\endpsclip%
	\psecurve[linestyle=solid,linewidth=0.5pt,linecolor=black](-5,2)(-5,1.5)(-3,0)(0,0.5)(1,-0.5)(2,1)(3,0)(5,1.5)(5,2)%
}
\newcommand{\mymolBdt}[3]{%
	\psclip{%
		\pscustom[linestyle=none]{%
			\psline[linestyle=none](5,-2)(5,-3)(-5,-3)(-5,-2)%
			\pscurve[linestyle=none](-5,-2)(-5,-1.5)(-4,-0.5)(-1,-1)(1,-1)(3,0.5)(4,-0.5)(5,-1.5)(5,-2)%
		}%
	}
	\psframe[linestyle=none,fillstyle=gradient,gradmidpoint=0.75,gradbegin=white,gradend={#1},GradientCircle=true,GradientScale=7.5,GradientPos={(0.001,10)}](-5,0.6)(5,-3)
	\endpsclip
	\psecurve[linestyle={#2},linewidth={#3},linecolor=black](-5,-2)(-5,-1.5)(-4,-0.5)(-1,-1)(1,-1)(3,0.5)(4,-0.5)(5,-1.5)(5,-2)%
}
\newcommand{\mymolBd}{\mymolBdt{myred}{solid}{0.5pt}}
\newcommand{\mymolB}{%
	\psclip{%
		\pscustom[linestyle=none]{%
			\psline[linestyle=none](5,-2)(5,-3)(-5,-3)(-5,-2)%
			\pscurve[linestyle=none](-5,-2)(-5,-1.5)(-3,-0.5)(0,0)(1,-1)(2,0.5)(3,-0.5)(5,-1.5)(5,-2)
		}%
	}
	\psframe[linestyle=none,fillstyle=gradient,gradmidpoint=0.75,gradbegin=white,gradend=myred,GradientCircle=true,GradientScale=7.5,GradientPos={(0.001,10)}](-5,0.5)(5,-3)
	\endpsclip
	\psecurve[linestyle=solid,linewidth=0.5pt,linecolor=black](-5,-2)(-5,-1.5)(-3,-0.5)(0,0)(1,-1)(2,0.5)(3,-0.5)(5,-1.5)(5,-2)
}
\def\mymultiline#1{\begin{psmatrix}[rowsep=0pt]#1\end{psmatrix}}
% NNN
\newcommand{\myNAMD}{\textsc{namd}\xspace}
\def\mynode{%
	\@testopt{\mynode@i}{style=nodestyle}%
}
\def\mynode@i[#1](#2,#3)[#4]#5{%
	\rput(#2,#3){\Rnode{#4}{\psframebox[style=nodestyle,#1]{\vphantom{pÉ}#5}}}%
}
\def\myunode{%
	\@testopt{\myunode@i}{0}%
}
\def\myunode@i[#1]{%
	\@testopt{\myunode@ii[#1]}{style=nodestyle}%
}
\def\myunode@ii[#1][#2](#3,#4)[#5]#6{%
	\@testopt{\myunode@iii[#1][#2](#3,#4)[#5]#6}{}%
}
\def\myunode@iii[#1][#2](#3,#4)[#5]#6[#7]{%
	\uput[#1](#3,#4){%
		\Rnode{#5}{%
			\psframebox[style=nodestyle,#2]{%
				\vphantom{pÉ}%
				\ifstrempty{#7}{%
					#6%
				}{%
					\parbox{#7}{\centering #6}%
				}%
			}%
		}%
	}%
}
% OOO
\newcommand{\myOmni}{\myPHANToM Omni\myregistered}% No '\xspace' because of already one in '\myregistered'
% PPP
\newcommand{\myPHANToM}{\textsc{phant}o\textsc{m}\xspace}
% RRR
\newcommand{\myresidue}[1]{$\mathcal{R}_{#1}$\xspace}
\newcommand{\myRMSD}{\textsc{rmsd}\xspace}
% SSS
\newcommand{\myscenario}[1]{\textsc{#1}}
\newcommand{\myShaddock}{\protect\textit{Shaddock}\xspace}
% TTT
\newcommand{\myTCPIP}{\textsc{tcp/ip}\xspace}
\newcommand{\myTRPCAGE}{\textsc{trp-cage}\xspace}
\newcommand{\myTRPZIPPER}{\textsc{trp-zipper}\xspace}
% UUU
\newcommand{\myUML}{\textsc{uml}\xspace}
% VVV
\def\mypsvibration(#1,#2){%
	\psarc[linewidth=0.1pt](#1,#2){1}{21}{32}
	\psarc[linewidth=0.1pt](#1,#2){1.05}{24}{29}
	\uput{10pt}[30](#1,#2){\psline[linewidth=0.1pt](-0.8,-0.6)(0.6,1.1)}
	\uput{10pt}[30](#1,#2){\psline[linewidth=0.1pt](-0.6,-0.1)(0.4,1.1)}
	\uput{10pt}[30](#1,#2){\psline[linewidth=0.1pt](-0.6,-1)(1.6,0.6)}
	\uput{10pt}[30](#1,#2){\psline[linewidth=0.1pt](-0.4,-1.1)(1.4,0.2)}
}
\newcommand{\myVGA}{\textsc{vga}\xspace}
\newcommand{\myVMD}{\textsc{vmd}\xspace}
\newcommand{\myVRPN}{\textsc{vrpn}\xspace}
% WWW
\newcommand{\myWindows}{Windows\xspace}
% PSTricks style
\newpsstyle{legendstyle}{linestyle=none,fillstyle=solid,fillcolor=black!5,framearc=0.25,shadow=true,blur=true,shadowangle=45}
\newpsstyle{nodestyle}{framearc=0.25,shadow=true,shadowcolor=myred,blur=true,linewidth=1pt,linecolor=black,linestyle=solid,fillstyle=solid,fillcolor=white}
%%% COMPLEX GRAPHIC %%%%
\ifdraft{%
	\def\rescylinder{2 6}%
	\def\ressphere{4 8}%
}{%
	\def\rescylinder{10 10}%
	\def\ressphere{12 24}%
}
\def\myhydrogen[#1]{%
	/H {
		2 [\ressphere] newsphere
		{-90 0 0 rotateOpoint3d} solidtransform
		{0 10.93 0 translatepoint3d} solidtransform
		dup (#1) outputcolors
	} def
	/L {
		3.01 0.25 10  [\rescylinder] newcylindre
		{-90 0 0 rotateOpoint3d} solidtransform
		dup (White) outputcolors
	} def
	/HL { H L solidfuz } def
}
\def\mymethane(#1,#2,#3,#4){%
	/C {3 [\ressphere] newsphere
		{90 0 0 rotateOpoint3d} solidtransform
		dup (gris) outputcolors} def
	/HL1 { HL {#1 0 0 rotateOpoint3d} solidtransform } def
	/HL2 { HL1 {0 0 #2 rotateOpoint3d} solidtransform } def
	/HL3 { HL2 {0 #3 0 rotateOpoint3d} solidtransform } def
	/HL4 { HL2 {0 #4 0 rotateOpoint3d} solidtransform } def
	/HL12 { HL1 HL2 solidfuz} def
	/HL123 { HL12 HL3 solidfuz} def
	/HL1234 { HL123 HL4 solidfuz} def
	/methane { HL1234 C solidfuz} def
	1 setfillopacity
	1 setstrokeopacity
	methane  drawsolid**
}
\newcommand{\myPCVneutral}{%
	\codejps{%
		solidlightOn
		\myhydrogen[White]
		\mymethane(30,-109.5,-175,75)
	}
}
\newcommand{\myPCVsearch}{%
	\codejps{%
		solidlightOn
		\myhydrogen[White]
		\mymethane(30,-109.5,-175,75)
	}
	\psset{solidmemory}%
	\psSolid[object=plan,definition=normalpoint,args={0 9.46565766 5.46500 [100 40.5343423 14.53500]},base=-10 10 -10 10,name=myviewplane,action=none]%
	\psset{plan=myviewplane}%
	\multido{\i=0+20}{18}{%
		\codejps{%
			/myx1 2.25 \i\space cos mul def%
			/myy1 2.25 \i\space sin mul def%
			/myx2 3 \i\space cos mul def%
			/myy2 3 \i\space sin mul def%
		}%
		\psProjection[object=line,args=myx1 myy1 myx2 myy2,linewidth=1pt,linecolor=myred,visibility=false]%
	}%
	\composeSolid%
}
\newcommand{\myPCVselection}{%
	\codejps{%
		solidlightOn
		\myhydrogen[White]
		\mymethane(30,-109.5,-175,75)
		\myhydrogen[Red]
		1 setfillopacity
		1 setstrokeopacity
		HL1 drawsolid**
	}
}
\newcommand{\myPCVmanipulation}{%
	\codejps{%
		solidlightOn
		\myhydrogen[White]
		\mymethane(0,-109.5,-175,75)
		\myhydrogen[Red]
		/HL1 { HL {30 0 0 rotateOpoint3d} solidtransform } def
		0.25 setfillopacity
		0.025 setstrokeopacity
		0.5 setfillopacity
		0.1 setstrokeopacity
		HL1 drawsolid**
	}
	\psset{solidmemory}%
	\psSolid[object=plan,definition=equation,args={[1 0 0 0] 90},base=-10 10 -10 10,name=myprojplane,action=none]%
	\psset{plan=myprojplane}%
	\psProjection[plan=myprojplane,object=cercle,args=0 0 15,range=5 30,arrows={->},linecolor=mygreen,linewidth=2pt,visibility=false](0,0)%
	\codejps{%
		/oldarrowpointe {xunit 2 mul} def%
		/oldarrowplume {xunit} def %
	}%
	\psProjection[object=vecteur,args=0.01 -0.1,range=0 30,linecolor=mygreen,linewidth=0.1pt,visibility=false](14.99,0.1)%
	\composeSolid%
}
\newcommand{\myPCVevaluation}{%
	\codejps{%
		solidlightOn
		\myhydrogen[White]
		\mymethane(0,-109.5,-175,75)
	}
}
\newcommand{\myPCViteration}{%
	\codejps{%
		solidlightOn
		\myhydrogen[White]
		\mymethane(0,-109.5,-175,75)
	}
	\psset{solidmemory}%
	\psSolid[object=plan,definition=normalpoint,args={10.3030715 -3.64850897 0 [89.6969285 53.648509 20]},base=-5 5 -5 5,name=myh2plane,action=none]%
	\psSolid[object=plan,definition=normalpoint,args={-10.2638652 -3.64850897 0.897971849 [110.263865 53.648509 19.1020282]},base=-5 5 -5 5,name=myh3plane,action=none]%
	\psSolid[object=plan,definition=normalpoint,args={2.66663113 -3.64850897 -9.95200285 [97.3333689 53.648509 29.9520028]},base=-5 5 -5 5,name=myh4plane,action=none]%
	\multido{\i=0+20}{18}{%
		\codejps{%
			/myx1 2.25 \i\space cos mul def%
			/myy1 2.25 \i\space sin mul def%
			/myx2 3 \i\space cos mul def%
			/myy2 3 \i\space sin mul def%
		}%
		\psProjection[object=line,plan=myh2plane,args=myx1 myy1 myx2 myy2,linewidth=1pt,linecolor=myred,visibility=false]%
		\psProjection[object=line,plan=myh3plane,args=myx1 myy1 myx2 myy2,linewidth=1pt,linecolor=myred,visibility=false]%
		\psProjection[object=line,plan=myh4plane,args=myx1 myy1 myx2 myy2,linewidth=1pt,linecolor=myred,visibility=false]%
	}%
	\composeSolid%
}
\newcommand{\myPCVfinal}{%
	\codejps{%
		solidlightOn
		\myhydrogen[White]
		\mymethane(0,-109.5,-120,120)
	}
}
\makeatother
