\makeatletter
% Modify the bibliography style
\newcounter{mymaxcitenames}
\AtBeginDocument{%
	\setcounter{mymaxcitenames}{\value{maxnames}}%
}
\renewbibmacro{begentry}{%
	\printtext[brackets]{%
		\defcounter{maxnames}{\value{mymaxcitenames}}%
		\printnames{labelname}~\usebibmacro{cite:labelyear+extrayear}%
	}%
	\newline%
}
% BBB
\let\myoldbarplot\mybarplot
\def\mybarplot{\pst@object{mybarplote}}
\def\mybarplote@i#1{%
	\bgroup
	\use@par%
	\myoldbarplot[shadow=true,shadowangle=45,blur=true]{#1}%
	\egroup
}
\let\myoldboxplot\myboxplot
\def\myboxplot{\pst@object{myboxplote}}
\def\myboxplote@i#1{%
	\use@par%
	\myoldboxplot[shadow=true,shadowangle=45,blur=true]{#1}%
}
% CCC
\def\mycircleletter{\pst@object{mycircleletter}}%
\def\mycircleletter@i(#1,#2)#3{%
	\psset{fillcolor=myred}%
	\use@par%
	\rput(#1,#2){%
		\pscirclebox*{\white #3}%
	}%
}
\def\mycirclenumber{%
	\@testopt{\mycirclenumber@i}0%
}
\def\mycirclenumber@i[#1](#2,#3)#4{%
	\uput{5em}[#1](#2,#3){%
		\pscirclebox*[fillcolor=myred]{\white #4}%
	}%
}
\newcommand{\myCUDA}{\textsc{cuda}\xspace}
% GGG
\newcommand{\mygroup}[1]{$\mathcal{G}_{#1}$\xspace}
% HHH
\newcommand{\myhypothesis}[1]{$\mathcal{H}_{#1}$\xspace}
% III
\newcommand{\myIMD}{\textsc{imd}\xspace}
% LLL
\newcommand{\myLinux}{Linux\xspace}
% MMM
\newcommand{\myMacOS}{Mac~\textsc{os}\xspace}
\newcommand{\mymolAd}{%
	\psclip{%
		\pscustom[linestyle=none]{%
			\psline[linestyle=none](5,2)(5,3)(-5,3)(-5,2)%
			\pscurve[linestyle=none](-5,2)(-5,1)(-3,0)(0,0.5)(1,0)(2,1)(2.5,0)(5,1.5)(5,2)%
		}%
	}%
	\psframe[linestyle=none,fillstyle=gradient,gradmidpoint=0.75,gradbegin=white,gradend=myblue,GradientCircle=true,GradientScale=7.5,GradientPos={(0.001,-10)}](-5,-0.5)(5,3)%
	\endpsclip%
	\psecurve[linestyle=solid,linewidth=0.5pt,linecolor=black](-5,2)(-5,1)(-3,0)(0,0.5)(1,0)(2,1)(2.5,0)(5,1.5)(5,2)%
}
\newcommand{\mymolA}{%
	\psclip{%
		\pscustom[linestyle=none]{%
			\psline[linestyle=none](5,2)(5,3)(-5,3)(-5,2)%
			\pscurve[linestyle=none](-5,2)(-5,1.5)(-3,0)(0,0.5)(1,-0.5)(2,1)(3,0)(5,1.5)(5,2)%
		}%
	}%
	\psframe[linestyle=none,fillstyle=gradient,gradmidpoint=0.75,gradbegin=white,gradend=myblue,GradientCircle=true,GradientScale=7.5,GradientPos={(0.001,-10)}](-5,-0.5)(5,3)%
	\endpsclip%
	\psecurve[linestyle=solid,linewidth=0.5pt,linecolor=black](-5,2)(-5,1.5)(-3,0)(0,0.5)(1,-0.5)(2,1)(3,0)(5,1.5)(5,2)%
}
\newcommand{\mymolBd}{%
	\psclip{%
		\pscustom[linestyle=none]{%
			\psline[linestyle=none](5,-2)(5,-3)(-5,-3)(-5,-2)%
			\pscurve[linestyle=none](-5,-2)(-5,-1.5)(-3,-0.5)(-1,0)(1,-1)(2.5,0.5)(3,-0.5)(5,-1.5)(5,-2)
		}%
	}
	\psframe[linestyle=none,fillstyle=gradient,gradmidpoint=0.75,gradbegin=white,gradend=myred,GradientCircle=true,GradientScale=7.5,GradientPos={(0.001,10)}](-5,0.5)(5,-3)
	\endpsclip
	\psecurve[linestyle=solid,linewidth=0.5pt,linecolor=black](-5,-2)(-5,-1.5)(-3,-0.5)(-1,0)(1,-1)(2.5,0.5)(3,-0.5)(5,-1.5)(5,-2)
}
\newcommand{\mymolB}{%
	\psclip{%
		\pscustom[linestyle=none]{%
			\psline[linestyle=none](5,-2)(5,-3)(-5,-3)(-5,-2)%
			\pscurve[linestyle=none](-5,-2)(-5,-1.5)(-3,-0.5)(0,0)(1,-1)(2,0.5)(3,-0.5)(5,-1.5)(5,-2)
		}%
	}
	\psframe[linestyle=none,fillstyle=gradient,gradmidpoint=0.75,gradbegin=white,gradend=myred,GradientCircle=true,GradientScale=7.5,GradientPos={(0.001,10)}](-5,0.5)(5,-3)
	\endpsclip
	\psecurve[linestyle=solid,linewidth=0.5pt,linecolor=black](-5,-2)(-5,-1.5)(-3,-0.5)(0,0)(1,-1)(2,0.5)(3,-0.5)(5,-1.5)(5,-2)
}
% NNN
\newcommand{\myNAMD}{\textsc{namd}\xspace}
\def\mynode{%
	\@testopt{\mynode@i}{style=nodestyle}%
}
\def\mynode@i[#1](#2,#3)[#4]#5{%
	\rput(#2,#3){\Rnode{#4}{\psframebox[style=nodestyle,#1]{\vphantom{pÉ}#5}}}%
}
\def\myunode{%
	\@testopt{\myunode@i}{0}%
}
\def\myunode@i[#1]{%
	\@testopt{\myunode@ii[#1]}{style=nodestyle}%
}
\def\myunode@ii[#1][#2](#3,#4)[#5]#6{%
	\@testopt{\myunode@iii[#1][#2](#3,#4)[#5]#6}{}%
}
\def\myunode@iii[#1][#2](#3,#4)[#5]#6[#7]{%
	\uput[#1](#3,#4){%
		\Rnode{#5}{%
			\psframebox[style=nodestyle,#2]{%
				\vphantom{pÉ}%
				\ifstrempty{#7}{%
					#6%
				}{%
					\parbox{#7}{\centering #6}%
				}%
			}%
		}%
	}%
}
% OOO
\newcommand{\myOmni}{\myPHANToM Omni\myregistered}% No '\xspace' because of already one in '\myregistered'
% PPP
\newcommand{\myPHANToM}{\textsc{phant}o\textsc{m}\xspace}
% RRR
\newcommand{\myresidue}[1]{$\mathcal{R}_{#1}$\xspace}
\newcommand{\myRMSD}{\textsc{rmsd}\xspace}
% SSS
\newcommand{\myscenario}[1]{\textsc{#1}}
\newcommand{\myShaddock}{Shaddock\xspace}
% TTT
\newcommand{\myTCPIP}{\textsc{tcp/ip}\xspace}
% UUU
\newcommand{\myUML}{\textsc{uml}\xspace}
% VVV
\newcommand{\myVGA}{\textsc{vga}\xspace}
\newcommand{\myVMD}{\textsc{vmd}\xspace}
\newcommand{\myVRPN}{\textsc{vrpn}\xspace}
% WWW
\newcommand{\myWindows}{Windows\xspace}
% PSTricks style
\newpsstyle{legendstyle}{linestyle=none,fillstyle=solid,fillcolor=black!5,framearc=0.25,shadow=true,blur=true,shadowangle=45}
\newpsstyle{nodestyle}{framearc=0.25,shadow=true,shadowcolor=myred,blur=true,linewidth=1pt,linecolor=black,linestyle=solid,fillstyle=solid,fillcolor=white}
\makeatother
