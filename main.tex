\documentclass[myfrancais]{mythesis}
\usepackage{mydate}
\usepackage{graphicx}
\usepackage{caption}
\newcommand{\name}[1]{\textsc{#1}}
\newcommand{\acro}[1]{\textsc{#1}}
\newcommand{\LIMSI}{\acro{CNRS--LIMSI}}

\hypersetup{%
	pdftitle={Interactions haptiques collaboratives pour la manipulation moléculaire},%
	pdfauthor={Jean SIMARD},%
	pdfsubject={Mémoire de thèse en informatique},%
	pdfdisplaydoctitle=true,%
	pdflang={FR-fr}%
}

\title{Interactions haptiques collaboratives pour la manipulation moléculaire}
\author{Jean~\name{Simard}}
\documenttype{Thèse en Informatique}
\university{École Doctorale d'Informatique de Paris Sud}
\date{\mydate[datestyle=long]{01/12/2011}}
\jury{%
	Martin & \name{DUPONT} & (rapporteur) & Directeur de recherche au \LIMSI \\
	Martin & \name{DUPOND} & (examinateur) & Directeur de recherche au \LIMSI}
	\addglobalbib[datatype=bibtex]{biblio.bib}
\begin{document}
	\frontmatter
	\maketitle
	\mytoc
	\mylof
	\mylot
	\mainmatter
	\part{Le sujet}
	\chapter{Introduction}
	\part{Étude du travail collaboratif}
	\chapter{La recherche collaborative}
	\chapter{La manipulation collaborative}
	\chapter{Les dynamiques de groupe}
	\part{Propositions pour le travail collaboratif}
	\chapter{Travail collaboratif assisté par haptique}
	\part{Synthèse}
	\chapter{Conclusion et perspectives}

	\appendix
	\chapter{\acro{Shaddock} -- Collaborative Virtual Environment for Molecular Design}
\end{document}
