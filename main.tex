\documentclass[myfrancais]{mythesis}
\usepackage{mydate}
\usepackage{mytodo}
\usepackage{mycolor}
\usepackage{myps}
\usepackage{mymacro}

\makeatletter
% Modify the bibliography style
\newcounter{mymaxcitenames}
\AtBeginDocument{%
	\setcounter{mymaxcitenames}{\value{maxnames}}%
}
\renewbibmacro{begentry}{%
	\printtext[brackets]{%
		\defcounter{maxnames}{\value{mymaxcitenames}}%
		\printnames{labelname}~\usebibmacro{cite:labelyear+extrayear}%
	}%
	\newline%
}
% AAA
\newcommand{\myACER}{\textsc{acer}\xspace}
\newcommand{\myAlanine}{Alanine\xspace}
\newcommand{\myanalysis}[1]{\input{files/#1}\%}
\newcommand{\myangstrom}{\AA ngström\xspace}
\newcommand{\myanova}[1]{\input{files/#1}}
\newcommand{\myatom}[2][]{%
	{%
		\ifstrempty{#1}%
			{\makefirstuc{\textsf{#2}}}%
			{\textcolor{#1}{\makefirstuc{\textsf{#2}}}}%
		\xspace%
	}%
}
\newcommand{\myAudacity}{\textsc{audacity}\myregistered}% No '\xspace' because of already one in '\myregistered'
% CCC
\newcommand{\mycarbon}{\myatom[mycarboncolor]{C}}
\newcommand{\myCasioXJ}{\textsc{Casio xj}\xspace}
\newcommand{\myCHARMM}{\textsc{charmm}\xspace}
\newcommand{\myChimera}{\textsc{chimera}\xspace}
\newcommand{\myClayWorks}{\textsc{Clayworks}\xspace}
\newcommand{\mycondition}[1]{$\left(\mathcal{C}_{#1}\right)$\xspace}
\newcommand{\myCPK}{\textsc{cpk}\xspace}
% DDD
\newcommand{\myDesktop}{\myPHANToM Desktop\myregistered}% No '\xspace' because of already one in '\myregistered'
% FFF
\newcommand{\myfeuillet}{feuillet-$\beta$\xspace}
\WithSuffix\newcommand\myfeuillet*{feuillets-$\beta$\xspace}
\newcommand{\myform}[1]{\textbf{\sffamily\MakeUppercase{#1}}}
% GGG
\newcommand{\myGhost}{\textsc{Ghost}\xspace}
\newcommand{\myGromacs}{\textsc{Gromacs}\xspace}
\newcommand{\mygroup}[1]{$\left(\mathcal{G}_{#1}\right)$\xspace}
% HHH
\newcommand{\myHaption}{\textsc{Haption}\xspace}
\newcommand{\myHawthorne}{\myemph{Hawthorne Works}\xspace}
\newcommand{\myHBonds}{\textit{HBonds}\xspace}
\newcommand{\myhelice}{hélice-$\alpha$\xspace}
\WithSuffix\newcommand\myhelice*{hélices-$\alpha$\xspace}
\newcommand{\myhypothesis}[1]{$\left(\mathcal{H}_{#1}\right)$\xspace}
% III
\newcommand{\myIntelCore}{Intel\myregistered Core\mytrademark~2 \textsc{q9450} (\mynum[GHz]{2.66})\xspace}
% JJJ
\newcommand{\myJmol}{\textsc{Jmol}\xspace}
% LLL
\newcommand{\myLCD}{\textsc{lcd}\xspace}
\newcommand{\myLicorice}{\textit{Licorice}\xspace}
\newcommand{\myLinux}{\textsc{Linux}\xspace}
% MMM
\newcommand{\myMacOS}{Mac~\textsc{OS}\xspace}
\newcommand{\myMDDriver}{\textsc{MDDriver}\xspace}
% NNN
\newcommand{\myNewRibbon}{\textit{NewRibbon}\xspace}
\def\mynode{%
	\@ifnextchar[{\mynode@i}{\mynode@i[style=nodestyle]}%
}
\def\mynode@i[#1](#2,#3)[#4]#5{%
	\rput(#2,#3){\Rnode{#4}{\psframebox[style=nodestyle,#1]{\vphantom{pÉ}#5}}}%
}
\newcommand{\mynytrogen}{\myatom[mynytrogencolor]{A}}
\newcommand{\myNusE}{\textsc{NusE}\xspace}
\newcommand{\myNusENusG}{\textsc{NusE:NusG}\xspace}
\newcommand{\myNusG}{\textsc{NusG}\xspace}
% OOO
\newcommand{\myOmni}{\myPHANToM Omni\myregistered}% No '\xspace' because of already one in '\myregistered'
\newcommand{\myOpenHaptics}{\textsc{OpenHaptics}\mytrademark}% No '\xspace' because of already one in '\mytrademark'
\newcommand{\myoxygen}{\myatom[myoxygencolor]{O}}
% PPP
\newcommand{\myPC}{\textsc{pc}\xspace}
\newcommand{\myPDB}{\textsc{pdb}\xspace}
\newcommand{\myPDBbase}{\emph{Protein~DataBase}\xspace}
\newcommand{\myPDBlink}[2]{\href{#1}{\textsc{\MakeLowercase{#2}}}}
\newcommand{\myPHANToM}{\textsc{phant}o\textsc{m}\xspace}
\newcommand{\myPremium}{\myPHANToM Premium\myregistered}% No '\xspace' because of already one in '\myregistered'
\newcommand{\myPrion}{Prion\xspace}
\newcommand{\myPSF}{\textsc{psf}\xspace}
\newcommand{\mypvalue}{$p$-value\xspace}
\newcommand{\myPyMOL}{\textsc{p}y\textsc{mol}\xspace}
% RRR
\newcommand{\myRAM}[2][Go]{\mynum[#1]{#2} de \textsc{ram}}
\newcommand{\myRasmol}{\textsc{RasMol}\xspace}
\newcommand{\myresidue}[1]{$\left(\mathcal{R}_{#1}\right)$\xspace}
% SSS
\newcommand{\myscenario}[1]{\textsc{#1}}
\newcommand{\mySensAble}{\textsc{SensAble}\xspace}
\newcommand{\myShaddock}{\textsc{Shaddock}\xspace}
\newcommand{\mySony}{\textsc{sony}\myregistered}% No '\xspace' because of already one in '\myregistered'
\newcommand{\mySpaceNavigator}{SpaceNavigator\myregistered}% No '\xspace' because of already one in '\myregistered'
\newcommand{\mysubject}[1]{$\mathcal{S}_#1$}
\newcommand{\mysulfur}{\myatom[mysulfurcolor]{S}}
\newcommand{\mysummary}[1]{\input{files/#1}}
% TTT
\newcommand{\myTCPIP}{\textsc{tcp/ip}\xspace}
\newcommand{\myThreeD}{\textsc{3d}\xspace}
\newcommand{\mytool}[1]{\myemph{#1}}
\newcommand{\myTRPCAGE}{\textsc{trp-cage}\xspace}
\newcommand{\myTRPZIPPER}{\textsc{trp-zipper}\xspace}
% UUU
\newcommand{\myUbiquitin}{Ubiquitin\xspace}
\newcommand{\myUbuntu}{\textsc{Ubuntu}~v$10.04$\xspace}
\newcommand{\myUSB}{\textsc{usb}\xspace}
\newcommand{\myuser}[1]{$\mathcal{#1}$}
% VVV
\newcommand{\myvar}[2]{$\left(\mathcal{V}_{\mathrm{#1}#2}\right)$\xspace}
\newcommand{\myvard}[1]{\myvar{d}{#1}}
\newcommand{\myvari}[1]{\myvar{i}{#1}}
\newcommand{\myVGA}{\textsc{vga}\xspace}
\newcommand{\myVirtuose}{\textsc{Virtuose}\mytrademark~\textsc{6d}\mynum{35}--\mynum{45}\xspace}
% WWW
\newcommand{\myWindows}{\textsc{Windows}\xspace}

% Needed lengths
\newlength{\mywidth}
\newlength{\myheight}

% PSTricks style
\newpsstyle{nodestyle}{framearc=0.25,shadow=true,shadowcolor=myblue,blur=true}
\makeatother

\NeedsTeXFormat{LaTeX2e}[1999/01/01]
\ProvidesPackage{mycolor}[2011/04/29]

%%%%%%%%%%%%%%%%%%%%%%%%%%%%%
%% Declare package options %%
%%%%%%%%%%%%%%%%%%%%%%%%%%%%%
% In case of unknown options
\DeclareOption*{%
	\PackageWarning{mycolor}{Unknown option `\CurrentOption'}%
}

\ProcessOptions

%% Options to pass to packages

%% Packages to call
\RequirePackage{xcolor}

%%%%%%%%%%%%%%%%%%%%%%%
%% New configuration %%
%%%%%%%%%%%%%%%%%%%%%%%
\definecolor{mydarkred}{rgb}{0.5265625 0.25 0.25}
\definecolor{myred}{rgb}{0.7265625 0 0}
\definecolor{mylightred}{rgb}{0.9265625 0.5 0.5}
\definecolor{mylightestred}{rgb}{1 0.66 0.66}
\definecolor{mydarkblue}{rgb}{0 0 0.2265625}
\definecolor{myblue}{rgb}{0 0 0.7265625}
\definecolor{mylightblue}{rgb}{0.500 0.500 0.9265625}
\definecolor{mylightestblue}{rgb}{0.7500 0.7500 0.9265625}
\definecolor{mygreen}{rgb}{0 0.7265625 0}
\definecolor{mylightgreen}{rgb}{0.500 0.9265625 0.500}
\definecolor{mylightestgreen}{rgb}{0.7500 0.9265625 0.7500}
\definecolor{mygray}{gray}{0.6666667}

%%%%%%%%%%%%%%%%%%
%% New commands %%
%%%%%%%%%%%%%%%%%%

% End of package
\endinput

% AAA
\mynewacro{acr-AFM}%
{%
	name={\textsc{afm}},
	first={microscope à force atomique (\textsc{afm} pour \myemph{Atomic Force Microscope})},%
	plural={\textsc{afm}s},%
	firstplural={microscopes à force atomique (\textsc{afm} pour \myemph{Atomic Force Microscope})},%
	description={Microscope permettant l'observation de la topologie de la surface d'un échantillon au niveau atomique}
}
\mynewacro{acr-API}%
{%
	name={\textsc{api}},%
	first={interface de programmation (\textsc{api})},%
	plural={\textsc{api}s},%
	firstplural={interfaces de programmation (\textsc{api}s)},%
	description={\textsc{api} vient de l'anglais \myemph{Application Programming Interface} et désigne une interface avec un programme informatique}%
}
% BBB
% Be careful, because this word has no plural form, but the femina word in plural form
\mynewglos{glo-Bimanuel}%
{%
	name={bimanuel},%
	description={Qui se fait avec les deux mains},%
	plural={bimanuelle}%
}
\mynewglos{glo-Binome}%
{%
	name={binôme},%
	description={Groupe constitué de \mynum{2}~personnes},%
	plural={binômes}%
}
% CCC
\mynewacro{acr-CAO}%
{%
	name={\textsc{cao}},%
	first={conception assistée par ordinateur (\textsc{cao})},%
	description={La \textsc{cao} permet de concevoir et de tester virtuellement, à l'aide d'outils informatique, des produits manufacturés}%
}
\mynewglos{glo-ConflitDeCoordination}%
{%
	name={conflit de coordination},%
	description={Conflit entre deux sujets qui peut survenir lorsque les deux sujets tente d'accéder ou de déformer un objet au même instant},%
	plural={conflits de coordination}%
}
\mynewacro{acr-CUDA}%
{%
	name={\textsc{cuda}},%
	first={\textsc{cuda} (\myemph{Compute Unified Device Architecture})},%
	description={Technologie permettant d'utiliser l'unité graphique d'un ordinateur pour effectuer des calculs à hautes performances}%
}
\mynewglos{glo-Curseur}%
{%
	name={curseur},%
	description={Élément virtuel associé à un élément physique que le sujet manipule; il est lié à l'\myglos{glo-EffecteurTerminal}},%
	plural={curseurs}%
}
% DDD
\mynewacro{acr-DDL}%
{%
	name={\textsc{ddl}},%
	first={degré de liberté (\textsc{ddl})},%
	plural={\textsc{ddl}s},%
	firstplural={degrés de liberté (\textsc{ddl}s)},%
	description={Mouvements relatifs indépendants d'un solide par rapport à un autre}%
}
\mynewglos{glo-DockingMoleculaire}%
{%
	name={\myemph{docking} moléculaire},%
	description={Méthode permettant de déterminer l'orientation et la déformation optimale de \mynum{2}~molécules afin qu'elle s'assemble pour former un complexe de molécules stable},%
	plural={\myemph{docking} moléculaires}%
}
% EEE
\mynewglos{glo-EffecteurTerminal}%
{%
	name={effecteur terminal},%
	description={Élément physique que le sujet manipule; il est lié au \myglos{glo-Curseur} du monde virtuel},%
	plural={effecteurs terminaux}%
}
\mynewacro{acr-EVC}%
{%
	name={\textsc{evc}},%
	first={Environnement Virtuel Collaboratif (\textsc{evc})},%
	firstplural={Environnements Virtuels Collaboratifs (\textsc{evc})},%
	description={Ensemble logiciel et matériel permettant de faire interagir plusieurs utilisateurs au sein d'un même environnement; ils jouent un rôle important dans le développement de nouvelles méthodes de travail collaboratives}%
}
% HHH
\mynewglos{glo-Homoscedasticite}%
{%
	name={homoscedasticité},%
	description={Équivalent à homogénéité des variances; permet de comparer des variables aléatoires possédant des variances similaires},%
	plural={homoscedasticités}%
}
% III
\mynewacro{acr-IBPC}%
{%
	name={\textsc{ibpc}},%
	first={Institut de Biologie Physico-Chimie (\textsc{ibpc})},%
	description={Institut de recherche, géré par la fédération de recherche \textsc{frc}~\mynum{550}, étudiant les bases structurales, génétiques et physico-chimiques à leur différents niveaux d'intégration}%
}
\mynewacro{acr-IMD}%
{%
	name={\textsc{imd}},%
	first={\textsc{imd} (\myemph{Interactive Molecular Dynamics})},%
	description={Programme permettant de connecter le logiciel de visualisation moléculaire \myacro-{acr-VMD} avec le logiciel de simulation \myacro-{acr-NAMD} pour une simulation interactive en temps-réel \mycite{Stadler-1997}}%
}
\mynewacro{acr-ITAP}%
{%
	name={\textsc{itap}},%
	first={\myemph{Institut für Theoretische und Angewandte Physik} (\textsc{itap})},%
	description={Institut de Physique Théorique et Appliquée de \myname{Stuttgart} à l'origine du développement du logiciel \myacro{acr-IMD}}%
}
% LLL
\mynewacro{acr-LIMSI}%
{%
	name={\textsc{cnrs--limsi}},%
	first={Laboratoire pour l'Informatique, la Mécanique et les Sciences de l'Ingénieur (\textsc{cnrs--limsi})},%
	description={Unité Propre de Recherche du \textsc{cnrs} (\textsc{upr}~3251) associé aux universités \textsc{Paris} Sud et Pierre et Marie \textsc{Curie}}%
}
% MMM
\mynewglos{glo-Meneur}%
{%
	name={meneur},%
	description={En anglais \myemph{leader}, personne qui dirige un groupe afin d'atteindre des objectifs communs à ce groupe; c'est celui qui prend les décisions (voir aussi \myglos{glo-Suiveur})},%
	plural={meneurs}%
}
\mynewglos{glo-Monomanuel}%
{%
	name={monomanuel},%
	description={Qui se fait avec une main},%
	plural={monomanuelle}%
}
\mynewglos{glo-Monome}%
{%
	name={monôme},%
	description={\myemph{Groupe} constitué d'une unique personne},%
	plural={monômes}%
}
\mynewglos{glo-MotivationSociale}%
{%
	name={motivation sociale},%
	description={En anglais \myemph{social facilitation} \mycite{Triplett-1900}, phénomène de groupe où les personnes fournissent plus d'efforts grâce à la présence de partenaires},%
	plural={motivation sociale}%
}
\mynewacro{acr-TRM}%
{%
	name={\textsc{trm}},
	first={Théorie des Ressources Multiples (\textsc{trm})},%
	description={Cette théorie, élaborée par \mycite[author]{Wickens-1984} (\textsc{mrt} pour \myemph{Multiple Resource Theory}), propose un modèle pour la gestion des charges de travail pour un humain}
}
% NNN
\mynewacro{acr-NAMD}%
{%
	name={\textsc{namd}},%
	first={\textsc{namd} (\myemph{Scalable Molecular Dynamics})},%
	description={Programme de simulation pour la dynamique moléculaire \mycite{Phillips-2005}}%
}
% PPP
\mynewglos{glo-ParesseSociale}%
{%
	name={paresse sociale},%
	description={En anglais \myemph{social loafing} \mycite{Ringelmann-1913}, phénomène de groupe où les personnes fournissent moins d'effort pour la réalisation d'une tâche que s'ils effectuaient la tâche seuls},%
	plural={paresse sociale}%
}
\mynewacro{acr-PCV}%
{%
	name={\textsc{pcv}},
	first={Primitive Comportementale Virtuelle (\textsc{pcv})},%
	plural={\textsc{pcv}s},%
	firstplural={Primitives Comportementales Virtuelles (\textsc{pcv}s)},%
	description={Dans une application de réalité virtuelle, les activités d'un sujet peuvent toujours être décomposées en quatre comportements de base, appelés \myacro+{acr-PCV}, qui sont : observer, se déplacer, agir et communiquer \mycite{Fuchs-2006}}
}
% QQQ
\mynewglos{glo-Quadrinome}%
{%
	name={quadrinôme},%
	description={Groupe constitué de \mynum{4}~personnes},%
	plural={quadrinômes}%
}
% RRR
\mynewglos{glo-Residu}%
{%
	name={résidu},%
	description={Groupe d'atomes constituant un des blocs élémentaires d'une molécule},%
	plural={résidus}%
}
\mynewacro{acr-RMSD}%
{%
	name={\textsc{rmsd}},%
	first={\myemph{Root Mean Square Deviation} (\textsc{rmsd})},%
	description={Appelé Écart Quadratique Moyen en français, il permet -- dans le cadre de la biologie moléculaire -- de mesurer la différence entre deux déformations d'une même molécule}%
}
% SSS
\mynewglos{glo-StructureInformelle}%
{%
	name={structure informelle},%
	description={Groupe de personnes sans structures ni hiérarchie},%
	plural={structures informelles}%
}
\mynewglos{glo-Suiveur}%
{%
	name={suiveur},%
	description={En anglais \myemph{follower}, personne qui se laisse diriger dans un groupe afin d'atteindre des objectifs communs à ce groupe; c'est une personne qui ne prend pas de décision (voir aussi \myglos{glo-Meneur})},%
	plural={suiveurs}%
}
\mynewacro{acr-SUS}%
{%
	name={\textsc{sus}},%
	first={\textsc{sus} (\myemph{System Usability Scale})},%
	description={Échelle de notation entre \mynum{0} et \mynum{100} proposée par \mycite[author]{Brooke-1996} permettant d'évaluer l'utilisabilité d'un système}%
}
% TTT
\mynewglos{glo-Tetranome}%
{%
	name={tetranôme},%
	description={Groupe constitué de \mynum{4}~personnes},%
	plural={tetranômes}%
}
\mynewglos{glo-Trinome}%
{%
	name={trinôme},%
	description={Groupe constitué de \mynum{3}~personnes},%
	plural={trinômes}%
}
% UUU
\mynewacro{acr-UDP}%
{%
	name={\textsc{udp}},
	first={\textsc{udp} (\myemph{User Datagram Protocol} pour protocole de datagramme utilisateur)},%
	plural={\textsc{afm}s},%
	firstplural={\textsc{udp} (\myemph{User Datagram Protocol} pour protocole de datagramme utilisateur)},%
	description={c'est un des principaux protocole de télécommunication sur internet ; il a pour distinction de ne pas vérifier l'intégrité des données transmises}
}
\mynewacro{acr-UML}%
{%
	name={\textsc{uml}},%
	first={\textsc{uml} (\myemph{Unified Modeling Language})},%
	description={C'est un langage graphique de modélisation utilisé principalement en génie logiciel}%
}
% VVV
\mynewglos{glo-VariableDependante}%
{%
	name={variable dépendante},%
	description={Facteur mesuré sur une expérimentation (nombre de sélections, trajectoire, \myetc); ces variables sont influencées par les \myglos*{glo-VariableIndependante}},%
	plural={variables dépendantes}%
}
\mynewglos{glo-VariableIndependante}%
{%
	name={variable indépendante},%
	description={Facteur pouvant varier et être manipuler sur une expérimentation (nombre de participants, tâche, \myetc); ces variables vont avoir une incidence sur les \myglos*{glo-VariableDependante}},%
	plural={variables indépendantes}%
}
\mynewglos{glo-VariableInterSujets}%
{%
	name={variable inter-sujets},%
	description={Variables pour lesquelles les sujets sont confrontés à une et une seule des modalités de la variable},%
	plural={variables inter-sujets}%
}
\mynewglos{glo-VariableIntraSujets}%
{%
	name={variable intra-sujets},%
	description={Variables pour lesquelles les sujets sont confrontés à toutes les modalités de la variable},%
	plural={variables intra-sujets}%
}
\mynewacro{acr-VMD}%
{%
	name={\textsc{vmd}},%
	first={\textsc{vmd} (\myemph{Visual Molecular Dynamics})},%
	description={Programme de visualisation moléculaire \mycite{Humphrey-1996}}%
}
\mynewacro{acr-VRPN}%
{%
	name={\textsc{vrpn}},%
	first={\textsc{vrpn} (\myemph{Virtual Reality Protocol Network})},%
	description={Logiciel permettant de connecter différents périphériques de réalité virtuelle à une même application sous forme d'une architecture client/serveur \mycite{Taylor-II-2001}}%
}


\hypersetup{%
	pdftitle={Interactions haptiques collaboratives pour la manipulation moléculaire},%
	pdfauthor={Jean SIMARD},%
	pdfsubject={Mémoire de thèse en informatique},%
	pdfdisplaydoctitle=true,%
	pdflang={FR-fr}%
}
\addglobalbib[datatype=bibtex]{biblio.bib}

\title{Interactions haptiques collaboratives pour la manipulation moléculaire}
\author{Jean~\myname{Simard}}
\documenttype{Thèse en Informatique}
\university{École Doctorale d'Informatique de Paris Sud}
\date{\mydate[datestyle=long]{01/12/2011}}
\jury{%
	Martin & \myname{DUPONT} & (rapporteur) & Directeur de recherche au \myacro-{acr-LIMSI} \\
	Martin & \myname{DUPOND} & (examinateur) & Directeur de recherche au \myacro-{acr-LIMSI}}

\begin{document}
	\frontmatter
	\maketitle
	\mytoc
	\mylof
	\mylot
	\mylotodo
	\mainmatter
	\begin{mypart}[prt-LeSujet]{Le sujet}
		\begin{mychapter}[cha-Introduction]{Introduction}
			\begin{mysection}[sec-EtatDeLArt]{État de l'art}
			\end{mysection}
			\begin{mysection}[sec-Contexte]{Contexte}
				\begin{mysubsection}[sse-LAmarrageMoleculaire]{L'\myglos{glo-AmarrageMoleculaire}}
					Le contexte de l'expérimentation est l'\myglos{glo-AmarrageMoleculaire} plus communément nommé \myglos{glo-DockingMoleculaire}.
					Ce processus implique une analyse et une manipulation complexe reposant sur plusieurs expertises.
					Il est basé sur une décomposition en trois niveaux de modélisation, traités du niveau le plus grossier au niveau le plus fin :
					\begin{description}
						\item[Niveau inter-moléculaire] Cette déformation au niveau macro-moléculaire applique des transformations de grande amplitude sur chaque molécule.
							L'objectif est de trouver la meilleure concordance entre les molécule en terme de position et d'orientation.
						\item[Niveau intra-moléculaire] Cette déformation au niveau moléculaire fait suite à la déformation inter-moléculaire.
							L'amarrage de ces deux molécules (ou plus) introduit de nombreuses interfaces qui doivent être optimisées en fonction de critères variés (la complémentarité des surfaces, les forces électrostatiques, les forces de \myname[van der]{Waals} \mycite{Muller-1994}, \myetc).
						\item[Niveau atomique] Cette déformation très fine va chercher à optimiser la position des atomes au niveau de l'interface.
							L'intérêt de cette étape sera portée sur plusieurs types d'interaction (les ponts hydrogènes, les zones hydrophobiques et hydrophylliques, les ponts salins, \myetc).
					\end{description}

					Pour chacun de ces différents niveaux, le processus de manipulation est similaire et peut être séparé en sous-tâches :
					\begin{description}
						\item[Recherche] Cette tâche concerne l'identification et la recherche d'une cible (atome, \myglos{glo-Residu}, hélices $\alpha$, feuillets $\beta$, \myetc) en fonction de critères multiples (articulations, bilan énergétique, régions hydrophobique, \myetc).
						\item[Sélection] Une fois la cible trouvée, la tâche consiste à accéder puis à sélectionner la cible par l'intermédiaire d'un périphérique d'entrée (une souris, une interface haptique, \myetc).
						\item[Déformation] La tâche consiste à déformer la structure en manipulant la cible précédemment sélectionnée, que ce soit au niveau inter-moléculaire, intra-moléculaire ou atomique.
							L'objectif inhérent à cette tâche et d'atteindre l'objectif fixé (par exemple, minimiser l'énergie totale du système).
						\item[Évaluation] Cette dernière partie va évaluer le travail précédemment réalisé en observant différents indicateurs (énergie potentielle, énergie électrostatique, complémentarité des surfaces, \myetc).
							En fonction de la synthèse des résultats de cette dernière phase, un nouveau cycle pourra recommencer (recherche, sélection, déformation, évaluation, \myetc).
					\end{description}

				\end{mysubsection}
			\end{mysection}
		\end{mychapter}
	\end{mypart}
	\begin{mypart}[prt-EtudeDuTravailCollaboratif]{Étude du travail collaboratif}
		\begin{mychapter}[cha-LaRechercheCollaborative]{La recherche collaborative}
			\begin{mysection}[sec-exp1-Presentation]{Présentation}
				\begin{mysubsection}[sse-exp1-Objectifs]{Objectifs}
					Dans cette première expérimentation, nous proposons d'étudier la première des quatre sous-tâches élémentaires \mysubsectionref*{sse-LAmarrageMoleculaire}: la \myemph{recherche}.
					Cette sous-tâche est cruciale car elle a un impact important sur les sous-tâches suivantes.
					Les difficultés liées à la complexité de l'environnement virtuel moléculaire seront étudiées à travers l'étude de cette sous-tâche.

					Cette première expérimentation a pour objectif principal de comparer un \myglos{glo-Monome} et un \myglos{glo-Binome}.
					Deux facteurs seront étudier lors de cette comparaison.

					Le premier facteur concerne les performances.
					Les performances représente à la fois le temps total pour réaliser la tâche mais aussi les ressources mises en place pour accéder à ce résultat.
					Un \myglos{glo-Binome} sera-t-il plus performant qu'un \myglos{glo-Monome} ?

					Le second facteur concerne les méthodes et les stratégies de travail.
					C'est principalement l'évolution de ces stratégies au sein des \myglos*{glo-Binome} qui focalisera notre attention.
					Le travail en \myglos{glo-Binome} permettra de mettre en avant différentes stratégies de travail discriminées en fonction de la communication, des espaces de travail, de la répartition des tâches, \myetc
				\end{mysubsection}
				\begin{mysubsection}[sse-exp1-Hypotheses]{Hypothèses}
					\begin{myparagraph}[par-exp1-AmeliorationDesPerformances]{\myhypothesis{1} Amélioration des performances en \myglos{glo-Binome}}
						La première hypothèse est une amélioration des performances pour les \myglos*{glo-Binome} comparés aux \myglos*{glo-Monome}.
						Cette amélioration se traduira principalement par une réalisation de la même tâche en un temps réduit.
					\end{myparagraph}
					\begin{myparagraph}[par-exp1-StrategiesVariablesEnFonctionDesBinomes]{\myhypothesis{2} Stratégies variables en fonction des \myglos*{glo-Binome}}
						Cette second hypothèse concerne uniquement les \myglos*{glo-Binome} et suppose que les stratégies adoptés seront différentes en fonction des \myglos*{glo-Binome}.
						Cette différence sera liée aux différentes personnalités et aux différentes affinités au sein du \myglos{glo-Binome}.
					\end{myparagraph}
					\begin{myparagraph}[par-exp1-LesSujetsPreferentLeTravailEnBinome]{\myhypothesis{3} Les sujets préfèrent le travail en \myglos{glo-Binome}}
						La troisième hypothèse est de l'ordre du qualitatif.
						Elle s'intéresse aux conditions de travail en \myglos{glo-Binome}.
						L'hypothèse est basée \myglos*{glo-Binome} sur l'effet stimulant de travailler à plusieurs mais aussi sur l'aspect humain de ne pas rester seul pour effectuer une tâche répétitive.
						Il est cependant important que chaque sujet au sein du \myglos{glo-Binome} se considère utile à la réalisation de la tâche.
					\end{myparagraph}
				\end{mysubsection}
			\end{mysection}
			\begin{mysection}[sec-exp1-DispositifExperimentalEtMateriel]{Dispositif expérimental et matériel}
				L'\myacro{acr-EVC} utilisé est illustré sur la \myfigureref{fig-exp1-IllustrationDuDispositifExperimental}.
				L'\myacro{acr-EVC} propose une visualisation partagée (vue publique) à l'aide d'un vidéoprojecteur sur un grand écran.
				Le ou les sujets font face à l'écran avec à leur disposition :
				\begin{itemize}
					\item un interface haptique de manipulation \mytool{grab};
					\item deux interfaces haptiques de déformation \mytool{tug}.
				\end{itemize}

				\begin{myfigure}
					\begin{mysubfigure}
						\myimage[width=0.49\textwidth]{exp1-schema}
						\mysubcaption[fig-exp1-IllustrationDuDispositifExperimental-SchemaDuDispositifExperimental]{Schéma du dispositif expérimental}
					\end{mysubfigure}
					\begin{mysubfigure}
						\myimage[width=0.49\textwidth]{exp1-photo}
						\mysubcaption[fig-exp1-IllustrationDuDispositifExperimental-PhotographieDuDispositifExperimental]{Photographie du dispositif expérimental}
					\end{mysubfigure}
					\mycaption[fig-exp1-IllustrationDuDispositifExperimental]{Illustration du dispositif expérimental}
				\end{myfigure}

				Les sujets ont la possibilité de communiquer entre eux sans restriction.
				Pour les \myglos*{glo-Monome}, le sujet peut utiliser chaque outil comme il le souhaite.
				Pour les \myglos*{glo-Binome}, chaque sujet se voit attribuer un outil de déformation \mytool{tug}.
				L'outil de manipulation \mytool{grab} sera attribuer à un seul des deux sujets après une négociation au sein du \myglos{glo-Binome}.
				Le sujet désigné pour l'outil de manipulation \mytool{grab} le sera pour toute la durée de l'expérimentation.

				Un micro de bureau est placé en face des deux sujets afin de capter toutes les communications orales.
				L'enregistrement, réalisé à l'aide du logiciel Audacity, débute à la fin de la phase d'entraînement.

				Pour les détails techniques concernant la plate-forme et les outils de manipulation et de déformation, se reporter au \mychapterref{app-ShaddockCollaborativeVirtualEnvironmentForMolecularDesign}.
			\end{mysection}
			\begin{mysection}[sec-exp1-Methode]{Méthode}
				\begin{mysubsection}[sse-exp1-Sujets]{Sujets}
					\mynum{24}~sujets (\mynum{4}~femmes et \mynum{20}~hommes) avec une moyenne d'âge de $\mu = 27.8$ ($\sigma = 7.19$) ont participés à cette expérimentation.
					Ils ont tous été recrutés au sein du laboratoire \myacro{acr-LIMSI} et sont chercheurs ou assistants de recherche dans les domaines suivants~:
					\begin{itemize}
						\item linguistique et traitement automatique de la parole;
						\item réalité virtuelle et système immersifs;
						\item audio-acoustique.
					\end{itemize}
					Ils ont tous le français comme langue principale.
					Aucun participant n'a de déficience visuelle (ou corrigée le cas échéant) ni de déficience audio.

					Chaque participants est complètement naïf concernant les détails de l'expérimentation.
					Une explication détaillée de la procédure expérimentale leur est donnée au commencement de l'expérimentation mais en omettant l'objectif de l'étude.
				\end{mysubsection}
				\begin{mysubsection}[sec-exp1-Variables]{Variables}
					\begin{mysubsubsection}[sss-exp1-VariablesIndependantes]{Variables indépendantes}
						\begin{myparagraph}[par-exp1-NombreDeSujets]{\myvari{1} Nombre de sujets}
							La première \myglos{glo-VariableIndependante} est une \myglos{glo-VariableIntraPopulation}, c'est-à-dire que tous les sujets sont expérimentés dans toutes les modalités de cette variable.
							\myvari{1} possède deux valeurs possibles: \og un sujet (\mycf \myemph{\myglos{glo-Monome}}) \fg ou \og deux sujets (\mycf \myemph{\myglos{glo-Binome}}) \fg.
							Le sujets seuls et les sujets en couples ont à leur disposition deux outils haptiques de déformation et un outil haptique de manipulation.
							Pour les \myglos*{glo-Binome}, seulement un des deux sujets est désigné pour l'utilisation exclusive de l'outil de manipulation.
							\mynum{24}~\myglos*{glo-Monome} et \mynum{12}~\myglos*{glo-Binome} ont été testés ce qui fait deux fois plus de \myglos*{glo-Monome} que de \myglos*{glo-Binome}.
						\end{myparagraph}
						\begin{myparagraph}[par-exp1-ResiduRecherche]{\myvari{2} Résidu recherché}
							La seconde \myglos{glo-VariableIndependante} est une \myglos{glo-VariableIntraPopulation}.
							\myvari{2} concerne les \myglos*{glo-Residu} recherchés qui sont au nombre de \mynum{10} répartis à part égale dans deux molécules \mytableref*{tab-exp1-ListeDesResidusRecherches}.
						\end{myparagraph}
					\end{mysubsubsection}
					\begin{mysubsubsection}[sec-exp1-VariablesDependantes]{Variables dépendantes}
						\begin{myparagraph}[par-exp1-LeTempsDeCompletion]{\myvard{1} Le temps de complétion}
							Ce temps est le temps total pour réaliser la tâche demandée, c'est-à-dire trouver le \myglos{glo-Residu} et l'extraire de la molécule.
							Ce temps est divisé en deux phases bien distinctes :
							\begin{description}
								\item[La recherche] C'est la phase pendant laquelle les sujets cherchent le \myglos{glo-Residu}.
									Cette recherche peut être simplement visuelle en orientant et en déplaçant la molécule mais elle peut aussi amener les sujets à déformer la molécule afin d'explorer les \myglos{glo-Residu} inaccessibles.
								\item[La sélection] La phase de sélection débute dès l'instant où un des deux sujets a trouvé le \myglos{glo-Residu}.
									Elle est constitué d'une phase de sélection puis d'une phase d'extraction.
							\end{description}
							Il n'y a pas de limite de temps pour réaliser la tâche.
						\end{myparagraph}
						\begin{myparagraph}[par-exp1-LaDistanceEntreLesEspacesDeTravail]{\myvard{2} La distance entre les espaces de travail}
							Cette distance est la distance moyenne entre les deux \myglos*{glo-EffecteurTerminal} présents durant l'expérimentation.
							Cette distance représente donc une distance physique du monde réel, pas une distance virtuelle.
							Elle est de l'ordre du centimètre.
						\end{myparagraph}
						\begin{myparagraph}[par-exp1-LesCommunicationsOrales]{\myvard{3} Les communications orales}
							L'enregistrement audio permet de mesurer la quantité de temps de parole pendant chaque tâche de l'expérimentation.
							Ces mesures discrimine la phase de recherche de la phase de sélection (voir \myvard{1}) comme indiqué plus précisément sur la \myfigureref{fig-exp1-SchemaDesPhasesDeLaCommunicationVerbale}.

							\begin{myfigure}
								\psset{unit=0.1\textwidth} % Fill entirely the page width
								\begin{myps}(0,-1.75)(10,1.5)
									\psset{linewidth=1pt,linecolor=black}%
									\psframe(0,-0.5)(10,0.5)%
									\psset{fillstyle=solid}%
									\psframe[fillcolor=mylightblue](0,-0.5)(6,0.5)%
									\psframe[fillcolor=mylightred](6,-0.5)(10,0.5)%
									\psbrace[ref=lC,rot=-90,nodesepA=-3,nodesepB=-0.25](6,0.5)(0,0.5){%
										\parbox{6\psxunit}{%
											\centering\textcolor{myblue}{Temps de recherche}%
										}%
									}%
									\psbrace[ref=lC,rot=-90,nodesepA=-2,nodesepB=-0.25](10,0.5)(6,0.5){%
										\parbox{4\psxunit}{%
											\centering\textcolor{myred}{Temps de sélection}%
										}%
									}%
									\psframe[fillcolor=myblue](1,-0.5)(1.5,0.5)
									\psframe[fillcolor=myblue](3,-0.5)(4.5,0.5)
									\psframe[fillcolor=myblue](4.8,-0.5)(5,0.5)
									\psframe[fillcolor=myred](6.5,-0.5)(7.5,0.5)
									\psframe[fillcolor=myred](8,-0.5)(8.25,0.5)
									\pnode(1.25,-0.5){verbal1}
									\pnode(3.75,-0.5){verbal2}
									\pnode(4.9,-0.5){verbal3}
									\pnode(7,-0.5){verbal4}
									\pnode(8.125,-0.5){verbal5}
									\rput(5,-1.5){%
										\Rnode{verbal}{%
											\psframebox[linestyle=none]{\centering Communication verbale}%
										}%
									}%
									\psset{linearc=0.1,angleA=-90}
									\ncdiagg{<-}{verbal1}{verbal}
									\ncdiagg{<-}{verbal2}{verbal}
									\ncdiagg{<-}{verbal3}{verbal}
									\ncdiagg{<-}{verbal4}{verbal}
									\ncdiagg{<-}{verbal5}{verbal}
								\end{myps}
								\mycaption[fig-exp1-SchemaDesPhasesDeLaCommunicationVerbale]{Schéma des phases de la communication verbale}
							\end{myfigure}
						\end{myparagraph}
						\begin{myparagraph}[par-exp1-LAffiniteEntreLesSujets]{\myvard{4} L'affinité entre les sujets}
							Le degré d'affinité -- concernant uniquement les \myglos*{glo-Binome} -- est compris entre \mynum{1} et \mynum{5} selon les critères suivants :
							\begin{enumerate}
								\item Les sujets ne se connaissent pas;
								\item Les sujets travaillent dans la même entreprise, le même laboratoire;
								\item Les sujets travaillent dans la même équipe;
								\item Les sujets travaillent dans le même bureau;
								\item Les sujets sont amis.
							\end{enumerate}
						\end{myparagraph}
						\begin{myparagraph}[par-exp1-ReponsesQualitatives]{\myvard{5} Réponses qualitatives}
							Un questionnaire est proposé à tous les sujets.
							Il est constitué de plusieurs questions (notées sur échelle de \mycite[author]{Likert-1932} à cinq niveaux).

							Le questionnaire est le suivant (les questions sont posées à chaque sujet dans le cas du \myglos{glo-Binome}) :
							\begin{enumerate}
								\item Dans quelle configuration vous êtes-vous senti le plus efficace : \myemph{seul} ou \myemph{en collaboratif} ?
								\item Vous êtes vous senti utile dans la configuration collaborative (par opposition à pénalisant) ?
								\item Pensez-vous avoir une position de meneur dans la configuration collaborative ?
								\item Comment évalueriez-vous votre taux de communication\dots{}
									\begin{itemize}
										\item verbale ?
										\item gestuelle ?
										\item virtuelle ?
									\end{itemize}
							\end{enumerate}

							Concernant la communication, les communications verbales concernent tous les échanges, dialogues exposés par la voix.
							La communication gestuelle représente les gestes que les sujets peuvent effectuer dans le monde réel pour expliquer, désigner ou pour tout autre explication à son partenaire.
							Enfin, la communication virtuelle concerne les informations données au partenaire par l'intermédiaire de l'environnement virtuel (par exemple, une désignation avec le curseur).
						\end{myparagraph}
					\end{mysubsubsection}
				\end{mysubsection}
				\begin{mysubsection}[sse-exp1-Tache]{Tâche}
					La tâche proposée est la recherche et la sélection dans un \myacro{acr-EVC} sur des molécules complexes.
					Les motifs à rechercher dans les structures moléculaires sont les \myglos*{glo-Residu} du \mytableref{tab-exp1-ListeDesResidusRecherches}.
					Une fois le \myglos{glo-Residu} trouvé, les sujets doivent le sélectionner puis l'extraire hors de la sphère virtuelle englobant la molécule.
					Les sujets possèdent deux outils pour trouver, sélectionner puis extraire ces motifs :
					\begin{itemize}
						\item ils peuvent explorer la molécule en la déplaçant ou en la tournant à l'aide de l'outil \mytool{grab};
						\item ils peuvent déformer la molécule à l'aide de l'outil \mytool{tug}.
					\end{itemize}

					\begin{mytable}
						\mycaption[tab-exp1-ListeDesResidusRecherches]{Liste des résidus recherchés}
						\setlength{\myheight}{10ex}
						\newcommand{\mypatternpicture}[1]{\myimage[width=\myheight]{exp1-#1}}
						\begin{mysubtable}
							\mysubcaption[tab-exp1-ListeDesResidusRecherches-ResidusSurLaMoleculeTRPCAGE]{Residus sur la molécule \myTRPCAGE}
							\begin{mytabular}[0.49\textwidth]{^C-C}
								\mytoprule
								\myrowstyle{\bfseries}
								Résidu & Image \\
								\mymiddlerule
								\myresidue{1} & \mypatternpicture{pattern1} \\
								\myresidue{2} & \mypatternpicture{pattern2} \\
								\myresidue{3} & \mypatternpicture{pattern3} \\
								\myresidue{4} & \mypatternpicture{pattern4} \\
								\myresidue{5} & \mypatternpicture{pattern5} \\
								\mybottomrule
							\end{mytabular}
						\end{mysubtable}
						\begin{mysubtable}
							\mysubcaption[tab-exp1-ListeDesResidusRecherches-ResidusSurLaMoleculePrion]{Residus sur la molécule \myPrion}
							\begin{mytabular}[0.49\textwidth]{^C-C}
								\mytoprule
								\myrowstyle{\bfseries}
								Résidu & Image \\
								\mymiddlerule
								\myresidue{6}  & \mypatternpicture{pattern6}  \\
								\myresidue{7}  & \mypatternpicture{pattern7}  \\
								\myresidue{8}  & \mypatternpicture{pattern8}  \\
								\myresidue{9}  & \mypatternpicture{pattern9}  \\
								\myresidue{10} & \mypatternpicture{pattern10} \\
								\mybottomrule
							\end{mytabular}
						\end{mysubtable}
					\end{mytable}

					La première molécule est couramment nommée \myTRPCAGE \mycite{Neidigh-2002} et a pour identifiant \myPDB \myPDBlink{http://www.rcsb.org/pdb/explore/explore.do?structureId=1L2Y}{1L2Y} sur la \myPDBbase\footnote{\url{http://www.pdb.org/}}.
					La seconde molécule nommée \myPrion \mycite{Christen-2009} avec l'identifiant \myPDB \myPDBlink{http://www.rcsb.org/pdb/explore/explore.do?structureId=2KFL}{2KFL}.
					Cinq \myglos*{glo-Residu} sont présents sur chaque molécule \myfigureref*{fig-exp1-RepartitionDesResidusSurLesMolecules} et chacun présente différents niveaux de complexité.
					Les critères de complexité, résumés dans le \mytableref{tab-exp1-ParametresDeComplexiteDesResidus}, sont les suivants :
					\begin{description}
						\item[Position] La position du \myglos{glo-Residu} peut se trouver sur le pourtour de la molécule, en position \myemph{externe} ou à l'intérieur, au milieu de l'amas d'atome (position \myemph{interne}).
							Un \myglos{glo-Residu} en position externe ne nécessite pas de déformer la molécule pour le trouver et l'atteindre contrairement à un \myglos{glo-Residu} en position interne qui sera plus complexe d'accès.
						\item[Forme] La forme du \myglos{glo-Residu} influe énormément sur la complexité de la recherche.
							On distingue trois formes différentes :
							\begin{description}
								\item[Chaîne] Un enchaînement d'atomes seuls les atomes d'hydrogène sont de part et d'autres de cet enchaînement.
								\item[Cercle] Une chaîne d'atomes de carbone ou d'azote qui boucle sur elle-même.
								\item[Étoile] Séries de chaînes d'atomes toutes reliées sur un atome central (la plupart du temps, un atome de carbone).
							\end{description}
						\item[Couleurs] Les atomes sont colorés en fonction de leur nature (rouge pour l'oxygène, blanc pour l'hydrogène, \myetc).
							Les atomes qui sont rares seront donc rapidement trouvés grâce à leur couleur différente.
							Par contre, les atomes nombreux (comme les hydrogènes ou les carbones) seront plus difficiles à filtrer à cause de leur nombre important.
						\item[Similarité] Certains \myglos*{glo-Residu} à chercher sont très similaires à d'autres \myglos*{glo-Residu} également présents sur la molécule.
							De par leur similarité, ils vont mobilier la recherche sur des \myglos*{glo-Residu} incorrects.
					\end{description}

					\begin{myfigure}
						\newcommand{\schemafactor}{0.20}
						\newlength{\schemaunit}\setlength{\schemaunit}{\schemafactor\textwidth}
						\psset{unit=\schemaunit}
						\mycaption[fig-exp1-RepartitionDesResidusSurLesMolecules]{Répartition des résidus sur les molécules}
						\begin{myps}(-2.5,-3)(2.5,3)
							\rput(0,1.75){%
								\myimage[height=2\schemaunit,angle=90]{exp1-trp-cage}}
							\rput(0,-1.25){%
								\myimage[height=2\schemaunit,angle=90]{exp1-prion}}
							\rput(-1.5,2){%
								\myimage[height=\schemaunit]{exp1-pattern1}}
							\rput(1.5,2){%
								\myimage[width=\schemaunit]{exp1-pattern3}}
							\rput(1.5,-0){%
								\myimage[width=\schemaunit]{exp1-pattern2}}
							\rput(-1.5,-0){%
								\myimage[width=\schemaunit]{exp1-pattern4}}
							\rput(-1.5,-2){%
								\myimage[width=\schemaunit]{exp1-pattern5}}
							\rput(1.5,-2){%
								\myimage[height=\schemaunit]{exp1-pattern6}}

							\psset{framesize=1 1}
							\fnode(-1.5,2){P1}
							\uput[90](-1.5,2.5){\myresidue{1}}
							\fnode(1.5,2){P38}
							\uput[90](1.5,2.5){\myresidue{3} et \myresidue{8}}
							\fnode(1.5,-0){P27}
							\uput[90](1.5,0.5){\myresidue{2} et \myresidue{7}}
							\fnode(-1.5,-0){P49}
							\uput[90](-1.5,0.5){\myresidue{4} et \myresidue{9}}
							\fnode(-1.5,-2){P510}
							\uput[90](-1.5,-1.5){\myresidue{5} et \myresidue{10}}
							\fnode(1.5,-2){P6}
							\uput[90](1.5,-1.5){\myresidue{6}}

							\psset{linecolor=myred}
							\cnode(0.3,1.5){0.2}{TRPP1}
							\cnode(0.15,2){0.2}{TRPP38}
							\cnode(-0.1,1.25){0.2}{TRPP27}
							\cnode(-0.5,2.2){0.2}{TRPP49}
							\cnode(-0.65,1.25){0.2}{TRPP510}
							\ncline{-}{P1}{TRPP1}
							\ncline{-}{P38}{TRPP38}
							\ncline{-}{P27}{TRPP27}
							\ncline{-}{P49}{TRPP49}
							\ncline{-}{P510}{TRPP510}

							\psset{linecolor=myblue}
							\cnode(0.4,0.2){0.2}{PrionP38}
							\cnode(0.6,-2.8){0.2}{PrionP27}
							\cnode(0.2,-0.8){0.2}{PrionP49}
							\cnode(-0.7,-1.7){0.2}{PrionP510}
							\cnode(0.0,-1.4){0.2}{PrionP6}
							\ncline{-}{P38}{PrionP38}
							\ncline{-}{P27}{PrionP27}
							\ncline{-}{P49}{PrionP49}
							\ncline{-}{P510}{PrionP510}
							\ncline{-}{P6}{PrionP6}
						\end{myps}
					\end{myfigure}

					\begin{mytable}
						\newcommand{\myatom}[2]{%
							{%
								\bfseries%
								\sffamily%
								\textcolor{#2}{\MakeUppercase{#1}}%
								\xspace%
							}%
						}
						\newcommand{\mycarbon}{\myatom{C}{mycarboncolor}}
						\newcommand{\mynytrogen}{\myatom{A}{mynytrogencolor}}
						\newcommand{\myoxygen}{\myatom{O}{myoxygencolor}}
						\newcommand{\mysulfur}{\myatom{S}{mysulfurcolor}}
						\newcommand{\myatomincolor}[3]{\csname my#1\endcsname{}{}#2 en \myemph{#3}}
						\mycaption[tab-exp1-ParametresDeComplexiteDesResidus]{Paramètres de complexité des résidus -- \myatomincolor{carbon}{arbone}{cyan}, \myatomincolor{nytrogen}{zote}{bleu}, \myatomincolor{oxygen}{xygène}{rouge} et \myatomincolor{sulfur}{oufre}{jaune}}
						\begin{mytabular}{^C-C-C-C-C}
							\mytoprule
							\myrowstyle{\bfseries}
							Résidu & Position & Forme & Couleurs & Similarité \\
							\mymiddlerule[\heavyrulewidth]
							\myresidue{1}  & Interne & Cercle & \mynum{8}~\mycarbon, \mynum{1}~\mynytrogen & Non \\
							\mymiddlerule
							\myresidue{2}  & Interne & Étoile & \mynum{1}~\mycarbon, \mynum{3}~\mynytrogen & Non \\
							\mymiddlerule
							\myresidue{3}  & Interne & Cercle & \mynum{6}~\mycarbon, \mynum{1}~\myoxygen   & Non \\
							\mymiddlerule
							\myresidue{4}  & Externe & Chaîne & \mynum{4}~\mycarbon                        & Non \\
							\mymiddlerule
							\myresidue{5}  & Externe & Chaîne & \mynum{4}~\mycarbon, \mynum{1}~\mynytrogen & Non \\
							\mymiddlerule[\heavyrulewidth]
							\myresidue{6}  & Interne & Chaîne & \mynum{2}~\mycarbon, \mynum{2}~\mysulfur   & Non \\
							\mymiddlerule
							\myresidue{7}  & Externe & Étoile & \mynum{1}~\mycarbon, \mynum{3}~\mynytrogen & Non \\
							\mymiddlerule
							\myresidue{8}  & Externe & Cercle & \mynum{6}~\mycarbon, \mynum{1}~\myoxygen   & Non \\
							\mymiddlerule
							\myresidue{9}  & Interne & Chaîne & \mynum{4}~\mycarbon                        & Oui \\
							\mymiddlerule
							\myresidue{10} & Interne & Chaîne & \mynum{4}~\mycarbon, \mynum{1}~\mynytrogen & Oui \\
							\mybottomrule
						\end{mytabular}
					\end{mytable}
				\end{mysubsection}
				\begin{mysubsection}[sse-exp1-Procedure]{Procédure}
					Pour débuter cette expérimentation, les sujets sont confrontés à un exemple sur la molécule \myTRPZIPPER \mycite{Christen-2009} avec l'identifiant \myPDB \myPDBlink{http://www.rcsb.org/pdb/explore/explore.do?structureId=2KFL}{2KFL}.
					Pendant la phase d'entraînement, les outils sont introduits et expliqués un par un.
					Chaque sujet a la possibilité de tester les outils et peut questionner l'expérimentateur.

					Un résumé du protocole expérimental est exprimé dans le \mytableref{tab-exp1-SyntheseDeLaProcedureExperimentale}.

					\begin{mytable}
						\mycaption[tab-exp1-SyntheseDeLaProcedureExperimentale]{Synthèse de la procédure expérimentale}
						\newcommand{\mytitlecolumn}[2]{%
							\multirow{#1}*{%
								\begin{minipage}{6em}%
									\raggedleft #2%
								\end{minipage}%
							}
						}
						\newlength{\exponefirstcolumn}
						\newlength{\exponesecondcolumn}
						\setlength{\exponefirstcolumn}{7em}
						\setlength{\exponesecondcolumn}{\textwidth}
						\addtolength{\exponesecondcolumn}{-\exponefirstcolumn}
						\addtolength{\exponesecondcolumn}{-4\tabcolsep}
						\begin{mytabular}{>{\bfseries}p{\exponefirstcolumn}p{\exponesecondcolumn}}
							\mytoprule
							\mytitlecolumn{1}{Tâche}                  & Recherche et sélection de motifs \\
							\mymiddlerule[\heavyrulewidth]
							\mytitlecolumn{1}{Hypothèses}             & \myhypothesis{1} Amélioration des performances en \myglos{glo-Binome} \\
																		& \myhypothesis{2} Stratégies variables en fonction des \myglos*{glo-Binome} \\
																		& \myhypothesis{3} Les sujets préfèrent le travail en \myglos{glo-Binome} \\
							\mymiddlerule
							\mytitlecolumn{2}{Variable indépendantes} & \myvari{1} Nombre de sujets \\
																		& \myvari{2} Résidu à chercher \\
							\mymiddlerule
							\mytitlecolumn{4}{Variable dépendantes}   & \myvard{1} Temps de complétion \\
																		& \myvard{2} Distance entre les espaces de travail \\
																		& \myvard{3} Communication orales \\
																		& \myvard{4} Affinités entre les sujets \\
							\mymiddlerule[\heavyrulewidth]
							\multicolumn{2}{c}{%
								\begin{tabular}{^C-C-C}
									\myrowstyle{\bfseries}
									Condition \mycondition{1} & Condition \mycondition{2} & Condition \mycondition{3} \\
									\mymiddlerule
									Sujet~$A$ & Sujet~$A$ & Sujet~$A$ et $B$ \\
									\mynum{10}~résidus & \mynum{10}~résidus & \mynum{10}~résidus \\
									\mymiddlerule
									Sujet~$B$ & Sujet~$A$ et $B$ & Sujet~$A$ \\
									\mynum{10}~résidus & \mynum{10}~résidus & \mynum{10}~résidus \\
									\mymiddlerule
									Sujet~$A$ et $B$ & Sujet~$B$ & Sujet~$B$ \\
									\mynum{10}~résidus & \mynum{10}~résidus & \mynum{10}~résidus \\
								\end{tabular}
							} \\
							\mybottomrule
						\end{mytabular}
					\end{mytable}
				\end{mysubsection}
			\end{mysection}
			\begin{mysection}[sec-exp1-Resultats]{Résultats}
				\begin{myfigure}
					\psset{xunit=0.1\textwidth,yunit=0.01cm}
					\begin{myps}(0,-100)(10,400)
						\myaxes(0,10){x-axis}(0,400)[50]{y-axis}
						\readpsbardata[header=true,chartstyle=boxplot,barcolsep=0.16,barsep=0]{\data}{files/exp1-time-residue.csv}
						\psbarchart[header=true,chartstyle=boxplot,barcolsep=0.16,barsep=0]{\data}
					\end{myps}
					\mycaption{Test}
				\end{myfigure}
			\end{mysection}
		\end{mychapter}
		\begin{mychapter}[cha-LaManipulationCollaborative]{La manipulation collaborative}
			\begin{mysection}[sec-exp2-Presentation]{Présentation}
				\begin{mysubsection}[sse-exp2-Objectifs]{Objectifs}
					Après avoir traité la sous-tâche élémentaire de \myemph{recherche}, la seconde expérimentation traitera des sous-tâches élémentaires de \myemph{sélection} et de \myemph{déformation} \mysubsectionref*{sse-LAmarrageMoleculaire}.
					Ces sous-tâches introduisent des actions qui nécessitent une grande synchronisation et permet de stimuler les collaborations étroites.
					La précédente expérimentation \mychapterref*{cha-LaRechercheCollaborative} a souligné l'avantage de la collaboration sur des tâches nécessitant un couplage fort.
					Les tâches proposées dans cette expérimentation sont élaborées pour stimuler les interactions entre les sujets.

					L'expérimentation va de nouveau comparer un \myglos{glo-Monome} et un \myglos{glo-Binome}.
					La manipulation \myglos*{glo-Bimanuel} est opposée à la manipulation collaborative afin de tester les performances de synchronisation.
					En effet, la synchronisation d'un seul sujet utilisant ces deux mains face à deux sujets utilisant chacun une seule de leur main.
					Trois facteurs seront étudier lors de cette comparaison.

					Le premier facteur concerne les performances.
					Les performances représente à la fois le temps total pour réaliser la tâche mais aussi les ressources mises en place pour accéder à ce résultat.
					Un \myglos{glo-Binome} en configuration collaborative sera-t-il plus performant qu'un \myglos{glo-Monome} en configuration \myglos*{glo-Bimanuel} ?

					Le second facteur observé sera la complexité de la tâche proposée.
					Le lien entre la complexité de la tâche et la configuration (collaborative ou \myglos*{glo-Bimanuel}) est étudié en fonction des performances.

					Le troisième facteur concerne l'apprentissage.
					En effet, quelque soit l'application et l'expérimentation proposée à des sujets, un phénomène d'apprentissage peut être observé.
					Cette expérimentation compare l'évolution de l'apprentissage entre les configurations collaboratives et \myglos*{glo-Bimanuel}.
				\end{mysubsection}
				\begin{mysubsection}[sse-exp2-Hypotheses]{Hypothèses}
					\begin{myparagraph}[par-exp2-AmeliorationDesPerformancesEnBinome]{\myhypothesis{1} Amélioration des performances en \myglos{glo-Binome}}
						La première hypothèse est une amélioration des performances pour les \myglos*{glo-Binome} en collaboratif comparés aux \myglos*{glo-Monome} en \myglos{glo-Bimanuel}.
						Cette amélioration se traduira principalement par une réalisation de la même tâche en un temps réduit.
						D'autres variables seront observées comme le nombre de sélections et la vitesse moyenne afin d'observer la répartition du travail entre les ressources disponibles.
					\end{myparagraph}
					\begin{myparagraph}[par-exp2-LesBinomesSontPlusPerformantsSurLesTachesComplexes]{\myhypothesis{2} Les \myglos*{glo-Binome} sont plus performants sur les tâches complexes}
						Cette second hypothèse concerne la corrélation entre la complexité de la tâche et la configuration (\myglos{glo-Binome} en collaboratif ou \myglos{glo-Monome} en \myglos{glo-Bimanuel}) sur les performances.
						L'hypothèse formule que les \myglos*{glo-Binome} seront plus performants que les \myglos*{glo-Monome} sur les tâches les plus complexes.
					\end{myparagraph}
					\begin{myparagraph}[par-exp2-LApprentissageEstPlusPerformantPourLesBinomes]{\myhypothesis{3} L'apprentissage est plus performant pour les \myglos*{glo-Binome}}
						L'hypothèse suppose que le travail collaboratif va stimuler l'apprentissage.
						L'échange des connaissances améliore l'apprentissage mais aussi grâce la multiplicité des avis et des idées pour répondre à un problème ou comprendre un événement.
					\end{myparagraph}
				\end{mysubsection}
			\end{mysection}
			\begin{mysection}[sec-exp2-DispositifExperimentalEtMateriel]{Dispositif expérimental et matériel}
				L'\myacro{acr-EVC} utilisé est illustré sur la \myfigureref{fig-exp2-IllustrationDuDispositifExperimental}.
				Comme pour la première expérimentation \mysectionref*{sec-exp1-DispositifExperimentalEtMateriel}, l'\myacro{acr-EVC} propose une visualisation partagée (vue publique) à l'aide d'un vidéoprojecteur sur un grand écran.
				Le ou les sujets font face à l'écran avec à leur disposition :
				\begin{itemize}
					\item un interface de manipulation de type souris \myThreeD;
					\item deux interfaces haptiques de déformation \mytool{tug}.
				\end{itemize}

				\begin{myfigure}
					\begin{mysubfigure}
						\myimage[width=0.49\textwidth]{exp2-schema}
						\mysubcaption[fig-exp2-IllustrationDuDispositifExperimental-SchemaDuDispositifExperimental]{Schéma du dispositif expérimental}
					\end{mysubfigure}
					\begin{mysubfigure}
						\myimage[width=0.49\textwidth]{exp2-photo}
						\mysubcaption[fig-exp2-IllustrationDuDispositifExperimental-PhotographieDuDispositifExperimental]{Photographie du dispositif expérimental}
					\end{mysubfigure}
					\mycaption[fig-exp2-IllustrationDuDispositifExperimental]{Illustration du dispositif expérimental}
				\end{myfigure}

				Les sujets ont la possibilité de communiquer entre eux sans restriction.
				Pour les \myglos*{glo-Monome}, le sujet peut utiliser chaque outil comme il le souhaite.
				Pour les \myglos*{glo-Binome}, chaque sujet se voit attribuer un outil de déformation \mytool{tug}.
				L'outil de manipulation (la souris \myThreeD) est laissé libre d'utilisation pour chacun des deux sujets.

				Pour les détails techniques concernant la plate-forme et les outils de manipulation et de déformation, se reporter au \mychapterref{app-ShaddockCollaborativeVirtualEnvironmentForMolecularDesign}.
			\end{mysection}
			\begin{mysection}[sec-exp2-Methode]{Méthode}
				\begin{mysubsection}[sse-exp2-Sujets]{Sujets}
					\mynum{36}~sujets (\mynum{8}~femmes et \mynum{28}~hommes) avec une moyenne d'âge de $\mu = 25.9$ ($\sigma = 4.70$) ont participés à cette expérimentation.
					Ils ont tous été recrutés au sein du laboratoire \myacro{acr-LIMSI} et sont chercheurs ou assistants de recherche dans les domaines suivants~:
					\begin{itemize}
						\item linguistique et traitement automatique de la parole;
						\item réalité virtuelle et système immersifs;
						\item audio-acoustique.
					\end{itemize}
					Ils ont tous le français comme langue principale.
					Aucun participant n'a de déficience visuelle (ou corrigée le cas échéant) ni de déficience audio.

					Chaque participants est complètement naïf concernant les détails de l'expérimentation.
					Une explication détaillée de la procédure expérimentale leur est donnée au commencement de l'expérimentation mais en omettant l'objectif de l'étude.
				\end{mysubsection}
				\begin{mysubsection}[sec-exp2-Variables]{Variables}
					\begin{mysubsubsection}[sss-exp2-VariablesIndependantes]{Variables indépendantes}
						\begin{myparagraph}[par-exp2-NombreDeSujets]{\myvari{1} Nombre de sujets}
							La première \myglos{glo-VariableIndependante} est une \myglos{glo-VariableInterPopulation}, c'est-à-dire que les sujets sont expérimentés dans une seule modalité de cette variable.
							\myvari{1} possède deux valeurs possibles: \og un sujet (\mycf \myemph{\myglos{glo-Monome}}) \fg ou \og deux sujets (\mycf \myemph{\myglos{glo-Binome}}) \fg.
							Le sujets seuls et les sujets en couples ont à leur disposition deux outils haptiques de déformation et un outil haptique de manipulation.
							Pour les \myglos*{glo-Binome}, seulement un des deux sujets est désigné pour l'utilisation exclusive de la souris~\myThreeD.
							\mynum{12}~\myglos*{glo-Monome} et \mynum{12}~\myglos*{glo-Binome} sont testés.
						\end{myparagraph}
						\begin{myparagraph}[par-exp2-ComplexiteDeLaTache]{\myvari{2} Complexité de la tâche}
							La seconde \myglos{glo-VariableIndependante} est une \myglos{glo-VariableIntraPopulation}.
							Deux molécules de complexités différentes sont testées.
							Deux tâches de déformation sur chacune de molécules sont proposées : une déformation au niveau inter-moléculaire et une déformation au niveau intra-moléculaire.
						\end{myparagraph}
						\begin{myparagraph}[par-exp2-LeNiveauDApprentissage]{\myvari{3} Le niveau d'apprentissage}
							La troisième \myglos{glo-VariableIndependante} est une \myglos{glo-VariableIntraPopulation}.
							Tous les sujets seront confrontés trois fois à la même tâche afin de voir l'effet de l'apprentissage en \myglos{glo-Monome} et en \myglos{glo-Binome}.
						\end{myparagraph}
					\end{mysubsubsection}
					\begin{mysubsubsection}[sec-exp2-VariablesDependantes]{Variables dépendantes}
						\begin{myparagraph}[par-exp2-LeTempsDeCompletion]{\myvard{1} Le temps de complétion}
							Ce temps est le temps total pour réaliser la tâche demandée, c'est-à-dire manipuler et déformer la molécule afin d'atteindre l'objectif fixé.
							Il n'y a pas de limite de temps pour réaliser la tâche.
						\end{myparagraph}
						\begin{myparagraph}[par-exp2-LeNombreDeSelections]{\myvard{2} Le nombre de sélections}
							\myvard{2} représente le nombre de sélections réalisées durant chaque tâche à réaliser.
							Une sélection est comptabilisée lorsque un atome ou un \myglos{glo-Residu} est sélectionné par un des deux \myglos{glo-EffecteurTerminal}.
							Un compteur est affecté pour chacun des \myglos*{glo-EffecteurTerminal}.
						\end{myparagraph}
						\begin{myparagraph}[par-exp2-LaDistancePassiveEntreLesEspacesDeTravail]{\myvard{3} La distance passive entre les espaces de travail}
							Cette distance est la distance moyenne entre les deux \myglos*{glo-EffecteurTerminal} présents durant l'ensemble de l'expérimentation.
							Cette distance représente donc une distance physique du monde réel, pas une distance virtuelle.
							Elle est de l'ordre du centimètre.
						\end{myparagraph}
						\begin{myparagraph}[par-exp2-LaDistanceActiveEntreLesEspacesDeTravail]{\myvard{4} La distance active entre les espaces de travail}
							Cette distance est la distance moyenne entre les deux \myglos*{glo-EffecteurTerminal} présents seulement lorsque ces deux \myglos*{glo-EffecteurTerminal} sont en cours de manipulation (un atome ou un \myglos{glo-Residu} est sélectionné).
						\end{myparagraph}
						\begin{myparagraph}[par-exp2-ReponsesQualitatives]{\myvard{5} Réponses qualitatives}
							Un questionnaire est proposé à tous les sujets (variable en fonction des \myglos*{glo-Monome} et des \myglos*{glo-Binome}).
							Il est constitué de plusieurs questions (notées sur échelle de \mycite[author]{Likert-1932} à cinq niveaux).
							Un questionnaire différent est adressé aux \myglos*{glo-Monome} et aux \myglos*{glo-Binome}.

							Pour les \myglos*{glo-Monome}, le questionnaire est le suivant :
							\begin{enumerate}
								\item Vous êtes-vous senti efficace ?
								\item Pensez-vous que vous auriez été plus à l'aise seul avec un seul outil de déformation ?
								\item Pensez-vous que vous auriez été plus à l'aise avec un partenaire ?
								\item Quelle solution choisiriez-vous entre les trois configurations ?
							\end{enumerate}

							Pour les \myglos*{glo-Binome}, le questionnaire est le suivant (les questions sont posées à chaque sujet du \myglos{glo-Binome}) :
							\begin{enumerate}
								\item Vous êtes-vous senti efficace ?
								\item Comment évalueriez-vous votre taux de communication\dots{}
									\begin{itemize}
										\item verbale ?
										\item gestuelle ?
										\item virtuelle ?
									\end{itemize}
								\item Vous sentez-vous utile dans le groupe (par opposition à pénalisant) ?
								\item Pensez-vous avoir une position de meneur dans le groupe ?
								\item Pensez-vous que vous auriez été plus à l'aise seul avec votre outil de déformation ?
								\item Pensez-vous que vous auriez été plus à l'aise seul avec deux outils de déformation ?
								\item Quelle solution choisiriez-vous entre les trois configurations ?
							\end{enumerate}

							Concernant la communication, les communications verbales concernent tous les échanges, dialogues exposés par la voix.
							La communication gestuelle représente les gestes que les sujets peuvent effectuer dans le monde réel pour expliquer, désigner ou pour tout autre explication à son partenaire.
							Enfin, la communication virtuelle concerne les informations données au partenaire par l'intermédiaire de l'environnement virtuel (par exemple, une désignation avec le curseur).
						\end{myparagraph}
					\end{mysubsubsection}
				\end{mysubsection}
				\begin{mysubsection}[sse-exp2-Tache]{Tâche}
					La tâche proposée est la déformation dans un \myacro{acr-EVC} sur des molécules complexes.
					Deux niveaux différents de manipulation sont proposés :
					\begin{itemize}
						\item inter-moléculaire (à l'échelle d'un \myglos{glo-Residu});
						\item intra-moléculaire (à l'échelle d'un atome).
					\end{itemize}

					\begin{mysubsubsection}[sss-exp2-DescriptionDeLaTache]{Description de la tâche}
						La tâche proposée est la déformation d'une molécule afin de la rendre conforme à un modèle.
						L'intégralité des atomes de la molécule à déformer est affiché.
						De plus, un \myemph{ruban} de cette molécule est affiché.
						En ce qui concerne la molécule cible (le modèle), seul un affichage de type \myemph{ruban} est utilisé.
						Cet affichage est appliqué en filigrane.

						Lorsqu'un sujet sélectionne un atome ou un \myglos{glo-Residu}, ce dernier est mis en surbrillance.
						De plus, l'atome ou le \myglos{glo-Residu} correspondant sur la molécule cible est affiché afin de connaître la position finale de la sélection courante.
						La \myfigureref{fig-exp2-AffichageDeLaMoleculeADeformerEtDeLaMoleculeCible} illustre ces différents effets graphiques.

						\begin{myfigure}
							\psset{unit=0.08\textwidth}
							\def\myexptwolabel(#1,#2)[#3]#4#5{\rput(#1,#2){\rnode{#3}{\textcolor{#4}{\sffamily #5}}}}
							\begin{myps}(0,0)(12,9)
								\rput[bl](1,0){\myimage[width=0.8\textwidth]{exp2-trp-zipper}}
								\rput[bl](8,0.5){\myimage[width=3.5cm,angle=-20]{exp2-green-cursor}}
								\myexptwolabel(8.8,2.6)[deformed-label]{myred}{Molécule à déformer}
								\myexptwolabel(1.3,5.5)[ghost-label]{myred}{Molécule cible}
								\myexptwolabel(7.1,7.0)[deformed-residue-label]{myblue}{Résidu à déformer}
								\myexptwolabel(1.2,2.75)[ghost-residue-label]{myblue}{Résidu cible}
								\myexptwolabel(4.0,7.75)[fixed-residue-label]{mygray}{Résidu fixe}
								\pnode(6.8,3.6){deformed}
								\pnode(1.8,4){ghost}
								\psset{linecolor=myblue}
								\cnode(6.2,4.9){1.0}{deformed-residue}
								\cnode(2.3,1.6){0.8}{ghost-residue}
								\psset{linecolor=mygray}
								\cnode(2.0,6.6){0.8}{fixed-residue}
								\psset{linewidth=1pt,linecolor=myred,linearc=.025,arrowsize=1pt 3,arrowinset=.2,nodesepA=3pt}
								\ncangle[angleA=90,angleB=0]{c->}{deformed-label}{deformed}
								\ncangle[angleA=-90,angleB=180,offsetA=-0.5]{c->}{ghost-label}{ghost}
								\psset{linecolor=myblue,nodesepB=0pt}
								\ncdiagg[angleA=-90,offsetA=0.5]{c->}{deformed-residue-label}{deformed-residue}
								\ncdiagg[angleA=-90,offsetA=-0.5]{c->}{ghost-residue-label}{ghost-residue}
								\ncdiagg[angleA=180,linecolor=mygray]{c->}{fixed-residue-label}{fixed-residue}
								\ncline[linewidth=10pt,linecolor=myblue,arrowsize=2pt 2,nodesepA=4pt]{C->}{deformed-residue}{ghost-residue}
								\psframe*[linecolor=red](0,8)(12,9)
								\psframe*[linecolor=green](0,8)(2,9)
								\rput(6,8.5){\textcolor{white}{\bfseries\sffamily\LARGE Score RMSD}}
								\psframe[linewidth=1pt,linecolor=black](0,0)(12,9)
							\end{myps}
							\mycaption[fig-exp2-AffichageDeLaMoleculeADeformerEtDeLaMoleculeCible]{Affichage de la molécule à déformer et de la molécule cible}
						\end{myfigure}
					\end{mysubsubsection}
					\begin{mysubsubsection}[sss-exp2-LaDescriptionDesTaches]{La description des tâches}
						Une tâche pour chacun de ces deux échelles de manipulation est proposé sur chacune des deux molécules.
						La première molécule est couramment nommée \myTRPZIPPER \mycite{Cochran-2001} a pour identifiant \myPDB \myPDBlink{http://www.rcsb.org/pdb/explore/explore.do?structureId=1LE1}{1LE1}.
						La second molécule est couramment nommée \myTRPCAGE \mycite{Neidigh-2002} a pour identifiant \myPDB \myPDBlink{http://www.rcsb.org/pdb/explore/explore.do?structureId=1L2Y}{1L2Y}.
						On peut donc distinguer quatre tâches différentes :
						\begin{description}
							\item[\mytask{1a}]
								Cette tâche concerne la manipulation de la molécule \myTRPZIPPER à l'échelle inter-moléculaire.
								Un \myglos{glo-Residu} à l'extrémité -- la molécule formant une chaîne -- est fixé afin d'\myemph{ancrer} la molécule au sein de l'environnement virtuel et éviter d'éventuelles dérives hors du champ visuel.
								L'intégralité des onze autres \myglos*{glo-Residu} est libre de mouvement.
								La forme général de la molécule peut être comparée à un \myform{V} : la chaîne de \myglos*{glo-Residu} de la molécule contient une cassure.
							\item[\mytask{1b}]
								Cette tâche concerne la manipulation de la molécule \myTRPCAGE à l'échelle inter-moléculaire.
								Comme la tâche \mytask{1a}, elle contient un \myglos{glo-Residu} fixe à une extrémité.
								L'intégralité des dix neuf autres \myglos*{glo-Residu} est libre de mouvement.
								La forme général de la molécule peut être comparée à un \myform{W} : la chaîne de \myglos*{glo-Residu} de la molécule contient deux cassures.
							\item[\mytask{2a}]
								Cette tâche concerne la manipulation de la molécule \myTRPZIPPER à l'échelle intra-moléculaire.
								Seulement trois \myglos*{glo-Residu} sont laissés libres tandis que tous les autres sont fixés.
								Les contraintes physiques de cette tâche sont relativement faibles.
								Cependant, la difficulté de cette tâche réside dans la recherche des \myglos*{glo-Residu} à déformer qui ne sont pas aisés à trouver.
							\item[\mytask{2b}]
								Cette tâche concerne la manipulation de la molécule \myTRPCAGE à l'échelle intra-moléculaire.
								Seulement six \myglos*{glo-Residu} sont laissés libres tandis que les autres sont fixés.
								La déformation requise demande une grande dépense d'énergie.
								En effet, la molécule proposée se trouve dans une sorte de puit de potentiel (un \myemph{minima} local) et l'objectif est d'atteindre un autre puit de potentiel (un autre \myemph{minima} local).
								L'énergie nécessaire pour passer d'un puit à l'autre est relativement importante, à tel point qu'un seul outil de déformation n'est pas suffisant.
								La manipulation synchrone de deux \myglos*{glo-Residu} est la seule solution pour atteindre l'objectif.
						\end{description}

						Un résumé de la complexité des quatre tâches est exposé dans le \mytableref{tab-exp2-ParametresDeComplexiteDesTaches} selon les critères suivants :
						\begin{description}
							\item[Nombre d'atomes] C'est le nombre total d'atomes que contient la molécule à manipuler;
							\item[\myglos{glo-Residu} libre] C'est le nombre de \myglos*{glo-Residu} non fixés sur la molécule;
							\item[Cassure] Ce sont les cassures de la chaîne principale de la molécule; elles représentent les jonctions entre hélices $\alpha$ et/ou les feuillets $\beta$;
							\item[Champ de force] Il représente la difficulté en terme de contrainte physique; il exprime l'énergie minimum nécessaire pour atteindre l'objectif et se traduit par trois niveaux (\myemph{faible}, \myemph{moyen} et \myemph{fort}).
						\end{description}

						\begin{mytable}
							\mycaption[tab-exp2-ParametresDeComplexiteDesTaches]{Paramètres de complexité des tâches}
							\begin{mytabular}{^>{\bfseries}p{9em}-C-C-C-C}
								\mytoprule
								\myrowstyle{\bfseries}
								& \mytask{1a} & \mytask{1b} & \mytask{2a} & \mytask{2b} \\
								\mymiddlerule[\heavyrulewidth]
								Nombre d'atomes           & \mynum{218} & \mynum{304} & \mynum{218} & \mynum{304} \\
								\mymiddlerule
								\myglos{glo-Residu} libre & \mynum{11}  & \mynum{19}  & \mynum{3}   & \mynum{7}   \\
								\mymiddlerule
								Cassure                   & \mynum{1}   & \mynum{2}   & \mynum{0}   & \mynum{1}   \\
								\mymiddlerule
								Champ de force            & Moyen       & Moyen       & Faible      & Fort        \\
								\mybottomrule
							\end{mytabular}
						\end{mytable}
					\end{mysubsubsection}
					\begin{mysubsubsection}[sss-exp2-LesOutilsDisponibles]{Les outils disponibles}
						Des outils de déformation légérement différents sont proposés en fonction de la tâche à réaliser.
						Pour les tâches de déformation au niveau inter-moléculaire, l'outil de déformation est l'outil \mytool{tug} : il permet de déformer d'un tenant l'intégralité d'un \myglos{glo-Residu}.
						Pour les tâches de déformation au niveau intra-moléculaire, l'outil de déformation est l'outil \mytool{tug} : il permet d'appliquer une force sur un unique atome.
						L'outil \mytool{tug} pour les \myglos*{glo-Residu} applique la même force à chaque atome du \myglos{glo-Residu}.
						Il en résulte que l'outil \mytool{tug} pour les \myglos*{glo-Residu} permet de développer plus d'énergie.
					\end{mysubsubsection}
				\end{mysubsection}
				\begin{mysubsection}[sse-exp2-Procedure]{Procédure}
					Pour débuter cette expérimentation, les sujets sont confrontés à un exemple sur la molécule \myPrion \mycite{Cochran-2001} ayant pour identifiant \myPDB \myPDBlink{http://www.rcsb.org/pdb/explore/explore.do?structureId=1LE1}{1LE1}.
					Pendant la phase d'entraînement, les outils sont introduits et expliqués un par un.
					Chaque sujet a la possibilité de tester les outils et peut questionner l'expérimentateur.
					Dans le cas des \myglos*{glo-Binome}, cette phase d'entraînement est également l'occasion de choisir qui, parmi les deux sujets, sera en charge de la manipulation de la molécule à l'aide de l'outil de manipulation \mytool{grab}.

					Un résumé du protocole expérimental est exprimé dans le \mytableref{tab-exp2-SyntheseDeLaProcedureExperimentale}.

					\begin{mytable}
						\mycaption[tab-exp2-SyntheseDeLaProcedureExperimentale]{Synthèse de la procédure expérimentale}
						\newcommand{\mytitlecolumn}[2]{%
							\multirow{#1}*{%
								\begin{minipage}{6em}%
									\raggedleft #2%
								\end{minipage}%
							}
						}
						\newlength{\exptwofirstcolumn}
						\newlength{\exptwosecondcolumn}
						\setlength{\exptwofirstcolumn}{7em}
						\setlength{\exptwosecondcolumn}{\textwidth}
						\addtolength{\exptwosecondcolumn}{-\exptwofirstcolumn}
						\addtolength{\exptwosecondcolumn}{-4\tabcolsep}
						\begin{mytabular}{>{\bfseries}p{\exptwofirstcolumn}p{\exptwosecondcolumn}}
							\mytoprule
							\mytitlecolumn{1}{Tâche}                  & Déformation d'une molécule                                                      \\
							\mymiddlerule[\heavyrulewidth]
							\mytitlecolumn{1}{Hypothèses}             & \myhypothesis{1} Amélioration des performances en \myglos{glo-Binome}           \\
							                                          & \myhypothesis{2} \myGlos*{glo-Binome} plus performants sur les tâches complexes \\
							                                          & \myhypothesis{3} Apprentissage plus performant en \myglos{glo-Binome}           \\
							\mymiddlerule
							\mytitlecolumn{3}{Variable indépendantes} & \myvari{1} Nombre de sujets                                                     \\
							                                          & \myvari{2} Complexité de la tâche                                               \\
							                                          & \myvari{3} Niveau d'apprentissage                                               \\
							\mymiddlerule
							\mytitlecolumn{5}{Variable dépendantes}   & \myvard{1} Temps de complétion                                                  \\
							                                          & \myvard{2} Nombre de sélections                                                 \\
							                                          & \myvard{3} Distance passive entre les espaces de travail                        \\
							                                          & \myvard{4} Distance active entre les espaces de travail                         \\
							                                          & \myvard{5} Réponses qualitatives                                                \\
							\mymiddlerule[\heavyrulewidth]
							\multicolumn{2}{c}{%
								\begin{tabular}{^C-C-C-C}
									\myrowstyle{\bfseries}
									Condition \mycondition{1} & Condition \mycondition{2} & Condition \mycondition{3} & Condition \mycondition{4} \\
									\mymiddlerule
									Bimanuel ($N=1$)          & Bimanuel ($N=1$)          & Collaboratif ($N=2$)      & Collaboratif ($N=2$)      \\
									\mymiddlerule
									\mytask{1a}               & \mytask{1b}               & \mytask{1a}               & \mytask{1b}               \\
									\mytask{1b}               & \mytask{1a}               & \mytask{1b}               & \mytask{1a}               \\
									\mytask{2a}               & \mytask{2b}               & \mytask{2a}               & \mytask{2b}               \\
									\mytask{2b}               & \mytask{2a}               & \mytask{2b}               & \mytask{2a}               \\
								\end{tabular}
							} \\
							\mybottomrule
						\end{mytabular}
					\end{mytable}
				\end{mysubsection}
			\end{mysection}
		\end{mychapter}
		\begin{mychapter}[cha-LesDynamiquesDeGroupe]{Les dynamiques de groupe}
		\end{mychapter}
	\end{mypart}
	\begin{mypart}[prt-PropositionsPourLeTravailCollaboratif]{Propositions pour le travail collaboratif}
		\begin{mychapter}[cha-TravailCollaboratifAssisteParHaptique]{Travail collaboratif assisté par haptique}
		\end{mychapter}
	\end{mypart}
	\begin{mypart}[prt-Synthese]{Synthèse}
		\begin{mychapter}[cha-ConclusionEtPerspectives]{Conclusion et perspectives}
		\end{mychapter}
	\end{mypart}

	\myglossary
	\myappendix
	\begin{mychapter}[app-ShaddockCollaborativeVirtualEnvironmentForMolecularDesign]{\textsc{Shaddock} -- Collaborative Virtual Environment for Molecular Design}
	\end{mychapter}
\end{document}
