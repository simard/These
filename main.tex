\documentclass[myfrancais]{mythesis}
\usepackage{mydate}
\usepackage{graphicx}
\usepackage{caption}
\newcommand{\name}[1]{\textsc{#1}}
\newcommand{\acro}[1]{\textsc{#1}}
\newcommand{\LIMSI}{\acro{CNRS--LIMSI}}

\hypersetup{%
	pdftitle={Interactions haptiques collaboratives pour la manipulation moléculaire},%
	pdfauthor={Jean SIMARD},%
	pdfsubject={Mémoire de thèse en informatique},%
	pdfdisplaydoctitle=true,%
	pdflang={FR-fr}%
}

\title{Interactions haptiques collaboratives pour la manipulation moléculaire}
\author{Jean~\name{Simard}}
\documenttype{Thèse en Informatique}
\university{École Doctorale d'Informatique de Paris Sud}
\date{\mydate[datestyle=long]{01/12/2011}}
\jury{%
	Martin & \name{DUPONT} & (rapporteur) & Directeur de recherche au \LIMSI \\
	Martin & \name{DUPOND} & (examinateur) & Directeur de recherche au \LIMSI}
	\addglobalbib[datatype=bibtex]{biblio.bib}
\begin{document}
	\frontmatter
	\maketitle
	\mytoc
	\mylof
	\mylot
	\mainmatter
	\part{Introduction}\label{par-test}
	\chapter{État de l'art}\label{cha-state}
	\mypartref{par-test}\par
	\mypartref*{par-test}\par
	\mypartref[refname={Part.},refsee=true]{par-test}\par
	\mypartref[reffont={\bf},refsee=true]{par-test}\par
	\mypartref[refsee=true]{par-test}\par
	\mypartref*[refname={Part.}]{par-test}\par
	\mypartref*[reffont={\bf}]{par-test}\par
	\mypartref*{par-test}\par
	\mypartref*[refsee=false]{par-test}\par
	\myfigureref{fig-MaFenga}\par
	\myfigureref*{fig-MaFenga}\par
	\mytableref{tab-TableDeMerde}\par
	\mytableref*{tab-TableDeMerde}\par
	\myminitoc
	\section{Le collaboratif}
	\subsection{Test 1}
	\mychapterref{cha-state}
	Une citation pour Feng \mycite{Xiong-2010}.

	\subsection{Test 2}
	Une citation pour Feng \mycite{Simard-2010a}.
	%Une citation pour Feng \mycite{Simard-2010b}.
	%Une citation pour Feng \mycite{Simard-2010c}.
	%Une citation pour Feng \mycite{Simard-2011}.
	\myminibiblio

	\section{La manipulation moléculaire}
	La citation de \mycite[author]{Simard-2010a,Simard-2010b}.
	\begin{myfigure}
		\begin{mysubfigure}[0.25\textwidth]
			\myimage[width=0.25\textwidth]{feng}
			\mysubcaption[fig-MaFenga]{Ma première Feng}
		\end{mysubfigure}
		\begin{mysubfigure}[0.25\textwidth]
			\myimage[width=0.25\textwidth]{feng}
			\mysubcaption[fig-MaFengb]{Ma seconde Feng}
		\end{mysubfigure}
		\begin{mysubfigure}[0.25\textwidth]
			\myimage[width=0.25\textwidth]{feng}
			\mysubcaption[fig-MaFengc]{Ma troisième Feng}
		\end{mysubfigure}
		\mycaption[fig-MaFeng]{Ma Feng}
	\end{myfigure}
	\myminibiblio

	\chapter{Plate-forme}
	\mychapterref{cha-state}
	\mycite{Xiong-2010}\mycite[author]{Simard-2011}\footnote{Une citation toute seule, vous pourrez remarquer !}.
	Et une référence sur la table \ref{tab-TableDeMerde} !
	\begin{mytable}
		\mycaption[tab-TableDeMerde]{Table de merde}
		\begin{mytabular}{cr@{,}l}
			\mytoprule
			Constante & {\small <entier>} & {\small <décimale>} \\
			\mymiddlerule[\heavyrulewidth]
			$\pi$ & 3 & 14159 \\
			\mymiddlerule
			$e$ & 2 & 71828 \\
			\mymiddlerule
			$\bar{h}$ & 6 & 62606896$\times 10^{-34}$ \\
			\mybottomrule
		\end{mytabular}
	\end{mytable}
	\myminibiblio

	\chapter{Études sur le travail collaboratif}
	\myminibiblio

	\appendix

	\chapter{Le matériel utilisé}
\end{document}
