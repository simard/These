\documentclass[myfrancais]{mythesis}
\usepackage{mydate}
\usepackage{mytodo}
\usepackage{mycolor}
\usepackage{myps}
\usepackage{mymacro}

\makeatletter
% Modify the bibliography style
\newcounter{mymaxcitenames}
\AtBeginDocument{%
	\setcounter{mymaxcitenames}{\value{maxnames}}%
}
\renewbibmacro{begentry}{%
	\printtext[brackets]{%
		\defcounter{maxnames}{\value{mymaxcitenames}}%
		\printnames{labelname}~\usebibmacro{cite:labelyear+extrayear}%
	}%
	\newline%
}
% AAA
\newcommand{\myACER}{\textsc{acer}\xspace}
\newcommand{\myAlanine}{Alanine\xspace}
\newcommand{\myanalysis}[1]{\input{files/#1}\%}
\newcommand{\myangstrom}{\AA ngström\xspace}
\newcommand{\myanova}[1]{\input{files/#1}}
\newcommand{\myatom}[2][]{%
	{%
		\ifstrempty{#1}%
			{\makefirstuc{\textsf{#2}}}%
			{\textcolor{#1}{\makefirstuc{\textsf{#2}}}}%
		\xspace%
	}%
}
\newcommand{\myAudacity}{\textsc{audacity}\myregistered}% No '\xspace' because of already one in '\myregistered'
% CCC
\newcommand{\mycarbon}{\myatom[mycarboncolor]{C}}
\newcommand{\myCasioXJ}{\textsc{Casio xj}\xspace}
\newcommand{\myCHARMM}{\textsc{charmm}\xspace}
\newcommand{\myChimera}{\textsc{chimera}\xspace}
\newcommand{\myClayWorks}{\textsc{Clayworks}\xspace}
\newcommand{\mycondition}[1]{$\left(\mathcal{C}_{#1}\right)$\xspace}
\newcommand{\myCPK}{\textsc{cpk}\xspace}
% DDD
\newcommand{\myDesktop}{\myPHANToM Desktop\myregistered}% No '\xspace' because of already one in '\myregistered'
% FFF
\newcommand{\myfeuillet}{feuillet-$\beta$\xspace}
\WithSuffix\newcommand\myfeuillet*{feuillets-$\beta$\xspace}
\newcommand{\myform}[1]{\textbf{\sffamily\MakeUppercase{#1}}}
% GGG
\newcommand{\myGhost}{\textsc{Ghost}\xspace}
\newcommand{\myGromacs}{\textsc{Gromacs}\xspace}
\newcommand{\mygroup}[1]{$\left(\mathcal{G}_{#1}\right)$\xspace}
% HHH
\newcommand{\myHaption}{\textsc{Haption}\xspace}
\newcommand{\myHawthorne}{\myemph{Hawthorne Works}\xspace}
\newcommand{\myHBonds}{\textit{HBonds}\xspace}
\newcommand{\myhelice}{hélice-$\alpha$\xspace}
\WithSuffix\newcommand\myhelice*{hélices-$\alpha$\xspace}
\newcommand{\myhypothesis}[1]{$\left(\mathcal{H}_{#1}\right)$\xspace}
% III
\newcommand{\myIntelCore}{Intel\myregistered Core\mytrademark~2 \textsc{q9450} (\mynum[GHz]{2.66})\xspace}
% JJJ
\newcommand{\myJmol}{\textsc{Jmol}\xspace}
% LLL
\newcommand{\myLCD}{\textsc{lcd}\xspace}
\newcommand{\myLicorice}{\textit{Licorice}\xspace}
\newcommand{\myLinux}{\textsc{Linux}\xspace}
% MMM
\newcommand{\myMacOS}{Mac~\textsc{OS}\xspace}
\newcommand{\myMDDriver}{\textsc{MDDriver}\xspace}
% NNN
\newcommand{\myNewRibbon}{\textit{NewRibbon}\xspace}
\def\mynode{%
	\@ifnextchar[{\mynode@i}{\mynode@i[style=nodestyle]}%
}
\def\mynode@i[#1](#2,#3)[#4]#5{%
	\rput(#2,#3){\Rnode{#4}{\psframebox[style=nodestyle,#1]{\vphantom{pÉ}#5}}}%
}
\newcommand{\mynytrogen}{\myatom[mynytrogencolor]{A}}
\newcommand{\myNusE}{\textsc{NusE}\xspace}
\newcommand{\myNusENusG}{\textsc{NusE:NusG}\xspace}
\newcommand{\myNusG}{\textsc{NusG}\xspace}
% OOO
\newcommand{\myOmni}{\myPHANToM Omni\myregistered}% No '\xspace' because of already one in '\myregistered'
\newcommand{\myOpenHaptics}{\textsc{OpenHaptics}\mytrademark}% No '\xspace' because of already one in '\mytrademark'
\newcommand{\myoxygen}{\myatom[myoxygencolor]{O}}
% PPP
\newcommand{\myPC}{\textsc{pc}\xspace}
\newcommand{\myPDB}{\textsc{pdb}\xspace}
\newcommand{\myPDBbase}{\emph{Protein~DataBase}\xspace}
\newcommand{\myPDBlink}[2]{\href{#1}{\textsc{\MakeLowercase{#2}}}}
\newcommand{\myPHANToM}{\textsc{phant}o\textsc{m}\xspace}
\newcommand{\myPremium}{\myPHANToM Premium\myregistered}% No '\xspace' because of already one in '\myregistered'
\newcommand{\myPrion}{Prion\xspace}
\newcommand{\myPSF}{\textsc{psf}\xspace}
\newcommand{\mypvalue}{$p$-value\xspace}
\newcommand{\myPyMOL}{\textsc{p}y\textsc{mol}\xspace}
% RRR
\newcommand{\myRAM}[2][Go]{\mynum[#1]{#2} de \textsc{ram}}
\newcommand{\myRasmol}{\textsc{RasMol}\xspace}
\newcommand{\myresidue}[1]{$\left(\mathcal{R}_{#1}\right)$\xspace}
% SSS
\newcommand{\myscenario}[1]{\textsc{#1}}
\newcommand{\mySensAble}{\textsc{SensAble}\xspace}
\newcommand{\myShaddock}{\textsc{Shaddock}\xspace}
\newcommand{\mySony}{\textsc{sony}\myregistered}% No '\xspace' because of already one in '\myregistered'
\newcommand{\mySpaceNavigator}{SpaceNavigator\myregistered}% No '\xspace' because of already one in '\myregistered'
\newcommand{\mysubject}[1]{$\mathcal{S}_#1$}
\newcommand{\mysulfur}{\myatom[mysulfurcolor]{S}}
\newcommand{\mysummary}[1]{\input{files/#1}}
% TTT
\newcommand{\myTCPIP}{\textsc{tcp/ip}\xspace}
\newcommand{\myThreeD}{\textsc{3d}\xspace}
\newcommand{\mytool}[1]{\myemph{#1}}
\newcommand{\myTRPCAGE}{\textsc{trp-cage}\xspace}
\newcommand{\myTRPZIPPER}{\textsc{trp-zipper}\xspace}
% UUU
\newcommand{\myUbiquitin}{Ubiquitin\xspace}
\newcommand{\myUbuntu}{\textsc{Ubuntu}~v$10.04$\xspace}
\newcommand{\myUSB}{\textsc{usb}\xspace}
\newcommand{\myuser}[1]{$\mathcal{#1}$}
% VVV
\newcommand{\myvar}[2]{$\left(\mathcal{V}_{\mathrm{#1}#2}\right)$\xspace}
\newcommand{\myvard}[1]{\myvar{d}{#1}}
\newcommand{\myvari}[1]{\myvar{i}{#1}}
\newcommand{\myVGA}{\textsc{vga}\xspace}
\newcommand{\myVirtuose}{\textsc{Virtuose}\mytrademark~\textsc{6d}\mynum{35}--\mynum{45}\xspace}
% WWW
\newcommand{\myWindows}{\textsc{Windows}\xspace}

% Needed lengths
\newlength{\mywidth}
\newlength{\myheight}

% PSTricks style
\newpsstyle{nodestyle}{framearc=0.25,shadow=true,shadowcolor=myblue,blur=true}
\makeatother

\NeedsTeXFormat{LaTeX2e}[1999/01/01]
\ProvidesPackage{mycolor}[2011/04/29]

%%%%%%%%%%%%%%%%%%%%%%%%%%%%%
%% Declare package options %%
%%%%%%%%%%%%%%%%%%%%%%%%%%%%%
% In case of unknown options
\DeclareOption*{%
	\PackageWarning{mycolor}{Unknown option `\CurrentOption'}%
}

\ProcessOptions

%% Options to pass to packages

%% Packages to call
\RequirePackage{xcolor}

%%%%%%%%%%%%%%%%%%%%%%%
%% New configuration %%
%%%%%%%%%%%%%%%%%%%%%%%
\definecolor{mydarkred}{rgb}{0.5265625 0.25 0.25}
\definecolor{myred}{rgb}{0.7265625 0 0}
\definecolor{mylightred}{rgb}{0.9265625 0.5 0.5}
\definecolor{mylightestred}{rgb}{1 0.66 0.66}
\definecolor{mydarkblue}{rgb}{0 0 0.2265625}
\definecolor{myblue}{rgb}{0 0 0.7265625}
\definecolor{mylightblue}{rgb}{0.500 0.500 0.9265625}
\definecolor{mylightestblue}{rgb}{0.7500 0.7500 0.9265625}
\definecolor{mygreen}{rgb}{0 0.7265625 0}
\definecolor{mylightgreen}{rgb}{0.500 0.9265625 0.500}
\definecolor{mylightestgreen}{rgb}{0.7500 0.9265625 0.7500}
\definecolor{mygray}{gray}{0.6666667}

%%%%%%%%%%%%%%%%%%
%% New commands %%
%%%%%%%%%%%%%%%%%%

% End of package
\endinput

% AAA
\mynewacro{acr-AFM}%
{%
	name={\textsc{afm}},
	first={microscope à force atomique (\textsc{afm} pour \myemph{Atomic Force Microscope})},%
	plural={\textsc{afm}s},%
	firstplural={microscopes à force atomique (\textsc{afm} pour \myemph{Atomic Force Microscope})},%
	description={Microscope permettant l'observation de la topologie de la surface d'un échantillon au niveau atomique}
}
\mynewacro{acr-API}%
{%
	name={\textsc{api}},%
	first={interface de programmation (\textsc{api})},%
	plural={\textsc{api}s},%
	firstplural={interfaces de programmation (\textsc{api}s)},%
	description={\textsc{api} vient de l'anglais \myemph{Application Programming Interface} et désigne une interface avec un programme informatique}%
}
% BBB
% Be careful, because this word has no plural form, but the femina word in plural form
\mynewglos{glo-Bimanuel}%
{%
	name={bimanuel},%
	description={Qui se fait avec les deux mains},%
	plural={bimanuelle}%
}
\mynewglos{glo-Binome}%
{%
	name={binôme},%
	description={Groupe constitué de \mynum{2}~personnes},%
	plural={binômes}%
}
% CCC
\mynewacro{acr-CAO}%
{%
	name={\textsc{cao}},%
	first={conception assistée par ordinateur (\textsc{cao})},%
	description={La \textsc{cao} permet de concevoir et de tester virtuellement, à l'aide d'outils informatique, des produits manufacturés}%
}
\mynewglos{glo-ConflitDeCoordination}%
{%
	name={conflit de coordination},%
	description={Conflit entre deux sujets qui peut survenir lorsque les deux sujets tente d'accéder ou de déformer un objet au même instant},%
	plural={conflits de coordination}%
}
\mynewacro{acr-CUDA}%
{%
	name={\textsc{cuda}},%
	first={\textsc{cuda} (\myemph{Compute Unified Device Architecture})},%
	description={Technologie permettant d'utiliser l'unité graphique d'un ordinateur pour effectuer des calculs à hautes performances}%
}
\mynewglos{glo-Curseur}%
{%
	name={curseur},%
	description={Élément virtuel associé à un élément physique que le sujet manipule; il est lié à l'\myglos{glo-EffecteurTerminal}},%
	plural={curseurs}%
}
% DDD
\mynewacro{acr-DDL}%
{%
	name={\textsc{ddl}},%
	first={degré de liberté (\textsc{ddl})},%
	plural={\textsc{ddl}s},%
	firstplural={degrés de liberté (\textsc{ddl}s)},%
	description={Mouvements relatifs indépendants d'un solide par rapport à un autre}%
}
\mynewglos{glo-DockingMoleculaire}%
{%
	name={\myemph{docking} moléculaire},%
	description={Méthode permettant de déterminer l'orientation et la déformation optimale de \mynum{2}~molécules afin qu'elle s'assemble pour former un complexe de molécules stable},%
	plural={\myemph{docking} moléculaires}%
}
% EEE
\mynewglos{glo-EffecteurTerminal}%
{%
	name={effecteur terminal},%
	description={Élément physique que le sujet manipule; il est lié au \myglos{glo-Curseur} du monde virtuel},%
	plural={effecteurs terminaux}%
}
\mynewacro{acr-EVC}%
{%
	name={\textsc{evc}},%
	first={Environnement Virtuel Collaboratif (\textsc{evc})},%
	firstplural={Environnements Virtuels Collaboratifs (\textsc{evc})},%
	description={Ensemble logiciel et matériel permettant de faire interagir plusieurs utilisateurs au sein d'un même environnement; ils jouent un rôle important dans le développement de nouvelles méthodes de travail collaboratives}%
}
% HHH
\mynewglos{glo-Homoscedasticite}%
{%
	name={homoscedasticité},%
	description={Équivalent à homogénéité des variances; permet de comparer des variables aléatoires possédant des variances similaires},%
	plural={homoscedasticités}%
}
% III
\mynewacro{acr-IBPC}%
{%
	name={\textsc{ibpc}},%
	first={Institut de Biologie Physico-Chimie (\textsc{ibpc})},%
	description={Institut de recherche, géré par la fédération de recherche \textsc{frc}~\mynum{550}, étudiant les bases structurales, génétiques et physico-chimiques à leur différents niveaux d'intégration}%
}
\mynewacro{acr-IMD}%
{%
	name={\textsc{imd}},%
	first={\textsc{imd} (\myemph{Interactive Molecular Dynamics})},%
	description={Programme permettant de connecter le logiciel de visualisation moléculaire \myacro-{acr-VMD} avec le logiciel de simulation \myacro-{acr-NAMD} pour une simulation interactive en temps-réel \mycite{Stadler-1997}}%
}
\mynewacro{acr-ITAP}%
{%
	name={\textsc{itap}},%
	first={\myemph{Institut für Theoretische und Angewandte Physik} (\textsc{itap})},%
	description={Institut de Physique Théorique et Appliquée de \myname{Stuttgart} à l'origine du développement du logiciel \myacro{acr-IMD}}%
}
% LLL
\mynewacro{acr-LIMSI}%
{%
	name={\textsc{cnrs--limsi}},%
	first={Laboratoire pour l'Informatique, la Mécanique et les Sciences de l'Ingénieur (\textsc{cnrs--limsi})},%
	description={Unité Propre de Recherche du \textsc{cnrs} (\textsc{upr}~3251) associé aux universités \textsc{Paris} Sud et Pierre et Marie \textsc{Curie}}%
}
% MMM
\mynewglos{glo-Meneur}%
{%
	name={meneur},%
	description={En anglais \myemph{leader}, personne qui dirige un groupe afin d'atteindre des objectifs communs à ce groupe; c'est celui qui prend les décisions (voir aussi \myglos{glo-Suiveur})},%
	plural={meneurs}%
}
\mynewglos{glo-Monomanuel}%
{%
	name={monomanuel},%
	description={Qui se fait avec une main},%
	plural={monomanuelle}%
}
\mynewglos{glo-Monome}%
{%
	name={monôme},%
	description={\myemph{Groupe} constitué d'une unique personne},%
	plural={monômes}%
}
\mynewglos{glo-MotivationSociale}%
{%
	name={motivation sociale},%
	description={En anglais \myemph{social facilitation} \mycite{Triplett-1900}, phénomène de groupe où les personnes fournissent plus d'efforts grâce à la présence de partenaires},%
	plural={motivation sociale}%
}
\mynewacro{acr-TRM}%
{%
	name={\textsc{trm}},
	first={Théorie des Ressources Multiples (\textsc{trm})},%
	description={Cette théorie, élaborée par \mycite[author]{Wickens-1984} (\textsc{mrt} pour \myemph{Multiple Resource Theory}), propose un modèle pour la gestion des charges de travail pour un humain}
}
% NNN
\mynewacro{acr-NAMD}%
{%
	name={\textsc{namd}},%
	first={\textsc{namd} (\myemph{Scalable Molecular Dynamics})},%
	description={Programme de simulation pour la dynamique moléculaire \mycite{Phillips-2005}}%
}
% PPP
\mynewglos{glo-ParesseSociale}%
{%
	name={paresse sociale},%
	description={En anglais \myemph{social loafing} \mycite{Ringelmann-1913}, phénomène de groupe où les personnes fournissent moins d'effort pour la réalisation d'une tâche que s'ils effectuaient la tâche seuls},%
	plural={paresse sociale}%
}
\mynewacro{acr-PCV}%
{%
	name={\textsc{pcv}},
	first={Primitive Comportementale Virtuelle (\textsc{pcv})},%
	plural={\textsc{pcv}s},%
	firstplural={Primitives Comportementales Virtuelles (\textsc{pcv}s)},%
	description={Dans une application de réalité virtuelle, les activités d'un sujet peuvent toujours être décomposées en quatre comportements de base, appelés \myacro+{acr-PCV}, qui sont : observer, se déplacer, agir et communiquer \mycite{Fuchs-2006}}
}
% QQQ
\mynewglos{glo-Quadrinome}%
{%
	name={quadrinôme},%
	description={Groupe constitué de \mynum{4}~personnes},%
	plural={quadrinômes}%
}
% RRR
\mynewglos{glo-Residu}%
{%
	name={résidu},%
	description={Groupe d'atomes constituant un des blocs élémentaires d'une molécule},%
	plural={résidus}%
}
\mynewacro{acr-RMSD}%
{%
	name={\textsc{rmsd}},%
	first={\myemph{Root Mean Square Deviation} (\textsc{rmsd})},%
	description={Appelé Écart Quadratique Moyen en français, il permet -- dans le cadre de la biologie moléculaire -- de mesurer la différence entre deux déformations d'une même molécule}%
}
% SSS
\mynewglos{glo-StructureInformelle}%
{%
	name={structure informelle},%
	description={Groupe de personnes sans structures ni hiérarchie},%
	plural={structures informelles}%
}
\mynewglos{glo-Suiveur}%
{%
	name={suiveur},%
	description={En anglais \myemph{follower}, personne qui se laisse diriger dans un groupe afin d'atteindre des objectifs communs à ce groupe; c'est une personne qui ne prend pas de décision (voir aussi \myglos{glo-Meneur})},%
	plural={suiveurs}%
}
\mynewacro{acr-SUS}%
{%
	name={\textsc{sus}},%
	first={\textsc{sus} (\myemph{System Usability Scale})},%
	description={Échelle de notation entre \mynum{0} et \mynum{100} proposée par \mycite[author]{Brooke-1996} permettant d'évaluer l'utilisabilité d'un système}%
}
% TTT
\mynewglos{glo-Tetranome}%
{%
	name={tetranôme},%
	description={Groupe constitué de \mynum{4}~personnes},%
	plural={tetranômes}%
}
\mynewglos{glo-Trinome}%
{%
	name={trinôme},%
	description={Groupe constitué de \mynum{3}~personnes},%
	plural={trinômes}%
}
% UUU
\mynewacro{acr-UDP}%
{%
	name={\textsc{udp}},
	first={\textsc{udp} (\myemph{User Datagram Protocol} pour protocole de datagramme utilisateur)},%
	plural={\textsc{afm}s},%
	firstplural={\textsc{udp} (\myemph{User Datagram Protocol} pour protocole de datagramme utilisateur)},%
	description={c'est un des principaux protocole de télécommunication sur internet ; il a pour distinction de ne pas vérifier l'intégrité des données transmises}
}
\mynewacro{acr-UML}%
{%
	name={\textsc{uml}},%
	first={\textsc{uml} (\myemph{Unified Modeling Language})},%
	description={C'est un langage graphique de modélisation utilisé principalement en génie logiciel}%
}
% VVV
\mynewglos{glo-VariableDependante}%
{%
	name={variable dépendante},%
	description={Facteur mesuré sur une expérimentation (nombre de sélections, trajectoire, \myetc); ces variables sont influencées par les \myglos*{glo-VariableIndependante}},%
	plural={variables dépendantes}%
}
\mynewglos{glo-VariableIndependante}%
{%
	name={variable indépendante},%
	description={Facteur pouvant varier et être manipuler sur une expérimentation (nombre de participants, tâche, \myetc); ces variables vont avoir une incidence sur les \myglos*{glo-VariableDependante}},%
	plural={variables indépendantes}%
}
\mynewglos{glo-VariableInterSujets}%
{%
	name={variable inter-sujets},%
	description={Variables pour lesquelles les sujets sont confrontés à une et une seule des modalités de la variable},%
	plural={variables inter-sujets}%
}
\mynewglos{glo-VariableIntraSujets}%
{%
	name={variable intra-sujets},%
	description={Variables pour lesquelles les sujets sont confrontés à toutes les modalités de la variable},%
	plural={variables intra-sujets}%
}
\mynewacro{acr-VMD}%
{%
	name={\textsc{vmd}},%
	first={\textsc{vmd} (\myemph{Visual Molecular Dynamics})},%
	description={Programme de visualisation moléculaire \mycite{Humphrey-1996}}%
}
\mynewacro{acr-VRPN}%
{%
	name={\textsc{vrpn}},%
	first={\textsc{vrpn} (\myemph{Virtual Reality Protocol Network})},%
	description={Logiciel permettant de connecter différents périphériques de réalité virtuelle à une même application sous forme d'une architecture client/serveur \mycite{Taylor-II-2001}}%
}


\hypersetup{%
	pdftitle={Interactions haptiques collaboratives pour la manipulation moléculaire},%
	pdfauthor={Jean SIMARD},%
	pdfsubject={Mémoire de thèse en informatique},%
	pdfdisplaydoctitle=true,%
	pdflang={FR-fr}%
}
\addglobalbib[datatype=bibtex]{biblio.bib}

\title{Interactions haptiques collaboratives pour la manipulation moléculaire}
\author{Jean~\myname{Simard}}
\documenttype{Thèse en Informatique}
\university{École Doctorale d'Informatique de Paris Sud}
\date{\mydate[datestyle=long]{01/12/2011}}
\jury{%
	Martin & \myname{DUPONT} & (rapporteur) & Directeur de recherche au \myacro-{acr-LIMSI} \\
	Martin & \myname{DUPOND} & (examinateur) & Directeur de recherche au \myacro-{acr-LIMSI}}

\begin{document}
	\frontmatter
	\maketitle
	\mytoc
	\mylof
	\mylot
	\mylotodo
	\mainmatter
	\begin{mypart}[prt-Introduction]{Introduction}
		\begin{mychapter}[cha-LeSujet]{Le sujet}
			\begin{mysection}[sec-EtatDeLArt]{État de l'art}
			\end{mysection}
			\begin{mysection}[sec-Contexte]{Contexte}
				\begin{mysubsection}[sse-LAmarrageMoleculaire]{L'\myglosnl{glo-AmarrageMoleculaire}}
					Le contexte de l'expérimentation est l'\myglos{glo-AmarrageMoleculaire} plus communément nommé \myglos{glo-DockingMoleculaire}.
					Ce processus implique une analyse et une manipulation complexe reposant sur plusieurs expertises.
					Il est basé sur une décomposition en trois niveaux de modélisation, traités du niveau le plus grossier au niveau le plus fin :
					\begin{description}
						\item[Niveau inter-moléculaire] Cette déformation au niveau macro-moléculaire applique des transformations de grande amplitude sur chaque molécule.
							L'objectif est de trouver la meilleure concordance entre les molécule en terme de position et d'orientation.
						\item[Niveau intra-moléculaire] Cette déformation au niveau moléculaire fait suite à la déformation inter-moléculaire.
							L'amarrage de ces deux molécules (ou plus) introduit de nombreuses interfaces qui doivent être optimisées en fonction de critères variés (la complémentarité des surfaces, les forces électrostatiques, les forces de \myname[van der]{Waals} \mycite{Muller-1994}, \myetc).
						\item[Niveau atomique] Cette déformation très fine va chercher à optimiser la position des atomes au niveau de l'interface.
							L'intérêt de cette étape sera portée sur plusieurs types d'interaction (les ponts hydrogènes, les zones hydrophobiques et hydrophylliques, les ponts salins, \myetc).
					\end{description}

					Pour chacun de ces différents niveaux, le processus de manipulation est similaire et peut être séparé en sous-tâches :
					\begin{description}
						\item[Recherche] Cette tâche concerne l'identification et la recherche d'une cible (atome, \myglos{glo-Residu}, \myhelice*, \myfeuillet*, \myetc) en fonction de critères multiples (articulations, bilan énergétique, régions hydrophobique, \myetc).
						\item[Sélection] Une fois la cible trouvée, la tâche consiste à accéder puis à sélectionner la cible par l'intermédiaire d'un périphérique d'entrée (une souris, une interface haptique, \myetc).
						\item[Déformation] La tâche consiste à déformer la structure en manipulant la cible précédemment sélectionnée, que ce soit au niveau inter-moléculaire, intra-moléculaire ou atomique.
							L'objectif inhérent à cette tâche et d'atteindre l'objectif fixé (par exemple, minimiser l'énergie totale du système).
						\item[Évaluation] Cette dernière partie va évaluer le travail précédemment réalisé en observant différents indicateurs (énergie potentielle, énergie électrostatique, complémentarité des surfaces, \myetc).
							En fonction de la synthèse des résultats de cette dernière phase, un nouveau cycle pourra recommencer (recherche, sélection, déformation, évaluation, \myetc).
					\end{description}

				\end{mysubsection}
			\end{mysection}
		\end{mychapter}
		\begin{mychapter}[app-ShaddockSystemeCollaboratifPourLaManipulationDeMolecules]{\myShaddock\ -- Système collaboratif pour la manipulation de molécules}
			\begin{mysection}[sec-SimulationMoleculaireEnTempsReel]{Simulation moléculaire en temps-réel}
				Tout d'abord, la plate-forme \myShaddock doit permettre de manipuler des molécules.
				Nous commencerons par identifier les besoins dans la \myref{sse-simulation-LesBesoins} puis par exposer les outils utilisés dans la \myref{sse-simulation-LesOutilsExistants}.
				\begin{mysubsection}[sse-simulation-LesBesoins]{Les besoins}
					\myShaddock doit permettre d'effectuer de la manipulation de molécules en temps-réel.
					Il faut donc commencer par avoir la possibilité de visualiser les molécules.
					La visualisation est un processus complexe qui nécessite des rendus très variés et complet.
					Cette tâche sera effectuée par le logiciel \myacro{acr-VMD} \myref*{sss-simulation-VMD}.

					Ensuite, \myShaddock doit simuler un environnement moléculaire de façon physique.
					On fera appel à un logiciel de simulation pour ce besoin.
					Il faut que ce logiciel puisse interagir avec \myacro{acr-VMD} et \myacro{acr-NAMD} est le parfait candidat \myref*{sss-simulation-NAMD}.

					Cependant, \myacro{acr-NAMD} n'est pas conçu pour effectuer des simulations en temps-réel.
					La solution à ce problème vient avec \myacro{acr-IMD} développé par le \myacro{acr-ITAP} et présenté en \myref{sss-simulation-IMD}.
				\end{mysubsection}
				\begin{mysubsection}[sse-simulation-LesOutilsExistants]{Les outils existants}
					Trois outils permettent de fournir une simulation en temps-réel de molécules : \myacro{acr-VMD} pour la visualisation, \myacro{acr-NAMD} pour la simulation et \myacro{acr-IMD} pour la simulation en temps-réel.
					Ces trois outils sont présentés dans les sections suivantes.
					\begin{mysubsubsection}[sss-simulation-VMD]{\myacronl+{acr-VMD}}
						Les outils de visualisation moléculaire disponibles sont relativement nombreux.
						Parmi les plus populaires, on peut citer \myPyMOL \mycite{PyMOL-2010}, \myacro{acr-VMD} \mycite{Humphrey-1996}, \myChimera \mycite{Pettersen-2004}, \myRasmol \mycite{Sayle-1995} sans compter les nombreux dérivés permettant un affichage en ligne tel que \myJmol \mycite{Jmol-2006} par exemple pour ne citer que le plus connu.
						\myPyMOL et \myacro{acr-VMD} se distinguent particulièrement par leurs fonctionnalités et leur large utilisation dans le milieu spécialisé.

						\myPyMOL est probablement le logiciel de visualisation le plus utilisé par les experts du domaine car c'est le plus complet pour fournir des rendus graphiques élaborés de molécules.
						Cependant, \myPyMOL n'est pas adapté à l'affichage de simulation temps-réel et à la manipulation interactive, point nécessaire dans le cadre de notre étude et de nos expérimentations.
						\myacro{acr-VMD} possède également une large gamme de rendus graphiques possible.
						Contrairement à \myPyMOL, \myacro{acr-VMD} est adapté pour le rendu en temps-réel de simulation ainsi que pour la manipulation.
						Les fonctionnalités de \myacro{acr-VMD} sont nombreuses et seulement certaines on été utilisés dans le cadre des expérimentations qui vont suivre.

						\begin{myparagraph}[par-simulation-LesRendusGraphiques]{Les rendus graphiques}
							La possibilité d'avoir accès à des rendus graphiques divers et complets est primordiale pour la visualisation moléculaire.
							La complexité des molécules, le nombre important d'atomes, les nombreuses meta-informations, la structure particulière nécessite d'avoir à sa disposition des moyens évolués et variés pour afficher une molécule.
							Quatre représentations différentes \myref*{fig-simulation-IllustrationsDesRepresentationsDeMoleculesSurVMD} ont été utilisées sur la plate-forme \myShaddock :
							\begin{description}
								\item[\myCPK] affiche tous les atomes de la molécule sous forme de sphère en les reliant par des cylindres; c'est un affichage très chargé lorsque le nombre d'atomes est importants mais on peut modifier la taille des sphères et des cylindres \myref*{fig-simulation-CPK};
								\item[\myLicorice] représente tous les liens entre les atomes par des cylindres, sans représenter les atomes; la taille des cylindres peut être modifiée \myref*{fig-simulation-Licorice};
								\item[\myNewRibbon] produit une courbe spline sur les éléments $C_{\alpha}$ représentant l'armature principale de la molécule; la courbe est représentée sous forme de ruban avec une orientation \myref*{fig-simulation-NewRibbon};
								\item[\myHBonds] affiche les potentielles liaisons hydrogène sous forme de traits en pointillés; les seuils d'affichage ainsi que les paramètres de la ligne en pointillés (couleur, largeur, \myetc) sont modifiables \myref*{fig-simulation-HBonds}.
							\end{description}

							\begin{myfigure}
								\begin{mysubfigure}
									\myimage[width=0.49\textwidth]{simulation-CPK}
									\mysubcaption[fig-simulation-CPK]{\myCPK}
								\end{mysubfigure}
								\begin{mysubfigure}
									\myimage[width=0.49\textwidth]{simulation-Licorice}
									\mysubcaption[fig-simulation-Licorice]{\myLicorice}
								\end{mysubfigure}
								\begin{mysubfigure}
									\myimage[width=0.49\textwidth]{simulation-NewRibbon}
									\mysubcaption[fig-simulation-NewRibbon]{\myNewRibbon}
								\end{mysubfigure}
								\begin{mysubfigure}
									\myimage[width=0.49\textwidth]{simulation-HBonds}
									\mysubcaption[fig-simulation-HBonds]{\myHBonds}
								\end{mysubfigure}
								\mycaption[fig-simulation-IllustrationsDesRepresentationsDeMoleculesSurVMD]{Illustration des représentations de molécules sur \myacronl{acr-VMD}}
							\end{myfigure}

							Chacune de ces représentations visuelles peut être affectée à tout ou partie de la molécule comme \og le \myglos{glo-Residu} \mynum{13} \fg, \og seulement les atomes de carbone \fg, \og tous les \myglos*{glo-Residu} entre \mynum{1} et \mynum{16} sauf les atomes d'hydrogène \fg, \myetc
							De plus, pour chacune des représentations précédentes, différentes colorations sont possibles :
							\begin{description}
								\item[Couleur fixe] donne une couleur unie prédéfinie;
								\item[Couleur des atomes] donne une couleur différente à chaque atome selon un code couleur standard (rouge pour l'oxygène, blanc pour l'hydrogène, \myetc);
								\item[Couleur des \myglosnl*{glo-Residu}] donne une couleur différente pour chaque atome selon une palette de couleurs prédéfinie par \myacro{acr-VMD};
								\item[Transparence] rend transparent les objets tout en conservant la teinte colorée;
								\item[\textit{GoodSell}] accentuant les contours des objets sous le principe du \myemph{cell shading}.
							\end{description}
						\end{myparagraph}
						\begin{myparagraph}[par-simulation-LesOutilsDeManipulation]{Les outils de manipulation}
							Notre plate-forme doit permettre la manipulation des molécules et la possibilité d'avoir accès à des outils de manipulation est nécessaire.
							\myAcro{acr-VMD} dispose déjà de différents outils permettant d'effectuer différentes manipulation sur les molécules.

							Par défaut et sans configuration, la souris permet d'orienter la scène sur trois degrés de libertés afin d'observer la molécule sous différents angles.

							Il est également possible de bénéficier d'une souris \myThreeD, automatiquement détectée lorsqu'elle est branchée sur l'ordinateur.
							Une souris \myThreeD permet de translater et d'orienter la scène.
							La souris \myThreeD \mySpaceNavigator est utilisée dans le cadre de certaines de nos études.

							Enfin, des outils spécifiques sont disponibles par l'intermédiaire d'une connexion à travers le logiciel \myacro{acr-VRPN} \myref*{sss-interaction-VRPN}.
							Ces outils sont liés à des périphériques externes (des interfaces \myOmni dans le cas de notre étude).
							Les outils disponibles dans \myacro{acr-VMD} qui ont été utilisés dans la première expérimentation \myref*{cha-LaRechercheCollaborative} sont :
							\begin{description}
								\item[\mytool{grab}] qui permet de sélectionner une molécule dans son intégralité et de la déplacer dans la scène;
								\item[\mytool{tug}] qui permet de sélectionner un atome de la molécule et de lui appliquer une force (qui sera transmise à la simulation).
							\end{description}

							Cependant, de nombreux outils supplémentaires ont été développés au-fur-et-à-mesure des besoins identifiés durant les expérimentations.
							Ces nouveaux outils sont détaillés dans la \myref{sec-NouveauxOutilsPourLInteraction}.
						\end{myparagraph}
						\begin{myparagraph}[par-simulation-LaGenerationAutomatiqueDeFichierDeSimulation]{La génération automatique de fichier de simulation}
							La simulation nécessite différents fichiers d'information et notamment, l'ensemble des liaisons entre atomes, des angles simples, des angles dihédrals et des angles de torsion.
							Ce fichier peut être obtenu à partir d'un fichier de description de molécule (fichiers \myPDB).
							\myacro{acr-VMD} permet de générer ce fichier nécessaire à la simulation (fichier \myPSF) par l'intermédiaire d'une extension: \myemph{Automatic \textsc{psf} builder}.
						\end{myparagraph}
					\end{mysubsubsection}
					\begin{mysubsubsection}[sss-simulation-NAMD]{\myacronl+{acr-NAMD}}
						Les deux logiciels de simulation principaux existants sont \myacro{acr-NAMD} \mycite{Phillips-2005} et \myGromacs \mycite{Berendsen-1995}.
						Bien que \myGromacs soit plus performant que \myacro{acr-NAMD}, surtout dans les dernières versions \mycite{Hess-2008} qui est jusqu'à quatre fois plus rapide que \myacro{acr-NAMD}.
						Cependant, \myacro{acr-NAMD} est dévéloppé par la même université que \myacro{acr-VMD} et l'interaction entre les deux logiciels est donc extrêmement facilitée.
						De plus, les petites molécules que nous utiliserons lors de nos simulations ne nécessitent pas des performances exceptionnelles.
						Enfin, \myacro{acr-NAMD} peut être aisément connecté à \myacro{acr-VMD} dans le cadre d'une simulation interactive \myref*{sss-simulation-IMD} contrairement à \myGromacs.
						C'est pourquoi le logiciel \myacro{acr-NAMD} a été retenu pour notre plate-forme.

						Une des fonctionnalités de \myacro{acr-NAMD} utilisée est la possibilité de \myemph{fixer} des atomes.
						En effet, la fixation d'atomes permet d'avoir certains atomes de la molécules qui ne sont soumis à aucune force de la simulation.
						Bien évidemment, c'est un choix qui perturbe la simulation physique mais c'est un moindre mal.
						De plus, cette fonctionnalité est nécessaire pour avoir un point d'ancrage de la molécule dans l'environnement virtuel.
						Sans ce point d'ancrage, la molécule pourrait dériver et sortir de l'espace de travail des utilisateurs.
					\end{mysubsubsection}
					\begin{mysubsubsection}[sss-simulation-IMD]{\myacronl+{acr-IMD}}
						Les logiciels de simulation ne sont pas développés afin d'effectuer des simulations interactives en temps-réel.
						Cependant, le \myacro{acr-ITAP} a développé le protocole \myacro{acr-IMD} permettant d'utiliser \myacro{acr-NAMD} couplé à \myacro{acr-VMD} pour des simulations interactives en temps-réel \mycite{Stadler-1997}.
						Une extension au sein du logiciel \myacro{acr-VMD} permet de le connecter facilement avec \myacro{acr-NAMD}.

						Cependant, entre le début du développement de notre plate-forme en 2008 et aujourd'hui, une nouvelle solution plus générique a été développée au sein de l'\myacro{acr-IBPC}.
						En effet, \myMDDriver \mycite{Delalande-2009} est un petit logiciel permettant d'utiliser le protocole \myacro{acr-IMD} avec l'autre logiciel de simulation \myGromacs.
						Il se présente sous forme d'interface permettant de choisir le logiciel de simulation ainsi que le logiciel de visualisation.
						Il permet également de connecter plusieurs logiciels de visualisation à une même simulation.
						Cependant, cette nouvelle solution n'a pas encore été implémentée dans notre plate-forme mais c'est une amélioration technique qui sera effectuée dans les prochaines versions de la plate-forme.
					\end{mysubsubsection}
				\end{mysubsection}
			\end{mysection}
			\begin{mysection}[sec-InteractionAvecLaSimulationMoleculaire]{Interaction avec la simulation moléculaire}
				\begin{mysubsection}[sse-interaction-LesBesoins]{Les besoins}
				\end{mysubsection}
				\begin{mysubsection}[sse-interaction-InteractionParInterfaceHaptique]{Interaction par interface haptique}
					\begin{mysubsubsection}[sss-interaction-OmniEtOpenHaptics]{\myOmni et \myOpenHaptics}
						Une plate-forme de simulation interactive en temps-réel nécessite des outils d'interaction.
						Parmi les périphériques d'interaction existants, il faut choisir un périphérique permettant au minimum six \myacro*{acr-DDL} en entrée et au minimum trois \myacro*{acr-DDL} en retour haptique.
						En effet, l'outil \mytool{grab} nécessite six \myacro*{acr-DDL} en entrée et l'outil \mytool{tug} nécessite trois \myacro*{acr-DDL} en entrée et en retour haptique.
						L'interface \myOmni \mycite{Massie-1994} de l'entreprise \mySensAble répond aux attentes de la plate-forme \myref*{fig-simulation-InterfaceOmniSixDDLTroisDDL}.

						\begin{myfigure}
							\myimage{simulation-omni}
							\mycaption[fig-simulation-InterfaceOmniSixDDLTroisDDL]{Interface \myOmni 6~\myacronl-{acr-DDL}/3~\myacronl-{acr-DDL}}
						\end{myfigure}

						À l'origine, les interfaces haptiques de \mySensAble était programmable à l'aide de l'\myacro{acr-API} \myGhost \mycite{SensAble-2002}.
						Le travail de \mycite[author]{Itkowitz-2005} a permis de fournir une nouvelle \myacro{acr-API} plus facile à utiliser : \myOpenHaptics.
						C'est à partir de cette \myacro{acr-API} que les interfaces haptiques sont utilisées sur la plate-forme.
					\end{mysubsubsection}
					\begin{mysubsubsection}[sss-interaction-VRPN]{\myacronl+{acr-VRPN}}
						\myacro{acr-VMD} offre un moyen simple et relativement universel de connecter un périphérique.
						En effet, il gère le connexions par l'intermédiaire de \myacro{acr-VRPN} \mycite{Taylor-II-2001}.
						\myacro{acr-VRPN} fonctionne sous la forme d'une architecture client/serveur.
						\myacro{acr-VMD} est l'application cliente.
						L'interface haptique est connectée physiquement à un autre ordinateur (le même ordinateur le cas échéant) et un serveur \myacro{acr-VRPN} communique avec cette interface.
						C'est seulement par l'intermédiaire de \myacro{acr-VRPN} et à travers le réseau \myemph{Ethernet} que \myacro{acr-VMD} va percevoir les mouvements de l'interface haptique et lui envoyer des efforts à fournir.
						La compilation de \myacro{acr-VRPN} en tant que serveur de \myOmni sous le système d'exploitation \myLinux (\myUbuntu $v10.04$) a demandé quelques modifications dans le code.
						Ces modifications sont à présent disponibles dans les dernières versions de \myacro{acr-VRPN}.

						L'avantage de cette architecture est la possibilité d'ajouter autant de serveurs et donc autant d'interfaces haptiques que voulu.
						Cependant, cela suppose également d'avoir autant d'ordinateurs que de serveurs ce qui complique la logistique.
						On pourra noter que la chaleur dégagée par l'ensemble de ces machines additionnée à celle du vidéo-projectuer provoquait des conditions d'expérimentation rapidement désagréables.
						C'est pourquoi, aucune des expérimentations proposées ne durait plus de \mynum[mn]{30}.
					\end{mysubsubsection}
				\end{mysubsection}
			\end{mysection}
			\begin{mysection}[sec-NouveauxOutilsPourLInteraction]{Nouveaux outils pour l'interaction}
				Durant les différentes études présentées dans la \myref{prt-EtudeDuTravailCollaboratif}, les analyses et les remarques d'utilisateurs ont permis d'améliorer les outils d'interaction et d'en proposer des nouveaux.
				Le développement de ces nouveaux outils a nécessité une modification du programme \myacro{acr-VMD} par l'extension du nombre d'outils disponibles.
				Des fonctionnalités ont été ajoutées et sont présentées dans les sections suivantes.
				\begin{mysubsection}[sse-AmeliorationDeLaSelection]{Amélioration de la sélection}
					Durant le processus de recherche et de sélection, les utilisateurs ont souvent évoqué le besoin de connaître en continu leur position et de savoir à priori l'élément qui va être sélectionné.
					Pour que les utilisateurs connaissent à chaque instant l'élément qui peut être sélectionné, une information visuelle met en surbrillance l'élément pointé à chaque instant.
					La mise en surbrillance est effectué par l'intermédiaire d'un agrandissement en transparence de l'élément pointé.
					La couleur de cette mise en surbrillance est de la même couleur que le curseur de l'utilisateur.

					Cependant, lorsque seul un atome est mis en surbrillance, il peut être difficile de l'apercevoir à cause de l'affichage chargé dû au grand nombre d'atomes.
					C'est pourquoi, l'ensemble du \myglos{glo-Residu} auquel appartient l'atome pointé est également mis en surbrillance.
					Seul l'atome pointé est agrandi.

					Enfin, lorsque les utilisateurs sélectionnent l'élément, la surbrillance passera de la transparence à de l'opacité.
					Une illustration des effets visuels relatifs au pointage et à la sélection est affiché sur la \myref{fig-selection-improvement-DifferenceVisuelleEntreLesElementsPointesEtSelectionnes}.

					\begin{myfigure}
						\begin{mysubfigure}
							\myimage[width=0.49\textwidth]{selection-improvement-pointed}
							\mysubcaption[fig-selection-improvement-ElementPointe]{Élément pointé}
						\end{mysubfigure}
						\begin{mysubfigure}
							\myimage[width=0.49\textwidth]{selection-improvement-targeted}
							\mysubcaption[fig-selection-improvement-ElementSelectionne]{Élément sélectionné}
						\end{mysubfigure}
						\mycaption[fig-selection-improvement-DifferenceVisuelleEntreLesElementsPointesEtSelectionnes]{Différence visuelle entre les éléments pointés et sélectionnés}
					\end{myfigure}
				\end{mysubsection}
				\begin{mysubsection}[sse-DeformationParGroupeDAtomes]{Déformation par groupe d'atomes}
					L'outil \mytool{tug} permet de déformer la molécule en appliquant un effort à l'atome sélectionné.
					Cependant, la déformation par l'intermédiaire d'un seul atome possède deux désavantages.

					Tout d'abord, la déformation d'une molécule atome par atome est un processus très fastidieux.
					Il serait plus efficace de déplacer un groupe d'atomes en une seule fois.

					De plus, l'application d'un effort sur un atome provoque l'étirement de la molécule.
					Au repos, la molécule est dans état relativement stable.
					Étirer un atome perturbe cet état de stabilité.
					Pourtant, certains atomes sont fortement liés et les séparer n'est pas désirable.
					Il serait donc préférable de déplacer tous ces atomes en une seule manipulation.

					C'est pourquoi un outil permettant d'appliquer un effort à un groupe d'atomes permettrait de déplacer un bloc d'atomes tout en conservant plus de stabilité.
					Les groupes d'atomes dignes d'intérêt sont les \myglos*{glo-Residu} (une vingtaine d'atomes), les \myhelice* ou \myfeuillet* (une vingtaine de \myglos*{glo-Residu}) et les molécules.
					Cependant, \myacro{acr-VMD} ne permet de regrouper les atomes selon les \myhelice* ou \myfeuillet*.

					La fonctionnalité de l'outil \mytool{tug} a donc été étendue aux \myglos*{glo-Residu} et aux molécules.
					Appliquer le même effort à l'ensemble des atomes de la molécule produit un effort total très important.
					Si l'effort total est trop important, les perturbations envoyées à la simulation sont trop importantes et produisent des incohérences dans la simulation voire même un arrêt de la simulation.
					Il est donc nécessaire de diviser l'intensité des forces proportionnellement au nombre d'atomes sélectionnés.
				\end{mysubsection}
				\begin{mysubsection}[sse-OutilDeDesignationEtAttraction]{Outil de désignation et attraction}
					Un aspect récurrent constaté durant les expérimentations est la nécessité de désigner un élément de la molécule.
					Parfois les utilisateurs éprouvent le besoin de désigner mais la plupart du temps, ce sont les enregistrement audio qui ont permis d'identifier ce besoin.

					L'outil de désignation a été conçu avec un processus en quatre étapes:
					\begin{enumerate}[label={\alph*.},ref={\alph*}]
						\item Recherche d'une cible \myref*{fig-designation-RechercheDUneCible};\label{enu-designation-RechercheDUneCible}
						\item Désignation d'une cible \myref*{fig-designation-DesignationDUneCible};\label{enu-designation-DesignationDUneCible}
						\item Acceptation d'une cible \myref*{fig-designation-AcceptationDUneCible};\label{enu-designation-AcceptationDUneCible}
						\item Sélection d'une cible \myref*{fig-designation-SelectionDUneCible}.\label{enu-designation-SelectionDUneCible}
					\end{enumerate}

					\begin{myfigure}
						\begin{mysubfigure}
							\myimage[width=0.49\textwidth]{designation-normal}
							\mysubcaption[fig-designation-RechercheDUneCible]{Recherche d'une cible}
						\end{mysubfigure}
						\begin{mysubfigure}
							\myimage[width=0.49\textwidth]{designation-called}
							\mysubcaption[fig-designation-DesignationDUneCible]{Cible désigné}
						\end{mysubfigure}
						\begin{mysubfigure}
							\myimage[width=0.49\textwidth]{designation-accepted}
							\mysubcaption[fig-designation-AcceptationDUneCible]{Cible acceptée}
						\end{mysubfigure}
						\begin{mysubfigure}
							\myimage[width=0.49\textwidth]{designation-selected}
							\mysubcaption[fig-designation-SelectionDUneCible]{Cible sélectionnée}
						\end{mysubfigure}
						\mycaption[fig-designation-LesQuatreEtapesDeLaDesignation]{Les quatre étapes de la désignation}
					\end{myfigure}

					L'\myref{enu-designation-RechercheDUneCible} consiste pour un utilisateur~\myuser{A} à rechercher une cible à désigner.
					Cette recherche sera réalisée selon différents critères comme par exemple, la nécessité d'obtenir de l'aide d'un utilisateur \myuser{B}.

					L'\myref{enu-designation-DesignationDUneCible} permet à l'utilisateur~\myuser{A} de désigner la cible identifiée.
					La cible est alors mise en surbrillance de façon à être vue des autres utilisateurs.

					L'\myref{enu-designation-AcceptationDUneCible} fait intervenir l'utilisateur~\myuser{B}.
					L'utilisateur~\myuser{B} peut accepter ou non cette désignation.
					S'il accepte la désignation, la cible est alors colorée de la couleur du curseur de l'utilisateur~\myuser{B} qui a accepté.

					L'\myref{enu-designation-SelectionDUneCible} est la dernière étape.
					L'utilisateur~\myuser{B} ayant accepté doit maintenant sélectionner la cible pour achever le processus de désignation.
					
					En supplément, des aides haptiques ont été ajoutées lors de la dernière expérimentation \myref*{cha-TravailCollaboratifAssisteParHaptique}.
					Ce sont ces aides haptiques qui sont testées.
					Pour l'\myref{enu-designation-DesignationDUneCible}, des vibrations sont générées sur tous les utilisateurs concernés par la désignation.
					De plus, dès l'instant qu'un utilisateur a accepté la désignation \myref*{enu-designation-AcceptationDUneCible}, il est guidé vers la cible.
					La vibration chez tous les autres utilisateurs est arrêtée.
				\end{mysubsection}
			\end{mysection}
		\end{mychapter}
	\end{mypart}
	\begin{mypart}[prt-EtudeDuTravailCollaboratif]{Étude du travail collaboratif}
		\begin{mychapter}[cha-LaRechercheCollaborative]{La recherche collaborative}
			\begin{mysection}[sec-exp1-Presentation]{Présentation}
				\begin{mysubsection}[sse-exp1-Objectifs]{Objectifs}
					Dans cette première expérimentation, nous proposons d'étudier les deux premières des quatre sous-tâches élémentaires \myref*{sse-LAmarrageMoleculaire}: la \myemph{recherche} et la \myemph{sélection}.
					Ces sous-tâches sont cruciales car elles ont un impact important sur les sous-tâches suivantes que sont la \myemph{déformation} et la \myemph{manipulation}.
					Les difficultés liées à la complexité de l'environnement virtuel moléculaire seront étudiées à travers cette étude.

					Cette première expérimentation a pour objectif principal de comparer un \myglos{glo-Monome} et un \myglos{glo-Binome}.
					Plusieurs facteurs seront étudiés lors de cette comparaison.

					Le premier facteur concerne les performances.
					Les performances représentent à la fois le temps total pour réaliser la tâche mais aussi les ressources mises en place pour accéder à ce résultat.
					Un \myglos{glo-Binome} sera-t-il plus performant qu'un \myglos{glo-Monome} ?

					Le second facteur concerne les méthodes et les stratégies de travail.
					C'est principalement l'évolution de ces stratégies au sein des \myglos*{glo-Binome} qui focalisera notre attention.
					Le travail en \myglos{glo-Binome} permettra de mettre en avant différentes stratégies de travail discriminées en fonction de la communication, des espaces de travail, de la répartition des tâches, \myetc

					Enfin, il est nécessaire de valider la plate-forme de manipulation proposée.
					Pour cela, l'évaluation sera principalement confiée aux sujets qui noteront la plate-forme.
					L'objectif est ici de vérifier la pertinence de la plate-forme mais aussi d'en améliorer les points faibles.
				\end{mysubsection}
				\begin{mysubsection}[sse-exp1-Hypotheses]{Hypothèses}
					\begin{myparagraph}[par-exp1-AmeliorationDesPerformancesEnBinome]{\myhypothesis{1} Amélioration des performances en \myglosnl{glo-Binome}}
						La première hypothèse est une amélioration des performances des \myglos*{glo-Binome} face aux performances des \myglos*{glo-Monome}.
						L'évaluation des performances sera principalement le temps de réalisation de la tâche.
					\end{myparagraph}
					\begin{myparagraph}[par-exp1-StrategiesVariablesEnFonctionDesBinomes]{\myhypothesis{2} Stratégies variables en fonction des \myglosnl*{glo-Binome}}
						Cette second hypothèse concerne uniquement les \myglos*{glo-Binome} et suppose que les stratégies adoptées seront différentes en fonction des \myglos*{glo-Binome}.
						Cette différence sera liée aux différentes personnalités et aux différentes affinités au sein du \myglos{glo-Binome}.
						L'identification des différentes stratégies permettra des les évaluer et de trouver celles qui donnent les meilleurs résultats.
					\end{myparagraph}
					\begin{myparagraph}[par-exp1-LesSujetsPreferentLeTravailEnBinome]{\myhypothesis{3} Les sujets préfèrent le travail en \myglosnl{glo-Binome}}
						La troisième hypothèse est de l'ordre du qualitatif.
						Elle s'intéresse aux conditions de travail en \myglos{glo-Binome}.
						L'hypothèse est basée sur l'effet stimulant de travailler à plusieurs mais aussi sur l'intérêt de la collaboration sur des tâches répétitives.
						Il est cependant important que chaque sujet au sein du \myglos{glo-Binome} se considère utile à la réalisation de la tâche.
					\end{myparagraph}
				\end{mysubsection}
			\end{mysection}
			\begin{mysection}[sec-exp1-DispositifExperimentalEtMateriel]{Dispositif expérimental et matériel}
				L'\myacro{acr-EVC} utilisé est illustré sur la \myref{fig-exp1-IllustrationDuDispositifExperimental}.
				L'\myacro{acr-EVC} propose une visualisation partagée sur grand écran (vue publique à tous les utilisateurs) à l'aide d'un vidéoprojecteur.
				Le ou les sujets font face à l'écran avec à leur disposition :
				\begin{itemize}
					\item un interface haptique de manipulation \mytool{grab};
					\item deux interfaces haptiques de déformation \mytool{tug}.
				\end{itemize}

				\begin{myfigure}
					\begin{mysubfigure}
						\myimage[width=0.49\textwidth]{exp1-schema}
						\mysubcaption[fig-exp1-IllustrationDuDispositifExperimental-SchemaDuDispositifExperimental]{Schéma du dispositif expérimental}
					\end{mysubfigure}
					\begin{mysubfigure}
						\myimage[width=0.49\textwidth]{exp1-photo}
						\mysubcaption[fig-exp1-IllustrationDuDispositifExperimental-PhotographieDuDispositifExperimental]{Photographie du dispositif expérimental}
					\end{mysubfigure}
					\mycaption[fig-exp1-IllustrationDuDispositifExperimental]{Illustration du dispositif expérimental}
				\end{myfigure}

				Les sujets ont la possibilité de communiquer entre eux sans restriction.
				Pour les \myglos*{glo-Monome}, le sujet a à sa disposition un outil de manipulation \myemph{grab} et un outil de déformation \myemph{tug} qu'il peut utiliser librement.
				Pour les \myglos*{glo-Binome}, chaque sujet se voit attribuer un outil de déformation \mytool{tug}.
				L'outil de manipulation \mytool{grab} est attribué à un seul des deux sujets après une négociation au sein du \myglos{glo-Binome}.
				Le sujet choisi pour gérer l'outil de manipulation \mytool{grab} le sera pour toute la durée de l'expérimentation.

				Un micro de bureau est placé en face des sujets afin de capter toutes les communications orales.
				L'enregistrement, réalisé à l'aide du logiciel Audacity, débute à la fin de la phase d'apprentissage jusqu'à la fin de l'expérimentation.
				Un découpage par tâche est réalisé en post-traitement.

				Pour les détails techniques concernant la plate-forme et les outils de manipulation et de déformation, se reporter au \myref{app-ShaddockSystemeCollaboratifPourLaManipulationDeMolecules}.
			\end{mysection}
			\begin{mysection}[sec-exp1-Methode]{Méthode}
				\begin{mysubsection}[sse-exp1-Sujets]{Sujets}
					\mynum{24}~sujets (\mynum{4}~femmes et \mynum{20}~hommes) avec une distribution d'âge de \mysummary{exp1-age.tex} ont participé à cette expérimentation.
					Ils ont tous été recrutés au sein du laboratoire \myacro{acr-LIMSI} et sont chercheurs ou assistants de recherche dans les domaines suivants~:
					\begin{itemize}
						\item linguistique et traitement automatique de la parole;
						\item réalité virtuelle et système immersifs;
						\item audio-acoustique.
					\end{itemize}
					Ils ont tous le français comme langue principale.
					Aucun participant n'a de déficience visuelle (ou corrigée le cas échéant), de déficience audio ou de déficience moteur du haut du corps.
					Les sujets ne sont pas rémunérés pour l'expérimentation.

					Chaque participant est complètement naïf concernant les détails de l'expérimentation.
					Une explication détaillée de la procédure expérimentale leur est donnée au commencement de l'expérimentation mais en omettant l'objectif de l'étude.
				\end{mysubsection}
				\begin{mysubsection}[sec-exp1-Variables]{Variables}
					\begin{mysubsubsection}[sss-exp1-VariablesIndependantes]{Variables indépendantes}
						\begin{myparagraph}[par-exp1-NombreDeSujets]{\myvari{1} Nombre de sujets}
							La première \myglos{glo-VariableIndependante} est une \myglos{glo-VariableIntraPopulation}.
							\myvari{1} possède deux valeurs possibles: \og un sujet \fg (\mycf \myemph{\myglos{glo-Monome}}) ou \og deux sujets \fg (\mycf \myemph{\myglos{glo-Binome}}).
							\mynum{24}~\myglos*{glo-Monome} et \mynum{12}~\myglos*{glo-Binome} ont été testés ce qui fait deux fois plus de \myglos*{glo-Monome} que de \myglos*{glo-Binome}.
						\end{myparagraph}
						\begin{myparagraph}[par-exp1-ResiduRecherche]{\myvari{2} \myGlosnl{glo-Residu} recherché}
							La seconde \myglos{glo-VariableIndependante} est une \myglos{glo-VariableIntraPopulation}.
							\myvari{2} concerne les \myglos*{glo-Residu} recherchés qui sont au nombre de \mynum{10} répartis à part égale dans deux molécules \myref*{tab-exp1-ListeDesResidusRecherches}.
						\end{myparagraph}
					\end{mysubsubsection}
					\begin{mysubsubsection}[sec-exp1-VariablesDependantes]{Variables dépendantes}
						\begin{myparagraph}[par-exp1-LeTempsDeCompletion]{\myvard{1} Le temps de complétion}
							Ce temps est le temps total pour réaliser la tâche demandée, c'est-à-dire trouver le \myglos{glo-Residu} et l'extraire de la molécule.
							Il n'y a pas de limite de temps pour réaliser la tâche.
							Ce temps est divisé en deux phases bien distinctes :
							\begin{description}
								\item[La recherche] C'est la phase pendant laquelle les sujets cherchent le \myglos{glo-Residu}.
									Cette recherche peut être visuelle en orientant et en déplaçant la molécule.
									Elle peut aussi amener les sujets à déformer la molécule afin d'explorer les \myglos{glo-Residu} inaccessibles du centre de la molécule.
								\item[La sélection] La phase de sélection débute dès l'instant où un des deux sujets a identifié visuellement le \myglos{glo-Residu}.
									Elle est constituée d'une phase de sélection puis d'une phase d'extraction.
							\end{description}
						\end{myparagraph}
						\begin{myparagraph}[par-exp1-LaDistanceEntreLesEspacesDeTravail]{\myvard{2} La distance entre les espaces de travail}
							Cette distance est la distance moyenne entre les deux \myglos*{glo-EffecteurTerminal} présents durant l'expérimentation.
							Elle est mesurée dans le monde réel mais peut être convertie dans l'environnement virtuel (à l'échelle de la molécule).
							Cette distance est de l'ordre du centimètre.
						\end{myparagraph}
						\begin{myparagraph}[par-exp1-LesCommunicationsVerbales]{\myvard{3} Les communications verbales}
							L'enregistrement audio permet de mesurer la quantité de temps de parole pour chaque tâche de l'expérimentation.
							Ces mesures différencie la phase de recherche et la phase de sélection (voir \myvard{1}) comme indiqué plus précisément sur la \myref{fig-exp1-SchemaDesPhasesDeLaCommunicationVerbale}.

							\begin{myfigure}
								\psset{unit=0.1\textwidth} % Fill entirely the page width
								\begin{myps}(0,-1.75)(10,1.5)
									\psset{linewidth=1pt,linecolor=black}%
									%\psframe(0,-0.5)(10,0.5)%
									\psset{fillstyle=solid}%
									\psframe[fillcolor=mylightblue](0,-0.5)(6,0.5)%
									\pspolygon[fillcolor=mylightred](6,-0.5)(6,0.5)(9,0.5)(10,0)(9,-0.5)%
									\uput{16pt}[180](10,0){\LARGE\sl\textcolor{white!33}{temps}}
									\psbrace[ref=lC,rot=-90,nodesepA=-3,nodesepB=-0.25](6,0.5)(0,0.5){%
										\parbox{6\psxunit}{%
											\centering\textcolor{myblue}{Temps de recherche}%
										}%
									}%
									\psbrace[ref=lC,rot=-90,nodesepA=-2,nodesepB=-0.25](10,0.5)(6,0.5){%
										\parbox{4\psxunit}{%
											\centering\textcolor{myred}{Temps de sélection}%
										}%
									}%
									\psframe[fillcolor=myblue](1,-0.5)(1.5,0.5)
									\psframe[fillcolor=myblue](3,-0.5)(4.5,0.5)
									\psframe[fillcolor=myblue](4.8,-0.5)(5,0.5)
									\psframe[fillcolor=myred](6.5,-0.5)(7.5,0.5)
									\psframe[fillcolor=myred](8,-0.5)(8.25,0.5)
									\pnode(1.25,-0.5){verbal1}
									\pnode(3.75,-0.5){verbal2}
									\pnode(4.9,-0.5){verbal3}
									\pnode(7,-0.5){verbal4}
									\pnode(8.125,-0.5){verbal5}
									\rput(5,-1.5){%
										\Rnode{verbal}{%
											\psframebox[linestyle=none]{\centering Communication verbale}%
										}%
									}%
									\psset{linearc=0.1,angleA=-90}
									\ncdiagg{<-}{verbal1}{verbal}
									\ncdiagg{<-}{verbal2}{verbal}
									\ncdiagg{<-}{verbal3}{verbal}
									\ncdiagg{<-}{verbal4}{verbal}
									\ncdiagg{<-}{verbal5}{verbal}
								\end{myps}
								\mycaption[fig-exp1-SchemaDesPhasesDeLaCommunicationVerbale]{Schéma des phases de la communication verbale}
							\end{myfigure}
						\end{myparagraph}
						\begin{myparagraph}[par-exp1-LAffiniteEntreLesSujets]{\myvard{4} L'affinité entre les sujets}
							Le degré d'affinité -- concernant uniquement les \myglos*{glo-Binome} -- est compris entre \mynum{1} et \mynum{5} selon les critères suivants :
							\begin{enumerate}
								\item Les sujets ne se connaissent pas;
								\item Les sujets travaillent dans la même entreprise, le même laboratoire;
								\item Les sujets travaillent dans la même équipe, sur les mêmes projets;
								\item Les sujets travaillent dans ensemble, sont dans le même bureau;
								\item Les sujets sont amis proches.
							\end{enumerate}
						\end{myparagraph}
						\begin{myparagraph}[par-exp1-ReponsesQualitatives]{\myvard{5} Réponses qualitatives}
							Un questionnaire est proposé à tous les sujets.
							Il est constitué de plusieurs questions (notées sur échelle de \mycite[author]{Likert-1932} à cinq niveaux).

							Le questionnaire est le suivant (les questions sont posées à chaque sujet dans le cas du \myglos{glo-Binome}) :
							\begin{enumerate}
								\item Dans quelle configuration vous êtes-vous senti le plus efficace : \myemph{seul} ou \myemph{en collaboratif} (choisissez \mynum{1} pour \myemph{seul} et \mynum{5} pour \myemph{en collaboratif}) ?
								\item Avez-vous eu le sentiment de travailler en collaboration (par opposition au travail seul) ?
								\item Pensez-vous avoir une position de meneur dans la configuration collaborative ?
								\item Vous êtes vous senti utile dans la configuration collaborative (par opposition à pénalisant) ?
								\item Comment évalueriez-vous votre taux de communication\dots{}
									\begin{itemize}
										\item verbale ?
										\item gestuelle ?
										\item virtuelle ?
									\end{itemize}
								\item Avez-vous trouvé les différents effets visuels intuitifs ?
								\item Avez-vous trouvé les différents effets visuels confortables ?
								\item Avez-vous trouvé les interactions intuitives ?
								\item Avez-vous trouvé les interactions confortables ?
							\end{enumerate}

							Concernant la communication, les communications verbales concernent tous les échanges, dialogues exposés par la voix.
							La communication gestuelle représente les gestes que les sujets peuvent effectuer dans le monde réel pour expliquer ou pour désigner par exemple.
							Enfin, la communication virtuelle concerne les informations données au partenaire par l'intermédiaire de l'environnement virtuel (par exemple, une désignation avec le curseur).
						\end{myparagraph}
					\end{mysubsubsection}
				\end{mysubsection}
				\begin{mysubsection}[sse-exp1-Tache]{Tâche}
					La tâche proposée est la recherche et la sélection dans un \myacro{acr-EVC} sur des molécules complexes.
					Les motifs à rechercher dans les structures moléculaires sont les \myglos*{glo-Residu} de la \myref{tab-exp1-ListeDesResidusRecherches}.
					Une fois le \myglos{glo-Residu} trouvé, les sujets doivent le sélectionner puis l'extraire hors de la sphère virtuelle englobant la molécule.
					Les sujets possèdent deux outils pour trouver, sélectionner puis extraire ces motifs :
					\begin{itemize}
						\item ils peuvent explorer la molécule en la déplaçant ou en la tournant à l'aide de l'outil \mytool{grab};
						\item ils peuvent déformer la molécule à l'aide de l'outil \mytool{tug}.
					\end{itemize}

					\begin{mytable}
						\mycaption[tab-exp1-ListeDesResidusRecherches]{Liste des \myglosnl*{glo-Residu} recherchés}
						\setlength{\myheight}{10ex}
						\newcommand{\mypatternpicture}[1]{\myimage[width=\myheight]{exp1-#1}}
						\begin{mysubtable}
							\mysubcaption[tab-exp1-ListeDesResidusRecherches-ResidusSurLaMoleculeTRPCAGE]{Residus sur la molécule \myTRPCAGE}
							\begin{mytabular}[0.49\textwidth]{^C-C}
								\mytoprule
								\myrowstyle{\bfseries}
								\myGlosnl{glo-Residu} & Image \\
								\mymiddlerule
								\myresidue{1} & \mypatternpicture{pattern1} \\
								\myresidue{2} & \mypatternpicture{pattern2} \\
								\myresidue{3} & \mypatternpicture{pattern3} \\
								\myresidue{4} & \mypatternpicture{pattern4} \\
								\myresidue{5} & \mypatternpicture{pattern5} \\
								\mybottomrule
							\end{mytabular}
						\end{mysubtable}
						\begin{mysubtable}
							\mysubcaption[tab-exp1-ListeDesResidusRecherches-ResidusSurLaMoleculePrion]{Residus sur la molécule \myPrion}
							\begin{mytabular}[0.49\textwidth]{^C-C}
								\mytoprule
								\myrowstyle{\bfseries}
								\myGlosnl{glo-Residu} & Image \\
								\mymiddlerule
								\myresidue{6}  & \mypatternpicture{pattern6}  \\
								\myresidue{7}  & \mypatternpicture{pattern7}  \\
								\myresidue{8}  & \mypatternpicture{pattern8}  \\
								\myresidue{9}  & \mypatternpicture{pattern9}  \\
								\myresidue{10} & \mypatternpicture{pattern10} \\
								\mybottomrule
							\end{mytabular}
						\end{mysubtable}
					\end{mytable}

					La première molécule nommée \myTRPCAGE \mycite{Neidigh-2002} a pour identifiant \myPDB \myPDBlink{http://www.rcsb.org/pdb/explore/explore.do?structureId=1L2Y}{1L2Y} sur \myPDBbase\footnote{\url{http://www.pdb.org/}}.
					La seconde molécule nommée \myPrion \mycite{Christen-2009} avec l'identifiant \myPDB \myPDBlink{http://www.rcsb.org/pdb/explore/explore.do?structureId=2KFL}{2KFL}.
					Cinq \myglos*{glo-Residu} à chercher sont présents sur chaque molécule \myref*{fig-exp1-RepartitionDesResidusSurLesMolecules} et chacun présente différents niveaux de complexité.
					Les critères de complexité, résumés pour chaque \myglos{glo-Residu} dans la \myref{tab-exp1-ParametresDeComplexiteDesResidus}, sont les suivants :
					\begin{description}
						\item[Position] La position du \myglos{glo-Residu} peut se trouver sur le pourtour de la molécule (position \myemph{externe}) ou à l'intérieur, au milieu de l'amas d'atomes que constitue la molécule (position \myemph{interne}).
							Un \myglos{glo-Residu} en position externe ne nécessite pas de déformer la molécule pour le trouver et l'atteindre contrairement à un \myglos{glo-Residu} en position interne qui sera plus complexe d'accès.
						\item[Forme] La forme du \myglos{glo-Residu} influe énormément sur la complexité de la recherche.
							On distingue trois formes différentes :
							\begin{description}
								\item[Chaîne] Une chaîne d'atomes (la plupart du temps carbonés) avec des atomes d'hydrogène de part et d'autres.
								\item[Cycle] Une chaîne d'atomes de carbone ou d'azote qui boucle sur elle-même.
								\item[Étoile] Séries de chaînes d'atomes toutes reliées sur un atome central (un atome de carbone pour la plupart du temps).
							\end{description}
						\item[Couleurs] Les atomes sont colorés en fonction de leur nature (rouge pour l'oxygène, blanc pour l'hydrogène, \myetc).
							Les atomes rares seront donc rapidement trouvés grâce à leur couleur différente.
							Par contre, les atomes nombreux (comme les hydrogènes ou les carbones) seront plus difficiles à filtrer à cause de leur fréquence importante.
						\item[Similarité] Certains \myglos*{glo-Residu} à chercher sont très similaires à d'autres \myglos*{glo-Residu} également présents sur la molécule.
							Les \myglos*{glo-Residu} similaires possèdent un atome de moins ou de plus par rapport au \myglos{glo-Residu} recherché.
							De par leur similarité, ils vont mobiliser les capacités de recherche des sujets sur des \myglos*{glo-Residu} ressemblants mais incorrects.
					\end{description}

					\begin{myfigure}
						\newcommand{\schemafactor}{0.20}
						\newlength{\schemaunit}\setlength{\schemaunit}{\schemafactor\textwidth}
						\psset{unit=\schemaunit}
						\mycaption[fig-exp1-RepartitionDesResidusSurLesMolecules]{Répartition des \myglosnl*{glo-Residu} sur les molécules}
						\begin{myps}(-2.5,-3)(2.5,3)
							\rput(0,1.75){%
								\myimage[height=2\schemaunit,angle=90]{exp1-trp-cage}}
							\rput(0,-1.25){%
								\myimage[height=2\schemaunit,angle=90]{exp1-prion}}
							\rput(-1.5,2){%
								\myimage[height=\schemaunit]{exp1-pattern1}}
							\rput(1.5,2){%
								\myimage[width=\schemaunit]{exp1-pattern3}}
							\rput(1.5,-0){%
								\myimage[width=\schemaunit]{exp1-pattern2}}
							\rput(-1.5,-0){%
								\myimage[width=\schemaunit]{exp1-pattern4}}
							\rput(-1.5,-2){%
								\myimage[width=\schemaunit]{exp1-pattern5}}
							\rput(1.5,-2){%
								\myimage[height=\schemaunit]{exp1-pattern6}}

							\psset{framesize=1 1}
							\fnode(-1.5,2){P1}
							\uput[90](-1.5,2.5){\myresidue{1}}
							\fnode(1.5,2){P38}
							\uput[90](1.5,2.5){\myresidue{3} et \myresidue{8}}
							\fnode(1.5,-0){P27}
							\uput[90](1.5,0.5){\myresidue{2} et \myresidue{7}}
							\fnode(-1.5,-0){P49}
							\uput[90](-1.5,0.5){\myresidue{4} et \myresidue{9}}
							\fnode(-1.5,-2){P510}
							\uput[90](-1.5,-1.5){\myresidue{5} et \myresidue{10}}
							\fnode(1.5,-2){P6}
							\uput[90](1.5,-1.5){\myresidue{6}}

							\psset{linecolor=myred}
							\cnode(0.3,1.5){0.2}{TRPP1}
							\cnode(0.15,2){0.2}{TRPP38}
							\cnode(-0.1,1.25){0.2}{TRPP27}
							\cnode(-0.5,2.2){0.2}{TRPP49}
							\cnode(-0.65,1.25){0.2}{TRPP510}
							\ncline{-}{P1}{TRPP1}
							\ncline{-}{P38}{TRPP38}
							\ncline{-}{P27}{TRPP27}
							\ncline{-}{P49}{TRPP49}
							\ncline{-}{P510}{TRPP510}

							\psset{linecolor=myblue}
							\cnode(0.4,0.2){0.2}{PrionP38}
							\cnode(0.6,-2.8){0.2}{PrionP27}
							\cnode(0.2,-0.8){0.2}{PrionP49}
							\cnode(-0.7,-1.7){0.2}{PrionP510}
							\cnode(0.0,-1.4){0.2}{PrionP6}
							\ncline{-}{P38}{PrionP38}
							\ncline{-}{P27}{PrionP27}
							\ncline{-}{P49}{PrionP49}
							\ncline{-}{P510}{PrionP510}
							\ncline{-}{P6}{PrionP6}
						\end{myps}
					\end{myfigure}

					\begin{mytable}
						\newcommand{\myatomincolor}[3]{\csname my#1\endcsname{}{}#2 en \myemph{#3}}
						\mycaption[tab-exp1-ParametresDeComplexiteDesResidus]{Paramètres de complexité des \myglosnl*{glo-Residu} -- \myatomincolor{carbon}{arbone}{cyan}, \myatomincolor{nytrogen}{zote}{bleu}, \myatomincolor{oxygen}{xygène}{rouge} et \myatomincolor{sulfur}{oufre}{jaune}}
						\begin{mytabular}{^C-C-C-C-C}
							\mytoprule
							\myrowstyle{\bfseries}
							\myGlosnl{glo-Residu} & Position & Forme & Couleurs & Similarité \\
							\mymiddlerule[\heavyrulewidth]
							\myresidue{1}  & Interne & Cycle  & \mynum{8}~\mycarbon, \mynum{1}~\mynytrogen & Non \\
							\mymiddlerule
							\myresidue{2}  & Interne & Étoile & \mynum{1}~\mycarbon, \mynum{3}~\mynytrogen & Non \\
							\mymiddlerule
							\myresidue{3}  & Interne & Cycle  & \mynum{6}~\mycarbon, \mynum{1}~\myoxygen   & Non \\
							\mymiddlerule
							\myresidue{4}  & Externe & Chaîne & \mynum{4}~\mycarbon                        & Non \\
							\mymiddlerule
							\myresidue{5}  & Externe & Chaîne & \mynum{4}~\mycarbon, \mynum{1}~\mynytrogen & Non \\
							\mymiddlerule[\heavyrulewidth]
							\myresidue{6}  & Interne & Chaîne & \mynum{2}~\mycarbon, \mynum{2}~\mysulfur   & Non \\
							\mymiddlerule
							\myresidue{7}  & Externe & Étoile & \mynum{1}~\mycarbon, \mynum{3}~\mynytrogen & Non \\
							\mymiddlerule
							\myresidue{8}  & Externe & Cycle  & \mynum{6}~\mycarbon, \mynum{1}~\myoxygen   & Non \\
							\mymiddlerule
							\myresidue{9}  & Interne & Chaîne & \mynum{4}~\mycarbon                        & Oui \\
							\mymiddlerule
							\myresidue{10} & Interne & Chaîne & \mynum{4}~\mycarbon, \mynum{1}~\mynytrogen & Oui \\
							\mybottomrule
						\end{mytabular}
					\end{mytable}
				\end{mysubsection}
				\begin{mysubsection}[sse-exp1-Procedure]{Procédure}
					Pour débuter cette expérimentation, les sujets sont confrontés à un exemple sur la molécule \myTRPZIPPER \mycite{Christen-2009} ayant pour identifiant \myPDB \myPDBlink{http://www.rcsb.org/pdb/explore/explore.do?structureId=2KFL}{2KFL}.
					Pendant la phase d'apprentissage, les outils sont introduits et expliqués un par un.
					Chaque sujet a la possibilité de tester les outils et peut questionner l'expérimentateur.

					Dès que la phase d'apprentissage est terminée, l'enregistrement audio démarre.
					Un premier \myglos{glo-Residu} est affiché et les sujets peuvent débuter la recherche.
					Lorsque le \myglos{glo-Residu} a été trouvé, sélectionné puis extrait, le système s'arrête.
					Un second \myglos{glo-Residu} est affiché et ainsi de suite pour les dix \myglos*{glo-Residu}.

					L'ensemble des \myglos*{glo-Residu} est proposé dans un ordre différent pour chaque \myglos{glo-Monome} ou \myglos{glo-Binome}.
					Les sujets sont tenus d'effectuer l'ensemble des dux \myglos*{glo-Residu} deux fois, en \myglos{glo-Monome} et en \myglos{glo-Binome}.
					Les configurations \myglos{glo-Monome} et \myglos{glo-Binome} sont également alternées suivant les groupes selon les trois combinaisons possibles suivantes :
					\begin{enumerate}
						\item Le sujet \myuser{A}, puis le sujet \myuser{B}, puis le \myglos{glo-Binome} \myuser{AB};
						\item Le sujet \myuser{A}, puis le \myglos{glo-Binome} \myuser{AB}, puis le sujet \myuser{B};
						\item Le \myglos{glo-Binome} \myuser{AB}, puis le sujet \myuser{A}, puis le sujet \myuser{B}.
					\end{enumerate}

					Lorsque les sujets ont réalisé toutes les tâches dans les deux configurations possibles (\myglos{glo-Monome} et \myglos{glo-Binome}, ils sont soumis au questionnaire.
					Chaque sujet est tenu de répondre au questionnaire seul, sans communiquer avec les autres sujets.

					Un résumé du protocole expérimental est exprimé dans la \myref{tab-exp1-SyntheseDeLaProcedureExperimentale}.

					\begin{mytable}
						\mycaption[tab-exp1-SyntheseDeLaProcedureExperimentale]{Synthèse de la procédure expérimentale}
						\newcommand{\mytitlecolumn}[2]{%
							\multirow{#1}*{%
								\begin{minipage}{6em}%
									\raggedleft #2%
								\end{minipage}%
							}
						}
						\newlength{\exponefirstcolumn}
						\newlength{\exponesecondcolumn}
						\setlength{\exponefirstcolumn}{7em}
						\setlength{\exponesecondcolumn}{\textwidth}
						\addtolength{\exponesecondcolumn}{-\exponefirstcolumn}
						\addtolength{\exponesecondcolumn}{-4\tabcolsep}
						\begin{mytabular}{>{\bfseries}p{\exponefirstcolumn}p{\exponesecondcolumn}}
							\mytoprule
							\mytitlecolumn{1}{Tâche}                  & Recherche et sélection de motifs                                             \\
							\mymiddlerule[\heavyrulewidth]
							\mytitlecolumn{3}{Hypothèses}             & \myhypothesis{1} Amélioration des performances en \myglosnl{glo-Binome}      \\
							                                          & \myhypothesis{2} Stratégies variables en fonction des \myglosnl*{glo-Binome} \\
							                                          & \myhypothesis{3} Les sujets préfèrent le travail en \myglosnl{glo-Binome}    \\
							\mymiddlerule
							\mytitlecolumn{2}{Variable indépendantes} & \myvari{1} Nombre de sujets                                                  \\
							                                          & \myvari{2} \myGlosnl{glo-Residu} à chercher                                  \\
							\mymiddlerule
							\mytitlecolumn{4}{Variable dépendantes}   & \myvard{1} Temps de complétion                                               \\
							                                          & \myvard{2} Distance entre les espaces de travail                             \\
							                                          & \myvard{3} Communication verbales                                            \\
							                                          & \myvard{4} Affinités entre les sujets                                        \\
							\mymiddlerule[\heavyrulewidth]
							\multicolumn{2}{c}{%
								\begin{tabular}{^C-C-C}
									\myrowstyle{\bfseries}
									Condition \mycondition{1}         & Condition \mycondition{2}         & Condition \mycondition{3}         \\
									\mymiddlerule
									Sujet~\myuser{A}                  & Sujet~\myuser{A}                  & \myGlosnl{glo-Binome}~\myuser{AB} \\
									\mynum{10}~\myglosnl*{glo-Residu} & \mynum{10}~\myglosnl*{glo-Residu} & \mynum{10}~\myglosnl*{glo-Residu} \\
									\mymiddlerule
									Sujet~\myuser{B}                  & \myGlosnl{glo-Binome}~\myuser{AB} & Sujet~\myuser{A}                  \\
									\mynum{10}~\myglosnl*{glo-Residu} & \mynum{10}~\myglosnl*{glo-Residu} & \mynum{10}~\myglosnl*{glo-Residu} \\
									\mymiddlerule
									\myGlosnl{glo-Binome}~\myuser{AB} & Sujet~\myuser{B}                  & Sujet~\myuser{B}                  \\
									\mynum{10}~\myglosnl*{glo-Residu} & \mynum{10}~\myglosnl*{glo-Residu} & \mynum{10}~\myglosnl*{glo-Residu} \\
								\end{tabular}
							} \\
							\mybottomrule
						\end{mytabular}
					\end{mytable}
				\end{mysubsection}
			\end{mysection}
			\begin{mysection}[sec-exp1-Resultats]{Résultats}
				Cette section va présenter tous les résultats de cette première étude concernant la recherche et la sélection sur une tâche complexe de collaboration.
				Les données, confrontées à un test de \mycite[author]{Shapiro-1965}, ne sont pas distribuées selon une loi normale.
				Cependant, un test de \mycite[author]{Brown-1974} permet de confirmer l'\myglos{glo-Homoscedasticite}.
				L'analyse de la variance est alors pratiquée à l'aide d'un test de \mycite[author]{Friedman-1940}, adapté pour les \myglos*{glo-VariableIntraPopulation} non-paramètriques.
				\begin{mysubsection}[sse-exp1-AmeliorationDesPerformancesEnBinome]{Amélioration des performances en \myglosnl{glo-Binome}}
					\begin{myfigure}
						\psset{xunit=0.0889\textwidth,yunit=0.008cm}
						\begin{myps}(-1.25,-125)(10,425)
							\myaxes(0,10){\myglosnl*{glo-Residu}}(0,400)[100]{time~(s)}
							\myboxplot{exp1-time-residue.csv}
						\end{myps}
						\mycaption[fig-exp1-TempsDeCompletionParResidu]{Temps de complétion par \myglosnl{glo-Residu}}
					\end{myfigure}

					La \myref{fig-exp1-TempsDeCompletionParResidu} présente le temps de complétion \myvard{1} de chaque \myglos{glo-Residu} \myvari{2}.
					L'analyse montre qu'il y a un effet significatif des \myglos*{glo-Residu} \myvari{2} sur le temps de complétion \myvard{1} (\myanova{exp1-time-residue-anova.tex}).
					Un test post-hoc de \mycite[author]{Mann-1947} avec une correction de \mycite[author]{Holm-1979} permet de déterminer que les \myglos*{glo-Residu} \myresidue{6}, \myresidue{9} et \myresidue{10} obtiennent des temps de complétion significativement plus longs que les autres \myglos*{glo-Residu}.

					\begin{myfigure}
						\psset{xunit=0.0889\textwidth,yunit=0.008cm}
						\begin{myps}(-1.25,-125)(10,475)
							\myaxes(0,10){\myglosnl*{glo-Residu}}(0,400)[100]{time~(s)}
							\myboxplot{exp1-time-residue-group.csv}
							\mylegend{\myleg{\myGlosnl{glo-Monome}}{myblue}\myand\myleg{\myGlosnl{glo-Binome}}{myblue!70}}
						\end{myps}
						\mycaption[fig-exp1-TempsDeCompletionComparesMonomeOuBinomeParResidu]{Temps de complétion comparés (\myglosnl{glo-Monome} ou \myglosnl{glo-Binome}) par \myglosnl{glo-Residu}}
					\end{myfigure}

					La \myref{fig-exp1-TempsDeCompletionComparesMonomeOuBinomeParResidu} présente les temps de complétion \myvard{1} de chaque \myglos{glo-Residu} \myvari{2} en fonction du nombre de participants \myvari{1}.
					L'analyse ne montre pas d'effet significatif du nombre de participants \myvari{1} sur le temps de complétion \myvard{1} (\myanova{exp1-time-residue-group-anova.tex}).
					Cependant, en se limitant au groupe de trois \myglos*{glo-Residu} \myresidue{6}, \myresidue{9} et \myresidue{10} identifiés précédemment comme significativement plus longs à trouver et extraire, on montre un effet significatif du nombre de participants \myvari{1} sur le temps de complétion \myvard{1} (\myanova{exp1-time-residue-group-anova-restricted.tex}).

					\begin{myfigure}
						\psset{xunit=0.0889\textwidth,yunit=0.0069cm}
						\begin{myps}(-1.25,-150)(10,475)
							\myaxes(0,10){\myglosnl*{glo-Residu}}(0,400)[100]{time~(s)}
							\myboxplot{exp1-timeaudio-residue-searchselection.csv}
							\mylegend{\myleg{Recherche}{myblue}\myand\myleg{Sélection}{myblue!70}}
						\end{myps}
						\mycaption[fig-exp1-TempsDeRechercheEtDeSelectionComparesParResidu]{Temps de recherche et de sélection comparés par \myglosnl{glo-Residu}}
					\end{myfigure}

					La \myref{fig-exp1-TempsDeRechercheEtDeSelectionComparesParResidu} présente les temps de recherche et de sélection par \myglos{glo-Residu} \myvari{2}.
					L'analyse montre un effet significatif des \myglos*{glo-Residu} \myvari{2} sur les temps de recherche (\myanova{exp1-timeaudio-residue-searchselection-anova-search.tex}).
					Un test post-hoc de \mycite[author]{Mann-1947} avec une correction de \mycite[author]{Holm-1979} permet de déterminer que les \myglos*{glo-Residu} \myresidue{9} et \myresidue{10} obtiennent des temps de recherche significativement plus longs que les autres \myglos*{glo-Residu}.
					L'analyse montre également un effet significatif des \myglos*{glo-Residu} \myvari{2} sur les temps de sélection (\myanova{exp1-timeaudio-residue-searchselection-anova-selection.tex}).
					Un test post-hoc de \mycite[author]{Mann-1947} avec une correction de \mycite[author]{Holm-1979} permet de déterminer que le \myglos{glo-Residu} \myresidue{6} obtient un temps de sélection significativement plus long que les autres \myglos*{glo-Residu}.

					La molécule \myTRPCAGE présente un nombre de \myglos*{glo-Residu} à examiner relativement limité.
					Les sujets construisent rapidement une carte mentale de la molécule afin de trouver rapidement les \myglos*{glo-Residu} recherchés.
					De plus, les faibles contraintes physiques de la molécule la rende facile à déformer.
					Cela facilite la recherche des \myglos*{glo-Residu} qui sont en position interne à la molécule et qui nécessite une déformation.
					Tous ces facteurs rendent les tâches de recherche et de sélection peu complexes sur la molécule \myTRPCAGE.

					La molécule \myPrion possède un nombre de \myglos*{glo-Residu} beaucoup plus important.
					La construction complète d'une carte mentale est très complexe, d'autant plus que les sujets ne sont confrontés à la molécule que dix fois (cinq fois en \myglos{glo-Monome} et cinq fois en \myglos{glo-Binome}).
					Les sujets adoptent différentes stratégies suivant les \myglos*{glo-Residu}.
					Tout d'abord, ils débutent par une recherche exploratoire qui permet de découvrir tous les \myglos*{glo-Residu}  en position externe à la molécule (\myresidue{7} et \myresidue{8}).
					Ensuite, lorsque cette première phase d'exploration ne permet pas d'achever la tâche de recherche, les sujets déforment la molécule afin d'accéder aux \myglos*{glo-Residu} en position interne (\myresidue{6}, \myresidue{9} et \myresidue{10}).

					Le travail en \myglos{glo-Binome} n'améliore pas les performances sur la tâche par rapport au travail en \myglos{glo-Monome} bien que la \mypvalue soit seulement très légèrement au-dessus du seuil.
					Cependant, une évaluation de l'ensemble des \myglos*{glo-Residu} proposés a permis d'identifier des tâches de complexités variables : les \myglos*{glo-Residu} \myresidue{6}, \myresidue{9} et \myresidue{10} apparaissent comme significativement plus complexes que les autres \myglos*{glo-Residu}.
					Sur ce groupe de \myglos*{glo-Residu}, les \myglos*{glo-Binome} obtiennent une amélioration significative des performances par rapport aux \myglos*{glo-Monome} ce qui confirme notre hypothèse \myhypothesis{1} sur les tâches complexes.

					Comme développé dans la procédure expérimentale, le temps de complétion de la tâche peut être séparé en deux parties : le temps de recherche et le temps de sélection \myref*{fig-exp1-SchemaDesPhasesDeLaCommunicationVerbale}.
					Les \myglos*{glo-Residu} \myresidue{9} et \myresidue{10} se distinguent par un temps de recherche significativement plus important que les autres \myglos*{glo-Residu}.
					En effet, ce sont les deux \myglos*{glo-Residu} qui possèdent des similarités avec d'autres \myglos*{glo-Residu} également présents dans la molécule \myref*{tab-exp1-ParametresDeComplexiteDesResidus}.
					À cause de ces similarités, l'œil des sujets est continuellement attiré vers ces \myglos*{glo-Residu} similaires et une perte de temps significative se fait ressentir dans la phase de recherche.

					En contre-partie, le \myglos{glo-Residu} \myresidue{6} se distingue par un temps de sélection significativement plus important que les autres \myglos*{glo-Residu}.
					Ce \myglos{glo-Residu} possède deux atomes de \myatom{Soufre} de couleur jaune ce qui les rends très distinguables malgré le nombre importants d'atomes de la molécules.
					Le temps de recherche s'en trouve extrêmement réduit.
					Cependant, ce \myglos{glo-Residu} est positionné au centre de la molécule.
					La sélection nécessite de \myemph{déplier} en grande partie la molécule afin de pouvoir l'atteindre et le sélectionner.

					L'analyse met en évidence trois configurations concernant l'allocation du temps pour achever la tâche :
					\begin{description}
						\item[Recherche et sélection]
							Cette configuration se constitue d'un temps identique alloué à la recherche et à la sélection.
							Ces \myglos*{glo-Residu} sont ceux qui ne présentent pas de réelle complexité (la molécule \myTRPCAGE et les \myglos*{glo-Residu} \myresidue{7} et \myresidue{8} de la molécule \myPrion) et sur lesquels, le travail collaboratif n'améliore pas les performances.
						\item[Prédominance de la recherche]
							Cette configuration alloue un temps relativement important à la recherche de l'objectif.
							Une fois le \myglos{glo-Residu} identifié, les sujets peuvent accéder et sélectionner rapidement l'objectif.
							Les \myglos*{glo-Residu} \myresidue{9} et \myresidue{10} sont concernés.
							Le travail collaboratif améliore les performances grâce au caractère très parallélisable de la phase de recherche.
						\item[Prédominance de la sélection]
							Cette configuration alloue un temps relativement important à la sélection de l'objectif.
							Le \myglos{glo-Residu} est rapidement localisé mais il est difficile d'y accéder directement.
							Une phase de déformation est nécessaire pour le sélectionner.
							Le \myglos{glo-Residu} \myresidue{6} est concerné.
							Le travail collaboratif améliore les performances grâce à la multiplication des ressources allouées pour la déformation.
					\end{description}
				\end{mysubsection}
				\begin{mysubsection}[sse-exp1-StrategiesDeTravail]{Stratégies de travail}
					Dans cette section, les données concernent exclusivement les \myglos*{glo-Binome}.
					Elles sont utilisées afin d'étudier les différentes stratégies adoptées.

					\begin{myfigure}
						\psset{xunit=0.074\textwidth,yunit=0.15cm}
						\begin{myps}(-1.5,-6)(12,21)
							\myaxes(0,12){groupes}(0,20)[4]{distance~(mm)}
							% Once header are readed, they are defined for other barplot
							% That's why barplots without headers are in first position
							\mybarplot[header=false,barstyle=third-barstyle]{exp1-diff-groups3.csv}
							\mybarplot[header=false,barstyle=second-barstyle]{exp1-diff-groups2.csv}
							\mybarplot[barstyle=first-barstyle]{exp1-diff-groups1.csv}
							\psset{linecolor=myred,linewidth=1pt,linestyle=solid}
							\psline(0,14)(12,14)
							\psline(0,8)(12,8)
							\psset{linewidth=0.1pt,linecolor=white,fillstyle=solid,fillcolor=myred}
							\uput[180](12,5){\pscharpath{\LARGE\bf\sffamily Champ proche}}
							\uput[180](12,11){\pscharpath{\LARGE\bf\sffamily Champ voisin}}
							\uput[180](12,17){\pscharpath{\LARGE\bf\sffamily Champ distant}}
						\end{myps}
						\mycaption[fig-exp1-DistanceMoyenneEntreLesSujetsPourChaqueBinomeSurLesResidusSixNeufEtDix]{Distance moyenne entre les sujets pour chaque \myglosnl{glo-Binome} sur les \myglosnl*{glo-Residu} \myresidue{6}, \myresidue{9} et \myresidue{10}}
					\end{myfigure}

					La \myref{fig-exp1-DistanceMoyenneEntreLesSujetsPourChaqueBinomeSurLesResidusSixNeufEtDix} présente la distance moyenne entre les espaces de travail \myvard{2} de chaque \myglos{glo-Binome}.
					Les \myglos*{glo-Binome} peuvent être classés en trois groupes : \myemph{espace distant}, \myemph{espace voisin} et \myemph{espace proche}.

					\begin{myfigure}
						\psset{xunit=0.074\textwidth,yunit=0.5cm}
						\begin{myps}(-1.5,-1.75)(12,5.25)
							\myaxes(0,12){groupes}(0,5)[1]{affinité~(\mynum{1}--\mynum{5})}
							\mybarplot{exp1-affinity-groups.csv}
						\end{myps}
						\mycaption[fig-exp1-AffiniteEntreLesSujetsPourChaqueBinome]{Affinité entre les sujets pour chaque \myglosnl{glo-Binome}}
					\end{myfigure}

					La \myref{fig-exp1-AffiniteEntreLesSujetsPourChaqueBinome} présente les affinités \myvard{4} de chaque \myglos{glo-Binome}.
					Les notes, comprises entre un et cinq, montre que les \myglos*{glo-Binome} choisis ont des affinités relativement variées.

					\begin{myfigure}
						\psset{xunit=0.074\textwidth,yunit=0.0075cm}
						\begin{myps}(-1.5,-125)(12,325)
							\myaxes(0,12){groupes}(0,300)[50]{time~(s)}
							\mybarplot{exp1-time-groups.csv}
						\end{myps}
						\mycaption[fig-exp1-TempsDeCompletionEntreLesSujetsPourChaqueBinome]{Temps de complétion entre les sujets pour chaque \myglosnl{glo-Binome}}
					\end{myfigure}

					La \myref{fig-exp1-TempsDeCompletionEntreLesSujetsPourChaqueBinome} présente les temps de complétion \myvard{1} de chaque \myglos{glo-Binome}.
					Le temps de complétion de \mygroup{1} est particulièrement important (plus d'une fois et demi les autres groupes les plus longs).
					À l'opposé, on note que \mygroup{2}, \mygroup{3} et \mygroup{4} obtiennent des temps extrêmement bas.

					\begin{myfigure}
						\psset{xunit=0.074\textwidth,yunit=0.04cm}
						\begin{myps}(-1.5,-22)(12,75)
							\myaxes(0,12){groupes}(0,70)[10]{time~(s)}
							\mybarplot{exp1-timeaudio-groups.csv}
						\end{myps}
						\mycaption[fig-exp1-TempsDeCommunicationVerbaleEntreLesSujetsPourChaqueBinome]{Temps de communication verbale entre les sujets pour chaque \myglosnl{glo-Binome}}
					\end{myfigure}

					La \myref{fig-exp1-TempsDeCommunicationVerbaleEntreLesSujetsPourChaqueBinome} présente les temps de communication verbale \myvard{3} de chaque \myglos{glo-Binome}.
					\mygroup{2}, \mygroup{3} et \mygroup{4} ont des temps de communication verbale inférieurs à \mynum[s]{20}.
					À l'opposé, \mygroup{1}, \mygroup{5} et \mygroup{11} ont des temps de communication verbale qui approche les \mynum[s]{60}.

					\begin{myfigure}
						\psset{xunit=0.074\textwidth,yunit=0.03cm}
						\begin{myps}(-1.5,-30)(12,120)
							\myaxes(0,12){groupes}(0,100)[25]{time~(\%)}
							\mybarplot{exp1-timeaudio-groups-searchselection.csv}
							\mylegend{\myleg{Recherche}{myblue}\myand\myleg{Sélection}{myblue!70}}
						\end{myps}
						\mycaption[fig-exp1-PourcentageDeTempsDeCommunicationVerbalePendantLaRechercheEtLaSelectionDesSujetsPourChaqueBinome]{Pourcentage de temps de communication verbale pendant la recherche et la sélection des sujets pour chaque \myglosnl{glo-Binome}}
					\end{myfigure}

					La \myref{fig-exp1-PourcentageDeTempsDeCommunicationVerbalePendantLaRechercheEtLaSelectionDesSujetsPourChaqueBinome} présente les pourcentages de temps de communication verbale durant la phase de recherche et durant la phase de sélection de chaque \myglos{glo-Binome}.
					Le pourcentage représente le rapport du temps de communication verbale durant la phase recherche ou de sélection rapporté respectivement au temps total de la phase de recherche ou de sélection.
					Les \myglos*{glo-Binome} \mygroup{1} à \mygroup{4} ainsi que \mygroup{9} communiquent plus durant la phase de sélection.
					Les \myglos*{glo-Binome} \mygroup{5} à \mygroup{8} et \mygroup{10} à \mygroup{12} communiquent plus durant la phase de recherche.
					Notons également que \mygroup{1} communique assez peu par rapport aux autres \myglos*{glo-Binome}.

					\begin{myfigure}
						\psset{xunit=0.074\textwidth,yunit=1cm}
						\begin{myps}(-1.5,-1)(12,3.5)
							\myaxes(0,12){groupes}(0,3)[1]{force~(N)}
							\mybarplot{exp1-force-groups-meandiff.csv}
							\mylegend{\myleg{Force moyenne}{myblue}\myand\myleg{Différence de force}{myblue!70}}
						\end{myps}
						\mycaption[fig-exp1-ForceMoyenneEtDifferenceDeForceEntreLesSujetsPourChaqueBinome]{Force moyenne et différence de force entre les sujets pour chaque \myglosnl{glo-Binome}}
					\end{myfigure}

					La \myref{fig-exp1-ForceMoyenneEtDifferenceDeForceEntreLesSujetsPourChaqueBinome} représente la force moyenne et la différence de force entre les sujets.
					La force moyenne est la moyenne des forces moyennes des deux sujets.
					La différence de force est la différence des forces moyennes des deux sujets.
					\mygroup{9} et \mygroup{11} apporte un effort moyen très important par rapport aux autres \myglos*{glo-Binome}.
					\mygroup{2}, \mygroup{3} et \mygroup{4} apporte un effort moyen important également tout en ayant une différence de force quasiment nulle entre les deux membres du \myglos{glo-Binome}.

					Les distances moyennes entre les espaces de travail, données dans le référentiel du monde réel, sont comprises entre \mynum[mm]{3} et \mynum[mm]{18}.
					Ces distances mènent à trois différents types de stratégies :
					\begin{description}
						\item[Champ proche] pour les distances inférieures à \mynum[mm]{8};
						\item[Champ voisin] pour les distances comprises entre \mynum[mm]{8} et \mynum[mm]{14};
						\item[Champ distant] pour les distances supérieures à \mynum[mm]{14}.
					\end{description}

					Ces différentes stratégies, qui confirme notre hypothèse \myhypothesis{2}, sont caractérisées de façon précise dans les sections suivantes.

					\begin{mysubsubsection}[sss-exp1-ChampProche]{Champ proche}
						Les interactions en champs proches, inférieure à \mynum[mm]{8}, correspondent, dans l'environnement virtuel, à des distances inférieures à \mynum[\AA]{10} ce qui est environ l'envergure d'un \myglos{glo-Residu}\footnote{\AA\ désigne l'\AA ngström qui est une unité de mesure telle que $\mynum[\AA]{1} = \mynum[m]{e-10}$}.
						\mynum{8}~\myglos*{glo-Binome} sur \mynum{12} sont concernés par cette catégorie (\myglos*{glo-Binome} \mygroup{5}, \mygroup{6}, \mygroup{7}, \mygroup{8}, \mygroup{9}, \mygroup{10}, \mygroup{11} et \mygroup{12}).
						Ces \myglos*{glo-Binome} travaillent en collaboration étroite.

						Sur la \myref{fig-exp1-AffiniteEntreLesSujetsPourChaqueBinome}, tous les \myglos*{glo-Binome} manipulant en collaboration étroite ont de fortes affinités ($\mu = 4$) : ce sont des collègues proches ou des amis.
						D'après la \myref{fig-exp1-TempsDeCompletionEntreLesSujetsPourChaqueBinome}, ces \myglos*{glo-Binome} obtiennent des temps de complétion de la tâche relativement moyens comparés aux autres stratégies de travail.
						Cela se traduit également par une communication relativement moyenne comme affiché sur la \myref{fig-exp1-TempsDeCommunicationVerbaleEntreLesSujetsPourChaqueBinome}.

						En observant plus précisément les temps de communication verbale sur la \myref{fig-exp1-PourcentageDeTempsDeCommunicationVerbalePendantLaRechercheEtLaSelectionDesSujetsPourChaqueBinome}, les \myglos*{glo-Binome} de ce groupe passent plus de temps à communiquer dans la phase de recherche que dans la phase de sélection (excepté pour \mygroup{9}).
						Ces résultats tendent à prouver les difficultés du travail en champ proche liées aux nombreux conflits durant la phase de recherche.

						La \myref{fig-exp1-ForceMoyenneEtDifferenceDeForceEntreLesSujetsPourChaqueBinome} montre des disparités entre les \myglos*{glo-Binome} concernant la force moyenne.
						Des observations durant l'expérimentation ont permis de déterminer deux stratégies différentes :
						\begin{description}
							\item[Contrôle redondant] où les deux sujets effectuent la même action pour obtenir un meilleur contrôle sur les structures manipulées;
							\item[Guidage gestuel] où un des deux sujets indique à son partenaire la déformation à effectuer ou la position à atteindre.
						\end{description}
						Ces deux stratégies impliquent une communication étroite entre les sujets afin de coordonner au mieux les actions.

						Cependant, les différences importantes de forces appliquées \myref*{fig-exp1-ForceMoyenneEtDifferenceDeForceEntreLesSujetsPourChaqueBinome} ou l'analyse des communications verbales montre des conflits importants au sein des \myglos*{glo-Binome}.
						En effet, la grande complexité des tâches considérées couplé à une mauvaise conscience de l'environnement et de son partenaire mène à une mauvaise coordination et à des inter-référencements inexacts.
						Les communications verbales révèlent de nombreuses incompréhension dans l'inter-référencement (\og Pas dans cette direction \fg, \og Pas ici mais ici \fg, \og C'est juste derrière \fg, \myetc).
						Ces conflits et incompréhensions limitent les performances du \myglos{glo-Binome}.
					\end{mysubsubsection}
					\begin{mysubsubsection}[sss-exp1-ChampVoisin]{Champ voisin}
						Les interactions en champ voisin, comprises entre \mynum[mm]{8} et \mynum[mm]{14}, correspondent, dans l'environnement virtuel, à des distances de l'ordre de \myglos*{glo-Residu} voisins (entre \mynum[\AA]{10} et \mynum[\AA]{20}).
						\mynum{3}~\myglos*{glo-Binome} sur \mynum{12} sont concernés par cette catégorie (\myglos*{glo-Binome} \mygroup{2}, \mygroup{3} et \mygroup{4}).
						Ces \myglos*{glo-Binome} travaillent en collaboration relativement étroite sur des \myglos*{glo-Residu} voisins.
						Les \myglos*{glo-Residu} voisins sont dépendants physiquement ou structurellement comme indiqué sur la \myref{fig-exp1-CouplagePhysiqueEtStructurelEntreLesResidus}.

						\begin{myfigure}
							\setlength{\mywidth}{10pc}
							\setlength{\myheight}{10pc}
							\psset{xunit=\mywidth,yunit=\myheight}
							\begin{myps}(-1,-0.2)(1.15,1)
								\psset{ref=c}
								\rput(0.5,0.5){\myimage[width=\mywidth,angle=90]{exp1-trp-zipper}}
								\rput(0.35,-0.1){\rnode{connected-residues-label}{\begin{tabular}{c}Connected residues\\[-1ex]\textcolor{black!70}{\scriptsize Structural dependencies}\end{tabular}}}
								\rput(0.2,0.3){\pnode{connected-residues1}}
								\rput(0.5,0.3){\pnode{connected-residues2}}
								\rput(-0.5,0.7){\rnode{close-macro-label}{\begin{tabular}{c}Close macrostructures\\[-1ex]\textcolor{black!70}{\scriptsize Physical dependencies}\end{tabular}}}
								\rput(0,0.9){\pnode{close-macro1}}
								\rput(-0.1,0.5){\pnode{close-macro2}}
								\nccurve[angleA=135,angleB=-90]{->}{connected-residues-label}{connected-residues1}
								\nccurve[angleA=45,angleB=-90]{->}{connected-residues-label}{connected-residues2}
								\nccurve[angleA=90,angleB=180]{->}{close-macro-label}{close-macro1}
								\nccurve[angleA=-90,angleB=180]{->}{close-macro-label}{close-macro2}
							\end{myps}
							\mycaption[fig-exp1-CouplagePhysiqueEtStructurelEntreLesResidus]{Couplage physique et structure entre les \myglosnl*{glo-Residu}}
						\end{myfigure}

						Sur la \myref{fig-exp1-AffiniteEntreLesSujetsPourChaqueBinome}, tous les \myglos*{glo-Binome} manipulant en collaboration moyennement couplées ont des affinités moyennes ($\mu = 3$) : ce sont des collègues.
						D'après la \myref{fig-exp1-TempsDeCompletionEntreLesSujetsPourChaqueBinome}, ces \myglos*{glo-Binome} obtiennent de très bonnes performances sur les temps de complétion de la tâche.
						De plus, la \myref{fig-exp1-TempsDeCommunicationVerbaleEntreLesSujetsPourChaqueBinome} montre une communication verbale assez limitée.
						La manipulation en champ voisin permet d'être continuellement conscient des actions du partenaire ce qui évite les communications verbales.
						Cependant, les sujets manipulent des \myglos*{glo-Residu} différents ce qui limite les conflits d'interactions qui interviennent en champ proche.

						La \myref{fig-exp1-PourcentageDeTempsDeCommunicationVerbalePendantLaRechercheEtLaSelectionDesSujetsPourChaqueBinome} montre que la stratégie complémentaire de travail en champ voisin est plus performante dans la phase de recherche.

						La \myref{fig-exp1-ForceMoyenneEtDifferenceDeForceEntreLesSujetsPourChaqueBinome} illustre une bonne répartition des efforts entre les deux membres du \myglos{glo-Binome}.
						En effet, la force moyenne est assez élevée par rapport à la plupart des autres \myglos*{glo-Binome} ce qui montre qu'aucun des deux sujets n'est moins actif (ce qui entraînerait une force moyenne moins élevée).
						La différence des forces moyennes quasi-nulle entre les deux sujets confirme ce résultat.
						Ceci peut s'expliquer par une bonne coordination pendant laquelle les deux membres du \myglos{glo-Binome} vont effectuer des actions complémentaires mais de même intensité.
						La stratégie adoptée peut être définie de la façon suivante :
						\begin{description}
							\item[Manipulation complémentaire] où les deux sujets sont attentifs aux actions de leur partenaire afin d'avoir un meilleur contrôle du processus de déformation par une synchronisation améliorée.
						\end{description}
						L'analyse des communication verbales concerne beaucoup les phases de synchronisation (\og Maintenant, prends ça \fg, \og peux-tu m'aider ici ? \fg, \og Bien ! \fg, \myetc).
						Les performances des \myglos*{glo-Binome} travaillant en champ voisin sont relativement élevées bien que quelques conflits similaires à ceux rencontrés en champs proches soient présents bien que plus limités en nombre.
					\end{mysubsubsection}
					\begin{mysubsubsection}[sss-exp1-ChampDistant]{Champ distant}
						Les interactions en champ voisin, supérieures à \mynum[mm]{14}, correspondent, dans l'environnement virtuel, à des \myglos*{glo-Residu} sans interaction physique (supérieur à \mynum[\AA]{20}).
						\mynum{1}~\myglos{glo-Binome} sur \mynum{12} est concerné par cette catégorie (\myglos{glo-Binome} \mygroup{1}).
						Ce \myglos{glo-Binome} travaille de façon faiblement couplée.
						En effet, les membres de ce \myglos{glo-Binome} travaillent de façon complétement indépendante, en limitant au maximum le nombre d'interactions.

						Les affinités des membres de ce \myglos{glo-Binome} sont très faibles \myref*{fig-exp1-AffiniteEntreLesSujetsPourChaqueBinome} : les membres ne se connaissent presque pas.
						De plus, le \myglos{glo-Binome} obtient de très mauvaises performances en ce qui concerne le temps de complétion de la tâche comme le montre la \myref{fig-exp1-TempsDeCompletionEntreLesSujetsPourChaqueBinome}.
						La \myref{fig-exp1-TempsDeCommunicationVerbaleEntreLesSujetsPourChaqueBinome} montre que le temps de communication verbale est assez important.
						Cependant, le temps de complétion étant nettement plus important, le taux de communication verbale est beaucoup plus faible que les autres groupes \myref*{fig-exp1-PourcentageDeTempsDeCommunicationVerbalePendantLaRechercheEtLaSelectionDesSujetsPourChaqueBinome}.
						En effet, les membres du \myglos{glo-Binome} travaillant à distance, leurs interactions sont peu nombreuses ce qui implique peu de gestion de conflits et donc peu de communication verbale.
						La \myref{fig-exp1-PourcentageDeTempsDeCommunicationVerbalePendantLaRechercheEtLaSelectionDesSujetsPourChaqueBinome} montre également que ce \myglos{glo-Binome} communique plus dans les phases de sélection que dans les phases de recherche.
						En effet, les phases de sélection ont tendance à force la collaboration étroite et favorise ainsi les conflits.

						La \myref{fig-exp1-ForceMoyenneEtDifferenceDeForceEntreLesSujetsPourChaqueBinome} illustre un effort moyen moins élevé que les \myglos*{glo-Binome} manipulant en champ voisin.
						De plus, on constate une forte disparité entre les deux membres du \myglos{glo-Binome}.
						Ces observations montre encore une fois la faible collaboration.

						Chaque sujet manipulant en champ distant défini chacun sa propre stratégie et son propre espace de travail.
						Les interactions sont limitées au maximum.
						Cette configuration réduit considérablement les conflits d'interaction ainsi que la communication.
						Cependant, elle nuit beaucoup aux performances du groupe dans son ensemble.
					\end{mysubsubsection}
				\end{mysubsection}
				\begin{mysubsection}[sse-exp1-ResultatsQualitatifs]{Résultats qualitatifs}
					Les résultats qualitatifs sont constitués de deux parties.
					La première permet de déterminer les impressions des sujets concernant la collaboration, les rôles et efficacité de chacun durant la tâche.
					La seconde partie a pour but d'évaluer la plate-forme.
					Toutes les notes sont comprises entre un et cinq (échelle de \mycite[author]{Likert-1932} à cinq niveaux).
					\begin{mysubsubsection}[sss-exp1-EvaluationDeLaCollaboration]{Évaluation de la collaboration}
						Les résultats du questionnaire montre qu'une majorité des sujets de cette expérimentation ont préféré et apprécié la réalisation de la tâche en configuration collaborative (\mysummary{exp1-evaluation-group.tex}).
						De plus, le sentiment d'effectuer une tâche en collaboration est fort.
						L'hypothèse \myhypothesis{3} est confirmée par les sujets qui préférent la configuration collaborative à la configuration seule.
						Ceci valide à la fois la pertinence de la plate-forme mais également la pertinence des travaux de recherche sur le travail collaboratif.

						Durant ce travail collaboratif, chaque sujet considère qu'il a effectivement contribué à la réalisation de la tâche (\mysummary{exp1-evaluation-help.tex}).
						Cependant, les sujets ne considèrent qu'ils se soient imposés en meneur ou en suiveur (\mysummary{exp1-evaluation-leader.tex}).
						En effet, des questions supplémentaires ont permis de mettre en évidence que chaque sujet a tendance à surestimer le rôle du partenaire ($\approx$~\mynum[\%]{70}).

						L'estime du partenaire est important dans le travail collaboratif.
						En effet, la collaboration mène à une confiance mutuelle concernant la qualité du travail effectué par le partenaire, nécessaire à toute collaboration.
						Tout sujet n'ayant aucune estime pour le partenaire passerait son temps à vérifier le travail déjà effectué par le partenaire et entraînerait une perte significative des performances générales du groupe.
						De plus, chaque sujet se considère utile pour la réalisation de la tâche.
						Le cas contraire aurait pour effet qu'un des participants s'isole de lui-même, soustrayant ainsi son potentiel de l'action collaborative.


						Concernant la communication, les participants estiment que le principal canal de communication a été verbal (\mysummary{exp1-evaluation-verbal.tex}) et, dans une proportion plus faible mais tout de même importante, virtuel (\mysummary{exp1-evaluation-virtual.tex}).
						En ce qui concerne la communication gestuelle, ils la considèrent comme quasiment inexistante (\mysummary{exp1-evaluation-gestural.tex}).

						La communication gestuelle n'est pas ou peu utilisée pour plusieurs raisons.
						La principale raison est la difficulté de communiquer par geste lorsque les mains sont occupées à la manipulation.
						Deuxièmement, les sujets ont rapidement adopté le canal virtuel qui est plus précis dans les tâches de désignation qui constituent la plupart des besoins de communication.
						La communication verbale reste le canal principal de communication : c'est le canal le plus naturel pour communiquer.
						Cependant, il vient aussi en soutien du canal virtuel.
						En effet, aucun outil visuel ou haptique n'a été fourni pour effectuer des tâches de désignation et le canal virtuel seul serait incapable de remplir seul cette mission.
					\end{mysubsubsection}
					\begin{mysubsubsection}[sss-exp1-EvaluationDuSysteme]{Évaluation du système}
						L'évaluation du système en terme d'intuitivité comme en terme de confort est relativement satisfaisante.
						En effet, en ce qui concerne l'intuitivité des graphismes et effets visuels, les participants les trouve accessibles (\mysummary{exp1-platform-visual-intuitive.tex}).
						Il est en va de même en ce qui concerne l'intuitivité des interactions avec le système (\mysummary{exp1-platform-interaction-intuitive.tex}).
						Pour le confort, le visuel (\mysummary{exp1-platform-visual-confortable.tex}) et les interactions (\mysummary{exp1-platform-interaction-confortable.tex}) jouissent d'une évaluation similaire.

						Là encore, les sujets valident l'hypothèse \myhypothesis{3}.
						La plate-forme est relativement bien évaluée.
						Il semble cependant nécessaire d'apporter encore des améliorations afin de répondre au mieux aux attentes des utilisateurs.

						Ces résultats sont cependant à nuancer.
						Les écart-types sont relativement élevés ce qui veut dire qu'il y a de fortes disparités dans ces notations entre les différents sujets : certains sujets se sont déclarés plutôt insatisfaits concernant le confort (visuel : \mynum{2}, interaction : \mynum{2}).
						De plus, les outils proposés durant cette expérimentation sont relativement simples et peu envahissants.
						Des outils plus complexes, plus informatifs seraient peut-être moins intuitifs au premier abord et pourrait mener à un inconfort.
					\end{mysubsubsection}
				\end{mysubsection}
			\end{mysection}
			\begin{mysection}[sec-exp1-Synthese]{Synthèse}
				\begin{mysubsection}[sse-exp1-ResumeDesResultats]{Résumé des résultats}
					Dans ce chapitre, nous avons observé et comparé les performances de \myglos*{glo-Monome} et de \myglos*{glo-Binome} pendant une tâche de recherche et de sélection sur une simulation moléculaire en temps-réel.
					L'objectif était de montrer l'intérêt du travail collaboratif dans l'amélioration des performances et d'identifier les différentes stratégies de travail.
					De plus, il fallait valider la pertinence du système mis en place.

					La collaboration a prouvé son intérêt, notamment sur les tâches les plus complexes.
					La complexité d'une tâche est relativement difficile à établir.
					Le nombre d'atomes (et donc le nombre de \myglos*{glo-Residu}) joue un rôle important dans cette complexité : l'environnement virtuel devient surchargé et difficile à appréhender.
					Un autre facteur de complexité est l'amplitude des contraintes physiques qui interviennent dans la molécule.

					En observant et en analysant les différentes stratégies de travail, il ressort que le travail en champ proche comme le travail en champ distant ne sont pas des stratégies très performantes.
					En effet, le nombre de conflits en champ proche augmente significativement; en champ distant, tout l'intérêt du travail collaboratif est perdu.
					Le travail en champ voisin constitue un juste milieu performant.

					Enfin, il paraît nécessaire d'avoir de bonnes relations avec ces partenaires afin d'apporter à la fois, une communication saine et un respect mutuel du travail effectué.
					Cependant, les résultats montrent de façon évidente que tout déséquilibre dans le groupe mène à des performances dégradées.
				\end{mysubsection}
				\begin{mysubsection}[sse-exp1-Perspectives]{Perspectives}
					Basés sur les résultats précédents, certaines perspectives assez évidentes s'imposent et ont guidé les expérimentations qui suivent.
					Tout d'abord, il semble nécessaire de proposer des tâches suffisamment complexes pour pouvoir étudier plus en détail le travail collaboratif.
					Ceci se traduit soit par des tâches à fortes zones de contraintes \myref*{cha-LaManipulationCollaborative} ou par la manipulation de molécules de taille importante \myref*{cha-LesDynamiquesDeGroupe}.

					Les différentes stratégies observées ont permis de mettre en évidence l'intérêt du travail en champ voisin.
					Les propositions d'outils visuo-haptiques devront tenir compte de ce paramètre : ils devront encouragert le travail rapproché en fournissant une assistance en champ voisin tout en maintenant une distance minimum afin de limiter les conflits liés au travail en champ proche.

					L'évaluation qualitative par questionnaire apporte également de nombreuses réponses intéressantes.
					Tout d'abord, les sujets ont mis en avant un élément primordial de la communication : le canal virtuel est important.
					À l'aide d'observations durant les phases expérimentales, ce canal de communication est principalement exploité pour des actions de désignation.
					Fournir des outils spécifiquement conçus pour la désignation devient une nécessité.

					Enfin, ces évaluations qualitatives ont permis de valider l'\myacro{acr-EVC} proposé.
					Des améliorations sont cependant nécessaires en ce qui concerne le rendu visuel et les interactions.
					De nombreux sujets ont par exemple demandé une mise en surbrillance du \myglos{glo-Residu} survolé.
					Une assistance haptique pour la sélection est également une des améliorations possibles.
					Ces améliorations ne sont pas implémentées dans les deux expérimentations suivantes pour ne pas alourdir les outils et ainsi ne pas biaiser l'étude.
					Cependant, ils sont implémentés pour la dernière expérimentation \myref*{cha-TravailCollaboratifAssisteParHaptique}.
				\end{mysubsection}
			\end{mysection}
		\end{mychapter}
		\begin{mychapter}[cha-LaManipulationCollaborative]{La manipulation collaborative}
			\begin{mysection}[sec-exp2-Presentation]{Présentation}
				\begin{mysubsection}[sse-exp2-Objectifs]{Objectifs}
					Après avoir traité la sous-tâche élémentaire de \myemph{recherche}, la seconde expérimentation traitera des sous-tâches élémentaires de \myemph{sélection} et de \myemph{déformation} \myref*{sse-LAmarrageMoleculaire}.
					Ces sous-tâches introduisent des actions qui nécessitent une grande synchronisation et permet de stimuler les collaborations étroites.
					La précédente expérimentation \myref*{cha-LaRechercheCollaborative} a souligné l'avantage de la collaboration sur des tâches nécessitant un couplage fort.
					Les tâches proposées dans cette expérimentation sont élaborées pour stimuler les interactions entre les sujets.

					L'expérimentation va de nouveau comparer un \myglos{glo-Monome} et un \myglos{glo-Binome}.
					La manipulation \myglos*{glo-Bimanuel} est opposée à la manipulation collaborative afin de tester les performances de synchronisation.
					En effet, la synchronisation d'un seul sujet utilisant ces deux mains face à deux sujets utilisant chacun une seule de leur main.
					Trois facteurs seront étudier lors de cette comparaison.

					Le premier facteur concerne les performances.
					Les performances représente à la fois le temps total pour réaliser la tâche mais aussi les ressources mises en place pour accéder à ce résultat.
					Un \myglos{glo-Binome} en configuration collaborative sera-t-il plus performant qu'un \myglos{glo-Monome} en configuration \myglos*{glo-Bimanuel} ?

					Le second facteur observé sera la complexité de la tâche proposée.
					Le lien entre la complexité de la tâche et la configuration (collaborative ou \myglos*{glo-Bimanuel}) est étudié en fonction des performances.

					Le troisième facteur concerne l'apprentissage.
					En effet, quelque soit l'application et l'expérimentation proposée à des sujets, un phénomène d'apprentissage peut être observé.
					Cette expérimentation compare l'évolution de l'apprentissage entre les configurations collaboratives et \myglos*{glo-Bimanuel}.
				\end{mysubsection}
				\begin{mysubsection}[sse-exp2-Hypotheses]{Hypothèses}
					\begin{myparagraph}[par-exp2-AmeliorationDesPerformancesEnBinome]{\myhypothesis{1} Amélioration des performances en \myglosnl{glo-Binome}}
						La première hypothèse est une amélioration des performances pour les \myglos*{glo-Binome} en collaboratif comparés aux \myglos*{glo-Monome} en \myglos{glo-Bimanuel}.
						Cette amélioration se traduira principalement par une réalisation de la même tâche en un temps réduit.
						D'autres variables seront observées comme le nombre de sélections et la vitesse moyenne afin d'observer la répartition du travail entre les ressources disponibles.
					\end{myparagraph}
					\begin{myparagraph}[par-exp2-LesBinomesSontPlusPerformantsSurLesTachesComplexes]{\myhypothesis{2} Les \myglosnl*{glo-Binome} sont plus performants sur les tâches complexes}
						Cette second hypothèse concerne la corrélation entre la complexité de la tâche et la configuration (\myglos{glo-Binome} en collaboratif ou \myglos{glo-Monome} en \myglos{glo-Bimanuel}) sur les performances.
						L'hypothèse formule que les \myglos*{glo-Binome} seront plus performants que les \myglos*{glo-Monome} sur les tâches les plus complexes.
					\end{myparagraph}
					\begin{myparagraph}[par-exp2-LApprentissageEstPlusPerformantPourLesBinomes]{\myhypothesis{3} L'apprentissage est plus performant pour les \myglosnl*{glo-Binome}}
						L'hypothèse suppose que le travail collaboratif va stimuler l'apprentissage.
						L'échange des connaissances améliore l'apprentissage mais aussi grâce la multiplicité des avis et des idées pour répondre à un problème ou comprendre un événement.
					\end{myparagraph}
				\end{mysubsection}
			\end{mysection}
			\begin{mysection}[sec-exp2-DispositifExperimentalEtMateriel]{Dispositif expérimental et matériel}
				L'\myacro{acr-EVC} utilisé est illustré sur la \myref{fig-exp2-IllustrationDuDispositifExperimental}.
				Comme pour la première expérimentation \myref*{sec-exp1-DispositifExperimentalEtMateriel}, l'\myacro{acr-EVC} propose une visualisation partagée (vue publique) à l'aide d'un vidéoprojecteur sur un grand écran.
				Le ou les sujets font face à l'écran avec à leur disposition :
				\begin{itemize}
					\item un interface de manipulation de type souris \myThreeD;
					\item deux interfaces haptiques de déformation \mytool{tug}.
				\end{itemize}

				\begin{myfigure}
					\begin{mysubfigure}
						\myimage[width=0.49\textwidth]{exp2-schema}
						\mysubcaption[fig-exp2-IllustrationDuDispositifExperimental-SchemaDuDispositifExperimental]{Schéma du dispositif expérimental}
					\end{mysubfigure}
					\begin{mysubfigure}
						\myimage[width=0.49\textwidth]{exp2-photo}
						\mysubcaption[fig-exp2-IllustrationDuDispositifExperimental-PhotographieDuDispositifExperimental]{Photographie du dispositif expérimental}
					\end{mysubfigure}
					\mycaption[fig-exp2-IllustrationDuDispositifExperimental]{Illustration du dispositif expérimental}
				\end{myfigure}

				Les sujets ont la possibilité de communiquer entre eux sans restriction.
				Pour les \myglos*{glo-Monome}, le sujet peut utiliser chaque outil comme il le souhaite.
				Pour les \myglos*{glo-Binome}, chaque sujet se voit attribuer un outil de déformation \mytool{tug}.
				L'outil de manipulation (la souris \myThreeD) est laissé libre d'utilisation pour chacun des deux sujets.

				Pour les détails techniques concernant la plate-forme et les outils de manipulation et de déformation, se reporter au \myref{app-ShaddockSystemeCollaboratifPourLaManipulationDeMolecules}.
			\end{mysection}
			\begin{mysection}[sec-exp2-Methode]{Méthode}
				\begin{mysubsection}[sse-exp2-Sujets]{Sujets}
					\mynum{36}~sujets (\mynum{8}~femmes et \mynum{28}~hommes) avec une moyenne d'âge de $\mu = 25.9$ ($\sigma = 4.70$) ont participés à cette expérimentation.
					Ils ont tous été recrutés au sein du laboratoire \myacro{acr-LIMSI} et sont chercheurs ou assistants de recherche dans les domaines suivants~:
					\begin{itemize}
						\item linguistique et traitement automatique de la parole;
						\item réalité virtuelle et système immersifs;
						\item audio-acoustique.
					\end{itemize}
					Ils ont tous le français comme langue principale.
					Aucun participant n'a de déficience visuelle (ou corrigée le cas échéant) ni de déficience audio.

					Chaque participant est complètement naïf concernant les détails de l'expérimentation.
					Une explication détaillée de la procédure expérimentale leur est donnée au commencement de l'expérimentation mais en omettant l'objectif de l'étude.
				\end{mysubsection}
				\begin{mysubsection}[sec-exp2-Variables]{Variables}
					\begin{mysubsubsection}[sss-exp2-VariablesIndependantes]{Variables indépendantes}
						\begin{myparagraph}[par-exp2-NombreDeSujets]{\myvari{1} Nombre de sujets}
							La première \myglos{glo-VariableIndependante} est une \myglos{glo-VariableInterPopulation}, c'est-à-dire que les sujets sont expérimentés dans une seule modalité de cette variable.
							\myvari{1} possède deux valeurs possibles: \og un sujet (\mycf \myemph{\myglos{glo-Monome}}) \fg ou \og deux sujets (\mycf \myemph{\myglos{glo-Binome}}) \fg.
							Le sujets seuls et les sujets en couples ont à leur disposition deux outils haptiques de déformation et un outil haptique de manipulation.
							Pour les \myglos*{glo-Binome}, seulement un des deux sujets est désigné pour l'utilisation exclusive de la souris~\myThreeD.
							\mynum{12}~\myglos*{glo-Monome} et \mynum{12}~\myglos*{glo-Binome} sont testés.
						\end{myparagraph}
						\begin{myparagraph}[par-exp2-ComplexiteDeLaTache]{\myvari{2} Complexité de la tâche}
							La seconde \myglos{glo-VariableIndependante} est une \myglos{glo-VariableIntraPopulation}.
							Deux molécules de complexités différentes sont testées.
							Deux tâches de déformation sur chacune de molécules sont proposées : une déformation au niveau inter-moléculaire et une déformation au niveau intra-moléculaire.
						\end{myparagraph}
						\begin{myparagraph}[par-exp2-LeNiveauDApprentissage]{\myvari{3} Le niveau d'apprentissage}
							La troisième \myglos{glo-VariableIndependante} est une \myglos{glo-VariableIntraPopulation}.
							Tous les sujets seront confrontés trois fois à la même tâche afin de voir l'effet de l'apprentissage en \myglos{glo-Monome} et en \myglos{glo-Binome}.
						\end{myparagraph}
					\end{mysubsubsection}
					\begin{mysubsubsection}[sec-exp2-VariablesDependantes]{Variables dépendantes}
						\begin{myparagraph}[par-exp2-LeTempsDeCompletion]{\myvard{1} Le temps de complétion}
							Ce temps est le temps total pour réaliser la tâche demandée, c'est-à-dire manipuler et déformer la molécule afin d'atteindre l'objectif fixé.
							Il n'y a pas de limite de temps pour réaliser la tâche.
						\end{myparagraph}
						\begin{myparagraph}[par-exp2-LeNombreDeSelections]{\myvard{2} Le nombre de sélections}
							\myvard{2} représente le nombre de sélections réalisées durant chaque tâche à réaliser.
							Une sélection est comptabilisée lorsque un atome ou un \myglos{glo-Residu} est sélectionné par un des deux \myglos{glo-EffecteurTerminal}.
							Un compteur est affecté pour chacun des \myglos*{glo-EffecteurTerminal}.
						\end{myparagraph}
						\begin{myparagraph}[par-exp2-LaDistancePassiveEntreLesEspacesDeTravail]{\myvard{3} La distance passive entre les espaces de travail}
							Cette distance est la distance moyenne entre les deux \myglos*{glo-EffecteurTerminal} présents durant l'ensemble de l'expérimentation.
							Cette distance représente donc une distance physique du monde réel, pas une distance virtuelle.
							Elle est de l'ordre du centimètre.
						\end{myparagraph}
						\begin{myparagraph}[par-exp2-LaDistanceActiveEntreLesEspacesDeTravail]{\myvard{4} La distance active entre les espaces de travail}
							Cette distance est la distance moyenne entre les deux \myglos*{glo-EffecteurTerminal} présents seulement lorsque ces deux \myglos*{glo-EffecteurTerminal} sont en cours de manipulation (un atome ou un \myglos{glo-Residu} est sélectionné).
						\end{myparagraph}
						\begin{myparagraph}[par-exp2-ReponsesQualitatives]{\myvard{5} Réponses qualitatives}
							Un questionnaire est proposé à tous les sujets (variable en fonction des \myglos*{glo-Monome} et des \myglos*{glo-Binome}).
							Il est constitué de plusieurs questions (notées sur échelle de \mycite[author]{Likert-1932} à cinq niveaux).
							Un questionnaire différent est adressé aux \myglos*{glo-Monome} et aux \myglos*{glo-Binome}.

							Pour les \myglos*{glo-Monome}, le questionnaire est le suivant :
							\begin{enumerate}
								\item Vous êtes-vous senti efficace ?
								\item Pensez-vous que vous auriez été plus à l'aise seul avec un seul outil de déformation ?
								\item Pensez-vous que vous auriez été plus à l'aise avec un partenaire ?
								\item Quelle solution choisiriez-vous entre les trois configurations ?
							\end{enumerate}

							Pour les \myglos*{glo-Binome}, le questionnaire est le suivant (les questions sont posées à chaque sujet du \myglos{glo-Binome}) :
							\begin{enumerate}
								\item Vous êtes-vous senti efficace ?
								\item Comment évalueriez-vous votre taux de communication\dots{}
									\begin{itemize}
										\item verbale ?
										\item gestuelle ?
										\item virtuelle ?
									\end{itemize}
								\item Vous sentez-vous utile dans le groupe (par opposition à pénalisant) ?
								\item Pensez-vous avoir une position de meneur dans le groupe ?
								\item Pensez-vous que vous auriez été plus à l'aise seul avec votre outil de déformation ?
								\item Pensez-vous que vous auriez été plus à l'aise seul avec deux outils de déformation ?
								\item Quelle solution choisiriez-vous entre les trois configurations ?
							\end{enumerate}

							Concernant la communication, les communications verbales concernent tous les échanges, dialogues exposés par la voix.
							La communication gestuelle représente les gestes que les sujets peuvent effectuer dans le monde réel pour expliquer, désigner ou pour tout autre explication à son partenaire.
							Enfin, la communication virtuelle concerne les informations données au partenaire par l'intermédiaire de l'environnement virtuel (par exemple, une désignation avec le curseur).
						\end{myparagraph}
					\end{mysubsubsection}
				\end{mysubsection}
				\begin{mysubsection}[sse-exp2-Tache]{Tâche}
					La tâche proposée est la déformation dans un \myacro{acr-EVC} sur des molécules complexes.
					Deux niveaux différents de manipulation sont proposés :
					\begin{itemize}
						\item inter-moléculaire (à l'échelle d'un \myglos{glo-Residu});
						\item intra-moléculaire (à l'échelle d'un atome).
					\end{itemize}

					\begin{mysubsubsection}[sss-exp2-DescriptionDeLaTache]{Description de la tâche}
						La tâche proposée est la déformation d'une molécule afin de la rendre conforme à un modèle.
						L'intégralité des atomes de la molécule à déformer est affiché.
						De plus, un \myemph{ruban} de cette molécule est affiché.
						En ce qui concerne la molécule cible (le modèle), seul un affichage de type \myemph{ruban} est utilisé.
						Cet affichage est appliqué en filigrane.

						Lorsqu'un sujet sélectionne un atome ou un \myglos{glo-Residu}, ce dernier est mis en surbrillance.
						De plus, l'atome ou le \myglos{glo-Residu} correspondant sur la molécule cible est affiché afin de connaître la position finale de la sélection courante.
						La \myref{fig-exp2-AffichageDeLaMoleculeADeformerEtDeLaMoleculeCible} illustre ces différents effets graphiques.

						\begin{myfigure}
							\psset{unit=0.08\textwidth}
							\def\myexptwolabel(#1,#2)[#3]#4#5{\rput(#1,#2){\rnode{#3}{\textcolor{#4}{\sffamily #5}}}}
							\begin{myps}(0,0)(12,9)
								\rput[bl](1,0){\myimage[width=0.8\textwidth]{exp2-trp-zipper}}
								\rput[bl](8,0.5){\myimage[width=3.5cm,angle=-20]{exp2-green-cursor}}
								\myexptwolabel(8.8,2.6)[deformed-label]{myred}{Molécule à déformer}
								\myexptwolabel(1.3,5.5)[ghost-label]{myred}{Molécule cible}
								\myexptwolabel(7.1,7.0)[deformed-residue-label]{myblue}{\myGlosnl{glo-Residu} à déformer}
								\myexptwolabel(1.2,2.75)[ghost-residue-label]{myblue}{\myGlosnl{glo-Residu} cible}
								\myexptwolabel(4.0,7.75)[fixed-residue-label]{mygray}{\myGlosnl{glo-Residu} fixe}
								\pnode(6.8,3.6){deformed}
								\pnode(1.8,4){ghost}
								\psset{linecolor=myblue}
								\cnode(6.2,4.9){1.0}{deformed-residue}
								\cnode(2.3,1.6){0.8}{ghost-residue}
								\psset{linecolor=mygray}
								\cnode(2.0,6.6){0.8}{fixed-residue}
								\psset{linewidth=1pt,linecolor=myred,linearc=.1,arrowsize=1pt 3,arrowinset=.2,nodesepA=3pt}
								\ncangle[angleA=90,angleB=0]{c->}{deformed-label}{deformed}
								\ncangle[angleA=-90,angleB=180,offsetA=-0.5]{c->}{ghost-label}{ghost}
								\psset{linecolor=myblue,nodesepB=0pt}
								\ncdiagg[angleA=-90,offsetA=0.5]{c->}{deformed-residue-label}{deformed-residue}
								\ncdiagg[angleA=-90,offsetA=-0.5]{c->}{ghost-residue-label}{ghost-residue}
								\ncdiagg[angleA=180,linecolor=mygray]{c->}{fixed-residue-label}{fixed-residue}
								\ncline[linewidth=10pt,linecolor=myblue,arrowsize=2pt 2,nodesepA=4pt]{C->}{deformed-residue}{ghost-residue}
								\psframe*[linecolor=red](0,8)(12,9)
								\psframe*[linecolor=green](0,8)(2,9)
								\rput(6,8.5){\textcolor{white}{\bfseries\sffamily\LARGE Score RMSD}}
								\psframe[linewidth=1pt,linecolor=black](0,0)(12,9)
							\end{myps}
							\mycaption[fig-exp2-AffichageDeLaMoleculeADeformerEtDeLaMoleculeCible]{Affichage de la molécule à déformer et de la molécule cible}
						\end{myfigure}

						Afin de pouvoir évaluer la déformation effectuée, des scores sont affichés en temps-réel \myref*{fig-exp2-AffichageDeLaMoleculeADeformerEtDeLaMoleculeCible}.
						Le premier score est le score \myacro{acr-RMSD} qui permet de mesurer la différence entre deux déformations d'une même molécule en calculant la différence entre chaque paire d'atomes \myref*{equ-RMSD}.
						\begin{equation}\label{equ-RMSD}
							\mathrm{RMSD}\left(\mathbf{c},\mathbf{m}\right) = \sqrt{\frac{1}{N}\sum_{i=1}^{N}\mynorm{c_i - m_i}^2}
						\end{equation}
						où $N$~est le nombre total d'atomes et~$c_i$, $m_i$~sont respectivement l'atome~$i$ de la molécule à comparer et de la molécule modèle.
					\end{mysubsubsection}
					\begin{mysubsubsection}[sss-exp2-LaDescriptionDesTaches]{La description des tâches}
						Une tâche pour chacun de ces deux échelles de manipulation est proposé sur chacune des deux molécules.
						La première molécule est couramment nommée \myTRPZIPPER \mycite{Cochran-2001} a pour identifiant \myPDB \myPDBlink{http://www.rcsb.org/pdb/explore/explore.do?structureId=1LE1}{1LE1}.
						La second molécule est couramment nommée \myTRPCAGE \mycite{Neidigh-2002} a pour identifiant \myPDB \myPDBlink{http://www.rcsb.org/pdb/explore/explore.do?structureId=1L2Y}{1L2Y}.
						On peut donc distinguer quatre tâches différentes :
						\begin{description}
							\item[\mytask{1a}]
								Cette tâche concerne la manipulation de la molécule \myTRPZIPPER à l'échelle inter-moléculaire.
								Un \myglos{glo-Residu} à l'extrémité -- la molécule formant une chaîne -- est fixé afin d'\myemph{ancrer} la molécule au sein de l'environnement virtuel et éviter d'éventuelles dérives hors du champ visuel.
								L'intégralité des onze autres \myglos*{glo-Residu} est libre de mouvement.
								La forme général de la molécule peut être comparée à un \myform{V} : la chaîne de \myglos*{glo-Residu} de la molécule contient une cassure.
							\item[\mytask{1b}]
								Cette tâche concerne la manipulation de la molécule \myTRPCAGE à l'échelle inter-moléculaire.
								Comme la tâche \mytask{1a}, elle contient un \myglos{glo-Residu} fixe à une extrémité.
								L'intégralité des dix neuf autres \myglos*{glo-Residu} est libre de mouvement.
								La forme général de la molécule peut être comparée à un \myform{W} : la chaîne de \myglos*{glo-Residu} de la molécule contient deux cassures.
							\item[\mytask{2a}]
								Cette tâche concerne la manipulation de la molécule \myTRPZIPPER à l'échelle intra-moléculaire.
								Seulement trois \myglos*{glo-Residu} sont laissés libres tandis que tous les autres sont fixés.
								Les contraintes physiques de cette tâche sont relativement faibles.
								Cependant, la difficulté de cette tâche réside dans la recherche des \myglos*{glo-Residu} à déformer qui ne sont pas aisés à trouver.
							\item[\mytask{2b}]
								Cette tâche concerne la manipulation de la molécule \myTRPCAGE à l'échelle intra-moléculaire.
								Seulement six \myglos*{glo-Residu} sont laissés libres tandis que les autres sont fixés.
								La déformation requise demande une grande dépense d'énergie.
								En effet, la molécule proposée se trouve dans une sorte de puit de potentiel (un \myemph{minima} local) et l'objectif est d'atteindre un autre puit de potentiel (un autre \myemph{minima} local).
								L'énergie nécessaire pour passer d'un puit à l'autre est relativement importante, à tel point qu'un seul outil de déformation n'est pas suffisant.
								La manipulation synchrone de deux \myglos*{glo-Residu} est la seule solution pour atteindre l'objectif.
						\end{description}

						Un résumé de la complexité des quatre tâches est exposé dans la \myref{tab-exp2-ParametresDeComplexiteDesTaches} selon les critères suivants :
						\begin{description}
							\item[Nombre d'atomes] C'est le nombre total d'atomes que contient la molécule à manipuler;
							\item[\myglosnl{glo-Residu} libre] C'est le nombre de \myglos*{glo-Residu} non fixés sur la molécule;
							\item[Cassure] Ce sont les cassures de la chaîne principale de la molécule; elles représentent les jonctions entre \myhelice* et/ou les \myfeuillet*;
							\item[Champ de force] Il représente la difficulté en terme de contrainte physique; il exprime l'énergie minimum nécessaire pour atteindre l'objectif et se traduit par trois niveaux (\myemph{faible}, \myemph{moyen} et \myemph{fort}).
						\end{description}

						\begin{mytable}
							\mycaption[tab-exp2-ParametresDeComplexiteDesTaches]{Paramètres de complexité des tâches}
							\begin{mytabular}{^>{\bfseries}p{9em}-C-C-C-C}
								\mytoprule
								\myrowstyle{\bfseries}
								& \mytask{1a} & \mytask{1b} & \mytask{2a} & \mytask{2b} \\
								\mymiddlerule[\heavyrulewidth]
								Nombre d'atomes           & \mynum{218} & \mynum{304} & \mynum{218} & \mynum{304} \\
								\mymiddlerule
								\myglosnl{glo-Residu} libre & \mynum{11}  & \mynum{19}  & \mynum{3}   & \mynum{7}   \\
								\mymiddlerule
								Cassure                   & \mynum{1}   & \mynum{2}   & \mynum{0}   & \mynum{1}   \\
								\mymiddlerule
								Champ de force            & Moyen       & Moyen       & Faible      & Fort        \\
								\mybottomrule
							\end{mytabular}
						\end{mytable}
					\end{mysubsubsection}
					\begin{mysubsubsection}[sss-exp2-LesOutilsDisponibles]{Les outils disponibles}
						Des outils de déformation légérement différents sont proposés en fonction de la tâche à réaliser.
						Pour les tâches de déformation au niveau inter-moléculaire, l'outil de déformation est l'outil \mytool{tug} : il permet de déformer d'un tenant l'intégralité d'un \myglos{glo-Residu}.
						Pour les tâches de déformation au niveau intra-moléculaire, l'outil de déformation est l'outil \mytool{tug} : il permet d'appliquer une force sur un unique atome.
						L'outil \mytool{tug} pour les \myglos*{glo-Residu} applique la même force à chaque atome du \myglos{glo-Residu}.
						Il en résulte que l'outil \mytool{tug} pour les \myglos*{glo-Residu} permet de développer plus d'énergie.
					\end{mysubsubsection}
				\end{mysubsection}
				\begin{mysubsection}[sse-exp2-Procedure]{Procédure}
					Pour débuter cette expérimentation, les sujets sont confrontés à un exemple sur la molécule \myPrion \mycite{Cochran-2001} ayant pour identifiant \myPDB \myPDBlink{http://www.rcsb.org/pdb/explore/explore.do?structureId=1LE1}{1LE1}.
					Pendant la phase d'apprentissage, les outils sont introduits et expliqués un par un.
					Chaque sujet a la possibilité de tester les outils et peut questionner l'expérimentateur.
					Dans le cas des \myglos*{glo-Binome}, cette phase d'apprentissage est également l'occasion de choisir qui, parmi les deux sujets, sera en charge de la manipulation de la molécule à l'aide de l'outil de manipulation \mytool{grab}.

					Un résumé du protocole expérimental est exprimé dans la \myref{tab-exp2-SyntheseDeLaProcedureExperimentale}.

					\begin{mytable}
						\mycaption[tab-exp2-SyntheseDeLaProcedureExperimentale]{Synthèse de la procédure expérimentale}
						\newcommand{\mytitlecolumn}[2]{%
							\multirow{#1}*{%
								\begin{minipage}{6em}%
									\raggedleft #2%
								\end{minipage}%
							}
						}
						\newlength{\exptwofirstcolumn}
						\newlength{\exptwosecondcolumn}
						\setlength{\exptwofirstcolumn}{7em}
						\setlength{\exptwosecondcolumn}{\textwidth}
						\addtolength{\exptwosecondcolumn}{-\exptwofirstcolumn}
						\addtolength{\exptwosecondcolumn}{-4\tabcolsep}
						\begin{mytabular}{>{\bfseries}p{\exptwofirstcolumn}p{\exptwosecondcolumn}}
							\mytoprule
							\mytitlecolumn{1}{Tâche}                  & Déformation d'une molécule                                                        \\
							\mymiddlerule[\heavyrulewidth]
							\mytitlecolumn{3}{Hypothèses}             & \myhypothesis{1} Amélioration des performances en \myglosnl{glo-Binome}           \\
							                                          & \myhypothesis{2} \myglosnl*{glo-Binome} plus performants sur les tâches complexes \\
							                                          & \myhypothesis{3} Apprentissage plus performant en \myglosnl{glo-Binome}           \\
							\mymiddlerule
							\mytitlecolumn{3}{Variable indépendantes} & \myvari{1} Nombre de sujets                                                       \\
							                                          & \myvari{2} Complexité de la tâche                                                 \\
							                                          & \myvari{3} Niveau d'apprentissage                                                 \\
							\mymiddlerule
							\mytitlecolumn{5}{Variable dépendantes}   & \myvard{1} Temps de complétion                                                    \\
							                                          & \myvard{2} Nombre de sélections                                                   \\
							                                          & \myvard{3} Distance passive entre les espaces de travail                          \\
							                                          & \myvard{4} Distance active entre les espaces de travail                           \\
							                                          & \myvard{5} Réponses qualitatives                                                  \\
							\mymiddlerule[\heavyrulewidth]
							\multicolumn{2}{c}{%
								\begin{tabular}{^C-C-C-C}
									\myrowstyle{\bfseries}
									Condition \mycondition{1}       & Condition \mycondition{2}       & Condition \mycondition{3} & Condition \mycondition{4} \\
									\mymiddlerule
									\myGlosnl{glo-Bimanuel} ($N=1$) & \myGlosnl{glo-Bimanuel} ($N=1$) & Collaboratif ($N=2$)      & Collaboratif ($N=2$)      \\
									\mymiddlerule
									\mytask{1a}                     & \mytask{1b}                     & \mytask{1a}               & \mytask{1b}               \\
									\mytask{1b}                     & \mytask{1a}                     & \mytask{1b}               & \mytask{1a}               \\
									\mytask{2a}                     & \mytask{2b}                     & \mytask{2a}               & \mytask{2b}               \\
									\mytask{2b}                     & \mytask{2a}                     & \mytask{2b}               & \mytask{2a}               \\
								\end{tabular}
							} \\
							\mybottomrule
						\end{mytabular}
					\end{mytable}
				\end{mysubsection}
			\end{mysection}
		\end{mychapter}
		\begin{mychapter}[cha-LesDynamiquesDeGroupe]{Les dynamiques de groupe}
		\end{mychapter}
	\end{mypart}
	\begin{mypart}[prt-PropositionsPourLeTravailCollaboratif]{Propositions pour le travail collaboratif}
		\begin{mychapter}[cha-TravailCollaboratifAssisteParHaptique]{Travail collaboratif assisté par haptique}
			\begin{mysection}[sec-exp4-Presentation]{Présentation}
				\begin{mysubsection}[sse-exp4-Objectifs]{Objectifs}
					Cette dernière expérimentation aura pour objectif d'introduire et de valider des outils de communication haptique dans le cadre d'une tâche d'\myglos{glo-AmarrageMoleculaire}.
					Sur la base des précédentes expérimentations, des outils haptiques censés améliorer les interactions et les communications entre les manipulateurs sont proposés.
					L'expérimentation testera l'intérêt et l'apport de ces outils sur la collaboration de groupe.

					Le principal facteur observé sera les performances du groupe.
					Les performances regardées seront le temps mis pour achever la tâche mais également la qualité de la solution trouvée.
					En effet, la qualité de la solution est une variable non-négligeable dans le cadre d'une tâche d'\myglos{glo-AmarrageMoleculaire}.

					Le second facteur concernera l'évaluation qualitative du système par les utilisateurs.
					Il est primordial de recueillir l'avis des utilisateurs en ce qui concerne une plate-forme de travail.
					Des outils haptiques inconfortables, des détails visuels incohérents, des interactions peu intuitives sont autant de paramètres qui peuvent rendre un système inefficace.
				\end{mysubsection}
				\begin{mysubsection}[sse-exp4-Hypotheses]{Hypothèses}
					\begin{myparagraph}[par-exp4-AmeliorationDesPerformancesAvecLAssistanceHaptique]{\myhypothesis{1} Amélioration des performances avec l'assistance haptique}
						La première hypothèse est une amélioration des performances liée à l'utilisation des assistances haptiques proposées à travers des outils.
						Le temps de complétion de la tâche et la qualité de la solution proposée par les sujets seront les \myglos*{glo-VariableDependante} principales pour observer cette amélioration des performances.
					\end{myparagraph}
				\end{mysubsection}
			\end{mysection}
			\begin{mysection}[sec-exp4-DispositifExperimentalEtMateriel]{Dispositif expérimental et matériel}
				L'\myacro{acr-EVC} utilisé est illustré sur la \myref{fig-exp4-IllustrationDuDispositifExperimental}.
				L'\myacro{acr-EVC} propose une visualisation partagée sur grand écran (vue publique à tous les utilisateurs) à l'aide d'un vidéoprojecteur.
				Les \mytodo{3 ou 4}{Quelle configuration choisissons nous ?} sujets font face à l'écran avec à leur disposition :
				\begin{itemize}
					\item une interface d'orientation de la scène;
					\item une interface haptique de manipulation de la molécule;
					\item une interface haptique pour la coordination;
					\item deux interfaces haptiques de déformation \mytool{tug}.
				\end{itemize}

				\begin{myTodo}{Images à compléter}{Il va falloir créer la scène Blender correspondante et faire des photos du dispositif expérimental}
					\begin{myfigure}
						\begin{mysubfigure}
							%\myimage[width=0.49\textwidth]{exp4-schema}
							\mysubcaption[fig-exp4-IllustrationDuDispositifExperimental-SchemaDuDispositifExperimental]{Schéma du dispositif expérimental}
						\end{mysubfigure}
						\begin{mysubfigure}
							%\myimage[width=0.49\textwidth]{exp4-photo}
							\mysubcaption[fig-exp4-IllustrationDuDispositifExperimental-PhotographieDuDispositifExperimental]{Photographie du dispositif expérimental}
						\end{mysubfigure}
						\mycaption[fig-exp4-IllustrationDuDispositifExperimental]{Illustration du dispositif expérimental}
					\end{myfigure}
				\end{myTodo}

				Une caméra vidéo de marque \mySony (\textsc{hdr-sr11e}) sera installée afin de filmer l'expérimentation.
				L'écran de vidéo-projection ainsi que les sujets (de dos) sont dans le plan de la vidéo.

				Pour les détails techniques concernant la plate-forme et les outils de manipulation et de déformation, se reporter au \myref{app-ShaddockSystemeCollaboratifPourLaManipulationDeMolecules}.

			\end{mysection}
			\begin{mysection}[sec-exp4-Methode]{Méthode}
				\begin{mysubsection}[sse-exp4-Sujets]{Sujets}
					\begin{myTodo}{Nombre de sujets}{Remplir toutes les informations statistiques concernant les sujets}
						\mynum{000}~sujets (\mynum{000}~femmes et \mynum{000}~hommes) avec une moyenne d'âge de $\mu = 00.0$ ($\sigma = 0.00$) ont participés à cette expérimentation.
					\end{myTodo}
					Ils ont été recrutés au sein du laboratoire \myacro{acr-IBPC} et sont chercheurs en biologie moléculaire.
					Ils ont tous le français comme langue principale.
					Aucun participant n'a de déficience visuelle (ou corrigée le cas échéant) ni de déficience audio.

					Chaque participant est complètement naïf concernant les détails de l'expérimentation.
					Une explication détaillée de la procédure expérimentale leur est donnée au commencement de l'expérimentation mais en omettant l'objectif de l'étude.
				\end{mysubsection}
				\begin{mysubsection}[sec-exp4-Variables]{Variables}
					\begin{mysubsubsection}[sss-exp4-VariablesIndependantes]{Variables indépendantes}
						\begin{myparagraph}[par-exp4-PresenceDeLAssistance]{\myvari{1} Présence de l'assistance}
							La première \myglos{glo-VariableIndependante} est une \myglos{glo-VariableIntraPopulation}, c'est-à-dire que tous les sujets sont expérimentés dans toutes les modalités de cette variable.
							\myvari{1} possède deux valeurs possibles : \og sans assistance \fg ou \og avec assistance \fg.
							L'assistance est l'aide haptique ajouté aux différents outils de manipulation, de désignation et de déformation afin d'améliorer l'intéraction et la communication entre les sujets pendant la tâche.
						\end{myparagraph}
						\begin{myparagraph}[par-exp4-ComplexeDeMoleculesAAssembler]{\myvari{2} Complexe de molécules à assembler}
							La seconde \myglos{glo-VariableIndependante} est une \myglos{glo-VariableIntraPopulation}.
							\myvari{2} concerne les complexes de molécules à assembler : \og \myNusE \fg.
						\end{myparagraph}
					\end{mysubsubsection}
					\begin{mysubsubsection}[sec-exp4-VariablesDependantes]{Variables dépendantes}
						\begin{myparagraph}[par-exp4-LeTempsDeCompletion]{\myvard{1} Le temps de complétion}
							Ce temps est le temps total pour réaliser la tâche demandée, c'est-à-dire manipuler et déformer la molécule afin d'atteindre l'objectif fixé.
							Il n'y a pas de limite de temps pour réaliser la tâche.
						\end{myparagraph}
						\begin{myparagraph}[par-exp4-LeNombreDeSelections]{\myvard{2} Le nombre de sélections}
							\myvard{2} représente le nombre de sélections réalisées durant chaque tâche à réaliser.
							Une sélection est comptabilisée lorsque un atome ou un \myglos{glo-Residu} est sélectionné par un des deux \myglos{glo-EffecteurTerminal}.
							Un compteur est affecté pour chacun des \myglos*{glo-EffecteurTerminal}.
						\end{myparagraph}
						\begin{myparagraph}[par-exp4-LesCommunicationsOrales]{\myvard{3} Les communications verbales et gestuelles}
							L'enregistrement vidéo permet de mesurer la quantité de temps de parole pendant chaque tâche de l'expérimentation.
							Elle permet également d'observer les phases de communication gestuelle.
							Les communications gestuelles sont les mouvements physiques des sujets destinés à donner une information à un ou plusieurs autres sujets.
						\end{myparagraph}
						\begin{myparagraph}[par-exp4-EvaluationQualitativeDuSysteme]{\myvard{4} Évaluation qualitative du système}
							Un questionnaire est proposé à tous les sujets.
							Il est constitué de plusieurs questions (notées sur échelle de \mycite[author]{Likert-1932} à cinq niveaux).

							\begin{myTodo}{Le questionnaire}{Écrire le questionnaire soumis au sujets}
								Le questionnaire est le suivant (les questions sont posées à chaque sujet dans le cas du \myglos{glo-Binome}) :
								\begin{enumerate}
									\item Quelle note donneriez-vous\dots{}
										\begin{enumerate}
											\item au système interactif que vous venez de tester ?
											\item aux effets visuels offerts par le système ?
											\item aux outils proposés ?
										\end{enumerate}
									\item Quelle configuration avez-vous préféré : \myemph{sans assistance} ou \myemph{avec assistance} ?
									\item Vous êtes vous senti utile dans le groupe ?
									\item Pensez-vous avoir une position de meneur dans la configuration collaborative ?
									\item Comment évalueriez-vous votre taux de communication\dots{}
										\begin{itemize}
											\item verbale ?
											\item gestuelle ?
											\item virtuelle ?
										\end{itemize}
								\end{enumerate}
							\end{myTodo}

							Concernant la communication, les communications verbales concernent tous les échanges, dialogues exposés par la voix.
							La communication gestuelle représente les gestes que les sujets peuvent effectuer dans le monde réel pour expliquer, désigner ou pour tout autre explication à son partenaire.
							Enfin, la communication virtuelle concerne les informations données au partenaire par l'intermédiaire de l'environnement virtuel (par exemple, une désignation avec le curseur).
						\end{myparagraph}
					\end{mysubsubsection}
				\end{mysubsection}
				\begin{mysubsection}[sse-exp4-Tache]{Tâche}
					La tâche proposée est la déformation dans un \myacro{acr-EVC} sur des complexes de molécules : c'est une tâche d'\myglos{glo-AmarrageMoleculaire} simplifié.

					\begin{mysubsubsection}[sss-exp4-DescriptionDeLaTache]{Description de la tâche}
						La tâche proposée est la déformation d'un complexe de molécules afin d'obtenir le meilleur score énergétique possible.
						L'intégralité des atomes de la molécule à déformer est affiché de façon discrète en transparence.
						De plus, un \myemph{ruban} de cette molécule est affiché.

						Afin de réaliser la tâche, différentes mesures sont disponibles en temps-réel pour les sujets.
						La première de ces mesures est le score \myacro{acr-RMSD} qui est décrit dans la \myref{sss-exp2-DescriptionDeLaTache}.
						La seconde mesure est l'énergie totale du système, valeur calculée par \myacro{acr-NAMD}.

						Le premier complexe de molécules proposé, couramment nommé \myNusE \mycite{Burmann-2010}, a pour identifiant \myPDB \myPDBlink{http://www.rcsb.org/pdb/explore/explore.do?structureId=2KVQ}{2KVQ}.
						Il est constitué de deux molécules \textsc{NusE} et \textsc{NusG} possédant respectivement \mynum{1294}~atomes et \mynum{929}~atomes.
					\end{mysubsubsection}
					\begin{mysubsubsection}[sss-exp4-LesOutilsDisponibles]{Les outils disponibles}
						\begin{myTodo}{Deux configurations possibles}{Les deux configurations font intervenir trois ou quatre sujets; il faut enlever un des 2 paragraphes}
							\begin{myparagraph}[par-exp4-ConfigurationATroisSujets]{Configuration à trois sujets}
								Les sujets effectueront l'expérimentation par groupe de trois utilisateurs.
								Ils ont la possibilité de communiquer entre eux sans restriction.
								Un des sujets aura le rôle du \myemph{coordinateur} avec des outils haptiques différents des deux autres sujets qui auront des rôles d'\myemph{opérateurs}.

								Le \myemph{coordinateur} pourra déplacer et orienter la molécule à l'aide de deux outils:
								\begin{itemize}
									\item un outil haptique attaché virtuellement à la molécule permettant de déplacer la molécule;
									\item un \mySpaceNavigator permettant d'orienter la molécule.
								\end{itemize}
								De plus, le \myemph{coordinateur} aura à sa disposition un outil de désignation complexe permettant de :
								\begin{itemize}
									\item afficher ou masquer les \myglos*{glo-Residu} en fonction de leur intérêt pour la tâche en cours de réalisation;
									\item désigner un \myglos{glo-Residu} nécessitant une manipulation par les \myemph{opérateurs}.
								\end{itemize}
							\end{myparagraph}
							\begin{myparagraph}[par-exp4-ConfigurationAQuatreSujets]{Configuration à quatre sujets}
								Les sujets effectueront l'expérimentation par groupe de quatre utilisateurs.
								Ils ont la possibilité de communiquer entre eux sans restriction.

								Un des sujets aura la gestion du déplacement de la molécule et du point de vue de l'application; il sera le \myemph{manipulateur}.
								Un deuxième sujet aura le rôle du \myemph{coordinateur}.
								Enfin, les deux derniers sujets seront les \myemph{opérateurs}.

								Le \myemph{manipulateur} aura à sa disposition deux outils.
								Le premier outil est une interface haptique, attachée virtuellement à la molécule permettant de déplacer la molécule.
								Le second outil est un \mySpaceNavigator permettant d'orienter la molécule.

								Le \myemph{coordinateur} aura à sa disposition un outil de désignation complexe permettant de :
								\begin{itemize}
									\item afficher ou masquer les \myglos*{glo-Residu} en fonction de leur intérêt pour la tâche en cours de réalisation;
									\item désigner un \myglos{glo-Residu} nécessitant une manipulation par les \myemph{opérateurs}.
								\end{itemize}
							\end{myparagraph}
						\end{myTodo}

						Les \myemph{opérateurs} auront chacun à leur disposition un outil haptique permettant de :
						\begin{itemize}
							\item déplacer les atomes de la molécule afin de la déformer;
							\item désigner un \myglos{glo-Residu} nécessitant une manipulation par l'autre \myemph{opérateur}.
						\end{itemize}
					\end{mysubsubsection}
				\end{mysubsection}
				\begin{mysubsection}[sse-exp4-Procedure]{Procédure}
					Pour débuter cette expérimentation, les sujets sont confrontés à un exemple sur la molécule \mytodo{à déterminer}{Il faudra trouver une molécule d'apprentissage}.
					Pendant la phase d'apprentissage, les outils sont introduits et expliqués un par un.
					Chaque sujet a la possibilité de tester les outils et peut questionner l'expérimentateur.

					Dès que la phase d'apprentissage est terminée, l'enregistrement vidéo démarre.
					Un premier complexe est proposé aux sujets.
					\mytodo{Dix minutes}{C'est une proposition de temps mais il faudra peut-être adapter en fonction des tests alpha} sont laissées pour réaliser la tâche.
					Si les sujets estiment avoir obtenu le meilleur score possible avant la durée limite, ils peuvent décider d'arrêter la tâche.

					Lorsque toutes les tâches sont réalisées, les sujets sont soumis au questionnaire.
					Chaque sujet est tenu de répondre au questionnaire seul, sans communiquer avec les autres sujets.

					Un résumé du protocole expérimental est exprimé dans la \myref{tab-exp4-SyntheseDeLaProcedureExperimentale}.

					\begin{mytable}
						\mycaption[tab-exp4-SyntheseDeLaProcedureExperimentale]{Synthèse de la procédure expérimentale}
						\newcommand{\mytitlecolumn}[2]{%
							\multirow{#1}*{%
								\begin{minipage}{6em}%
									\raggedleft #2%
								\end{minipage}%
							}
						}
						\newlength{\expfourfirstcolumn}
						\newlength{\expfoursecondcolumn}
						\setlength{\expfourfirstcolumn}{7em}
						\setlength{\expfoursecondcolumn}{\textwidth}
						\addtolength{\expfoursecondcolumn}{-\expfourfirstcolumn}
						\addtolength{\expfoursecondcolumn}{-4\tabcolsep}
						\begin{mytabular}{>{\bfseries}p{\expfourfirstcolumn}p{\expfoursecondcolumn}}
							\mytoprule
							\mytitlecolumn{1}{Tâche}                  & Amarrage d'un complexe de molécule                                        \\
							\mymiddlerule[\heavyrulewidth]
							\mytitlecolumn{1}{Hypothèses}             & \myhypothesis{1} Amélioration de performance par assistance haptique      \\
							\mymiddlerule
							\mytitlecolumn{2}{Variable indépendantes} & \myvari{1} Présence de l'assistance                                       \\
							                                          & \myvari{2} Complexe de molécules à assembler                              \\
							\mymiddlerule
							\mytitlecolumn{4}{Variable dépendantes}   & \myvard{1} Temps de complétion                                            \\
							                                          & \myvard{2} Nombre de sélections                                           \\
							                                          & \myvard{3} Communications verbales et gestuelles                          \\
							                                          & \myvard{4} Évaluation qualitative du système                              \\
							\mymiddlerule[\heavyrulewidth]
							\multicolumn{2}{c}{%
								\begin{tabular}{^C-C}
									\myrowstyle{\bfseries}
									Condition \mycondition{1} & Condition \mycondition{2} \\
									\mymiddlerule
									Sans assistance           & Avec assistance           \\
									\mymiddlerule
									\myNusE                   & \myNusE                   \\
								\end{tabular}
							} \\
							\mybottomrule
						\end{mytabular}
					\end{mytable}
				\end{mysubsection}
			\end{mysection}
		\end{mychapter}
	\end{mypart}
	\begin{mypart}[prt-Synthese]{Synthèse}
		\begin{mychapter}[cha-ConclusionEtPerspectives]{Conclusion et perspectives}
		\end{mychapter}
	\end{mypart}

	\myglossary
	%\myappendix
\end{document}
