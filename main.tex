\documentclass[myfrancais]{mythesis}
\usepackage{mydate}
\usepackage{mytodo}
\usepackage{graphicx}
\usepackage{caption}
\usepackage{pstricks}
\usepackage{pstricks-add}
\newcommand{\name}[1]{\textsc{#1}}
\newcommand{\mynum}[1]{\nombre{#1}}
\newcommand{\myhyp}[1]{$\left(\mathcal{H}_#1\right)$\xspace}
\newcommand{\mygroup}[1]{$\left(\mathcal{G}_#1\right)$\xspace}
\newcommand{\myvar}[2]{$\left(\mathcal{V}_{\mathrm{#1}#2}\right)$\xspace}
\newcommand{\mysubject}[1]{\textsc{p}\nombre{#1}}
\newcommand{\myvari}[1]{\myvar{i}{#1}}
\newcommand{\myvard}[1]{\myvar{d}{#1}}
\newcommand{\mypattern}[1]{$\left(\mathcal{M}_{#1}\right)$\xspace}
\newcommand{\myPDB}{\textsc{pdb}\xspace}
\newcommand{\myPDBbase}{\emph{Protein~DataBase}\xspace}
\newcommand{\myPDBlink}[2]{\href{#1}{\textsc{\MakeLowercase{#2}}}}
\newcommand{\TRPCAGE}{TRP-CAGE\xspace}
\newcommand{\TRPZIPPER}{TRP-ZIPPER\xspace}
\newcommand{\Prion}{Prion\xspace}
\newcommand{\Ubiquitin}{Ubiquitin\xspace}
\newcommand{\myemph}[1]{\emph{#1}}
\newcommand{\myThreeD}{\textsc{3d}\xspace}
\newcommand{\myregistered}{\textsuperscript{\textregistered}}
\newcommand{\mySpaceNavigator}{SpaceNavigator\myregistered\xspace}
\newcommand{\mycf}{\textit{c.f.}\xspace}
\newcommand{\myetc}{\textit{etc.}\xspace}
\newrgbcolor{myred}{0.7265625 0 0}
\newrgbcolor{myblue}{0 0 0.7265625}
\newlength{\mywidth}
\newlength{\myheight}

\hypersetup{%
	pdftitle={Interactions haptiques collaboratives pour la manipulation moléculaire},%
	pdfauthor={Jean SIMARD},%
	pdfsubject={Mémoire de thèse en informatique},%
	pdfdisplaydoctitle=true,%
	pdflang={FR-fr}%
}

\title{Interactions haptiques collaboratives pour la manipulation moléculaire}
\author{Jean~\name{Simard}}
\documenttype{Thèse en Informatique}
\university{École Doctorale d'Informatique de Paris Sud}
\date{\mydate[datestyle=long]{01/12/2011}}
\jury{%
	Martin & \name{DUPONT} & (rapporteur) & Directeur de recherche au \name{LIMSI} \\
	Martin & \name{DUPOND} & (examinateur) & Directeur de recherche au \name{LIMSI}}

\addglobalbib[datatype=bibtex]{biblio.bib}

\mynewglos{glo-VariableIndependante}{%
	name={variable indépendante},%
	description={BlahBlahBlah},%
	plural={variables indépendantes}%
}
\mynewacro{acr-LIMSI}{%
	name={\textsc{cnrs--limsi}},%
	description={BlahBlahBlah},%
	first={Laboratoire pour l'Informatique, la Mécanique et les Sciences de l'Ingénieur (\textsc{cnrs--limsi})}%
}
\mynewacro{acr-CD}{%
	name={\textsc{cd}},
	description={Un Compact Disc permet de stocker de la musique par exemple},
	first={Compact Disc (\textsc{cd})},
	plural={\textsc{cd}s},
	firstplural={Compacts Discs (\textsc{cd}s)}
}
\begin{document}
	\frontmatter
	\maketitle
	\mytoc
	\mylof
	\mylot
	\mylotodo
	\mainmatter
	\part{Le sujet}
	\chapter{Introduction}
	\part{Étude du travail collaboratif}
	\chapter{La recherche collaborative}
	\section{Présentation}
	\subsection{Objectifs}
	\subsection{Hypothèses}
	\section{Dispositif expérimental}
	\section{Méthode}
	\subsection{Sujets}
	\mynum{24}~sujets (\mynum{4}~femmes et \mynum{20}~hommes) avec une moyenne d'âge de $\mu = 27.8$ ($\sigma = 7.19$) ont participés à cette expérimentation.
	Ils ont tous été recrutés au sein du laboratoire \myacro{acr-LIMSI} et sont chercheurs ou assistants de recherche dans les domaines suivants~:
	\begin{itemize}
		\item linguistique et traitement automatique de la parole;
		\item réalité virtuelle et système immersifs;
		\item audio-acoustique.
	\end{itemize}
	Ils ont tous le français comme langue principale.
	Aucun participant n'a de déficience visuelle (ou corrigée le cas échéant) ni déficience audio.
	
	Chaque participants est complètement naïf concernant les détails de l'expérimentation.
	Une explication détaillée de la procédure expérimentale leur est donnée au commencement de l'expérimentation mais en omettant l'objectif de l'étude.

	\subsection{Variables}
	\subsubsection{Variables indépendantes}
	\paragraph{\myvari{1} Nombre de sujets}
	La première \myglos{glo-VariableIndependante} est une variable intra-population, c'est-à-dire que tous les sujets seront expérimentés dans toutes les conditions de cette variable.
	\myvari{1} possède \mynum{2}~valeurs possibles: \og \mynum{1}~sujet (\mycf \myemph{monôme}) \fg ou \og \mynum{2}~sujets (\mycf \myemph{binôme}) \fg.
	Le sujets seuls et les sujets en couples ont à leur disposition \mynum{2}~interfaces haptiques et une souris~\myThreeD (\mySpaceNavigator).
	Pour les binômes, seulement un des deux sujets est désigné pour l'utilisation exclusive de la souris~\myThreeD.
	\mynum{24}~monômes et \mynum{12}~binômes ont été testés ce qui fait deux fois plus de monômes que de binômes.

	\paragraph{\myvari{2} Motif recherché}
	La seconde \myglos{glo-VariableIndependante} est une variable intra-population.
	\myvari{2} concerne les motifs recherchés qui sont au nombre de \mynum{10} répartis à part égale dans \mynum{2}~molécules \mytableref*{tab-exp1-ListeDesMotifsRecherches}.
	La première molécule est couramment nommée \TRPCAGE \mycite{Neidigh-2002} et a pour identifiant \myPDB \myPDBlink{http://www.rcsb.org/pdb/explore/explore.do?structureId=1L2Y}{1L2Y} sur la \myPDBbase\footnote{\url{http://www.pdb.org/}}.
	La seconde molécule nommée \Prion \mycite{Christen-2009} avec l'identifiant \myPDB \myPDBlink{http://www.rcsb.org/pdb/explore/explore.do?structureId=2KFL}{2KFL}.
	\mynum{5}~motifs sont présents sur chaque molécule \myfigureref*{fig-exp1-RepartitionDesMotifsSurLesMolecules} et chacun présente différents niveaux de complexité \mytableref*{tab-exp1-ParametresDeComplexiteDesMotifs} :
	\begin{description}
		\item[Position] La position du motif peut se trouver sur le pourtour de la molécule, en position \myemph{externe} ou à l'intérieur, au milieu de l'amas d'atome (position \myemph{interne}).
			Un motif en position externe ne nécessite pas de déformer la molécule pour le trouver et l'atteindre contrairement à un motif en position interne qui sera plus complexe d'accès.
		\item[Forme] La forme du motif influe énormément sur la complexité de la recherche.
			On distingue \mynum{3}~formes différentes :
			\begin{description}
				\item[Chaîne] Un enchaînement d'atomes seuls les atomes d'hydrogène sont de part et d'autres de cet enchaînement.
				\item[Cercle] Une chaîne d'atomes de carbone ou d'azote qui boucle sur elle-même.
				\item[Étoile] Séries de chaînes d'atomes toutes reliées sur un atome central (la plupart du temps, un atome de carbone).
			\end{description}
		\item[Couleurs] Les atomes sont colorés en fonction de leur nature (rouge pour l'oxygène, blanc pour l'hydrogène, \myetc).
			Les atomes qui sont rares seront donc rapidement trouvés grâce à leur couleur différente.
			Par contre, les atomes nombreux (comme les hydrogènes ou les carbones) seront plus difficiles à filtrer à cause de leur nombre important.
		\item[Similarité] Certains résidus à chercher sont très similaires à d'autres résidus également présents sur la molécule.
			De par leur similarité, ils vont mobilier la recherche sur des résidus incorrects.
	\end{description}

	\begin{mytable}
		\mycaption[tab-exp1-ListeDesMotifsRecherches]{Liste des motifs recherchés}
		\setlength{\myheight}{10ex}
		\newcommand{\mypatternpicture}[1]{\myimage[width=\myheight]{exp1-#1}}
		\begin{mysubtable}
			\mysubcaption[tab-exp1-ListeDesMotifsRecherches-MotifsSurLaMoleculeTRPCAGE]{Motifs sur la molécule \TRPCAGE}
			\begin{mytabular}[0.49\textwidth]{^C-C}
				\mytoprule
				\myrowstyle{\bfseries}
				Motif & Image \\
				\mymiddlerule
				\mypattern{1} & \mypatternpicture{pattern1} \\
				\mypattern{2} & \mypatternpicture{pattern2} \\
				\mypattern{3} & \mypatternpicture{pattern3} \\
				\mypattern{4} & \mypatternpicture{pattern4} \\
				\mypattern{5} & \mypatternpicture{pattern5} \\
				\mybottomrule
			\end{mytabular}
		\end{mysubtable}
		\begin{mysubtable}
			\mysubcaption[tab-exp1-ListeDesMotifsRecherches-MotifsSurLaMoleculePrion]{Motifs sur la molécule \Prion}
			\begin{mytabular}[0.49\textwidth]{^C-C}
				\mytoprule
				\myrowstyle{\bfseries}
				Motif & Image \\
				\mymiddlerule
				\mypattern{6}  & \mypatternpicture{pattern6}  \\
				\mypattern{7}  & \mypatternpicture{pattern7}  \\
				\mypattern{8}  & \mypatternpicture{pattern8}  \\
				\mypattern{9}  & \mypatternpicture{pattern9}  \\
				\mypattern{10} & \mypatternpicture{pattern10} \\
				\mybottomrule
			\end{mytabular}
		\end{mysubtable}
	\end{mytable}
	
	\begin{myfigure}
		\newcommand{\schemafactor}{0.16}
		\newlength{\schemaunit}\setlength{\schemaunit}{\schemafactor\textwidth}
		\psset{unit=\schemaunit}
		\mycaption[fig-exp1-RepartitionDesMotifsSurLesMolecules]{Répartition des motifs sur les molécules}
		\begin{pspicture}(-3,-2.5)(3,2.5)
			\rput(-1.75,0){%
				\myimage[height=2\schemaunit]{exp1-trp-cage}}
			\rput(1.25,0){%
				\myimage[height=2\schemaunit]{exp1-prion}}
			\rput(-2,-1.5){%
				\myimage[height=\schemaunit]{exp1-pattern1}}
			\rput(-2,1.5){%
				\myimage[width=\schemaunit]{exp1-pattern3}}
			\rput(0,1.5){%
				\myimage[width=\schemaunit]{exp1-pattern2}}
			\rput(0,-1.5){%
				\myimage[width=\schemaunit]{exp1-pattern4}}
			\rput(2,-1.5){%
				\myimage[width=\schemaunit]{exp1-pattern5}}
			\rput(2,1.5){%
				\myimage[height=\schemaunit]{exp1-pattern6}}

			\psset{framesize=1 1}
			\fnode(-2,-1.5){P1}
			\uput[-90](-2,-2){\mypattern{1}}
			\fnode(-2,1.5){P38}
			\uput[90](-2,2){\mypattern{3} et \mypattern{8}}
			\fnode(0,1.5){P27}
			\uput[90](0,2){\mypattern{2} et \mypattern{7}}
			\fnode(0,-1.5){P49}
			\uput[-90](0,-2){\mypattern{4} et \mypattern{9}}
			\fnode(2,-1.5){P510}
			\uput[-90](2,-2){\mypattern{5} et \mypattern{10}}
			\fnode(2,1.5){P6}
			\uput[90](2,2){\mypattern{6}}

			\psset{linecolor=myred}
			\cnode(-1.5,0.3){0.2}{TRPP1}
			\cnode(-2,0.15){0.2}{TRPP38}
			\cnode(-1.25,-0.1){0.2}{TRPP27}
			\cnode(-2.2,-0.5){0.2}{TRPP49}
			\cnode(-1.25,-0.65){0.2}{TRPP510}
			\ncline{-}{P1}{TRPP1}
			\ncline{-}{P38}{TRPP38}
			\ncline{-}{P27}{TRPP27}
			\ncline{-}{P49}{TRPP49}
			\ncline{-}{P510}{TRPP510}

			\psset{linecolor=myblue}
			\cnode(-0.2,0.4){0.2}{PrionP38}
			\cnode(2.8,0.6){0.2}{PrionP27}
			\cnode(0.8,0.2){0.2}{PrionP49}
			\cnode(1.7,-0.7){0.2}{PrionP510}
			\cnode(1.4,0.0){0.2}{PrionP6}
			\ncline{-}{P38}{PrionP38}
			\ncline{-}{P27}{PrionP27}
			\ncline{-}{P49}{PrionP49}
			\ncline{-}{P510}{PrionP510}
			\ncline{-}{P6}{PrionP6}
		\end{pspicture}
	\end{myfigure}

	\begin{mytable}
		\mycaption[tab-exp1-ParametresDeComplexiteDesMotifs]{Paramètres de complexité des motifs}
		\begin{mytabular}{^C-L-L-L-L}
			\mytoprule
			\myrowstyle{\bfseries}
			Motif & Position & Forme & Couleurs & Similarité \\
			\mymiddlerule[\heavyrulewidth]
			\mypattern{1}  & Interne & Cercle & 8~C, 1~N & Non \\
			\mymiddlerule
			\mypattern{2}  & Interne & Étoile & 1~C, 3~N & Non \\
			\mymiddlerule
			\mypattern{3}  & Interne & Cercle & 6~C, 1~O & Non \\
			\mymiddlerule
			\mypattern{4}  & Externe & Chaîne & 4~C      & Non \\
			\mymiddlerule
			\mypattern{5}  & Externe & Chaîne & 4~C, 1~N & Non \\
			\mymiddlerule[\heavyrulewidth]
			\mypattern{6}  & Interne & Chaîne & 2~C, 2~S & Non \\
			\mymiddlerule
			\mypattern{7}  & Externe & Étoile & 1~C, 3~N & Non \\
			\mymiddlerule
			\mypattern{8}  & Externe & Cercle & 6~C, 1~O & Non \\
			\mymiddlerule
			\mypattern{9}  & Interne & Chaîne & 4~C      & Oui \\
			\mymiddlerule
			\mypattern{10} & Interne & Chaîne & 4~C, 1~N & Oui \\
			\mybottomrule
		\end{mytabular}
	\end{mytable}

	\subsubsection{Variables dépendantes}
	\subsection{Tâche}
	\subsection{Procédure}
	\section{Résultats}
	\chapter{La manipulation collaborative}
	\chapter{Les dynamiques de groupe}
	\part{Propositions pour le travail collaboratif}
	\chapter{Travail collaboratif assisté par haptique}
	\part{Synthèse}
	\chapter{Conclusion et perspectives}

	\myglossary
	\appendix
	\chapter{\textsc{Shaddock} -- Collaborative Virtual Environment for Molecular Design}
\end{document}
