\documentclass[myfrancais]{mythesis}
\usepackage{mydate}
\usepackage{mytodo}
\usepackage{mycolor}
\usepackage{myps}
\usepackage{mymacro}
\usepackage{mypdf}

\makeatletter
% Modify the bibliography style
\newcounter{mymaxcitenames}
\AtBeginDocument{%
	\setcounter{mymaxcitenames}{\value{maxnames}}%
}
\renewbibmacro{begentry}{%
	\printtext[brackets]{%
		\defcounter{maxnames}{\value{mymaxcitenames}}%
		\printnames{labelname}~\usebibmacro{cite:labelyear+extrayear}%
	}%
	\newline%
}
% AAA
\newcommand{\myACER}{\textsc{acer}\xspace}
\newcommand{\myAlanine}{Alanine\xspace}
\newcommand{\myanalysis}[1]{\input{files/#1}\%}
\newcommand{\myangstrom}{\AA ngström\xspace}
\newcommand{\myanova}[1]{\input{files/#1}}
\newcommand{\myatom}[2][]{%
	{%
		\ifstrempty{#1}%
			{\makefirstuc{\textsf{#2}}}%
			{\textcolor{#1}{\makefirstuc{\textsf{#2}}}}%
		\xspace%
	}%
}
\newcommand{\myAudacity}{\textsc{audacity}\myregistered}% No '\xspace' because of already one in '\myregistered'
% CCC
\newcommand{\mycarbon}{\myatom[mycarboncolor]{C}}
\newcommand{\myCasioXJ}{\textsc{Casio xj}\xspace}
\newcommand{\myCHARMM}{\textsc{charmm}\xspace}
\newcommand{\myChimera}{\textsc{chimera}\xspace}
\newcommand{\myClayWorks}{\textsc{Clayworks}\xspace}
\newcommand{\mycondition}[1]{$\left(\mathcal{C}_{#1}\right)$\xspace}
\newcommand{\myCPK}{\textsc{cpk}\xspace}
% DDD
\newcommand{\myDesktop}{\myPHANToM Desktop\myregistered}% No '\xspace' because of already one in '\myregistered'
% FFF
\newcommand{\myfeuillet}{feuillet-$\beta$\xspace}
\WithSuffix\newcommand\myfeuillet*{feuillets-$\beta$\xspace}
\newcommand{\myform}[1]{\textbf{\sffamily\MakeUppercase{#1}}}
% GGG
\newcommand{\myGhost}{\textsc{Ghost}\xspace}
\newcommand{\myGromacs}{\textsc{Gromacs}\xspace}
\newcommand{\mygroup}[1]{$\left(\mathcal{G}_{#1}\right)$\xspace}
% HHH
\newcommand{\myHaption}{\textsc{Haption}\xspace}
\newcommand{\myHawthorne}{\myemph{Hawthorne Works}\xspace}
\newcommand{\myHBonds}{\textit{HBonds}\xspace}
\newcommand{\myhelice}{hélice-$\alpha$\xspace}
\WithSuffix\newcommand\myhelice*{hélices-$\alpha$\xspace}
\newcommand{\myhypothesis}[1]{$\left(\mathcal{H}_{#1}\right)$\xspace}
% III
\newcommand{\myIntelCore}{Intel\myregistered Core\mytrademark~2 \textsc{q9450} (\mynum[GHz]{2.66})\xspace}
% JJJ
\newcommand{\myJmol}{\textsc{Jmol}\xspace}
% LLL
\newcommand{\myLCD}{\textsc{lcd}\xspace}
\newcommand{\myLicorice}{\textit{Licorice}\xspace}
\newcommand{\myLinux}{\textsc{Linux}\xspace}
% MMM
\newcommand{\myMacOS}{Mac~\textsc{OS}\xspace}
\newcommand{\myMDDriver}{\textsc{MDDriver}\xspace}
% NNN
\newcommand{\myNewRibbon}{\textit{NewRibbon}\xspace}
\def\mynode{%
	\@ifnextchar[{\mynode@i}{\mynode@i[style=nodestyle]}%
}
\def\mynode@i[#1](#2,#3)[#4]#5{%
	\rput(#2,#3){\Rnode{#4}{\psframebox[style=nodestyle,#1]{\vphantom{pÉ}#5}}}%
}
\newcommand{\mynytrogen}{\myatom[mynytrogencolor]{A}}
\newcommand{\myNusE}{\textsc{NusE}\xspace}
\newcommand{\myNusENusG}{\textsc{NusE:NusG}\xspace}
\newcommand{\myNusG}{\textsc{NusG}\xspace}
% OOO
\newcommand{\myOmni}{\myPHANToM Omni\myregistered}% No '\xspace' because of already one in '\myregistered'
\newcommand{\myOpenHaptics}{\textsc{OpenHaptics}\mytrademark}% No '\xspace' because of already one in '\mytrademark'
\newcommand{\myoxygen}{\myatom[myoxygencolor]{O}}
% PPP
\newcommand{\myPC}{\textsc{pc}\xspace}
\newcommand{\myPDB}{\textsc{pdb}\xspace}
\newcommand{\myPDBbase}{\emph{Protein~DataBase}\xspace}
\newcommand{\myPDBlink}[2]{\href{#1}{\textsc{\MakeLowercase{#2}}}}
\newcommand{\myPHANToM}{\textsc{phant}o\textsc{m}\xspace}
\newcommand{\myPremium}{\myPHANToM Premium\myregistered}% No '\xspace' because of already one in '\myregistered'
\newcommand{\myPrion}{Prion\xspace}
\newcommand{\myPSF}{\textsc{psf}\xspace}
\newcommand{\mypvalue}{$p$-value\xspace}
\newcommand{\myPyMOL}{\textsc{p}y\textsc{mol}\xspace}
% RRR
\newcommand{\myRAM}[2][Go]{\mynum[#1]{#2} de \textsc{ram}}
\newcommand{\myRasmol}{\textsc{RasMol}\xspace}
\newcommand{\myresidue}[1]{$\left(\mathcal{R}_{#1}\right)$\xspace}
% SSS
\newcommand{\myscenario}[1]{\textsc{#1}}
\newcommand{\mySensAble}{\textsc{SensAble}\xspace}
\newcommand{\myShaddock}{\textsc{Shaddock}\xspace}
\newcommand{\mySony}{\textsc{sony}\myregistered}% No '\xspace' because of already one in '\myregistered'
\newcommand{\mySpaceNavigator}{SpaceNavigator\myregistered}% No '\xspace' because of already one in '\myregistered'
\newcommand{\mysubject}[1]{$\mathcal{S}_#1$}
\newcommand{\mysulfur}{\myatom[mysulfurcolor]{S}}
\newcommand{\mysummary}[1]{\input{files/#1}}
% TTT
\newcommand{\myTCPIP}{\textsc{tcp/ip}\xspace}
\newcommand{\myThreeD}{\textsc{3d}\xspace}
\newcommand{\mytool}[1]{\myemph{#1}}
\newcommand{\myTRPCAGE}{\textsc{trp-cage}\xspace}
\newcommand{\myTRPZIPPER}{\textsc{trp-zipper}\xspace}
% UUU
\newcommand{\myUbiquitin}{Ubiquitin\xspace}
\newcommand{\myUbuntu}{\textsc{Ubuntu}~v$10.04$\xspace}
\newcommand{\myUSB}{\textsc{usb}\xspace}
\newcommand{\myuser}[1]{$\mathcal{#1}$}
% VVV
\newcommand{\myvar}[2]{$\left(\mathcal{V}_{\mathrm{#1}#2}\right)$\xspace}
\newcommand{\myvard}[1]{\myvar{d}{#1}}
\newcommand{\myvari}[1]{\myvar{i}{#1}}
\newcommand{\myVGA}{\textsc{vga}\xspace}
\newcommand{\myVirtuose}{\textsc{Virtuose}\mytrademark~\textsc{6d}\mynum{35}--\mynum{45}\xspace}
% WWW
\newcommand{\myWindows}{\textsc{Windows}\xspace}

% Needed lengths
\newlength{\mywidth}
\newlength{\myheight}

% PSTricks style
\newpsstyle{nodestyle}{framearc=0.25,shadow=true,shadowcolor=myblue,blur=true}
\makeatother

\NeedsTeXFormat{LaTeX2e}[1999/01/01]
\ProvidesPackage{mycolor}[2011/04/29]

%%%%%%%%%%%%%%%%%%%%%%%%%%%%%
%% Declare package options %%
%%%%%%%%%%%%%%%%%%%%%%%%%%%%%
% In case of unknown options
\DeclareOption*{%
	\PackageWarning{mycolor}{Unknown option `\CurrentOption'}%
}

\ProcessOptions

%% Options to pass to packages

%% Packages to call
\RequirePackage{xcolor}

%%%%%%%%%%%%%%%%%%%%%%%
%% New configuration %%
%%%%%%%%%%%%%%%%%%%%%%%
\definecolor{mydarkred}{rgb}{0.5265625 0.25 0.25}
\definecolor{myred}{rgb}{0.7265625 0 0}
\definecolor{mylightred}{rgb}{0.9265625 0.5 0.5}
\definecolor{mylightestred}{rgb}{1 0.66 0.66}
\definecolor{mydarkblue}{rgb}{0 0 0.2265625}
\definecolor{myblue}{rgb}{0 0 0.7265625}
\definecolor{mylightblue}{rgb}{0.500 0.500 0.9265625}
\definecolor{mylightestblue}{rgb}{0.7500 0.7500 0.9265625}
\definecolor{mygreen}{rgb}{0 0.7265625 0}
\definecolor{mylightgreen}{rgb}{0.500 0.9265625 0.500}
\definecolor{mylightestgreen}{rgb}{0.7500 0.9265625 0.7500}
\definecolor{mygray}{gray}{0.6666667}

%%%%%%%%%%%%%%%%%%
%% New commands %%
%%%%%%%%%%%%%%%%%%

% End of package
\endinput

% AAA
\mynewacro{acr-AFM}%
{%
	name={\textsc{afm}},
	first={microscope à force atomique (\textsc{afm} pour \myemph{Atomic Force Microscope})},%
	plural={\textsc{afm}s},%
	firstplural={microscopes à force atomique (\textsc{afm} pour \myemph{Atomic Force Microscope})},%
	description={Microscope permettant l'observation de la topologie de la surface d'un échantillon au niveau atomique}
}
\mynewacro{acr-API}%
{%
	name={\textsc{api}},%
	first={interface de programmation (\textsc{api})},%
	plural={\textsc{api}s},%
	firstplural={interfaces de programmation (\textsc{api}s)},%
	description={\textsc{api} vient de l'anglais \myemph{Application Programming Interface} et désigne une interface avec un programme informatique}%
}
% BBB
% Be careful, because this word has no plural form, but the femina word in plural form
\mynewglos{glo-Bimanuel}%
{%
	name={bimanuel},%
	description={Qui se fait avec les deux mains},%
	plural={bimanuelle}%
}
\mynewglos{glo-Binome}%
{%
	name={binôme},%
	description={Groupe constitué de \mynum{2}~personnes},%
	plural={binômes}%
}
% CCC
\mynewacro{acr-CAO}%
{%
	name={\textsc{cao}},%
	first={conception assistée par ordinateur (\textsc{cao})},%
	description={La \textsc{cao} permet de concevoir et de tester virtuellement, à l'aide d'outils informatique, des produits manufacturés}%
}
\mynewglos{glo-ConflitDeCoordination}%
{%
	name={conflit de coordination},%
	description={Conflit entre deux sujets qui peut survenir lorsque les deux sujets tente d'accéder ou de déformer un objet au même instant},%
	plural={conflits de coordination}%
}
\mynewacro{acr-CUDA}%
{%
	name={\textsc{cuda}},%
	first={\textsc{cuda} (\myemph{Compute Unified Device Architecture})},%
	description={Technologie permettant d'utiliser l'unité graphique d'un ordinateur pour effectuer des calculs à hautes performances}%
}
\mynewglos{glo-Curseur}%
{%
	name={curseur},%
	description={Élément virtuel associé à un élément physique que le sujet manipule; il est lié à l'\myglos{glo-EffecteurTerminal}},%
	plural={curseurs}%
}
% DDD
\mynewacro{acr-DDL}%
{%
	name={\textsc{ddl}},%
	first={degré de liberté (\textsc{ddl})},%
	plural={\textsc{ddl}s},%
	firstplural={degrés de liberté (\textsc{ddl}s)},%
	description={Mouvements relatifs indépendants d'un solide par rapport à un autre}%
}
\mynewglos{glo-DockingMoleculaire}%
{%
	name={\myemph{docking} moléculaire},%
	description={Méthode permettant de déterminer l'orientation et la déformation optimale de \mynum{2}~molécules afin qu'elle s'assemble pour former un complexe de molécules stable},%
	plural={\myemph{docking} moléculaires}%
}
% EEE
\mynewglos{glo-EffecteurTerminal}%
{%
	name={effecteur terminal},%
	description={Élément physique que le sujet manipule; il est lié au \myglos{glo-Curseur} du monde virtuel},%
	plural={effecteurs terminaux}%
}
\mynewacro{acr-EVC}%
{%
	name={\textsc{evc}},%
	first={Environnement Virtuel Collaboratif (\textsc{evc})},%
	firstplural={Environnements Virtuels Collaboratifs (\textsc{evc})},%
	description={Ensemble logiciel et matériel permettant de faire interagir plusieurs utilisateurs au sein d'un même environnement; ils jouent un rôle important dans le développement de nouvelles méthodes de travail collaboratives}%
}
% HHH
\mynewglos{glo-Homoscedasticite}%
{%
	name={homoscedasticité},%
	description={Équivalent à homogénéité des variances; permet de comparer des variables aléatoires possédant des variances similaires},%
	plural={homoscedasticités}%
}
% III
\mynewacro{acr-IBPC}%
{%
	name={\textsc{ibpc}},%
	first={Institut de Biologie Physico-Chimie (\textsc{ibpc})},%
	description={Institut de recherche, géré par la fédération de recherche \textsc{frc}~\mynum{550}, étudiant les bases structurales, génétiques et physico-chimiques à leur différents niveaux d'intégration}%
}
\mynewacro{acr-IMD}%
{%
	name={\textsc{imd}},%
	first={\textsc{imd} (\myemph{Interactive Molecular Dynamics})},%
	description={Programme permettant de connecter le logiciel de visualisation moléculaire \myacro-{acr-VMD} avec le logiciel de simulation \myacro-{acr-NAMD} pour une simulation interactive en temps-réel \mycite{Stadler-1997}}%
}
\mynewacro{acr-ITAP}%
{%
	name={\textsc{itap}},%
	first={\myemph{Institut für Theoretische und Angewandte Physik} (\textsc{itap})},%
	description={Institut de Physique Théorique et Appliquée de \myname{Stuttgart} à l'origine du développement du logiciel \myacro{acr-IMD}}%
}
% LLL
\mynewacro{acr-LIMSI}%
{%
	name={\textsc{cnrs--limsi}},%
	first={Laboratoire pour l'Informatique, la Mécanique et les Sciences de l'Ingénieur (\textsc{cnrs--limsi})},%
	description={Unité Propre de Recherche du \textsc{cnrs} (\textsc{upr}~3251) associé aux universités \textsc{Paris} Sud et Pierre et Marie \textsc{Curie}}%
}
% MMM
\mynewglos{glo-Meneur}%
{%
	name={meneur},%
	description={En anglais \myemph{leader}, personne qui dirige un groupe afin d'atteindre des objectifs communs à ce groupe; c'est celui qui prend les décisions (voir aussi \myglos{glo-Suiveur})},%
	plural={meneurs}%
}
\mynewglos{glo-Monomanuel}%
{%
	name={monomanuel},%
	description={Qui se fait avec une main},%
	plural={monomanuelle}%
}
\mynewglos{glo-Monome}%
{%
	name={monôme},%
	description={\myemph{Groupe} constitué d'une unique personne},%
	plural={monômes}%
}
\mynewglos{glo-MotivationSociale}%
{%
	name={motivation sociale},%
	description={En anglais \myemph{social facilitation} \mycite{Triplett-1900}, phénomène de groupe où les personnes fournissent plus d'efforts grâce à la présence de partenaires},%
	plural={motivation sociale}%
}
\mynewacro{acr-TRM}%
{%
	name={\textsc{trm}},
	first={Théorie des Ressources Multiples (\textsc{trm})},%
	description={Cette théorie, élaborée par \mycite[author]{Wickens-1984} (\textsc{mrt} pour \myemph{Multiple Resource Theory}), propose un modèle pour la gestion des charges de travail pour un humain}
}
% NNN
\mynewacro{acr-NAMD}%
{%
	name={\textsc{namd}},%
	first={\textsc{namd} (\myemph{Scalable Molecular Dynamics})},%
	description={Programme de simulation pour la dynamique moléculaire \mycite{Phillips-2005}}%
}
% PPP
\mynewglos{glo-ParesseSociale}%
{%
	name={paresse sociale},%
	description={En anglais \myemph{social loafing} \mycite{Ringelmann-1913}, phénomène de groupe où les personnes fournissent moins d'effort pour la réalisation d'une tâche que s'ils effectuaient la tâche seuls},%
	plural={paresse sociale}%
}
\mynewacro{acr-PCV}%
{%
	name={\textsc{pcv}},
	first={Primitive Comportementale Virtuelle (\textsc{pcv})},%
	plural={\textsc{pcv}s},%
	firstplural={Primitives Comportementales Virtuelles (\textsc{pcv}s)},%
	description={Dans une application de réalité virtuelle, les activités d'un sujet peuvent toujours être décomposées en quatre comportements de base, appelés \myacro+{acr-PCV}, qui sont : observer, se déplacer, agir et communiquer \mycite{Fuchs-2006}}
}
% QQQ
\mynewglos{glo-Quadrinome}%
{%
	name={quadrinôme},%
	description={Groupe constitué de \mynum{4}~personnes},%
	plural={quadrinômes}%
}
% RRR
\mynewglos{glo-Residu}%
{%
	name={résidu},%
	description={Groupe d'atomes constituant un des blocs élémentaires d'une molécule},%
	plural={résidus}%
}
\mynewacro{acr-RMSD}%
{%
	name={\textsc{rmsd}},%
	first={\myemph{Root Mean Square Deviation} (\textsc{rmsd})},%
	description={Appelé Écart Quadratique Moyen en français, il permet -- dans le cadre de la biologie moléculaire -- de mesurer la différence entre deux déformations d'une même molécule}%
}
% SSS
\mynewglos{glo-StructureInformelle}%
{%
	name={structure informelle},%
	description={Groupe de personnes sans structures ni hiérarchie},%
	plural={structures informelles}%
}
\mynewglos{glo-Suiveur}%
{%
	name={suiveur},%
	description={En anglais \myemph{follower}, personne qui se laisse diriger dans un groupe afin d'atteindre des objectifs communs à ce groupe; c'est une personne qui ne prend pas de décision (voir aussi \myglos{glo-Meneur})},%
	plural={suiveurs}%
}
\mynewacro{acr-SUS}%
{%
	name={\textsc{sus}},%
	first={\textsc{sus} (\myemph{System Usability Scale})},%
	description={Échelle de notation entre \mynum{0} et \mynum{100} proposée par \mycite[author]{Brooke-1996} permettant d'évaluer l'utilisabilité d'un système}%
}
% TTT
\mynewglos{glo-Tetranome}%
{%
	name={tetranôme},%
	description={Groupe constitué de \mynum{4}~personnes},%
	plural={tetranômes}%
}
\mynewglos{glo-Trinome}%
{%
	name={trinôme},%
	description={Groupe constitué de \mynum{3}~personnes},%
	plural={trinômes}%
}
% UUU
\mynewacro{acr-UDP}%
{%
	name={\textsc{udp}},
	first={\textsc{udp} (\myemph{User Datagram Protocol} pour protocole de datagramme utilisateur)},%
	plural={\textsc{afm}s},%
	firstplural={\textsc{udp} (\myemph{User Datagram Protocol} pour protocole de datagramme utilisateur)},%
	description={c'est un des principaux protocole de télécommunication sur internet ; il a pour distinction de ne pas vérifier l'intégrité des données transmises}
}
\mynewacro{acr-UML}%
{%
	name={\textsc{uml}},%
	first={\textsc{uml} (\myemph{Unified Modeling Language})},%
	description={C'est un langage graphique de modélisation utilisé principalement en génie logiciel}%
}
% VVV
\mynewglos{glo-VariableDependante}%
{%
	name={variable dépendante},%
	description={Facteur mesuré sur une expérimentation (nombre de sélections, trajectoire, \myetc); ces variables sont influencées par les \myglos*{glo-VariableIndependante}},%
	plural={variables dépendantes}%
}
\mynewglos{glo-VariableIndependante}%
{%
	name={variable indépendante},%
	description={Facteur pouvant varier et être manipuler sur une expérimentation (nombre de participants, tâche, \myetc); ces variables vont avoir une incidence sur les \myglos*{glo-VariableDependante}},%
	plural={variables indépendantes}%
}
\mynewglos{glo-VariableInterSujets}%
{%
	name={variable inter-sujets},%
	description={Variables pour lesquelles les sujets sont confrontés à une et une seule des modalités de la variable},%
	plural={variables inter-sujets}%
}
\mynewglos{glo-VariableIntraSujets}%
{%
	name={variable intra-sujets},%
	description={Variables pour lesquelles les sujets sont confrontés à toutes les modalités de la variable},%
	plural={variables intra-sujets}%
}
\mynewacro{acr-VMD}%
{%
	name={\textsc{vmd}},%
	first={\textsc{vmd} (\myemph{Visual Molecular Dynamics})},%
	description={Programme de visualisation moléculaire \mycite{Humphrey-1996}}%
}
\mynewacro{acr-VRPN}%
{%
	name={\textsc{vrpn}},%
	first={\textsc{vrpn} (\myemph{Virtual Reality Protocol Network})},%
	description={Logiciel permettant de connecter différents périphériques de réalité virtuelle à une même application sous forme d'une architecture client/serveur \mycite{Taylor-II-2001}}%
}


\hypersetup{%
	pdftitle={Interactions haptiques collaboratives pour la manipulation moléculaire},%
	pdfauthor={Jean SIMARD},%
	pdfkeywords={collaboration,haptique,environnement virtuel,simulation moléculaire},%
	pdflang={FR-fr},%
	pdfsubject={Mémoire de thèse en informatique}%
}
\addglobalbib[datatype=bibtex]{biblio.bib}

\title{Interactions haptiques collaboratives pour la manipulation moléculaire}
\author{Jean~\myname{Simard}}
\documenttype{Thèse en Informatique}
\university{École Doctorale d'Informatique de Paris Sud}
\date{\mydate[datestyle=long]{01/12/2011}}
\jury{%
	Martin & \myname{DUPONT} & (rapporteur) & Directeur de recherche au \myacro-{acr-LIMSI} \\
	Martin & \myname{DUPOND} & (examinateur) & Directeur de recherche au \myacro-{acr-LIMSI}}

\begin{document}
	\frontmatter
	\maketitle
	\mytoc
	\mylof
	\mylot
	\mylotodo
	\mainmatter
	\begin{mypart}[prt-Introduction]{Introduction}
		\begin{mychapter}[cha-LeSujet]{Le sujet}
			\begin{mysection}[sec-EtatDeLArt]{État de l'art}
			\end{mysection}
			\begin{mysection}[sec-Contexte]{Contexte}
				\begin{mysubsection}[sse-LAmarrageMoleculaire]{L'\myglosnl{glo-AmarrageMoleculaire}}
					Le contexte de l'expérimentation est l'\myglos{glo-AmarrageMoleculaire} plus communément nommé \myglos{glo-DockingMoleculaire}.
					Ce processus implique une analyse et une manipulation complexe reposant sur plusieurs expertises.
					Il est basé sur une décomposition en trois niveaux de modélisation, traités du niveau le plus grossier au niveau le plus fin :
					\begin{description}
						\item[Niveau inter-moléculaire] Cette déformation au niveau macro-moléculaire applique des transformations de grande amplitude sur chaque molécule.
							L'objectif est de trouver la meilleure concordance entre les molécule en terme de position et d'orientation.
						\item[Niveau intra-moléculaire] Cette déformation au niveau moléculaire fait suite à la déformation inter-moléculaire.
							L'amarrage de ces deux molécules (ou plus) introduit de nombreuses interfaces qui doivent être optimisées en fonction de critères variés (la complémentarité des surfaces, les forces électrostatiques, les forces de \myname[van der]{Waals} \mycite{Muller-1994}, \myetc).
						\item[Niveau atomique] Cette déformation très fine va chercher à optimiser la position des atomes au niveau de l'interface.
							L'intérêt de cette étape sera portée sur plusieurs types d'interaction (les ponts hydrogènes, les zones hydrophobiques et hydrophylliques, les ponts salins, \myetc).
					\end{description}

					Pour chacun de ces différents niveaux, le processus de manipulation est similaire et peut être séparé en tâches élémentaires \myref*{fig-ProcessusDeDeformationMoleculaireEnQuatreEtapes} :
					\begin{description}
						\item[Recherche] Cette tâche concerne l'identification et la recherche d'une cible (atome, \myglos{glo-Residu}, \myhelice*, \myfeuillet*, \myetc) en fonction de critères multiples (articulations, bilan énergétique, régions hydrophobique, \myetc).
						\item[Sélection] Une fois la cible trouvée, la tâche consiste à accéder puis à sélectionner la cible par l'intermédiaire d'un périphérique d'entrée (une souris, une interface haptique, \myetc).
						\item[Déformation] La tâche consiste à déformer la structure en manipulant la cible précédemment sélectionnée, que ce soit au niveau inter-moléculaire, intra-moléculaire ou atomique.
							L'objectif inhérent à cette tâche et d'atteindre l'objectif fixé (par exemple, minimiser l'énergie totale du système).
						\item[Évaluation] Cette dernière partie va évaluer le travail précédemment réalisé en observant différents indicateurs (énergie potentielle, énergie électrostatique, complémentarité des surfaces, \myetc).
							En fonction de la synthèse des résultats de cette dernière phase, un nouveau cycle pourra recommencer (recherche, sélection, déformation, évaluation, \myetc).
					\end{description}

					\begin{myfigure}
						\psset{xunit=0.2\textwidth}
						\def\mycirclenum(#1,#2)#3{%
							\uput{5em}[0](#1,#2){\pscirclebox*[fillcolor=myblue!70]{\white #3}}%
						}
						\begin{myps}(-2.5,-1)(2.5,5)
							\mynode(0,4)[Search]{Recherche}
							\mycirclenum(0,4){1}
							\mynode(0,3)[Selection]{Sélection}
							\mycirclenum(0,3){2}
							\mynode(0,2)[Manipulation]{Manipulation}
							\mycirclenum(0,2){3}
							\mynode(0,1)[Evaluation]{Évaluation}
							\mycirclenum(0,1){4}
							\mynode[fillstyle=solid,fillcolor=myblue!25](0,-0.5)[Objective]{Objectif atteint}
							\ncline{->}{Search}{Selection}
							\ncline{->}{Selection}{Manipulation}
							\ncline{->}{Manipulation}{Evaluation}
							\ncline{->}{Evaluation}{Objective}
							\ncloop[loopsize=4em,angleA=-90,angleB=90,linearc=0.05]{->}{Evaluation}{Search}
						\end{myps}
						\mycaption[fig-ProcessusDeDeformationMoleculaireEnQuatreEtapes]{Processus de déformation moléculaire en quatre étapes}
					\end{myfigure}
				\end{mysubsection}
			\end{mysection}
		\end{mychapter}
		\begin{mychapter}[cha-ShaddockSystemeCollaboratifPourLaManipulationDeMolecules]{\myShaddock\ -- Système collaboratif de manipulation de molécules}
			\begin{mysection}[sec-PlateFormesCollaborativesExistantes]{Plates-formes collaboratives existantes}
				\myShaddock est le nom de l'\myacro{acr-EVC} développé dans le cadre de cette thèse.
				Les \myacro*{acr-EVC} existants sont nombreux mais aucun ne convient à nos besoins.
				Plusieurs choix d'architectures ont été nécessaires et sont exposés dans cette section.
				Une vue schématique de la plate-forme \myShaddock est détaillée sur la \myref{fig-SchemaDeLaPlateFormeShaddock}.
				Les différents éléments de ce schéma seront expliqués dans les sections qui vont suivre.
				La \myref{sse-ArchitectureDuSysteme} tranchera sur la question de l'architecture pair-à-pair comparée à l'architecture client/serveur.
				Ensuite, la gestion d'une vue partagée est abordée dans la \myref{sse-VuePartagee}.
				\begin{myfigure}
					\psset{unit=0.05\textwidth}
					\def\mycomputer#1{%
						\textcolor{black!33}{\scriptsize Ordinateur~#1}%
					}
					\def\myVRPNclient{%
						\mymultiline{
							Client \\
							\myacronl-{acr-VRPN} \\
						}%
					}
					\def\myVRPNclient{%
						\mymultiline{
							Client \\
							\myacronl-{acr-VRPN} \\
						}%
					}
					\def\myVRPNserver#1{%
						\mymultiline[r]{
							\mycomputer{#1} \\[-1ex]
							Serveur \myacronl-{acr-VRPN}
						}%
					}
					\def\myNAMDserver#1{%
						\mymultiline[r]{
							\mycomputer{#1} \\[-1ex]
							Serveur \myacronl-{acr-NAMD}
						}%
					}
					\def\myvmdnode(#1,#2){%
						\rput(#1,#2){%
							\rnode{VMD}{%
								\psframebox[style=nodestyle,framesep=0pt]{%
									\begin{tabular}{c|c}%
										\multicolumn{2}{r}{\mycomputer{0}} \\[-0.5ex]
										\multicolumn{2}{c}{\psframebox*[linecolor=white]{\Large\myacronl-{acr-VMD}}} \\
											\hline
											\myVRPNclient & \psframebox*[linecolor=white]{\myacronl-{acr-IMD}} \\
										\end{tabular}%
									}%
								}%
							}%
						}
						\begin{myps}(-10,-5)(10,5)
							\rput(-7.5,4){\textit{Interaction}}
							\psline[linewidth=0.1pt,linestyle=dashed,linecolor=black!33](-4,5)(-4,-5)
							\rput(0,4){\textit{Visualisation}}
							\psline[linewidth=0.1pt,linestyle=dashed,linecolor=black!33](4,5)(4,-5)
							\rput(7.5,4){\textit{Simulation}}
							\myvmdnode(0,0)
							\mynode(7.5,0)[NAMD]{\myNAMDserver{n+1}}
							\mynode(-7.5,2)[VRPN1]{\myVRPNserver{1}}
							\mynode(-7.5,0)[VRPN2]{\myVRPNserver{2}}
							\mynode(-7.5,-3)[VRPNn]{\myVRPNserver{n}}
							\rput(-7.5,-1.5){\LARGE$\vdots$}
							\psset{arm=15pt,linearc=.5}
							\ncdiag[offsetA=-10pt,angleA=0,angleB=180]{<->}{VMD}{NAMD}
							\psset{angleA=180,angleB=0,offsetA=10pt,ncurvA=1}
							\ncdiag{<->}{VMD}{VRPN1}
							\ncdiag{<->}{VMD}{VRPN2}
							\ncdiag{<->}{VMD}{VRPNn}
						\end{myps}
						\mycaption[fig-SchemaDeLaPlateFormeShaddock]{Schéma de la plate-forme \myShaddock}
					\end{myfigure}
				\begin{mysubsection}[sse-ArchitectureDuSysteme]{Architecture du système}
					Deux types d'architectures sont possibles pour les \myacro*{acr-EVC} : client/serveur ou pair-à-pair.
					Parmi les architectures pair-à-pair, \mycite[author]{Iglesias-2008} propose une tâche d'assemblage collaboratif assisté par l'haptique.
					\mycite[author]{Kim-2004} étudie le déplacement collaboratif d'une boîte également assisté par l'haptique.
					Aucune des deux plate-formes proposées ne souffre d'instabilités notables dans le rendu haptique.
					Dans les \myacro*{acr-EVC}, l'architecture pair-à-pair permet un rendu haptique relativement stable puisque chaque nœud du système gère sa propre simulation et n'est pas lié au latence voire au coupure du réseau.
					
					Cependant, des travaux proposent également des architectures client/serveur avec des interactions haptiques.
					\mycite[author]{Huang-2010} propose une interaction haptique client/serveur pour la manipulation d'un jeu de construction par blocs.
					\mycite[author]{Norman-2010} s'intéresse particulièrement aux influences du réseau sur les interactions visuo-haptiques.
					L'avantage d'une architecture client/serveur est la cohérence de la simulation entre les différents nœuds du système.
					En effet, seul le serveur effectue la simulation et ensuite distribue les données de simulation aux différentes nœuds.

					\mycite[author]{Marsh-2006} propose une comparaison de ces deux types d'architectures et en vient à la conclusion que l'architecture pair-à-pair est la plus performante en terme de latence.
					En effet, l'architecture pair-à-pair permet un lien direct entre tous les nœuds du système.
					Dans le cas d'une architecture client/serveur, le serveur sert toujours d'intermédiaire ce qui augmente le nombre de rebonds des paquets réseaux.

					Cependant, les systèmes pair-à-pair sont adaptés pour les environnements virtuels statiques ou faiblement dynamiques.
					En effet, des latences ou des coupures du réseau introduisent des erreurs de synchronisation entre les simulations ayant pour conséquence une divergence des résultats de simulation entre les différents nœuds.
					L'architecture client/serveur, bien que moins performante en terme de latence, permettra de conserver une cohérence de la simulation.
					La simulation est effectuée sur un serveur et les données de simulation sont transmises à tous les nœuds de l'\myacro{acr-EVC}.
					Ce type d'architecture est très adapté pour les environnements virtuels dynamiques.

					Étant donné les besoins en simulation, la plate-forme \myShaddock est une architecture client/serveur.
				\end{mysubsection}
				\begin{mysubsection}[sse-VuePartagee]{Vue partagée}
					Dans un \myacro{acr-EVC}, un point important est la disponibilité de la vue.
					Elle peut être soit privée pour chaque utilisateur, soit partagée par tous les utilisateurs.

					L'un des premiers travaux sur la collaboration haptique est fourni par \mycite[author]{Basdogan-2000} dans lequel deux utilisateurs placés dans deux pièces différentes doivent effectuer une tâche nécessitant de la synchronisation.
					\mycite[author]{Oliveira-2002} propose un \myacro{acr-EVC} intégrant chaque utilisateur comme un avatar dans l'environnement virtuel.
					Dans les deux cas, chaque utilisateur n'a conscience des autres utilisateurs que par l'intermédiaire de l'\myacro{acr-EVC}.
					La communication est alors restreinte aux modalités autorisées (strictement haptique et partiellement visuelle dans ces deux cas).
					L'impossibilité de communiquer oralement peut devenir un frein dans la collaboration.

					Être conscient des faits et gestes des autres utilisateurs dans un \myacro{acr-EVC} a prouvé son efficacité.
					\mycite[author]{Sallnas-2010} a beaucoup travaillé sur l'apport de l'haptique sur la conscience périphérique.
					Elle montre à plusieurs reprises l'intérêt d'être conscients des faits et gestes des autres utilisateurs ainsi que de l'état de l'environnement.
					\mycite[author]{Casera-2006} privilégie une vue publique à tous les utilisateurs pour améliorer la conscience périphérique.
					Enfin, \mycite[author]{Tang-2006} a exploré la collaboration sur une table tactile : tous les utilisateurs évoluent autour de la même table tactile et possède donc une conscience accrue des faits et gestes de chacun.

					\myShaddock doit permettre la collaboration sur des tâches complexes.
					L'objectif est d'obtenir une collaboration la plus complète possible afin d'exploiter au maximum le potentiel de chaque utilisateur présent.
					C'est pourquoi \myShaddock propose une vue publique permettant à tous les utilisateurs de se trouver au même endroit pour effectuer les tâches demandées.
					Ils ont ainsi la possibilité de communiquer librement.
					La vue publique est assurée par un vidéoprojecteur projetant la vue sur un grand écran en face des utilisateurs.
					Les utilisateurs sont disposés de front face à l'écran.
				\end{mysubsection}
			\end{mysection}
			\begin{mysection}[sec-SimulationMoleculaireEnTempsReel]{Simulation moléculaire en temps-réel}
				La plate-forme \myShaddock permet de visualiser des simulations moléculaires en temps-réel.
				Dans cette section, nous commençons par identifier les besoins en terme de simulation dans la \myref{sse-simulation-LesBesoins} puis par exposer les solutions logicielles retenues dans la \myref{sse-simulation-LesOutilsExistants}.
				\begin{mysubsection}[sse-simulation-LesBesoins]{Les besoins}
					\myShaddock permet d'effectuer la visualisation des molécules.
					La visualisation est un processus complexe qui nécessite des rendus variés et complets.
					En effet, devant le nombre important d'informations disponibles concernant une molécule, il est primordial d'avoir un rendu graphiques de molécules clair et complet sans être surchargé.
					Cette tâche est effectuée par le logiciel \myacro{acr-VMD} \myref*{sss-simulation-VMD}.

					Ensuite, \myShaddock simule un environnement moléculaire.
					Un logiciel de simulation est nécessaire pour réaliser cette tâche.
					Il faut que ce logiciel puisse interagir avec \myacro{acr-VMD}.
					De plus, il est nécessaire de pouvoir paramétrer finement la simulation (par exemple, fixer certains atomes).
					\myacro{acr-NAMD} est le parfait candidat \myref*{sss-simulation-NAMD}.

					Cependant, \myacro{acr-NAMD} n'est pas conçu pour effectuer des simulations en temps-réel.
					Pour proposer une interaction aux utilisateurs, il est nécessaire d'avoir accès à une simulation interactive et en temps-réel.
					La solution à ce problème vient avec \myacro{acr-IMD} développé par l'\myacro{acr-ITAP} et présenté en \myref{sss-simulation-IMD}.
				\end{mysubsection}
				\begin{mysubsection}[sse-simulation-LesOutilsExistants]{Les outils existants}
					Trois outils permettent de fournir une simulation en temps-réel de molécules : \myacro{acr-VMD} pour la visualisation, \myacro{acr-NAMD} pour la simulation et \myacro{acr-IMD} pour la simulation en temps-réel.
					Ces trois outils sont présentés dans les sections suivantes.
					\begin{mysubsubsection}[sss-simulation-VMD]{\myacronl+{acr-VMD}}
						Les outils de visualisation moléculaire disponibles sont relativement nombreux.
						Parmi les plus populaires, on peut citer \myPyMOL \mycite{PyMOL-2010}, \myacro{acr-VMD} \mycite{Humphrey-1996}, \myChimera \mycite{Pettersen-2004}, \myRasmol \mycite{Sayle-1995} sans compter les nombreux dérivés permettant un affichage en ligne tel que \myJmol \mycite{Jmol-2006} pour ne citer que le plus connu.
						\myPyMOL et \myacro{acr-VMD} se distinguent particulièrement par leurs nombreuses fonctionnalités et leur large utilisation dans le milieu spécialisé.

						\myPyMOL est probablement le logiciel de visualisation le plus utilisé par les experts du domaine car c'est le plus complet pour fournir des rendus graphiques de molécules très complets.
						Cependant, \myPyMOL ne permet pas l'affichage de simulations temps-réel ni la manipulation interactive de molécules.

						\myacro{acr-VMD} possède également une large gamme de rendus graphiques.
						Contrairement à \myPyMOL, \myacro{acr-VMD} est adapté pour le rendu graphique en temps-réel de données de simulation.
						Il permet également la manipulation interactive de molécules.
						Les fonctionnalités de \myacro{acr-VMD} sont nombreuses et seulement certaines on été utilisées dans le cadre des expérimentations qui vont suivre.
						Elles sont exposées dans les paragraphes suivants.

						\begin{myparagraph}[par-simulation-LesRendusGraphiques]{Les rendus graphiques}
							La possibilité d'avoir accès à des rendus graphiques divers et complets est primordiale pour la visualisation moléculaire.
							La complexité des molécules, le nombre important d'atomes, les nombreuses meta-informations, les structures particulières nécessitent d'avoir à sa disposition des moyens évolués et variés pour afficher une molécule.
							Quatre représentations différentes \myref*{fig-simulation-IllustrationsDesRepresentationsDeMoleculesSurVMD} ont été utilisées sur la plate-forme \myShaddock :
							\begin{description}
								\item[\myCPK] affiche tous les atomes de la molécule sous forme de sphères en les reliant par des cylindres; c'est un affichage très chargé lorsque le nombre d'atomes est importants mais on peut modifier la taille des sphères et des cylindres \myref*{fig-simulation-CPK};
								\item[\myLicorice] représente tous les liens entre les atomes par des cylindres, sans représenter les atomes; la taille des cylindres peut être modifiée \myref*{fig-simulation-Licorice};
								\item[\myNewRibbon] produit une courbe spline sur les atomes $C_{\alpha}$ représentant l'armature principale de la molécule; la courbe est représentée sous forme de ruban \myref*{fig-simulation-NewRibbon};
								\item[\myHBonds] affiche les potentielles liaisons hydrogène sous forme de traits en pointillés; les seuils d'affichage ainsi que les paramètres de la ligne en pointillés (couleur, largeur, \myetc) sont modifiables \myref*{fig-simulation-HBonds}.
							\end{description}

							\begin{myfigure}
								\begin{mysubfigure}
									\myimage[width=0.49\textwidth]{simulation-CPK}
									\mysubcaption[fig-simulation-CPK]{\myCPK}
								\end{mysubfigure}
								\begin{mysubfigure}
									\myimage[width=0.49\textwidth]{simulation-Licorice}
									\mysubcaption[fig-simulation-Licorice]{\myLicorice}
								\end{mysubfigure}
								\begin{mysubfigure}
									\myimage[width=0.49\textwidth]{simulation-NewRibbon}
									\mysubcaption[fig-simulation-NewRibbon]{\myNewRibbon}
								\end{mysubfigure}
								\begin{mysubfigure}
									\myimage[width=0.49\textwidth]{simulation-HBonds}
									\mysubcaption[fig-simulation-HBonds]{\myHBonds}
								\end{mysubfigure}
								\mycaption[fig-simulation-IllustrationsDesRepresentationsDeMoleculesSurVMD]{Illustration des représentations de molécules sur \myacronl{acr-VMD}}
							\end{myfigure}

							Chacune de ces représentations visuelles peut être affectée à tout ou partie de la molécule comme par exemple \og le \myglos{glo-Residu} \mynum{13} \fg, \og seulement les atomes de carbone \fg ou \og tous les \myglos*{glo-Residu} entre \mynum{1} et \mynum{16} sauf les atomes d'hydrogène \fg.
							De plus, pour chacune des représentations précédentes, différentes colorations sont possibles :
							\begin{description}
								\item[Couleur fixe] donne une couleur unie prédéfinie (pour la couleur du curseur par exemple);
								\item[Couleur des atomes] donne une couleur différente à chaque atome selon un code couleur standard dépendant de sa nature (rouge pour oxygène, blanc pour hydrogène, \myetc);
								\item[Couleur des \myglosnl*{glo-Residu}] donne une couleur différente pour chaque atome selon une palette de couleurs prédéfinie par \myacro{acr-VMD};
								\item[Transparence] rend transparent les objets tout en conservant la teinte;
								\item[\textit{GoodSell}] accentuant les contours des objets sous le principe du \myemph{cell shading}.
							\end{description}
						\end{myparagraph}
						\begin{myparagraph}[par-simulation-LesOutilsDeManipulation]{Les outils de manipulation}
							La manipulation des molécules est nécessaire sur la plate-forme \myShaddock.
							\myAcro{acr-VMD} dispose déjà de différents outils permettant d'effectuer différentes manipulation sur les molécules.

							Par défaut et sans configuration, la souris permet d'orienter la scène sur trois \myacro*{acr-DDL} afin d'observer la molécule sous différents angles.

							Il est également possible d'utiliser une souris \myThreeD, automatiquement détectée lorsqu'elle est branchée sur l'ordinateur.
							Une souris \myThreeD permet de translater et d'orienter la scène.
							La souris \myThreeD \mySpaceNavigator est utilisée dans le cadre de certaines de nos expérimentations.

							Enfin, des outils spécifiques sont disponibles par l'intermédiaire d'une connexion avec \myacro{acr-VRPN} \myref*{sss-interaction-VRPN}.
							Ces outils sont liés à des périphériques externes (des interfaces \myOmni dans notre cas).
							Les outils disponibles par défaut dans \myacro{acr-VMD} ont été utilisés dans la première expérimentation \myref*{cha-RechercheCollaborativeDeResidusEnSimulationMoleculaire} et sont :
							\begin{description}
								\item[\mytool{grab}] qui permet de sélectionner une molécule dans son intégralité et de la déplacer dans la scène;
								\item[\mytool{tug}] qui permet de sélectionner un atome de la molécule et de lui appliquer une force (qui sera transmise à la simulation).
							\end{description}

							Cependant, de nombreux outils supplémentaires ont été développés au-fur-et-à-mesure des besoins identifiés durant les expérimentations.
							Ces nouveaux outils sont détaillés dans la \myref{sec-NouveauxOutilsPourLInteraction}.
						\end{myparagraph}
						\begin{myparagraph}[par-simulation-LaGenerationAutomatiqueDeFichierDeSimulation]{La génération automatique de fichier de simulation}
							La simulation nécessite de nombreuses informations dont l'ensemble des liaisons entre atomes, des angles simples, des angles dihédraux et des angles de torsion.
							La simple description des atomes et de leurs positions (fichier \myPDB) couplée aux résultats de \myCHARMM \mycite{Brooks-1983} permet de générer les fichiers nécessaires à la simulation.
							\myacro{acr-VMD} fournit tous les outils permettant de générer ce fichier nécessaire à la simulation (fichier \myPSF) par l'intermédiaire d'une extension: \myemph{Automatic \textsc{psf} builder}.
						\end{myparagraph}
					\end{mysubsubsection}
					\begin{mysubsubsection}[sss-simulation-NAMD]{\myacronl+{acr-NAMD}}
						Les deux logiciels de simulation principaux existants sont \myacro{acr-NAMD} \mycite{Phillips-2005} et \myGromacs \mycite{Berendsen-1995}.
						Bien que \myGromacs soit plus performant que \myacro{acr-NAMD}, surtout dans les dernières versions \mycite{Hess-2008} qui offre des performances jusqu'à quatre fois plus rapide que \myacro{acr-NAMD}.
						Cependant, \myacro{acr-NAMD} est développé par la même université que \myacro{acr-VMD} et l'interaction entre les deux logiciels est donc extrêmement facilitée.
						De plus, les petites molécules que nous utiliserons lors de nos simulations ne nécessitent pas des performances exceptionnelles.
						Enfin, \myacro{acr-NAMD} peut être aisément connecté à \myacro{acr-VMD} dans le cadre d'une simulation interactive \myref*{sss-simulation-IMD} contrairement à \myGromacs.
						C'est pourquoi le logiciel \myacro{acr-NAMD} a été retenu pour notre plate-forme.

						Une des fonctionnalités de \myacro{acr-NAMD} utilisée est la possibilité de \myemph{fixer} des atomes.
						En effet, la fixation d'atomes permet d'exclure partiellement certains atomes durant la simulation.
						Ces atomes interviennent dans le calcul des forces de la simulation mais eux-mêmes ne sont pas soumis aux forces de l'environnement.
						Cette fonctionnalité est nécessaire pour simuler un point d'ancrage de la molécule dans l'environnement virtuel.
						Sans ce point d'ancrage, la molécule pourrait dériver et sortir de l'espace de travail des utilisateurs sans possibilité de récupération.
					\end{mysubsubsection}
					\begin{mysubsubsection}[sss-simulation-IMD]{\myacronl+{acr-IMD}}
						Les logiciels de simulation ne sont pas développés pour des simulations interactives en temps-réel.
						Cependant, l'\myacro{acr-ITAP} a développé le protocole \myacro{acr-IMD} permettant d'utiliser \myacro{acr-NAMD} couplé à \myacro{acr-VMD} pour des simulations interactives en temps-réel \mycite{Stadler-1997}.
						L'extension \myemph{\myacronl{acr-IMD} connect} permet de connecter rapidement le logiciel \myacro{acr-VMD} avec la simulation de \myacro{acr-NAMD}.

						Cependant, entre le début du développement de notre plate-forme en 2008 et aujourd'hui, une nouvelle solution plus générique a été développée au sein de l'\myacro{acr-IBPC}.
						En effet, \myMDDriver \mycite{Delalande-2009} est une interface permettant d'utiliser le protocole \myacro{acr-IMD} avec l'autre logiciel de simulation \myGromacs.
						Il se présente sous la forme d'une interface permettant de choisir le logiciel de simulation ainsi que le logiciel de visualisation.
						Il permet également de connecter plusieurs logiciels de visualisation à une même simulation.
						Cependant, cette nouvelle solution n'a pas encore été implémentée dans notre plate-forme mais c'est une amélioration technique qui sera effectuée dans les prochaines versions de la plate-forme.
					\end{mysubsubsection}
				\end{mysubsection}
			\end{mysection}
			\begin{mysection}[sec-InteractionAvecLaSimulationMoleculaire]{Interaction avec la simulation moléculaire}
				La plate-forme \myShaddock implémente également un moyen d'interagir avec cette simulation en temps-réel.
				Les besoins particuliers pour la manipulation sont présentés dans la \myref{sse-interaction-LesBesoins} puis les solutions logicielles dans la \myref{sse-interaction-InteractionParInterfaceHaptique}.
				\begin{mysubsection}[sse-interaction-LesBesoins]{Les besoins}
					Afin de pouvoir modifier la simulation en temps-réel, il faut pouvoir manipuler des éléments de la molécule.
					Il faut donc être capable de sélectionner les atomes puis de leur appliquer une force.
					L'environnement virtuel est en \myThreeD et il paraît donc nécessaire d'avoir un périphérique de manipulation en \myThreeD.
					Cette manipulation sera effectuée par le \myOmni \myref*{sss-interaction-OmniEtOpenHaptics}.

					De plus, \myShaddock est un \myacro{acr-EVC} : il doit pouvoir accueillir plusieurs utilisateurs.
					La connexion de plusieurs périphériques apporte des contraintes matérielles et logicielles.
					Une architecture sous forme client/serveur sera assurée grâce au logiciel \myacro{acr-VRPN} \myref*{sss-interaction-VRPN}.
				\end{mysubsection}
				\begin{mysubsection}[sse-interaction-InteractionParInterfaceHaptique]{Interaction par interface haptique}
					\begin{mysubsubsection}[sss-interaction-OmniEtOpenHaptics]{\myOmni et \myOpenHaptics}
						Une plate-forme de simulation interactive en temps-réel nécessite des outils d'interaction.
						Parmi les périphériques d'interaction existants, il faut choisir un périphérique permettant au minimum six \myacro*{acr-DDL} en entrée et au minimum trois \myacro*{acr-DDL} en retour haptique.
						En effet, l'outil \mytool{grab} nécessite six \myacro*{acr-DDL} en entrée et l'outil \mytool{tug} nécessite trois \myacro*{acr-DDL} en entrée et trois \myacro*{acr-DDL} en retour haptique.
						L'interface \myOmni \mycite{Massie-1994} de l'entreprise \mySensAble répond aux attentes de la plate-forme \myShaddock \myref*{fig-simulation-InterfaceOmniSixDDLTroisDDL}.

						\begin{myfigure}
							\myimage{simulation-omni}
							\mycaption[fig-simulation-InterfaceOmniSixDDLTroisDDL]{Interface \myOmni 6~\myacronl-{acr-DDL}/3~\myacronl-{acr-DDL}}
						\end{myfigure}

						À l'origine, les interfaces haptiques de \mySensAble était programmable à l'aide de l'\myacro{acr-API} \myGhost \mycite{SensAble-2002}.
						Le travail de \mycite[author]{Itkowitz-2005} a permis de fournir une nouvelle \myacro{acr-API} plus facile à utiliser : \myOpenHaptics.
						C'est à partir de cette \myacro{acr-API} que les interfaces haptiques sont utilisées sur \myShaddock.
					\end{mysubsubsection}
					\begin{mysubsubsection}[sss-interaction-VRPN]{\myacronl+{acr-VRPN}}
						\myacro{acr-VMD} offre un moyen simple et relativement universel de connecter un périphérique.
						En effet, il gère les connexions de périphériques par l'intermédiaire de \myacro{acr-VRPN} \mycite{Taylor-II-2001}.
						\myacro{acr-VRPN} fonctionne sous la forme d'une architecture client/serveur où \myacro{acr-VMD} est l'application cliente.
						L'interface haptique est connectée physiquement à un autre ordinateur (le même ordinateur le cas échéant) et un serveur \myacro{acr-VRPN} communique avec cette interface.
						C'est seulement par l'intermédiaire de \myacro{acr-VRPN} et à travers le réseau \myemph{Ethernet} que \myacro{acr-VMD} va percevoir les mouvements de l'interface haptique et lui envoyer les efforts à fournir.
						La compilation de \myacro{acr-VRPN} en tant que serveur de \myOmni sous le système d'exploitation \myLinux (\myUbuntu $v10.04$) a demandé quelques modifications dans le code.
						Ces modifications ont été soumises au développeur de \myacro{acr-VRPN} qui les a intégrées dans les dernières versions.

						L'avantage de cette architecture est la possibilité d'ajouter autant de serveurs et donc autant d'interfaces haptiques que voulu.
						Cependant, cela suppose également d'avoir autant d'ordinateurs que de serveurs ce qui complique une logistique complexe.
						On pourra noter que la chaleur dégagée par l'ensemble de ces machines additionnée à celle du vidéo-projectuer provoque des conditions d'expérimentation rapidement désagréables.
						C'est pourquoi aucune des expérimentations proposées ne durait plus de \mynum[mn]{30} ou, le cas échéant, une pause est effectuée au bout de \mynum[mn]{30} afin d'aérer la salle d'expérimentation.
					\end{mysubsubsection}
				\end{mysubsection}
			\end{mysection}
			\begin{mysection}[sec-NouveauxOutilsPourLInteraction]{Nouveaux outils pour l'interaction}
				Durant les différentes études présentées dans la \myref{prt-EtudeDuTravailCollaboratif}, les analyses et les remarques d'utilisateurs ont permis d'améliorer les outils d'interaction et d'en proposer de nouveaux.
				Le développement de ces nouveaux outils a nécessité une modification du programme \myacro{acr-VMD} par extension des outils déjà existants.
				Des fonctionnalités ont été ajoutées et sont présentées dans les sections suivantes.
				\begin{mysubsection}[sse-AmeliorationDeLaSelection]{Amélioration de la sélection}
					Durant le processus de recherche et de sélection, les utilisateurs ont souvent évoqué le besoin de connaître en continu leur position et de savoir à priori l'élément qui va être sélectionné.
					Pour que les utilisateurs connaissent à chaque instant l'élément qui peut être sélectionné, une information visuelle met en surbrillance l'élément pointé à chaque instant.
					La mise en surbrillance est un agrandissement en transparence de l'élément pointé.
					La couleur de cette mise en surbrillance est de la même couleur que le curseur de l'utilisateur.

					Cependant, lorsque seul un atome est mis en surbrillance, il peut être difficile de l'apercevoir.
					En effet, le nombre important d'atomes d'une molécule peut surcharger le rendu graphique.
					C'est pourquoi, l'ensemble du \myglos{glo-Residu} auquel appartient l'atome pointé est également mis en surbrillance.
					Seul l'atome pointé est agrandi.

					Une fois l'élément pointé, l'utilisateur peut sélectionner l'élément.
					Lorsque les utilisateurs sélectionnent l'élément, la surbrillance passera de la transparence à l'opacité.
					Une illustration des effets visuels relatifs au pointage et à la sélection est affichée sur la \myref{fig-selection-improvement-DifferenceVisuelleEntreLesElementsPointesEtSelectionnes}.

					\begin{myfigure}
						\begin{mysubfigure}
							\myimage[width=0.49\textwidth]{selection-improvement-pointed}
							\mysubcaption[fig-selection-improvement-ElementPointe]{Élément pointé}
						\end{mysubfigure}
						\begin{mysubfigure}
							\myimage[width=0.49\textwidth]{selection-improvement-targeted}
							\mysubcaption[fig-selection-improvement-ElementSelectionne]{Élément sélectionné}
						\end{mysubfigure}
						\mycaption[fig-selection-improvement-DifferenceVisuelleEntreLesElementsPointesEtSelectionnes]{Différence visuelle entre les éléments pointés et sélectionnés}
					\end{myfigure}
				\end{mysubsection}
				\begin{mysubsection}[sse-DeformationParGroupeDAtomes]{Déformation par groupe d'atomes}
					L'outil \mytool{tug} permet de déformer la molécule en appliquant un effort à l'atome sélectionné.
					Cependant, la déformation par l'intermédiaire d'un seul atome possède deux désavantages.

					Tout d'abord, la déformation d'une molécule atome par atome est un processus très fastidieux.
					Il serait plus efficace de déplacer un groupe d'atomes en une seule fois.

					De plus, l'application d'un effort sur un atome provoque l'étirement de la molécule.
					Au repos, la molécule est dans état relativement stable.
					Déplacer un atome perturbe cet état de stabilité.
					De plus, certains atomes sont fortement liés et les éloigner peut perturber grandement l'état de stabilité.
					Il est donc préférable de déplacer tous ces atomes liés en une seule manipulation.

					C'est pourquoi un outil appliquant un effort à un groupe d'atomes permet de déplacer un bloc d'atomes tout en conservant une certaine stabilité.
					Les groupes d'atomes dignes d'intérêt sont les \myglos*{glo-Residu} (une vingtaine d'atomes), les \myhelice* ou \myfeuillet* (une vingtaine de \myglos*{glo-Residu}) et les molécules.
					Cependant, \myacro{acr-VMD} n'est pas capable de fournir l'information sémantique regroupant les atomes en \myhelice* ou en \myfeuillet*.
					La fonctionnalité de l'outil \mytool{tug} a donc été étendue aux \myglos*{glo-Residu} et aux molécules.

					Cependant, appliquer le même effort à l'ensemble des atomes de la molécule produit un effort total très important.
					Si l'effort total est trop important, les perturbations envoyées à la simulation sont trop importantes et produisent des incohérences dans la simulation voire même un arrêt de la simulation.
					Il est donc nécessaire de diviser l'intensité des forces proportionnellement au nombre d'atomes sélectionnés.
				\end{mysubsection}
				\begin{mysubsection}[sse-OutilDeDesignationEtAttraction]{Outil de désignation et attraction}
					Un aspect récurrent constaté durant les expérimentations est la nécessité de désigner un élément de la molécule.
					Parfois les utilisateurs éprouvent le besoin de désigner mais la plupart du temps, ce sont les enregistrement audio qui ont permis d'identifier ce besoin.

					L'outil de désignation a été conçu pour répondre à un processus en quatre étapes:
					\begin{enumerate}[label={\alph*.},ref={\alph*}]
						\item Recherche d'une cible \myref*{fig-designation-RechercheDUneCible};\label{enu-designation-RechercheDUneCible}
						\item Désignation d'une cible \myref*{fig-designation-CibleDesignee};\label{enu-designation-DesignationDUneCible}
						\item Acceptation d'une cible \myref*{fig-designation-CibleAcceptee};\label{enu-designation-AcceptationDUneCible}
						\item Sélection d'une cible \myref*{fig-designation-CibleSelectionnee}.\label{enu-designation-SelectionDUneCible}
					\end{enumerate}

					\begin{myfigure}
						\begin{mysubfigure}
							\myimage[width=0.49\textwidth]{designation-normal}
							\mysubcaption[fig-designation-RechercheDUneCible]{Recherche d'une cible}
						\end{mysubfigure}
						\begin{mysubfigure}
							\myimage[width=0.49\textwidth]{designation-called}
							\mysubcaption[fig-designation-CibleDesignee]{Cible désignée}
						\end{mysubfigure}
						\begin{mysubfigure}
							\myimage[width=0.49\textwidth]{designation-accepted}
							\mysubcaption[fig-designation-CibleAcceptee]{Cible acceptée}
						\end{mysubfigure}
						\begin{mysubfigure}
							\myimage[width=0.49\textwidth]{designation-selected}
							\mysubcaption[fig-designation-CibleSelectionnee]{Cible sélectionnée}
						\end{mysubfigure}
						\mycaption[fig-designation-LesQuatreEtapesDeLaDesignation]{Les quatre étapes de la désignation}
					\end{myfigure}

					L'\myref{enu-designation-RechercheDUneCible} consiste pour un utilisateur~\myuser{A} à rechercher une cible à désigner.
					La recherche de cette cible répondra au besoin d'obtenir de l'aide d'un utilisateur \myuser{B}.

					L'\myref{enu-designation-DesignationDUneCible} consiste pour l'utilisateur~\myuser{A} à désigner la cible identifiée.
					La cible est alors mise en surbrillance de façon à être vue des autres utilisateurs.

					L'\myref{enu-designation-AcceptationDUneCible} fait intervenir l'utilisateur~\myuser{B}.
					L'utilisateur~\myuser{B} peut accepter ou non cette désignation.
					S'il accepte la désignation, la cible est alors colorée de la couleur du curseur de l'utilisateur~\myuser{B} qui a accepté.
					Tant qu'elle n'est pas acceptée, la \myglos{glo-Residu} reste en surbrillance jusqu'à ce que la requête soit accepté ou modifiée.

					L'\myref{enu-designation-SelectionDUneCible} est la dernière étape.
					L'utilisateur~\myuser{B} ayant accepté doit maintenant sélectionner la cible pour achever le processus de désignation.
					Tant que l'utilisateur~\myuser{B} n'a pas sélectionné le \myglos{glo-Residu} ciblé, le processus ne peut pas être considéré comme terminé et l'effet de surbrillance reste actif.
					
					Parallèlement à ces quatre étapes, des aides haptiques ont été ajoutées pour la dernière expérimentation \myref*{cha-TravailCollaboratifAssisteParHaptique}.
					Pour l'\myref{enu-designation-DesignationDUneCible}, des vibrations sont générées sur tous les utilisateurs concernés par la désignation.
					De plus, dès l'instant qu'un utilisateur a accepté la désignation \myref*{enu-designation-AcceptationDUneCible}, il est guidé de façon haptique vers la cible.
					La vibration chez tous les autres utilisateurs est arrêtée.
				\end{mysubsection}
			\end{mysection}
		\end{mychapter}
	\end{mypart}
	\begin{mypart}[prt-EtudeDuTravailCollaboratif]{Étude du travail collaboratif}
		\begin{mychapter}[cha-RechercheCollaborativeDeResidusEnSimulationMoleculaire]{Recherche collaborative de résidus en simulation moléculaire}
			\begin{mysection}[sec-exp1-Introduction]{Introduction}
				La majeure partie de la thèse est l'étude du travail collaboratif en environnement virtuel complexe.
				L'environnement virtuel complexe choisi est la simulation moléculaire.
				Nous nous intéresserons à la déformation moléculaire qui se découpe en quatre tâches élémentaires \myref*{fig-ProcessusDeDeformationMoleculaireEnQuatreEtapes}.
				Nous nous proposons d'étudier ces quatre tâches élémentaires indépendamment les unes des autres.
				La première de ces expérimentations s'intéresse particulièrement au deux premières tâches élémentaires : la \myemph{sélection} et la \myemph{recherche}.
				La seconde et la troisième expérimentation permettront d'étudier plus particulièrement les tâches élémentaires de \myemph{manipulation} et d'\myemph{évaluation}.

				La thèse portant sur le travail collaboratif, cette première étude met en collaboration deux individus sur la tâche de recherche.
				La seconde étude utilisera des \myglos*{glo-Binome} dans une tâche de manipulation.
				Finalement, la collaboration de groupe sera abordée au niveau de la troisième et dernière étude sur le travail collaboratif.

				L'expérimentation est présentée en plusieurs parties.
				Nous commencerons par présenter les objectifs et les hypothèses de cette première expérimentation dans la \myref{sec-exp1-Presentation}.
				Le dispositif expérimental est présenté dans la \myref{sec-exp1-DispositifExperimentalEtMateriel}.
				La \myref{sec-exp1-Methode} expose la tâche réalisée ainsi que les différentes variables de l'expérimentation.
				Enfin, les résultats sont analysés dans la \myref{sec-exp1-Resultats} puis une synthèse de cette expérimentation est exposée dans la \myref{sec-exp1-Synthese}.
			\end{mysection}
			\begin{mysection}[sec-exp1-Presentation]{Présentation}
				\begin{mysubsection}[sse-exp1-Objectifs]{Objectifs}
					Dans cette première expérimentation, nous proposons d'étudier les deux premières des quatre tâches élémentaires \myref*{sse-LAmarrageMoleculaire}: la \myemph{recherche} et la \myemph{sélection}.
					Ces tâches sont cruciales car elles ont un impact important sur les tâches suivantes que sont la \myemph{déformation} et la \myemph{manipulation}.
					Les difficultés liées à la complexité de l'environnement virtuel moléculaire seront étudiées à travers cette étude.

					La recherche en environnement virtuel est une tâche élémentaire relativement peu explorée en biologie moléculaire.
					Cependant, on trouve de nombreux travaux concernant la \og recherche de chemin \fg ou \myemph{wayfinding} en anglais.
					\mycite[author]{Darken-1996a} consacre une thèse à la recherche de chemin en environnement virtuel en adaptant des solutions du monde réel.
					Plus récemment, \mycite[author]{Menelas-2010} s'est intéressé au rendu multi-modal pour la perception de la mécanique des fluides.
					Une des problématiques est la localisation de structures mécaniques particulières (vortex) dans le fluide à l'aide de retour audio-haptique.

					La recherche étant une tâche qui peut s'avérer complexe devant la quantité d'informations toujours grandissante, la collaboration s'est imposé comme une des solutions de ces dernières années.
					La recherche collaborative est un champ de recherche assez largement étudié, notamment en ce qui concerne les moteurs de recherche collaboratifs.
					\mycite[author]{Pickens-2007} explorent la recherche collaborative indirecte en utilisant les données de certains utilisateurs pour améliorer les recherches d'autres utilisateurs.
					La thèse de \mycite[author]{Foley-2008} s'intéresse plus précisément à l'interaction synchrone des utilisateurs pour l'amélioration des résultats de recherche.
					Certaines études proposent même représenter les informations de l'internet dans un environnement virtuel en \myThreeD comme \mycite[author]{Benford-1999} par exemple.

					Cette première expérimentation a pour objectif d'explorer la recherche de \myglos*{glo-Residu} durant une simulation moléculaire.
					La recherche s'effectuera dans un \myacro{acr-EVC}.
					L'expérimentation a pour objectif principal de comparer les performances d'un \myglos{glo-Monome} et d'un \myglos{glo-Binome} sur cette tâche de recherche.
					Les performances représentent à la fois le temps total pour réaliser la tâche mais aussi les ressources mises en place pour accéder à ce résultat.
					Un \myglos{glo-Binome} sera-t-il plus performant qu'un \myglos{glo-Monome} ?

					De plus, les méthodes et les stratégies de travail seront étudiées.
					C'est principalement l'évolution de ces stratégies au sein des \myglos*{glo-Binome} qui focalisera notre attention.
					Le travail en \myglos{glo-Binome} permet de mettre en avant différentes stratégies de travail discriminées en fonction de la communication, des espaces de travail, de la répartition des tâches, \myetc

					Enfin, il est nécessaire de valider la plate-forme de manipulation proposée.
					Pour cela, l'évaluation sera confiée aux sujets.
					L'objectif est de vérifier l'utilisabilité de la plate-forme afin d'identifier les points faibles.
				\end{mysubsection}
				\begin{mysubsection}[sse-exp1-Hypotheses]{Hypothèses}
					Nous émettons plusieurs hypothèses concernant cette première expérimentation.
					Les hypothèses concernent les performances des \myglos*{glo-Binome} ainsi que leurs stratégies de travail.
					Deuxièmement, une évaluation de la plate-forme est nécessaire.
					Des hypothèses sont formulées pour noter l'utilisabilité de la plate-forme ainsi que la sensation de collaboration des utilisateurs.
					\begin{myparagraph}[par-exp1-AmeliorationDesPerformancesEnBinome]{\myhypothesis{1} Amélioration des performances en \myglosnl{glo-Binome}}
						Nous émettons l'hypothèse que les performances des \myglos*{glo-Binome} seront meilleures que les performances des \myglos*{glo-Monome}.
						Les performances seront évaluées en terme de temps de réalisation de la tâche mais aussi en terme de ressources utilisées comme le nombre de sélections.
					\end{myparagraph}
					\begin{myparagraph}[par-exp1-StrategiesVariablesEnFonctionDesBinomes]{\myhypothesis{2} Stratégies variables en fonction des \myglosnl*{glo-Binome}}
						Nous émettons l'hypothèse que les \myglos*{glo-Binome} adopteront des stratégies de collaboration différentes en fonction des affinités des sujets et de leurs espaces de travail respectifs.
						L'identification des différentes stratégies permettra de les évaluer et de trouver la plus performante.
					\end{myparagraph}
					\begin{myparagraph}[par-exp1-LesSujetsPreferentLeTravailEnBinome]{\myhypothesis{3} Les sujets préfèrent le travail en \myglosnl{glo-Binome}}
						Notre troisième hypothèse est de nature qualitative et suppose que les utilisateurs auront une préférence pour le travail en \myglos{glo-Binome} comparé au travail en \myglos{glo-Monome}.
						Le travail en \myglos{glo-Binome} créé une collaboration sociale qui est préférée en général.
					\end{myparagraph}
					\begin{myparagraph}[par-exp1-LaPlateFormeEstApprecieeDesUtilisateurs]{\myhypothesis{4} La plate-forme est appréciée des utilisateurs}
						Notre dernière hypothèse concerne la validation de notre plate-forme en terme d'utilisabilité (intuitivité, ergonomie, \myetc).
						Elle est nécessaire pour la poursuite des études de cette thèse.
					\end{myparagraph}
				\end{mysubsection}
			\end{mysection}
			\begin{mysection}[sec-exp1-DispositifExperimentalEtMateriel]{Dispositif expérimental et matériel}
				Dans cette section est exposé l'ensemble du dispositif expérimental utilisé lors de l'expérience.
				L'expérience est basée sur l'\myacro{acr-EVC} présenté dans le \myref{cha-ShaddockSystemeCollaboratifPourLaManipulationDeMolecules}.
				Nous commencerons par présenter le matériel nécessaire dans la \myref{sse-exp1-DispositifTechnique}.
				Ensuite, nous détaillerons la disposition de chaque élément dans le dispositif expérimental dans la \myref{sse-exp1-DispositionDesElements}.
				Enfin, nous terminerons par les visualisations proposées et les interactions possibles des sujets sur le dispositif expérimental dans la \myref{sse-exp1-VisualisationEtInteraction}.
				\begin{mysubsection}[sse-exp1-DispositifTechnique]{Dispositif technique}
					La réalisation de cette expérimentation nécessite l'ensemble des matériels suivants :
					\begin{itemize}
						\item \mynum{1}~ordinateur quatre cœurs \myIntelCore avec \myRAM[Go]{4};
						\item \mynum{2}~ordinateurs de faible puissance;
						\item \mynum{3}~interfaces haptiques \myOmni;
						\item \mynum{1}~vidéoprojecteur \myCasioXJ;
						\item \mynum{1}~grand écran de vidéoprojection;
						\item \mynum{1}~écran \myLCD \mynum[pouces]{17};
						\item \mynum{1}~microphone de bureau.
					\end{itemize}
				\end{mysubsection}
				\begin{mysubsection}[sse-exp1-DispositionDesElements]{Disposition des éléments}
					Durant la phase expérimentale, les sujets sont placés en face de l'écran de vidéoprojection avec le vidéoprojecteur derrière eux.
					La visualisation du grand écran est accessible à tous les sujets : c'est une vue \myemph{partagée}.
					Dans le cas d'un \myglos{glo-Binome}, les sujets sont placés à côté l'un de l'autre.
					Il n'y a aucun obstacle entre eux afin de ne pas gêner les éventuelles communications (verbales ou gestuelles).
					De plus, les sujets auront face à eux un écran \myLCD.

					Face aux sujets se trouvent trois interfaces haptiques.
					L'interface du milieu est l'outil de manipulation \mytool{grab}.
					Les deux interfaces sur les côtés sont des outils de déformation \mytool{tug} \myref*{par-simulation-LesOutilsDeManipulation}.

					L'expérimentateur est placé derrière les sujets afin de ne pas gêner le champ visuel.
					Il dispose d'un écran permettant de lancer et de stopper les scénarios.
					La surveillance du bon déroulement de l'expérimentation est effectuée depuis ce poste.

					Finalement, le microphone est placé sur la table face aux sujets afin d'enregistrer tous les échanges verbaux.
					Les enregistrements sont effectués à l'aide du logiciel \myAudacity.

					La \myref{fig-exp1-SchemaDuDispositifExperimental} est un schéma récapitulatif de la disposition des tous les éléments dans la salle d'expérimentation.
					La \myref{fig-exp1-PhotographieDuDispositifExperimental} est une photographie de la salle d'expérimentation.

					\begin{myfigure}
						\myimage{exp1-schema}
						\mycaption[fig-exp1-SchemaDuDispositifExperimental]{Schéma du dispositif expérimental}
					\end{myfigure}
					\begin{myfigure}
						\myimage{exp1-photo}
						\mycaption[fig-exp1-PhotographieDuDispositifExperimental]{Photographie du dispositif expérimental}
					\end{myfigure}
				\end{mysubsection}
				\begin{mysubsection}[sse-exp1-VisualisationEtInteraction]{Visualisation et interactions}
					La vue complète de la molécule est projetée sur le grand écran.
					Les rendus graphiques utilisés sur \myacro{acr-VMD} pour afficher la molécule sont les rendus suivants \myref*{par-simulation-LesRendusGraphiques} :
					\begin{itemize}
						\item un rendu \myCPK avec des atomes de taille assez petite afin de pouvoir apprécier l'ensemble des atomes de la molécule;
						\item un second rendu \myCPK pour agrandir tous les atomes sauf les atomes d'hydrogène qui sont peu informatifs;
						\item un rendu \myNewRibbon pour apprécier la structure globale de la molécule.
					\end{itemize}

					L'écran \myLCD est utilisé pour afficher les \myglos*{glo-Residu} à chercher dans la molécule \myref*{sse-exp1-Tache}.
					Un simple rendu \myCPK est utilisé comme on peut le voir sur la \myref{tab-exp1-ListeDesResidusRecherches}.

					Parmi les trois interfaces haptiques \myOmni, l'outil \mytool{grab}, placé au centre est accessible depuis chacune des deux chaises à disposition.
					Dans le cas d'un \myglos{glo-Binome}, un seul des deux sujets utilisera l'outil durant tout la durée de l'expérimentation.
					Cependant, le choix du sujet qui utilisera cet outil est déterminé par négociation au sein du \myglos{glo-Binome}.

					Concernant les deux outils \mytool{tug}, ils sont répartis entre les deux sujets dans le cas d'un \myglos{glo-Binome}.
					Cependant, seulement un outil \mytool{tug} est laissé à disposition dans le cas d'un \myglos{glo-Monome}.
					Les sujets peuvent se placer sur la chaise (droite ou gauche) qu'ils souhaitent et ainsi se mettre dans les meilleures conditions concernant leur main dominante.
				\end{mysubsection}
			\end{mysection}
			\begin{mysection}[sec-exp1-Methode]{Méthode}
				\begin{mysubsection}[sse-exp1-Sujets]{Sujets}
					\mysummary{exp1-subjects.tex} avec une distribution d'âge de \mysummary{exp1-age.tex} ont participé à cette expérimentation.
					Ils ont tous été recrutés au sein du \myacro{acr-LIMSI} et sont chercheurs, assistants de recherche, étudiants en thèse ou stagiaires dans les domaines suivants~:
					\begin{itemize}
						\item linguistique et traitement automatique de la parole;
						\item réalité virtuelle et système immersifs;
						\item audio-acoustique.
					\end{itemize}

					Tous les sujets sont francophones.
					Aucun participant n'a de déficience visuelle (ou corrigée le cas échéant), de déficience audio ou de déficience moteur du haut du corps.
					Les sujets ne sont pas rémunérés pour l'expérimentation.

					Chaque participant est complètement naïf concernant les détails de l'expérimentation.
					Une explication détaillée de la procédure expérimentale leur est donnée au commencement de l'expérimentation.
					Cependant, l'objectif de l'expérimentation n'est pas révélé.
				\end{mysubsection}
				\begin{mysubsection}[sse-exp1-Variables]{Variables}
					\begin{mysubsubsection}[sss-exp1-VariablesIndependantes]{Variables indépendantes}
						\begin{myparagraph}[par-exp1-NombreDeSujets]{\myvari{1} Nombre de sujets}
							La première \myglos{glo-VariableIndependante} est une \myglos{glo-VariableIntraSujets}.
							\myvari{1} possède deux valeurs possibles: \og un sujet \fg (\mycf \myemph{\myglos{glo-Monome}}) ou \og deux sujets \fg (\mycf \myemph{\myglos{glo-Binome}}).
							\mynum{24}~\myglos*{glo-Monome} et \mynum{12}~\myglos*{glo-Binome} ont été testés ce qui fait deux fois plus de \myglos*{glo-Monome} que de \myglos*{glo-Binome}.
						\end{myparagraph}
						\begin{myparagraph}[par-exp1-ResiduRecherche]{\myvari{2} \myGlosnl{glo-Residu} recherché}
							La seconde \myglos{glo-VariableIndependante} est une \myglos{glo-VariableIntraSujets}.
							\myvari{2} concerne les \myglos*{glo-Residu} recherchés qui sont au nombre de \mynum{10} répartis à part égale dans deux molécules \myref*{tab-exp1-ListeDesResidusRecherches}.
							Différents niveaux de complexité caractérisent chaque \myglos{glo-Residu} \myref*{tab-exp1-ParametresDeComplexiteDesResidus}.
						\end{myparagraph}
					\end{mysubsubsection}
					\begin{mysubsubsection}[sss-exp1-VariablesDependantes]{Variables dépendantes}
						\begin{myparagraph}[par-exp1-LeTempsDeRealisation]{\myvard{1} Le temps de réalisation}
							Ce temps est le temps total pour réaliser la tâche demandée, c'est-à-dire trouver le \myglos{glo-Residu} et l'extraire de la molécule.
							Il n'y a pas de limite de temps pour réaliser la tâche.
							Ce temps est divisé en deux phases bien distinctes :
							\begin{description}
								\item[La recherche] C'est la phase pendant laquelle les sujets cherchent le \myglos{glo-Residu}.
									Cette recherche peut être visuelle en orientant et en déplaçant la molécule.
									Elle peut aussi amener les sujets à déformer la molécule afin d'explorer les \myglos{glo-Residu} inaccessibles du centre de la molécule.
								\item[La sélection] La phase de sélection débute dès l'instant où un des deux sujets a identifié visuellement le \myglos{glo-Residu}.
									Elle est constituée d'une phase de sélection puis d'une phase d'extraction hors de la molécule.
							\end{description}
						\end{myparagraph}
						\begin{myparagraph}[par-exp1-LaDistanceEntreLesEspacesDeTravail]{\myvard{2} La distance entre les espaces de travail}
							Cette mesure est la distance moyenne entre les deux \myglos*{glo-EffecteurTerminal} correspondant aux outils \mytool{tug}.
							Elle est mesurée dans le monde réel mais peut être convertie dans l'environnement virtuel (à l'échelle de la molécule).
							L'ordre de grandeur de cette mesure est le centimètre.
						\end{myparagraph}
						\begin{myparagraph}[par-exp1-LesCommunicationsVerbales]{\myvard{3} Les communications verbales}
							L'enregistrement des communications verbales permet de mesurer la quantité de temps de parole de chaque sujets pour chaque étape de l'expérimentation.
							Ces mesures différencie la phase de recherche et la phase de sélection (voir \myvard{1}) comme indiqué plus précisément sur la \myref{fig-exp1-EtapesDeLaCommunicationVerbalePourLaRechercheDUnResidu}.

							\begin{myfigure}
								\psset{unit=0.1\textwidth} % Fill entirely the page width
								\begin{myps}(0,-1.75)(10,1.5)
									\psset{linewidth=1pt,linecolor=black}%
									\psset{fillstyle=solid}%
									\psframe[fillcolor=mylightblue](0,-0.5)(6,0.5)%
									\pspolygon[fillcolor=mylightred](6,-0.5)(6,0.5)(9,0.5)(10,0)(9,-0.5)%
									\uput{16pt}[180](10,0){\LARGE\sl\textcolor{white!33}{temps}}
									\psbrace[ref=lC,rot=-90,nodesepA=-3,nodesepB=-0.25](6,0.5)(0,0.5){%
										\parbox{6\psxunit}{%
											\centering\textcolor{myblue}{Temps de recherche}%
										}%
									}%
									\psbrace[ref=lC,rot=-90,nodesepA=-2,nodesepB=-0.25](10,0.5)(6,0.5){%
										\parbox{4\psxunit}{%
											\centering\textcolor{myred}{Temps de sélection}%
										}%
									}%
									\psframe[fillcolor=myblue](1,-0.5)(1.5,0.5)
									\psframe[fillcolor=myblue](3,-0.5)(4.5,0.5)
									\psframe[fillcolor=myblue](4.8,-0.5)(5,0.5)
									\psframe[fillcolor=myred](6.5,-0.5)(7.5,0.5)
									\psframe[fillcolor=myred](8,-0.5)(8.25,0.5)
									\pnode(1.25,-0.5){verbal1}
									\pnode(3.75,-0.5){verbal2}
									\pnode(4.9,-0.5){verbal3}
									\pnode(7,-0.5){verbal4}
									\pnode(8.125,-0.5){verbal5}
									\rput(5,-1.5){%
										\Rnode{verbal}{%
											\psframebox[linestyle=none]{\centering Communication verbale}%
										}%
									}%
									\psset{linearc=0.1,angleA=-90}
									\ncdiagg{<-}{verbal1}{verbal}
									\ncdiagg{<-}{verbal2}{verbal}
									\ncdiagg{<-}{verbal3}{verbal}
									\ncdiagg{<-}{verbal4}{verbal}
									\ncdiagg{<-}{verbal5}{verbal}
								\end{myps}
								\mycaption[fig-exp1-EtapesDeLaCommunicationVerbalePourLaRechercheDUnResidu]{Étapes de la communication verbale pour la recherche d'un \myglosnl{glo-Residu}}
							\end{myfigure}
						\end{myparagraph}
						\begin{myparagraph}[par-exp1-LAffiniteEntreLesSujets]{\myvard{4} L'affinité entre les sujets}
							Le degré d'affinité -- concernant uniquement les \myglos*{glo-Binome} -- est compris entre \mynum{1} et \mynum{5} selon les critères suivants :
							\begin{enumerate}
								\item Les sujets ne se connaissent pas;
								\item Les sujets travaillent dans la même entreprise, le même laboratoire;
								\item Les sujets travaillent dans la même équipe, sur les mêmes projets;
								\item Les sujets travaillent ensemble, sont dans le même bureau;
								\item Les sujets sont amis proches.
							\end{enumerate}
						\end{myparagraph}
						\begin{myparagraph}[par-exp1-LaForceMoyenneAppliqueeParLesSujets]{\myvard{5} La force moyenne appliquée par les sujets}
							Le force appliquée par chaque sujet sur les atomes durant la simulation est mesurée.
							Une valeur moyenne de cette force est calculée pour être analysée.
						\end{myparagraph}
						\begin{myparagraph}[par-exp1-ReponsesQualitatives]{\myvard{6} Réponses qualitatives}
							Un questionnaire est proposé à tous les sujets.
							Il est constitué de trois ou quatre parties respectivement destinés aux \myglos*{glo-Monome} et \myglos*{glo-Binome}.
							Le questionnaire fourni aux sujets est disponible dans la \myref{sec-Questionnaires-PremiereExperimentation}.
						\end{myparagraph}
					\end{mysubsubsection}
				\end{mysubsection}
				\begin{mysubsection}[sse-exp1-Tache]{Tâche}
					La tâche proposée consiste à trouver puis à extraire des \myglos*{glo-Residu} d'une molécule.
					Les \myglos*{glo-Residu} sont des groupes d'atomes.
					Tous les \myglos*{glo-Residu} à rechercher sont affichés dans la \myref{tab-exp1-ListeDesResidusRecherches}.
					Trois molécules sont proposées dans le cadre de cette expérimentation :
					\begin{description}
						\item[\myTRPZIPPER]
							La molécule \myTRPZIPPER \mycite{Christen-2009} a pour identifiant \myPDB \myPDBlink{http://www.rcsb.org/pdb/explore/explore.do?structureId=2KFL}{2KFL} sur la \myPDBbase\footnote{\url{http://www.pdb.org/}}.
							Cette molécule contient \mynum{218}~atomes dont \mynum{12}~\myglos*{glo-Residu}.
							Elle est peu complexe et est seulement être utilisée pour un entraînement et un apprentissage des outils de manipulation.
						\item[\myTRPCAGE]
							La molécule nommée \myTRPCAGE \mycite{Neidigh-2002} a pour identifiant \myPDB \myPDBlink{http://www.rcsb.org/pdb/explore/explore.do?structureId=1L2Y}{1L2Y} sur la \myPDBbase\footnotemark[\value{footnote}].
							Cette molécule contient \mynum{304}~atomes dont \mynum{20}~\myglos*{glo-Residu}.
							C'est une des deux molécules proposées pour la tâche de recherche et d'extraction de \mynum{5}~\myglos*{glo-Residu} \myref*{tab-exp1-ListeDesResidusRecherches-ResidusSurLaMoleculeTRPCAGE}.
						\item[\myPrion]
							La molécule nommée \myPrion \mycite{Christen-2009} avec l'identifiant \myPDB \myPDBlink{http://www.rcsb.org/pdb/explore/explore.do?structureId=2KFL}{2KFL} sur la \myPDBbase\footnotemark[\value{footnote}].
							Cette molécule contient \mynum{1779}~atomes dont \mynum{112}~\myglos*{glo-Residu}.
							C'est une des deux molécules proposées pour la tâche de recherche et d'extraction de \mynum{5}~\myglos*{glo-Residu} \myref*{tab-exp1-ListeDesResidusRecherches-ResidusSurLaMoleculePrion}.
					\end{description}

					\begin{mytable}
						\mycaption[tab-exp1-ListeDesResidusRecherches]{Liste des \myglosnl*{glo-Residu} recherchés}
						\setlength{\myheight}{10ex}
						\newcommand{\mypatternpicture}[1]{\myimage[width=\myheight]{exp1-#1}}
						\begin{mysubtable}
							\mysubcaption[tab-exp1-ListeDesResidusRecherches-ResidusSurLaMoleculeTRPCAGE]{Residus sur la molécule \myTRPCAGE}
							\begin{mytabular}[0.49\textwidth]{^C-C}
								\mytoprule
								\myrowstyle{\bfseries}
								\myGlosnl{glo-Residu} & Image \\
								\mymiddlerule
								\myresidue{1} & \mypatternpicture{pattern1} \\
								\myresidue{2} & \mypatternpicture{pattern2} \\
								\myresidue{3} & \mypatternpicture{pattern3} \\
								\myresidue{4} & \mypatternpicture{pattern4} \\
								\myresidue{5} & \mypatternpicture{pattern5} \\
								\mybottomrule
							\end{mytabular}
						\end{mysubtable}
						\begin{mysubtable}
							\mysubcaption[tab-exp1-ListeDesResidusRecherches-ResidusSurLaMoleculePrion]{Residus sur la molécule \myPrion}
							\begin{mytabular}[0.49\textwidth]{^C-C}
								\mytoprule
								\myrowstyle{\bfseries}
								\myGlosnl{glo-Residu} & Image \\
								\mymiddlerule
								\myresidue{6}  & \mypatternpicture{pattern6}  \\
								\myresidue{7}  & \mypatternpicture{pattern7}  \\
								\myresidue{8}  & \mypatternpicture{pattern8}  \\
								\myresidue{9}  & \mypatternpicture{pattern9}  \\
								\myresidue{10} & \mypatternpicture{pattern10} \\
								\mybottomrule
							\end{mytabular}
						\end{mysubtable}
					\end{mytable}

					La \myref{fig-exp1-RepartitionDesResidusSurLesMolecules} montre la répartition des \myglos*{glo-Residu} sur les deux molécules.
					Chaque \myglos{glo-Residu} possède ses propres spécificités (position, couleurs\myetc).
					Les critères de complexité, résumés pour chaque \myglos{glo-Residu} dans la \myref{tab-exp1-ParametresDeComplexiteDesResidus}, sont les suivants :
					\begin{description}
						\item[Nombre de \myglosnl*{glo-Residu}] Le nombre total de \myglos*{glo-Residu} présents dans la molécule.
							Un nombre important des \myglos*{glo-Residu} surcharge visuellement l'environnement virtuel et augmente le nombre de cibles potentielles.
						\item[Position] Le \myglos{glo-Residu} peut se trouver soit à la périphérie de la molécule (position \myemph{externe}) ou au centre de la molécule (position \myemph{interne}).
							Un \myglos{glo-Residu} en position externe ne nécessite pas de déformer la molécule pour le trouver et l'atteindre contrairement à un \myglos{glo-Residu} en position interne qui sera plus complexe d'accès.
						\item[Forme] La forme du \myglos{glo-Residu} est un motif graphique plus ou moins complexe à identifier.
							On distingue trois formes différentes :
							\begin{description}
								\item[Chaîne] Une chaîne d'atomes (la plupart du temps carbonés) avec des atomes d'hydrogène de chaque côté.
								\item[Cycle] Une chaîne fermée d'atomes de carbone ou d'azote.
								\item[Étoile] Séries de chaînes d'atomes toutes reliées sur un atome central (un atome de carbone pour la plupart du temps).
							\end{description}
						\item[Couleurs] Les atomes sont colorés en fonction de leur nature (rouge pour l'oxygène, blanc pour l'hydrogène, \myetc).
							Les atomes rares seront donc rapidement identifiés grâce à leur couleur singulière.
							Par contre, les atomes nombreux (comme les hydrogènes ou les carbones) seront plus difficiles à filtrer à cause de leur fréquence d'apparition.
						\item[Similarité] Certains \myglos*{glo-Residu} à chercher sont très similaires à d'autres \myglos*{glo-Residu} également présents sur la molécule.
							Les \myglos*{glo-Residu} similaires possèdent un atome de moins ou de plus par rapport au \myglos{glo-Residu} recherché.
							À cause de cette similarité, les sujets vont mobiliser une partie du temps à identifier des \myglos*{glo-Residu} incorrects.
					\end{description}

					\begin{mytable}
						\newcommand{\myatomincolor}[3]{\csname my#1\endcsname{}{}#2 en \myemph{#3}}
						\mycaption[tab-exp1-ParametresDeComplexiteDesResidus]{Paramètres de complexité des \myglosnl*{glo-Residu} -- \myatomincolor{carbon}{arbone}{cyan}, \myatomincolor{nytrogen}{zote}{bleu}, \myatomincolor{oxygen}{xygène}{rouge} et \myatomincolor{sulfur}{oufre}{jaune}}
						\begin{mytabular}{^C-C-C-C-C-C}
							\mytoprule
							\myrowstyle{\bfseries}
							\myGlosnl{glo-Residu} & Nombre de \myglosnl*{glo-Residu} & Position & Forme  & Couleurs                                   & Similarité \\
							\mymiddlerule[\heavyrulewidth]
							\myresidue{1}         & \mynum{20}                       & Interne  & Cycle  & \mynum{8}~\mycarbon, \mynum{1}~\mynytrogen & Non        \\
							\mymiddlerule
							\myresidue{2}         & \mynum{20}                       & Interne  & Étoile & \mynum{1}~\mycarbon, \mynum{3}~\mynytrogen & Non        \\
							\mymiddlerule
							\myresidue{3}         & \mynum{20}                       & Interne  & Cycle  & \mynum{6}~\mycarbon, \mynum{1}~\myoxygen   & Non        \\
							\mymiddlerule
							\myresidue{4}         & \mynum{20}                       & Externe  & Chaîne & \mynum{4}~\mycarbon                        & Non        \\
							\mymiddlerule
							\myresidue{5}         & \mynum{20}                       & Externe  & Chaîne & \mynum{4}~\mycarbon, \mynum{1}~\mynytrogen & Non        \\
							\mymiddlerule[\heavyrulewidth]
							\myresidue{6}         & \mynum{112}                      & Interne  & Chaîne & \mynum{2}~\mycarbon, \mynum{2}~\mysulfur   & Non        \\
							\mymiddlerule
							\myresidue{7}         & \mynum{112}                      & Externe  & Étoile & \mynum{1}~\mycarbon, \mynum{3}~\mynytrogen & Non        \\
							\mymiddlerule
							\myresidue{8}         & \mynum{112}                      & Externe  & Cycle  & \mynum{6}~\mycarbon, \mynum{1}~\myoxygen   & Non        \\
							\mymiddlerule
							\myresidue{9}         & \mynum{112}                      & Interne  & Chaîne & \mynum{4}~\mycarbon                        & Oui        \\
							\mymiddlerule
							\myresidue{10}        & \mynum{112}                      & Interne  & Chaîne & \mynum{4}~\mycarbon, \mynum{1}~\mynytrogen & Oui        \\
							\mybottomrule
						\end{mytabular}
					\end{mytable}
					\begin{myfigure}
						\newcommand{\schemafactor}{0.20}
						\newlength{\schemaunit}\setlength{\schemaunit}{\schemafactor\textwidth}
						\psset{unit=\schemaunit}
						\mycaption[fig-exp1-RepartitionDesResidusSurLesMolecules]{Répartition des \myglosnl*{glo-Residu} sur les molécules}
						\begin{myps}(-2.5,-3)(2.5,3)
							\rput(0,1.75){%
								\myimage[height=2\schemaunit,angle=90]{exp1-trp-cage}}
							\rput(0,2.9){%
								\textcolor{black!25}{\Huge\bfseries\myTRPCAGE}}
							\rput(0,-1.25){%
								\myimage[height=2\schemaunit,angle=90]{exp1-prion}}
							\rput(0,0){%
								\textcolor{black!25}{\Huge\bfseries\myPrion}}
							\rput(-1.5,2){%
								\myimage[height=\schemaunit]{exp1-pattern1}}
							\rput(1.5,2){%
								\myimage[width=\schemaunit]{exp1-pattern3}}
							\rput(1.5,-0){%
								\myimage[width=\schemaunit]{exp1-pattern2}}
							\rput(-1.5,-0){%
								\myimage[width=\schemaunit]{exp1-pattern4}}
							\rput(-1.5,-2){%
								\myimage[width=\schemaunit]{exp1-pattern5}}
							\rput(1.5,-2){%
								\myimage[height=\schemaunit]{exp1-pattern6}}

							\psset{framesize=1 1}
							\fnode(-1.5,2){P1}
							\uput[90](-1.5,2.5){\myresidue{1}}
							\fnode(1.5,2){P38}
							\uput[90](1.5,2.5){\myresidue{3} et \myresidue{8}}
							\fnode(1.5,-0){P27}
							\uput[90](1.5,0.5){\myresidue{2} et \myresidue{7}}
							\fnode(-1.5,-0){P49}
							\uput[90](-1.5,0.5){\myresidue{4} et \myresidue{9}}
							\fnode(-1.5,-2){P510}
							\uput[90](-1.5,-1.5){\myresidue{5} et \myresidue{10}}
							\fnode(1.5,-2){P6}
							\uput[90](1.5,-1.5){\myresidue{6}}

							\psset{linecolor=myred}
							\cnode(0.3,1.5){0.2}{TRPP1}
							\cnode(0.15,2){0.2}{TRPP38}
							\cnode(-0.1,1.25){0.2}{TRPP27}
							\cnode(-0.5,2.2){0.2}{TRPP49}
							\cnode(-0.65,1.25){0.2}{TRPP510}
							\ncline{-}{P1}{TRPP1}
							\ncline{-}{P38}{TRPP38}
							\ncline{-}{P27}{TRPP27}
							\ncline{-}{P49}{TRPP49}
							\ncline{-}{P510}{TRPP510}

							\psset{linecolor=myblue}
							\cnode(0.4,0.2){0.2}{PrionP38}
							\cnode(0.6,-2.8){0.2}{PrionP27}
							\cnode(0.2,-0.8){0.2}{PrionP49}
							\cnode(-0.7,-1.7){0.2}{PrionP510}
							\cnode(0.0,-1.4){0.2}{PrionP6}
							\ncline{-}{P38}{PrionP38}
							\ncline{-}{P27}{PrionP27}
							\ncline{-}{P49}{PrionP49}
							\ncline{-}{P510}{PrionP510}
							\ncline{-}{P6}{PrionP6}
						\end{myps}
					\end{myfigure}

					La tâche proposée nécessite deux étapes.
					Selon \mycite[author]{Bowman-1999}, on distingue tout d'abord l'étape de recherche de l'objectif.
					Pour explorer la molécule afin d'identifier l'objectif, les sujets disposent de l'outil \mytool{grab}.
					Lorsque l'objectif recherché est identifié, les sujets entrent dans une seconde étape de sélection.
					Pour effectuer ces étape de sélection, les sujets disposent de l'outil \mytool{tug}.
					Les outils \mytool{grab} et \mytool{tug} sont décrits dans la \myref{par-simulation-LesOutilsDeManipulation}.
				\end{mysubsection}
				\begin{mysubsection}[sse-exp1-Procedure]{Procédure}
					L'expérimentation débute par une étape d'apprentissage avec la molécule \myTRPZIPPER.
					L'apprentissage est destiné à familiariser les sujets avec la plate-forme, les outils de manipulation et la tâche à réaliser.
					Cette phase dure maximum \mynum[mn]{5}.
					L'expérimentateur est disponible pour répondre aux questions des sujets.

					Lorsque l'étape d'apprentissage est terminée, les sujets sont soumis successivement au \mynum{10}~\myglos*{glo-Residu}.
					Lorsqu'un \myglos{glo-Residu} est trouvé, sélectionné puis extrait, l'application est arrêtée.
					Un second \myglos{glo-Residu} est affiché, l'application est de nouveau démarrée et ainsi de suite pour les \myglos*{glo-Residu} à chercher.
					L'enregistrement audio est démarré à la fin de l'étape d'apprentissage.

					L'ensemble des \myglos*{glo-Residu} est proposé dans un ordre aléatoire afin d'éviter un biais lié à l'apprentissage de la plate-forme et de la tâche.
					Les sujets doivent trouver et extraire dix \myglos*{glo-Residu} en \myglos{glo-Monome} et dix \myglos{glo-Residu} en \myglos{glo-Binome}.
					Toujours pour éviter un biais lié à l'apprentissage, les sujets sont soumis aux tâches en \myglos{glo-Monome} et en \myglos{glo-Binome} de façon alternée selon les trois combinaisons suivantes :
					\begin{enumerate}
						\item Le \myglos{glo-Monome} \myuser{A}, puis le \myglos{glo-Monome} \myuser{B}, puis le \myglos{glo-Binome} \myuser{AB};
						\item Le \myglos{glo-Monome} \myuser{A}, puis le \myglos{glo-Binome} \myuser{AB}, puis le \myglos{glo-Monome} \myuser{B};
						\item Le \myglos{glo-Binome} \myuser{AB}, puis le \myglos{glo-Monome} \myuser{A}, puis le \myglos{glo-Monome} \myuser{B}.
					\end{enumerate}

					Lorsque les sujets ont réalisé toutes les tâches dans les deux configurations possibles (\myglos{glo-Monome} et \myglos{glo-Binome}), ils sont soumis au remplissage du questionnaire.
					Chaque sujet répond au questionnaire seul, sans communiquer avec son partenaire.

					Un résumé du protocole expérimental est exprimé dans la \myref{tab-exp1-SyntheseDeLaProcedureExperimentale}.

					\begin{mytable}
						\mycaption[tab-exp1-SyntheseDeLaProcedureExperimentale]{Synthèse de la procédure expérimentale}
						\newcommand{\mytitlecolumn}[2]{%
							\multirow{#1}*{%
								\begin{minipage}{6em}%
									\raggedleft #2%
								\end{minipage}%
							}
						}
						\newlength{\exponefirstcolumn}
						\newlength{\exponesecondcolumn}
						\setlength{\exponefirstcolumn}{7em}
						\setlength{\exponesecondcolumn}{\textwidth}
						\addtolength{\exponesecondcolumn}{-\exponefirstcolumn}
						\addtolength{\exponesecondcolumn}{-4\tabcolsep}
						\begin{mytabular}{>{\bfseries}p{\exponefirstcolumn}p{\exponesecondcolumn}}
							\mytoprule
							\mytitlecolumn{1}{Tâche}                  & Recherche et sélection de motifs                                             \\
							\mymiddlerule[\heavyrulewidth]
							\mytitlecolumn{3}{Hypothèses}             & \myhypothesis{1} Amélioration des performances en \myglosnl{glo-Binome}      \\
							                                          & \myhypothesis{2} Stratégies variables en fonction des \myglosnl*{glo-Binome} \\
							                                          & \myhypothesis{3} Les sujets préfèrent le travail en \myglosnl{glo-Binome}    \\
							\mymiddlerule
							\mytitlecolumn{2}{Variable indépendantes} & \myvari{1} Nombre de sujets                                                  \\
							                                          & \myvari{2} \myGlosnl{glo-Residu} à chercher                                  \\
							\mymiddlerule
							\mytitlecolumn{4}{Variable dépendantes}   & \myvard{1} Temps de réalisation                                              \\
							                                          & \myvard{2} Distance entre les espaces de travail                             \\
							                                          & \myvard{3} Communication verbales                                            \\
							                                          & \myvard{4} Affinités entre les sujets                                        \\
							                                          & \myvard{5} La force moyenne appliquée par le sujet                           \\
							                                          & \myvard{6} Réponses qualitatives                                             \\
							\mymiddlerule[\heavyrulewidth]
							\multicolumn{2}{c}{%
								\begin{tabular}{^C-C-C}
									\myrowstyle{\bfseries}
									Condition \mycondition{1}         & Condition \mycondition{2}         & Condition \mycondition{3}         \\
									\mymiddlerule
									Sujet~\myuser{A}                  & Sujet~\myuser{A}                  & \myGlosnl{glo-Binome}~\myuser{AB} \\
									\mynum{10}~\myglosnl*{glo-Residu} & \mynum{10}~\myglosnl*{glo-Residu} & \mynum{10}~\myglosnl*{glo-Residu} \\
									\mymiddlerule
									Sujet~\myuser{B}                  & \myGlosnl{glo-Binome}~\myuser{AB} & Sujet~\myuser{A}                  \\
									\mynum{10}~\myglosnl*{glo-Residu} & \mynum{10}~\myglosnl*{glo-Residu} & \mynum{10}~\myglosnl*{glo-Residu} \\
									\mymiddlerule
									\myGlosnl{glo-Binome}~\myuser{AB} & Sujet~\myuser{B}                  & Sujet~\myuser{B}                  \\
									\mynum{10}~\myglosnl*{glo-Residu} & \mynum{10}~\myglosnl*{glo-Residu} & \mynum{10}~\myglosnl*{glo-Residu} \\
								\end{tabular}
							} \\
							\mybottomrule
						\end{mytabular}
					\end{mytable}
				\end{mysubsection}
			\end{mysection}
			\begin{mysection}[sec-exp1-Resultats]{Résultats}
				Cette section présente et analyse l'ensemble des mesures expérimentales de cette première étude concernant la recherche et la sélection sur une tâche complexe de collaboration.
				Les données, confrontées à un test de \mycite[author]{Shapiro-1965}, ne sont pas distribuées selon une loi normale.
				Cependant, un test de \mycite[author]{Brown-1974} permet de confirmer l'\myglos{glo-Homoscedasticite}.
				L'analyse de la variance est alors pratiquée à l'aide d'un test de \mycite[author]{Friedman-1940} adapté pour les \myglos*{glo-VariableIntraSujets} non-paramètriques.
				\begin{mysubsection}[sse-exp1-AmeliorationDesPerformancesEnBinome]{Amélioration des performances en \myglosnl{glo-Binome}}
					\begin{myfigure}
						\psset{xunit=0.0889\textwidth,yunit=0.01cm}
						\begin{myps}(-1.25,-115)(10,425)
							\myaxes(0,10){\myglosnl*{glo-Residu}}(0,400)[100]{temps~(s)}
							\myboxplot{exp1-time-residue.csv}
						\end{myps}
						\mycaption[fig-exp1-TempsDeRealisationParResidu]{Temps de réalisation par \myglosnl{glo-Residu}}
					\end{myfigure}

					La \myref{fig-exp1-TempsDeRealisationParResidu} présente le temps de réalisation \myvard{1} pour l'identification et l'extraction de chaque \myglos{glo-Residu} \myvari{2}.
					L'analyse montre qu'il y a un effet significatif des \myglos*{glo-Residu} \myvari{2} sur le temps de réalisation \myvard{1} (\myanova{exp1-time-residue-anova.tex}).
					Un test post-hoc de \mycite[author]{Mann-1947} avec une correction de \mycite[author]{Holm-1979} permet de déterminer que les \myglos*{glo-Residu} \myresidue{6}, \myresidue{9} et \myresidue{10} obtiennent des temps de réalisation significativement plus longs que les autres \myglos*{glo-Residu}.

					\begin{myfigure}
						\psset{xunit=0.0889\textwidth,yunit=0.01cm}
						\begin{myps}(-1.25,-115)(10,460)
							\myaxes(0,10){\myglosnl*{glo-Residu}}(0,400)[100]{temps~(s)}
							\myboxplot{exp1-time-residue-group.csv}
							\mylegend{\myleg{\myGlosnl{glo-Monome}}{myblue}\myand\myleg{\myGlosnl{glo-Binome}}{myblue!70}}
						\end{myps}
						\mycaption[fig-exp1-TempsDeRealisationComparesMonomeOuBinomeParResidu]{Temps de réalisation comparés (\myglosnl{glo-Monome} ou \myglosnl{glo-Binome}) par \myglosnl{glo-Residu}}
					\end{myfigure}

					La \myref{fig-exp1-TempsDeRealisationComparesMonomeOuBinomeParResidu} présente les temps de réalisation \myvard{1} de chaque \myglos{glo-Residu} \myvari{2} en fonction du nombre de participants \myvari{1}.
					L'analyse ne montre pas d'effet significatif du nombre de participants \myvari{1} sur le temps de réalisation \myvard{1} (\myanova{exp1-time-residue-group-anova.tex}).
					Cependant, en se limitant au groupe des trois \myglos*{glo-Residu} \myresidue{6}, \myresidue{9} et \myresidue{10} identifiés précédemment comme significativement plus longs à trouver et extraire, on montre un effet significatif du nombre de participants \myvari{1} sur le temps de réalisation \myvard{1} (\myanova{exp1-time-residue-group-anova-restricted.tex}).

					\begin{myfigure}
						\psset{xunit=0.0889\textwidth,yunit=0.01cm}
						\begin{myps}(-1.25,-115)(10,460)
							\myaxes(0,10){\myglosnl*{glo-Residu}}(0,400)[100]{temps~(s)}
							\myboxplot{exp1-timeaudio-residue-searchselection.csv}
							\mylegend{\myleg{Recherche}{myblue}\myand\myleg{Sélection}{myblue!70}}
						\end{myps}
						\mycaption[fig-exp1-TempsDeRechercheEtDeSelectionComparesParResidu]{Temps de recherche et de sélection comparés par \myglosnl{glo-Residu}}
					\end{myfigure}

					La \myref{fig-exp1-TempsDeRechercheEtDeSelectionComparesParResidu} présente les temps de recherche et de sélection par \myglos{glo-Residu} \myvari{2}.
					L'analyse montre un effet significatif des \myglos*{glo-Residu} \myvari{2} sur les temps de recherche (\myanova{exp1-timeaudio-residue-searchselection-anova-search.tex}).
					Un test post-hoc de \mycite[author]{Mann-1947} avec une correction de \mycite[author]{Holm-1979} permet de déterminer que les \myglos*{glo-Residu} \myresidue{9} et \myresidue{10} obtiennent des temps de recherche significativement plus longs que les autres \myglos*{glo-Residu}.
					L'analyse montre également un effet significatif des \myglos*{glo-Residu} \myvari{2} sur les temps de sélection (\myanova{exp1-timeaudio-residue-searchselection-anova-selection.tex}).
					Un test post-hoc de \mycite[author]{Mann-1947} avec une correction de \mycite[author]{Holm-1979} permet de déterminer que le \myglos{glo-Residu} \myresidue{6} obtient un temps de sélection significativement plus long que les autres \myglos*{glo-Residu}.

					Les cinq \myglos*{glo-Residu} \myresidue{1}, \myresidue{2}, \myresidue{3}, \myresidue{4} et \myresidue{5} sont au sein de la molécule \myTRPCAGE qui en compte un nombre total relativement limité (\mynum{20}~\myglos*{glo-Residu}).
					Les sujets construisent rapidement une carte mentale de la molécule ce qui leur permet de d'identifier rapidement les \myglos*{glo-Residu} recherchés.
					De plus, les faibles contraintes physiques de la molécule (énergie totale du système peu élevée à cause du faible nombre d'atomes) la rende facile à déformer.
					Cela facilite la recherche des \myglos*{glo-Residu} qui sont dans une position interne à la molécule et qui nécessitent une déformation.
					Tous ces facteurs rendent les tâches de recherche et de sélection peu complexes sur la molécule \myTRPCAGE ce qui explique des temps de réalisation de la tâche très courts.

					Les cinq \myglos*{glo-Residu} \myresidue{6}, \myresidue{7}, \myresidue{8}, \myresidue{9} et \myresidue{10} sont au sein de la molécule \myPrion qui en compte un nombre total relativement important (\mynum{112}~\myglos*{glo-Residu}).
					La construction complète d'une carte mentale est très complexe à cause du nombre importants d'atomes qui sont continuellement en mouvement (dû à la simulation en temps-réel).
					Les sujets n'étant jamais confronté plus de deux fois à la même tâche (une fois en \myglos{glo-Monome} et une fois en \myglos{glo-Binome}), le phénomène d'apprentissage ne peut pas être effectué.
					En effet, les sujets ne se souviennent pas de la position d'un \myglos{glo-Residu} d'une confrontation à l'autre (contrairement à la molécule \myTRPCAGE pour certains cas).
					Les sujets adoptent une stratégie en plusieurs étapes en fonction de la caractéristique de la tâche et du \myglos{glo-Residu} à trouver.
					Tout d'abord, une recherche exploratoire permet d'identifier les \myglos*{glo-Residu} \myresidue{7} et \myresidue{8} qui se trouvent en position externe.
					Ensuite, lorsque cette première étape exploratoire ne permet pas d'identifier le \myglos{glo-Residu} recherché, les sujets déforment la molécule afin d'accéder aux \myglos*{glo-Residu} \myresidue{6}, \myresidue{9} et \myresidue{10} qui se trouvent en position interne.

					Le travail en \myglos{glo-Binome} comparé au travail en \myglos{glo-Monome} ne montre pas d'amélioration significative bien que la \mypvalue soit très proche du seuil de significativité.
					Cependant, un test post-hoc a permis de d'identifier les \myglos*{glo-Residu} \myresidue{6}, \myresidue{9} et \myresidue{10} comme ayant un temps de réalisation significativement plus long.
					Sur ce groupe de \myglos*{glo-Residu} plus complexes, les \myglos*{glo-Binome} obtiennent une amélioration significative des performances par rapport aux \myglos*{glo-Monome}.
					Ce résultat confirme notre hypothèse \myhypothesis{1} exclusivement sur des tâches de fortes complexité.

					Comme développé dans la procédure expérimentale, le temps de réalisation de la tâche peut être séparé en deux parties : le temps de recherche et le temps de sélection \myref*{fig-exp1-EtapesDeLaCommunicationVerbalePourLaRechercheDUnResidu}.
					Les \myglos*{glo-Residu} \myresidue{9} et \myresidue{10} se distinguent par un temps de recherche significativement plus long que les autres \myglos*{glo-Residu} (excepté \myresidue{6}).
					En effet, ces deux \myglos*{glo-Residu} sont en présence d'autres \myglos*{glo-Residu} similaires au sein de la même molécule \myref*{tab-exp1-ParametresDeComplexiteDesResidus}.
					Ces similarités ont pour effet de monopoliser l'attention des sujets ce qui provoque une hausse significative du temps de recherche du \myglos{glo-Residu} au sein de la molécule.

					De la même façon, le \myglos{glo-Residu} \myresidue{6} se distingue par un temps de sélection significativement plus long que les autres \myglos*{glo-Residu} (excepté \myresidue{9} et \myresidue{10}).
					Ce \myglos{glo-Residu} possède deux atomes de \myatom{Soufre} de couleur jaune.
					Cette particularité aisément identifiable malgré le nombre importants d'atomes de la molécules.
					Le temps de recherche est alors extrêmement court.
					Cependant, ce \myglos{glo-Residu} est positionné au centre de la molécule.
					L'accès au \myglos{glo-Residu} nécessite de \myemph{déplier} en grande partie la molécule afin de pouvoir le sélectionner et l'extraire.

					L'analyse du rapport entre les temps de recherche et de sélection met en évidence trois configurations en fonction des différents \myglos*{glo-Residu} :
					\begin{description}
						\item[Temps de recherche et de sélection égaux]
							Les sujets ont un temps similaire alloué à l'étape de recherche et de sélection.
							Les \myglos*{glo-Residu} concernés ne présentent pas de forte complexité (tous les \myglos*{glo-Residu} de la molécule \myTRPCAGE et les \myglos*{glo-Residu} \myresidue{7} et \myresidue{8} de la molécule \myPrion) et sur lesquels, le travail collaboratif n'améliore pas les performances.
						\item[Temps de recherche prédominant]
							Les sujets ont un temps important alloué à l'identification du \myglos{glo-Residu} recherché.
							Une fois identifié, le \myglos{glo-Residu} est facile à sélectionner puis à extraire.
							Les \myglos*{glo-Residu} \myresidue{9} et \myresidue{10} sont concernés.
							Dans cette configuration, le travail collaboratif améliore significativement les performances.
							En effet, l'étape de recherche est fortement parallélisable : l'espace de recherche est séparé entre les sujets (stratégie \myemph{diviser pour régner}).
						\item[Temps de sélection prédominant]
							Les sujets ont un temps important alloué à la sélection et à l'extraction du \myglos{glo-Residu} recherché.
							Le \myglos{glo-Residu} est rapidement identifié mais il est difficile d'y accéder directement.
							Une phase de déformation est nécessaire pour le sélectionner.
							Le \myglos{glo-Residu} \myresidue{6} est concerné.
							Dans cette configuration, le travail collaboratif améliore significativement les performances.
							En effet, l'étape de déformation peut bénéficier d'une action synchronisée entre plusieurs sujets : l'effort déployé est alors plus important ce qui permet de déformer la molécule plus rapidement.
					\end{description}
				\end{mysubsection}
				\begin{mysubsection}[sse-exp1-StrategiesDeTravail]{Stratégies de travail}
					Dans cette section, les données concernent exclusivement les \myglos*{glo-Binome}.
					Une numérotation des \myglos*{glo-Binome} a été effectuée afin de pouvoir comparer les mesures effectuées et ainsi, étudier les différentes stratégies.

					\begin{myfigure}
						\psset{xunit=0.074\textwidth,yunit=0.15cm}
						\begin{myps}(-1.5,-7)(12,22)
							\myaxes(0,12){groupes}(0,20)[4]{distance~(mm)}
							% Once header are readed, they are defined for other barplot
							% That's why barplots without headers are in first position
							\mybarplot[header=false,barstyle=third-barstyle]{exp1-diff-groups3.csv}
							\mybarplot[header=false,barstyle=second-barstyle]{exp1-diff-groups2.csv}
							\mybarplot[barstyle=first-barstyle]{exp1-diff-groups1.csv}
							\psset{linecolor=myred,linewidth=1pt,linestyle=solid}
							\psline(0,14)(12,14)
							\psline(0,8)(12,8)
							\psset{linewidth=0.1pt,linecolor=white,fillstyle=solid,fillcolor=myred}
							\uput[180](12,5){\pscharpath{\LARGE\bf\sffamily Champ proche}}
							\uput[180](12,11){\pscharpath{\LARGE\bf\sffamily Champ voisin}}
							\uput[180](12,17){\pscharpath{\LARGE\bf\sffamily Champ distant}}
						\end{myps}
						\mycaption[fig-exp1-DistanceMoyenneEntreLesSujetsPourChaqueBinomeSurLesResidusSixNeufEtDix]{Distance moyenne entre les sujets pour chaque \myglosnl{glo-Binome} sur les \myglosnl*{glo-Residu} \myresidue{6}, \myresidue{9} et \myresidue{10}}
					\end{myfigure}

					La \myref{fig-exp1-DistanceMoyenneEntreLesSujetsPourChaqueBinomeSurLesResidusSixNeufEtDix} présente la distance moyenne entre les espaces de travail \myvard{2} de chaque \myglos{glo-Binome}.
					Les \myglos*{glo-Binome} peuvent être classés en trois groupes : \myemph{espace distant}, \myemph{espace voisin} et \myemph{espace proche}.

					\begin{myfigure}
						\psset{xunit=0.074\textwidth,yunit=0.5cm}
						\begin{myps}(-1.5,-2)(12,5.5)
							\myaxes(0,12){groupes}(0,5)[1]{affinité~( \mynum{1}--\mynum{5})}
							\mybarplot{exp1-affinity-groups.csv}
						\end{myps}
						\mycaption[fig-exp1-AffiniteEntreLesSujetsPourChaqueBinome]{Affinité entre les sujets pour chaque \myglosnl{glo-Binome}}
					\end{myfigure}

					La \myref{fig-exp1-AffiniteEntreLesSujetsPourChaqueBinome} présente les affinités \myvard{4} de chaque \myglos{glo-Binome}.
					Les notes, comprises entre un et cinq, montre que les \myglos*{glo-Binome} choisis ont des affinités relativement variées.
					L'affinité entre les sujets du groupe \mygroup{1} est très basse contrairement aux groupes \mygroup{8} et \mygroup{12} pour lesquelles l'affinité est très élevée.

					\begin{myfigure}
						\psset{xunit=0.074\textwidth,yunit=0.01cm}
						\begin{myps}(-1.5,-110)(12,325)
							\myaxes(0,12){groupes}(0,300)[50]{temps~(s)}
							\mybarplot{exp1-time-groups.csv}
						\end{myps}
						\mycaption[fig-exp1-TempsDeRealisationEntreLesSujetsPourChaqueBinome]{Temps de réalisation entre les sujets pour chaque \myglosnl{glo-Binome}}
					\end{myfigure}

					La \myref{fig-exp1-TempsDeRealisationEntreLesSujetsPourChaqueBinome} présente les temps de réalisation \myvard{1} de chaque \myglos{glo-Binome}.
					Le temps de réalisation de \mygroup{1} est particulièrement important (plus d'une fois et demi les autres groupes les plus longs).
					À l'opposé, on note que \mygroup{2}, \mygroup{3} et \mygroup{4} obtiennent des temps de réalisation extrêmement bas.

					\begin{myfigure}
						\psset{xunit=0.074\textwidth,yunit=0.05cm}
						\begin{myps}(-1.5,-22)(12,75)
							\myaxes(0,12){groupes}(0,70)[10]{temps~(s)}
							\mybarplot{exp1-timeaudio-groups.csv}
						\end{myps}
						\mycaption[fig-exp1-TempsDeCommunicationVerbaleEntreLesSujetsPourChaqueBinome]{Temps de communication verbale entre les sujets pour chaque \myglosnl{glo-Binome}}
					\end{myfigure}

					La \myref{fig-exp1-TempsDeCommunicationVerbaleEntreLesSujetsPourChaqueBinome} présente les temps de communication verbale \myvard{3} de chaque \myglos{glo-Binome}.
					\mygroup{2}, \mygroup{3} et \mygroup{4} ont des temps de communication verbale inférieurs à \mynum[s]{20}.
					À l'opposé, \mygroup{1}, \mygroup{5} et \mygroup{11} ont des temps de communication verbale qui approche les \mynum[s]{60}.

					\begin{myfigure}
						\psset{xunit=0.074\textwidth,yunit=0.03cm}
						\begin{myps}(-1.5,-35)(12,120)
							\myaxes(0,12){groupes}(0,100)[25]{temps~(\%)}
							\mybarplot{exp1-timeaudio-groups-searchselection.csv}
							\mylegend{\myleg{Recherche}{myblue}\myand\myleg{Sélection}{myblue!70}}
						\end{myps}
						\mycaption[fig-exp1-PourcentageDeTempsDeCommunicationVerbalePendantLaRechercheEtLaSelectionDesSujetsPourChaqueBinome]{Pourcentage de temps de communication verbale pendant la recherche et la sélection des sujets pour chaque \myglosnl{glo-Binome}}
					\end{myfigure}

					La \myref{fig-exp1-PourcentageDeTempsDeCommunicationVerbalePendantLaRechercheEtLaSelectionDesSujetsPourChaqueBinome} présente les pourcentages de temps de communication verbale durant la phase de recherche et durant la phase de sélection de chaque \myglos{glo-Binome} par rapport au temps total de réalisation de la tâche.
					Le pourcentage représente le rapport du temps de communication verbale durant la phase recherche ou de sélection rapporté respectivement au temps total de la phase de recherche ou de sélection.
					Les \myglos*{glo-Binome} \mygroup{1} à \mygroup{4} ainsi que \mygroup{9} communiquent plus durant la phase de sélection.
					Les \myglos*{glo-Binome} \mygroup{5} à \mygroup{8} et \mygroup{10} à \mygroup{12} communiquent plus durant la phase de recherche.
					Notons également que \mygroup{1} communique assez peu par rapport aux autres \myglos*{glo-Binome}.

					\begin{myfigure}
						\psset{xunit=0.074\textwidth,yunit=1cm}
						\begin{myps}(-1.5,-1)(12,3.6)
							\myaxes(0,12){groupes}(0,3)[1]{force~(N)}
							\mybarplot{exp1-force-groups-meandiff.csv}
							\mylegend{\myleg{Force moyenne}{myblue}\myand\myleg{Différence de force}{myblue!70}}
						\end{myps}
						\mycaption[fig-exp1-ForceMoyenneEtDifferenceDeForceEntreLesSujetsPourChaqueBinome]{Force moyenne et différence de force entre les sujets pour chaque \myglosnl{glo-Binome}}
					\end{myfigure}

					La \myref{fig-exp1-ForceMoyenneEtDifferenceDeForceEntreLesSujetsPourChaqueBinome} représente la force moyenne appliquée par les sujets \myvard{5} et la différence de force entre les sujets.
					La différence de force est la différence entre les forces moyennes de chaque sujet.
					\mygroup{9} et \mygroup{11} apporte un effort moyen très important par rapport aux autres \myglos*{glo-Binome}.
					\mygroup{2}, \mygroup{3} et \mygroup{4} apporte un effort moyen important également tout en ayant une différence de force quasiment nulle entre les deux membres du \myglos{glo-Binome}.

					L'ensemble des résultats et analyses précédentes permet de différencier les \myglos*{glo-Binome} ce qui confirme notre hypothèse \myhypothesis{2}.
					Les \myglos*{glo-Binome} se différencient pas des stratégies de travail variables.
					Les sections suivantes caractérisent les différentes stratégies de travail en fonction de plusieurs paramètres (distance entre les espaces de travail, affinités, temps de réalisation de la tâche, communication verbale, forces moyennes appliquées).
					Trois stratégies sont décrites distinguées en fonction des distances entre les espaces de travail.
					\begin{description}
						\item[Interaction en champ proche] pour les distances inférieures à \mynum[mm]{8};
						\item[Interaction en champ voisin] pour les distances comprises entre \mynum[mm]{8} et \mynum[mm]{14};
						\item[Interaction en champ distant] pour les distances supérieures à \mynum[mm]{14}.
					\end{description}
					Les mesures de distances sont données dans le référentiel du monde réel.

					\begin{mysubsubsection}[sss-exp1-InteractionEnChampProche]{Interaction en champ proche}
						Les interactions en champs proches, inférieure à \mynum[mm]{8}, correspondent, dans l'environnement virtuel, à des distances inférieures à \mynum[\AA]{10} ce qui est environ l'envergure d'un \myglos{glo-Residu}\footnote{\og \AA \fg désigne l'\myangstrom qui est une unité de mesure telle que $\mynum[\AA]{1} = \mynum[m]{e-10}$}.
						\mynum{8}~\myglos*{glo-Binome} sur \mynum{12} sont concernés par cette catégorie (\myglos*{glo-Binome} \mygroup{5}, \mygroup{6}, \mygroup{7}, \mygroup{8}, \mygroup{9}, \mygroup{10}, \mygroup{11} et \mygroup{12}).
						Ces \myglos*{glo-Binome} travaillent en collaboration étroite sur les \myglos*{glo-Residu}.
						Étant donné la distance inférieure \mynum[\AA]{10}, les \myglos*{glo-Binome} concernés travaillent sur les mêmes \myglos*{glo-Residu}.

						Sur la \myref{fig-exp1-AffiniteEntreLesSujetsPourChaqueBinome}, tous les \myglos*{glo-Binome} manipulant en collaboration étroite ont de fortes affinités ($\mu = 4$) : ce sont des collègues proches ou des amis.
						D'après la \myref{fig-exp1-TempsDeRealisationEntreLesSujetsPourChaqueBinome}, ces \myglos*{glo-Binome} obtiennent des temps de réalisation de la tâche relativement moyens comparés aux autres stratégies de travail.
						Cela se traduit également par une communication variable selon les \myglos*{glo-Binome} comme affichée sur la \myref{fig-exp1-TempsDeCommunicationVerbaleEntreLesSujetsPourChaqueBinome}.

						En observant plus précisément les temps de communication verbale sur la \myref{fig-exp1-PourcentageDeTempsDeCommunicationVerbalePendantLaRechercheEtLaSelectionDesSujetsPourChaqueBinome}, les \myglos*{glo-Binome} de ce groupe passent plus de temps à communiquer durant la phase de recherche que durant la phase de sélection (excepté pour \mygroup{9}).
						Ces résultats mettent en évidence les difficultés du travail en champ proche liées aux nombreux conflits durant la phase de recherche.

						En effet, les \myglos*{glo-Binome} avec de fortes affinités travaillent sur les mêmes \myglos*{glo-Residu}.
						Ils doivent donc coordonner leurs mouvements de manipulation pour déplacer un \myglos{glo-Residu} et cette coordination nécessite une communication verbale importante.
						La collaboration est alors étroitement couplée mais il en résulte une perte de temps à cause de cette communication.

						La \myref{fig-exp1-ForceMoyenneEtDifferenceDeForceEntreLesSujetsPourChaqueBinome} montre de fortes disparités entre les \myglos*{glo-Binome} concernant la force moyenne appliquée durant la manipulation.
						Des observations durant l'expérimentation ont permis de d'identifier deux stratégies adoptées par les sujets :
						\begin{description}
							\item[par contrôle] où les deux sujets effectuent la même action pour obtenir un meilleur contrôle sur les structures manipulées;
							\item[par guidage] où un des deux sujets indique à son partenaire la déformation à effectuer ou la position à atteindre.
						\end{description}
						Ces deux stratégies impliquent une communication étroite entre les sujets afin de coordonner au mieux les actions \myref*{fig-exp1-DistanceMoyenneEntreLesSujetsPourChaqueBinomeSurLesResidusSixNeufEtDix}.

						Les \myglos*{glo-Binome} ne travaillent pas de façon partagée comme le montre les différences importantes de forces appliquées \myref*{fig-exp1-ForceMoyenneEtDifferenceDeForceEntreLesSujetsPourChaqueBinome}.
						Un des deux sujets effectue une majorité du travail contrairement à l'autre sujet.
						De plus, les interactions en champ proche générent de nombreux conflits nécessitant une communication verbale accrue.
						D'ailleurs, les communications verbales révèlent de nombreuses incompréhension dans l'inter-référencement (\og Pas dans cette direction \fg, \og Pas ici mais ici \fg, \og C'est juste derrière \fg, \myetc).
						En effet, la grande complexité des tâches considérées et une conscience incomplète de l'environnement et de l'état de son partenaire provoque des inter-référencements imprécis et entraîne une mauvaise coordination.
						Ces conflits et incompréhensions restreignent les performances du \myglos{glo-Binome}.
					\end{mysubsubsection}
					\begin{mysubsubsection}[sss-exp1-InteractionEnChampVoisin]{Interaction en champ voisin}
						Les interactions en champ voisin, comprises entre \mynum[mm]{8} et \mynum[mm]{14}, correspondent, dans l'environnement virtuel, à des distances de l'ordre de \myglos*{glo-Residu} voisins (entre \mynum[\AA]{10} et \mynum[\AA]{20}).
						\mynum{3}~\myglos*{glo-Binome} sur \mynum{12} se trouvent dans cette catégorie (\myglos*{glo-Binome} \mygroup{2}, \mygroup{3} et \mygroup{4}).
						Ces \myglos*{glo-Binome} travaillent en collaboration relativement étroite sur des \myglos*{glo-Residu} voisins.
						Les \myglos*{glo-Residu} voisins sont dépendants physiquement ou structurellement comme indiqué sur la \myref{fig-exp1-CouplagePhysiqueEtStructurelEntreLesResidus}.
						En effet, les \myglos*{glo-Residu} interagissent entre eux : plus les distances sont courtes, plus les contraintes physiques sont fortes.

						\begin{myfigure}
							\setlength{\mywidth}{10pc}
							\setlength{\myheight}{10pc}
							\psset{xunit=\mywidth,yunit=\myheight}
							\begin{myps}(-1,-0.2)(1.15,1)
								\psset{ref=c}
								\rput(0.5,0.5){\myimage[width=\mywidth,angle=90]{exp1-trp-zipper}}
								\rput(0.35,-0.1){\rnode{connected-residues-label}{\begin{tabular}{c}Connected residues\\[-1ex]\textcolor{black!70}{\scriptsize Structural dependencies}\end{tabular}}}
								\rput(0.2,0.3){\pnode{connected-residues1}}
								\rput(0.5,0.3){\pnode{connected-residues2}}
								\rput(-0.5,0.7){\rnode{close-macro-label}{\begin{tabular}{c}Close macrostructures\\[-1ex]\textcolor{black!70}{\scriptsize Physical dependencies}\end{tabular}}}
								\rput(0,0.9){\pnode{close-macro1}}
								\rput(-0.1,0.5){\pnode{close-macro2}}
								\nccurve[angleA=135,angleB=-90]{->}{connected-residues-label}{connected-residues1}
								\nccurve[angleA=45,angleB=-90]{->}{connected-residues-label}{connected-residues2}
								\nccurve[angleA=90,angleB=180]{->}{close-macro-label}{close-macro1}
								\nccurve[angleA=-90,angleB=180]{->}{close-macro-label}{close-macro2}
							\end{myps}
							\mycaption[fig-exp1-CouplagePhysiqueEtStructurelEntreLesResidus]{Couplage physique et structure entre les \myglosnl*{glo-Residu}}
						\end{myfigure}

						Sur la \myref{fig-exp1-AffiniteEntreLesSujetsPourChaqueBinome}, tous les \myglos*{glo-Binome} manipulant en collaboration moyennement couplées ont des affinités moyennes ($\mu = 3$) : ce sont des collègues de bureau ou d'équipe ne travaillant pas forcément sur les mêmes projets.
						La \myref{fig-exp1-TempsDeRealisationEntreLesSujetsPourChaqueBinome} montre que les \myglos*{glo-Binome} obtiennent de très bonnes performances sur les temps de réalisation de la tâche.
						De plus, la communication verbale est faible comme le montre la \myref{fig-exp1-TempsDeCommunicationVerbaleEntreLesSujetsPourChaqueBinome}.
						La manipulation en champ voisin permet d'être continuellement conscient des actions du partenaire ce qui évite les communications verbales.
						Cependant, les sujets manipulent des \myglos*{glo-Residu} différents ce qui limite les conflits d'interactions qui interviennent en champ proche.

						La \myref{fig-exp1-PourcentageDeTempsDeCommunicationVerbalePendantLaRechercheEtLaSelectionDesSujetsPourChaqueBinome} montre un nombre de conflits plus faible durant la phase de recherche.
						En effet, la communication verbale est nettement moins importante durant la phase de recherche que durant la phase de sélection.

						La \myref{fig-exp1-ForceMoyenneEtDifferenceDeForceEntreLesSujetsPourChaqueBinome} illustre une bonne répartition des efforts entre les deux membres du \myglos{glo-Binome}.
						En effet, la force moyenne est assez élevée par rapport à la plupart des autres \myglos*{glo-Binome} ce qui montre qu'aucun des deux sujets n'est moins actif (ce qui entraînerait une force moyenne moins élevée).
						La différence des forces moyennes quasi-nulle entre les deux sujets confirme ce résultat.
						Ceci peut s'expliquer par une bonne coordination pendant laquelle les deux membres du \myglos{glo-Binome} vont effectuer des actions complémentaires mais de même intensité.
						La stratégie adoptée peut être définie comme une stratégie \myemph{par manipulation complémentaire} : les deux sujets sont attentifs aux actions de leur partenaire afin d'avoir un meilleur contrôle du processus de déformation par une synchronisation améliorée.

						L'analyse des communication verbales met en évidence les phases de communication de coordination (\og Maintenant, prends ça \fg, \og peux-tu m'aider ici ? \fg, \og Bien ! \fg, \myetc).
						Les performances des \myglos*{glo-Binome} travaillant en champ voisin sont relativement élevées bien que quelques conflits similaires à ceux rencontrés en champs proches soient présents bien que plus limités en nombre.
					\end{mysubsubsection}
					\begin{mysubsubsection}[sss-exp1-InteractionEnChampDistant]{Interaction en champ distant}
						Les interactions en champ voisin, supérieures à \mynum[mm]{14}, correspondent, dans l'environnement virtuel, à des \myglos*{glo-Residu} sans interaction physique (supérieur à \mynum[\AA]{20}).
						\mynum{1}~\myglos{glo-Binome} sur \mynum{12} est concerné par cette catégorie (\myglos{glo-Binome} \mygroup{1}).
						Ce \myglos{glo-Binome} travaille de façon faiblement couplée.
						En effet, les membres de ce \myglos{glo-Binome} travaillent de façon complétement indépendante, en limitant au maximum le nombre d'interactions.

						Les affinités des membres de ce \myglos{glo-Binome} sont très faibles \myref*{fig-exp1-AffiniteEntreLesSujetsPourChaqueBinome} : les membres ne se connaissent presque pas.
						De plus, le \myglos{glo-Binome} obtient de très mauvaises performances en ce qui concerne le temps de réalisation de la tâche comme le montre la \myref{fig-exp1-TempsDeRealisationEntreLesSujetsPourChaqueBinome}.
						La \myref{fig-exp1-TempsDeCommunicationVerbaleEntreLesSujetsPourChaqueBinome} montre que le temps de communication verbale est assez important.
						Cependant, le temps de réalisation étant nettement plus important, le taux de communication verbale est beaucoup plus faible que les autres groupes \myref*{fig-exp1-PourcentageDeTempsDeCommunicationVerbalePendantLaRechercheEtLaSelectionDesSujetsPourChaqueBinome}.
						En effet, les membres du \myglos{glo-Binome} travaillent à distance et ont peu d'interactions entre eux.
						Le peu d'interaction permet de limiter le nombre de conflits ce qui implique le peu de communication verbale comme on peut le voir sur la \myref{fig-exp1-PourcentageDeTempsDeCommunicationVerbalePendantLaRechercheEtLaSelectionDesSujetsPourChaqueBinome}.
						Cette figure montre également que ce \myglos{glo-Binome} communique plus dans les phases de sélection que dans les phases de recherche.
						En effet, les phases de sélection forcent une collaboration étroite (spécificité de la tâche proposée) et favorisent les conflits.

						La \myref{fig-exp1-ForceMoyenneEtDifferenceDeForceEntreLesSujetsPourChaqueBinome} montre un effort moyen appliqué par les \myglos*{glo-Binome} peu élevé (comparé aux stratégies en champ voisin).
						De plus, les forces moyennes appliquées par chacun des deux sujets sont très inégales.
						Il y a une mauvaise répartition de la charge de travail au sein du \myglos{glo-Binome}.

						En effectuant des interactions en champ distants, les sujets se définissent leur propre espace de travail mais également leur propre stratégie en fonction des événement locaux à leur espace de travail.
						Les interactions entre les sujets sont limitées au maximum.
						Cette configuration réduit considérablement les conflits d'interaction ainsi que la communication.
						Cependant, elle nuit beaucoup aux performances du groupe dans son ensemble.
						En effet, les stratégies sont différentes et la phase de sélection nécessite une collaboration.
						De plus, l'inégalité de la répartition des charges de travail dévalue les performances.
					\end{mysubsubsection}
					\begin{mysubsubsection}[sss-exp1-SyntheseDesStrategiesDeTravail]{Synthèse des stratégies de travail}
						Les \myglos*{glo-Binome} sont susceptibles d'adopter une des trois stratégies de travail vues dans les sections précédentes.
						Pour certaines, les interactions en champ distants semblent convenir mais au détriment des performances : la collaboration est quasiment inexistante.
						D'autres \myglos*{glo-Binome} interagissent en champ proches et obtiennent des performances moyennes : la collaboration est étroitement couplée mais souffre des nombreux conflits.

						Cependant, ce sont les interactions en champ voisins qui produisent les meilleures performances.
						En effet, les conflits sont plus limités que pour des interactions en champ proche mais la collaboration est tout de même couplée.
						Les résultats montrent à la fois de bonnes performances en terme de temps de réalisation mais aussi en terme de répartition des charges de travail tout en limitant les communication verbales.
						La plupart du temps, les communications verbales sont destinées à la résolution de conflit : elles sont très chronophages et peuvent être évitées.
						C'est pour cette raison que nous proposerons des outils haptiques pour améliorer cette gestion des conflits \myref*{cha-TravailCollaboratifAssisteParHaptique}.
					\end{mysubsubsection}
				\end{mysubsection}
				\begin{mysubsection}[sse-exp1-ResultatsQualitatifs]{Résultats qualitatifs}
					Les résultats qualitatifs sont constitués de deux parties.
					La première permet de déterminer les impressions des sujets concernant la collaboration, les rôles et efficacité de chacun durant la tâche.
					La seconde partie a pour but d'évaluer la plate-forme.
					Toutes les notes sont comprises entre un et cinq (échelle de \mycite[author]{Likert-1932} à cinq niveaux).
					\begin{mysubsubsection}[sss-exp1-EvaluationDuTravailEnCollaboration]{Évaluation du travail en collaboration}
						Les résultats du questionnaire montre qu'une majorité des sujets de cette expérimentation ont apprécié et préféré la réalisation de la tâche en configuration collaborative (\mysummary{exp1-evaluation-group.tex}).
						De plus, le sentiment d'effectuer une tâche en collaboration est fort.
						L'hypothèse \myhypothesis{3} est confirmée par les sujets qui préfèrent le travail en collaboration que le travail en \myglos{glo-Monome}.
						C'est un point important pour la poursuite de nos travaux de recherche sur le travail collaboratif : les sujets apprécient le travail en collaboration.

						Durant les tâches collaboratives, les sujets considèrent qu'ils ont effectivement contribués à la réalisation de la tâche (\mysummary{exp1-evaluation-help.tex}).
						Cependant, les sujets considèrent qu'ils ne se sont imposés ni en meneur ou ni en suiveur (\mysummary{exp1-evaluation-leader.tex}).
						En effet, des questions supplémentaires ont permis de mettre en évidence que chaque sujet a tendance à surestimer le rôle du partenaire ($\approx\mynum[\%]{70}$).

						La collaboration fonctionne grâce à une confiance mutuelle : chaque sujet considère que son partenaire a effectué le travail demandé.
						Aucune vérification n'est effectuée par un sujet sur le travail effectué par le partenaire : ceci permet de construire cette confiance mutuelle.
						De plus, ceci permet à chaque sujet de se sentir utile à la réalisation de la tâche et d'éviter les phénomènes de paresse sociale.
						Dans le cas où un sujet remet en cause le travail de son partenaire, le partenaire ne se sent plus utile et peut s'isoler, soustrayant ainsi son potentiel de l'action collaborative.

						Concernant la communication, les participants estiment qu'ils exploitent principalement la communication verbale (\mysummary{exp1-evaluation-verbal.tex}) et, dans une proportion plus faible mais tout de même importante, virtuelle (\mysummary{exp1-evaluation-virtual.tex}).
						En ce qui concerne la communication gestuelle, ils la considèrent quasiment inexistante (\mysummary{exp1-evaluation-gestural.tex}).

						La communication gestuelle n'est pas ou peu utilisée pour plusieurs raisons.
						La principale raison est la difficulté de communiquer avec des gestes lorsque les mains sont occupées par la manipulation.
						Deuxièmement, les sujets ont rapidement adopté le canal virtuel qui est plus précis dans les tâches de désignation qui constituent la plupart des besoins de communication.
						La communication verbale reste le canal principal de communication : c'est le canal le plus naturel pour communiquer.
						Cependant, il vient aussi en soutien du canal virtuel.
						En effet, aucun outil visuel ou haptique n'a été fourni pour effectuer des tâches de désignation et le canal virtuel seul serait incapable de remplir seul cette mission.
					\end{mysubsubsection}
					\begin{mysubsubsection}[sss-exp1-EvaluationDuSysteme]{Évaluation du système}
						L'évaluation du système en terme d'intuitivité comme en terme de confort est relativement satisfaisante.
						En effet, en ce qui concerne l'intuitivité des graphismes et effets visuels, les participants les trouve accessibles (\mysummary{exp1-platform-visual-intuitive.tex}).
						Il est en va de même en ce qui concerne l'intuitivité des interactions avec le système (\mysummary{exp1-platform-interaction-intuitive.tex}).
						Pour le confort, le visuel (\mysummary{exp1-platform-visual-confortable.tex}) et les interactions (\mysummary{exp1-platform-interaction-confortable.tex}) jouissent d'une évaluation similaire.

						Là encore, les sujets valident l'hypothèse \myhypothesis{4}.
						La plate-forme est relativement bien évaluée.
						Il semble cependant nécessaire d'apporter encore des améliorations afin de répondre au mieux aux attentes des utilisateurs.

						Ces résultats sont cependant à nuancer.
						Les écart-types sont relativement élevés ce qui veut dire qu'il y a de fortes disparités dans ces notations entre les différents sujets : certains sujets se sont déclarés plutôt insatisfaits concernant le confort (visuel : \mynum{2}, interaction : \mynum{2}).
						De plus, les outils proposés durant cette expérimentation sont relativement simples et peu envahissants.
						Des outils plus complexes, plus informatifs seraient peut-être moins intuitifs au premier abord et pourrait mener à un inconfort.
					\end{mysubsubsection}
				\end{mysubsection}
			\end{mysection}
			\begin{mysection}[sec-exp1-Synthese]{Synthèse}
				\begin{mysubsection}[sse-exp1-ResumeDesResultats]{Résumé des résultats}
					Dans ce chapitre, nous avons observé et comparé les performances de \myglos*{glo-Monome} et de \myglos*{glo-Binome} pendant une tâche de recherche et de sélection sur une simulation moléculaire en temps-réel.
					L'objectif était de montrer l'intérêt du travail collaboratif dans l'amélioration des performances et d'identifier les différentes stratégies de travail.
					De plus, il fallait valider la pertinence du système mis en place.

					La collaboration a prouvé son intérêt, notamment sur les tâches les plus complexes.
					Cependant, la complexité d'une tâche est relativement difficile à établir.
					Le nombre d'atomes (et donc le nombre de \myglos*{glo-Residu}) joue un rôle important dans cette complexité.
					Un grand nombre d'atomes surcharge l'environnement virtuel qui difficile à appréhender.
					Un deuxième facteur de complexité à prendre en compte est l'amplitude des contraintes physiques de la molécule.
					Certaines zones de la molécule sont dans un état de stabilité tel qu'il est difficile de déformer les \myglos*{glo-Residu} de cette zone.

					En observant et en analysant les différentes stratégies de travail, il ressort que les interactions en champ proche et les interactions en champ distant ne sont pas des stratégies très performantes.
					En effet, le nombre de conflits durant les interactions en champ proche est très important alors que le potentiel de la collaboration est perdu dans des interaction en champ distant.
					Ce sont les interactions en champ voisin qui offre les meilleures performances, générant un bon compromis entre collaboration étroite et gestion des conflits.

					Enfin, il paraît nécessaire d'avoir de bonnes relations sociales avec ces partenaires afin d'apporter à la fois, une communication saine et un respect mutuel du travail effectué.
					Les résultats montrent de façon évidente que tout déséquilibre dans le groupe mène à des performances dégradées.
				\end{mysubsection}
				\begin{mysubsection}[sse-exp1-Perspectives]{Perspectives}
					Basés sur les résultats précédents, certaines perspectives assez évidentes s'imposent et ont guidé les expérimentations qui suivent.
					Tout d'abord, il semble nécessaire de proposer des tâches suffisamment complexes pour pouvoir étudier plus en détail le travail collaboratif.
					Ceci se traduit soit par des tâches à fortes zones de contraintes \myref*{cha-DeformationCollaborativeDeMolecule} ou par la manipulation de molécules de taille importante \myref*{cha-LesDynamiquesDeGroupe}.

					Les différentes stratégies observées ont permis de mettre en évidence l'intérêt du travail en champ voisin.
					Les propositions d'outils visuo-haptiques devront tenir compte de ce paramètre : ils devront encouragert le travail rapproché en fournissant une assistance en champ voisin tout en maintenant une distance minimum afin de limiter les conflits liés au travail en champ proche.

					L'évaluation qualitative par questionnaire apporte également de nombreuses réponses intéressantes.
					Tout d'abord, les sujets ont mis en avant un élément primordial de la communication : le canal virtuel est important.
					À l'aide d'observations durant les phases expérimentales, ce canal de communication est principalement exploité pour des actions de désignation.
					Fournir des outils spécifiquement conçus pour la désignation devient une nécessité.

					Enfin, ces évaluations qualitatives ont permis de valider l'\myacro{acr-EVC} proposé.
					Des améliorations sont cependant nécessaires en ce qui concerne le rendu visuel et les interactions.
					De nombreux sujets ont par exemple demandé une mise en surbrillance du \myglos{glo-Residu} survolé.
					Une assistance haptique pour la sélection est également une des améliorations possibles.
					Ces améliorations ne sont pas implémentées dans les deux expérimentations suivantes pour ne pas alourdir les outils et ainsi ne pas biaiser l'étude.
					Cependant, ils sont implémentés pour la dernière expérimentation \myref*{cha-TravailCollaboratifAssisteParHaptique}.
				\end{mysubsection}
			\end{mysection}
		\end{mychapter}
		\begin{mychapter}[cha-DeformationCollaborativeDeMolecule]{Déformation collaborative de molécule}
			\begin{mysection}[sec-exp2-Introduction]{Introduction}
				Dans le précédent chapitre, nous avons effectué une première étude du travail collaboratif sur une tâche de recherche et de sélection.
				Cette étude montre la pertinence d'un travail en \myglos{glo-Binome} pour la \myemph{recherche} et la \myemph{sélection} de \myglos*{glo-Residu} dans un environnement complexe.
				En effet, les \myglos*{glo-Binome} sont plus performants sur des tâches très complexes et gardent des performances équivalentes aux \myglos*{glo-Monome} sur des tâches de complexité faible.
				Cette première expérimentation nous permet d'identifier certains points importants nécessaires pour nos prochaines études.

				Tout d'abord, nous allons focaliser nos études sur des tâches de complexité importante.
				La caractérisation de la complexité d'une tâche dépend en grande partie de la nature de la tâche elle-même (par exemple, la forme ou la couleur d'un \myglos{glo-Residu}.
				Cependant, le nombre total d'atomes (et donc de \myglos*{glo-Residu}) présents dans la molécule est un facteur de complexité qui ne dépendra pas ou peu de la nature de la tâche.
				Ce facteur pourra être utilisé pour proposer des tâches complexes.

				Les deux premières tâches élémentaires du processus de déformation moléculaire \myref*{fig-ProcessusDeDeformationMoleculaireEnQuatreEtapes} ont été étudiées dans cette première études.
				Cette seconde étude se rapproche d'un processus complet de déformation moléculaire puisqu'elle se focalisera sur la tâche élémentaire de \myemph{manipulation}.

				Tout d'abord, nous présenterons les objectifs et les hypothèses de cette seconde expérimentation dans la \myref{sec-exp2-Presentation}.
				Ensuite, le dispositif expérimental, modifié pour cette nouvelle étude, est présenté dans la \myref{sec-exp2-DispositifExperimentalEtMateriel}.
				La \myref{sec-exp2-Methode} expose les différents détails concernant la tâche proposée et les variables étudiées.
				Les résultats seront analysés dans la \myref{sec-exp2-Resultats}.
				Enfin, la \myref{sec-exp2-Synthese} effectue une synthèse de cette seconde étude et propose des perspectives pour les expérimentations qui vont suivre.
			\end{mysection}
			\begin{mysection}[sec-exp2-Presentation]{Présentation}
				\begin{mysubsection}[sse-exp2-Objectifs]{Objectifs}
					Après avoir étudié la tâche élémentaire de \myemph{recherche} et de \myemph{sélection}, cette seconde expérimentation se portera plus en détails sur la tâche élémentaire de \myemph{déformation} en collaboration \myref*{fig-ProcessusDeDeformationMoleculaireEnQuatreEtapes}.
					Cette tâche nécessite une grande synchronisation et favorise les collaborations étroites.
					La précédente expérimentation \myref*{cha-RechercheCollaborativeDeResidusEnSimulationMoleculaire} a souligné l'avantage de la collaboration sur des tâches nécessitant un couplage fort.
					Les tâches proposées dans cette expérimentation sont élaborées pour stimuler les interactions entre les sujets.

					L'utilisation d'interfaces haptiques pour la déformation d'objets flexibles n'est pas une idée nouvelle.
					\mycite[author]{Shen-2006} propose déjà une solution pour déformer des objets non-rigides et teste son modèle avec des interfaces \myOmni.
					Les objets concernés sont plutôt simples comme des sphères.
					Quelques années plus tard, \mycite[author]{Peterlik-2009} effectue une thèse sur les déformations de tissus cellulaires.
					Jusqu'à présent, l'utilisation de l'haptique pour la déformation en temps-réel de molécules n'a pas été étudié.
					Nous pouvons tout de même citer \mycite[author]{Subasi-2006} qui propose une approche de manipulation de molécule rigide utilisant l'haptique.

					Les processus de déformation collaboratifs sont également peu étudiés.
					La littérature s'attarde principalement sur des \myacro*{acr-EVC}.
					\mycite[author]{Sumengen-2007} propose une plate-forme permettant la déformation d'objets dans un \myacro{acr-EVC}.
					L'objectif de cette plate-forme est de répondre à des contraintes de latence liées au réseau.
					Pour répondre à la problématique de latence, \mycite[author]{Tang-2010a} encode les paramètres du maillage pour accélérer la transmission.
					Les \myacro*{acr-EVC} proposés sont destinés à des collaborations distantes et la problématique du temps-réel en est le sujet d'intérêt principal.
					Notre plate-forme se place dans un contexte colocalisé où les contraintes de réseau n'ont pas d'existence.

					On peut tout de même trouver quelques exemples de manipulation collaborative à l'aide d'interface haptique.
					\mycite[author]{Gautier-2008} propose par exemple une plate-forme de travail collaboratif pour la \myacro{acr-CAO}.
					Son projet repose sur la manipulation de corps rigides.
					Pour des corps flexibles, \mycite[author]{Muller-2006} développe le logiciel \myClayWorks qui permet la sculpture sur glaise.
					La manipulation s'effectue alors sur un objet malléable.
					Ici encore, l'accent est mis sur la collaboration distante comme le montre cette autre publication sur \myClayWorks \mycite{Gorlatch-2009}.

					L'expérimentation proposée dans cette nouvelle étude va de nouveau mettre en opposition un \myglos{glo-Monome} et un \myglos{glo-Binome}.
					L'objectif est comparer une manipulation \myglos*{glo-Bimanuel} en \myglos{glo-Monome} avec une manipulation \myglos*{glo-Monomanuel} en \myglos{glo-Binome}.
					L'expérimentation s'appuie sur un \myacro{acr-EVC} permettant une collaboration colocalisée, s'affranchissant ainsi des problématiques de la collaboration distante.

					De plus, nous mettrons en relation la performance des groupes en fonction de la complexité de la tâche.
					Ce point nous permet de confirmer les conclusions de la première étude mais dans un contexte de manipulation moléculaire.

					Enfin, cette seconde étude est l'occasion d'observer l'effet du travail collaboratif sur l'apprentissage.
					En effet, le travail de groupe peut stimuler l'apprentissage et donc être bénéfique en terme de performances sur le court terme.
				\end{mysubsection}
				\begin{mysubsection}[sse-exp2-Hypotheses]{Hypothèses}
					Les hypothèses de cette nouvelle étude sont en grande partie basée sur l'étude précédente.
					Nous souhaitons confirmer l'intérêt du travail collaboratif dans la tâche élémentaire de \myemph{manipulation}, notamment sur les tâches de complexité importante.
					De plus, cette expérimentation propose d'étudier l'apprentissage de la tâche et d'en observer l'évolution dans le cadre du travail collaboratif.
					\begin{myparagraph}[par-exp2-AmeliorationDesPerformancesEnBinome]{\myhypothesis{1} Amélioration des performances en \myglosnl{glo-Binome}}
						Nous émettons l'hypothèse que les performances des \myglos*{glo-Binome} seront meilleures que les performances des \myglos*{glo-Monome}.
						Cette hypothèse a pour objectif de confirmer les conclusions obtenues dans la première étude dans un contexte de \myemph{manipulation}.
						La première hypothèse est une amélioration des performances pour les \myglos*{glo-Binome} en collaboratif comparés aux \myglos*{glo-Monome} en \myglos{glo-Bimanuel}.
					\end{myparagraph}
					\begin{myparagraph}[par-exp2-MeilleurGainDePerformancesSurLesTachesComplexes]{\myhypothesis{2} Meilleur gain de performances sur les tâches complexes}
						Nous émettons l'hypothèse que plus la tâche est complexe et plus une configuration de travail collaboratif produira un gain significatif de performances comparé à un \myglos{glo-Monome}.
					\end{myparagraph}
					\begin{myparagraph}[par-exp2-LApprentissageEstPlusPerformantPourLesBinomes]{\myhypothesis{3} L'apprentissage est plus performant pour les \myglosnl*{glo-Binome}}
						Nous émettons l'hypothèse que le travail en collaboration augmente la vitesse d'apprentissage de la tâche.
						En effet, nous supposons que l'interaction entre les partenaires va stimuler l'apprentissage et permettre l'échange des connaissances.
					\end{myparagraph}
				\end{mysubsection}
			\end{mysection}
			\begin{mysection}[sec-exp2-DispositifExperimentalEtMateriel]{Dispositif expérimental et matériel}
				Cette seconde expérimentation se base sur un dispositif expérimental relativement similaire à celui de la première expérimentation.
				L'expérience est basée sur l'\myacro{acr-EVC} présenté dans le \myref{cha-ShaddockSystemeCollaboratifPourLaManipulationDeMolecules}.
				Dans un premier temps, une liste du matériel utilisé est présenté dans la \myref{sse-exp2-DispositifTechnique} puis la disposition de chacun de ces éléments dans la \myref{sse-exp2-DispositionDesElements}.
				Puis, dans un second temps, les aspects logiciels seront détaillés dans la \myref{sse-exp2-VisualisationEtInteractions}.
				\begin{mysubsection}[sse-exp2-DispositifTechnique]{Dispositif technique}
					La réalisation de cette expérimentation nécessite l'ensemble des matériels suivants :
					\begin{itemize}
						\item \mynum{1}~ordinateur quatre cœurs \myIntelCore avec \myRAM[Go]{4};
						\item \mynum{2}~ordinateurs de faible puissance;
						\item \mynum{2}~interfaces haptiques \myOmni;
						\item \mynum{1}~souris \myThreeD \mySpaceNavigator;
						\item \mynum{1}~vidéoprojecteur \myCasioXJ;
						\item \mynum{1}~grand écran de vidéoprojection.
					\end{itemize}
				\end{mysubsection}
				\begin{mysubsection}[sse-exp2-DispositionDesElements]{Disposition des éléments}
					Cette nouvelle étude propose une disposition relativement similaire à la première expérimentation \myref*{sse-exp1-DispositionDesElements}.
					Les sujets sont face à un écran de vidéoprojection, le vidéoprojecteur se trouvant derrière eux.
					La vue partagée permet à tous les sujets de voir le grand écran et de communiquer librement.

					Deux interfaces haptiques se trouvent faces aux sujets : ce sont deux outils de déformation \mytool{tug} \myref*{par-simulation-LesOutilsDeManipulation}.
					Le sujet d'un \myglos{glo-Monome} aura accès aux deux interfaces en manipulation \myglos*{glo-Bimanuel}.
					Dans le cas d'un \myglos{glo-Binome}, chaque sujet aura un outil à sa disposition.
					De plus, le \mySpaceNavigator est placé entre les deux interfaces haptiques afin d'être accessibles à tous les sujets.

					L'expérimentateur dirige les opérations depuis un poste de travail placé derrière les sujets.
					C'est lui qui lance et stoppe les différents scénarios proposés.

					La \myref{fig-exp2-SchemaDuDispositifExperimental} illustre le dispositif expérimental par un schéma.
					La \myref{fig-exp2-PhotographieDuDispositifExperimental} est une photographie de la salle d'expérimentation.

					\begin{myfigure}
						\myimage{exp2-schema}
						\mysubcaption[fig-exp2-SchemaDuDispositifExperimental]{Schéma du dispositif expérimental}
					\end{myfigure}
					\begin{myfigure}
						\myimage{exp2-photo}
						\mysubcaption[fig-exp2-PhotographieDuDispositifExperimental]{Photographie du dispositif expérimental}
					\end{myfigure}
				\end{mysubsection}
				\begin{mysubsection}[sse-exp2-VisualisationEtInteractions]{Visualisation et interactions}
					Chaque molécule proposée est projetée dans son intégralité sur l'écran de vidéoprojection avec les rendus graphiques suivants \myref*{par-simulation-LesRendusGraphiques} :
					\begin{itemize}
						\item un rendu \myCPK avec des atomes de taille assez petite afin de pouvoir apprécier l'ensemble des atomes de la molécule;
						\item un second rendu \myCPK pour agrandir tous les atomes sauf les atomes d'hydrogène qui sont peu informatifs;
						\item un rendu \myNewRibbon pour apprécier la structure globale de la molécule.
					\end{itemize}
					De plus, un affichage de l'état stable de la molécule est affiché par un rendu \myNewRibbon transparent.
					Pour les besoins de la simulation, certains \myglos*{glo-Residu} sont fixes et sont alors représentés avec la couleur grise.
					Enfin, les \myglos*{glo-Residu} sélectionné sont affichés avec un rendu \myCPK en transparence et de la couleur du curseur concerné.
					La \myref{fig-exp2-IllustrationDesRendusPourLAffichageDeLaMolecule} illustre l'ensemble des rendus graphiques précédemment décrits.

					\begin{myfigure}
						\psset{unit=0.08\textwidth}
						\def\myexptwolabel(#1,#2)[#3]#4#5{\rput(#1,#2){\rnode{#3}{\textcolor{#4}{\sffamily #5}}}}
						\begin{myps}(0,0)(11,8.5)
							\rput[bl](1,0){\myimage{exp2-trp-zipper}}
							\myexptwolabel(9.4,2.6)[deformed-label]{myred}{Molécule à déformer}
							\myexptwolabel(1,5.5)[ghost-label]{myred}{Molécule cible}
							\myexptwolabel(7.1,7.3)[deformed-residue-label]{myblue}{\myGlosnl{glo-Residu} sélectionné}
							\myexptwolabel(1,3)[ghost-residue-label]{myblue}{\myGlosnl{glo-Residu} cible}
							\myexptwolabel(4.0,8.15)[fixed-residue-label]{mygray}{\myGlosnl{glo-Residu} fixe}
							\pnode(7.4,3.6){deformed}
							\pnode(1.8,4){ghost}
							\psset{linecolor=myblue}
							\cnode(6.2,5.2){1.0}{deformed-residue}
							\cnode(2.3,1.5){0.8}{ghost-residue}
							\psset{linecolor=mygray}
							\cnode(2.0,7){0.8}{fixed-residue}
							\psset{linewidth=1pt,linecolor=myred,linearc=.1,arrowsize=1pt 3,arrowinset=.2,nodesepA=3pt}
							\ncangle[angleA=90,angleB=0]{c->}{deformed-label}{deformed}
							\ncangle[angleA=-90,angleB=180,offsetA=-0.5]{c->}{ghost-label}{ghost}
							\psset{linecolor=myblue,nodesepB=0pt}
							\ncdiagg[angleA=-90,offsetA=0.5]{c->}{deformed-residue-label}{deformed-residue}
							\ncdiagg[angleA=-90,offsetA=-0.5]{c->}{ghost-residue-label}{ghost-residue}
							\ncdiagg[angleA=180,linecolor=mygray]{c->}{fixed-residue-label}{fixed-residue}
							\ncline[linewidth=10pt,linecolor=myblue,arrowsize=2pt 2,nodesepA=4pt]{C->}{deformed-residue}{ghost-residue}
						\end{myps}
						\mycaption[fig-exp2-IllustrationDesRendusPourLAffichageDeLaMolecule]{Illustration des rendus pour l'affichage de la molécule}
					\end{myfigure}

					Les outils \mytool{tug} vont permettre aux sujets de déformer la molécule.
					Les \myglos*{glo-Monome} se placent devant les deux interfaces haptiques de façon à pouvoir manipuler en configuration \myglos*{glo-Bimanuel}.
					Dans le cas d'un \myglos{glo-Binome}, chaque sujet est installé face à un outil.
					Le \mySpaceNavigator est laissé libre d'utilisation pour les sujets quelque soit le nombre de sujets.
				\end{mysubsection}
			\end{mysection}
			\begin{mysection}[sec-exp2-Methode]{Méthode}
				\begin{mysubsection}[sse-exp2-Sujets]{Sujets}
					\mysummary{exp2-subjects.tex} avec une moyenne d'âge de \mysummary{exp2-age.tex} ont participé à cette expérimentation.
					Ils ont tous été recrutés au sein du laboratoire \myacro{acr-LIMSI} et sont chercheurs ou assistants de recherche dans les domaines suivants~:
					\begin{itemize}
						\item linguistique et traitement automatique de la parole;
						\item réalité virtuelle et système immersifs;
						\item audio-acoustique.
					\end{itemize}
					Ils ont tous le français comme langue principale.
					Aucun participant n'a de déficience visuelle (ou corrigée le cas échéant) ni de déficience audio.

					Chaque participant est complètement naïf concernant les détails de l'expérimentation.
					Une explication détaillée de la procédure expérimentale leur est donnée au commencement de l'expérimentation mais en omettant l'objectif de l'étude.
				\end{mysubsection}
				\begin{mysubsection}[sec-exp2-Variables]{Variables}
					\begin{mysubsubsection}[sss-exp2-VariablesIndependantes]{Variables indépendantes}
						\begin{myparagraph}[par-exp2-NombreDeSujets]{\myvari{1} Nombre de sujets}
							La première \myglos{glo-VariableIndependante} est une \myglos{glo-VariableInterSujets}.
							\myvari{1} possède deux valeurs possibles: \og un sujet (\mycf \myemph{\myglos{glo-Monome}}) \fg ou \og deux sujets (\mycf \myemph{\myglos{glo-Binome}}) \fg.
							\mynum{12}~\myglos*{glo-Monome} et \mynum{12}~\myglos*{glo-Binome} sont testés.
						\end{myparagraph}
						\begin{myparagraph}[par-exp2-ComplexiteDeLaTache]{\myvari{2} Complexité de la tâche}
							La seconde \myglos{glo-VariableIndependante} est une \myglos{glo-VariableIntraSujets}.
							Deux tâches de déformation sont proposées sur chacune des deux molécules : une déformation au niveau inter-moléculaire et une déformation au niveau intra-moléculaire.
						\end{myparagraph}
						\begin{myparagraph}[par-exp2-LeNiveauDApprentissage]{\myvari{3} Le niveau d'apprentissage}
							La troisième \myglos{glo-VariableIndependante} est une \myglos{glo-VariableIntraSujets}.
							Tous les sujets seront confrontés trois fois à la même tâche afin de voir l'effet de l'apprentissage en \myglos{glo-Monome} et en \myglos{glo-Binome}.
						\end{myparagraph}
					\end{mysubsubsection}
					\begin{mysubsubsection}[sec-exp2-VariablesDependantes]{Variables dépendantes}
						\begin{myparagraph}[par-exp2-LeTempsDeRealisation]{\myvard{1} Le temps de réalisation}
							C'est le temps total pour réaliser la tâche demandée, c'est-à-dire manipuler et déformer la molécule afin d'atteindre l'objectif fixé.
							Le temps est limité à \mynum[mn]{10}.
						\end{myparagraph}
						\begin{myparagraph}[par-exp2-LeNombreDeSelections]{\myvard{2} Le nombre de sélections}
							\myvard{2} représente le nombre de sélections réalisées durant chaque tâche à réaliser.
							Une sélection est comptabilisée lorsqu'un atome ou un \myglos{glo-Residu} est sélectionné par un des deux \myglos{glo-EffecteurTerminal}.
							Un compteur est affecté pour chacun des \myglos*{glo-EffecteurTerminal} qui lui-même est associé à un sujet.
						\end{myparagraph}
						\begin{myparagraph}[par-exp2-LaDistancePassiveEntreLesEspacesDeTravail]{\myvard{3} La distance passive entre les espaces de travail}
							Cette distance est la distance moyenne entre les deux \myglos*{glo-EffecteurTerminal} présents durant l'ensemble de l'expérimentation.
							Cette distance représente donc une distance physique du monde réel, pas une distance virtuelle.
							Elle est de l'ordre du centimètre.
						\end{myparagraph}
						\begin{myparagraph}[par-exp2-LaDistanceActiveEntreLesEspacesDeTravail]{\myvard{4} La distance active entre les espaces de travail}
							Cette distance est la distance moyenne entre les deux \myglos*{glo-EffecteurTerminal} présents seulement lorsque ces deux \myglos*{glo-EffecteurTerminal} sont en cours de manipulation (un atome ou un \myglos{glo-Residu} est sélectionné).
							C'est une distance physique du monde réel de l'ordre du centimètre.
						\end{myparagraph}
						\begin{myparagraph}[par-exp2-ReponsesQualitatives]{\myvard{5} Vitesse moyenne}
							Cette variable est une mesure de la vitesse moyenne de chaque \myglos*{glo-EffecteurTerminal}.
							Elle est calculée par intégration numérique des positions successives en fonction du temps.
						\end{myparagraph}
						\begin{myparagraph}[par-exp2-ReponsesQualitatives]{\myvard{6} Réponses qualitatives}
							Un questionnaire est proposé à tous les sujets (variable en fonction des \myglos*{glo-Monome} et des \myglos*{glo-Binome}).
							Il se décline en deux versions destinées soit aux \myglos*{glo-Monome}, soit aux \myglos*{glo-Binome}.
							Le questionnaire soumis aux sujets est disponible dans la \myref{sec-Questionnaires-SecondeExperimentation}.
						\end{myparagraph}
					\end{mysubsubsection}
				\end{mysubsection}
				\begin{mysubsection}[sse-exp2-Tache]{Tâche}
					La tâche proposée est la déformation dans un \myacro{acr-EVC} sur des molécules complexes.

					\begin{mysubsubsection}[sss-exp2-DescriptionDeLaTache]{Description de la tâche}
						La tâche proposée est la déformation d'une molécule.
						L'objectif est de la rendre conforme à un modèle.
						L'intégralité des atomes de la molécule à déformer est affiché.
						Trois molécules sont utilisées dans le cadre de cette expérimentation :
						\begin{description}
							\item[\myPrion]
								La molécule nommée \myPrion \mycite{Christen-2009} avec l'identifiant \myPDB \myPDBlink{http://www.rcsb.org/pdb/explore/explore.do?structureId=2KFL}{2KFL} sur la \myPDBbase\footnote{\url{http://www.pdb.org/}}.
								Cette molécule contient \mynum{1779}~atomes dont \mynum{112}~\myglos*{glo-Residu}.
								Elle est très complexe et est seulement utilisée pour un entraînement complet et un apprentissage approfondi des outils de manipulation.
							\item[\myTRPZIPPER]
								La molécule \myTRPZIPPER \mycite{Christen-2009} a pour identifiant \myPDB \myPDBlink{http://www.rcsb.org/pdb/explore/explore.do?structureId=2KFL}{2KFL} sur la \myPDBbase\footnotemark[\value{footnote}].
								Cette molécule contient \mynum{218}~atomes dont \mynum{12}~\myglos*{glo-Residu}.
							\item[\myTRPCAGE]
								La molécule nommée \myTRPCAGE \mycite{Neidigh-2002} a pour identifiant \myPDB \myPDBlink{http://www.rcsb.org/pdb/explore/explore.do?structureId=1L2Y}{1L2Y} sur la \myPDBbase\footnotemark[\value{footnote}].
								Cette molécule contient \mynum{304}~atomes dont \mynum{20}~\myglos*{glo-Residu}.
						\end{description}

						Lorsqu'un sujet sélectionne un atome ou un \myglos{glo-Residu}, ce dernier est mis en surbrillance.
						De plus, l'atome ou le \myglos{glo-Residu} correspondant sur la molécule cible est affiché afin de connaître la position finale de la sélection courante.

						Afin de pouvoir évaluer la déformation effectuée, un score est affiché en temps-réel \myref*{fig-exp2-AffichageDeLaMoleculeADeformerEtDeLaMoleculeCible}.
						Le score affiché est le \myacro{acr-RMSD} qui permet de mesurer la différence entre deux déformations d'une même molécule en calculant la différence entre chaque paire d'atomes \myref*{equ-RMSD}.
						\begin{equation}\label{equ-RMSD}
							\mathrm{RMSD}\left(\mathbf{c},\mathbf{m}\right) = \sqrt{\frac{1}{N}\sum_{i=1}^{N}\mynorm{c_i - m_i}^2}
						\end{equation}
						où $N$~est le nombre total d'atomes et~$c_i$, $m_i$~sont respectivement l'atome~$i$ de la molécule à comparer et de la molécule modèle.

						\begin{myfigure}
							\psset{unit=0.08\textwidth}
							\begin{myps}(0,0)(12,9)
								\rput[bl](1,0){\myimage[width=0.6\textwidth]{exp2-trp-zipper}}
								\rput[bl](6.2,5){\myimage[width=5cm,angle=0]{exp2-red-cursor}}
								\rput[bl](8.5,0.5){\myimage[width=3.5cm,angle=-20]{exp2-green-cursor}}
								\psframe*[linecolor=red](0,8)(12,9)
								\psframe*[linecolor=green](0,8)(2,9)
								\rput(6,8.5){\textcolor{white}{\bfseries\sffamily\LARGE Score RMSD}}
								\psframe[linewidth=1pt,linecolor=black](0,0)(12,9)
							\end{myps}
							\mycaption[fig-exp2-AffichageDeLaMoleculeADeformerEtDeLaMoleculeCible]{Affichage de la molécule à déformer et de la molécule cible}
						\end{myfigure}
					\end{mysubsubsection}
					\begin{mysubsubsection}[sss-exp2-DescriptionDesScenarios]{Description des scénarios}
						Quatre scénarios sont proposés : deux molécules expérimentées dans deux niveaux de manipulation différents.
						Les deux niveaux différents de manipulation sont :
						\begin{itemize}
							\item inter-moléculaire (à l'échelle d'un \myglos{glo-Residu});
							\item intra-moléculaire (à l'échelle d'un atome).
						\end{itemize}

						Les paragraphes suivants décrivent les quatre scénarios :
						\begin{description}
							\item[Scénario~\myscenario{1a}]
								Cette tâche concerne la manipulation de la molécule \myTRPZIPPER à l'échelle inter-moléculaire.
								Un \myglos{glo-Residu} à l'extrémité -- la molécule formant une chaîne -- est fixé afin d'\myemph{ancrer} la molécule au sein de l'environnement virtuel et éviter d'éventuelles dérives hors du champ visuel.
								L'intégralité des onze autres \myglos*{glo-Residu} est libre de mouvement.
								La forme général de la molécule peut être comparée à un \myform{V} : la chaîne de \myglos*{glo-Residu} de la molécule contient une cassure.
							\item[Scénario~\myscenario{1b}]
								Cette tâche concerne la manipulation de la molécule \myTRPCAGE à l'échelle inter-moléculaire.
								Comme le scénario~\myscenario{1a}, elle contient un \myglos{glo-Residu} fixe à une extrémité.
								L'intégralité des dix neuf autres \myglos*{glo-Residu} est libre de mouvement.
								La forme général de la molécule peut être comparée à un \myform{W} : la chaîne de \myglos*{glo-Residu} de la molécule contient deux cassures.
							\item[Scénario~\myscenario{2a}]
								Cette tâche concerne la manipulation de la molécule \myTRPZIPPER à l'échelle intra-moléculaire.
								Seulement trois \myglos*{glo-Residu} sont laissés libres tandis que tous les autres sont fixés.
								Les contraintes physiques de cette tâche sont relativement faibles.
								Cependant, la difficulté de cette tâche réside dans la recherche des \myglos*{glo-Residu} à déformer qui ne sont pas aisés à trouver.
							\item[Scénario~\myscenario{2b}]
								Cette tâche concerne la manipulation de la molécule \myTRPCAGE à l'échelle intra-moléculaire.
								Seulement six \myglos*{glo-Residu} sont laissés libres tandis que les autres sont fixés.
								La déformation requise demande une grande dépense d'énergie.
								En effet, la molécule proposée se trouve dans une sorte de puit de potentiel (un \myemph{minima} local) et l'objectif est d'atteindre un autre puit de potentiel (un autre \myemph{minima} local).
								L'énergie nécessaire pour passer d'un puit à l'autre est relativement importante, à tel point qu'un seul outil de déformation n'est pas suffisant.
								La manipulation synchrone de deux \myglos*{glo-Residu} est la seule solution pour atteindre l'objectif.
						\end{description}

						Un résumé de la complexité des quatre tâches est exposé dans la \myref{tab-exp2-ParametresDeComplexiteDesTaches} selon les critères suivants :
						\begin{description}
							\item[Nombre d'atomes] C'est le nombre total d'atomes que contient la molécule à manipuler;
							\item[\myGlosnl{glo-Residu} libre] C'est le nombre de \myglos*{glo-Residu} de la molécules non fixés dans la simulation;
							\item[Cassure] Ce sont les cassures de la chaîne principale de la molécule; elles représentent les jonctions entre \myhelice* et/ou les \myfeuillet*;
							\item[Champ de force] Il représente la difficulté en terme de contrainte physique; il exprime l'énergie minimum nécessaire pour atteindre l'objectif et se traduit par trois niveaux (\myemph{faible}, \myemph{moyen} et \myemph{fort}).
						\end{description}

						\begin{mytable}
							\mycaption[tab-exp2-ParametresDeComplexiteDesTaches]{Paramètres de complexité des tâches}
							\begin{mytabular}{^>{\bfseries}p{9em}-C-C-C-C}
								\mytoprule
								\myrowstyle{\bfseries}
								Scénario                    & \myscenario{1a} & \myscenario{1b} & \myscenario{2a} & \myscenario{2b} \\
								\mymiddlerule[\heavyrulewidth]
								Nombre d'atomes             & \mynum{218}     & \mynum{304}     & \mynum{218}     & \mynum{304}     \\
								\mymiddlerule
								\myGlosnl{glo-Residu} libre & \mynum{11}      & \mynum{19}      & \mynum{3}       & \mynum{7}       \\
								\mymiddlerule
								Cassure                     & \mynum{1}       & \mynum{2}       & \mynum{0}       & \mynum{1}       \\
								\mymiddlerule
								Champ de force              & Moyen           & Moyen           & Faible          & Fort            \\
								\mybottomrule
							\end{mytabular}
						\end{mytable}
					\end{mysubsubsection}
					\begin{mysubsubsection}[sss-exp2-LesOutilsDisponibles]{Les outils disponibles}
						Des outils de déformation légérement différents sont proposés en fonction de la tâche à réaliser.
						Pour les tâches de déformation au niveau inter-moléculaire, l'outil de déformation est l'outil \mytool{tug} pour les \myglos*{glo-Residu} : il permet de déformer d'un tenant l'intégralité d'un \myglos{glo-Residu}.
						Pour les tâches de déformation au niveau intra-moléculaire, l'outil de déformation est l'outil \mytool{tug} pour les atomes : il permet d'appliquer une force sur un seul atome.
						L'outil \mytool{tug} pour les \myglos*{glo-Residu} applique la même force à chaque atome du \myglos{glo-Residu}.
						Il en résulte que l'outil \mytool{tug} pour les \myglos*{glo-Residu} permet d'appliquer un effort total plus important et perturbe plus fortement la simulation.
					\end{mysubsubsection}
				\end{mysubsection}
				\begin{mysubsection}[sse-exp2-Procedure]{Procédure}
					L'expérimentation débute par une étape d'entraînement avec la molécule \myPrion.
					Pendant cette phase, les outils sont introduits et expliqués un par un.
					Cette phase dure entre \mynum[mn]{5} et \mynum[mn]{10}.
					Chaque sujet a la possibilité de tester les outils et peut questionner l'expérimentateur.

					Lorsque la phase d'entraînement est terminée, les sujets sont confrontées aux scénarios \myscenario{1a} et \myscenario{1b}.
					Les scénarios sont alternés entre les groupes de sujets afin d'éviter les biais d'apprentissage.
					L'application s'arrête automatiquement lorsque le seuil \myacro{acr-RMSD} désiré est atteint.

					Dès que les scénarios \myscenario{1a} et \myscenario{1b} ont été terminés par les sujets, les sujets sont confrontés aux scénarios \myscenario{2a} et \myscenario{2b} également de façon alternée.
					De la même façon, l'application s'arrête automatiquement lorsque le seuil \myacro{acr-RMSD} désiré est atteint.

					Tous les sujets sont confrontés aux quatre scénarios trois fois avec un jour d'intervalle entre chaque confrontation.
					L'objectif de cette multiple confrontation est l'étude de l'apprentissage.

					Un résumé du protocole expérimental est exprimé dans la \myref{tab-exp2-SyntheseDeLaProcedureExperimentale}.

					\begin{mytable}
						\mycaption[tab-exp2-SyntheseDeLaProcedureExperimentale]{Synthèse de la procédure expérimentale}
						\newcommand{\mytitlecolumn}[2]{%
							\multirow{#1}*{%
								\begin{minipage}{6em}%
									\raggedleft #2%
								\end{minipage}%
							}
						}
						\newlength{\exptwofirstcolumn}
						\newlength{\exptwosecondcolumn}
						\setlength{\exptwofirstcolumn}{7em}
						\setlength{\exptwosecondcolumn}{\textwidth}
						\addtolength{\exptwosecondcolumn}{-\exptwofirstcolumn}
						\addtolength{\exptwosecondcolumn}{-4\tabcolsep}
						\begin{mytabular}{>{\bfseries}p{\exptwofirstcolumn}p{\exptwosecondcolumn}}
							\mytoprule
							\mytitlecolumn{1}{Tâche}                  & Déformation d'une molécule                                                        \\
							\mymiddlerule[\heavyrulewidth]
							\mytitlecolumn{3}{Hypothèses}             & \myhypothesis{1} Amélioration des performances en \myglosnl{glo-Binome}           \\
							                                          & \myhypothesis{2} \myglosnl*{glo-Binome} plus performants sur les tâches complexes \\
							                                          & \myhypothesis{3} Apprentissage plus performant en \myglosnl{glo-Binome}           \\
							\mymiddlerule
							\mytitlecolumn{3}{Variable indépendantes} & \myvari{1} Nombre de sujets                                                       \\
							                                          & \myvari{2} Complexité de la tâche                                                 \\
							                                          & \myvari{3} Niveau d'apprentissage                                                 \\
							\mymiddlerule
							\mytitlecolumn{5}{Variable dépendantes}   & \myvard{1} Temps de réalisation                                                   \\
							                                          & \myvard{2} Nombre de sélections                                                   \\
							                                          & \myvard{3} Distance passive entre les espaces de travail                          \\
							                                          & \myvard{4} Distance active entre les espaces de travail                           \\
							                                          & \myvard{5} Vitesse moyenne                                                        \\
							                                          & \myvard{6} Réponses qualitatives                                                  \\
							\mymiddlerule[\heavyrulewidth]
							\multicolumn{2}{c}{%
								\begin{tabular}{^C-C-C-C}
									\myrowstyle{\bfseries}
									Condition \mycondition{1}           & Condition \mycondition{2}           & Condition \mycondition{3}     & Condition \mycondition{4} \\
									\mymiddlerule
									\myGlosnl{glo-Bimanuel} ($N=1$)     & \myGlosnl{glo-Bimanuel} ($N=1$)     & Collaboratif ($N=2$)          & Collaboratif ($N=2$)      \\
									\mymiddlerule
									Scénario~\myscenario{1a}            & Scénario~\myscenario{1b}            & Scénario~\myscenario{1a}      & Scénario~\myscenario{1b}  \\
									Scénario~\myscenario{1b}            & Scénario~\myscenario{1a}            & Scénario~\myscenario{1b}      & Scénario~\myscenario{1a}  \\
									Scénario~\myscenario{2a}            & Scénario~\myscenario{2b}            & Scénario~\myscenario{2a}      & Scénario~\myscenario{2b}  \\
									Scénario~\myscenario{2b}            & Scénario~\myscenario{2a}            & Scénario~\myscenario{2b}      & Scénario~\myscenario{2a}  \\
								\end{tabular}
							} \\
							\mybottomrule
						\end{mytabular}
					\end{mytable}
				\end{mysubsection}
			\end{mysection}
			\begin{mysection}[sec-exp2-Resultats]{Résultats}
				Cette section présente et analyse l'ensemble des mesures expérimentales de cette première étude concernant la recherche et la sélection sur une tâche complexe de collaboration.
				Les données, confrontées à un test de \mycite[author]{Shapiro-1965}, ne sont pas distribuées selon une loi normale.
				Cependant, un test de \mycite[author]{Brown-1974} permet de confirmer l'\myglos{glo-Homoscedasticite}.
				L'analyse de la variance est alors pratiquée avec différents tests statistiques suivant les cas :
				\begin{itemize}
					\item test de \mycite[author]{Friedman-1940} pour les \myglos*{glo-VariableIntraSujets} non-paramètriques;
					\item test de \mycite[author]{Kruskal-1952} pour les \myglos*{glo-VariableInterSujets} non-paramètriques.
				\end{itemize}
				\begin{mysubsection}[sse-exp2-AmeliorationDesPerformancesEnBinome]{Amélioration des performances en \myglosnl{glo-Binome}}
					\begin{myfigure}
						\psset{xunit=0.272108844\textwidth,yunit=0.02cm}
						\begin{myps}(-0.45,-55)(2,210)
							\myaxes(0,2){nombre de sujets}(0,200)[50]{temps~(s)}
							\myboxplot{exp2-time-group.csv}
						\end{myps}
						\mycaption[fig-exp2-TempsDeRealisationEnFonctionDuNombreDeSujets]{Temps de réalisation en fonction du nombre de sujets}
					\end{myfigure}

					La \myref{fig-exp2-TempsDeRealisationEnFonctionDuNombreDeSujets} présente le temps de réalisation \myvard{1} en fonction du nombre de sujets \myvari{1} ayant réalisé l'expérience.
					L'analyse montre qu'il y a un effet significatif du nombre de sujets \myvari{1} sur le temps de réalisation \myvard{1} (\myanova{exp2-time-group-anova.tex}).

					\begin{myfigure}
						\psset{xunit=0.272108844\textwidth,yunit=1.25cm}
						\begin{myps}(-0.45,-0.85)(2,3.5)
							\myaxes(0,2){distance}(0,3)[1]{distance~(mm)}
							\myboxplot{exp2-diff-activepassive-group.csv}
							\mylegend{\myleg{\myglosnl{glo-Monome}}{myblue}\myand\myleg{\myglosnl{glo-Binome}}{myblue!70}}
						\end{myps}
						\mycaption[fig-exp2-DistancePassiveEtActiveEntreLesEffecteursTerminauxEnFonctionDuNombreDeSujets]{Distance passive et active entre les \myglosnl*{glo-EffecteurTerminal} en fonction du nombre de sujets}
					\end{myfigure}

					La \myref{fig-exp2-DistancePassiveEtActiveEntreLesEffecteursTerminauxEnFonctionDuNombreDeSujets} présente la distance passive \myvard{3} et active \myvard{4} entre les \myglos*{glo-EffecteurTerminal} en fonction du nombre de sujets \myvari{1} ayant réalisé l'expérience.
					L'analyse montre qu'il n'y a pas d'effet significatif du nombre de sujets \myvari{1} sur la distance passive \myvard{3} (\myanova{exp2-diff-activepassive-group-anova-passive.tex}).
					De la même façon, l'analyse montre qu'il y a un effet significatif du nombre de sujets \myvari{1} sur la distance active \myvard{4} (\myanova{exp2-diff-activepassive-group-anova-active.tex}).

					On peut également observer l'évolution entre distance passive et active.
					L'analyse montre qu'il y a un effet significatif de la nature de la distance (passive ou active) au sein d'un \myglos{glo-Monome} (\myanova{exp2-diff-activepassive-group-anova-monome.tex}).
					Par contre, l'analyse ne montre pas d'effet significatif de la nature de la distance (passive ou active) au sein d'un \myglos{glo-Binome} (\myanova{exp2-diff-activepassive-group-anova-binome.tex}).

					\begin{myfigure}
						\psset{xunit=0.272108844\textwidth,yunit=0.075cm}
						\begin{myps}(-0.45,-15)(2,58)
							\myaxes(0,2){nombre de sujets}(0,50)[10]{nombre de sélections~(nb)}
							\myboxplot{exp2-numsel-group-dominant.csv}
							\mylegend{\myleg{main dominante}{myblue}\myand\myleg{main dominée}{myblue!70}}
						\end{myps}
						\mycaption[fig-exp2-NombreDeSelectionsDeLaMainDominanteEtDomineeEnFonctionDuNombreDeSujets]{Nombre de sélection de la main dominante et dominée en fonction du nombre de sujets}
					\end{myfigure}

					La \myref{fig-exp2-NombreDeSelectionsDeLaMainDominanteEtDomineeEnFonctionDuNombreDeSujets} présente le nombre de sélections de la main dominante et dominée \myvard{2} en fonction du nombre de sujets \myvari{1} ayant réalisé l'expérience.
					On constate un déséquilibre entre la main dominante et dominée des \myglos*{glo-Monome}.
					Les \myglos*{glo-Binome} n'utilisent que leur main dominante ce qui explique l'absence de valeur pour la main dominée.
					En cumulant le nombre total de sélections (main dominante et main dominée) pour les \myglos*{glo-Monome} et les \myglos*{glo-Binome}, on montre qu'il y a un effet significatif du nombre de sujets \myvari{1} sur le nombre de sélections \myvard{2} (\myanova{exp2-numsel-group-dominant-anova-cumulative.tex}).

					Il faut noter que le nombre de sélections pour la main dominante des \myglos*{glo-Binome} comptabilise les sélections des deux sujets présents contrairement aux \myglos*{glo-Monome} : ceci explique le nombre plus élevé de sélections en \myglos*{glo-Binome}.
					Cependant, si on compare le nombre moyen de sélections par sujet (pour sa main dominante), on montre qu'il n'y a pas d'effet significatif du nombre de sujets \myvari{1} sur le nombre de sélections de la main dominante \myvard{2} (\myanova{exp2-numsel-group-dominant-anova-dominant.tex}).

					\begin{myfigure}
						\psset{xunit=0.272108844\textwidth,yunit=2.5cm}
						\begin{myps}(-0.45,-0.45)(2,1.75)
							\myaxes(0,2){nombre de sujets}(0,1.5)[0.5]{speed~(mm/s)}
							\myboxplot{exp2-speed-group-dominant.csv}
							\mylegend{\myleg{main dominante}{myblue}\myand\myleg{main dominée}{myblue!70}}
						\end{myps}
						\mycaption[fig-exp2-VitesseMoyenneDeLaMainDominanteEtDomineeEnFonctionDuNombreDeSujets]{Vitesse moyenne de la main dominante et dominée en fonction du nombre de sujets}
					\end{myfigure}

					La \myref{fig-exp2-VitesseMoyenneDeLaMainDominanteEtDomineeEnFonctionDuNombreDeSujets} présente la vitesse moyenne \myvard{5} des \myglos*{glo-EffecteurTerminal} en fonction du nombre de sujets \myvari{1} ayant réalisé l'expérience.
					L'analyse montre un effet significatif du nombre de sujets \myvari{1} sur la vitesse moyenne \myvard{5} (\myanova{exp2-speed-group-dominant-anova.tex}).
					On observe un déséquilibre de vitesse moyenne entre la main dominante et dominée des \myglos*{glo-Monome} avec un effet significatif (\myanova{exp2-speed-group-dominant-anova-monome.tex}).
					On observe également un effet significatif du nombre de sujets \myvari{1} sur la vitesse moyenne \myvard{5} de la main dominante (\myanova{exp2-speed-group-dominant-anova-dominant.tex}).

					Le premier résultat sur la \myref{fig-exp2-TempsDeRealisationEnFonctionDuNombreDeSujets} nous permet de confirmer notre hypothèse \myhypothesis{1} : les \myglos*{glo-Binome} sont plus performants que les \myglos*{glo-Monome}.
					Cependant, la suite de l'analyse va permettre de mettre en avant les paramètres précis pour lesquels il y a un gain de performances ainsi que les scénarios les plus favorables.

					Pour commencer, l'observation des distances moyennes entre les \myglos*{glo-EffecteurTerminal} nous permet d'observer un désequilibre de performances entre  les \myglos*{glo-Monome} et les \myglos*{glo-Binome} \myref*{fig-exp2-DistancePassiveEtActiveEntreLesEffecteursTerminauxEnFonctionDuNombreDeSujets}.
					En effet, lorsqu'on mesure cette distance de façon continue sur toute l'expérimentation, on constate une distance moyenne plus importante chez les \myglos*{glo-Monome}.
					Cependant, la mesure de cette distance seulement dans le cas où les deux \myglos*{glo-EffecteurTerminal} sont effectivement actifs nous montre que ce sont les \myglos*{glo-Binome} qui couvre le plus grand espace de travail.
					En effet, les \myglos*{glo-Monome} sont en manipulation \myglos*{glo-Bimanuel} ce qui est une configuration complexe sur le plan cognitif.
					Le sujet doit alors être capable de gérer deux \myglos*{glo-EffecteurTerminal} à chaque instant.
					La difficulté de cette configuration a mené la plupart des sujets à utiliser seulement un \myglos{glo-EffecteurTerminal} en laissant le second sur le côté afin que le curseur ne gêne pas à l'écran.
					Ceci a pour effet d'augmenter la distance moyenne bien que le second \myglos{glo-EffecteurTerminal} ne soit pas utilisé.

					La distance active permet d'éviter ce biais de mesure.
					Cette mesure ne concerne alors que les \myglos*{glo-Monome} capables d'effectuer la tâche en utilisant leurs deux mains.
					On constate alors que les \myglos*{glo-Binome} sont capables de couvrir un plus grand espace de travail.
					En effet, les \myglos*{glo-Monome} ne sont pas capables de couvrir un espace aussi grand car ils ne peuvent focaliser visuellement que sur une zone de travail à la fois.
					Par conséquent, les deux \myglos*{glo-EffecteurTerminal} se retrouvent dans cette zone limitée dans l'espace.

					La \myref{fig-exp2-NombreDeSelectionsDeLaMainDominanteEtDomineeEnFonctionDuNombreDeSujets} confirme ce déséquilibre.
					En effet, on constate un plus grand nombre total de sélections pour les \myglos*{glo-Binome} (\myanova{exp2-numsel-group-dominant-mean-binome.tex}~sélections) que pour les \myglos*{glo-Monome} (\myanova{exp2-numsel-group-dominant-mean-monome.tex}~sélections).
					Là encore, le sujet effectuant la tâche en \myglos{glo-Monome} n'est pas capable d'exploiter pleinement les deux outils en sa possession : la charge cognitive est trop importante.
					Cependant, les analyses statistiques montre que l'outil utilisé par la main dominante obtient un taux d'utilisation identique en \myglos{glo-Monome} ou en \myglos{glo-Binome}.
					Le sujet en \myglos{glo-Monome} n'est pas capable d'exploiter pleinement les possibilités des deux outils en configuration \myglos*{glo-Bimanuel}.
					Par opposition, les \myglos*{glo-Binome} en configuration \myglos*{glo-Monomanuel} répartissent correctement la charge de travail entre les deux outils.

					On constate que la configuration \myglos*{glo-Bimanuel} a également un effet néfaste sur l'utilisation de l'outil associé à la main dominante.
					En effet, l'analyse montre une différence significative entre la vitesse moyenne de la main dominante des \myglos*{glo-Monome} et des \myglos*{glo-Binome}.
					La configuration \myglos*{glo-Bimanuel} provoque une séquentialité dans les actions du sujet : il manipule avec un outil, puis avec l'autre mais rarement les deux en même temps.
					Cette séquentialité a pour effet des pauses alternatives entre les outils ce qui explique cette vitesse moyenne plus basse.

					Cette section nous a permis de constater que le travail en \myglos{glo-Binome} permet de meilleures performances que le travail en \myglos{glo-Monome}.
					Une analyse plus détaillé a mis en avant la difficulté du travail en configuration \myglos*{glo-Bimanuel} : la charge de travail cognitive à assumer avec deux outils est trop importante.
					Cette difficulté a pour effet de fortement dégrader le taux d'utilisation d'un des deux outils.
					On constate également une légére dégradation de l'utilisation de l'outil associé à la main dominante.
					Pour résumer, il est préférable de partager les ressources disponibles (outils de manipulation dans notre cas) entre différents participants.
				\end{mysubsection}
				\begin{mysubsection}[sse-exp2-GainDePerformanceSurLesTachesComplexes]{Gain de performance sur les tâches complexes}
					\begin{myfigure}
						\psset{xunit=0.222222222\textwidth,yunit=0.015cm}
						\begin{myps}(-0.5,-75)(4,290)
							\myaxes(0,4){scénario}(0,250)[50]{temps~(s)}
							\myboxplot{exp2-time-task.csv}
							\mylegend{\myleg{\myglosnl{glo-Monome}}{myblue}\myand\myleg{\myglosnl{glo-Binome}}{myblue!70}}
						\end{myps}
						\mycaption[fig-exp2-TempsDeRealisationDesScenarios]{Temps de réalisation des scénarios}
					\end{myfigure}

					La \myref{fig-exp2-TempsDeRealisationDesScenarios} présente le temps de réalisation \myvard{1} des différents scénarios \myvari{2}.
					L'analyse montre un effet significatif des scénarios \myvari{2} sur le temps de réalisation \myvard{1} (\myanova{exp2-time-task-anova.tex}).
					Un test post-hoc de \mycite[author]{Mann-1947} avec une correction de \mycite[author]{Holm-1979} permet de trier les scénarios en deux classes de complexité : $\left\{\myscenario{1a}, \myscenario{2a}\right\}$ et $\left\{\myscenario{1b}, \myscenario{2b}\right\}$.

					\begin{myfigure}
						\psset{xunit=0.222222222\textwidth,yunit=0.015cm}
						\begin{myps}(-0.5,-75)(4,290)
							\myaxes(0,4){scénario}(0,250)[50]{temps~(s)}
							\myboxplot{exp2-time-task-group.csv}
							\mylegend{\myleg{\myglosnl{glo-Monome}}{myblue}\myand\myleg{\myglosnl{glo-Binome}}{myblue!70}}
						\end{myps}
						\mycaption[fig-exp2-TempsDeRealisationDesScenariosEnFonctionDuNombreDeSujets]{Temps de réalisation des scénarios en fonction du nombre de sujets}
					\end{myfigure}

					La \myref{fig-exp2-TempsDeRealisationDesScenariosEnFonctionDuNombreDeSujets} présente le temps de réalisation \myvard{1} des différents scénarios \myvari{2} en fonction du nombre de sujets \myvari{1} ayant réalisé l'expérience.
					Les analyses suivantes regroupent les scénarios par classe de complexité : \myscenario{1a} avec \myscenario{2a} et \myscenario{1b} avec \myscenario{2b}.
					L'analyse montre qu'il n'y a pas d'effet significatif du nombre de sujets \myvari{1} sur le temps de réalisation \myvard{1} pour les scénarios \myscenario{1a} et \myscenario{2a} (\myanova{exp2-time-task-group-anova-trpzipper.tex}).
					Cependant, l'analyse montre un effet significatif du nombre de sujets \myvari{1} sur le temps de réalisation \myvard{1} pour les scénarios \myscenario{1b} et \myscenario{2b} (\myanova{exp2-time-task-group-anova-trpcage.tex}).
				\end{mysubsection}
				\begin{mysubsection}[sse-exp2-AmeliorationDeLApprentissagePourLesBinomes]{Amélioration de l'apprentissage pour les \myglosnl*{glo-Binome}}
				\end{mysubsection}
			\end{mysection}
			\begin{mysection}[sec-exp2-Synthese]{Synthèse}
			\end{mysection}
		\end{mychapter}
		\begin{mychapter}[cha-LesDynamiquesDeGroupe]{Les dynamiques de groupe}
		\end{mychapter}
	\end{mypart}
	\begin{mypart}[prt-PropositionsPourLeTravailCollaboratif]{Propositions pour le travail collaboratif}
		\begin{mychapter}[cha-TravailCollaboratifAssisteParHaptique]{Travail collaboratif assisté par haptique}
			\begin{mysection}[sec-exp4-Presentation]{Présentation}
				\begin{mysubsection}[sse-exp4-Objectifs]{Objectifs}
					Cette dernière expérimentation aura pour objectif d'introduire et de valider des outils de communication haptique dans le cadre d'une tâche d'\myglos{glo-AmarrageMoleculaire}.
					Sur la base des précédentes expérimentations, des outils haptiques censés améliorer les interactions et les communications entre les manipulateurs sont proposés.
					L'expérimentation testera l'intérêt et l'apport de ces outils sur la collaboration de groupe.

					Le principal facteur observé sera les performances du groupe.
					Les performances regardées seront le temps mis pour achever la tâche mais également la qualité de la solution trouvée.
					En effet, la qualité de la solution est une variable non-négligeable dans le cadre d'une tâche d'\myglos{glo-AmarrageMoleculaire}.

					Le second facteur concernera l'évaluation qualitative du système par les utilisateurs.
					Il est primordial de recueillir l'avis des utilisateurs en ce qui concerne une plate-forme de travail.
					Des outils haptiques inconfortables, des détails visuels incohérents, des interactions peu intuitives sont autant de paramètres qui peuvent rendre un système inefficace.
				\end{mysubsection}
				\begin{mysubsection}[sse-exp4-Hypotheses]{Hypothèses}
					\begin{myparagraph}[par-exp4-AmeliorationDesPerformancesAvecLAssistanceHaptique]{\myhypothesis{1} Amélioration des performances avec l'assistance haptique}
						La première hypothèse est une amélioration des performances liée à l'utilisation des assistances haptiques proposées à travers des outils.
						Le temps de réalisation de la tâche et la qualité de la solution proposée par les sujets seront les \myglos*{glo-VariableDependante} principales pour observer cette amélioration des performances.
					\end{myparagraph}
				\end{mysubsection}
			\end{mysection}
			\begin{mysection}[sec-exp4-DispositifExperimentalEtMateriel]{Dispositif expérimental et matériel}
				L'\myacro{acr-EVC} utilisé est illustré sur la \myref{fig-exp4-IllustrationDuDispositifExperimental}.
				L'\myacro{acr-EVC} propose une visualisation partagée sur grand écran (vue publique à tous les utilisateurs) à l'aide d'un vidéoprojecteur.
				Les \mytodo{3 ou 4}{Quelle configuration choisissons nous ?} sujets font face à l'écran avec à leur disposition :
				\begin{itemize}
					\item une interface d'orientation de la scène;
					\item une interface haptique de manipulation de la molécule;
					\item une interface haptique pour la coordination;
					\item deux interfaces haptiques de déformation \mytool{tug}.
				\end{itemize}

				\begin{myTodo}{Images à compléter}{Il va falloir créer la scène Blender correspondante et faire des photos du dispositif expérimental}
					\begin{myfigure}
						\begin{mysubfigure}
							%\myimage[width=0.49\textwidth]{exp4-schema}
							\mysubcaption[fig-exp4-IllustrationDuDispositifExperimental-SchemaDuDispositifExperimental]{Schéma du dispositif expérimental}
						\end{mysubfigure}
						\begin{mysubfigure}
							%\myimage[width=0.49\textwidth]{exp4-photo}
							\mysubcaption[fig-exp4-IllustrationDuDispositifExperimental-PhotographieDuDispositifExperimental]{Photographie du dispositif expérimental}
						\end{mysubfigure}
						\mycaption[fig-exp4-IllustrationDuDispositifExperimental]{Illustration du dispositif expérimental}
					\end{myfigure}
				\end{myTodo}

				Une caméra vidéo de marque \mySony (\textsc{hdr-sr11e}) sera installée afin de filmer l'expérimentation.
				L'écran de vidéo-projection ainsi que les sujets (de dos) sont dans le plan de la vidéo.

				Pour les détails techniques concernant la plate-forme et les outils de manipulation et de déformation, se reporter au \myref{cha-ShaddockSystemeCollaboratifPourLaManipulationDeMolecules}.

			\end{mysection}
			\begin{mysection}[sec-exp4-Methode]{Méthode}
				\begin{mysubsection}[sse-exp4-Sujets]{Sujets}
					\begin{myTodo}{Nombre de sujets}{Remplir toutes les informations statistiques concernant les sujets}
						\mynum{000}~sujets (\mynum{000}~femmes et \mynum{000}~hommes) avec une moyenne d'âge de $\mu = 00.0$ ($\sigma = 0.00$) ont participés à cette expérimentation.
					\end{myTodo}
					Ils ont été recrutés au sein du laboratoire \myacro{acr-IBPC} et sont chercheurs en biologie moléculaire.
					Ils ont tous le français comme langue principale.
					Aucun participant n'a de déficience visuelle (ou corrigée le cas échéant) ni de déficience audio.

					Chaque participant est complètement naïf concernant les détails de l'expérimentation.
					Une explication détaillée de la procédure expérimentale leur est donnée au commencement de l'expérimentation mais en omettant l'objectif de l'étude.
				\end{mysubsection}
				\begin{mysubsection}[sec-exp4-Variables]{Variables}
					\begin{mysubsubsection}[sss-exp4-VariablesIndependantes]{Variables indépendantes}
						\begin{myparagraph}[par-exp4-PresenceDeLAssistance]{\myvari{1} Présence de l'assistance}
							La première \myglos{glo-VariableIndependante} est une \myglos{glo-VariableIntraSujets}, c'est-à-dire que tous les sujets sont expérimentés dans toutes les modalités de cette variable.
							\myvari{1} possède deux valeurs possibles : \og sans assistance \fg ou \og avec assistance \fg.
							L'assistance est l'aide haptique ajouté aux différents outils de manipulation, de désignation et de déformation afin d'améliorer l'intéraction et la communication entre les sujets pendant la tâche.
						\end{myparagraph}
						\begin{myparagraph}[par-exp4-ComplexeDeMoleculesAAssembler]{\myvari{2} Complexe de molécules à assembler}
							La seconde \myglos{glo-VariableIndependante} est une \myglos{glo-VariableIntraSujets}.
							\myvari{2} concerne les complexes de molécules à assembler : \og \myNusE \fg.
						\end{myparagraph}
					\end{mysubsubsection}
					\begin{mysubsubsection}[sec-exp4-VariablesDependantes]{Variables dépendantes}
						\begin{myparagraph}[par-exp4-LeTempsDeRealisation]{\myvard{1} Le temps de réalisation}
							Ce temps est le temps total pour réaliser la tâche demandée, c'est-à-dire manipuler et déformer la molécule afin d'atteindre l'objectif fixé.
							Il n'y a pas de limite de temps pour réaliser la tâche.
						\end{myparagraph}
						\begin{myparagraph}[par-exp4-LeNombreDeSelections]{\myvard{2} Le nombre de sélections}
							\myvard{2} représente le nombre de sélections réalisées durant chaque tâche à réaliser.
							Une sélection est comptabilisée lorsque un atome ou un \myglos{glo-Residu} est sélectionné par un des deux \myglos{glo-EffecteurTerminal}.
							Un compteur est affecté pour chacun des \myglos*{glo-EffecteurTerminal}.
						\end{myparagraph}
						\begin{myparagraph}[par-exp4-LesCommunicationsVerbalesEtGestuelles]{\myvard{3} Les communications verbales et gestuelles}
							L'enregistrement vidéo permet de mesurer la quantité de temps de parole pendant chaque tâche de l'expérimentation.
							Elle permet également d'observer les phases de communication gestuelle.
							Les communications gestuelles sont les mouvements physiques des sujets destinés à donner une information à un ou plusieurs autres sujets.
						\end{myparagraph}
						\begin{myparagraph}[par-exp4-EvaluationQualitativeDuSysteme]{\myvard{4} Évaluation qualitative du système}
							Un questionnaire est proposé à tous les sujets.
							Il est constitué de plusieurs questions (notées sur échelle de \mycite[author]{Likert-1932} à cinq niveaux).

							\begin{myTodo}{Le questionnaire}{Écrire le questionnaire soumis au sujets}
								Le questionnaire est le suivant (les questions sont posées à chaque sujet dans le cas du \myglos{glo-Binome}) :
								\begin{enumerate}
									\item Quelle note donneriez-vous\dots{}
										\begin{enumerate}
											\item au système interactif que vous venez de tester ?
											\item aux effets visuels offerts par le système ?
											\item aux outils proposés ?
										\end{enumerate}
									\item Quelle configuration avez-vous préféré : \myemph{sans assistance} ou \myemph{avec assistance} ?
									\item Vous êtes vous senti utile dans le groupe ?
									\item Pensez-vous avoir une position de meneur dans la configuration collaborative ?
									\item Comment évalueriez-vous votre taux de communication\dots{}
										\begin{itemize}
											\item verbale ?
											\item gestuelle ?
											\item virtuelle ?
										\end{itemize}
								\end{enumerate}
							\end{myTodo}

							Concernant la communication, les communications verbales concernent tous les échanges, dialogues exposés par la voix.
							La communication gestuelle représente les gestes que les sujets peuvent effectuer dans le monde réel pour expliquer, désigner ou pour tout autre explication à son partenaire.
							Enfin, la communication virtuelle concerne les informations données au partenaire par l'intermédiaire de l'environnement virtuel (par exemple, une désignation avec le curseur).
						\end{myparagraph}
					\end{mysubsubsection}
				\end{mysubsection}
				\begin{mysubsection}[sse-exp4-Tache]{Tâche}
					La tâche proposée est la déformation dans un \myacro{acr-EVC} sur des complexes de molécules : c'est une tâche d'\myglos{glo-AmarrageMoleculaire} simplifié.

					\begin{mysubsubsection}[sss-exp4-DescriptionDeLaTache]{Description de la tâche}
						La tâche proposée est la déformation d'un complexe de molécules afin d'obtenir le meilleur score énergétique possible.
						L'intégralité des atomes de la molécule à déformer est affiché de façon discrète en transparence.
						De plus, un \myemph{ruban} de cette molécule est affiché.

						Afin de réaliser la tâche, différentes mesures sont disponibles en temps-réel pour les sujets.
						La première de ces mesures est le score \myacro{acr-RMSD} qui est décrit dans la \myref{sss-exp2-DescriptionDeLaTache}.
						La seconde mesure est l'énergie totale du système, valeur calculée par \myacro{acr-NAMD}.

						Le premier complexe de molécules proposé, couramment nommé \myNusE \mycite{Burmann-2010}, a pour identifiant \myPDB \myPDBlink{http://www.rcsb.org/pdb/explore/explore.do?structureId=2KVQ}{2KVQ}.
						Il est constitué de deux molécules \textsc{NusE} et \textsc{NusG} possédant respectivement \mynum{1294}~atomes et \mynum{929}~atomes.
					\end{mysubsubsection}
					\begin{mysubsubsection}[sss-exp4-LesOutilsDisponibles]{Les outils disponibles}
						\begin{myTodo}{Deux configurations possibles}{Les deux configurations font intervenir trois ou quatre sujets; il faut enlever un des 2 paragraphes}
							\begin{myparagraph}[par-exp4-ConfigurationATroisSujets]{Configuration à trois sujets}
								Les sujets effectueront l'expérimentation par groupe de trois utilisateurs.
								Ils ont la possibilité de communiquer entre eux sans restriction.
								Un des sujets aura le rôle du \myemph{coordinateur} avec des outils haptiques différents des deux autres sujets qui auront des rôles d'\myemph{opérateurs}.

								Le \myemph{coordinateur} pourra déplacer et orienter la molécule à l'aide de deux outils:
								\begin{itemize}
									\item un outil haptique attaché virtuellement à la molécule permettant de déplacer la molécule;
									\item un \mySpaceNavigator permettant d'orienter la molécule.
								\end{itemize}
								De plus, le \myemph{coordinateur} aura à sa disposition un outil de désignation complexe permettant de :
								\begin{itemize}
									\item afficher ou masquer les \myglos*{glo-Residu} en fonction de leur intérêt pour la tâche en cours de réalisation;
									\item désigner un \myglos{glo-Residu} nécessitant une manipulation par les \myemph{opérateurs}.
								\end{itemize}
							\end{myparagraph}
							\begin{myparagraph}[par-exp4-ConfigurationAQuatreSujets]{Configuration à quatre sujets}
								Les sujets effectueront l'expérimentation par groupe de quatre utilisateurs.
								Ils ont la possibilité de communiquer entre eux sans restriction.

								Un des sujets aura la gestion du déplacement de la molécule et du point de vue de l'application; il sera le \myemph{manipulateur}.
								Un deuxième sujet aura le rôle du \myemph{coordinateur}.
								Enfin, les deux derniers sujets seront les \myemph{opérateurs}.

								Le \myemph{manipulateur} aura à sa disposition deux outils.
								Le premier outil est une interface haptique, attachée virtuellement à la molécule permettant de déplacer la molécule.
								Le second outil est un \mySpaceNavigator permettant d'orienter la molécule.

								Le \myemph{coordinateur} aura à sa disposition un outil de désignation complexe permettant de :
								\begin{itemize}
									\item afficher ou masquer les \myglos*{glo-Residu} en fonction de leur intérêt pour la tâche en cours de réalisation;
									\item désigner un \myglos{glo-Residu} nécessitant une manipulation par les \myemph{opérateurs}.
								\end{itemize}
							\end{myparagraph}
						\end{myTodo}

						Les \myemph{opérateurs} auront chacun à leur disposition un outil haptique permettant de :
						\begin{itemize}
							\item déplacer les atomes de la molécule afin de la déformer;
							\item désigner un \myglos{glo-Residu} nécessitant une manipulation par l'autre \myemph{opérateur}.
						\end{itemize}
					\end{mysubsubsection}
				\end{mysubsection}
				\begin{mysubsection}[sse-exp4-Procedure]{Procédure}
					Pour débuter cette expérimentation, les sujets sont confrontés à un exemple sur la molécule \mytodo{à déterminer}{Il faudra trouver une molécule d'apprentissage}.
					Pendant la phase d'apprentissage, les outils sont introduits et expliqués un par un.
					Chaque sujet a la possibilité de tester les outils et peut questionner l'expérimentateur.

					Dès que la phase d'apprentissage est terminée, l'enregistrement vidéo démarre.
					Un premier complexe est proposé aux sujets.
					\mytodo{Dix minutes}{C'est une proposition de temps mais il faudra peut-être adapter en fonction des tests alpha} sont laissées pour réaliser la tâche.
					Si les sujets estiment avoir obtenu le meilleur score possible avant la durée limite, ils peuvent décider d'arrêter la tâche.

					Lorsque toutes les tâches sont réalisées, les sujets sont soumis au questionnaire.
					Chaque sujet est tenu de répondre au questionnaire seul, sans communiquer avec les autres sujets.

					Un résumé du protocole expérimental est exprimé dans la \myref{tab-exp4-SyntheseDeLaProcedureExperimentale}.

					\begin{mytable}
						\mycaption[tab-exp4-SyntheseDeLaProcedureExperimentale]{Synthèse de la procédure expérimentale}
						\newcommand{\mytitlecolumn}[2]{%
							\multirow{#1}*{%
								\begin{minipage}{6em}%
									\raggedleft #2%
								\end{minipage}%
							}
						}
						\newlength{\expfourfirstcolumn}
						\newlength{\expfoursecondcolumn}
						\setlength{\expfourfirstcolumn}{7em}
						\setlength{\expfoursecondcolumn}{\textwidth}
						\addtolength{\expfoursecondcolumn}{-\expfourfirstcolumn}
						\addtolength{\expfoursecondcolumn}{-4\tabcolsep}
						\begin{mytabular}{>{\bfseries}p{\expfourfirstcolumn}p{\expfoursecondcolumn}}
							\mytoprule
							\mytitlecolumn{1}{Tâche}                  & Amarrage d'un complexe de molécule                                        \\
							\mymiddlerule[\heavyrulewidth]
							\mytitlecolumn{1}{Hypothèses}             & \myhypothesis{1} Amélioration de performance par assistance haptique      \\
							\mymiddlerule
							\mytitlecolumn{2}{Variable indépendantes} & \myvari{1} Présence de l'assistance                                       \\
							                                          & \myvari{2} Complexe de molécules à assembler                              \\
							\mymiddlerule
							\mytitlecolumn{4}{Variable dépendantes}   & \myvard{1} Temps de réalisation                                           \\
							                                          & \myvard{2} Nombre de sélections                                           \\
							                                          & \myvard{3} Communications verbales et gestuelles                          \\
							                                          & \myvard{4} Évaluation qualitative du système                              \\
							\mymiddlerule[\heavyrulewidth]
							\multicolumn{2}{c}{%
								\begin{tabular}{^C-C}
									\myrowstyle{\bfseries}
									Condition \mycondition{1} & Condition \mycondition{2} \\
									\mymiddlerule
									Sans assistance           & Avec assistance           \\
									\mymiddlerule
									\myNusE                   & \myNusE                   \\
								\end{tabular}
							} \\
							\mybottomrule
						\end{mytabular}
					\end{mytable}
				\end{mysubsection}
			\end{mysection}
		\end{mychapter}
	\end{mypart}
	\begin{mypart}[prt-Synthese]{Synthèse}
		\begin{mychapter}[cha-ConclusionEtPerspectives]{Conclusion et perspectives}
		\end{mychapter}
	\end{mypart}

	\myglossary
	\myappendix
	\begin{mychapter}[cha-Questionnaires]{Questionnaires}
		\begin{mysection}[sec-Questionnaires-PremiereExperimentation]{Première expérimentation}
			Le questionnaire proposé durant cette expérimentation est constitué de deux parties.
			La deuxième partie est exclusivement réservée aux \myglos*{glo-Binome} et n'était pas proposée au \myglos*{glo-Monome}.
			Ce questionnaire contient \mynum[pages]{5} (\mynum[pages]{3} pour les \myglos*{glo-Monome}).
			Les questions sont évaluées selon une échelle de \mycite[author]{Likert-1932} à cinq niveaux.
			\myinsertpdf[pdfpages={1-5}]{exp1-questionnary}
		\end{mysection}
		\begin{mysection}[sec-Questionnaires-SecondeExperimentation]{Seconde expérimentation}
			Le questionnaire proposé durant la seconde expérimentation est décliné en deux versions : une version pour les \myglos*{glo-Monome} et une version pour les \myglos*{glo-Binome}.
			Le questionnaire est soumis aux sujets oralement par l'expérimentateur et les réponses sont directement reportées dans une tableau.
			Il est constitué de plusieurs questions notées sur échelle de \mycite[author]{Likert-1932} à cinq niveaux.
			\begin{mysubsection}[sec-Questionnaires-SecondeExperimentation-QuestionnairePourLesMonomes]{Questionnaire pour les \myglosnl*{glo-Monome}}
				Pour les \myglos*{glo-Monome}, le questionnaire est le suivant :
				\begin{enumerate}
					\item Vous êtes-vous senti efficace ?
					\item Pensez-vous que vous auriez été plus à l'aise seul avec un seul outil de déformation ?
					\item Pensez-vous que vous auriez été plus à l'aise avec un partenaire ?
					\item Quelle solution choisiriez-vous entre les trois configurations ?
				\end{enumerate}
			\end{mysubsection}
			\begin{mysubsection}[sec-Questionnaires-SecondeExperimentation-QuestionnairePourLesBinomes]{Questionnaire pour les \myglosnl*{glo-Binome}}
				Chaque sujet dans un \myglos{glo-Binome} est interrogé séparement pour éviter que les réponses de l'un influence les réponses de l'autre.
				Pour les \myglos*{glo-Binome}, le questionnaire est le suivant :
				\begin{enumerate}
					\item Vous êtes-vous senti efficace ?
					\item Comment évalueriez-vous votre taux de communication\dots{}
						\begin{itemize}
							\item verbale ?
							\item gestuelle ?
							\item virtuelle ?
						\end{itemize}
					\item Vous sentez-vous utile dans le groupe (par opposition à pénalisant) ?
					\item Pensez-vous avoir une position de meneur dans le groupe ?
					\item Pensez-vous que vous auriez été plus à l'aise seul avec votre outil de déformation ?
					\item Pensez-vous que vous auriez été plus à l'aise seul avec deux outils de déformation ?
					\item Quelle solution choisiriez-vous entre les trois configurations ?
				\end{enumerate}

				Concernant les taux de communication, les communications verbales concernent tous les échanges, dialogues exposés par la voix.
				La communication gestuelle représente les gestes que les sujets peuvent effectuer dans le monde réel pour expliquer, désigner ou pour tout autre explication à son partenaire.
				Enfin, la communication virtuelle concerne les informations données au partenaire par l'intermédiaire de l'environnement virtuel (par exemple, une désignation avec le curseur).
			\end{mysubsection}
		\end{mysection}
	\end{mychapter}
\end{document}
