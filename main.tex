\documentclass[myfrancais]{mythesis}
\usepackage{mydate}
\usepackage{mytodo}
\usepackage{graphicx}
\usepackage{caption}
\newcommand{\name}[1]{\textsc{#1}}
\newcommand{\acro}[1]{\textsc{#1}}
\newcommand{\mynum}[1]{\nombre{#1}}
\newcommand{\LIMSI}{\acro{CNRS--LIMSI}\xspace}

\hypersetup{%
	pdftitle={Interactions haptiques collaboratives pour la manipulation moléculaire},%
	pdfauthor={Jean SIMARD},%
	pdfsubject={Mémoire de thèse en informatique},%
	pdfdisplaydoctitle=true,%
	pdflang={FR-fr}%
}

\title{Interactions haptiques collaboratives pour la manipulation moléculaire}
\author{Jean~\name{Simard}}
\documenttype{Thèse en Informatique}
\university{École Doctorale d'Informatique de Paris Sud}
\date{\mydate[datestyle=long]{01/12/2011}}
\jury{%
	Martin & \name{DUPONT} & (rapporteur) & Directeur de recherche au \LIMSI \\
	Martin & \name{DUPOND} & (examinateur) & Directeur de recherche au \LIMSI}
	\addglobalbib[datatype=bibtex]{biblio.bib}
\begin{document}
	\frontmatter
	\maketitle
	\mytoc
	\mylof
	\mylot
	\mylotodo
	\mainmatter
	\part{Le sujet}
	\chapter{Introduction}
	\part{Étude du travail collaboratif}
	\chapter{La recherche collaborative}
	\section{Présentation}
	\subsection{Objectifs}
	\subsection{Hypothèses}
	\section{Dispositif expérimental}
	\section{Méthode}
	\subsection{Sujets}
	\mynum{24}~sujets (\mynum{4}~femmes et \mynum{20}~hommes) avec une moyenne d'âge de $\mu = 27.8$ ($\sigma = 7.19$) ont participés à cette expérimentation.
	Ils ont tous été recrutés au sein du laboratoire \LIMSI et sont chercheurs ou assistants de recherche dans les domaines suivants~:
	\begin{itemize}
		\item linguistique et traitement automatique de la parole;
		\item réalité virtuelle et système immersifs;
		\item audio-acoustique.
	\end{itemize}
	Ils ont tous le français comme langue principale.
	Aucun participant n'a de déficience visuelle (ou corrigée le cas échéant) ni déficience audio.
	
	Chaque participants est complètement naïf concernant les détails de l'expérimentation.
	Une explication détaillée de la procédure expérimentale leur est donnée au commencement de l'expérimentation mais en omettant l'objectif de l'étude.

	\subsection{Variables}
	\subsubsection{Variables indépendantes}
	\subsubsection{Variables dépendantes}
	\subsection{Tâche}
	\subsection{Procédure}
	\section{Résultats}
	\chapter{La manipulation collaborative}
	\chapter{Les dynamiques de groupe}
	\part{Propositions pour le travail collaboratif}
	\chapter{Travail collaboratif assisté par haptique}
	\part{Synthèse}
	\chapter{Conclusion et perspectives}

	\appendix
	\chapter{\acro{Shaddock} -- Collaborative Virtual Environment for Molecular Design}
\end{document}
