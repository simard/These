\documentclass[myfrancais]{mythesis}
\usepackage{mydate}
\usepackage{mytodo}
\usepackage{mycolor}
\usepackage{myps}
\usepackage{myuml}
\usepackage{mymacro}
\usepackage{mypdf}
\usepackage{mysource}

\makeatletter
% Modify the bibliography style
\newcounter{mymaxcitenames}
\AtBeginDocument{%
	\setcounter{mymaxcitenames}{\value{maxnames}}%
}
\renewbibmacro{begentry}{%
	\printtext[brackets]{%
		\defcounter{maxnames}{\value{mymaxcitenames}}%
		\printnames{labelname}~\usebibmacro{cite:labelyear+extrayear}%
	}%
	\newline%
}
% AAA
\newcommand{\myACER}{\textsc{acer}\xspace}
\newcommand{\myAlanine}{Alanine\xspace}
\newcommand{\myanalysis}[1]{\input{files/#1}\%}
\newcommand{\myangstrom}{\AA ngström\xspace}
\newcommand{\myanova}[1]{\input{files/#1}}
\newcommand{\myatom}[2][]{%
	{%
		\ifstrempty{#1}%
			{\makefirstuc{\textsf{#2}}}%
			{\textcolor{#1}{\makefirstuc{\textsf{#2}}}}%
		\xspace%
	}%
}
\newcommand{\myAudacity}{\textsc{audacity}\myregistered}% No '\xspace' because of already one in '\myregistered'
% CCC
\newcommand{\mycarbon}{\myatom[mycarboncolor]{C}}
\newcommand{\myCasioXJ}{\textsc{Casio xj}\xspace}
\newcommand{\myCHARMM}{\textsc{charmm}\xspace}
\newcommand{\myChimera}{\textsc{chimera}\xspace}
\newcommand{\myClayWorks}{\textsc{Clayworks}\xspace}
\newcommand{\mycondition}[1]{$\left(\mathcal{C}_{#1}\right)$\xspace}
\newcommand{\myCPK}{\textsc{cpk}\xspace}
% DDD
\newcommand{\myDesktop}{\myPHANToM Desktop\myregistered}% No '\xspace' because of already one in '\myregistered'
% FFF
\newcommand{\myfeuillet}{feuillet-$\beta$\xspace}
\WithSuffix\newcommand\myfeuillet*{feuillets-$\beta$\xspace}
\newcommand{\myform}[1]{\textbf{\sffamily\MakeUppercase{#1}}}
% GGG
\newcommand{\myGhost}{\textsc{Ghost}\xspace}
\newcommand{\myGromacs}{\textsc{Gromacs}\xspace}
\newcommand{\mygroup}[1]{$\left(\mathcal{G}_{#1}\right)$\xspace}
% HHH
\newcommand{\myHaption}{\textsc{Haption}\xspace}
\newcommand{\myHawthorne}{\myemph{Hawthorne Works}\xspace}
\newcommand{\myHBonds}{\textit{HBonds}\xspace}
\newcommand{\myhelice}{hélice-$\alpha$\xspace}
\WithSuffix\newcommand\myhelice*{hélices-$\alpha$\xspace}
\newcommand{\myhypothesis}[1]{$\left(\mathcal{H}_{#1}\right)$\xspace}
% III
\newcommand{\myIntelCore}{Intel\myregistered Core\mytrademark~2 \textsc{q9450} (\mynum[GHz]{2.66})\xspace}
% JJJ
\newcommand{\myJmol}{\textsc{Jmol}\xspace}
% LLL
\newcommand{\myLCD}{\textsc{lcd}\xspace}
\newcommand{\myLicorice}{\textit{Licorice}\xspace}
\newcommand{\myLinux}{\textsc{Linux}\xspace}
% MMM
\newcommand{\myMacOS}{Mac~\textsc{OS}\xspace}
\newcommand{\myMDDriver}{\textsc{MDDriver}\xspace}
% NNN
\newcommand{\myNewRibbon}{\textit{NewRibbon}\xspace}
\def\mynode{%
	\@ifnextchar[{\mynode@i}{\mynode@i[style=nodestyle]}%
}
\def\mynode@i[#1](#2,#3)[#4]#5{%
	\rput(#2,#3){\Rnode{#4}{\psframebox[style=nodestyle,#1]{\vphantom{pÉ}#5}}}%
}
\newcommand{\mynytrogen}{\myatom[mynytrogencolor]{A}}
\newcommand{\myNusE}{\textsc{NusE}\xspace}
\newcommand{\myNusENusG}{\textsc{NusE:NusG}\xspace}
\newcommand{\myNusG}{\textsc{NusG}\xspace}
% OOO
\newcommand{\myOmni}{\myPHANToM Omni\myregistered}% No '\xspace' because of already one in '\myregistered'
\newcommand{\myOpenHaptics}{\textsc{OpenHaptics}\mytrademark}% No '\xspace' because of already one in '\mytrademark'
\newcommand{\myoxygen}{\myatom[myoxygencolor]{O}}
% PPP
\newcommand{\myPC}{\textsc{pc}\xspace}
\newcommand{\myPDB}{\textsc{pdb}\xspace}
\newcommand{\myPDBbase}{\emph{Protein~DataBase}\xspace}
\newcommand{\myPDBlink}[2]{\href{#1}{\textsc{\MakeLowercase{#2}}}}
\newcommand{\myPHANToM}{\textsc{phant}o\textsc{m}\xspace}
\newcommand{\myPremium}{\myPHANToM Premium\myregistered}% No '\xspace' because of already one in '\myregistered'
\newcommand{\myPrion}{Prion\xspace}
\newcommand{\myPSF}{\textsc{psf}\xspace}
\newcommand{\mypvalue}{$p$-value\xspace}
\newcommand{\myPyMOL}{\textsc{p}y\textsc{mol}\xspace}
% RRR
\newcommand{\myRAM}[2][Go]{\mynum[#1]{#2} de \textsc{ram}}
\newcommand{\myRasmol}{\textsc{RasMol}\xspace}
\newcommand{\myresidue}[1]{$\left(\mathcal{R}_{#1}\right)$\xspace}
% SSS
\newcommand{\myscenario}[1]{\textsc{#1}}
\newcommand{\mySensAble}{\textsc{SensAble}\xspace}
\newcommand{\myShaddock}{\textsc{Shaddock}\xspace}
\newcommand{\mySony}{\textsc{sony}\myregistered}% No '\xspace' because of already one in '\myregistered'
\newcommand{\mySpaceNavigator}{SpaceNavigator\myregistered}% No '\xspace' because of already one in '\myregistered'
\newcommand{\mysubject}[1]{$\mathcal{S}_#1$}
\newcommand{\mysulfur}{\myatom[mysulfurcolor]{S}}
\newcommand{\mysummary}[1]{\input{files/#1}}
% TTT
\newcommand{\myTCPIP}{\textsc{tcp/ip}\xspace}
\newcommand{\myThreeD}{\textsc{3d}\xspace}
\newcommand{\mytool}[1]{\myemph{#1}}
\newcommand{\myTRPCAGE}{\textsc{trp-cage}\xspace}
\newcommand{\myTRPZIPPER}{\textsc{trp-zipper}\xspace}
% UUU
\newcommand{\myUbiquitin}{Ubiquitin\xspace}
\newcommand{\myUbuntu}{\textsc{Ubuntu}~v$10.04$\xspace}
\newcommand{\myUSB}{\textsc{usb}\xspace}
\newcommand{\myuser}[1]{$\mathcal{#1}$}
% VVV
\newcommand{\myvar}[2]{$\left(\mathcal{V}_{\mathrm{#1}#2}\right)$\xspace}
\newcommand{\myvard}[1]{\myvar{d}{#1}}
\newcommand{\myvari}[1]{\myvar{i}{#1}}
\newcommand{\myVGA}{\textsc{vga}\xspace}
\newcommand{\myVirtuose}{\textsc{Virtuose}\mytrademark~\textsc{6d}\mynum{35}--\mynum{45}\xspace}
% WWW
\newcommand{\myWindows}{\textsc{Windows}\xspace}

% Needed lengths
\newlength{\mywidth}
\newlength{\myheight}

% PSTricks style
\newpsstyle{nodestyle}{framearc=0.25,shadow=true,shadowcolor=myblue,blur=true}
\makeatother

\NeedsTeXFormat{LaTeX2e}[1999/01/01]
\ProvidesPackage{mycolor}[2011/04/29]

%%%%%%%%%%%%%%%%%%%%%%%%%%%%%
%% Declare package options %%
%%%%%%%%%%%%%%%%%%%%%%%%%%%%%
% In case of unknown options
\DeclareOption*{%
	\PackageWarning{mycolor}{Unknown option `\CurrentOption'}%
}

\ProcessOptions

%% Options to pass to packages

%% Packages to call
\RequirePackage{xcolor}

%%%%%%%%%%%%%%%%%%%%%%%
%% New configuration %%
%%%%%%%%%%%%%%%%%%%%%%%
\definecolor{mydarkred}{rgb}{0.5265625 0.25 0.25}
\definecolor{myred}{rgb}{0.7265625 0 0}
\definecolor{mylightred}{rgb}{0.9265625 0.5 0.5}
\definecolor{mylightestred}{rgb}{1 0.66 0.66}
\definecolor{mydarkblue}{rgb}{0 0 0.2265625}
\definecolor{myblue}{rgb}{0 0 0.7265625}
\definecolor{mylightblue}{rgb}{0.500 0.500 0.9265625}
\definecolor{mylightestblue}{rgb}{0.7500 0.7500 0.9265625}
\definecolor{mygreen}{rgb}{0 0.7265625 0}
\definecolor{mylightgreen}{rgb}{0.500 0.9265625 0.500}
\definecolor{mylightestgreen}{rgb}{0.7500 0.9265625 0.7500}
\definecolor{mygray}{gray}{0.6666667}

%%%%%%%%%%%%%%%%%%
%% New commands %%
%%%%%%%%%%%%%%%%%%

% End of package
\endinput

% AAA
\mynewacro{acr-AFM}%
{%
	name={\textsc{afm}},
	first={microscope à force atomique (\textsc{afm} pour \myemph{Atomic Force Microscope})},%
	plural={\textsc{afm}s},%
	firstplural={microscopes à force atomique (\textsc{afm} pour \myemph{Atomic Force Microscope})},%
	description={Microscope permettant l'observation de la topologie de la surface d'un échantillon au niveau atomique}
}
\mynewacro{acr-API}%
{%
	name={\textsc{api}},%
	first={interface de programmation (\textsc{api})},%
	plural={\textsc{api}s},%
	firstplural={interfaces de programmation (\textsc{api}s)},%
	description={\textsc{api} vient de l'anglais \myemph{Application Programming Interface} et désigne une interface avec un programme informatique}%
}
% BBB
% Be careful, because this word has no plural form, but the femina word in plural form
\mynewglos{glo-Bimanuel}%
{%
	name={bimanuel},%
	description={Qui se fait avec les deux mains},%
	plural={bimanuelle}%
}
\mynewglos{glo-Binome}%
{%
	name={binôme},%
	description={Groupe constitué de \mynum{2}~personnes},%
	plural={binômes}%
}
% CCC
\mynewacro{acr-CAO}%
{%
	name={\textsc{cao}},%
	first={conception assistée par ordinateur (\textsc{cao})},%
	description={La \textsc{cao} permet de concevoir et de tester virtuellement, à l'aide d'outils informatique, des produits manufacturés}%
}
\mynewglos{glo-ConflitDeCoordination}%
{%
	name={conflit de coordination},%
	description={Conflit entre deux sujets qui peut survenir lorsque les deux sujets tente d'accéder ou de déformer un objet au même instant},%
	plural={conflits de coordination}%
}
\mynewacro{acr-CUDA}%
{%
	name={\textsc{cuda}},%
	first={\textsc{cuda} (\myemph{Compute Unified Device Architecture})},%
	description={Technologie permettant d'utiliser l'unité graphique d'un ordinateur pour effectuer des calculs à hautes performances}%
}
\mynewglos{glo-Curseur}%
{%
	name={curseur},%
	description={Élément virtuel associé à un élément physique que le sujet manipule; il est lié à l'\myglos{glo-EffecteurTerminal}},%
	plural={curseurs}%
}
% DDD
\mynewacro{acr-DDL}%
{%
	name={\textsc{ddl}},%
	first={degré de liberté (\textsc{ddl})},%
	plural={\textsc{ddl}s},%
	firstplural={degrés de liberté (\textsc{ddl}s)},%
	description={Mouvements relatifs indépendants d'un solide par rapport à un autre}%
}
\mynewglos{glo-DockingMoleculaire}%
{%
	name={\myemph{docking} moléculaire},%
	description={Méthode permettant de déterminer l'orientation et la déformation optimale de \mynum{2}~molécules afin qu'elle s'assemble pour former un complexe de molécules stable},%
	plural={\myemph{docking} moléculaires}%
}
% EEE
\mynewglos{glo-EffecteurTerminal}%
{%
	name={effecteur terminal},%
	description={Élément physique que le sujet manipule; il est lié au \myglos{glo-Curseur} du monde virtuel},%
	plural={effecteurs terminaux}%
}
\mynewacro{acr-EVC}%
{%
	name={\textsc{evc}},%
	first={Environnement Virtuel Collaboratif (\textsc{evc})},%
	firstplural={Environnements Virtuels Collaboratifs (\textsc{evc})},%
	description={Ensemble logiciel et matériel permettant de faire interagir plusieurs utilisateurs au sein d'un même environnement; ils jouent un rôle important dans le développement de nouvelles méthodes de travail collaboratives}%
}
% HHH
\mynewglos{glo-Homoscedasticite}%
{%
	name={homoscedasticité},%
	description={Équivalent à homogénéité des variances; permet de comparer des variables aléatoires possédant des variances similaires},%
	plural={homoscedasticités}%
}
% III
\mynewacro{acr-IBPC}%
{%
	name={\textsc{ibpc}},%
	first={Institut de Biologie Physico-Chimie (\textsc{ibpc})},%
	description={Institut de recherche, géré par la fédération de recherche \textsc{frc}~\mynum{550}, étudiant les bases structurales, génétiques et physico-chimiques à leur différents niveaux d'intégration}%
}
\mynewacro{acr-IMD}%
{%
	name={\textsc{imd}},%
	first={\textsc{imd} (\myemph{Interactive Molecular Dynamics})},%
	description={Programme permettant de connecter le logiciel de visualisation moléculaire \myacro-{acr-VMD} avec le logiciel de simulation \myacro-{acr-NAMD} pour une simulation interactive en temps-réel \mycite{Stadler-1997}}%
}
\mynewacro{acr-ITAP}%
{%
	name={\textsc{itap}},%
	first={\myemph{Institut für Theoretische und Angewandte Physik} (\textsc{itap})},%
	description={Institut de Physique Théorique et Appliquée de \myname{Stuttgart} à l'origine du développement du logiciel \myacro{acr-IMD}}%
}
% LLL
\mynewacro{acr-LIMSI}%
{%
	name={\textsc{cnrs--limsi}},%
	first={Laboratoire pour l'Informatique, la Mécanique et les Sciences de l'Ingénieur (\textsc{cnrs--limsi})},%
	description={Unité Propre de Recherche du \textsc{cnrs} (\textsc{upr}~3251) associé aux universités \textsc{Paris} Sud et Pierre et Marie \textsc{Curie}}%
}
% MMM
\mynewglos{glo-Meneur}%
{%
	name={meneur},%
	description={En anglais \myemph{leader}, personne qui dirige un groupe afin d'atteindre des objectifs communs à ce groupe; c'est celui qui prend les décisions (voir aussi \myglos{glo-Suiveur})},%
	plural={meneurs}%
}
\mynewglos{glo-Monomanuel}%
{%
	name={monomanuel},%
	description={Qui se fait avec une main},%
	plural={monomanuelle}%
}
\mynewglos{glo-Monome}%
{%
	name={monôme},%
	description={\myemph{Groupe} constitué d'une unique personne},%
	plural={monômes}%
}
\mynewglos{glo-MotivationSociale}%
{%
	name={motivation sociale},%
	description={En anglais \myemph{social facilitation} \mycite{Triplett-1900}, phénomène de groupe où les personnes fournissent plus d'efforts grâce à la présence de partenaires},%
	plural={motivation sociale}%
}
\mynewacro{acr-TRM}%
{%
	name={\textsc{trm}},
	first={Théorie des Ressources Multiples (\textsc{trm})},%
	description={Cette théorie, élaborée par \mycite[author]{Wickens-1984} (\textsc{mrt} pour \myemph{Multiple Resource Theory}), propose un modèle pour la gestion des charges de travail pour un humain}
}
% NNN
\mynewacro{acr-NAMD}%
{%
	name={\textsc{namd}},%
	first={\textsc{namd} (\myemph{Scalable Molecular Dynamics})},%
	description={Programme de simulation pour la dynamique moléculaire \mycite{Phillips-2005}}%
}
% PPP
\mynewglos{glo-ParesseSociale}%
{%
	name={paresse sociale},%
	description={En anglais \myemph{social loafing} \mycite{Ringelmann-1913}, phénomène de groupe où les personnes fournissent moins d'effort pour la réalisation d'une tâche que s'ils effectuaient la tâche seuls},%
	plural={paresse sociale}%
}
\mynewacro{acr-PCV}%
{%
	name={\textsc{pcv}},
	first={Primitive Comportementale Virtuelle (\textsc{pcv})},%
	plural={\textsc{pcv}s},%
	firstplural={Primitives Comportementales Virtuelles (\textsc{pcv}s)},%
	description={Dans une application de réalité virtuelle, les activités d'un sujet peuvent toujours être décomposées en quatre comportements de base, appelés \myacro+{acr-PCV}, qui sont : observer, se déplacer, agir et communiquer \mycite{Fuchs-2006}}
}
% QQQ
\mynewglos{glo-Quadrinome}%
{%
	name={quadrinôme},%
	description={Groupe constitué de \mynum{4}~personnes},%
	plural={quadrinômes}%
}
% RRR
\mynewglos{glo-Residu}%
{%
	name={résidu},%
	description={Groupe d'atomes constituant un des blocs élémentaires d'une molécule},%
	plural={résidus}%
}
\mynewacro{acr-RMSD}%
{%
	name={\textsc{rmsd}},%
	first={\myemph{Root Mean Square Deviation} (\textsc{rmsd})},%
	description={Appelé Écart Quadratique Moyen en français, il permet -- dans le cadre de la biologie moléculaire -- de mesurer la différence entre deux déformations d'une même molécule}%
}
% SSS
\mynewglos{glo-StructureInformelle}%
{%
	name={structure informelle},%
	description={Groupe de personnes sans structures ni hiérarchie},%
	plural={structures informelles}%
}
\mynewglos{glo-Suiveur}%
{%
	name={suiveur},%
	description={En anglais \myemph{follower}, personne qui se laisse diriger dans un groupe afin d'atteindre des objectifs communs à ce groupe; c'est une personne qui ne prend pas de décision (voir aussi \myglos{glo-Meneur})},%
	plural={suiveurs}%
}
\mynewacro{acr-SUS}%
{%
	name={\textsc{sus}},%
	first={\textsc{sus} (\myemph{System Usability Scale})},%
	description={Échelle de notation entre \mynum{0} et \mynum{100} proposée par \mycite[author]{Brooke-1996} permettant d'évaluer l'utilisabilité d'un système}%
}
% TTT
\mynewglos{glo-Tetranome}%
{%
	name={tetranôme},%
	description={Groupe constitué de \mynum{4}~personnes},%
	plural={tetranômes}%
}
\mynewglos{glo-Trinome}%
{%
	name={trinôme},%
	description={Groupe constitué de \mynum{3}~personnes},%
	plural={trinômes}%
}
% UUU
\mynewacro{acr-UDP}%
{%
	name={\textsc{udp}},
	first={\textsc{udp} (\myemph{User Datagram Protocol} pour protocole de datagramme utilisateur)},%
	plural={\textsc{afm}s},%
	firstplural={\textsc{udp} (\myemph{User Datagram Protocol} pour protocole de datagramme utilisateur)},%
	description={c'est un des principaux protocole de télécommunication sur internet ; il a pour distinction de ne pas vérifier l'intégrité des données transmises}
}
\mynewacro{acr-UML}%
{%
	name={\textsc{uml}},%
	first={\textsc{uml} (\myemph{Unified Modeling Language})},%
	description={C'est un langage graphique de modélisation utilisé principalement en génie logiciel}%
}
% VVV
\mynewglos{glo-VariableDependante}%
{%
	name={variable dépendante},%
	description={Facteur mesuré sur une expérimentation (nombre de sélections, trajectoire, \myetc); ces variables sont influencées par les \myglos*{glo-VariableIndependante}},%
	plural={variables dépendantes}%
}
\mynewglos{glo-VariableIndependante}%
{%
	name={variable indépendante},%
	description={Facteur pouvant varier et être manipuler sur une expérimentation (nombre de participants, tâche, \myetc); ces variables vont avoir une incidence sur les \myglos*{glo-VariableDependante}},%
	plural={variables indépendantes}%
}
\mynewglos{glo-VariableInterSujets}%
{%
	name={variable inter-sujets},%
	description={Variables pour lesquelles les sujets sont confrontés à une et une seule des modalités de la variable},%
	plural={variables inter-sujets}%
}
\mynewglos{glo-VariableIntraSujets}%
{%
	name={variable intra-sujets},%
	description={Variables pour lesquelles les sujets sont confrontés à toutes les modalités de la variable},%
	plural={variables intra-sujets}%
}
\mynewacro{acr-VMD}%
{%
	name={\textsc{vmd}},%
	first={\textsc{vmd} (\myemph{Visual Molecular Dynamics})},%
	description={Programme de visualisation moléculaire \mycite{Humphrey-1996}}%
}
\mynewacro{acr-VRPN}%
{%
	name={\textsc{vrpn}},%
	first={\textsc{vrpn} (\myemph{Virtual Reality Protocol Network})},%
	description={Logiciel permettant de connecter différents périphériques de réalité virtuelle à une même application sous forme d'une architecture client/serveur \mycite{Taylor-II-2001}}%
}


\hypersetup{%
	pdftitle={Interactions haptiques collaboratives pour la manipulation moléculaire},%
	pdfauthor={Jean SIMARD},%
	pdfkeywords={collaboration,haptique,environnement virtuel,simulation moléculaire},%
	pdflang={FR-fr},%
	pdfsubject={Mémoire de thèse en informatique}%
}
\addglobalbib[datatype=bibtex]{biblio.bib}

\title{Interactions haptiques collaboratives pour la manipulation moléculaire}
\author{Jean~\myname{Simard}}
\documenttype{Thèse en Informatique}
\university{École Doctorale d'Informatique de Paris Sud}
\date{\mydate[datestyle=long]{01/12/2011}}
\jury{%
	Martin & \myname{DUPONT} & (rapporteur) & Directeur de recherche au \myacro-{acr-LIMSI} \\
	Martin & \myname{DUPOND} & (examinateur) & Directeur de recherche au \myacro-{acr-LIMSI}}

\begin{document}
	\frontmatter
	\maketitle
	\mytoc
	\mylof
	\mylot
	\begin{mychapter+}{Résumé}
		\myemph{<Background>}\par
		\myemph{<Ma thèse>}\par
		\myemph{<Développement de la plate-forme>}\par
		\myemph{<Études de cas>}\par
		\myemph{<Proposition de solution>}\par
		\myemph{<Synthèse et perspectives>}\par
	\end{mychapter+}
	\mainmatter
	\begin{mychapter}[cha-LeSujet]{Le sujet}
		\begin{mysection}[sec-EtatDeLArt]{État de l'art}
			La distinction entre travail coopératif et travail collaboratif est expliquée par \mycite[author]{Roschelle-1995}.
			\begin{quote}
				\it Cooperative work is accomplished by the division of labour among participants, as an activity where each person is responsible for a portion of the problem solving.
				We focus on collaboration as the mutual engagement of participants in a coordinated effort to solve the problem together.
			\end{quote}
			traduit dans \mycite{Knauf-2010} en français
			\begin{quote}
				Le travail coopératif implique une division du travail entre les participants, chaque participant étant responsable d’une partie du problème à résoudre.
				Dans la collaboration, les participants s’engagent tous dans les mêmes tâches, en se coordonnant, afin de résoudre le problème ensemble.
			\end{quote}
			Le travail collaboratif s'est développé rapidement avec l'informatique.
			On peut citer les outils de gestions de versions tels que \mycite[author]{Git-2011}, \mycite[author]{Mercurial-2011} ou encore \mycite[author]{SVN-2011} qui permettent la collaboration de plusieurs développeurs pour la création ou la modification de programmes informatiques.
			Puis \myInternet a donné naissance à des outils de collaboration tels que les \myWiki* \mycite{Leuf-2001,Wagner-2004} avec le succès de \myWikipedia que l'on connaît.
			Cependant, ces deux exemples proposent une collaboration asynchrone où chaque acteur intervient derrière son prédécesseur pour ajouter, améliorer ou corriger une partie du travail.
			Aujourd'hui, les outils permettant la collaboration synchrone sont encore rares (systèmes de vidéoconférences, jeux vidéos en ligne).
		\end{mysection}
		\begin{mysection}[sec-Contexte]{Contexte}
			\begin{mysubsection}[sse-LAmarrageMoleculaire]{L'\myglosnl{glo-AmarrageMoleculaire}}
				Le contexte de l'expérimentation est l'\myglos{glo-AmarrageMoleculaire} plus communément nommé \myglos{glo-DockingMoleculaire}.
				Ce processus implique une analyse et une manipulation complexe reposant sur plusieurs expertises.
				Il est basé sur une décomposition en trois niveaux de modélisation, traités du niveau le plus grossier au niveau le plus fin :
				\begin{description}
					\item[Niveau inter-moléculaire] Cette déformation au niveau macro-moléculaire applique des transformations de grande amplitude sur chaque molécule.
						L'objectif est de trouver la meilleure concordance entre les molécule en terme de position et d'orientation.
					\item[Niveau intra-moléculaire] Cette déformation au niveau moléculaire fait suite à la déformation inter-moléculaire.
						L'amarrage de ces deux molécules (ou plus) introduit de nombreuses interfaces qui doivent être optimisées en fonction de critères variés (la complémentarité des surfaces, les forces électrostatiques, les forces de \myname[van der]{Waals} \mycite{Muller-1994}, \myetc).
					\item[Niveau atomique] Cette déformation très fine va chercher à optimiser la position des atomes au niveau de l'interface.
						L'intérêt de cette étape sera portée sur plusieurs types d'interaction (les ponts hydrogènes, les zones hydrophobiques et hydrophylliques, les ponts salins, \myetc).
				\end{description}

				Pour chacun de ces différents niveaux, le processus de manipulation est similaire et peut être séparé en tâches élémentaires \myref*{fig-ProcessusDeDeformationMoleculaireEnQuatreEtapes} :
				\begin{description}
					\item[Recherche] Cette tâche concerne l'identification et la recherche d'une cible (atome, \myglos{glo-Residu}, \myhelice*, \myfeuillet*, \myetc) en fonction de critères multiples (articulations, bilan énergétique, régions hydrophobique, \myetc).
					\item[Sélection] Une fois la cible trouvée, la tâche consiste à accéder puis à sélectionner la cible par l'intermédiaire d'un périphérique d'entrée (une souris, une interface haptique, \myetc).
					\item[Déformation] La tâche consiste à déformer la structure en manipulant la cible précédemment sélectionnée, que ce soit au niveau inter-moléculaire, intra-moléculaire ou atomique.
						L'objectif inhérent à cette tâche et d'atteindre l'objectif fixé (par exemple, minimiser l'énergie totale du système).
					\item[Évaluation] Cette dernière partie va évaluer le travail précédemment réalisé en observant différents indicateurs (énergie potentielle, énergie électrostatique, complémentarité des surfaces, \myetc).
						En fonction de la synthèse des résultats de cette dernière phase, un nouveau cycle pourra recommencer (recherche, sélection, déformation, évaluation, \myetc).
				\end{description}

				\begin{myfigure}
					\psset{xunit=0.2\textwidth}
					\def\mycirclenum(#1,#2)#3{%
						\uput{5em}[0](#1,#2){\pscirclebox*[fillcolor=myblue!70]{\white #3}}%
					}
					\begin{myps}(-2.5,-1)(2.5,5)
						\mynode(0,4)[Search]{Recherche}
						\mycirclenum(0,4){1}
						\mynode(0,3)[Selection]{Sélection}
						\mycirclenum(0,3){2}
						\mynode(0,2)[Manipulation]{Manipulation}
						\mycirclenum(0,2){3}
						\mynode(0,1)[Evaluation]{Évaluation}
						\mycirclenum(0,1){4}
						\mynode[fillstyle=solid,fillcolor=myblue!25](0,-0.5)[Objective]{Objectif atteint}
						\ncline{->}{Search}{Selection}
						\ncline{->}{Selection}{Manipulation}
						\ncline{->}{Manipulation}{Evaluation}
						\ncline{->}{Evaluation}{Objective}
						\ncloop[loopsize=4em,angleA=-90,angleB=90,linearc=0.05]{->}{Evaluation}{Search}
					\end{myps}
					\mycaption[fig-ProcessusDeDeformationMoleculaireEnQuatreEtapes]{Processus de déformation moléculaire en quatre étapes}
				\end{myfigure}
			\end{mysubsection}
		\end{mysection}
	\end{mychapter}
	\begin{mychapter}[cha-Shaddock-ManipulationCollaborativeDeMolecules]{\myShaddock\ -- Manipulation collaborative de molécules}
		\begin{mysection}[sec-Shaddock-Introduction]{Introduction}
			Le \myref{cha-LeSujet} nous a permis d'identifier des problématiques de recherche.
			C'est autour de ces problématiques que la plate-forme \myShaddock a été élaborée.

			Nous commencerons par présenter les choix de matériels et d'architecture logicielle \myref*{sec-Shaddock-ArchitectureMaterielleEtLogicielle}.
			Certaines propriétés particulières sont nécessaires pour le choix de l'interface haptique permettant la manipulation interactive; elles seront détaillées dans la \myref{sse-Shaddock-LInterfaceHaptique}.
			L'ensemble des éléments de la plate-forme sont organisés selon un architecture client/serveur; les raisons de ce choix sont expliquées dans la \myref{sse-Shaddock-UneArchitectureClientServeur}.

			Ensuite, la plate-forme de simulation moléculaire en temps-réel est présentée \myref*{sec-Shaddock-PlateFormeDeSimulationEtDeVisualisation}.
			Tout d'abord, un logiciel de visualisation complet est nécessaire pour obtenir des affichages détaillés et complexes de molécules; le logiciel est présenté dans la \myref{sse-Shaddock-LogicielDeVisualisationMoleculaire}.
			Puis un logiciel de simulation est utilisé afin d'obtenir un comportement physique réaliste de la molécule; les différentes solutions existantes ainsi que le logiciel retenu sont présentés dans la \myref{sse-Shaddock-LogicielDeSimulationMoleculaire}.
			Enfin, un module spécifique permettant d'obtenir des simulations moléculaires en temps-réel est présenté dans la \myref{sss-Shaddock-PlateFormeDeSimulationMoleculaireEnTempsReel}.

			Le logiciel de visualisation moléculaire utilisé propose des premiers outils d'interaction avec les molécules.
			Ces outils sont présentés dans la \myref{sse-Shaddock-LesOutilsExistants}.
			Cependant, les outils existants n'ont pas été suffisants dans certains cas.
			De plus, notre étude du travail collaboratif a mené à la proposition de nouveaux outils haptiques présentés dans la \myref{sse-Shaddock-LesNouveauxOutilsDInteraction}.

			Les différents éléments de cette plate-forme sont résumés dans deux diagrammes \myacro{acr-UML}.
			Un diagramme de déploiement \myacro{acr-UML} de la plate-forme \myShaddock est présenté sur la \myref{fig-Shaddock-DiagrammeDeDeploiementUMLDeLaPlateFormeShaddock}.
			L'application \myacro{acr-VMD} est détaillée dans un diagramme de composant \myacro{acr-UML} sur la \myref{fig-Shaddock-DiagrammeDeComposantUMLDuNoeudVMD}.

			\begin{myfigure}
				\psset{xunit=0.1\textwidth,yunit=0.0475\textheight}
				\psset{framearc=.1,shadow=true,blur=true}
				\begin{myps}(-5,-10)(5,10)
					\rotateleft{%
						\rput(0,0){%
							\myumlnode*<PCUtilisateur>[\myLinux et \myacronl-{acr-CUDA}]{Client : \myPC utilisateur}{%
								\myumlcomponent<VMD>[application]{\myacronl-{acr-VMD}}%
							}%
						}
						\rput(-5,0){%
							\myumlnode*<ServeurNAMD>[\myLinux et \myacronl-{acr-CUDA}]{Serveur : Serveur \myacronl-{acr-NAMD}}{%
								\begin{psmatrix}[rowsep=1]%
									\myumlcomponent<NAMD>[executable]{{\mysource{namd2}}} \\%
									\myumlcomponent<FichierSimulation>[fichier]{%
										\\[-1ex]%
											\begin{psmatrix}[rowsep=0]%
												Données de\\ simulation%
											\end{psmatrix}%
										}
									\end{psmatrix}%
								}%
							}
							\rput(5,4){%
								\myumlnode*<ServeurVRPN1>[\myLinux, \myMacOS ou \myWindows]{Serveur~\mynum{1} : Serveur \myacronl-{acr-VRPN}}{%
									\myumlcomponent<VRPN1>[executable]{{\mysource{vrpn_server}}}%
								}%
							}
							\rput(5,1){%
								\myumlnode<PHANToM1>[\myOmni]{Interface~1 : Interface haptique}%
							}
							\rput(5,-0.5){\Huge$\vdots$}
							\rput(5,-3){%
								\myumlnode*<ServeurVRPNn>[\myLinux, \myMacOS ou \myWindows]{Serveur~$n$ : Serveur \myacronl-{acr-VRPN}}{%
									\myumlcomponent<VRPNn>[executable]{{\mysource{vrpn_server}}}%
								}%
							}
							\rput(5,-6){%
								\myumlnode<PHANToMn>[\myOmni]{Interface~$n$ : Interface haptique}%
							}
							\rput(0,-5){%
								\myumlnode<VideoProjecteur>[vue partagée]{Affichage : Vidéoprojecteur}
							}
							\psset{shadow=false}

							\myumlrealization[angleA=-90,angleB=90]{NAMD}{FichierSimulation}[nccurve]%
							\myumlrealization[angleA=-90,angleB=90]{VRPN1}{PHANToM1}[nccurve]%
							\myumlrealization[angleA=-90,angleB=90]{VRPNn}{PHANToMn}[nccurve]%
							\myumlinterface[angleA=-90,angleB=90,ArrowInsidePos=0.666667]{VMD}{VideoProjecteur}[nccurve]
							\nbput[npos=0.666667]{\tiny\begin{psmatrix}[rowsep=0]Transmission\\ des données\\ d'affichage\end{psmatrix}}
							\ncput*[framesep=1pt,nrot=:D,npos=0.333333]{\textcolor{black!30}{\scriptsize \myVGA}}
							\myumlinterface[angleA=0,angleB=180,offsetB=-8pt,ArrowInsidePos=0.3]{NAMD}{VMD}[nccurve]
							\naput[npos=0.25]{\tiny\begin{psmatrix}[rowsep=0]Transmission\\ des données\\ de simulation\end{psmatrix}}
							\ncput*[framesep=1pt,nrot=:U,npos=0.55]{\textcolor{black!30}{\scriptsize \myTCPIP}}
							\myumlinterface[angleA=180,angleB=0,ncurvA=1.5,offsetA=8pt,ArrowInsidePos=0.5]{VRPN1}{VMD}[nccurve]
							\nbput[npos=0.4]{\tiny\begin{psmatrix}[rowsep=0]Transmission\\ des données\\ d'interaction\end{psmatrix}}
							\ncput*[framesep=1pt,nrot=:D,npos=0.6]{\textcolor{black!30}{\scriptsize \myTCPIP}}
							\myumlinterface[angleA=180,angleB=0,ncurvA=1.5,offsetA=8pt,ArrowInsidePos=0.5]{VRPNn}{VMD}[nccurve]
							\naput[npos=0.4]{\tiny\begin{psmatrix}[rowsep=0]Transmission\\ des données\\ d'interaction\end{psmatrix}}
							\ncput*[framesep=1pt,nrot=:D,npos=0.6]{\textcolor{black!30}{\scriptsize \myTCPIP}}
						}
					\end{myps}
					\mycaption[fig-Shaddock-DiagrammeDeDeploiementUMLDeLaPlateFormeShaddock]{Diagramme de déploiement \myacronl-{acr-UML} de la plate-forme \myShaddock}
			\end{myfigure}
			\begin{myfigure}
					\psset{unit=0.05\textwidth}
					\begin{myps}(-10,-5)(10,4)
						\rput(0,0){%
							\myumlcomponent*[framesep=10pt,framearc=0,shadow=false]<VMD>[application]{\myacronl-{acr-VMD}}{%
								\psset{framesep=5pt,framearc=.1,shadow=true,blur=true}%
								\psframebox[linestyle=none,fillstyle=none,shadow=false]{%
									\begin{psmatrix}[rowsep=1]%
										\myumlcomponent<IMD>[extension]{\myacronl-{acr-IMD}}%
										\hspace{3em}%
										\myumlcomponent<VRPNclient>[fonction]{Client \myacronl-{acr-VRPN}} \\%
										\myumlcomponent<Renderer>[fonction]{Moteur de rendus}%
									\end{psmatrix}%
								}%
							}%
						}

						\psset{fillstyle=none,shadow=false}
						\myumlrealization[angleA=90,angleB=-90]{Renderer}{IMD}[nccurve]%
						\myumlrelation[angleA=180,angleB=0,offsetA=8pt]{VRPNclient}[-135]<*>{IMD}[-45]<*>[nccurve]%
						\myumlinterface[angleA=180,angleB=180,outAngleB=0,offsetA=8pt,ArrowInside={}]{IMD}{VMD}[nccurve]
						\ncput[npos=1]{\uput[180](0,0){\tiny\begin{psmatrix}[rowsep=0]Chargement\\ des données\\ de simulation\end{psmatrix}}}
						\myumlinterface[angleA=0,angleB=0,outAngleB=180,ArrowInside={}]{VRPNclient}{VMD}[nccurve]
						\ncput[npos=1]{\uput[0](0,0){\tiny\begin{psmatrix}[rowsep=0]Communication\\ avec les outils\\ de manipulation\end{psmatrix}}}
						\myumlinterface[angleA=-90,angleB=-90,outAngleB=90,offsetA=8pt,ArrowInside={}]{Renderer}{VMD}[nccurve]
						\ncput[npos=1]{\uput[-90](0,0){\tiny\begin{psmatrix}[rowsep=0]Affichage\\ de la scène\end{psmatrix}}}
					\end{myps}
					\mycaption[fig-Shaddock-DiagrammeDeComposantUMLDuNoeudVMD]{Diagramme de composant \myacronl-{acr-UML} du nœud \myacronl-{acr-VMD}}
			\end{myfigure}
		\end{mysection}
		\begin{mysection}[sec-Shaddock-ArchitectureMaterielleEtLogicielle]{Architecture matérielle et logicielle}
			\begin{mysubsection}[sse-Shaddock-LInterfaceHaptique]{L'interface haptique}
				Une plate-forme de manipulation interactive en temps-réel nécessite des outils d'interaction.
				De plus, le but final de cette thèse est d'apporter des solutions d'assistance haptique pour le travail collaboratif.
				Les types d'interfaces haptiques disponibles sur le marché sont relativement nombreuses et variées.
				Cependant, plusieurs contraintes nous ont permis de choisir le \myOmni et le \myDesktop.

				Tout d'abord, nous cherchons à pouvoir effectuer de la manipulation dans un environnement en \myThreeD; il faut choisir une interface permettant au minimum six \myacro*{acr-DDL} en entrée et au minimum trois \myacro*{acr-DDL} en retour haptique.
				En effet, un outil permettant de manipuler une molécule en translation et en rotation nécessite six \myacro*{acr-DDL} en entrée.
				Cependant, il est également nécessaire que le périphérique possède au minimum trois \myacro*{acr-DDL} en retour haptique (en translation) afin de créer des solutions d'assistance haptique.
				Évidemment, trois \myacro*{acr-DDL} supplémentaires en retour haptique (pour la rotation) pourraient permettre des solutions d'assistance plus spécifiques mais les interfaces haptiques proposant six \myacro*{acr-DDL} sont rares et relativement chères.

				De nombreuses interfaces répondent au critères donnés comme le \myPremium de chez \mySensAble ou le \myVirtuose de chez \myHaption.
				Cependant, deux critères supplémentaires nous ont permis de choisir.
				Tout d'abord, nous souhaitons fournir des outils accessibles à des biologistes : il est préférable d'avoir un outil de taille modérée qui puisse se poser sur un bureau et se substituer à une souris.
				Deuxièmement, le \myOmni fournit actuellement le meilleur rapport qualité/prix du marché en fonction de nos contraintes.
				En effet, dans le cadre du travail collaboratif, plusieurs interfaces haptiques sont nécessaires.
				De plus, le prix modéré peut amener les biologistes à adopter les outils sans avoir à investir de gros budgets; ceci peut également faciliter l'intégration de ces solutions dans les laboratoires de biologistes.
				C'est donc l'interface \myOmni \mycite{Massie-1994} de l'entreprise \mySensAble qui répond le mieux à nos attentes pour la plate-forme \myShaddock \myref*{fig-simulation-InterfaceOmniSixDDLTroisDDL}.

				\begin{myfigure}
					\myimage{simulation-omni}
					\mycaption[fig-simulation-InterfaceOmniSixDDLTroisDDL]{Interface \myOmni 6~\myacronl-{acr-DDL}/3~\myacronl-{acr-DDL}}
				\end{myfigure}

				À l'origine, les interfaces haptiques de \mySensAble était programmable à l'aide de l'\myacro{acr-API} \myGhost \mycite{SensAble-2002}.
				Le travail de \mycite[author]{Itkowitz-2005} a permis de fournir une nouvelle \myacro{acr-API} plus facile à utiliser : \myOpenHaptics.
				C'est à partir de cette \myacro{acr-API} que les interfaces haptiques sont utilisées sur \myShaddock.
				Cependant, nous verrons dans la \myref{sse-Shaddock-UneArchitectureClientServeur} que c'est un logiciel spécifique qui s'occupera de cette communication avec l'interface.
			\end{mysubsection}
			\begin{mysubsection}[sse-Shaddock-ConfigurationDeTravailCollaboratif]{Configuration de travail collaboratif}
				Le \myref{cha-LeSujet} a permis de lister les principaux critères qui caractérisent une plate-forme pour le travail collaboratif.
				Le travail peut être synchrone ou asynchrone et il peut être distant ou colocalisé \myref*{fig-Shaddock-ClassificationDesTachesCollaborativesSelonEllis}.

				\begin{myfigure}
					\psset{unit=0.083333333\textwidth}
					\begin{myps}(-1,-1)(8,4)
						\myaxes[ticks=none,labels=none,arrows={->}](0,8){distance}(0,4){temps}
						\psclip{%
							\pscustom[linestyle=none]{%
								\pszigzag[coilwidth=0.5cm,coilheight=5,linearc=.5]{-}(0,2)(8,2)
								\lineto(0,2)
								\lineto(8,2)
							}
							\pscustom[linestyle=none]{%
								\pszigzag[coilwidth=0.5cm,coilheight=5,linearc=.5]{-}(4,0)(4,4)
								\lineto(0,-4)
								\lineto(4,-4)
							}
						}
						\psframe*[linecolor=myblue!30](0,0)(8,4)
						\endpsclip
						\pszigzag[coilwidth=0.5cm,coilheight=5,linestyle=dashed,linearc=.5,linewidth=.5pt]{-}(0,2)(8,2)
						\pszigzag[coilwidth=0.5cm,coilheight=5,linestyle=dashed,linearc=.5,linewidth=.5pt]{-}(4,0)(4,4)
						\rput(2,1){\begin{tabular}{c}Collaboration\\face-à-face\end{tabular}}
						\rput(2,3){\begin{tabular}{c}Collaboration\\asynchrone\end{tabular}}
						\rput(6,1){\begin{tabular}{c}Collaboration\\synchrone\\distribuée\end{tabular}}
						\rput(6,3){\begin{tabular}{c}Collaboration\\asynchrone\\distribuée\end{tabular}}
						\uput[-90](2,0){\small colocalisé}
						\uput[-90](6,0){\small distant}
						\uput[180](0,1){\rotateleft{\small synchrone}}
						\uput[180](0,3){\rotateleft{\small asynchrone}}
					\end{myps}
					\mycaption[fig-Shaddock-ClassificationDesTachesCollaborativesSelonEllis]{Classification des tâches collaboratives selon \mycite[author]{Ellis-1991}}
				\end{myfigure}

				Nous avons vu l'importance d'une bonne conscience périphérique des autres utilisateurs dans les travaux de \mycite[author]{Casera-2006} ou \mycite[author]{Tang-2006} et plus particulièrement dans l'étude proposée par \mycite[author]{Sallnas-2010} : elle montre l'amélioration des performances lorsque la conscience périphérique est augmentée.
				Les solutions de collaboration distantes doivent recréer cette conscience en transmettant les informations audio ou même visuelle.
				Pour conserver une bonne conscience périphérique, il est préférable de ne pas numériser les communications mais de conserver leur aspect réel : il faut une collaboration synchrone colocalisée ou collaboration face-à-face selon \mycite[author]{Ellis-1991} \myref*{fig-Shaddock-ClassificationDesTachesCollaborativesSelonEllis}.

				De plus, la conscience périphérique ne se limite pas seulement à la conscience physique des autres utilisateurs.
				En effet, les collaborateurs agissent sur l'environnement ce qui permet de leur donner une existence virtuelle par l'intermédiaire des modifications de la scène.
				Ceci participe également à la conscience périphérique.
				Afin d'obtenir la meilleure conscience périphérique concernant l'environnement virtuel, les utilisateurs doivent partager le même environnement virtuel.
				C'est pour cette raison que la plate-forme \myShaddock propose une visualisation vidéoprojetée sur un grand écran.
				De cette façon, la vue est partagée par l'ensemble des utilisateurs.
			\end{mysubsection}
			\begin{mysubsection}[sse-Shaddock-UneArchitectureClientServeur]{Une architecture client/serveur}
				Deux types d'architectures ont été explorés pour les \myacro*{acr-EVC} : client/serveur ou pair-à-pair\footnote{En anglais, \myemph{peer-to-peer} parfois abrégé en \textsc{p2p}.}.
				Parmi les études notables, \mycite[author]{Iglesias-2008} propose une tâche d'assemblage collaboratif assisté par l'haptique.
				Une des problématiques soulevée est la difficulté de maintenir un environnement virtuel cohérent et fidèle pour tous les utilisateurs.
				Afin de parer à ce problème, les développeurs ont retenu une architecture de type client/serveur : la simulation est effectuée par un processus qui distribue les informations aux différents clients.
				D'ailleurs, ce papier fournit un bon état de l'art des différents type d'architectures en soulignant les avantages et inconvénients de chacune.

				En effet, il est expliqué que les architectures de types pair-à-pair permettent d'avoir des synchronisations sur le réseau plus rapides : elles nécessitent moins de paquet réseau pour la communication.
				Cependant, ce type d'architecture génère des retours haptiques instables.
				D'ailleurs, \mycite[author]{Kim-2004} qui étudie le déplacement collaboratif d'une boîte virtuelle assisté par l'haptique signalent qu'ils ont dû ajouter une viscosité importante dans le retour haptique pour éviter les instabilités.
				Bien que les architectures pair-à-pair permettent de bons résultats sur une connexion réseau relativement lente voire même sur un réseau susceptible d'avoir des coupures, elles ne sont pas adaptées pour des application collaboratives utilisant l'haptique.
				
				D'autres travaux proposent des architectures client/serveur utilisant les interactions haptiques.
				\mycite[author]{Huang-2010} propose la manipulation d'un jeu de construction par blocs.
				La simulation est centralisée sur un serveur et les interaction haptiques sont produites par l'intermédiaire de clients.
				Il ne souligne aucune instabilité dans les interactions haptiques.
				\mycite[author]{Norman-2010} s'intéresse particulièrement aux influences du réseau sur les interactions visuo-haptiques.
				L'architecture client/serveur est la plus adaptée pour la gestion de simulation.
				Cependant, il conclue sur la nécessité d'avoir une information qui transite rapidement afin d'obtenir un rendu haptique le plus fidèle possible.

				\mycite[author]{Marsh-2006} propose une comparaison de ces deux types d'architectures et en vient à la conclusion que l'architecture pair-à-pair est la plus performante en terme de latence.
				Cependant, l'avantage d'une architecture client/serveur est la cohérence de la simulation entre les différents nœuds du système (et donc entre les utilisateurs).
				Cependant, ce type d'architecture nécessite deux fois plus de paquets sur le réseau qu'une architecture pair-à-pair.
				Par opposition, l'architecture de type pair-à-pair ne permet pas d'obtenir un rendu haptique stable alors que l'achitecture de type client/serveur le permet.

				Dans notre cas, tous les utilisateurs se trouvent confrontés à la même simulation de façon colocalisée.
				L'affichage étant partagé, c'est l'architecture client/serveur qui est utilisée sur la plate-forme \myShaddock.
				Cependant, nous venons de voir que l'architecture client/serveur n'est pas optimale en terme de performances sur le réseau.
				Heureusement, la proximité des utilisateurs durant l'expérimentation nous permet d'utiliser le réseau interne du laboratoire avec une bande passante suffisante pour compenser l'inefficacité de l'architecture client/serveur en terme de débit.
				La plate-forme \myShaddock se distinge donc sous la forme d'une architecture de type client/serveur.
				\begin{mysubsubsection}[sss-Shaddock-ServeurDePeripheriques]{Serveur de périphériques}
					Afin de gérer ces connexions client/serveur pour les interfaces haptiques, nous utilisons le logiciel \myacro{acr-VRPN} développé par \mycite[author]{Taylor-II-2001}.
					La connexion avec le moteur de simulation est gérée par un autre module qui sera détaillé plus tard dans la \myref{sss-Shaddock-PlateFormeDeSimulationMoleculaireEnTempsReel}.

					\myacro{acr-VRPN} offre un moyen simple et relativement universel de connecter des périphériques principalement utilisés en réalité virtuelle.
					En effet, il fournit un serveur pour chaque périphérique.
					Ensuite, l'application cliente peut envoyer et recevoir les informations nécessaires à la communication avec chacun des périphériques.

					Dans notre cas, l'interface haptique est connectée physiquement à un ordinateur et un serveur \myacro{acr-VRPN} commande cette interface.
					C'est seulement par l'intermédiaire de ce serveur \myacro{acr-VRPN} et à travers le réseau que le client (\myacro{acr-VMD} dans notre cas) va accéder aux informations de l'interface haptique.

					La compilation de \myacro{acr-VRPN} en tant que serveur de \myOmni sous le système d'exploitation \myLinux (\myUbuntu) a nécessité quelques modifications dans le code.
					Ces modifications ont été soumises au développeur de \myacro{acr-VRPN} qui les a intégrées dans les dernières versions.

					L'avantage de cette architecture est la possibilité d'ajouter autant de serveurs et donc autant d'interfaces haptiques que nécessaire.
					Cependant, cela suppose également d'avoir autant d'ordinateurs que de serveurs ce qui complexifie la logistique.
					On pourra noter que la chaleur dégagée par l'ensemble de ces machines additionnée à celle du vidéoprojecteur créé des températures durant l'expérimentation qui peuvent être désagréables.
					C'est pourquoi aucune des expérimentations proposée ne dure plus de \mynum[mn]{30} ou, le cas échéant, une pause est effectuée au bout de \mynum[mn]{30} afin d'aérer la salle d'expérimentation.
				\end{mysubsubsection}
			\end{mysubsection}
		\end{mysection}
		\begin{mysection}[sec-Shaddock-PlateFormeDeSimulationEtDeVisualisation]{Plate-forme de simulation et de visualisation}
			\myShaddock permet d'effectuer la visualisation de molécules.
			La visualisation est un processus complexe qui nécessite des rendus variés et complets.
			En effet, devant le nombre important d'informations que possède une molécule, il est primordial d'avoir accès à des rendus graphiques performants et complets sans surcharge.
			Cette tâche est effectuée par un logiciel de visualisation présenté dans la \myref{sse-Shaddock-LogicielDeVisualisationMoleculaire}.

			Ensuite, \myShaddock simule une dynamique moléculaire.
			Un logiciel de simulation est nécessaire pour réaliser cette tâche.
			Il faut que ce logiciel puisse interagir avec le logiciel de visualisation.
			De plus, il est nécessaire de pouvoir paramétrer finement la simulation.
			Le logiciel de simulation choisi est présenté dans la \myref{sse-Shaddock-LogicielDeSimulationMoleculaire}.
			Cependant, les logiciels de simulation ne sont pas conçus pour effectuer des simulations en temps-réel.
			Pourtant, afin de proposer une dynamique moléculaire interactive aux utilisateurs, il est nécessaire d'avoir accès à une simulation en temps-réel.
			Un module présenté dans la \myref{sss-Shaddock-PlateFormeDeSimulationMoleculaireEnTempsReel} permet de faire communiquer le logiciel de visualisation avec le logiciel de simulation afin d'obtenir une simulation en temps-réel.

			\begin{mysubsection}[sse-Shaddock-LogicielDeVisualisationMoleculaire]{Logiciel de visualisation moléculaire}
				Les outils de visualisation moléculaire disponibles sont relativement nombreux.
				Parmi les plus populaires, on peut citer \myPyMOL \mycite{PyMOL-2010}, \myacro{acr-VMD} \mycite{Humphrey-1996}, \myChimera \mycite{Pettersen-2004}, \myRasmol \mycite{Sayle-1995} sans compter les nombreux dérivés permettant un affichage en ligne tel que \myJmol \mycite{Jmol-2006} pour ne citer que le plus connu.
				\myPyMOL et \myacro{acr-VMD} se distinguent particulièrement par leurs nombreuses fonctionnalités et leur large utilisation dans le milieu spécialisé.

				\myPyMOL est probablement le logiciel de visualisation le plus utilisé par les experts du domaine car c'est le plus complet pour fournir des rendus graphiques de molécules très complets.
				Cependant, \myPyMOL ne permet pas l'affichage de simulations temps-réel ni la manipulation interactive de molécules.

				\myacro{acr-VMD} possède également une large gamme de rendus graphiques.
				Contrairement à \myPyMOL, \myacro{acr-VMD} est adapté pour le rendu graphique en temps-réel de données de simulation.
				Il permet également la manipulation interactive de molécules.
				Les fonctionnalités de \myacro{acr-VMD} sont nombreuses et seulement certaines on été utilisées dans le cadre des expérimentations qui vont suivre.
				Elles sont exposées dans les paragraphes suivants.

				\begin{mysubsubsection}[sss-Shaddock-LesRendusGraphiques]{Les rendus graphiques}
					La possibilité d'avoir accès à des rendus graphiques divers et complets est primordiale pour la visualisation moléculaire.
					La complexité des molécules, le nombre important d'atomes, les nombreuses meta-informations, les structures particulières nécessitent d'avoir à sa disposition des moyens évolués et variés pour afficher une molécule.
					Quatre représentations différentes \myref*{fig-simulation-IllustrationsDesRepresentationsDeMoleculesSurVMD} ont été utilisées sur la plate-forme \myShaddock :
					\begin{description}
						\item[\myCPK] affiche tous les atomes de la molécule sous forme de sphères en les reliant par des cylindres; c'est un affichage très chargé lorsque le nombre d'atomes est important mais la taille des sphères et des cylindres peut être modifiée \myref*{fig-simulation-CPK};
						\item[\myLicorice] représente tous les liens entre les atomes par des cylindres, sans représenter les atomes; la taille des cylindres peut être modifiée \myref*{fig-simulation-Licorice};
						\item[\myNewRibbon] produit une courbe spline sur les atomes $C_{\alpha}$ représentant l'armature principale de la molécule; la courbe est représentée sous forme de ruban \myref*{fig-simulation-NewRibbon};
						\item[\myHBonds] affiche les potentielles liaisons hydrogène sous forme de traits en pointillés; les seuils physiques ainsi que les paramètres d'affichage de la ligne (couleur, largeur, \myetc) sont modifiables \myref*{fig-simulation-HBonds}.
					\end{description}

					\begin{myfigure}
						\begin{mysubfigure}
							\myimage[width=0.49\textwidth]{simulation-CPK}
							\mysubcaption[fig-simulation-CPK]{\myCPK}
						\end{mysubfigure}
						\begin{mysubfigure}
							\myimage[width=0.49\textwidth]{simulation-Licorice}
							\mysubcaption[fig-simulation-Licorice]{\myLicorice}
						\end{mysubfigure}
						\begin{mysubfigure}
							\myimage[width=0.49\textwidth]{simulation-NewRibbon}
							\mysubcaption[fig-simulation-NewRibbon]{\myNewRibbon}
						\end{mysubfigure}
						\begin{mysubfigure}
							\myimage[width=0.49\textwidth]{simulation-HBonds}
							\mysubcaption[fig-simulation-HBonds]{\myHBonds}
						\end{mysubfigure}
						\mycaption[fig-simulation-IllustrationsDesRepresentationsDeMoleculesSurVMD]{Illustration des représentations de molécules sur \myacronl{acr-VMD}}
					\end{myfigure}

					Chacune de ces représentations visuelles peut être affectée à tout ou partie de la molécule comme par exemple \og le \myglos{glo-Residu} \mynum{13} \fg, \og seulement les atomes de carbone \fg ou \og tous les \myglos*{glo-Residu} entre \mynum{1} et \mynum{16} sauf les atomes d'hydrogène \fg.
					De plus, pour chacune des représentations précédentes, différentes colorations sont possibles :
					\begin{description}
						\item[Couleur fixe] donne une couleur unie prédéfinie (la couleur du curseur par exemple);
						\item[Couleur des atomes] donne une couleur différente à chaque atome selon un code couleur standard dépendant de sa nature (rouge pour oxygène, blanc pour hydrogène, \myetc);
						\item[Couleur des \myglosnl*{glo-Residu}] donne une couleur différente pour chaque atome selon une palette de couleurs prédéfinie par \myacro{acr-VMD};
						\item[Transparence] rend transparent les objets tout en conservant la teinte;
						\item[\textit{GoodSell}] accentue les contours des objets sous le principe du \myemph{cell shading}.
					\end{description}
				\end{mysubsubsection}
				\begin{mysubsubsection}[sss-Shaddock-LaGenerationAutomatiqueDeFichierDeSimulation]{La génération automatique de fichier de simulation}
					La simulation nécessite de nombreuses informations.
					Une partie de ces informations découle directement de la molécule à l'état d'équilibre; ces informations sont les suivantes :
					\begin{itemize}
						\item l'ensemble des liaisons entre atomes;
						\item des angles simples;
						\item des angles dihédraux;
						\item des angles de torsion.
					\end{itemize}
					La simple description des atomes et de leurs positions à l'état d'équilibre (fichier \myPDB) couplée aux données générées par \myCHARMM \mycite{Brooks-1983} permet de générer les fichiers nécessaires au logiciel de simulation.
					\myacro{acr-VMD} fournit tous les outils permettant de générer ce fichier nécessaire à la simulation (fichier \myPSF) par l'intermédiaire d'une extension : \myemph{Automatic \textsc{psf} builder}.
				\end{mysubsubsection}
			\end{mysubsection}
			\begin{mysubsection}[sse-Shaddock-LogicielDeSimulationMoleculaire]{Logiciel de simulation moléculaire}
				Les deux principaux logiciels de simulation existants sont \myacro{acr-NAMD} \mycite{Phillips-2005} et \myGromacs \mycite{Berendsen-1995}.
				\myGromacs est plus performant que \myacro{acr-NAMD}, surtout dans les dernières versions \mycite{Hess-2008} qui offre des performances jusqu'à quatre fois plus rapide que \myacro{acr-NAMD}.
				\myacro{acr-NAMD} est développé par la même université que \myacro{acr-VMD} et la connexion entre les deux logiciels est facilitée.
				Enfin, \myacro{acr-NAMD} peut être aisément connecté à \myacro{acr-VMD} dans le cadre d'une simulation interactive contrairement à \myGromacs.
				C'est pourquoi le logiciel \myacro{acr-NAMD} a été retenu pour notre plate-forme.

				Une des fonctionnalités de \myacro{acr-NAMD} utilisée est la possibilité de \og fixer \fg des atomes.
				En effet, la fixation d'atomes permet d'exclure partiellement certains atomes durant la simulation.
				Ces atomes interviennent dans le calcul des forces de la simulation mais eux-mêmes ne sont pas soumis aux forces de l'environnement.
				Cette fonctionnalité est nécessaire pour simuler un point d'ancrage de la molécule dans l'environnement virtuel.
				Sans ce point d'ancrage, la molécule pourrait dériver et sortir de l'espace de travail des utilisateurs sans possibilité de récupération.
				\begin{mysubsubsection}[sss-Shaddock-PlateFormeDeSimulationMoleculaireEnTempsReel]{Plate-forme de simulation moléculaire en temps-réel}
					Les logiciels de simulation ne sont pas prévus pour des simulations interactives en temps-réel.
					Cependant, l'\myacro{acr-ITAP} a développé le protocole \myacro{acr-IMD} permettant d'utiliser \myacro{acr-NAMD} couplé à \myacro{acr-VMD} pour des simulations en temps-réel \mycite{Stadler-1997}.
					L'extension \myemph{\myacronl{acr-IMD} connect} permet de connecter rapidement le logiciel \myacro{acr-VMD} avec la simulation offerte par \myacro{acr-NAMD}.

					Cependant, entre le début du développement de notre plate-forme en 2008 et aujourd'hui, une nouvelle solution plus générique a été développée au sein de l'\myacro{acr-IBPC}.
					En effet, \myMDDriver \mycite{Delalande-2009} est une interface permettant d'utiliser le protocole \myacro{acr-IMD} avec d'autre logiciel de simulation comme \myGromacs.
					C'est une interface capable de gérer différents logiciels de visualisation et de simulation.
					Cependant, cette nouvelle solution n'a pas encore été implémentée dans notre plate-forme.
				\end{mysubsubsection}
			\end{mysubsection}
		\end{mysection}
		\begin{mysection}[sec-Shaddock-LesOutilsDInteraction]{Les outils d'interaction}
			\begin{mysubsection}[sse-Shaddock-LesOutilsExistants]{Les outils existants}
				La manipulation des molécules est nécessaire sur la plate-forme \myShaddock.
				\myAcro{acr-VMD} dispose déjà de différents outils permettant d'effectuer différentes manipulation sur les molécules.

				Par défaut et sans configuration, la souris permet d'orienter la scène sur trois \myacro*{acr-DDL} afin d'observer la molécule sous différents angles.
				Elle peut également être configurées pour translater la molécule ou obtenir diverses informations sur la molécules.

				Il est également possible d'utiliser une souris \myThreeD, automatiquement détectée lorsqu'elle est branchée sur l'ordinateur.
				Une souris \myThreeD permet de translater et d'orienter la scène.
				La souris \myThreeD \mySpaceNavigator est utilisée dans le cadre de certaines de notre seconde expérimentation \myref*{cha-DeformationCollaborativeDeMolecule}.

				Enfin, des outils apportant une dimension haptique sont disponibles par l'intermédiaire d'une connexion avec \myacro{acr-VRPN} \myref*{sss-Shaddock-ServeurDePeripheriques}.
				Ces outils sont liés à des périphériques externes (des interfaces \myOmni dans notre cas).
				Les outils proposés par défaut dans \myacro{acr-VMD} ont été utilisés dans la première expérimentation \myref*{cha-RechercheCollaborativeDeResiduSurUneMolecule}.
				Ils sont les suivants :
				\begin{description}
					\item[\mytool{grab}] qui permet de sélectionner une molécule dans son intégralité et de la déplacer dans la scène;
					\item[\mytool{tug}] qui permet de sélectionner un atome de la molécule et de lui appliquer une force (qui sera transmise à la simulation) pour déformer la molécule.
				\end{description}

				Cependant, de nombreux outils supplémentaires ont été développés au-fur-et-à-mesure des besoins identifiés durant les expérimentations.
				Ces nouveaux outils sont détaillés dans la \myref{sse-Shaddock-LesNouveauxOutilsDInteraction}.
			\end{mysubsection}
			\begin{mysubsection}[sse-Shaddock-LesNouveauxOutilsDInteraction]{Les nouveaux outils d'interaction}
				Durant les différentes études présentées dans la \myref{prt-EtudeDuTravailCollaboratif}, les analyses et les remarques d'utilisateurs ont permis d'améliorer les outils d'interaction et d'en proposer de nouveaux.
				Le développement de ces nouveaux outils a nécessité une modification du programme \myacro{acr-VMD} par extension des outils déjà existants.
				Des fonctionnalités ont été ajoutées et sont présentées dans les sections suivantes.
				\begin{mysubsubsection}[sss-Shaddock-AmeliorationDeLaSelection]{Amélioration de la sélection}
					Durant le processus de recherche et de sélection, les utilisateurs ont souvent évoqué le besoin de connaître en continu leur position et de savoir à priori l'élément qui va être sélectionné.
					Pour que les utilisateurs connaissent à chaque instant l'élément qui peut être sélectionné, une information visuelle met en surbrillance l'élément pointé à chaque instant.
					La mise en surbrillance est un agrandissement en transparence de l'élément pointé.
					La couleur de cette mise en surbrillance est de la même couleur que le curseur de l'utilisateur.

					Dans le cas d'une sélection par \myglos{glo-Residu}, c'est l'ensemble du \myglos{glo-Residu} qui est mis en surbrillance.
					Cependant, dans le cas d'une sélection par atome, la mise en surbrillance d'un seul atome pourrait être difficile à apercevoir au sein d'une molécule complexe.
					En effet, le nombre important d'atomes d'une molécule peut surcharger le rendu graphique.
					C'est pourquoi dans ce cas, l'ensemble du \myglos{glo-Residu} auquel appartient l'atome pointé est mis en surbrillance.
					Cependant, l'atome pointé est agrandi afin de le distinguer du reste du \myglos{glo-Residu}.

					Une fois l'élément pointé, l'utilisateur peut sélectionner l'élément.
					Lorsque les utilisateurs sélectionnent l'élément, la surbrillance passera de la transparence à l'opacité.
					Une illustration des effets visuels relatifs au pointage et à la sélection est affichée sur la \myref{fig-selection-improvement-DifferenceVisuelleEntreLesElementsPointesEtSelectionnes}.

					\begin{myfigure}
						\begin{mysubfigure}
							\myimage[width=0.49\textwidth]{selection-improvement-pointed}
							\mysubcaption[fig-selection-improvement-ElementPointe]{Élément pointé}
						\end{mysubfigure}
						\begin{mysubfigure}
							\myimage[width=0.49\textwidth]{selection-improvement-targeted}
							\mysubcaption[fig-selection-improvement-ElementSelectionne]{Élément sélectionné}
						\end{mysubfigure}
						\mycaption[fig-selection-improvement-DifferenceVisuelleEntreLesElementsPointesEtSelectionnes]{Différence visuelle entre les éléments pointés et sélectionnés}
					\end{myfigure}
				\end{mysubsubsection}
				\begin{mysubsubsection}[sss-Shaddock-DeformationParGroupeDAtomes]{Déformation par groupe d'atomes}
					L'outil \mytool{tug} permet de déformer la molécule en appliquant un effort à l'atome sélectionné.
					Cependant, la déformation par l'intermédiaire d'un seul atome possède deux désavantages.

					Tout d'abord, la déformation d'une molécule atome par atome est un processus très fastidieux.
					Il serait plus efficace de déplacer un groupe d'atomes en une seule fois.

					De plus, l'application d'un effort sur un atome provoque l'étirement de la molécule.
					Au repos, la molécule est dans état relativement stable.
					Déplacer un atome perturbe cet état de stabilité.
					De plus, certains atomes sont fortement liés et les éloigner peut perturber grandement l'état de stabilité.
					Il est donc préférable de déplacer tous ces atomes liés en une seule manipulation.

					C'est pourquoi un outil appliquant une force à un groupe d'atomes permet de le déplacer tout en conservant la stabilité intrinsèque.
					Les groupes d'atomes dignes d'intérêt sont les \myglos*{glo-Residu} (une vingtaine d'atomes), les \myhelice* ou \myfeuillet* (une vingtaine de \myglos*{glo-Residu}) et les molécules.
					Cependant, \myacro{acr-VMD} n'est pas capable de fournir l'information sémantique regroupant les atomes en \myhelice* ou en \myfeuillet*.
					La fonctionnalité de l'outil \mytool{tug} a donc été étendue aux \myglos*{glo-Residu} et aux molécules.

					Cependant, appliquer le même effort à l'ensemble des atomes de la molécule produit un effort total très important.
					Si l'effort total est trop important, les perturbations envoyées à la simulation sont trop puissantes et peuvent produire des incohérences dans la simulation voire même un arrêt de la simulation.
					En effet, \myacro{acr-NAMD} peut décider d'arrêter la simulation s'il considère que la simulation diverge trop d'un état stable.
					Il est donc nécessaire de diviser l'intensité des forces appliquées proportionnellement au nombre d'atomes sélectionnés.
				\end{mysubsubsection}
				\begin{mysubsubsection}[sss-Shaddock-OutilDeDesignation]{Outil de désignation}
					Un besoin récurrent constaté durant les expérimentations est la nécessité pour les utilisateurs de désigner un élément de la molécule.
					Parfois les utilisateurs éprouvent le besoin de désigner une élément de l'environnement virtuel pour un autre utilisateur.
					Les enregistrements audio ont également permis d'identifier ce besoin.

					L'outil de désignation a été conçu pour répondre à un processus en quatre étapes:
					\begin{enumerate}[label={\alph*.},ref={\alph*}]
						\item Recherche d'une cible;\label{enu-designation-RechercheDUneCible}
						\item Désignation d'une cible;\label{enu-designation-DesignationDUneCible}
						\item Acceptation d'une cible;\label{enu-designation-AcceptationDUneCible}
						\item Sélection d'une cible.\label{enu-designation-SelectionDUneCible}
					\end{enumerate}

					\begin{myfigure}
						\begin{mysubfigure}
							\myimage[width=0.49\textwidth]{designation-normal}
							\mysubcaption[fig-designation-RechercheDUneCible]{Recherche d'une cible}
						\end{mysubfigure}
						\begin{mysubfigure}
							\myimage[width=0.49\textwidth]{designation-called}
							\mysubcaption[fig-designation-CibleDesignee]{Cible désignée}
						\end{mysubfigure}
						\begin{mysubfigure}
							\myimage[width=0.49\textwidth]{designation-accepted}
							\mysubcaption[fig-designation-CibleAcceptee]{Cible acceptée}
						\end{mysubfigure}
						\begin{mysubfigure}
							\myimage[width=0.49\textwidth]{designation-selected}
							\mysubcaption[fig-designation-CibleSelectionnee]{Cible sélectionnée}
						\end{mysubfigure}
						\mycaption[fig-designation-LesQuatreEtapesDeLaDesignation]{Les quatre étapes de la désignation}
					\end{myfigure}

					L'\myref{enu-designation-RechercheDUneCible} consiste pour un utilisateur~\myuser{A} à rechercher une cible à désigner.
					Cette cible est choisie en fonction des objectifs de la tâche à réaliser et sera indiquée à un utilisateur \myuser{B} \myref*{fig-designation-RechercheDUneCible}.

					Une fois la cible trouvée, l'utilisateur~\myuser{A} désigne la cible identifiée lors de l'\myref{enu-designation-DesignationDUneCible}.
					La cible est alors mise en surbrillance de façon à être vue des autres utilisateurs \myref*{fig-designation-CibleDesignee}.

					L'\myref{enu-designation-AcceptationDUneCible} fait intervenir l'utilisateur~\myuser{B}.
					L'utilisateur~\myuser{B} peut accepter ou non cette désignation.
					S'il accepte la désignation, la cible est alors colorée de la couleur du curseur de l'utilisateur~\myuser{B} qui a accepté \myref*{fig-designation-CibleAcceptee}.
					Tant qu'elle n'est pas acceptée, la \myglos{glo-Residu} reste en surbrillance jusqu'à ce que la requête soit accepté ou modifiée.

					L'\myref{enu-designation-SelectionDUneCible} est la dernière étape.
					L'utilisateur~\myuser{B} ayant accepté doit maintenant sélectionner la cible pour achever le processus de désignation.
					Tant que l'utilisateur~\myuser{B} n'a pas sélectionné le \myglos{glo-Residu} ciblé, le processus ne peut pas être considéré comme terminé et l'effet de surbrillance reste actif.
					Lorsque la cible est sélectionnée, le processus de manipulation reprend normalement \myref*{fig-designation-CibleSelectionnee}.
				\end{mysubsubsection}
			\end{mysubsection}
		\end{mysection}
	\end{mychapter}
	\begin{mychapter}[cha-RechercheCollaborativeDeResiduSurUneMolecule]{Recherche collaborative de résidu sur une molécule}
		\begin{mysection}[sec-exp1-Introduction]{Introduction}
			L'état de l'art du premier chapitre nous a permis d'identifier les principales tâches élémentaires concernant l'interaction en environnement virtuel, décrites par \mycite[author]{Bowman-1999} : les \myacro*{acr-PCV} \mycite{Fuchs-2006a}.
			Chaque primitive nécessite que la précédente ait été réalisée avec succès; on ne peut pas manipuler tant qu'une sélection n'a pas été effectuée; on ne peut pas sélectionner sans avoir naviguer, explorer et identifier les cibles à sélectionner.
			Le processus d'exploration et de sélection sont les étapes primordiales à toute manipulation ultérieure.
			Cependant, la sélection au sein d'une simulation moléculaire est une problématique complexe à part entière \mycite{Delalande-2010}.
			En effet, les molécules possèdent un nombre importants d'atomes sous la forme d'une longue chaîne carbonée repliée sur elle-même ce qui rend la recherche d'une cible complexe.
			L'appréhension d'un tel environnement virtuel peut être effectuée par des améliorations visuelles \mycite{Chavent-2011} ou des interfaces d'interactions adaptées \mycite{Delalande-2010} par exemple.
			Cependant, nous choisissons d'explorer une troisième solution : la distribution des charges cognitives de travail par l'intermédiaire d'une collaboration synchrone colocalisée.

			Dans ce chapitre, nous nous intéressons aux deux premières étapes du processus de déformation : la \myemph{recherche} et la \myemph{sélection} \myref*{fig-ProcessusDeDeformationMoleculaireEnQuatreEtapes}.
			Nous étudions tout particulièrement l'influence du travail collaboratif sur ces étapes.
			Le prétexte de cette étude est la recherche de \myglos*{glo-Residu}.
			L'étude proposée montre l'intérêt du travail collaboratif sur des tâches de nature complexe.
			Cependant, nous verrons que les \myglos*{glo-Binome} adoptent des stratégies différentes, menant à des performances hétérogènes.
			Nous identifions ainsi les avantages de la collaboration mais également les contraintes d'une telle configuration de travail.
		\end{mysection}
		\begin{mysection}[sec-exp1-RechercheEtSelectionCollaborative]{Recherche et sélection collaborative}
			\begin{mysubsection}[sse-exp1-TravauxExistants]{Travaux existants}
				La sélection en environnement virtuel est une tâche élémentaire relativement peu explorée pour la biologie moléculaire en environnement virtuel.
				Pourtant, elle est déjà relativement développée pour les \myacro*{acr-AFM} mais les contraintes techniques de cette technologie sont très différentes de nos contraintes en environnement virtuel; nous ne nous étendrons pas sur ce sujet.

				La taille et la complexité des molécules nécessite des solutions de sélection adaptées.
				Les logiciels les plus utilisés tel que \myPyMOL ou \myacro{acr-VMD} proposent des moteurs de sélection à base de chaîne de caractères.
				Par exemple, pour sélectionner tous les atomes de type \myatom{C}, \myatom{O}, \myatom{N} ou \myatom{CA}, on utilisera les commandes suivantes dans \myPyMOL
				\begin{mySource*}[language={}]
pymol> select mysel, name c+o+n+ca
				\end{mySource*}
				ou dans \myacro{acr-VMD}
				\begin{mySource*}[language={}]
vmd> set mysel [atomselect "top" "name C or name O or name N or name CA"]
				\end{mySource*}
				Cette méthode de sélection est seulement efficace dans le cas où la cible à sélectionner est connue à l'avance.
				Dans le cadre de l'exploration virtuelle d'une molécule en \myThreeD, \myacro{acr-VMD} propose également une sélection par \myemph{picking}\footnote{Méthode qui consiste à trouver l'élément pointé par la souris (\myTwoD) dans l'environnement virtuel (\myThreeD).} à l'aide de la souris.
				Cependant, ce type de méthode atteint ses limites en terme de précision dès que la molécule est de taille trop importante : la perception de la profondeur n'est pas permise avec cette méthode de sélection.

				Depuis déjà plusieurs années, la communauté de la réalité virtuelle s'est intéressé à cette problématique en proposant des périphériques d'interaction pour les environnements virtuels en \myThreeD.
				\mycite[author]{Pavlovic-1996} ont développé une plate-forme permettant d'interagir avec les molécules en utilisant la voix et les gestes.
				Cependant, les techniques de segmentations pour la reconnaissance des gestes sont encore difficiles à mettre en place, même si l'arrivée récente de la \myKinect de \myMicrosoft a amélioré cette démocratisation.
				On trouve également les travaux de \mycite[author]{Polys-2004} qui propose une interaction avec une \myemph{wand}\footnote{Dispositif d'interaction en réalité virtuelle permettant la manipulation sur \mynum{6}~\myacro*{acr-DDL} et disposant de boutons comme une souris.} ou encore les travaux de \mycite[author]{Obeysekare-1996} permettant l'interaction gestuelle avec un gant sur un \myemph{workbench}\footnote{Dispositif d'affichage permettant l'affichage en \myThreeD stéréoscopique sur deux écrans.}.
				Tous ces dispositifs sont relativement lourds à déployer et ne sont pas adaptés pour une utilisation sur un ordinateur de bureau.

				Tous les dispositifs présentés dans le paragraphe précédent sont testés sur des molécules ne contenant que quelques atomes et ayant peu d'intérêt pour les biologistes.
				Cependant, la sélection sur des tâches pertinentes pour les biologistes est un sujet encore peu exploré.
				\mycite[author]{Levine-1997} proposent une plate-forme d'interaction avec un environnement moléculaire virtuel afin d'explorer un complexe ligand-protéine pour réaliser un amarrage de molécules.
				Cependant, l'amarrage de molécule s'effectue ici sur des corps rigides ce qui simplifie grandement la complexité de l'exploration.
				On trouve également les travaux de \mycite[author]{Ferey-2008a} mais là encore, il s'agit de corps rigides.
				Cependant, les travaux ont évolué vers des corps flexibles avec \mycite[author]{Delalande-2010} qui utilisent les périphériques haptiques pour aider à la localisation de ponts ioniques au sein d'une simulation moléculaire en temps-réel.
				L'interface haptique utilisée permet de ressentir les forces en action dans la simulation et ainsi améliorer le processus d'exploration et de sélection.
			\end{mysubsection}
			\begin{mysubsection}[sse-exp1-Objectifs]{Objectifs}
				Dans ce chapitre, nous abordons les tâches de recherche et de sélection dans un \myacro{acr-EVC}.
				La recherche au sein d'une simulation moléculaire est une tâche très complexe en raison du grand nombre d'atomes et de la mobilité des atomes.
				Nous proposons d'étudier l'apport de la distribution des charges de travail par une configuration de travail collaboratif pour la réalisation de cette tâche.

				Les objectifs de cette première partie sont multiples.
				Tout d'abord, nous souhaitons observer les performances comparées d'un travail autonome face à une configuration de travail collaborative.
				Notre hypothèse va dans le sens d'une amélioration des performances pour les configurations de travail collaboratives sur des tâches de nature complexe.

				De plus, nous souhaitons observer les stratégies de travail en collaboration.
				Nous supposons que les stratégies vont varier d'un groupe à l'autre en fonction des affinités et des connaissances intrinsèques du groupe.
				Nous verrons qu'une configuration collaborative permet de mettre en avant des stratégies de travail distinctes en fonction de la communication entre les utilisateurs, des espaces de travail des manipulateurs, de la répartition des tâches, \myetc

				Ensuite, nous nous intéressons plus précisément aux avis des utilisateurs.
				Nous supposons que la configuration collaborative est plus apprécié des utilisateurs grâce à l'émulsion sociale qui naît des interactions et de la communication entre les manipulateurs.

				Enfin, nous souhaitons valider la plate-forme de manipulation proposée.
				Pour cela, l'évaluation sera confiée aux sujets.
				En effet, il est nécessaire de prendre en compte l'expérience de l'utilisateur afin d'améliorer l'ergonomie des outils proposés.
				L'objectif est de vérifier l'utilisabilité de la plate-forme afin d'identifier les points faibles et les contraintes.
			\end{mysubsection}
		\end{mysection}
		\begin{mysection}[sec-exp1-PresentationDeLExperimentation]{Présentation de l'expérimentation}
			Afin de répondre à nos hypothèses de travail, résumées dans la \myref{sse-met-exp1-Hypotheses}, nous mettons en place une expérimentation pour la recherche de \myglos{glo-Residu} au sein d'une molécule à travers une simulation en temps-réel.
			Cette section est destinée à décrire la tâche proposée puis les spécificités du protocole expérimental.
			\begin{mysubsection}[sse-exp1-DescriptionDeLaTache]{Description de la tâche}
				La tâche proposée consiste à trouver des \myglos*{glo-Residu} au sein d'une molécule puis à les extraire hors de la molécule.
				Les \myglos*{glo-Residu} sont des groupes d'atomes s'associant les uns aux autres le long d'une chaîne carbonée pour former une molécule.
				Trois molécules sont proposées dans le cadre de cette expérimentation.
				La molécule \myTRPZIPPER sera utilisée pour la procédure d'apprentissage de la plate-forme.
				Les molécules \myTRPCAGE et \myPrion sont utilisées pour la tâche de recherche et d'extraction de \myglos*{glo-Residu} : chaque molécule possède \mynum{5}~\myglos*{glo-Residu} à extraire.
				Tous les \myglos*{glo-Residu} à rechercher sont affichés dans la \myref{tab-exp1-ListeDesResidusRecherches}.
				Pour une description plus précises des molécules, se reporter à la \myref{sec-pro-PresentationsDesMolecules}.

				\begin{mytable}
					\mycaption[tab-exp1-ListeDesResidusRecherches]{Liste des \myglosnl*{glo-Residu} recherchés}
					\setlength{\myheight}{10ex}
					\newcommand{\mypatternpicture}[1]{\myimage[width=\myheight]{exp1-#1}}
					\begin{mysubtable}
						\mysubcaption[tab-exp1-ListeDesResidusRecherches-ResidusSurLaMoleculeTRPCAGE]{Residus sur la molécule \myTRPCAGE}
						\begin{mytabular}[0.49\textwidth]{^C-C}
							\mytoprule
							\myrowstyle{\bfseries}
							\myGlosnl{glo-Residu} & Image \\
							\mymiddlerule
							\myresidue{1} & \mypatternpicture{pattern1} \\
							\myresidue{2} & \mypatternpicture{pattern2} \\
							\myresidue{3} & \mypatternpicture{pattern3} \\
							\myresidue{4} & \mypatternpicture{pattern4} \\
							\myresidue{5} & \mypatternpicture{pattern5} \\
							\mybottomrule
						\end{mytabular}
					\end{mysubtable}
					\begin{mysubtable}
						\mysubcaption[tab-exp1-ListeDesResidusRecherches-ResidusSurLaMoleculePrion]{Residus sur la molécule \myPrion}
						\begin{mytabular}[0.49\textwidth]{^C-C}
							\mytoprule
							\myrowstyle{\bfseries}
							\myGlosnl{glo-Residu} & Image \\
							\mymiddlerule
							\myresidue{6}  & \mypatternpicture{pattern6}  \\
							\myresidue{7}  & \mypatternpicture{pattern7}  \\
							\myresidue{8}  & \mypatternpicture{pattern8}  \\
							\myresidue{9}  & \mypatternpicture{pattern9}  \\
							\myresidue{10} & \mypatternpicture{pattern10} \\
							\mybottomrule
						\end{mytabular}
					\end{mysubtable}
				\end{mytable}

				La \myref{fig-exp1-RepartitionDesResidusSurLesMolecules} montre la répartition des \myglos*{glo-Residu} sur les deux molécules.
				Chaque \myglos{glo-Residu} possède ses propres spécificités (position, couleur, \myetc).
				Les critères de complexité, résumés pour chaque \myglos{glo-Residu} dans la \myref{tab-exp1-ParametresDeComplexiteDesResidus}, sont les suivants :
				\begin{description}
					\item[Nombre de \myglosnl*{glo-Residu}] Le nombre total de \myglos*{glo-Residu} présents dans la molécule.
						Un nombre important des \myglos*{glo-Residu} surcharge visuellement l'environnement virtuel et augmente le nombre de cibles potentielles.
					\item[Position] Le \myglos{glo-Residu} peut se trouver soit à la périphérie de la molécule (position \myemph{externe}) ou au centre de la molécule (position \myemph{interne}).
						Un \myglos{glo-Residu} en position externe ne nécessite pas de déformer la molécule pour le trouver et l'atteindre contrairement à un \myglos{glo-Residu} en position interne qui sera plus complexe d'accès.
					\item[Forme] La forme du \myglos{glo-Residu} est un motif graphique plus ou moins complexe à identifier.
						On distingue trois formes différentes :
						\begin{description}
							\item[Chaîne] Une chaîne d'atomes (la plupart du temps carbonés) avec des atomes d'hydrogène de chaque côté.
							\item[Cycle] Une chaîne fermée d'atomes de carbone ou d'azote.
							\item[Étoile] Séries de chaînes d'atomes toutes reliées sur un atome central (un atome de carbone pour la plupart du temps).
						\end{description}
					\item[Couleurs] Les atomes sont colorés en fonction de leur nature (rouge pour l'oxygène, blanc pour l'hydrogène, \myetc).
						Les atomes rares seront donc rapidement identifiés grâce à leur couleur singulière.
						Par contre, les atomes nombreux (comme les hydrogènes ou les carbones) seront plus difficiles à filtrer à cause de leur fréquence d'apparition élevée.
					\item[Similarité] Certains \myglos*{glo-Residu} à chercher sont très similaires à d'autres \myglos*{glo-Residu} également présents sur la molécule.
						Les \myglos*{glo-Residu} similaires possèdent un atome de moins ou de plus par rapport au \myglos{glo-Residu} recherché.
						À cause de cette similarité, les sujets vont mobiliser une partie du temps à identifier des \myglos*{glo-Residu} incorrects.
				\end{description}

				\begin{mytable}
					\newcommand{\myatomincolor}[3]{\csname my#1\endcsname{}{}#2 en \myemph{#3}}
					\mycaption[tab-exp1-ParametresDeComplexiteDesResidus]{Paramètres de complexité des \myglosnl*{glo-Residu} -- \myatomincolor{carbon}{arbone}{cyan}, \myatomincolor{nytrogen}{zote}{bleu}, \myatomincolor{oxygen}{xygène}{rouge} et \myatomincolor{sulfur}{oufre}{jaune}}
					\begin{mytabular}{^C-C-C-C-C-C}
						\mytoprule
						\myrowstyle{\bfseries}
						\myGlosnl{glo-Residu} & Nombre de \myglosnl*{glo-Residu} & Position & Forme  & Couleurs                                   & Similarité \\
						\mymiddlerule[\heavyrulewidth]
						\myresidue{1}         & \mynum{20}                       & Interne  & Cycle  & \mynum{8}~\mycarbon, \mynum{1}~\mynytrogen & Non        \\
						\mymiddlerule
						\myresidue{2}         & \mynum{20}                       & Interne  & Étoile & \mynum{1}~\mycarbon, \mynum{3}~\mynytrogen & Non        \\
						\mymiddlerule
						\myresidue{3}         & \mynum{20}                       & Interne  & Cycle  & \mynum{6}~\mycarbon, \mynum{1}~\myoxygen   & Non        \\
						\mymiddlerule
						\myresidue{4}         & \mynum{20}                       & Externe  & Chaîne & \mynum{4}~\mycarbon                        & Non        \\
						\mymiddlerule
						\myresidue{5}         & \mynum{20}                       & Externe  & Chaîne & \mynum{4}~\mycarbon, \mynum{1}~\mynytrogen & Non        \\
						\mymiddlerule[\heavyrulewidth]
						\myresidue{6}         & \mynum{112}                      & Interne  & Chaîne & \mynum{2}~\mycarbon, \mynum{2}~\mysulfur   & Non        \\
						\mymiddlerule
						\myresidue{7}         & \mynum{112}                      & Externe  & Étoile & \mynum{1}~\mycarbon, \mynum{3}~\mynytrogen & Non        \\
						\mymiddlerule
						\myresidue{8}         & \mynum{112}                      & Externe  & Cycle  & \mynum{6}~\mycarbon, \mynum{1}~\myoxygen   & Non        \\
						\mymiddlerule
						\myresidue{9}         & \mynum{112}                      & Interne  & Chaîne & \mynum{4}~\mycarbon                        & Oui        \\
						\mymiddlerule
						\myresidue{10}        & \mynum{112}                      & Interne  & Chaîne & \mynum{4}~\mycarbon, \mynum{1}~\mynytrogen & Oui        \\
						\mybottomrule
					\end{mytabular}
				\end{mytable}
				\begin{myfigure}
					\newcommand{\schemafactor}{0.20}
					\newlength{\schemaunit}\setlength{\schemaunit}{\schemafactor\textwidth}
					\psset{unit=\schemaunit}
					\mycaption[fig-exp1-RepartitionDesResidusSurLesMolecules]{Répartition des \myglosnl*{glo-Residu} sur les molécules}
					\begin{myps}(-2.5,-3)(2.5,3)
						\rput(0,1.75){%
							\myimage[height=2\schemaunit,angle=90]{exp1-trp-cage}}
						\rput(0,2.9){%
							\textcolor{black!25}{\Huge\bfseries\myTRPCAGE}}
						\rput(0,-1.25){%
							\myimage[height=2\schemaunit,angle=90]{exp1-prion}}
						\rput(0,0){%
							\textcolor{black!25}{\Huge\bfseries\myPrion}}
						\rput(-1.5,2){%
							\myimage[height=\schemaunit]{exp1-pattern1}}
						\rput(1.5,2){%
							\myimage[width=\schemaunit]{exp1-pattern3}}
						\rput(1.5,-0){%
							\myimage[width=\schemaunit]{exp1-pattern2}}
						\rput(-1.5,-0){%
							\myimage[width=\schemaunit]{exp1-pattern4}}
						\rput(-1.5,-2){%
							\myimage[width=\schemaunit]{exp1-pattern5}}
						\rput(1.5,-2){%
							\myimage[height=\schemaunit]{exp1-pattern6}}

						\psset{framesize=1 1}
						\fnode(-1.5,2){P1}
						\uput[90](-1.5,2.5){\myresidue{1}}
						\fnode(1.5,2){P38}
						\uput[90](1.5,2.5){\myresidue{3} et \myresidue{8}}
						\fnode(1.5,-0){P27}
						\uput[90](1.5,0.5){\myresidue{2} et \myresidue{7}}
						\fnode(-1.5,-0){P49}
						\uput[90](-1.5,0.5){\myresidue{4} et \myresidue{9}}
						\fnode(-1.5,-2){P510}
						\uput[90](-1.5,-1.5){\myresidue{5} et \myresidue{10}}
						\fnode(1.5,-2){P6}
						\uput[90](1.5,-1.5){\myresidue{6}}

						\psset{linecolor=myred}
						\cnode(0.3,1.5){0.2}{TRPP1}
						\cnode(0.15,2){0.2}{TRPP38}
						\cnode(-0.1,1.25){0.2}{TRPP27}
						\cnode(-0.5,2.2){0.2}{TRPP49}
						\cnode(-0.65,1.25){0.2}{TRPP510}
						\ncline{-}{P1}{TRPP1}
						\ncline{-}{P38}{TRPP38}
						\ncline{-}{P27}{TRPP27}
						\ncline{-}{P49}{TRPP49}
						\ncline{-}{P510}{TRPP510}

						\psset{linecolor=myblue}
						\cnode(0.4,0.2){0.2}{PrionP38}
						\cnode(0.6,-2.8){0.2}{PrionP27}
						\cnode(0.2,-0.8){0.2}{PrionP49}
						\cnode(-0.7,-1.7){0.2}{PrionP510}
						\cnode(0.0,-1.4){0.2}{PrionP6}
						\ncline{-}{P38}{PrionP38}
						\ncline{-}{P27}{PrionP27}
						\ncline{-}{P49}{PrionP49}
						\ncline{-}{P510}{PrionP510}
						\ncline{-}{P6}{PrionP6}
					\end{myps}
				\end{myfigure}

				La tâche proposée nécessite deux étapes.
				Selon \mycite[author]{Bowman-1999}, on distingue tout d'abord l'étape de recherche de la cible.
				Pour explorer la molécule afin d'identifier cette cible, les sujets disposent de l'outil \mytool{grab}.
				Lorsque la cible recherchée est identifiée, les sujets entrent dans la seconde étape : la sélection.
				Pour effectuer cette étape de sélection, les sujets disposent de l'outil \mytool{tug}.
				Les outils \mytool{grab} et \mytool{tug} sont décrits dans la \myref{sse-Shaddock-LesOutilsExistants}.
			\end{mysubsection}
			\begin{mysubsection}[sse-exp1-SpecificitesDuProtocoleExperimental]{Spécificités du protocole expérimental}
				L'expérimentation est basée sur le dispositif expérimental décrit dans le \myref{cha-pro-DispositifExperimental}.
				Cependant, certains choix expérimentaux concernant cette expérimentation sont détaillés dans les sections suivantes.
				Les détails à propos de la méthode expérimentale, présents dans la \myref{sec-met-exp1-PremiereExperimentation}, sont résumés dans la \myref{tab-exp1-SyntheseDeLaMethodeExperimentale}.
				\begin{mysubsubsection}[sss-exp1-Materiel]{Matériel}
					Cette première expérimentation propose aux sujets d'effectuer une recherche de \myglos*{glo-Residu} au sein d'une molécule de taille importante.
					Les sujets disposent déjà de deux outils de déformation \mytool{tug}.
					Cependant, un outil d'orientation de la molécule est mis à disposition pour des raisons détaillés dans la \myref{sss-exp1-OutilsDeManipulation}.
					Ce nouvel outil nécessite l'ajout de quelques matériels.

					L'outil d'orientation de la molécule est assuré par un \myOmni associé à l'outil \mytool{grab} \myref*{sse-Shaddock-LesOutilsExistants}.
					L'ajout d'un outil nécessite également l'ajout d'un ordinateur pour faire office de serveur \myacro{acr-VRPN}.
					Dans ce cas, une machine de faible puissance est suffisante en tant que serveur.
					L'interface est placée devant le sujet en charge de cet outil.

					Durant l'expérimentation, il est demandé aux sujets de chercher des \myglos*{glo-Residu} sur la molécule.
					Le \myglos{glo-Residu} à rechercher est affiché aux sujets pendant toute la durée de la tâche.
					Afin de ne pas perturber la scène virtuelle, le \myglos{glo-Residu} n'est pas affiché sur l'écran de vidéoprojection.
					Un écran \myLCD \mynum[pouces]{17} est donc placé sur la table devant les sujets pour afficher en continu la cible à rechercher.

					Pour finir, cette première expérimentation est destinée à mettre en évidence les problèmes de communication qui peuvent survenir lors de la réalisation d'une tâche en collaboration.
					Afin d'analyser la communication entre les différents sujets de l'expérimentation, nous souhaitons enregistrer tous les échanges verbaux.
					C'est pourquoi, un microphone a été installé sur la table, face aux sujets, afin de capter toutes les communications orales.
					L'enregistrement est assuré par le logiciel \myAudacity.
					Un filtrage du bruit de fond est effectué \myafortiori afin de rendre plus audibles les enregistrements.

					La \myref{fig-exp1-SchemaDuDispositifExperimental} est un schéma récapitulatif de la disposition des tous les éléments dans la salle d'expérimentation.
					La \myref{fig-exp1-PhotographieDuDispositifExperimental} est une photographie de la salle d'expérimentation.

					\begin{myfigure}
						\myimage{exp1-schema}
						\mycaption[fig-exp1-SchemaDuDispositifExperimental]{Schéma du dispositif expérimental}
					\end{myfigure}
					\begin{myfigure}
						\myimage{exp1-photo}
						\mycaption[fig-exp1-PhotographieDuDispositifExperimental]{Photographie du dispositif expérimental}
					\end{myfigure}
				\end{mysubsubsection}
				\begin{mysubsubsection}[sss-exp1-Visualisation]{Visualisation}
					Dans cette première expérimentation, c'est la molécule \myPrion qui a été utilisée.
					Cette molécule est suffisamment volumineuse pour ne pas nécessiter d'atome fixes au niveau de la simulation.
					La molécule est donc laissée libre de tout mouvement.
					La \myref{fig-exp1-RepresentationDeLaMoleculePrionPourLExperimentation} est une représentation de cette molécule durant la tâche.

					\begin{myfigure}
						\myimage{exp1-scenario}
						\mycaption[fig-exp1-RepresentationDeLaMoleculePrionPourLExperimentation]{Représentation de la molécule \myPrion pour l'expérimentation}
					\end{myfigure}
				\end{mysubsubsection}
				\begin{mysubsubsection}[sss-exp1-OutilsDeManipulation]{Outils de manipulation}
					Durant cette tâche de recherche, nous donnons aux utilisateurs la possibilté de déformer la molécule à l'aide de l'outil \mytool{tug}.
					Cependant, afin de fournir un moyen d'explorer la molécule sous tous ses angles, il est nécessaire d'avoir également à sa disposition un moyen d'orienter la scène.
					Le cas échéant, tout \myglos{glo-Residu} qui se trouverait derrière la molécule ne pourrait être trouvé qu'après une longue et fastidieuse déformation.
					C'est pourquoi, nous mettons un outil d'orientation de la molécule à disposition des sujets.

					Cet outil, concrétisé par une interface haptique associé à l'outil \mytool{grab}, permet de sélectionner la molécule puis de la déplacer et de l'orienter.
					Aucune modification par rapport à l'outil déjà proposé par \myacro{acr-VMD} n'a été ajoutée.
					Cependant, l'outil n'est pas partagé entre les utilisateurs.
					En effet, dès le début de l'expérimentation, il est demandé aux sujets de choisir celui qui sera en charge de cet outil de manipulation et ceci, tout au long de l'expérimentation.
					Ce choix a été fait pour limiter les conflits entre les deux utilisateurs pendant la phase de recherche et de sélection.
					Il est à noter que pour les \myglos*{glo-Monome}, le sujet n'a accès qu'à un seul outil de déformation et un outil de manipulation.
				\end{mysubsubsection}
				\begin{mytable}
					\mycaption[tab-exp1-SyntheseDeLaMethodeExperimentale]{Synthèse de la méthode expérimentale}
					\newcommand{\mytitlecolumn}[2]{%
						\multirow{#1}*{%
							\begin{minipage}{6em}%
								\raggedleft #2%
							\end{minipage}%
						}
					}
					\newlength{\exponefirstcolumn}
					\newlength{\exponesecondcolumn}
					\setlength{\exponefirstcolumn}{7em}
					\setlength{\exponesecondcolumn}{\textwidth}
					\addtolength{\exponesecondcolumn}{-\exponefirstcolumn}
					\addtolength{\exponesecondcolumn}{-4\tabcolsep}
					\begin{mytabular}{>{\bfseries}p{\exponefirstcolumn}p{\exponesecondcolumn}}
						\mytoprule
						\mytitlecolumn{1}{Tâche}                   & Recherche et sélection de motifs                                             \\
						\mymiddlerule[\heavyrulewidth]
						\mytitlecolumn{4}{Hypothèses}              & \myhypothesis{1} Amélioration des performances en \myglosnl{glo-Binome}      \\
						                                           & \myhypothesis{2} Stratégies variables en fonction des \myglosnl*{glo-Binome} \\
						                                           & \myhypothesis{3} Les sujets préfèrent le travail en \myglosnl{glo-Binome}    \\
						                                           & \myhypothesis{4} La plate-forme est appréciée des utilisateurs               \\
						\mymiddlerule
						\mytitlecolumn{2}{Variables indépendantes} & \myvari{1} Nombre de sujets                                                  \\
						                                           & \myvari{2} \myGlosnl{glo-Residu} à chercher                                  \\
						\mymiddlerule
						\mytitlecolumn{6}{Variables dépendantes}   & \myvard{1} Temps de réalisation                                              \\
						                                           & \myvard{2} Distance entre les espaces de travail                             \\
						                                           & \myvard{3} Communication verbales                                            \\
						                                           & \myvard{4} Affinités entre les sujets                                        \\
						                                           & \myvard{5} Force moyenne appliquée par le sujet                              \\
						                                           & \myvard{6} Réponses qualitatives                                             \\
						\mymiddlerule[\heavyrulewidth]
						\multicolumn{2}{c}{%
							\small%
							\begin{tabular}{^C-C-C}
								\myrowstyle{\bfseries}
								Condition \mycondition{1}         & Condition \mycondition{2}         & Condition \mycondition{3}         \\
								\mymiddlerule
								Sujet~\myuser{A}                  & Sujet~\myuser{A}                  & \myGlosnl{glo-Binome}~\myuser{AB} \\
								\mynum{10}~\myglosnl*{glo-Residu} & \mynum{10}~\myglosnl*{glo-Residu} & \mynum{10}~\myglosnl*{glo-Residu} \\
								\mymiddlerule
								Sujet~\myuser{B}                  & \myGlosnl{glo-Binome}~\myuser{AB} & Sujet~\myuser{A}                  \\
								\mynum{10}~\myglosnl*{glo-Residu} & \mynum{10}~\myglosnl*{glo-Residu} & \mynum{10}~\myglosnl*{glo-Residu} \\
								\mymiddlerule
								\myGlosnl{glo-Binome}~\myuser{AB} & Sujet~\myuser{B}                  & Sujet~\myuser{B}                  \\
								\mynum{10}~\myglosnl*{glo-Residu} & \mynum{10}~\myglosnl*{glo-Residu} & \mynum{10}~\myglosnl*{glo-Residu} \\
							\end{tabular}
						} \\
						\mybottomrule
					\end{mytabular}
				\end{mytable}
			\end{mysubsection}
		\end{mysection}
		\begin{mysection}[sec-exp1-Resultats]{Résultats}
			Cette section présente et analyse l'ensemble des mesures expérimentales de cette première étude concernant la recherche et la sélection sur une tâche complexe de collaboration.
			Les données, confrontées à un test de \mycite[author]{Shapiro-1965}, ne sont pas distribuées selon une loi normale.
			Cependant, un test de \mycite[author]{Brown-1974} permet de confirmer l'\myglos{glo-Homoscedasticite}.
			L'analyse de la variance est alors pratiquée à l'aide d'un test de \mycite[author]{Friedman-1940} adapté pour les \myglos*{glo-VariableIntraSujets} non-paramètriques.

			Il est à noter que les données comparées entre les \myglos*{glo-Monome} et les \myglos*{glo-Binome} ne sont pas du même ordre de grandeur (\mynum{24}~\myglos*{glo-Monome} face à \mynum{12}~\myglos*{glo-Binome}).
			Afin de pouvoir effectuer une comparaison du même ordre de grandeur, les données des sujets ayant fait partie d'un même \myglos{glo-Binome} ont été moyennées.
			Ainsi, pour chaque variable correspond une donnée en \myglos{glo-Monome} et une donnée en \myglos{glo-Binome}.
			\begin{mysubsection}[sse-exp1-AmeliorationDesPerformancesEnBinome]{Amélioration des performances en \myglosnl{glo-Binome}}
				\begin{mysubsubsection}[sss-exp1-AmeliorationDesPerformancesEnBinome-DonneesEtStatistiques]{Données et statistiques}
					\begin{myfigure}
						\psset{xunit=0.0889\textwidth,yunit=0.01cm}
						\begin{myps}(-1.25,-115)(10,425)
							\myaxes(0,10){\myglosnl*{glo-Residu}}(0,400)[100]{temps~(s)}
							\myboxplot{exp1-time-residue.csv}
						\end{myps}
						\mycaption[fig-exp1-TempsDeRealisationParResidu]{Temps de réalisation par \myglosnl{glo-Residu}}
					\end{myfigure}

					La \myref{fig-exp1-TempsDeRealisationParResidu} présente le temps de réalisation \myvard{1} pour l'identification et l'extraction de chaque \myglos{glo-Residu} \myvari{2}.
					L'analyse montre qu'il y a un effet significatif des \myglos*{glo-Residu} \myvari{2} sur le temps de réalisation \myvard{1} (\myanova{exp1-time-residue-anova.tex}).
					Un test post-hoc de \mycite[author]{Mann-1947} avec une correction de \mycite[author]{Holm-1979} permet de déterminer que les \myglos*{glo-Residu} \myresidue{6}, \myresidue{9} et \myresidue{10} obtiennent des temps de réalisation significativement plus longs que les autres \myglos*{glo-Residu}.

					\begin{myfigure}
						\psset{xunit=0.0889\textwidth,yunit=0.01cm}
						\begin{myps}(-1.25,-115)(10,460)
							\myaxes(0,10){\myglosnl*{glo-Residu}}(0,400)[100]{temps~(s)}
							\myboxplot{exp1-time-residue-group.csv}
							\mylegend{\myleg{\myGlosnl{glo-Monome}}{myblue}\myand\myleg{\myGlosnl{glo-Binome}}{myblue!70}}
						\end{myps}
						\mycaption[fig-exp1-TempsDeRealisationComparesMonomeOuBinomeParResidu]{Temps de réalisation comparés (\myglosnl{glo-Monome} ou \myglosnl{glo-Binome}) par \myglosnl{glo-Residu}}
					\end{myfigure}

					La \myref{fig-exp1-TempsDeRealisationComparesMonomeOuBinomeParResidu} présente les temps de réalisation \myvard{1} de chaque \myglos{glo-Residu} \myvari{2} en fonction du nombre de participants \myvari{1}.
					L'analyse ne montre pas d'effet significatif du nombre de participants \myvari{1} sur le temps de réalisation \myvard{1} (\myanova{exp1-time-residue-group-anova.tex}).
					Cependant, en se limitant au groupe des trois \myglos*{glo-Residu} \myresidue{6}, \myresidue{9} et \myresidue{10} identifiés précédemment comme significativement plus longs à trouver et extraire, l'analyse montre un effet significatif du nombre de participants \myvari{1} sur le temps de réalisation \myvard{1} (\myanova{exp1-time-residue-group-anova-restricted.tex}).

					\begin{myfigure}
						\psset{xunit=0.0889\textwidth,yunit=0.01cm}
						\begin{myps}(-1.25,-115)(10,460)
							\myaxes(0,10){\myglosnl*{glo-Residu}}(0,400)[100]{temps~(s)}
							\myboxplot{exp1-timeaudio-residue-searchselection.csv}
							\mylegend{\myleg{Recherche}{myblue}\myand\myleg{Sélection}{myblue!70}}
						\end{myps}
						\mycaption[fig-exp1-TempsDeRechercheEtDeSelectionComparesParResidu]{Temps de recherche et de sélection comparés par \myglosnl{glo-Residu}}
					\end{myfigure}

					La \myref{fig-exp1-TempsDeRechercheEtDeSelectionComparesParResidu} présente les temps de recherche et de sélection par \myglos{glo-Residu} \myvari{2}.
					L'analyse montre un effet significatif des \myglos*{glo-Residu} \myvari{2} sur les temps de recherche (\myanova{exp1-timeaudio-residue-searchselection-anova-search.tex}).
					Un test post-hoc de \mycite[author]{Mann-1947} avec une correction de \mycite[author]{Holm-1979} permet de déterminer que les \myglos*{glo-Residu} \myresidue{9} et \myresidue{10} obtiennent des temps de recherche significativement plus longs que les autres \myglos*{glo-Residu}.
					L'analyse montre également un effet significatif des \myglos*{glo-Residu} \myvari{2} sur les temps de sélection (\myanova{exp1-timeaudio-residue-searchselection-anova-selection.tex}).
					Un test post-hoc de \mycite[author]{Mann-1947} avec une correction de \mycite[author]{Holm-1979} permet de déterminer que le \myglos{glo-Residu} \myresidue{6} obtient un temps de sélection significativement plus long que les autres \myglos*{glo-Residu}.
				\end{mysubsubsection}
				\begin{mysubsubsection}[sss-exp1-AmeliorationDesPerformancesEnBinome-AnalyseEtDiscussion]{Analyse et discussion}
					Les cinq \myglos*{glo-Residu} \myresidue{1}, \myresidue{2}, \myresidue{3}, \myresidue{4} et \myresidue{5} sont au sein de la molécule \myTRPCAGE qui en compte un nombre total relativement limité (\mynum{20}~\myglos*{glo-Residu}).
					Durant la phase d'exploration, les sujets construisent rapidement une carte mentale de la molécule ce qui leur permet de d'identifier rapidement les \myglos*{glo-Residu} recherchés.
					De plus, les faibles contraintes physiques de la molécule (énergie totale du système peu élevée à cause du faible nombre d'atomes) la rende facile à déformer et permet un accès rapide aux structures internes.
					Cela facilite la recherche des \myglos*{glo-Residu} qui sont dans une position interne à la molécule et qui nécessitent une déformation plus importante afin de pouvoir l'extraire.
					Tous ces facteurs rendent les tâches de recherche et de sélection peu complexes sur la molécule \myTRPCAGE ce qui explique des temps de réalisation de la tâche très courts, aussi bien pour les \myglos*{glo-Monome} que pour les \myglos*{glo-Binome}.

					Les cinq \myglos*{glo-Residu} \myresidue{6}, \myresidue{7}, \myresidue{8}, \myresidue{9} et \myresidue{10} sont au sein de la molécule \myPrion qui en compte un nombre total relativement important (\mynum{112}~\myglos*{glo-Residu}).
					La construction complète d'une carte mentale est très complexe à cause du nombre importants d'atomes qui sont continuellement en mouvement (dû à la simulation en temps-réel).
					Les sujets n'étant jamais confronté plus de deux fois à la même tâche (une fois en \myglos{glo-Monome} et une fois en \myglos{glo-Binome}), le phénomène d'apprentissage ne peut pas être effectué.
					En effet, les sujets ne se souviennent pas de la position d'un \myglos{glo-Residu} d'une confrontation à l'autre (contrairement à la molécule \myTRPCAGE pour certains cas).
					Les sujets adoptent une stratégie en plusieurs étapes en fonction de la caractéristique de la tâche et du \myglos{glo-Residu} à trouver.
					Tout d'abord, une recherche exploratoire permet d'identifier les \myglos*{glo-Residu} \myresidue{7} et \myresidue{8} qui se trouvent en position externe.
					Ensuite, lorsque cette première étape exploratoire ne permet pas d'identifier le \myglos{glo-Residu} recherché, les sujets déforment la molécule afin d'accéder aux \myglos*{glo-Residu} \myresidue{6}, \myresidue{9} et \myresidue{10} qui se trouvent en position interne.

					Le travail en \myglos{glo-Binome} comparé au travail en \myglos{glo-Monome} ne montre pas d'amélioration significative bien que la \mypvalue soit très proche du seuil de significativité.
					Cependant, un test post-hoc a permis de d'identifier les \myglos*{glo-Residu} \myresidue{6}, \myresidue{9} et \myresidue{10} comme ayant un temps de réalisation significativement plus long.
					Sur ce groupe de \myglos*{glo-Residu} plus complexes, les \myglos*{glo-Binome} obtiennent une amélioration significative des performances par rapport aux \myglos*{glo-Monome}.
					Ce résultat confirme notre hypothèse \myhypothesis{1} exclusivement sur des tâches de fortes complexité.

					Comme développé dans la procédure expérimentale, le temps de réalisation de la tâche peut être séparé en deux parties : le temps de recherche et le temps de sélection \myref*{fig-exp1-EtapesDeLaCommunicationVerbalePourLaRechercheDUnResidu}.
					Les \myglos*{glo-Residu} \myresidue{9} et \myresidue{10} se distinguent par un temps de recherche significativement plus long que les autres \myglos*{glo-Residu} (excepté \myresidue{6}).
					En effet, ces deux \myglos*{glo-Residu} sont en présence d'autres \myglos*{glo-Residu} similaires au sein de la même molécule \myref*{tab-exp1-ParametresDeComplexiteDesResidus}.
					Ces similarités ont pour effet de monopoliser l'attention des sujets ce qui provoque une hausse significative du temps de recherche du \myglos{glo-Residu} au sein de la molécule.

					De la même façon, le \myglos{glo-Residu} \myresidue{6} se distingue par un temps de sélection significativement plus long que les autres \myglos*{glo-Residu} (excepté \myresidue{9} et \myresidue{10}).
					Ce \myglos{glo-Residu} possède deux atomes de \myatom{Soufre} de couleur jaune.
					Cette particularité aisément identifiable malgré le nombre importants d'atomes de la molécules.
					Le temps de recherche est alors extrêmement court.
					Cependant, ce \myglos{glo-Residu} est positionné au centre de la molécule.
					L'accès au \myglos{glo-Residu} nécessite de \myemph{déplier} en grande partie la molécule afin de pouvoir le sélectionner et l'extraire.

					L'analyse du rapport entre les temps de recherche et de sélection met en évidence trois configurations en fonction des différents \myglos*{glo-Residu} :
					\begin{description}
						\item[Temps de recherche et de sélection égaux]
							Les sujets ont un temps similaire alloué à l'étape de recherche et de sélection.
							Les \myglos*{glo-Residu} concernés ne présentent pas de forte complexité (tous les \myglos*{glo-Residu} de la molécule \myTRPCAGE et les \myglos*{glo-Residu} \myresidue{7} et \myresidue{8} de la molécule \myPrion) et sur lesquels, le travail collaboratif n'améliore pas les performances.
						\item[Temps de recherche prédominant]
							Les sujets ont un temps important alloué à l'identification du \myglos{glo-Residu} recherché.
							Une fois identifié, le \myglos{glo-Residu} est facile à sélectionner puis à extraire.
							Les \myglos*{glo-Residu} \myresidue{9} et \myresidue{10} sont concernés.
							Dans cette configuration, le travail collaboratif améliore significativement les performances.
							En effet, l'étape de recherche est fortement parallélisable : l'espace de recherche est séparé entre les sujets (stratégie \myemph{diviser pour régner}).
						\item[Temps de sélection prédominant]
							Les sujets ont un temps important alloué à la sélection et à l'extraction du \myglos{glo-Residu} recherché.
							Le \myglos{glo-Residu} est rapidement identifié mais il est difficile d'y accéder directement.
							Une phase de déformation est nécessaire pour le sélectionner.
							Le \myglos{glo-Residu} \myresidue{6} est concerné.
							Dans cette configuration, le travail collaboratif améliore significativement les performances.
							En effet, l'étape de déformation bénéficie d'une action coordonnée entre les sujets : l'effort déployé est alors plus important et le contrôle sur la déformation meilleur ce qui permet une réalisation de la tâche plus rapide.
					\end{description}
				\end{mysubsubsection}
			\end{mysubsection}
			\begin{mysubsection}[sse-exp1-StrategiesDeTravail]{Stratégies de travail}
				\begin{mysubsubsection}[sss-exp1-StrategiesDeTravail-DonneesEtAnalyses]{Données et analyses}
					Dans cette section, les données concernent exclusivement les \myglos*{glo-Binome}.
					Une numérotation des \myglos*{glo-Binome} a été effectuée afin de pouvoir comparer les mesures effectuées et ainsi, étudier les différentes stratégies.

					\begin{myfigure}
						\psset{xunit=0.074\textwidth,yunit=0.15cm}
						\begin{myps}(-1.5,-7)(12,22)
							\myaxes(0,12){groupes}(0,20)[4]{distance~(mm)}
							% Once header are readed, they are defined for other barplot
							% That's why barplots without headers are in first position
							\mybarplot[header=false,barstyle=third-barstyle]{exp1-diff-groups3.csv}
							\mybarplot[header=false,barstyle=second-barstyle]{exp1-diff-groups2.csv}
							\mybarplot[barstyle=first-barstyle]{exp1-diff-groups1.csv}
							\psset{linecolor=myred,linewidth=1pt,linestyle=solid}
							\psline(0,14)(12,14)
							\psline(0,8)(12,8)
							\psset{linewidth=0.1pt,linecolor=white,fillstyle=solid,fillcolor=myred}
							\uput[180](12,5){\pscharpath{\LARGE\bf\sffamily Champ proche}}
							\uput[180](12,11){\pscharpath{\LARGE\bf\sffamily Champ voisin}}
							\uput[180](12,17){\pscharpath{\LARGE\bf\sffamily Champ distant}}
						\end{myps}
						\mycaption[fig-exp1-DistanceMoyenneEntreLesSujetsPourChaqueBinomeSurLesResidusSixNeufEtDix]{Distance moyenne entre les sujets pour chaque \myglosnl{glo-Binome} sur les \myglosnl*{glo-Residu} \myresidue{6}, \myresidue{9} et \myresidue{10}}
					\end{myfigure}

					La \myref{fig-exp1-DistanceMoyenneEntreLesSujetsPourChaqueBinomeSurLesResidusSixNeufEtDix} présente la distance moyenne entre les espaces de travail \myvard{2} de chaque \myglos{glo-Binome}.
					Les \myglos*{glo-Binome} peuvent être classés en trois groupes : \myemph{espace distant}, \myemph{espace voisin} et \myemph{espace proche}.

					\begin{myfigure}
						\psset{xunit=0.074\textwidth,yunit=0.5cm}
						\begin{myps}(-1.5,-2)(12,5.5)
							\myaxes(0,12){groupes}(0,5)[1]{affinité~(\mynum{1}--\mynum{5})}
							\mybarplot{exp1-affinity-groups.csv}
						\end{myps}
						\mycaption[fig-exp1-AffiniteEntreLesSujetsPourChaqueBinome]{Affinité entre les sujets pour chaque \myglosnl{glo-Binome}}
					\end{myfigure}

					La \myref{fig-exp1-AffiniteEntreLesSujetsPourChaqueBinome} présente les affinités \myvard{4} de chaque \myglos{glo-Binome}.
					Les notes, comprises entre un et cinq, montre que les \myglos*{glo-Binome} choisis ont des affinités relativement variées.
					L'affinité entre les sujets du groupe \mygroup{1} est très basse contrairement aux groupes \mygroup{8} et \mygroup{12} pour lesquelles l'affinité est très élevée.

					\begin{myfigure}
						\psset{xunit=0.074\textwidth,yunit=0.01cm}
						\begin{myps}(-1.5,-110)(12,325)
							\myaxes(0,12){groupes}(0,300)[50]{temps~(s)}
							\mybarplot{exp1-time-groups.csv}
						\end{myps}
						\mycaption[fig-exp1-TempsDeRealisationEntreLesSujetsPourChaqueBinome]{Temps de réalisation entre les sujets pour chaque \myglosnl{glo-Binome}}
					\end{myfigure}

					La \myref{fig-exp1-TempsDeRealisationEntreLesSujetsPourChaqueBinome} présente les temps de réalisation \myvard{1} de chaque \myglos{glo-Binome}.
					Le temps de réalisation de \mygroup{1} est particulièrement important (plus d'une fois et demi les autres groupes les plus longs).
					À l'opposé, on note que \mygroup{2}, \mygroup{3} et \mygroup{4} obtiennent des temps de réalisation extrêmement bas.

					\begin{myfigure}
						\psset{xunit=0.074\textwidth,yunit=0.05cm}
						\begin{myps}(-1.5,-22)(12,75)
							\myaxes(0,12){groupes}(0,70)[10]{temps~(s)}
							\mybarplot{exp1-timeaudio-groups.csv}
						\end{myps}
						\mycaption[fig-exp1-TempsDeCommunicationVerbaleEntreLesSujetsPourChaqueBinome]{Temps de communication verbale entre les sujets pour chaque \myglosnl{glo-Binome}}
					\end{myfigure}

					La \myref{fig-exp1-TempsDeCommunicationVerbaleEntreLesSujetsPourChaqueBinome} présente les temps de communication verbale \myvard{3} de chaque \myglos{glo-Binome}.
					\mygroup{2}, \mygroup{3} et \mygroup{4} ont des temps de communication verbale inférieurs à \mynum[s]{20}.
					À l'opposé, \mygroup{1}, \mygroup{5} et \mygroup{11} ont des temps de communication verbale qui approche les \mynum[s]{60}.

					\begin{myfigure}
						\psset{xunit=0.074\textwidth,yunit=0.03cm}
						\begin{myps}(-1.5,-35)(12,120)
							\myaxes(0,12){groupes}(0,100)[25]{temps~(\%)}
							\mybarplot{exp1-timeaudio-groups-searchselection.csv}
							\mylegend{\myleg{Recherche}{myblue}\myand\myleg{Sélection}{myblue!70}}
						\end{myps}
						\mycaption[fig-exp1-PourcentageDeTempsDeCommunicationVerbalePendantLaRechercheEtLaSelectionEntreLesSujetsPourChaqueBinome]{Pourcentage de temps de communication verbale pendant la recherche et la sélection entre les sujets pour chaque \myglosnl{glo-Binome}}
					\end{myfigure}

					La \myref{fig-exp1-PourcentageDeTempsDeCommunicationVerbalePendantLaRechercheEtLaSelectionEntreLesSujetsPourChaqueBinome} présente les pourcentages de temps de communication verbale durant la phase de recherche et durant la phase de sélection de chaque \myglos{glo-Binome} par rapport au temps total de réalisation de la tâche.
					Le pourcentage représente le rapport du temps de communication verbale durant la phase recherche ou de sélection rapporté respectivement au temps total de la phase de recherche ou de sélection.
					Les \myglos*{glo-Binome} \mygroup{1} à \mygroup{4} ainsi que \mygroup{9} communiquent plus durant la phase de sélection.
					Les \myglos*{glo-Binome} \mygroup{5} à \mygroup{8} et \mygroup{10} à \mygroup{12} communiquent plus durant la phase de recherche.
					Notons également que \mygroup{1} communique assez peu par rapport aux autres \myglos*{glo-Binome}.

					\begin{myfigure}
						\psset{xunit=0.074\textwidth,yunit=1cm}
						\begin{myps}(-1.5,-1)(12,3.6)
							\myaxes(0,12){groupes}(0,3)[1]{force~(N)}
							\mybarplot{exp1-force-groups-meandiff.csv}
							\mylegend{\myleg{Force moyenne}{myblue}\myand\myleg{Différence de force}{myblue!70}}
						\end{myps}
						\mycaption[fig-exp1-ForceMoyenneEtDifferenceDeForceEntreLesSujetsPourChaqueBinome]{Force moyenne et différence de force entre les sujets pour chaque \myglosnl{glo-Binome}}
					\end{myfigure}

					La \myref{fig-exp1-ForceMoyenneEtDifferenceDeForceEntreLesSujetsPourChaqueBinome} représente la force moyenne appliquée par les sujets \myvard{5} et la différence de force entre les sujets.
					La différence de force est la différence entre les forces moyennes de chaque sujet.
					\mygroup{9} et \mygroup{11} apporte un effort moyen très important par rapport aux autres \myglos*{glo-Binome}.
					\mygroup{2}, \mygroup{3} et \mygroup{4} apporte un effort moyen important également tout en ayant une différence de force quasiment nulle entre les deux membres du \myglos{glo-Binome}.

					L'ensemble des résultats et analyses précédentes permet de différencier les \myglos*{glo-Binome} ce qui confirme notre hypothèse \myhypothesis{2}.
					Les \myglos*{glo-Binome} se différencient pas des stratégies de travail variables.
					Les sections suivantes caractérisent les différentes stratégies de travail en fonction de plusieurs paramètres (distance entre les espaces de travail, affinités, temps de réalisation de la tâche, communication verbale, forces moyennes appliquées).
					Trois stratégies sont décrites, distinguées en fonction des distances entre les espaces de travail.
					\begin{description}
						\item[Collaboration en champ proche] pour les distances inférieures à \mynum[mm]{8};
						\item[Collaboration en champ voisin] pour les distances comprises entre \mynum[mm]{8} et \mynum[mm]{14};
						\item[Collaboration en champ distant] pour les distances supérieures à \mynum[mm]{14}.
					\end{description}
					Les mesures de distances sont données dans le référentiel du monde réel.
				\end{mysubsubsection}
				\begin{mysubsubsection}[sss-exp1-CollaborationEnChampProche]{Collaboration en champ proche}
					\begin{myparagraph}[par-exp1-CollaborationEnChampProche-Caracteristiques]{Caractéristiques}
						La collaboration en champ proche, inférieures à \mynum[mm]{8}, correspond, dans l'environnement virtuel, à des distances inférieures à \mynum[\AA]{10} ce qui est environ l'envergure d'un \myglos{glo-Residu}\footnote{\og \AA \fg désigne l'\myangstrom qui est une unité de mesure telle que $\mynum[\AA]{1} = \mynum[m]{e-10}$}.
						\mynum{8}~\myglos*{glo-Binome} sur \mynum{12} sont concernés par cette catégorie (\myglos*{glo-Binome} \mygroup{5}, \mygroup{6}, \mygroup{7}, \mygroup{8}, \mygroup{9}, \mygroup{10}, \mygroup{11} et \mygroup{12}).
						Étant donné la distance inférieure à \mynum[\AA]{10}, les \myglos*{glo-Binome} concernés manipulent en collaboration étroite sur les mêmes \myglos*{glo-Residu}.
						Ces \myglos*{glo-Binome} se caractérisent par une forte affinité ($\mu = 4$) : ce sont des collègues ou des amis \myref*{fig-exp1-AffiniteEntreLesSujetsPourChaqueBinome}.
						D'après la \myref{fig-exp1-TempsDeRealisationEntreLesSujetsPourChaqueBinome}, ces \myglos*{glo-Binome} obtiennent des temps de réalisation de la tâche moyens.
					\end{myparagraph}
					\begin{myparagraph}[par-exp1-CollaborationEnChampProche-PartageDeLaTache]{Partage de la tâche}
						La \myref{fig-exp1-ForceMoyenneEtDifferenceDeForceEntreLesSujetsPourChaqueBinome} montre de fortes disparités entre les \myglos*{glo-Binome} concernant la force moyenne appliquée pendant la manipulation.
						Des observations pendant l'expérimentation ont permis d'identifier deux stratégies adoptées par les sujets : \og par contrôle \fg où les deux sujets effectuent la même action pour obtenir un meilleur contrôle sur les structures manipulées; \og par guidage \fg où un des deux sujets indique à son partenaire la déformation à effectuer ou la position à atteindre.
						Le partage des tâches est donc très différent selon les initiatives de chacun des sujets.
						Cependant, les \myglos*{glo-Binome} se distribuent mal la charge de travail comme le montre les différences importantes entre les forces appliquées par les deux sujets \myref*{fig-exp1-ForceMoyenneEtDifferenceDeForceEntreLesSujetsPourChaqueBinome}.
						En effet, seul un des deux sujets réalise une grande partie de la tâche à réaliser.
						Le second sujet joue plutôt le rôle du suiveur.
					\end{myparagraph}
					\begin{myparagraph}[par-exp1-CollaborationEnChampProche-Communication]{Communication}
						Les temps de communication verbale sur la \myref{fig-exp1-PourcentageDeTempsDeCommunicationVerbalePendantLaRechercheEtLaSelectionEntreLesSujetsPourChaqueBinome} montrent une disparité entre les sujets.
						Les \myglos*{glo-Binome} de ce groupe passent plus de temps à communiquer pendant la phase de recherche que pendant la phase de sélection (excepté pour \mygroup{9}) ce qui met en évidence les difficultés du travail en champ proche liées aux nombreux \myglos*{glo-ConflitDeCoordination} pendant la phase de recherche.
						En effet, les \myglos*{glo-Binome} doivent coordonner leurs mouvements de manipulation pour déplacer un \myglos{glo-Residu} et cette coordination nécessite une communication verbale importante.
						La collaboration est alors étroitement couplée mais il en résulte une perte de temps à cause du temps alloué à la communication.
						D'ailleurs, l'analyse des communications verbales a permis de mettre en évidence de nombreuses incompréhensions dans l'inter-référencement (\og Pas dans cette direction \fg, \og Pas ici mais ici \fg, \og C'est juste derrière \fg, \myetc).
						En effet, la grande complexité des tâches ainsi qu'une conscience incomplète de l'environnement et de l'état de son partenaire provoque des inter-référencements imprécis entrainant une mauvaise coordination.
						Ces \myglos*{glo-ConflitDeCoordination} et ces incompréhensions diminuent les performances globales du \myglos{glo-Binome}.
					\end{myparagraph}
				\end{mysubsubsection}
				\begin{mysubsubsection}[sss-exp1-CollaborationEnChampVoisin]{Collaboration en champ voisin}
					\begin{myparagraph}[par-exp1-CollaborationEnChampVoisin-Caracteristiques]{Caractéristiques}
						La collaboration en champ voisin, comprises entre \mynum[mm]{8} et \mynum[mm]{14}, correspond, dans l'environnement virtuel, à des distances de l'ordre de \myglos*{glo-Residu} voisins (entre \mynum[\AA]{10} et \mynum[\AA]{20}).
						\mynum{3}~\myglos*{glo-Binome} sur \mynum{12} se trouvent dans cette catégorie (\myglos*{glo-Binome} \mygroup{2}, \mygroup{3} et \mygroup{4}).
						Ces \myglos*{glo-Binome} travaillent en collaboration relativement étroite sur des \myglos*{glo-Residu} voisins.
						La \myref{fig-exp1-CouplagePhysiqueEtStructurelEntreLesResidus} montre la dépendance physique ou structurelle entre les \myglos*{glo-Residu} voisins.
						En effet, les \myglos*{glo-Residu} interagissent entre eux à travers diverses forces physiques : plus les distances sont courtes, plus les contraintes physiques sont fortes.
						La \myref{fig-exp1-AffiniteEntreLesSujetsPourChaqueBinome} montre que les \myglos*{glo-Binome} concernés ont des affinités moyennes ($\mu = 3$) : ce sont des collègues de bureau ou d'équipe ne travaillant pas forcément sur les mêmes projets.
						Ces \myglos*{glo-Binome} obtiennent de très bonnes performances sur les temps de réalisation de la tâche \myref{fig-exp1-TempsDeRealisationEntreLesSujetsPourChaqueBinome}.

						\begin{myfigure}
							\setlength{\mywidth}{10pc}
							\setlength{\myheight}{10pc}
							\psset{xunit=\mywidth,yunit=\myheight}
							\begin{myps}(-1,-0.2)(1.15,1)
								\psset{ref=c}
								\rput(0.5,0.5){\myimage[width=\mywidth,angle=90]{exp1-trp-zipper}}
								\rput(0.35,-0.1){%
									\rnode{connected-residues-label}{%
										\begin{tabular}{c}%
											Connected residues\\[-1ex]%
											\textcolor{black!70}{\scriptsize Structural dependencies}%
										\end{tabular}%
									}%
								}
								\rput(0.2,0.3){\pnode{connected-residues1}}
								\rput(0.5,0.3){\pnode{connected-residues2}}
								\rput(-0.5,0.7){%
									\rnode{close-macro-label}{%
										\begin{tabular}{c}%
											Close macrostructures\\[-1ex]%
											\textcolor{black!70}{\scriptsize Physical dependencies}%
										\end{tabular}%
									}%
								}
								\rput(0,0.9){\pnode{close-macro1}}
								\rput(-0.1,0.5){\pnode{close-macro2}}
								\nccurve[angleA=135,angleB=-90]{->}{connected-residues-label}{connected-residues1}
								\nccurve[angleA=45,angleB=-90]{->}{connected-residues-label}{connected-residues2}
								\nccurve[angleA=90,angleB=180]{->}{close-macro-label}{close-macro1}
								\nccurve[angleA=-90,angleB=180]{->}{close-macro-label}{close-macro2}
							\end{myps}
							\mycaption[fig-exp1-CouplagePhysiqueEtStructurelEntreLesResidus]{Couplage physique et structure entre les \myglosnl*{glo-Residu}}
						\end{myfigure}
					\end{myparagraph}
					\begin{myparagraph}[par-exp1-CollaborationEnChampVoisin-PartageDeLaTache]{Partage de la tâche}
						La \myref{fig-exp1-ForceMoyenneEtDifferenceDeForceEntreLesSujetsPourChaqueBinome} illustre une bonne répartition des efforts entre les deux membres du \myglos{glo-Binome}.
						En effet, la force moyenne est assez élevée par rapport à la plupart des autres \myglos*{glo-Binome} ce qui montre qu'aucun des deux sujets n'est moins actif (ce qui entraînerait une force moyenne moins élevée).
						La différence des forces moyennes quasi-nulle entre les deux sujets confirme ce résultat.
						Ceci s'explique par une bonne coordination entre les sujets pendant laquelle les deux membres du \myglos{glo-Binome} vont effectuer des actions complémentaires mais de même intensité.
						La stratégie adoptée peut être définie comme une stratégie \myemph{par manipulation complémentaire} : les deux sujets sont attentifs aux actions de leur partenaire afin d'avoir un meilleur contrôle du processus de déformation par une coordination améliorée.
					\end{myparagraph}
					\begin{myparagraph}[par-exp1-CollaborationEnChampVoisin-Communication]{Communication}
						La communication verbale est faible comme le montre la \myref{fig-exp1-TempsDeCommunicationVerbaleEntreLesSujetsPourChaqueBinome}.
						La manipulation en champ voisin permet d'être continuellement conscient des actions du partenaire (grâce à la vision périphérique) ce qui limite les communications verbales.
						Cependant, les sujets manipulent des \myglos*{glo-Residu} différents restreignant ainsi les \myglos*{glo-ConflitDeCoordination} par rapport à la collaboration en champ proche.
						De plus, la \myref{fig-exp1-PourcentageDeTempsDeCommunicationVerbalePendantLaRechercheEtLaSelectionEntreLesSujetsPourChaqueBinome} montre un nombre de \myglos*{glo-ConflitDeCoordination} plus faible pendant la phase de recherche.
						En effet, la communication verbale est nettement moins importante pendant la phase de recherche que pendant la phase de sélection.
						L'analyse des communication verbales met en évidence des phases de communication de coordination (\og Maintenant, prends ça \fg, \og peux-tu m'aider ici ? \fg, \og Bien ! \fg, \myetc).
						Les performances des \myglos*{glo-Binome} travaillant en champ voisin sont relativement élevées bien que quelques \myglos*{glo-ConflitDeCoordination} similaires à ceux rencontrés dans une collaboration en champ proche soient présents.
						Cependant, le nombre de \myglos*{glo-ConflitDeCoordination} est plus limité.
					\end{myparagraph}
				\end{mysubsubsection}
				\begin{mysubsubsection}[sss-exp1-CollaborationEnChampDistant]{Collaboration en champ distant}
					\begin{myparagraph}[par-exp1-CollaborationEnChampDistant-Caracteristiques]{Caractéristiques}
						La collaboration en champ voisin, supérieures à \mynum[mm]{14}, correspond, dans l'environnement virtuel, à des \myglos*{glo-Residu} sans interaction physique (supérieur à \mynum[\AA]{20}).
						\mynum{1}~\myglos{glo-Binome} sur \mynum{12} est concerné par cette catégorie (\myglos{glo-Binome} \mygroup{1}).
						Ce \myglos{glo-Binome} travaille de façon faiblement couplée.
						En effet, les membres de ce \myglos{glo-Binome} travaillent de façon complétement indépendante, en limitant au maximum le nombre d'interactions.
						Les affinités des membres de ce \myglos{glo-Binome} sont très faibles \myref*{fig-exp1-AffiniteEntreLesSujetsPourChaqueBinome} : les membres ne se connaissent presque pas.
						Le \myglos{glo-Binome} obtient de très mauvaises performances en ce qui concerne le temps de réalisation de la tâche comme le montre la \myref{fig-exp1-TempsDeRealisationEntreLesSujetsPourChaqueBinome}.
					\end{myparagraph}
					\begin{myparagraph}[par-exp1-CollaborationEnChampDistant-PartageDeLaTache]{Partage de la tâche}
						La \myref{fig-exp1-ForceMoyenneEtDifferenceDeForceEntreLesSujetsPourChaqueBinome} montre un effort moyen appliqué par les \myglos*{glo-Binome} peu élevé (comparé aux stratégies de collaboration en champ voisin).
						De plus, les forces moyennes appliquées par chacun des deux sujets sont très inégales.
						Il y a une mauvaise répartition de la charge de travail au sein du \myglos{glo-Binome}.
					\end{myparagraph}
					\begin{myparagraph}[par-exp1-CollaborationEnChampDistant-Communication]{Communication}
						La \myref{fig-exp1-TempsDeCommunicationVerbaleEntreLesSujetsPourChaqueBinome} montre que le temps de communication verbale est assez important.
						Cependant, le temps de réalisation étant nettement plus important, le taux de communication verbale est beaucoup plus faible que les autres groupes \myref*{fig-exp1-PourcentageDeTempsDeCommunicationVerbalePendantLaRechercheEtLaSelectionEntreLesSujetsPourChaqueBinome}.
						En effet, les membres du \myglos{glo-Binome} travaillent à distance et ont peu d'interactions entre eux.
						Le peu d'interaction permet de limiter le nombre de \myglos*{glo-ConflitDeCoordination} ce qui implique le peu de communication verbale comme on peut le voir sur la \myref{fig-exp1-PourcentageDeTempsDeCommunicationVerbalePendantLaRechercheEtLaSelectionEntreLesSujetsPourChaqueBinome}.
						Cette figure montre également que ce \myglos{glo-Binome} communique plus dans les phases de sélection que dans les phases de recherche.
						En effet, les phases de sélection forcent une collaboration étroite (spécificité de la tâche proposée) et favorisent les \myglos*{glo-ConflitDeCoordination}.
						Cependant, les phases de recherche permettent aux sujets de manipuler de manière distante.
						Ainsi, ils se définissent leur propre espace de travail mais également leur propre stratégie en fonction des événement locaux à leur espace de travail.
						Pourtant, la phase de sélection nécessite une collaboration étroite et si les stratégies sont différentes, il en résulte de mauvaises performances dû au temps important pour se coordonner à nouveau.
					\end{myparagraph}
				\end{mysubsubsection}
				\begin{mysubsubsection}[sss-exp1-SyntheseDesStrategiesDeTravail]{Synthèse des stratégies de travail}
					Les \myglos*{glo-Binome} sont susceptibles d'adopter une des trois stratégies de travail vues dans les sections précédentes.
					Pour certaines, les interactions en champ distants semblent convenir mais au détriment des performances : la collaboration est quasiment inexistante.
					D'autres \myglos*{glo-Binome} interagissent en champ proches et obtiennent des performances moyennes : la collaboration est étroitement couplée mais souffre des nombreux \myglos*{glo-ConflitDeCoordination}.

					Cependant, ce sont les interactions en champ voisins qui produisent les meilleures performances.
					En effet, les \myglos*{glo-ConflitDeCoordination} sont plus limités que pour des interactions en champ proche mais la collaboration est tout de même couplée.
					Les résultats montrent à la fois de bonnes performances en terme de temps de réalisation mais aussi en terme de répartition des charges de travail tout en limitant les communication verbales.
					La plupart du temps, les communications verbales sont destinées à la résolution de \myglos*{glo-ConflitDeCoordination} : elles sont très chronophages et peuvent être évitées.
					C'est pour cette raison que nous proposerons des outils haptiques pour améliorer cette gestion des \myglos*{glo-ConflitDeCoordination} \myref*{cha-TravailCollaboratifAssisteParHaptique}.
				\end{mysubsubsection}
			\end{mysubsection}
			\begin{mysubsection}[sse-exp1-ResultatsQualitatifs]{Résultats qualitatifs}
				Les résultats qualitatifs sont constitués de deux parties.
				La première permet de déterminer les impressions des sujets concernant la collaboration, les rôles et efficacité de chacun pendant la tâche.
				La seconde partie a pour but d'évaluer la plate-forme\footnote{L'échelle de notation comprise entre \mynum{1} à \mynum{5} mais les moyennes ont été normalisées entre \mynum{0} et \mynum{4}.}.
				\begin{mysubsubsection}[sss-exp1-EvaluationDuTravailEnCollaboration]{Évaluation du travail en collaboration}
					Les résultats du questionnaire montre qu'une majorité des sujets de cette expérimentation ont apprécié et préféré la réalisation de la tâche en configuration collaborative (\mysummary{exp1-evaluation-group.tex}).
					De plus, le sentiment d'effectuer une tâche en collaboration est fort.
					L'hypothèse \myhypothesis{3} est confirmée par les sujets qui préfèrent le travail en collaboration que le travail en \myglos{glo-Monome}.

					Durant les tâches collaboratives, les sujets considèrent qu'ils ont effectivement contribués à la réalisation de la tâche (\mysummary{exp1-evaluation-help.tex}).
					Cependant, les sujets considèrent qu'ils ne se sont imposés ni en \myglos{glo-Meneur} ou ni en \myglos{glo-Suiveur} (\mysummary{exp1-evaluation-leader.tex}).
					En effet, des questions supplémentaires ont permis de mettre en évidence que chaque sujet a tendance à surestimer le rôle du partenaire ($\approx\mynum[\%]{70}$).

					Concernant la communication, les participants estiment qu'ils exploitent principalement la communication verbale (\mysummary{exp1-evaluation-verbal.tex}) et, dans une proportion plus faible mais tout de même importante, virtuelle (\mysummary{exp1-evaluation-virtual.tex}).
					En ce qui concerne la communication gestuelle, ils la considèrent quasiment inexistante (\mysummary{exp1-evaluation-gestural.tex}).

					La communication gestuelle n'est pas ou peu utilisée.
					La principale raison est la difficulté de communiquer avec des gestes lorsque les mains sont occupées par la manipulation.
					Les sujets ont rapidement adopté la désignation virtuelle qui est plus précise et plus adaptée dans les phases de désignation qui constituent la plupart des besoins de communication.
					La communication verbale reste le principal moyen de communication : c'est la manière la plus naturelle de communiquer.
					Cependant, il vient aussi en soutien de la désignation virtuelle.
					En effet, aucun outil visuel ou haptique n'a été fourni pour effectuer des désignations et le curseur ne suffit pas toujours à remplir cette mission.
				\end{mysubsubsection}
				\begin{mysubsubsection}[sss-exp1-EvaluationDuSysteme]{Évaluation du système}
					L'évaluation du système en terme d'utilisabilité est relativement satisfaisante.
					En effet, en ce qui concerne les graphismes et les effets visuels, les participants les ont trouvé accessibles (\mysummary{exp1-platform-visual-intuitive.tex}).
					De la même façon, l'utilisabilité des moyens d'interaction avec le système sont bien notés (\mysummary{exp1-platform-interaction-intuitive.tex}).
					En terme de confort d'utilisation, les effets visuels (\mysummary{exp1-platform-visual-confortable.tex}) et les interactions (\mysummary{exp1-platform-interaction-confortable.tex}) sont bien évalués également.

					Là encore, les résultats permettent de valider l'hypothèse \myhypothesis{4}.
					La plate-forme est relativement bien évaluée.
					Il semble cependant nécessaire d'apporter encore des améliorations afin de répondre au mieux aux attentes des utilisateurs.

					Ces résultats sont cependant à nuancer.
					Les écart-types sont relativement élevés ce qui veut dire qu'il y a de fortes disparités dans ces notations entre les différents sujets : certains sujets se sont déclarés plutôt insatisfaits concernant le confort (visuel : \mynum{2}, interaction : \mynum{2}).
					De plus, les outils proposés pendant cette expérimentation sont relativement simples et peu envahissants.
					Des outils plus complexes, plus informatifs seraient peut-être moins intuitifs au premier abord et pourrait mener à un inconfort.
				\end{mysubsubsection}
			\end{mysubsection}
		\end{mysection}
		\begin{mysection}[sec-exp1-Synthese]{Synthèse}
			\begin{mysubsection}[sse-exp1-ResumeDesResultats]{Résumé des résultats}
				Dans ce chapitre, nous avons observé et comparé les performances de \myglos*{glo-Monome} et de \myglos*{glo-Binome} pendant une tâche de recherche et de sélection sur une simulation moléculaire en temps-réel.
				L'objectif était de montrer l'intérêt des approches collaboratives pour l'amélioration des performances et d'identifier les différentes stratégies de travail.
				De plus, il fallait valider la pertinence de la plate-forme mise en place.

				Les approches collaboratives ont prouvé leur intérêt, notamment sur les tâches les plus complexes.
				Cependant, la complexité d'une tâche est relativement difficile à établir.
				Au-delà des facteur de position, de couleur ou de forme, le nombre d'atomes de la molécule (et donc le nombre de \myglos*{glo-Residu}) semble jouer un rôle important dans cette complexité.
				Un grand nombre d'atomes surcharge l'environnement virtuel qui difficile à appréhender.
				Un deuxième facteur de complexité à prendre en compte est l'amplitude des contraintes physiques de la molécule.
				Certaines zones de la molécule sont dans un état de stabilité tel qu'il est difficile d'en déformer les \myglos*{glo-Residu}.
				L'ensemble de ces contraintes rend pertinent une approche collaborative pour la réalisation d'une tâche de nature complexe.

				En observant et en analysant les différentes stratégies de travail, il ressort que les interactions en champ proche et les interactions en champ distant ne sont pas des stratégies très performantes.
				En effet, le nombre de \myglos*{glo-ConflitDeCoordination} durant les interactions en champ proche est très important alors que le potentiel de la collaboration est perdu dans des interaction en champ distant.
				Ce sont les interactions en champ voisin qui offre les meilleures performances, générant un bon compromis en terme de communication et de gestion des \myglos*{glo-ConflitDeCoordination}.

				Enfin, il paraît nécessaire d'avoir de bonnes relations sociales avec ces partenaires.
				Les résultats montrent de façon évidente que tout déséquilibre dans le groupe mène à des performances dégradées.
			\end{mysubsection}
			\begin{mysubsection}[sse-exp1-Conclusion]{Conclusion}
				Basés sur les résultats précédents, certaines perspectives assez évidentes s'imposent et ont guidé les expérimentations qui suivent.
				Tout d'abord, il semble nécessaire de proposer des tâches suffisamment complexes pour le travail collaboratif apporte une amélioration des performances.
				Ceci se traduit soit par des tâches à fortes zones de contraintes \myref*{cha-DeformationCollaborativeDeMolecule} ou par la manipulation de molécules de taille importante \myref*{cha-LaDynamiqueDeGroupe}.

				Les différentes stratégies observées ont permis de mettre en évidence l'intérêt de la collaboration en champ voisin.
				Il semble nécessaire de favoriser ce type de collaboration par des tâches stimulantes et des outils d'interaction adaptés.

				L'évaluation qualitative par questionnaire apporte également de nombreuses réponses intéressantes.
				Tout d'abord, les sujets ont mis en avant la communication virtuelle dans l'\myacro{acr-EVC} au détriment de la communication gestuelle.
				Tout d'abord, les sujets ont mis en avant un élément primordial de la communication : la modalité virtuelle est importante (désignation par exemple).
				Des observations durant les phases expérimentales nous ont permis de constater que ce moyen de communication est principalement utilisé pour des actions de désignation.
				Fournir des outils adaptés aux contraintes de la désignation en environnement complexe devient une nécessité.

				Enfin, ces évaluations qualitatives ont permis de valider l'\myacro{acr-EVC} proposé.
				Des améliorations sont cependant nécessaires en ce qui concerne le rendu visuel et les interactions.
				De nombreux sujets ont par exemple demandé une mise en surbrillance du \myglos{glo-Residu} survolé.
				Une assistance haptique pour la sélection est également une des améliorations possibles.
			\end{mysubsection}
		\end{mysection}
	\end{mychapter}
	\begin{mychapter}[cha-DeformationCollaborativeDeMolecule]{Déformation collaborative de molécule}
		\begin{mysection}[sec-exp2-Introduction]{Introduction}
			La précédente expérimentation nous a permis d'étudier les premières \myacro*{acr-PCV} que sont la \myemph{recherche} et la \myemph{sélection}.
			Afin de compléter notre étude, nous souhaitons à présent nous intéresser à la \myemph{déformation}.
			En effet, on trouve déjà des environnements virtuels permettant de manipuler des molécules rigides pour effectuer un \myglos{glo-AmarrageMoleculaire} comme les travaux de \mycite[author]{Levine-1997} ou encore de \mycite[author]{Ferey-2008a}.
			Cependant, un myglos{glo-AmarrageMoleculaire} nécessite de pouvoir déformer les molécules.
			Ceci est rendu possible par l'avénement des simulations moléculaires interactives en temps-réel, notamment avec \myacro{acr-IMD} développé par \mycite[author]{Stadler-1997}.
			Plus récemment, \mycite[author]{Delalande-2009} ont également amené une pierre à l'édifice avec \myMDDriver pour permettre une simulation moléculaire en temps-réel basée sur différents moteurs de simulation (\myacro{acr-NAMD} ou \myGromacs).
			Puis, \mycite[author]{Delalande-2010} améliorent la manipulation et la déformation interactive par l'utilisation d'une interface haptique.

			Dans ce chapitre, nous souhaitons étudier la pertinence d'une configuration collaborative pour appréhender la déformation d'une molécule.
			De plus, nous aborderons la question de l'apprentissage au sein d'un groupe.
			En effet, certains éléments de la première expérimentation semble indiquer qu'une configuration collaborative stimule l'apprentissage concernant l'utilisation des outils, de la plate-forme ou encore de la tâche à réaliser.
		\end{mysection}
		\begin{mysection}[sec-exp2-DeformationCollaborativeEnEnvironnementVirtuel]{Déformation collaborative en environnement virtuel}
			\begin{mysubsection}[sse-exp2-TravauxExistants]{Travaux existants}
				L'utilisation de retours haptiques pour la déformation d'objets flexibles n'est pas une idée nouvelle.
				\mycite[author]{Shen-2006} proposent déjà une solution pour déformer des objets non-rigides à l'aide de retour haptique.
				Les objets concernés sont de faible complexité, comme des sphères par exemple.
				Puis, \mycite[author]{Peterlik-2009} effectue une thèse sur les déformations de tissus cellulaires.
				Là encore, les éléments déformables sont de faible complexité et n'ont quasiment pas d'application utile et concrète dans le monde réel.

				Cependant, afin d'effectuer des déformations plus complexes, certains se sont intéressés aux processus de déformation collaboratifs dans les \myacro*{acr-EVC}.
				\mycite[author]{Sumengen-2007} proposent une plate-forme permettant la déformation de maillages destinés à des simulations d'objets déformables (tissus, organes, \myetc) dans un \myacro{acr-EVC}.
				Pour cela, il propose une architecture de type pair-à-pair basée sur le protocole \myacro{acr-UDP}.
				De son côté, \mycite[author]{Tang-2010a} proposent une plate-forme client/serveur de déformation collaborative de maillages.
				Ces deux plate-formes proposent chacun une plate-forme de déformation collaborative mais se focalisent principalement sur les contraintes techniques d'une telle plate-forme.
				\mycite[author]{Muller-2006} développent le logiciel \myClayWorks, complété plus tard par \mycite[author]{Gorlatch-2009}, permettant la sculpture virtuelle sur glaise.
				Dans cette étude, les problèmatiques d'accès exclusif à certains objets ou à certaines parties d'un objet sont brièvement évoquées afin de faciliter la coordination des différents acteurs.

				Tous les travaux présentés ci-dessus proposent une déformation collaborative distante où chaque utilisateur effectue une déformation localement.
				Les contraintes liées à la collaboration entre les acteurs n'est pas présentée.
				En effet, tous les \myacro*{acr-EVC} proposés sont consacrés aux problématiques techniques de la collaboration distante.
			\end{mysubsection}
			\begin{mysubsection}[sse-exp2-Objectifs]{Objectifs}
				Ce chapitre sera l'occasion d'aborder les problématiques de la déformation lors d'une configuration collaborative.
				La déformation est une tâche nécessitant plus de précision que la recherche car les cibles doivent être déplacées à un endroit défini.
				Nous souhaitons ainsi comparer les performances sur une tâche complexe nécessitant de la coordination.

				L'étude met en jeu un nombre de ressources fixe pour la déformation et compare une distribution des ressources (configuration collaborative) à une mutualisation des ressources (configuration individuelle).
				En effet, la première étude nous a montré les contraintes d'une configuration collaborative en terme de temps de communication.
				Paradoxalement, les utilisateurs qui manipulent seuls sont confrontés à une charge cognitive de travail importante.
				En fournissant les mêmes ressources (deux outils de déformation), nous souhaitons comparer la capacité de coordination d'un \myglos{glo-Binome} aux capacités cognitives de traitement d'un \myglos{glo-Monome} face à une importante charge de travail.
				Nous supposons que les \myglos*{glo-Binome} en configuration \myglos*{glo-Monomanuel} sont plus performants que les \myglos*{glo-Monome} en configuration \myglos*{glo-Bimanuel}.

				Dans un second temps, nous souhaitons définir un lien entre la complexité de la tâche et le nombre de sujets impliqués.
				En effet, les tâches complexes fournissent une charge cognitive de travail très importante; plus cette charge de travail est importante et plus les \myglos*{glo-Monome} devraient éprouver des difficultés à traiter l'ensemble des informations.
				Nous émettons l'hypothèse que les tâches les plus complexes seront plus difficiles à réaliser par les \myglos*{glo-Monome} que par les \myglos*{glo-Binome}.

				Enfin, cette seconde étude est l'occasion d'observer l'effet du travail collaboratif sur l'apprentissage.
				Nous comparons les performances des \myglos*{glo-Monome} et des \myglos*{glo-Binome} concernant la réalisation d'une même tâche répétée plusieurs fois.
				Nous supposons que la motivation et l'échange qui a lieu lors d'un travail collaboratif va permettre aux \myglos*{glo-Binome} d'appréhender plus rapidement la plate-forme, les outils ou encore la tâche.
			\end{mysubsection}
		\end{mysection}
		\begin{mysection}[sec-exp2-PresentationDeLExperimentation]{Présentation de l'expérimentation}
			\begin{mysubsection}[sse-exp2-DescriptionDeLaTache]{Description de la tâche}
				La tâche proposée est la déformation dans un \myacro{acr-EVC} sur des molécules complexes.
				L'objectif est de modifier la conformation initiale d'une molécule pour atteindre une conformation finale.
				En effet, la recherche d'un état stable est exactement ce qu'une tâche d'\myglos{glo-AmarrageMoleculaire} cherche à réaliser.
				De plus, la déformation est une tâche permettant de stimuler les actions coordonnées pour une collaboration étroitement couplée.

				Trois molécules sont utilisées dans le cadre de cette expérimentation.
				\myPrion est une molécule très complexe et sera simplement utilisé dans la phase d'entraînement.
				\myTRPZIPPER et \myTRPCAGE seront chacune utilisée dans deux scénarios distincts.
				Ces molécules sont détaillées dans la \myref{sse-pro-ListeDesMolecules}.

				Afin de pouvoir évaluer la déformation effectuée, un score est affiché en temps-réel \myref*{fig-exp2-AffichageDeLaMoleculeADeformerEtDeLaMoleculeCible}.
				Le score affiché est le \myacro{acr-RMSD} qui permet de mesurer la différence de forme entre deux déformations d'une même molécule en calculant la différence entre chaque paire d'atomes.
				L'\myref{equ-RMSD} est utilisée pour le calcul de cette différence.
				\begin{equation}\label{equ-RMSD}
					\mathrm{RMSD}\left(\mathbf{c},\mathbf{m}\right) = \sqrt{\frac{1}{N}\sum_{i=1}^{N}\mynorm{c_i - m_i}^2}
				\end{equation}
				où $N$~est le nombre total d'atomes et~$c_i$, $m_i$~sont respectivement les atomes~$i$ de la molécule à comparer $\mathbf{c}$ et de la molécule modèle $\mathbf{m}$.

				\begin{myfigure}
					\psset{unit=0.08\textwidth}
					\begin{myps}(0,0)(12,9)
						\rput[bl](1,0){\myimage[width=0.6\textwidth]{exp2-trp-zipper}}
						\rput[bl](6.2,5){\myimage[width=5cm,angle=0]{exp2-red-cursor}}
						\rput[bl](8.5,0.5){\myimage[width=3.5cm,angle=-20]{exp2-green-cursor}}
						\psframe*[linecolor=red](0,8)(12,9)
						\psframe*[linecolor=green](0,8)(2,9)
						\rput(6,8.5){\textcolor{white}{\bfseries\sffamily\LARGE Score RMSD}}
						\psframe[linewidth=1pt,linecolor=black](0,0)(12,9)
					\end{myps}
					\mycaption[fig-exp2-AffichageDeLaMoleculeADeformerEtDeLaMoleculeCible]{Affichage de la molécule à déformer et de la molécule cible}
				\end{myfigure}
				\begin{mysubsubsection}[sss-exp2-DescriptionDesScenarios]{Description des scénarios}
					Quatre scénarios sont proposés sur deux molécules avec deux niveaux de manipulation différents.
					Les deux niveaux différents de manipulation sont :
					\begin{itemize}
						\item inter-moléculaire (à l'échelle d'un \myglos{glo-Residu}) pour un niveau de déformation avec une granularité élevée;
						\item intra-moléculaire (à l'échelle d'un atome) pour un niveau de déformation avec une granularité fine.
					\end{itemize}

					Les paragraphes qui vont suivre décrivent les quatre scénarios basés sur les critères de complexité suivants :
					\begin{description}
						\item[Nombre d'atomes] C'est le nombre total d'atomes que contient la molécule à manipuler;
						\item[\myGlosnl{glo-Residu} libre] C'est le nombre de \myglos*{glo-Residu} de la molécules non fixés dans la simulation;
						\item[Cassure] Ce sont des angles dans la chaîne principale de la molécule; elles représentent les jonctions entre \myhelice* et/ou les \myfeuillet* et nécessitent deux points d'accroche pour être reformées;
						\item[Champ de force] C'est l'intensité des forces dans les zones de déformation; il exprime l'énergie minimum nécessaire à déployer pour atteindre l'objectif et se traduit par trois niveaux (\myemph{faible}, \myemph{moyen} et \myemph{fort}).
					\end{description}
					\begin{myparagraph}[par-exp2-Scenario1a]{Scénario~\myscenario{1a}}
						Cette tâche concerne la manipulation de la molécule \myTRPZIPPER à l'échelle inter-moléculaire.
						Un \myglos{glo-Residu} à l'extrémité\footnote{La molécule forme une chaîne carbonée; il s'agit ici d'une des extrémités de cette chaîne.} est fixé afin d'\myemph{ancrer} la molécule dans la scène virtuelle et éviter d'éventuelles dérives hors du champ visuel.
						L'intégralité des onze autres \myglos*{glo-Residu} est libre de mouvement ce qui en fait une molécule assez malléable avec un champ de force moyennement contraint.
						La forme général de la molécule peut être comparée à un \myform{V} : la chaîne de \myglos*{glo-Residu} de la molécule contient une cassure.
						La difficulté de ce scénario réside dans la nécessité de maintenir les \myglos*{glo-Residu} déjà placés pendant que le reste de la molécule est déformée.
					\end{myparagraph}
					\begin{myparagraph}[par-exp2-Scenario1b]{Scénario~\myscenario{1b}}
						Cette tâche concerne la manipulation de la molécule \myTRPCAGE à l'échelle inter-moléculaire.
						Comme le scénario~\myscenario{1a}, elle contient un \myglos{glo-Residu} fixe à une extrémité.
						L'intégralité des dix neuf autres \myglos*{glo-Residu} est libre de mouvement ce qui en fait une molécule assez malléable avec un champ de force moyennement contraint.
						La forme général de la molécule peut être comparée à un \myform{W} : la chaîne de \myglos*{glo-Residu} de la molécule contient deux cassures.
						Ce scénario est plus difficile que le scénario~\myscenario{1a} car le nombre d'atomes à placer est plus élevé et qu'il est nécessaire de maintenir en place deux cassures.
					\end{myparagraph}
					\begin{myparagraph}[par-exp2-Scenario2a]{Scénario~\myscenario{2a}}
						Cette tâche concerne la manipulation de la molécule \myTRPZIPPER à l'échelle intra-moléculaire.
						Seulement trois \myglos*{glo-Residu} sont laissés libres et tous les autres \myglos*{glo-Residu} sont fixés.
						Le champ de force au sein de la zone de déformation pour cette molécule est très faible et aucune cassure n'est à reformer.
						Cependant, la difficulté de ce scénario réside dans la précision de la déformation nécessaire.
						En effet, plutôt que de modifier la position des \myglos*{glo-Residu}, ce scénario nécessite la modification de l'orientation d'un \myglos*{glo-Residu} donc une précision accrue dans la sélection et la déformation des atomes.
					\end{myparagraph}
					\begin{myparagraph}[par-exp2-Scenario2b]{Scénario~\myscenario{2b}}
						Cette tâche concerne la manipulation de la molécule \myTRPCAGE à l'échelle intra-moléculaire.
						Seulement six \myglos*{glo-Residu} sont laissés libres et tous les autres \myglos*{glo-Residu} sont fixés.
						Le champ de force au sein de la zone de déformation est très important et l'énergie qu'il est nécessaire de déployer pour réussir cette déformation est importante.
						Cette déformation ne peut être réalisée qu'avec la manipulation simultanée et coordonnée de deux \myglos*{glo-Residu} : ceci permet de recréer la cassure.
					\end{myparagraph}

					Un résumé de la complexité des quatre tâches est exposé dans la \myref{tab-exp2-ParametresDeComplexiteDesTaches} selon les critères suivants :

					\begin{mytable}
						\mycaption[tab-exp2-ParametresDeComplexiteDesTaches]{Paramètres de complexité des tâches}
						\begin{mytabular}{^>{\bfseries}p{9em}-C-C-C-C}
							\mytoprule
							\myrowstyle{\bfseries}
							Scénario                      & \myscenario{1a} & \myscenario{1b} & \myscenario{2a} & \myscenario{2b} \\
							\mymiddlerule[\heavyrulewidth]
							Nombre d'atomes               & \mynum{218}     & \mynum{304}     & \mynum{218}     & \mynum{304}     \\
							\mymiddlerule
							\myGlosnl*{glo-Residu} libres & \mynum{11}      & \mynum{19}      & \mynum{3}       & \mynum{7}       \\
							\mymiddlerule
							Cassure                       & \mynum{1}       & \mynum{2}       & \mynum{0}       & \mynum{1}       \\
							\mymiddlerule
							Champ de force                & Moyen           & Moyen           & Faible          & Fort            \\
							\mybottomrule
						\end{mytabular}
					\end{mytable}
				\end{mysubsubsection}
			\end{mysubsection}
			\begin{mysubsection}[sse-exp2-SpecificitesDuProtocoleExperimental]{Spécificités du protocole expérimental}
				L'expérimentation, basée sur le dispositif expérimental présenté dans le \myref{cha-pro-DispositifExperimental}, a subi quelques modifications qui seront détaillées dans les sections suivantes.
				Un résumé de la methode expérimentale se trouve dans la \myref{tab-exp2-SyntheseDeLaProcedureExperimentale} qu'on pourra retrouver de manière détaillée dans la \myref{sec-met-exp2-SecondeExperimentation}.
				\begin{mysubsubsection}[sss-exp2-Materiel]{Matériel}
					Pour cette seconde expérimentation, une unique modification a été effectuée par rapport à la plate-forme de base \myref*{sec-pro-MaterielExperimental}.
					En effet, suite à la première expérimentation, nous avons beaucoup remis en cause la présence de l'outil d'orientation de la molécule.
					Cet outil permettant de modifier l'orientation de la molécule est nécessaire.
					Cependant, la forme sous laquelle il est présenté n'est pas idéale.
					L'outil d'orientation \mytool{grab} a posé des problèmes manifestes d'interaction à certains sujets qui ne réussissaient pas à s'approprier l'outil.

					Après une discussion avec un bio-informaticien, il est apparu qu'une souris~\myThreeD est un outil plus approprié qu'une interface haptique pour l'orientation de la scène.
					En effet, le périphérique haptique possède des contraintes mécaniques qui ne permettent pas des rotations complètes de l'objet.
					Cette contrainte amène des problématiques connues d'interaction avec les objets virtuels : le débrayage \mycite{Dominjon-2006}.
					La souris~\myThreeD ne souffre pas d'une telle contrainte et peut ainsi être proposée comme outil d'orientation en alternative à l'interface haptique associée à l'outil \mytool{grab}.
					Une souris~\myThreeD \mySpaceNavigator est placée sur la table entre les deux sujets.
					Aucune consigne particulière n'est donnée sur l'utilisation de cet outil et chaque sujet peut l'utiliser au moment où il le souhaite : nous créons ainsi artificiellement un point de conflit pour l'accès à cet outil.
					L'objectif est de stimuler les interactions.

					En ce qui concerne l'utilisation des outils de déformation en \myglos{glo-Binome}, chaque sujet possède à sa disposition un outil de déformation.
					La répartition de l'outil d'orientation est laissé à la responsabilités des deux membres du \myglos{glo-Binome}.
					Pour les \myglos*{glo-Monome}, le sujet peut utiliser les deux outils de déformation en configuration \myglos*{glo-Bimanuel}.
					Il peut également utiliser l'outil d'orientation mais dans ce cas, il est forcé de lâcher un des deux outils de déformation.
					Pour des raisons d'équité entre les \myglos*{glo-Monome} et les \myglos*{glo-Binome}, l'utilisation de la souris~\myThreeD désactive toutes les sélections effectuées avec un outil de déformation.
					De cette façon, les membres du \myglos{glo-Binome} ne peuvent pas utiliser l'outil d'orientation en même temps que les outils de déformations (contrainte physique inhérente au \myglos*{glo-Monome}).

					Les \myref{fig-exp2-SchemaDuDispositifExperimental} et \myref{fig-exp2-PhotographieDuDispositifExperimental} illustrent par un schéma et une photographie le dispositif expérimental.

					\begin{myfigure}
						\myimage{exp2-schema}
						\mycaption[fig-exp2-SchemaDuDispositifExperimental]{Schéma du dispositif expérimental}
					\end{myfigure}
					\begin{myfigure}
						\myimage{exp2-photo}
						\mycaption[fig-exp2-PhotographieDuDispositifExperimental]{Photographie du dispositif expérimental}
					\end{myfigure}
				\end{mysubsubsection}
				\begin{mysubsubsection}[sss-exp2-Visualisation]{Visualisation}
					Pour cette seconde expérimentation, quatre scénarios sont proposés et présentés dans la \myref{sss-exp2-DescriptionDesScenarios}.
					Ces molécules sont représentées avec les rendus graphiques de base (\myCPK et \myNewRibbon).
					Cependant, la tâche nécessite d'afficher la molécule dans son état stable qui est l'objectif que doivent atteindre les sujets.
					Cette molécule ne peut pas être représentée avec tous les atomes pour deux raisons.
					Tout d'abord, l'intégralité des atomes serait une surcharge du rendu visuel.
					De plus, la précision nécessaire pour les déformations demandées ne nécessite pas un affinement au niveau atomique.
					C'est pourquoi, la molécule dans son état stable sera discrètement affichée avec un rendu \myNewRibbon en transparence.

					Des images représentants les différents scénarios dans leurs états initiaux sont présentées.
					Les scénarios inter-moléculaires \myscenario{1a} et \myscenario{1b} sont respectivement représentés sur la \myref{fig-exp2-RepresentationDeLaMoleculeTRPZIPPERPourLeScenario1A} et la \myref{fig-exp2-RepresentationDeLaMoleculeTRPCAGEPourLeScenario1B}.
					Les scénarios intra-moléculaires \myscenario{2a} et \myscenario{2b} sont respectivement représentés sur la \myref{fig-exp2-RepresentationDeLaMoleculeTRPZIPPERPourLeScenario2A} et la \myref{fig-exp2-RepresentationDeLaMoleculeTRPCAGEPourLeScenario2B}.

					\begin{myfigure}
						\myimage[width=0.5\textwidth]{exp2-scenario1A}
						\mycaption[fig-exp2-RepresentationDeLaMoleculeTRPZIPPERPourLeScenario1A]{Représentation de la molécule \myTRPZIPPER pour le scénario~\myscenario{1A}}
					\end{myfigure}
					\begin{myfigure}
						\myimage[width=0.5\textwidth]{exp2-scenario1B}
						\mycaption[fig-exp2-RepresentationDeLaMoleculeTRPCAGEPourLeScenario1B]{Représentation de la molécule \myTRPCAGE pour le scénario~\myscenario{1B}}
					\end{myfigure}
					\begin{myfigure}
						\myimage{exp2-scenario2A}
						\mycaption[fig-exp2-RepresentationDeLaMoleculeTRPZIPPERPourLeScenario2A]{Représentation de la molécule \myTRPZIPPER pour le scénario~\myscenario{2A}}
					\end{myfigure}
					\begin{myfigure}
						\myimage{exp2-scenario2B}
						\mycaption[fig-exp2-RepresentationDeLaMoleculeTRPCAGEPourLeScenario2B]{Représentation de la molécule \myTRPCAGE pour le scénario~\myscenario{2B}}
					\end{myfigure}
				\end{mysubsubsection}
				\begin{mysubsubsection}[sss-exp2-OutilsDeManipulation]{Outils de manipulation}
					Concernant l'outil d'orientation de la molécule, maintenant assuré par une souris~\myThreeD, une légére modification a été effectuée.
					Grâce au choix du matériel qui permet de différencier aisément les translations et les rotations, nous avons choisi de ne conserver que les \myacro*{acr-DDL} en rotation.
					En effet, la molécule n'a pas besoin d'être déplacée à l'écran et c'est surtout l'orientation la molécule qui est nécessaire aux sujets.
					Ceci permet également d'éviter que les molécules ne sortent de l'espace visuel des sujets (scène vidéoprojetée) à cause d'une mauvaise manipulation.
					De plus, ce choix permet d'enlever une part de la charge cognitive aux manipulateurs.

					En ce qui concerne les outils de déformation, quelques modifications concernant le rendu visuel ont été effectuées.
					La tâche consiste à reconstituer une molécule dans son état d'équilibre.
					Pour aider les sujets dans cette tâche, nous avons affiché un rendu visuel en transparence de la molécule dans son état stable.
					Pour augmenter l'aide visuelle apportée, nous allons également indiquer l'emplacement final d'un \myglos{glo-Residu} sélectionné.
					En effet, dès qu'un sujet sélectionne un \myglos{glo-Residu}, ce \myglos{glo-Residu} est mis en surbrillance.
					Le \myglos{glo-Residu} correspondant sur la molécule stable est également mis en surbrillance comme expliqué sur la \myref{fig-exp2-IllustrationDesRendusPourLAffichageDeLaMolecule}.
					Le \myglos{glo-Residu} de la molécule stable est représenté par un rendu \myCPK coloré de la couleur du curseur du sujet concerné.

					\begin{myfigure}
						\psset{unit=0.08\textwidth}
						\def\myexptwolabel(#1,#2)[#3]#4#5{\rput(#1,#2){\rnode{#3}{\textcolor{#4}{\sffamily #5}}}}
						\begin{myps}(0,0)(11,8.5)
							\rput[bl](1,0){\myimage{exp2-trp-zipper}}
							\myexptwolabel(9.4,2.6)[deformed-label]{myred}{Molécule à déformer}
							\myexptwolabel(1,5.5)[ghost-label]{myred}{Molécule cible}
							\myexptwolabel(7.1,7.3)[deformed-residue-label]{myblue}{\myGlosnl{glo-Residu} sélectionné}
							\myexptwolabel(1,3)[ghost-residue-label]{myblue}{\myGlosnl{glo-Residu} cible}
							\myexptwolabel(4.0,8.15)[fixed-residue-label]{mygray}{\myGlosnl{glo-Residu} fixe}
							\pnode(7.4,3.6){deformed}
							\pnode(1.8,4){ghost}
							\psset{linecolor=myblue}
							\cnode(6.2,5.2){1.0}{deformed-residue}
							\cnode(2.3,1.5){0.8}{ghost-residue}
							\psset{linecolor=mygray}
							\cnode(2.0,7){0.8}{fixed-residue}
							\psset{linewidth=1pt,linecolor=myred,linearc=.1,arrowsize=1pt 3,arrowinset=.2,nodesepA=3pt}
							\ncangle[angleA=90,angleB=0]{c->}{deformed-label}{deformed}
							\ncangle[angleA=-90,angleB=180,offsetA=-0.5]{c->}{ghost-label}{ghost}
							\psset{linecolor=myblue,nodesepB=0pt}
							\ncdiagg[angleA=-90,offsetA=0.5]{c->}{deformed-residue-label}{deformed-residue}
							\ncdiagg[angleA=-90,offsetA=-0.5]{c->}{ghost-residue-label}{ghost-residue}
							\ncdiagg[angleA=180,linecolor=mygray]{c->}{fixed-residue-label}{fixed-residue}
							\ncline[linewidth=10pt,linecolor=myblue,arrowsize=2pt 2,nodesepA=4pt]{C->}{deformed-residue}{ghost-residue}
						\end{myps}
						\mycaption[fig-exp2-IllustrationDesRendusPourLAffichageDeLaMolecule]{Illustration des rendus pour l'affichage de la molécule}
					\end{myfigure}
				\end{mysubsubsection}
				\begin{mytable}
					\mycaption[tab-exp2-SyntheseDeLaProcedureExperimentale]{Synthèse de la procédure expérimentale}
					\newcommand{\mytitlecolumn}[2]{%
						\multirow{#1}*{%
							\begin{minipage}{6em}%
								\raggedleft #2%
							\end{minipage}%
						}
					}
					\newlength{\exptwofirstcolumn}
					\newlength{\exptwosecondcolumn}
					\setlength{\exptwofirstcolumn}{7em}
					\setlength{\exptwosecondcolumn}{\textwidth}
					\addtolength{\exptwosecondcolumn}{-\exptwofirstcolumn}
					\addtolength{\exptwosecondcolumn}{-4\tabcolsep}
					\begin{mytabular}{>{\bfseries}p{\exptwofirstcolumn}p{\exptwosecondcolumn}}
						\mytoprule
						\mytitlecolumn{1}{Tâche}                   & Déformation d'une molécule                                                        \\
						\mymiddlerule[\heavyrulewidth]
						\mytitlecolumn{4}{Hypothèses}              & \myhypothesis{1} Amélioration des performances en \myglosnl{glo-Binome}           \\
						                                           & \myhypothesis{2} \myglosnl*{glo-Binome} plus performants sur les tâches complexes \\
						                                           & \myhypothesis{3} Apprentissage plus performant en \myglosnl{glo-Binome}           \\
						                                           & \myhypothesis{4} Les sujets préfèrent le travail en collaboration                 \\
						\mymiddlerule
						\mytitlecolumn{3}{Variables indépendantes} & \myvari{1} Nombre de sujets                                                       \\
						                                           & \myvari{2} Complexité de la tâche                                                 \\
						                                           & \myvari{3} Niveau d'apprentissage                                                 \\
						\mymiddlerule
						\mytitlecolumn{6}{Variables dépendantes}   & \myvard{1} Temps de réalisation                                                   \\
						                                           & \myvard{2} Nombre de sélections                                                   \\
						                                           & \myvard{3} Distance passive entre les espaces de travail                          \\
						                                           & \myvard{4} Distance active entre les espaces de travail                           \\
						                                           & \myvard{5} Vitesse moyenne                                                        \\
						                                           & \myvard{6} Réponses qualitatives                                                  \\
						\mymiddlerule[\heavyrulewidth]
						\multicolumn{2}{c}{%
							\small%
							\begin{tabular}{^C-C-C-C}
								\myrowstyle{\bfseries}
								Condition \mycondition{1}       & Condition \mycondition{2}       & Condition \mycondition{3} & Condition \mycondition{4} \\
								\mymiddlerule
								\mynum{1}~sujet                 & \mynum{1}~sujet                 & \mynum{2}~sujets          & \mynum{2}~sujets          \\
								\myGlosnl{glo-Bimanuel}         & \myGlosnl{glo-Bimanuel}         & Collaboratif              & Collaboratif              \\
								\mymiddlerule
								Scénario~\myscenario{1a}        & Scénario~\myscenario{1b}        & Scénario~\myscenario{1a}  & Scénario~\myscenario{1b}  \\
								Scénario~\myscenario{1b}        & Scénario~\myscenario{1a}        & Scénario~\myscenario{1b}  & Scénario~\myscenario{1a}  \\
								Scénario~\myscenario{2a}        & Scénario~\myscenario{2b}        & Scénario~\myscenario{2a}  & Scénario~\myscenario{2b}  \\
								Scénario~\myscenario{2b}        & Scénario~\myscenario{2a}        & Scénario~\myscenario{2b}  & Scénario~\myscenario{2a}  \\
							\end{tabular}
						} \\
						\mybottomrule
					\end{mytabular}
				\end{mytable}
			\end{mysubsection}
		\end{mysection}
		\begin{mysection}[sec-exp2-Resultats]{Résultats}
			Cette section présente et analyse l'ensemble des mesures expérimentales de cette seconde étude concernant la déformation de molécules complexes en configuration collaborative.
			Les données, confrontées à un test de \mycite[author]{Shapiro-1965}, ne sont pas distribuées selon une loi normale.
			Cependant, un test de \mycite[author]{Brown-1974} permet de confirmer l'\myglos{glo-Homoscedasticite}.
			L'analyse de la variance est alors pratiquée avec différents tests statistiques suivant les cas :
			\begin{itemize}
				\item test de \mycite[author]{Friedman-1940} pour les \myglos*{glo-VariableIntraSujets} non-paramètriques;
				\item test de \mycite[author]{Kruskal-1952} pour les \myglos*{glo-VariableInterSujets} non-paramètriques.
			\end{itemize}
			\begin{mysubsection}[sse-exp2-AmeliorationDesPerformancesEnBinome]{Amélioration des performances en \myglosnl{glo-Binome}}
				\begin{mysubsubsection}[sss-exp2-AmeliorationDesPerformancesEnBinome-DonneesEtStatistiques]{Données et statistiques}
					\begin{myfigure}
						\psset{xunit=0.272108844\textwidth,yunit=0.02cm}
						\begin{myps}(-0.45,-55)(2,210)
							\myaxes(0,2){nombre de sujets}(0,200)[50]{temps~(s)}
							\myboxplot{exp2-time-group.csv}
						\end{myps}
						\mycaption[fig-exp2-TempsDeRealisationEnFonctionDuNombreDeSujets]{Temps de réalisation en fonction du nombre de sujets}
					\end{myfigure}

					La \myref{fig-exp2-TempsDeRealisationEnFonctionDuNombreDeSujets} présente le temps de réalisation \myvard{1} en fonction du nombre de sujets \myvari{1}.
					L'analyse montre qu'il y a un effet significatif du nombre de sujets \myvari{1} sur le temps de réalisation \myvard{1} (\myanova{exp2-time-group-anova.tex}).

					\begin{myfigure}
						\psset{xunit=0.272108844\textwidth,yunit=1.25cm}
						\begin{myps}(-0.45,-0.85)(2,3.5)
							\myaxes(0,2){distance}(0,3)[1]{distance~(mm)}
							\myboxplot{exp2-diff-activepassive-group.csv}
							\mylegend{\myleg{\myglosnl{glo-Monome}}{myblue}\myand\myleg{\myglosnl{glo-Binome}}{myblue!70}}
						\end{myps}
						\mycaption[fig-exp2-DistancePassiveEtActiveEntreLesEffecteursTerminauxEnFonctionDuNombreDeSujets]{Distance passive et active entre les \myglosnl*{glo-EffecteurTerminal} en fonction du nombre de sujets}
					\end{myfigure}

					La \myref{fig-exp2-DistancePassiveEtActiveEntreLesEffecteursTerminauxEnFonctionDuNombreDeSujets} présente la distance passive \myvard{3} et active \myvard{4} entre les \myglos*{glo-EffecteurTerminal} en fonction du nombre de sujets \myvari{1}.
					L'analyse montre qu'il n'y a pas d'effet significatif du nombre de sujets \myvari{1} sur la distance passive \myvard{3} (\myanova{exp2-diff-activepassive-group-anova-passive.tex}).
					Cependant, l'analyse montre qu'il y a un effet significatif du nombre de sujets \myvari{1} sur la distance active \myvard{4} (\myanova{exp2-diff-activepassive-group-anova-active.tex}).

					On peut également comparer les distances passive et active en fonction du nombre de sujets \myvari{1}.
					L'analyse montre qu'il y a un effet significatif de la nature de la distance (passive ou active) au sein d'un \myglos{glo-Monome} (\myanova{exp2-diff-activepassive-group-anova-monome.tex}).
					Par contre, l'analyse ne montre pas d'effet significatif de la nature de la distance (passive ou active) au sein d'un \myglos{glo-Binome} (\myanova{exp2-diff-activepassive-group-anova-binome.tex}).

					\begin{myfigure}
						\psset{xunit=0.272108844\textwidth,yunit=0.075cm}
						\begin{myps}(-0.45,-15)(2,58)
							\myaxes(0,2){nombre de sujets}(0,50)[10]{nombre de sélections~(nb)}
							\myboxplot{exp2-numsel-group-dominant.csv}
							\mylegend{\myleg{main dominante}{myblue}\myand\myleg{main dominée}{myblue!70}}
						\end{myps}
						\mycaption[fig-exp2-NombreDeSelectionsParMainDominanteDomineeEnFonctionDuNombreDeSujets]{Nombre de sélections par main dominante/dominée en fonction du nombre de sujets}
					\end{myfigure}

					La \myref{fig-exp2-NombreDeSelectionsParMainDominanteDomineeEnFonctionDuNombreDeSujets} présente le nombre de sélections par main dominante/dominée \myvard{2} en fonction du nombre de sujets \myvari{1}.
					Les \myglos*{glo-Binome} n'utilisant que leur main dominante, il n'y a pas de résultat pour la main dominée.
					On constate un déséquilibre du nombre de sélections entre la main dominante et la main dominée pour les \myglos*{glo-Monome}.
					L'analyse montre qu'il y a un effet significatif du nombre de sujets \myvari{1} sur le nombre total de sélections (somme des mains dominante et dominée pour les \myglos*{glo-Monome}) \myvard{2} (\myanova{exp2-numsel-group-dominant-anova-cumulative.tex}).

					Le nombre de sélections pour la main dominante comptabilise les sélections des deux sujets du \myglos{glo-Binome} contrairement aux \myglos*{glo-Monome} : ceci explique le nombre plus élevé de sélections en \myglos*{glo-Binome}.
					Cependant, si on compare le nombre moyen de sélections par sujet (pour la main dominante), l'analyse montre qu'il n'y a pas d'effet significatif du nombre de sujets \myvari{1} sur le nombre de sélections de la main dominante \myvard{2} (\myanova{exp2-numsel-group-dominant-anova-dominant.tex}).

					\begin{myfigure}
						\psset{xunit=0.272108844\textwidth,yunit=2.5cm}
						\begin{myps}(-0.45,-0.45)(2,1.75)
							\myaxes(0,2){nombre de sujets}(0,1.5)[0.5]{speed~(mm/s)}
							\myboxplot{exp2-speed-group-dominant.csv}
							\mylegend{\myleg{main dominante}{myblue}\myand\myleg{main dominée}{myblue!70}}
						\end{myps}
						\mycaption[fig-exp2-VitesseMoyenneDeLaMainDominanteEtDomineeEnFonctionDuNombreDeSujets]{Vitesse moyenne de la main dominante et dominée en fonction du nombre de sujets}
					\end{myfigure}

					La \myref{fig-exp2-VitesseMoyenneDeLaMainDominanteEtDomineeEnFonctionDuNombreDeSujets} présente la vitesse moyenne des \myglos*{glo-EffecteurTerminal} \myvard{5} des \myglos*{glo-EffecteurTerminal} en fonction du nombre de sujets \myvari{1}.
					L'analyse montre un effet significatif du nombre de sujets \myvari{1} sur la vitesse moyenne \myvard{5} (\myanova{exp2-speed-group-dominant-anova.tex}).
					L'analyse montre un déséquilibre de vitesse moyenne entre la main dominante et dominée des \myglos*{glo-Monome} avec un effet significatif (\myanova{exp2-speed-group-dominant-anova-monome.tex}).
					L'analyse montre également un effet significatif du nombre de sujets \myvari{1} sur la vitesse moyenne \myvard{5} de la main dominante (\myanova{exp2-speed-group-dominant-anova-dominant.tex}).
				\end{mysubsubsection}
				\begin{mysubsubsection}[sss-exp2-AmeliorationDesPerformancesEnBinome-AnalyseEtDiscussion]{Analyse et discussion}
					Le premier résultat sur la \myref{fig-exp2-TempsDeRealisationEnFonctionDuNombreDeSujets} nous permet de confirmer notre hypothèse \myhypothesis{1} : les \myglos*{glo-Binome} sont plus performants que les \myglos*{glo-Monome}.
					Cependant, la suite de l'analyse va permettre de mettre en avant les paramètres précis pour lesquels il y a un gain de performances ainsi que les scénarios les plus adaptés à cette configuration de travail.

					Pour commencer, les distances moyennes entre les \myglos*{glo-EffecteurTerminal} nous permet d'observer un désequilibre de performances entre les \myglos*{glo-Monome} et les \myglos*{glo-Binome} \myref*{fig-exp2-DistancePassiveEtActiveEntreLesEffecteursTerminauxEnFonctionDuNombreDeSujets}.
					En effet, la distance passive entre les \myglos*{glo-EffecteurTerminal} (distance moyenne sur toute la durée de la tâche) est plus importante pour les \myglos*{glo-Monome} que pour les \myglos*{glo-Binome}.
					Cependant, la distance active (seulement lorsque les deux \myglos*{glo-EffecteurTerminal} sont en phase de sélection) montre un effet inverse.
					En effet, la manipulation \myglos*{glo-Bimanuel} (pour les \myglos*{glo-Monome}) constitue une charge de travail cognitive importante.
					Le sujet doit alors être capable de gérer deux \myglos*{glo-EffecteurTerminal} à chaque instant.
					Cette configuration a mené la plupart des sujets à utiliser seulement un \myglos{glo-EffecteurTerminal} en laissant le second sur le côté afin que le curseur ne gêne pas à l'écran.
					La main dominée n'est utilisée que dans les cas où le sujet estime que c'est absolument nécessaire pour achever la tâche.
					Ceci a pour effet d'augmenter la distance passive moyenne.
					Cependant, la distance est censée représenter l'espace de travail couvert au sein de l'environnement virtuel.
					Dans ce cas, elle est incorrecte car bien que la distance soit importante, elle ne représente pas un espace de travail étant donné que le deuxième \myglos{glo-EffecteurTerminal} n'est pas utilisé.

					La distance active permet d'éviter ce biais de mesure.
					En effet, cette mesure ne prend pas en compte les phases d'inactivité d'un \myglos{glo-EffecteurTerminal}.
					On constate alors que les \myglos*{glo-Binome} couvrent un plus grand espace de travail.
					Les \myglos*{glo-Monome} couvrent un espace de travail plus restreint car ils peuvent focaliser visuellement sur une seule zone de travail à la fois.
					Par conséquent, les deux \myglos*{glo-EffecteurTerminal} se trouvent toujours proche de la zone de manipulation, dans la zone de focus du sujet.

					La \myref{fig-exp2-NombreDeSelectionsParMainDominanteDomineeEnFonctionDuNombreDeSujets} confirme ce déséquilibre.
					En effet, on constate un nombre total de sélections plus grand pour les \myglos*{glo-Binome} (\myanova{exp2-numsel-group-dominant-mean-binome.tex}~sélections) que pour les \myglos*{glo-Monome} (\myanova{exp2-numsel-group-dominant-mean-monome.tex}~sélections).
					Là encore, le sujet effectuant la tâche en \myglos{glo-Monome} n'exploite pas pleinement les deux outils en sa possession : la charge de travail cognitive est trop importante.
					En effet, la \myacro{acr-TRM} \mycite{Wickens-1984} estime que la gestion de plusieurs ressources pour la même modalité est impossible.
					Cependant, les analyses statistiques montre que l'outil utilisé par la main dominante obtient un taux d'utilisation identique entre les \myglos*{glo-Monome} et les \myglos*{glo-Binome}.
					Les \myglos*{glo-Binome} en configuration \myglos*{glo-Monomanuel} répartissent correctement la charge de travail entre les deux ressources disponibles ce qui n'est pas le cas des \myglos*{glo-Monome}.

					Cependant, l'outil associé à la main dominante est géné par la configuration \myglos*{glo-Bimanuel}.
					En effet, l'analyse montre une différence significative entre la vitesse moyenne de la main dominante des \myglos*{glo-Monome} et celle des \myglos*{glo-Binome}.
					La configuration \myglos*{glo-Bimanuel} provoque une séquentialité dans les actions du sujet : il manipule avec un outil, puis avec l'autre mais rarement les deux en même temps.
					Cette séquentialité a pour effet des pauses alternatives entre les outils ce qui explique une vitesse moyenne plus basse.

					Cette section nous a permis de constater que le travail en \myglos{glo-Binome} permet de meilleures performances que le travail en \myglos{glo-Monome}.
					Une analyse plus détaillé a mis en avant la difficulté du travail en configuration \myglos*{glo-Bimanuel} : la charge de travail cognitive à assumer avec deux outils est trop importante.
					Cette difficulté a pour effet de fortement dégrader le taux d'utilisation d'un des deux outils.
					On constate également une légére baisse de l'utilisation de l'outil associé à la main dominante.
					Pour résumer, il est préférable de distribuer les ressources disponibles (outils de manipulation dans notre cas) entre différents participants : la configuration \myglos*{glo-Bimanuel} apporte une charge de travail cognitive trop importante.
				\end{mysubsubsection}
			\end{mysubsection}
			\begin{mysubsection}[sse-exp2-EvolutionDesPerformancesEnFonctionDeLaComplexiteDeLaTache]{Évolution des performances en fonction de la complexité de la tâche}
				\begin{mysubsubsection}[sss-exp2-EvolutionDesPerformancesEnFonctionDeLaComplexiteDeLaTache-DonneesEtStatistiques]{Données et statistiques}
					\begin{myfigure}
						\psset{xunit=0.222222222\textwidth,yunit=0.015cm}
						\begin{myps}(-0.5,-75)(4,290)
							\myaxes(0,4){scénario}(0,250)[50]{temps~(s)}
							\myboxplot{exp2-time-task.csv}
						\end{myps}
						\mycaption[fig-exp2-TempsDeRealisationDesScenarios]{Temps de réalisation des scénarios}
					\end{myfigure}

					La \myref{fig-exp2-TempsDeRealisationDesScenarios} présente le temps de réalisation \myvard{1} en fonction de la complexité de la tâche \myvari{2} (temps cumulé des \myglos*{glo-Monome} et des \myglos*{glo-Binome}).
					L'analyse montre un effet significatif de la complexité de la tâche \myvari{2} sur le temps de réalisation \myvard{1} (\myanova{exp2-time-task-anova.tex}).
					Un test post-hoc de \mycite[author]{Mann-1947} avec une correction de \mycite[author]{Holm-1979} permet de trier les scénarios en deux classes de complexité : $\left\{\myscenario{1a}, \myscenario{2a}\right\}$ et $\left\{\myscenario{1b}, \myscenario{2b}\right\}$.

					\begin{myfigure}
						\psset{xunit=0.222222222\textwidth,yunit=0.0125cm}
						\begin{myps}(-0.5,-90)(4,300)
							\myaxes(0,4){scénario}(0,250)[50]{temps~(s)}
							\myboxplot{exp2-time-task-group.csv}
							\mylegend{\myleg{\myglosnl{glo-Monome}}{myblue}\myand\myleg{\myglosnl{glo-Binome}}{myblue!70}}
						\end{myps}
						\mycaption[fig-exp2-TempsDeRealisationDesScenariosEnFonctionDuNombreDeSujets]{Temps de réalisation des scénarios en fonction du nombre de sujets}
					\end{myfigure}

					La \myref{fig-exp2-TempsDeRealisationDesScenariosEnFonctionDuNombreDeSujets} présente le temps de réalisation \myvard{1} des différents scénarios \myvari{2} en fonction du nombre de sujets \myvari{1}.
					En regroupant les scénarios par classe de complexité, l'analyse montre qu'il n'y a pas d'effet significatif du nombre de sujets \myvari{1} sur le temps de réalisation \myvard{1} pour les scénarios \myscenario{1a} et \myscenario{2a} (\myanova{exp2-time-task-group-anova-trpzipper.tex}).
					Cependant, l'analyse montre un effet significatif du nombre de sujets \myvari{1} sur le temps de réalisation \myvard{1} pour les scénarios \myscenario{1b} et \myscenario{2b} (\myanova{exp2-time-task-group-anova-trpcage.tex}).

					\begin{myfigure}
						\psset{xunit=0.222222222\textwidth,yunit=0.04cm}
						\begin{myps}(-0.5,-30)(4,115)
							\myaxes(0,4){scénario}(0,100)[25]{nombre de sélection~(nb)}
							\myboxplot{exp2-numsel-task-group.csv}
							\mylegend{\myleg{\myglosnl{glo-Monome}}{myblue}\myand\myleg{\myglosnl{glo-Binome}}{myblue!70}}
						\end{myps}
						\mycaption[fig-exp2-NombreDeSelectionsDeChaqueScenarioEnFonctionDuNombreDeSujets]{Nombre de sélections de chaque scénario en fonction du nombre de sujets}
					\end{myfigure}

					La \myref{fig-exp2-NombreDeSelectionsDeChaqueScenarioEnFonctionDuNombreDeSujets} présente le nombre de sélections \myvard{2} des différents scénarios \myvari{2} en fonction du nombre de sujets \myvari{1}.
					En regroupant les scénarios par classe de complexité, l'analyse montre un effet significatif du nombre de sujets \myvari{1} sur le nombre de sélections \myvard{2} pour les scénarios \myscenario{1a} et \myscenario{2a} (\myanova{exp2-numsel-task-group-anova-trpzipper.tex}).
					Cependant, l'analyse montre qu'il n'y a pas d'effet significatif du nombre de sujets \myvari{1} sur le nombre de sélections \myvard{2} pour les scénarios \myscenario{1b} et \myscenario{2b} (\myanova{exp2-numsel-task-group-anova-trpcage.tex}).

					\begin{myfigure}
						\psset{xunit=0.222222222\textwidth,yunit=1.25cm}
						\begin{mysubfigure}[\textwidth]
							\begin{myps}(-0.5,-0.9)(4,3)
								\myaxes(0,4){scénario}(0,2.5)[0.5]{distance~(mm)}
								\myboxplot{exp2-passive-task-group.csv}
								\mylegend{\myleg{\myglosnl{glo-Monome}}{myblue}\myand\myleg{\myglosnl{glo-Binome}}{myblue!70}}
							\end{myps}
							\mysubcaption[fig-exp2-DistancePassiveEtActiveEntreLesEffecteursTerminauxSurChaqueScenarioEnFonctionDuNombreDeSujets-DistancePassive]{Distance passive}
						\end{mysubfigure}
						\begin{mysubfigure}[\textwidth]
							\begin{myps}(-0.5,-0.9)(4,3)
								\myaxes(0,4){scénario}(0,2.5)[0.5]{distance~(mm)}
								\myboxplot{exp2-active-task-group.csv}
								\mylegend{\myleg{\myglosnl{glo-Monome}}{myblue}\myand\myleg{\myglosnl{glo-Binome}}{myblue!70}}
							\end{myps}
							\mysubcaption[fig-exp2-DistancePassiveEtActiveEntreLesEffecteursTerminauxSurChaqueScenarioEnFonctionDuNombreDeSujets-DistanceActive]{Distance active}
						\end{mysubfigure}
						\mycaption[fig-exp2-DistancePassiveEtActiveEntreLesEffecteursTerminauxSurChaqueScenarioEnFonctionDuNombreDeSujets]{Distance passive et active entre les \myglosnl*{glo-EffecteurTerminal} sur chaque scénario en fonction du nombre de sujets}
					\end{myfigure}

					La \myref{fig-exp2-DistancePassiveEtActiveEntreLesEffecteursTerminauxSurChaqueScenarioEnFonctionDuNombreDeSujets} présente les distances passives \myvard{3} et actives \myvard{4} des différents scénarios \myvari{2} en fonction du nombre de sujets \myvari{1}.
					En regroupant les scénarios par classe de complexité, l'analyse montre un effet significatif du nombre de sujets \myvari{1} sur la distance passive \myvard{3} pour les scénarios \myscenario{1a} et \myscenario{2a} (\myanova{exp2-passive-task-group-anova-trpzipper.tex}) mais pas d'effet significatif sur les scénarios \myscenario{1b} et \myscenario{2b} (\myanova{exp2-passive-task-group-anova-trpcage.tex}).
					Cependant, on constate un effet significatif du nombre de sujets \myvari{1} sur la distance active \myvard{4} pour les scénarios \myscenario{1a} et \myscenario{2a} (\myanova{exp2-active-task-group-anova-trpzipper.tex}) ainsi que sur les scénarios \myscenario{1b} et \myscenario{2b} (\myanova{exp2-active-task-group-anova-trpcage.tex}).

					\begin{myfigure}
						\psset{xunit=0.222222222\textwidth,yunit=1.25cm}
						\begin{myps}(-0.5,-0.9)(4,3)
							\myaxes(0,4){scénario}(0,2.5)[0.5]{vitesse~(mm/s)}
							\myboxplot{exp2-speed-task-group.csv}
							\mylegend{\myleg{\myglosnl{glo-Monome}}{myblue}\myand\myleg{\myglosnl{glo-Binome}}{myblue!70}}
						\end{myps}
						\mycaption[fig-exp2-VitesseMoyenneSurChaqueScenarioEnFonctionDuNombreDeSujets]{Vitesse moyenne sur chaque scénario en fonction du nombre de sujets}
					\end{myfigure}

					La \myref{fig-exp2-VitesseMoyenneSurChaqueScenarioEnFonctionDuNombreDeSujets} présente la vitesse moyenne \myvard{5} des différents scénarios \myvari{2} en fonction du nombre de sujets \myvari{1}.
					En regroupant les scénarios par classe de complexité, l'analyse montre un effet significatif du nombre de sujets \myvari{1} sur la vitesse moyenne \myvard{5} pour les scénarios \myscenario{1a} et \myscenario{2a} (\myanova{exp2-speed-task-group-anova-trpzipper.tex}).
					De même, l'analyse montre un effet significatif du nombre de sujets \myvari{1} sur la vitesse moyenne \myvard{5} pour les scénarios \myscenario{1b} et \myscenario{2b} (\myanova{exp2-speed-task-group-anova-trpcage.tex}).
				\end{mysubsubsection}
				\begin{mysubsubsection}[sss-exp2-EvolutionDesPerformancesEnFonctionDeLaComplexiteDeLaTache-AnalyseEtDiscussion]{Analyse et discussion}
					L'analyse du temps de réalisation des différentes tâches ainsi que la \myref{tab-exp2-ParametresDeComplexiteDesTaches} nous permet de classifier ces tâches par niveau de complexité : les scénarios \myscenario{1a} et \myscenario{2a} sont simples alors que les scénarios \myscenario{1b} et \myscenario{2b} sont complexes.
					En effet, les scénarios \myscenario{1a} et \myscenario{2a} concernent la molécule \myTRPZIPPER contenant peu d'atomes et de \myglos*{glo-Residu} libres.
					Par contre, les scénarios \myscenario{1b} et \myscenario{2b}, dont le nombre d'atomes et de \myglos*{glo-Residu} libres est plus important, est constitué de champ de force à fortes contraintes physiques et nécessité la formation de plusieurs cassures.

					En observant les différences de performances entre les \myglos*{glo-Monome} et les \myglos*{glo-Binome} sur la \myref{fig-exp2-TempsDeRealisationDesScenariosEnFonctionDuNombreDeSujets}, on constate que l'apport du travail collaboratif n'est vrai que dans le cas des tâches complexes.
					La contrainte des tâches complexes réside dans la nécessité d'avoir recourt aux deux outils pour achever la tâche.
					En effet, en observant la \myref{fig-exp2-DistancePassiveEtActiveEntreLesEffecteursTerminauxSurChaqueScenarioEnFonctionDuNombreDeSujets-DistancePassive}, l'analyse de la distance active montre une différence significative entre les \myglos*{glo-Monome} et les \myglos*{glo-Binome} pour les scénarios simples.
					Sur la base des résultats de la section précédente \myref*{sse-exp2-AmeliorationDesPerformancesEnBinome}, les \myglos*{glo-Monome} ont tendance à délaisser le deuxième outil à cause de la forte charge cognitive qu'ajoute la configuration \myglos*{glo-Bimanuel}.
					L'outil délaissé augmente ainsi la valeur de la distance passive mesurée en étant mis à l'écart.
					En observant seulement les scénarios simples \myscenario{1a} et \myscenario{2a}, on constate que la distance passive des \myglos*{glo-Monome} est plus importante que celle des \myglos*{glo-Binome}.
					On en conclue que les \myglos*{glo-Monome} que la complexité de ces scénarios n'a pas nécessité une manipulation \myglos*{glo-Bimanuel} et que la tâche a pu être achevée avec un seul outil de déformation.
					Il y a donc peu d'intérêt d'effectuer ces tâches peu complexe en collaboration puisqu'il n'y a aucune amélioration significative des performances bien que le nombre de ressources utilisées (les outils de déformation) soit doublé.

					Cependant, pour les scénarios complexes, l'analyse ne montre pas de différence significative de la distance passive entre les \myglos*{glo-Monome} et les \myglos*{glo-Binome}.
					Pour ces scénarios, l'utilisation du deuxième outil est nécessaire et malgré la charge cognitive importante que cela représente pour les \myglos*{glo-Monome},la tâche est réalisée à l'aide des deux outils (configuration \myglos*{glo-Bimanuel}).
					Dans ce cadre, la configuration \myglos*{glo-Monomanuel} adoptée par les \myglos*{glo-Binome} permet de meilleures performances comme le montre les analyses pour une distance active similaire.
					En effet, l'espace de travail couvert par les \myglos*{glo-Monome} est identique à celui des \myglos*{glo-Binome} mais leur incapacité à traiter cognitivement cette charge supplémentaire de travail les rend moins performants.

					L'analyse du nombre de sélections vient appuyer ces conclusions.
					En effet, les \myglos*{glo-Monome} effectuent moins de sélections que les \myglos*{glo-Binome} dans la réalisation des scénarios simples.
					Cependant, on comptabilise un nombre de sélections similaires entre les \myglos*{glo-Monome} et les \myglos*{glo-Binome} dans les scénarios complexes.

					Dans cette section, nous avons montré que les améliorations de performances des \myglos*{glo-Binome} par rapport aux \myglos*{glo-Monome} étaient très liées à la complexité de la tâche.
					En effet, sur des tâches de faible complexité, les \myglos*{glo-Monome} obtiennent de bonnes performances (même en configuration \myglos*{glo-Monomanuel}) alors que les \myglos*{glo-Binome} souffrent de \myglos*{glo-ConflitDeCoordination} : les performances sont similaires.
					Cependant, dans le cas de tâche complexes, les \myglos*{glo-ConflitDeCoordination} ne sont pas suffisamment pénalisants et la collaboration permet d'obtenir de meilleures performances que le travail seul.
					Dans la section précédente, nous avons montré que la configuration \myglos*{glo-Bimanuel} ne permet pas d'égaler les performances d'un travail en collaboration.
					Complétons cette conclusion par le fait qu'elle est surtout vraie pour les scénarios complexes.
				\end{mysubsubsection}
			\end{mysubsection}
			\begin{mysubsection}[sse-exp2-AmeliorationDeLApprentissagePourLesBinomes]{Amélioration de l'apprentissage pour les \myglosnl*{glo-Binome}}
				\begin{mysubsubsection}[sss-exp2-AmeliorationDeLApprentissagePourLesBinomes-DonneesEtStatistiques]{Données et statistiques}
					\begin{myfigure}
						\psset{xunit=0.285714286\textwidth,yunit=0.0125cm}
						\begin{myps}(-0.45,-90)(3,250)
							\myaxes(0,3){essai}(0,200)[50]{temps~(s)}
							\myboxplot{exp2-time-try.csv}
						\end{myps}
						\mycaption[fig-exp2-TempsDeRealisationDeChaqueEssai]{Temps de réalisation de chaque essai}
					\end{myfigure}

					La \myref{fig-exp2-TempsDeRealisationDeChaqueEssai} présente le temps de réalisation \myvard{1} des différents essais \myvari{3}.
					L'analyse montre un effet significatif du numéro de l'essai \myvari{3} sur le temps de réalisation \myvard{1} (\myanova{exp2-time-try-anova.tex}).
					Un test post-hoc de \mycite[author]{Mann-1947} avec une correction de \mycite[author]{Holm-1979} montre une évolution significative entre le premier essai et le deuxième essai ainsi qu'entre le deuxième essai et le troisième.

					\begin{myfigure}
						\psset{xunit=0.285714286\textwidth,yunit=0.0125cm}
						\begin{myps}(-0.45,-90)(3,250)
							\myaxes(0,3){essai}(0,200)[50]{temps~(s)}
							\myboxplot{exp2-time-try-group.csv}
							\mylegend{\myleg{\myglosnl{glo-Monome}}{myblue}\myand\myleg{\myglosnl{glo-Binome}}{myblue!70}}
						\end{myps}
						\mycaption[fig-exp2-TempsDeRealisationDeChaqueEssaiEnFonctionDuNombreDeSujets]{Temps de réalisation de chaque essai en fonction du nombre de sujets}
					\end{myfigure}

					La \myref{fig-exp2-TempsDeRealisationDeChaqueEssaiEnFonctionDuNombreDeSujets} présente le temps de réalisation \myvard{1} des différents essais \myvari{3} en fonction du nombre de sujets \myvari{1}.
					L'analyse montre qu'il n'y a pas d'effet significatif du nombre de sujets \myvari{1} sur le temps de réalisation \myvard{1} pour le premier essai (\myanova{exp2-time-try-group-anova-try1.tex}), le deuxième essai (\myanova{exp2-time-try-group-anova-try2.tex}) ou le troisième essai (\myanova{exp2-time-try-group-anova-try3.tex}).

					De plus, l'analyse montre un effet significatif du numéro de l'essai \myvari{3} sur le temps de réalisation \myvard{1} pour les \myglos*{glo-Monome} (\myanova{exp2-time-try-group-anova-monome.tex}) et pour les \myglos*{glo-Binome} (\myanova{exp2-time-try-group-anova-binome.tex}).
					Un test post-hoc de \mycite[author]{Mann-1947} avec une correction de \mycite[author]{Holm-1979} montre une évolution significative seulement à partir de dernier essai pour les \myglos*{glo-Monome} alors que l'évolution est significative dès le deuxième essai pour les \myglos*{glo-Binome}.

					\begin{myfigure}
						\psset{xunit=0.285714286\textwidth,yunit=0.04cm}
						\begin{myps}(-0.5,-30)(3,115)
							\myaxes(0,3){essai}(0,100)[25]{nombre de sélections~(nb)}
							\myboxplot{exp2-numsel-try-group.csv}
							\mylegend{\myleg{\myglosnl{glo-Monome}}{myblue}\myand\myleg{\myglosnl{glo-Binome}}{myblue!70}}
						\end{myps}
						\mycaption[fig-exp2-NombreDeSelectionsDeChaqueEssaiEnFonctionDuNombreDeSujets]{Nombre de sélections de chaque essai en fonction du nombre de sujets}
					\end{myfigure}

					La \myref{fig-exp2-NombreDeSelectionsDeChaqueEssaiEnFonctionDuNombreDeSujets} présente le nombre de sélections \myvard{2} des différents essais \myvari{3} en fonction du nombre de sujets \myvari{1}.
					L'analyse montre qu'il n'y a pas d'effet significatif du nombre de sujets \myvari{1} sur le nombre de sélections \myvard{2} pour le premier essai (\myanova{exp2-numsel-try-group-anova-try1.tex}) ou le troisième essai (\myanova{exp2-numsel-try-group-anova-try3.tex}).
					Cependant, l'analyse montre un effet significatif du nombre de sujets \myvari{1} sur le nombre de sélections \myvard{2} pour le deuxième essai (\myanova{exp2-numsel-try-group-anova-try2.tex}).

					De plus, l'analyse montre qu'il n'y a pas d'effet significatif du numéro de l'essai \myvari{3} sur le nombre de sélections \myvard{2} pour les \myglos*{glo-Monome} (\myanova{exp2-numsel-try-group-anova-monome.tex}).
					Cependant, l'analyse montre un effet significatif du numéro de l'essai \myvari{3} sur le nombre de sélections \myvard{2} pour les \myglos*{glo-Binome} (\myanova{exp2-numsel-try-group-anova-binome.tex}).
					Le test post-hoc de \mycite[author]{Mann-1947} avec une correction de \mycite[author]{Holm-1979} montre une diminution significative du nombre de sélections pour les \myglos*{glo-Binome} entre le premier et le dernier essai.

					\begin{myfigure}
						\psset{xunit=0.285714286\textwidth,yunit=1.25cm}
						\begin{myps}(-0.5,-0.9)(3,3)
							\myaxes(0,3){essai}(0,2.5)[0.5]{distance~(mm)}
							\myboxplot{exp2-active-try-group.csv}
							\mylegend{\myleg{\myglosnl{glo-Monome}}{myblue}\myand\myleg{\myglosnl{glo-Binome}}{myblue!70}}
						\end{myps}
						\mycaption[fig-exp2-DistanceActiveEntreLesEffecteursTerminauxPourChaqueEssaiEnFonctionDuNombreDeSujets]{Distance active entre les \myglosnl*{glo-EffecteurTerminal} pour chaque essai en fonction du nombre de sujets}
					\end{myfigure}

					La \myref{fig-exp2-DistanceActiveEntreLesEffecteursTerminauxPourChaqueEssaiEnFonctionDuNombreDeSujets} présente la distance active \myvard{4} des différents essais \myvari{3} en fonction du nombre de sujets \myvari{1}.
					La distance passive n'a pas été prise en considération étant donné le biais de mesure décrit dans la \myref{sse-exp2-AmeliorationDesPerformancesEnBinome}.
					L'analyse montre un effet significatif du nombre de sujets \myvari{1} sur la distance active \myvard{4} pour le premier essai (\myanova{exp2-active-try-group-anova-try1.tex}) et pour le deuxième essai (\myanova{exp2-active-try-group-anova-try2.tex}) mais pas pour le troisième essai (\myanova{exp2-active-try-group-anova-try3.tex}).

					De plus, l'analyse montre qu'il n'y a pas d'effet significatif du numéro de l'essai \myvari{3} sur la distance active \myvard{4} pour les \myglos*{glo-Binome} (\myanova{exp2-active-try-group-anova-binome.tex}).
					Cependant, l'analyse montre un effet significatif du numéro de l'essai \myvari{3} sur la distance active \myvard{4} pour les \myglos*{glo-Monome} (\myanova{exp2-active-try-group-anova-monome.tex}).
					Un test post-hoc de \mycite[author]{Mann-1947} avec une correction de \mycite[author]{Holm-1979} montre une évolution significative entre le premier essai et le troisième essai.

					\begin{myfigure}
						\psset{xunit=0.285714286\textwidth,yunit=1.25cm}
						\begin{myps}(-0.5,-0.9)(3,3)
							\myaxes(0,3){essai}(0,2.5)[0.5]{vitesse~(mm/s)}
							\myboxplot{exp2-speed-try-group.csv}
							\mylegend{\myleg{\myglosnl{glo-Monome}}{myblue}\myand\myleg{\myglosnl{glo-Binome}}{myblue!70}}
						\end{myps}
						\mycaption[fig-exp2-VitesseMoyennePourChaqueEssaiEnFonctionDuNombreDeSujets]{Vitesse moyenne pour chaque essai en fonction du nombre de sujets}
					\end{myfigure}

					La \myref{fig-exp2-VitesseMoyennePourChaqueEssaiEnFonctionDuNombreDeSujets} présente la vitesse moyenne \myvard{5} des différents essais \myvari{3} en fonction du nombre de sujets \myvari{1}.
					L'analyse montre un effet significatif du nombre de sujets \myvari{1} sur la vitesse moyenne \myvard{5} pour le premier essai (\myanova{exp2-speed-try-group-anova-try1.tex}), le second essai (\myanova{exp2-speed-try-group-anova-try2.tex}) et le troisième essai (\myanova{exp2-speed-try-group-anova-try3.tex}).

					De plus, l'analyse montre un effet significatif du numéro de l'essai \myvari{3} sur la vitesse moyenne \myvard{5} pour les \myglos*{glo-Monome} (\myanova{exp2-speed-try-group-anova-monome.tex}) et les \myglos*{glo-Binome} (\myanova{exp2-speed-try-group-anova-binome.tex}).
					Le test post-hoc de \mycite[author]{Mann-1947} avec une correction de \mycite[author]{Holm-1979} montre dans chaque cas (\myglos{glo-Monome} et \myglos{glo-Binome}) une augmentation significative après le premier essai.
				\end{mysubsubsection}
				\begin{mysubsubsection}[sss-exp2-AmeliorationDeLApprentissagePourLesBinomes-AnalyseEtDiscussion]{Analyse et discussion}
					L'observation des temps de réalisation de la tâche \myref*{fig-exp2-TempsDeRealisationDeChaqueEssai} nous permet de caractériser un apprentissage réel sur l'ensemble des trois réalisations de la tâche.
					Le détail de l'apprentissage en fonction du nombre de sujets sur la \myref{fig-exp2-TempsDeRealisationDeChaqueEssaiEnFonctionDuNombreDeSujets} apporte cependant un point important : les \myglos*{glo-Binome} améliorent plus rapidement leurs performances que les \myglos*{glo-Monome}.
					En effet, on constate une amélioration franche des performances dès le second essai dans le cas des \myglos*{glo-Binome} alors que ce n'est que sur le dernier essai que les \myglos*{glo-Monome} montrent une évolution.
					L'amélioration plus rapide des performances chez les \myglos*{glo-Binome} suggère un apprentissage plus rapide de la tâche, des outils et de tous les éléments de la plate-forme\footnote{On observe une amélioration des performances par apprentissage mais rien ne permet de distinguer quel aspect de la tâche a été intégrée le plus vite.}.

					En observant l'évolution des variables \myvard{1} (temps de réalisation) et \myvard{2} (nombre de sélections), on constate que les \myglos*{glo-Binome} ont un apprentissage rapide.
					Le temps de réalisation décroît ainsi que le nombre de sélections ce qui n'est pas le cas des \myglos*{glo-Monome}.
					En effet, le temps de réalisation des \myglos*{glo-Monome} décroît alors que le nombre de sélections ne décroît pas de manière significative.
					Au-fur-et-à-mesure des essais, les \myglos*{glo-Monome} apprennent et intègre la manipulation en configuration \myglos*{glo-Bimanuel} : ils augmentent ainsi leurs performances (diminution du temps de réalisation et du nombre de sélections de la main dominante) tout conservant un nombre de sélections relativement constant (par une augmentation du nombre de sélections de la main dominée).

					On observe clairement l'apprentissage progressif du deuxième outil mis à disposition des \myglos*{glo-Monome} dans la \myref{fig-exp2-DistanceActiveEntreLesEffecteursTerminauxPourChaqueEssaiEnFonctionDuNombreDeSujets}.
					Alors que l'espace de travail des \myglos*{glo-Binome} reste stable sur l'ensemble des essais, celui des \myglos*{glo-Monome} s'étend au-fur-et-à-mesure des essais jusqu'à atteindre une valeur similaire à celle des \myglos*{glo-Binome}.
					En effet, seul l'apprentissage permet de s'affranchir en partie de la charge cognitive importante que représente la manipulation \myglos*{glo-Bimanuel} \mycite{Wickens-1984} : avec l'apprentissage, les \myglos*{glo-Monome} sont capables de gérer un espace de travail de plus en plus grand.
					Le potentiel du deuxième outil n'est pas ignoré et il est utilisé (avec la main dominée) comme un moyen de fixer un \myglos{glo-Residu} déjà déplacé pendant que l'autre outil déforme.
					Ceci permet de déformer une partie de la molécule tout en conservant la stabilité de la partie déjà déformée.
					Les \myglos*{glo-Monome} ont la capacité d'adopter une stratégie plus adaptée à la situation car aucune limite de surcharge cognitive ne contraint les sujets.

					En ce qui concerne les vitesses moyennes, les \myglos*{glo-Monome} comme les \myglos*{glo-Binome} s'améliorent en manipulant plus rapidement.
					Cependant, les \myglos*{glo-Binome} restent nettement plus rapides que les \myglos*{glo-Monome}.
					Cette amélioration peut être mise en relation avec l'amélioration des temps de réalisation : la tâche est réalisée plus rapidement car les sujets manipulent plus rapidement.

					Dans cette section, nous avons mis en évidence les améliorations en terme d'apprentissage pour les configurations collaboratives sans distinction sur les aspects de l'apprentissage (plate-forme, outils, tâche, \myetc).
					En effet, les \myglos*{glo-Binome} atteignent des performances optimales rapidement tandis que les \myglos*{glo-Monome} ont besoin de plus de temps pour converger vers de bonnes performances.
					La capacité des \myglos*{glo-Binome} à communiquer, échanger et conseiller permet de mutualiser l'apprentissage et de l'accélérer.
					De plus, un \myglos{glo-Binome} peut bénéficier des connaissances spécifiques ou de l'expérience d'un des membres du \myglos{glo-Binome} et ainsi mutualiser les aptitudes de chacun pour créer une vraie dynamique de groupe.
					La configuration \myglos*{glo-Bimanuel} offre une alternative de manipulation aux \myglos*{glo-Monome} avec une surcharge de travail trop importante : l'apprentissage est plus difficile.
					De plus, les \myglos*{glo-Monome} ont probablement atteint les limites de la charge cognitive maximum supportée avec cette configuration : l'ajout de nouvelles fonctionnalités serait probablement inefficace (contrairement aux \myglos*{glo-Binome}).
				\end{mysubsubsection}
			\end{mysubsection}
			\begin{mysubsection}[sse-exp2-ResultatsQualitatifs]{Résultats qualitatifs}
				Le questionnaire est destiné à évaluer la collaboration du point du vue de l'utilisateur.

				Tout d'abord, la grande majorité des sujets travaillant en \myglos{glo-Binome} se sont trouvés utiles dans cette tâche de collaboration (\myanova{exp2-evaluation-help.tex})\footnote{L'échelle de notation comprise entre \mynum{1} à \mynum{5} mais les moyennes ont été normalisées entre \mynum{0} et \mynum{4}.}.
				Ce résultat élevé permet de vérifier que les sujets ne se sentent pas mis de côté et participent activement à la réalisation de la tâche.
				Cette collaboration peut se traduire par une participation active à la déformation ou par une participation plutôt passive (échanges verbaux, conseils, remarques, \myetc).
				Dans un cas comme dans l'autre, les sujets ne sont pas isolés ce qui permet d'éviter les phénomènes de \myglos{glo-ParesseSociale}.

				Le sentiment d'avoir été \myglos{glo-Meneur} durant la réalisation de la tâche est relativement neutre (\myanova{exp2-evaluation-leader.tex}).
				Cependant, cette question semble biaisée.
				En effet, les sujets ne souhaitent pas prétendre avoir été \myglos{glo-Meneur} ou chef des opérations par modestie.
				Paradoxalement, ils ne souhaitent pas non plus avouer avoir été dirigé par quelqu'un d'autre par fierté.
				D'ailleurs, on observe un écart-type relativement bas concernant cette note ce qui signifie que la majorité des sujets ont répondu de façon neutre.

				L'évaluation de la communication confirme ce qui a été observé dans la précédente expérimentation \myref*{sss-exp1-EvaluationDuTravailEnCollaboration}.
				En effet, l'importance de la communication verbale a été mise en avant (\myanova{exp2-evaluation-verbal}).
				Par opposition, les sujets ont considéré qu'ils n'utilisaient quasiment pas la modalité virtuelle (\myanova{exp2-evaluation-virtual.tex}) et encore moins la modalité gestuelle (\myanova{exp2-evaluation-gestural.tex}) pour communiquer.
				La communication verbale étant la plus naturelle, il n'est pas étonnant d'obtenir un tel score.
				De la même façon, la communication gestuelle est compliquée étant donné que les sujets sont en train de manipuler une interface haptique.
				De plus, leur vision se focalise principalement sur le déroulement de la tâche à l'écran mais pas sur le partenaire ce qui laisse peu de place à la communication gestuelle.
				Cependant, les sujets estiment ne pas souvent avoir recours aux communications virtuelles.
				Cette modalité de communication offre des possibilités intéressantes puisqu'elle est intégrée à l'environnement de travail et matérialisée principalement par le curseur.
				L'expérimentation ne proposant aucun fonctionnalité particulière permettant d'exploiter cette modalité de communication explique probablement ce faible taux d'utilisation.
				La dernière expérimentation \myref*{cha-TravailCollaboratifAssisteParHaptique} propose des outils de désignation qui vont permettre d'exploiter le potentiel de ce canal de communication.

				Pour finir, les sujets ont été interrogés sur leur configuration de travail préférée.
				Le questionnaire propose aux sujets d'évaluer une configuration pour laquelle ils n'ont pas testés.
				La configuration \myglos*{glo-Monomanuel} en \myglos{glo-Monome} (qui n'a pas été testée) a été relativement peu choisie (\myanova{exp2-evaluation-monome-monomanual.tex}).
				Les sujets évalués en \myglos{glo-Monome} sont mitigés sur l'intérêt d'une configuration \myglos*{glo-Monomanuel} en \myglos{glo-Binome} (\myanova{exp2-evaluation-binome-monomanual.tex}).
				De la même façon, les sujets évalués en \myglos{glo-Binome} sont mitigés sur l'intérêt d'une configuration \myglos*{glo-Bimanuel} en \myglos{glo-Monome} (\myanova{exp2-evaluation-monome-bimanual.tex}).
				Quoiqu'il en soit, ils ont été seulement \myanova{exp2-evaluation-preference-monome-bimanual.tex} à préférer la configuration \myglos*{glo-Bimanuel} en \myglos{glo-Monome} alors qu'ils ont été \myanova{exp2-evaluation-preference-binome-monomanual.tex} à opter pour la configuration \myglos*{glo-Monomanuel} en \myglos{glo-Binome}.
				Une majorité des sujets semble préférer la configuration collaborative.
			\end{mysubsection}
		\end{mysection}
		\begin{mysection}[sec-exp2-Synthese]{Synthèse}
			\begin{mysubsection}[sse-exp2-ResumeDesResultats]{Résumé des résultats}
				Dans cette seconde expérimentation, nous avons comparé et étudié les performances de \myglos*{glo-Monome} et de \myglos*{glo-Binome} sur une tâche de déformation avec un nombre identique de ressources.
				De plus, nous avons cherché à observer l'apport de la configuration collaborative sur l'apprentissage sur les performances.
				L'objectif était de placer la configuration collaborative dans un contexte de déformation avec de nouvelles contraintes par rapport aux tâches de recherche et de sélection.

				Il a été montré qu'avec un nombre de ressources déterminées (un outil de manipulation et deux outils de déformation dans notre cas), il est préférable de les répartir sur plusieurs sujets.
				Cette répartition des ressources permet une meilleure distribution cognitive des charges de travail.
				En effet, la charge cognitive est trop importante pour un utilisateur seul.
				La configuration collaborative, bien que souffrant de \myglos*{glo-ConflitDeCoordination}, obtient tout de même des meilleures performances.

				Deuxièmement, nous avons montré que la configuration collaborative est particulièrement performante pour les scénarios à forte complexité.
				En ce qui concerne les scénarios à faible complexité, les performances d'une configuration collaborative ne sont ni meilleures, ni moins bonne que celle d'un seul manipulateur en configuration \myglos*{glo-Bimanuel}.
				On notera tout de même que les sujets semblent préférer la configuration collaborative.

				Le troisième résultat important concerne l'apprentissage.
				Nous avons montré que le travail en collaboration a une influence sur l'évolution de l'apprentissage.
				En effet, l'apprentissage est catalysé par la communication et les échanges entre les sujets.
				La complexité de la tâche ainsi que de la plate-forme (rendu visuel, outils, \myetc) nécessite un apprentissage important.
				L'apprentissage accéléré provoqué par une configuration collaborative est donc un avantage permettant d'appréhender plus rapidement la tâche à réaliser.
			\end{mysubsection}
			\begin{mysubsection}[sse-exp2-Conclusion]{Conclusion}
				Cette expérimentation nous a permis de comparer une configuration collaborative à une configuration \myglos*{glo-Bimanuel} possédant chacune le même nombre de ressources.
				Nous avons vu les avantages d'une configuration collaborative avec des \myglos*{glo-Binome}.
				L'étape suivante sera l'étude du travail collaboratif sur des groupes de plus de deux sujets.
				Ceci devrait permettre d'augmenter encore le potentiel cognitif du groupe.

				Pour mener une telle étude, il va falloir proposer des scénarios plus complexes.
				Cette deuxième expérimentation a montré une nouvelle fois le rôle prépondérant de la taille de la molécule dans la complexité de la tâche.
				Nous verrons que les molécules proposées dans la prochaine étude sont significativement plus importantes que celle utilisées jusqu'à présent.

				L'ajout de sujets supplémentaires va probablement générer des dynamiques de groupes qui n'avait pas de raison d'exister au sein d'un \myglos{glo-Binome}.
				Cette troisième étude permettra l'observation des dynamiques et de les caractériser.
				L'objectif sera de détecter les limites et les contraintes afin de pouvoir fournir des outils pour répondre aux problématiques soulevées.

				Cette deuxième expérimentation a également permis de remettre en cause la pertinence d'une manipulation en configuration \myglos*{glo-Bimanuel}.
				D'après les analyses, la charge cognitive qu'apporte la gestion d'un deuxième outil de déformation est trop importante.
				Cependant, l'outil de déformation est relativement complexe à appréhender.
				Il ne faut donc pas exclure la possibilité de fournir un outil simple et un outil complexe pour une manipulation en configuration \myglos*{glo-Bimanuel}.
				Nous verrons que la configuration de la dernière étude \myref*{cha-TravailCollaboratifAssisteParHaptique} propose une configuration \myglos*{glo-Bimanuel} avec un outil simple de déplacement et un outil plus complexe de désignation.

				Le questionnaire nous a également permis de mettre en avant les lacunes en ce qui concerne l'utilisation de la modalité virtuelle.
				La dernière expérimentation sera l'occasion d'introduire des nouveaux outils adaptés pour permettre d'utiliser efficacement cette modalité pour la communication et en particulier, un outil de désignation.
			\end{mysubsection}
		\end{mysection}
	\end{mychapter}
	\begin{mychapter}[cha-LaDynamiqueDeGroupe]{La dynamique de groupe}
		\begin{mysection}[sec-exp3-Introduction]{Introduction}
			À présent, les différentes \myacro*{acr-PCV} ont été étudiées dans un contexte de collaboration étroitement couplée à travers les deux précédents chapitres.
			Cependant, l'observation du travail collaboratif ne peut être restreint à l'étude des \myglos*{glo-Binome}.
			En effet, \mycite[author]{Roethlisberger-1939} a mis en évidence les dynamiques de groupe basés sur les travaux de Elton \myname{Mayo}.
			Ces dynamiques de groupe montre une collaboration très différente de que ce que nous avons pu observer chez les \myglos*{glo-Binome}.

			Ce chapitre constitue notre première étude sur le travail collaboratif avec des groupes d'utilisateurs\footnote{\mycite[author]{Bales-1950} considère qu'un groupe est constitué au minimum de trois personnes.}.
			Au regard des travaux existants, une dynamique de groupe devrait émerger.
			Cependant, notre contexte de travail est différent des précédents travaux sur le sujet : nous nous intéressons aux collaborations étroitement couplées.
			C'est dans ce contexte que nous allons observer les dynamiques de groupe qui émergent.
			Nous nous plaçons de nouveau dans un contexte de déformation moléculaire qui fournit un environnement d'étude propice aux collaboratio étroitement couplée.
			De plus, nous souhaitons tester l'utilité du \mybrainstorming\footnote{Pour la suite des développements, le mot \mybrainstorming sera utilisé plutôt que le mot \myemph{remue-méninges} car il est plus utilisé dans la littérature.} qui d'après \mycite[author]{Osborn-1963}, améliore les performances de groupe.
		\end{mysection}
		\begin{mysection}[sec-exp3-CollaborationDeGroupe]{Collaboration de groupe}
			\begin{mysubsection}[sse-exp3-TravauxExistants]{Travaux existants}
				L'ouvrage de \mycite[author]{Mugny-1995} aborde les problématiques de la psychologie sociale dans le cadre général et consacre une partie à la dynamique de groupe.
				Les premières études sur la dynamique de groupe date de la révolution industrielle entre la fin du \mycentury{18}~siècle et au début du \mycentury{19}~siècle avec en particulier, les travaux de Elton \myname{Mayo} au sein de l'entreprise \myHawthorne.
				Cette étude, destinée à étudier l'effet des conditions de travail (température, temps de pause, \myetc), a été effectuée entre les années 1927 et 1932.
				Cependant, l'étude a montré que l'amélioration de la productivité des ouvriers n'était pas liée aux conditions de travail.
				\mycite[author]{Roethlisberger-1939} explique cette amélioration par la stimulation sociale qu'exerce chaque individu sur ses partenaires : c'est la \myglos{glo-MotivationSociale}.
				Les résultats de cette étude sur les groupes de taille importante sont actuellement utilisés dans les techniques modernes de \myemph{management} \mycite{Wood-2004,Bruce-2006}.

				Cependant, en parallèle à cette théorie de la dynamique des groupes basée sur la \myglos{glo-MotivationSociale}, \mycite[author]{Ringelmann-1913} met en évidence une théorie radicalement différente.
				En effet, à travers un exercice de traction sur une corde, il montre que la somme des efforts individuels est plus importante que l'effort combiné du groupe, chaque sujet se fiant à son voisin pour réaliser la tâche.
				Ce phénomène, appelé \myglos{glo-ParesseSociale}, s'oppose aux résultats obtenus par \mycite[author]{Roethlisberger-1939} sur la \myglos{glo-MotivationSociale}.
				Une étude plus poussée de ce phénomène effectuée par \mycite[author]{Latane-1979} confirme les résultats obtenus sur la \myglos{glo-ParesseSociale}.
				Cependant, \mycite[author]{Latane-1979} propose de limiter ce problème en renforçant la responsabilité individuelle plutôt que de la diffuser sur le groupe.
				La responsabilisation par la définition de rôles distincts permet de ne pas se décharger des actions à réaliser sur ses partenaires tout en conservant les effets bénéfiques de la \myglos{glo-MotivationSociale}.

				Une part de ces études sur la dynamique de groupe est consacrée aux groupes de petites tailles appelés également groupes restreints.
				\mycite[author]{Bales-1950} proposent les premières analyses sur ces groupes de trois à une vingtaine de partenaires.
				Les résultats montrent que quelque soit la taille du groupe, le groupe sera dominé par un voire deux membres du groupe.
				Cependant, \mycite[author]{Zajonc-1965} montre que les groupes sont performants sur des tâches simples mais peu performants sur des tâches complexes.
				En effet, les tâches simples sont réalisées sans crainte du jugement ou de l'évaluation par les partenaires.
				Sur une tâche de nature complexe, l'évaluation et le jugement par les partenaires est un frein et a pour conséquences de faire baisser les performances du groupe.

				Au sein des groupes restreints, \mycite[author]{Osborn-1963} propose d'améliorer les performances dans les groupes restreints par l'introduction de la notion de \mybrainstorming.
				Pourtant, \mycite[author]{Diehl-1987} montre que le \mybrainstorming apporte moins de bénéfices en groupe que lorsqu'il est effectué individuellement.
				\mycite[author]{Poole-2005} expliquent que les groupes focalisent en priorité sur les informations qu'ils ont en commun.
				La peur de l'évaluation négative va empêcher l'émergence de solutions originales.
				Cependant, \mycite[author]{Tuckman-1965} considère que le \mybrainstorming permet tout de même de renforcer la cohésion sociale et d'améliorer les performances du groupe à long terme.

				Jusqu'à présent, les études concernant la dynamique des groupes et plus particulièrement celle concernant les groupes restreints sont nombreuses.
				Cependant, chaque étude proposée concerne des tâches autour d'une collaboration faiblement couplée.
				Dans ce chapitre, nous étudions la dynamique des groupes autour d'une collaboration fortement couplée afin d'observer les différences avec les configurations précédemment étudiées dans la littérature.
			\end{mysubsection}
			\begin{mysubsection}[sse-exp3-Objectifs]{Objectifs}
				Dans cette troisième étude, nous souhaitons étudier le travail collaboratif pour les groupes restreints.
				Jusqu'à présent, nous avons été confrontés à des \myglos*{glo-Binome}.
				La littérature montre des stratégies propres aux groupes restreints et distinctes de celles adoptées par les \myglos*{glo-Binome}.
				Nous souhaitons étudier cette dynamique de groupe pour la collaboration étroitement couplée.

				Étant donné les résultats obtenus dans nos précédentes études, nous souhaitons observer une amélioration des performances en fonction du nombre d'utilisateurs pour un scénario de collaboration étroitement couplée.
				Pourtant, les conclusions de \mycite[author]{Zajonc-1965} montrent que les groupes sont moins performants lorsqu'ils sont confrontés à une tâche complexe.
				Cependant, étant donné la coordination nécessaire demandée par la tâche, nous pensons que les conclusions obtenues pour la collaboration étroitement couplée seront différentes de celles obtenues par \mycite[author]{Zajonc-1965} dans le cadre d'une tâche à faible couplage.
				En effet, nous avons vu précédemment que la configuration \myglos*{glo-Bimanuel} menait à une surcharge cognitive difficile à traiter par les sujets; la coordination nécessaire pour les scénarios proposés devrait donner un avantage aux groupes restreints.

				D'après les conclusions de \mycite[author]{Bales-1950}, un groupe est toujours mené par un ou deux utilisateurs, quelque soit la taille du groupe.
				Nous émettons l'hypothèse que ces \myglos*{glo-Meneur} vont également apparaître dans le cadre d'une collaboration étroitement couplée.

				Finalement, nous souhaitons proposer une solution pour limiter les \myglos*{glo-ConflitDeCoordination}.
				\mycite[author]{Bales-1950} a noté que les groupes restreints consacrent du temps pour se connaître (sans rapport avec la tâche à réaliser) puis discutent à propos de la stratégie à adopter.
				Afin de répondre à ce besoin de se connaître, les groupes choisis dans cette expérimentation sont tous constitués de sujets se connaissant déjà dans le cadre professionnel.
				Puis, afin d'améliorer l'efficacité d'une discussion à propos de l'élaboration d'une stratégie, nous souhaitons tester la mise en place d'une période de \mybrainstorming au début de la tâche.
				Nous émettons l'hypothèse que cette période permettra aux groupes de s'organiser et d'élaborer une stratégie afin d'améliorer les performances globales du groupe.
				De plus, si l'hypothèse précédente se vérifie, le \mybrainstorming devrait permettre d'identifier plus rapidement le ou les \myglos*{glo-Meneur} du groupe.
			\end{mysubsection}
		\end{mysection}
		\begin{mysection}[sec-exp3-PresentationDeLExperimentation]{Présentation de l'expérimentation}
			\begin{mysubsection}[sse-exp3-DescriptionDeLaTache]{Description de la tâche}
				La tâche proposée est la déformation de molécules dans un \myacro{acr-EVC}.
				L'objectif est de rendre une molécule complexe conforme à une molécule modèle.
				Dans cette expérimentation, la molécule \myTRPCAGE est utilisée pour la phase d'entraînement.
				Des molécules plus complexes (\myPrion et \myUbiquitin) sont utilisées pour les scénarios de déformation collaborative.
				Ces molécules sont détaillées dans la \myref{sse-pro-ListeDesMolecules}.

				Le mécanisme de sélection et d'affichage est strictement identique à la seconde expérimentation \myref*{sse-exp2-SpecificitesDuProtocoleExperimental}.
				De la même façon, le système d'évaluation basé sur le score \myacro{acr-RMSD} est identique.
				On pourra trouver la description de ces éléments dans la \myref{sse-exp2-DescriptionDeLaTache}.
				\begin{mysubsubsection}[sss-exp3-DescriptionDesScenarios]{Description des scénarios}
					Deux scénarios sont proposés : un scénario avec des interactions faiblement couplées et un scénario avec des interactions fortement couplées.
					Les paragraphes suivants décrivent ces deux scénarios :
					\begin{description}
						\item[Scénario~\myscenario{1}]
							Basé sur la molécule \myPrion, il nécessite de replacer correctement une chaîne de \mynum{16}~\myglos*{glo-Residu} par rapport à un modèle.
							Cette chaîne se trouve en périphérie de la molécule et n'est donc pas soumise à de fortes contraintes physiques.
							Ce scénario est divisible en tâches élémentaires présentant de faibles interactions physiques.
							L'objectif est d'obtenir une collaboration faiblement couplée.
						\item[Scénario~\myscenario{2}]
							Basé sur la molécule \myUbiquitin, il nécessite de replacer correctement une chaîne de \mynum{19}~\myglos*{glo-Residu} par rapport à un modèle.
							Cette chaîne se trouve au sein de la molécule où elle est soumise à de fortes contraintes physiques, notamment au milieu de la chaîne; le contrôle précis de la déformation au milieu de la chaîne est complexe.
							La réalisation de ce scénario nécessite plusieurs points de contrôle et une coordination de l'ensemble des sujets.
							L'objectif est d'obtenir une collaboration étroitement couplée.
					\end{description}
				\end{mysubsubsection}
			\end{mysubsection}
			\begin{mysubsection}[sse-exp3-SpecificitesDuProtocoleExperimental]{Spécificités du protocole expérimental}
				Le dispositif expérimental utilisé, basé sur celui présenté dans le \myref{cha-pro-DispositifExperimental}, a été adapté pour les besoins de l'expérimentation.
				Les modifications sont présentées dans les sections qui vont suivre.
				Le protocole expérimental est détaillé dans la \myref{sec-met-exp3-TroisiemeExperimentation} avec un résumé dans la \myref{tab-exp3-SyntheseDeLaProcedureExperimentale}.
				\begin{mysubsubsection}[sss-exp3-Materiel]{Matériel}
					Cette expérimentation se focalise sur le travail de groupe et en particulier, sur les \myglos*{glo-Quadrinome}.
					Il est nécessaire d'ajouter deux outils de déformation supplémentaires à la plate-forme \myref*{sec-pro-MaterielExperimental}.
					Deux \myOmni supplémentaires sont posés sur la table, devant les sujets, de manière à ce que chacun puisse avoir accès à une interface haptique.
					Un serveur \myacro{acr-VRPN} exécuté par des machines de faible puissance est ajouté pour chaque nouvel \myOmni.

					Chaque sujet d'un \myglos{glo-Quadrinome} possède un outil de déformation à sa disposition.
					En ce qui concerne les \myglos*{glo-Binome}, chaque sujet possède deux outils pour une configuration \myglos*{glo-Bimanuel}.

					Cette expérimentation sur le travail collaboratif de groupe est l'occasion d'observer les communications.
					Afin d'enregistrer ces communications, une caméra vidéo \mySony (\textsc{pj50v hd}) a été placée derrière les sujets afin de filmer les sujets de dos et l'écran de vidéoprojection dans un même plan.
					Cet enregistrement permet de conserver toutes les communications orales ainsi que les actions effectuées en parallèle (action virtuelle ou réelle).
					Ces vidéos sont exportées et séquencées \myafortiori à l'aide du logiciel \myiMovie.

					La \myref{fig-exp3-SchemaDuDispositifExperimental} illustre le dispositif expérimental par un schéma.
					La \myref{fig-exp3-PhotographieDuDispositifExperimental} est une photographie de la salle d'expérimentation.

					\begin{myfigure}
						\myimage{exp3-schema}
						\mycaption[fig-exp3-SchemaDuDispositifExperimental]{Schéma du dispositif expérimental}
					\end{myfigure}
					\begin{myfigure}
						\myimage{exp3-photo}
						\mycaption[fig-exp3-PhotographieDuDispositifExperimental]{Photographie du dispositif expérimental}
					\end{myfigure}
				\end{mysubsubsection}
				\begin{mysubsubsection}[sss-exp3-Visualisation]{Visualisation}
					Cette expérimentation propose une tâche relativement similaire à la précédente expérimentation.
					La principale différence concerne la complexité des molécules puisque les molécules contiennent une centaine de \myglos*{glo-Residu} contrairement à la précédente expérimentation concernant des molécules d'une quinzaine de \myglos*{glo-Residu}.
					La molécule \myPrion est utilisée pour le scénario~\myscenario{1} \myref*{fig-exp3-RepresentationDeLaMoleculePrionPourLeScenario1}; la molécule \myUbiquitin est utilisée pour le scénario~\myscenario{2} \myref*{fig-exp3-RepresentationDeLaMoleculeUbiquitinPourLeScenario2}.

					\begin{myfigure}
						\myimage{exp3-scenario1}
						\mycaption[fig-exp3-RepresentationDeLaMoleculePrionPourLeScenario1]{Représentation de la molécule \myPrion pour le scénario~\myscenario{1}}
					\end{myfigure}
					\begin{myfigure}
						\myimage{exp3-scenario2}
						\mycaption[fig-exp3-RepresentationDeLaMoleculeUbiquitinPourLeScenario2]{Représentation de la molécule \myUbiquitin pour le scénario~\myscenario{2}}
					\end{myfigure}
				\end{mysubsubsection}
				\begin{mysubsubsection}[sss-exp3-OutilsDeManipulation]{Outils de manipulation}
					Cette expérimentation fait intervenir des \myglos*{glo-Quadrinome}.
					Cependant, c'est la première expérimentation pour laquelle aucun outil d'orientation de la molécule n'est fourni aux sujets.
					En effet, étant donné les observations des précédentes expérimentations, nous avons jugé que la présence de cet outil est générateur de \myglos*{glo-ConflitDeCoordination}.
					Durant les précédentes expérimentations, le nombre de \myglos*{glo-ConflitDeCoordination} était relativement limités car ils ne concernaient que des \myglos*{glo-Binome}.
					Avec des \myglos{glo-Quadrinome}, un tel outil pourrait produire beaucoup plus de chaos, ce que nous souhaitons éviter.

					En ce qui concerne les outils de déformation, ce sont exactement les mêmes que dans la seconde expérimentation \myref*{sss-exp2-OutilsDeManipulation}.
					Chaque \myglos{glo-Residu} qu'un sujet sélectionne est mis en surbrillance à la fois sur la molécule déformable et sur la molécule modèle.
				\end{mysubsubsection}
				\begin{mytable}
					\mycaption[tab-exp3-SyntheseDeLaProcedureExperimentale]{Synthèse de la procédure expérimentale}
					\newcommand{\mytitlecolumn}[2]{%
						\multirow{#1}*{%
							\begin{minipage}{6em}%
								\raggedleft #2%
							\end{minipage}%
						}
					}
					\newlength{\expthreefirstcolumn}
					\newlength{\expthreesecondcolumn}
					\setlength{\expthreefirstcolumn}{7em}
					\setlength{\expthreesecondcolumn}{\textwidth}
					\addtolength{\expthreesecondcolumn}{-\expthreefirstcolumn}
					\addtolength{\expthreesecondcolumn}{-4\tabcolsep}
					\begin{mytabular}{>{\bfseries}p{\expthreefirstcolumn}p{\expthreesecondcolumn}}
						\mytoprule
						\mytitlecolumn{1}{Tâche}                   & Déformation d'une molécule en groupe                                                  \\
						\mymiddlerule[\heavyrulewidth]
						\mytitlecolumn{2}{Hypothèses}              & \myhypothesis{1} Amélioration des performances en \myglosnl{glo-Quadrinome}           \\
						                                           & \myhypothesis{2} Émergence de \myglosnl{glo-Meneur} dans le \myglosnl{glo-Quadrinome} \\
						                                           & \myhypothesis{3} Le \mybrainstorming structure le \myglosnl{glo-Quadrinome}           \\
						\mymiddlerule
						\mytitlecolumn{3}{Variables indépendantes} & \myvari{1} Nombre de sujets                                                           \\
						                                           & \myvari{2} Complexité de la tâche                                                     \\
						                                           & \myvari{3} Temps alloué pour le \mybrainstorming                                      \\
						\mymiddlerule
						\mytitlecolumn{5}{Variables dépendantes}   & \myvard{1} Temps de réalisation                                                       \\
						                                           & \myvard{2} Fréquence des sélections                                                   \\
						                                           & \myvard{3} Vitesse moyenne                                                            \\
						                                           & \myvard{4} Force moyenne appliquée par les sujets                                     \\
						                                           & \myvard{5} Communications verbales                                                    \\
						\mymiddlerule[\heavyrulewidth]
						\multicolumn{2}{c}{%
							\small%
							\begin{tabular}{^C-C-C-C}
								\myrowstyle{\bfseries}
								Condition \mycondition{1} & Condition \mycondition{2}         & Condition \mycondition{3} & Condition \mycondition{4}         \\
								\mymiddlerule
								\mynum{2}~sujets          & \mynum{2}~sujets                  & \mynum{4}~sujets          & \mynum{4}~sujets                  \\
								\myGlosnl{glo-Bimanuel}   & \myGlosnl{glo-Bimanuel}           & \myGlosnl{glo-Monomanuel} & \myGlosnl{glo-Monomanuel}         \\
								\mymiddlerule
								Pas de \mybrainstorming   & \mynum[mn]{1} de \mybrainstorming & Pas de \mybrainstorming   & \mynum[mn]{1} de \mybrainstorming \\
								\mymiddlerule
								Scénario~\myscenario{1}   & Scénario~\myscenario{1}           & Scénario~\myscenario{1}   & Scénario~\myscenario{1}           \\
								Scénario~\myscenario{2}   & Scénario~\myscenario{2}           & Scénario~\myscenario{2}   & Scénario~\myscenario{2}           \\
							\end{tabular}
						} \\
						\mybottomrule
					\end{mytabular}
				\end{mytable}
			\end{mysubsection}
		\end{mysection}
		\begin{mysection}[sec-exp3-Resultats]{Résultats}
			\begin{mysubsection}[sse-exp3-AmeliorationDesPerformances]{Amélioration des performances}
				\begin{mysubsubsection}[sss-exp3-AmeliorationDesPerformances-DonneesEtTestsStatistiques]{Données et tests statistiques}
					\begin{myfigure}
						\psset{xunit=0.274914089\textwidth,yunit=0.005cm}
						\begin{myps}(-0.425,-220)(2,640)
							\myaxes(0,2){scénario}(0,600)[100]{temps~(s)}
							\myboxplot{exp3-time-molecule.csv}
						\end{myps}
						\mycaption[fig-exp3-TempsDeRealisationDesScenarios]{Temps de réalisation des scénarios}
					\end{myfigure}

					La \myref{fig-exp3-TempsDeRealisationDesScenarios} présente le temps de réalisation \myvard{1} de chaque scénario \myvari{2}.
					L'analyse montre un effet significatif des scénarios \myvari{2} sur le temps de réalisation \myvard{1} (\myanova{exp3-time-molecule-anova.tex}).

					\begin{myfigure}
						\psset{xunit=0.444444444\textwidth,yunit=0.005cm}
						\begin{myps}(-0.25,-220)(2,720)
							\myaxes(0,2){scénario}(0,600)[100]{temps~(s)}
							\myboxplot{exp3-time-molecule-group.csv}
							\mylegend{\myleg{\myglosnl{glo-Binome}}{myblue}\myand\myleg{\myglosnl{glo-Quadrinome}}{myblue!70}}
						\end{myps}
						\mycaption[fig-exp3-TempsDeRealisationDesScenariosEnFonctionDuNombreDeParticipants]{Temps de réalisation des scénarios en fonction du nombre de participants}
					\end{myfigure}

					La \myref{fig-exp3-TempsDeRealisationDesScenariosEnFonctionDuNombreDeParticipants} présente le temps de réalisation \myvard{1} de chaque scénario \myvari{2} en fonction du nombre de sujets \myvari{1}.
					L'analyse montre qu'il n'y a pas d'effet significatif du nombre de sujets \myvari{1} sur le temps de réalisation \myvard{1} de la molécule \myPrion (\myanova{exp3-time-molecule-group-anova-prion.tex}).
					De la même façon, l'analyse montre qu'il n'y a pas d'effet significatif du nombre de sujets \myvari{1} sur le temps de réalisation \myvard{1} de la molécule \myUbiquitin (\myanova{exp3-time-molecule-group-anova-ubiquitin.tex}).

					\begin{myfigure}
						\psset{xunit=0.43956044\textwidth,yunit=10cm}
						\begin{myps}(-0.275,-0.11)(2,0.36)
							\myaxes(0,2){scénario}(0,0.31)[0.05]{sélections~(nb/s)}
							\myboxplot{exp3-freqsel-molecule-group.csv}
							\mylegend{\myleg{\myglosnl{glo-Binome}}{myblue}\myand\myleg{\myglosnl{glo-Quadrinome}}{myblue!70}}
						\end{myps}
						\mycaption[fig-exp3-FréquenceDesSelectionsSurLesScenariosEnFonctionDuNombreDeParticipants]{Fréquence des sélections sur les scénarios en fonction du nombre de participants}
					\end{myfigure}

					La \myref{fig-exp3-FréquenceDesSelectionsSurLesScenariosEnFonctionDuNombreDeParticipants} présente la fréquence de sélection \myvard{2} de chaque scénario \myvari{2} en fonction du nombre de sujets \myvari{1}.
					L'analyse montre qu'il n'y a pas d'effet significatif du nombre de sujets \myvari{1} sur la fréquence de sélection \myvard{2} de la molécule \myPrion (\myanova{exp3-freqsel-molecule-group-anova-prion.tex}).
					De la même façon, l'analyse montre qu'il n'y a pas d'effet significatif du nombre de sujets \myvari{1} sur la fréquence de sélection \myvard{2} de la molécule \myUbiquitin (\myanova{exp3-freqsel-molecule-group-anova-ubiquitin.tex}).

					\begin{myfigure}
						\psset{xunit=0.43956044\textwidth,yunit=3cm}
						\begin{myps}(-0.275,-0.35)(2,1.45)
							\myaxes(0,2){scénario}(0,1.25)[0.25]{vitesse~(mm/s)}
							\myboxplot{exp3-speed-molecule-group.csv}
							\mylegend{\myleg{\myglosnl{glo-Binome}}{myblue}\myand\myleg{\myglosnl{glo-Quadrinome}}{myblue!70}}
						\end{myps}
						\mycaption[fig-exp3-VitesseMoyenneSurLesScenariosEnFonctionDuNombreDeParticipants]{Vitesse moyenne sur les scénarios en fonction du nombre de participants}
					\end{myfigure}

					La \myref{fig-exp3-VitesseMoyenneSurLesScenariosEnFonctionDuNombreDeParticipants} présente la vitesse moyenne \myvard{3} de chaque scénario \myvari{2} en fonction du nombre de sujets \myvari{1}.
					L'analyse montre un effet significatif du nombre de sujets \myvari{1} sur la vitesse moyenne \myvard{3} de la molécule \myPrion (\myanova{exp3-speed-molecule-group-anova-prion.tex}).
					De la même façon, l'analyse montre un effet significatif du nombre de sujets \myvari{1} sur la vitesse moyenne \myvard{3} de la molécule \myUbiquitin (\myanova{exp3-speed-molecule-group-anova-ubiquitin.tex}).

					\begin{myfigure}
						\psset{xunit=0.449438202\textwidth,yunit=0.1cm}
						\begin{myps}(-0.225,-11)(2,36)
							\myaxes(0,2){scénario}(0,30)[5]{échanges verbaux~(nb)}
							\myboxplot{exp3-talk-molecule-group.csv}
							\mylegend{\myleg{\myglosnl{glo-Binome}}{myblue}\myand\myleg{\myglosnl{glo-Quadrinome}}{myblue!70}}
						\end{myps}
						\mycaption[fig-exp3-NombreDEchangeVerbauxSurLesScenariosEnFonctionDuNombreDeParticipants]{Nombre d'échanges verbaux sur les scénarios en fonction du nombre de participants}
					\end{myfigure}

					La \myref{fig-exp3-NombreDEchangeVerbauxSurLesScenariosEnFonctionDuNombreDeParticipants} présente le nombre d'échanges verbaux \myvard{5} de chaque scénario \myvari{2} en fonction du nombre de sujets \myvari{1}.
					L'analyse montre un effet significatif du nombre de sujets \myvari{1} sur le nombre d'échanges verbaux \myvard{5} de la molécule \myPrion (\myanova{exp3-talk-molecule-group-anova-prion.tex}).
					De la même façon, l'analyse montre un effet significatif du nombre de sujets \myvari{1} sur le nombre d'échanges verbaux \myvard{5} de la molécule \myUbiquitin (\myanova{exp3-talk-molecule-group-anova-ubiquitin.tex}).
				\end{mysubsubsection}
				\begin{mysubsubsection}[sss-exp3-AmeliorationDesPerformances-AnalyseEtDiscussion]{Analyse et discussion}
					Les deux tâches proposées sont de natures très différentes.
					Malgré l'apprentissage, la \myref{fig-exp3-TempsDeRealisationDesScenarios} montre que la molécule \myUbiquitin a été la plus longue à réaliser : la tâche collaborative étroitement couplée est plus complexe que la tâche faiblement couplée.
					Pourtant, les molécules n'ont pas été alternées lors de la procédure expérimentale \myref*{sse-met-exp3-Procedure} : c'est toujours la molécule \myPrion qui a été présentée en premier aux sujets.
					De plus, de nombreux groupes ont atteint la limite de \mynum[mn]{10} lors de la réalisation du scénario \myscenario{2} (\myUbiquitin).
					Nous pouvons en déduire que la collaboration étroitement couplée est plus complexe à appréhender par les sujets.

					L'étude précédente présentée dans le \myref{cha-RechercheCollaborativeDeResiduSurUneMolecule} a montré que les performances sont meilleures lorsque les ressources disponibles (outils de manipulation) sont partagés entre plusieurs utilisateurs.
					Cependant, cette étude compare deux configurations collaboratives.
					On constate d'après la \myref{fig-exp3-TempsDeRealisationDesScenariosEnFonctionDuNombreDeParticipants} que les \myglos*{glo-Quadrinome} obtiennent des performances identiques aux \myglos*{glo-Binome}, quel que soit le scénario.
					D'ailleurs, les \myglos*{glo-Binome} et les \myglos*{glo-Quadrinome} ont également effectué des fréquences de sélections similaires ce qui confirme ce résultat \myref*{fig-exp3-FréquenceDesSelectionsSurLesScenariosEnFonctionDuNombreDeParticipants}.

					Pourtant, la \myref{fig-exp3-VitesseMoyenneSurLesScenariosEnFonctionDuNombreDeParticipants} montre des différences significatives entre les \myglos*{glo-Binome} et les \myglos*{glo-Quadrinome} concernant la vitesse moyenne des \myglos*{glo-EffecteurTerminal}.
					L'étude exposée par \mycite[author]{Roethlisberger-1939} a mis en évidence le phénomène de \myglos{glo-MotivationSociale} : les utilisateurs se motivent entre eux pour réaliser la tâche.
					Ceci permet aux \myglos*{glo-Quadrinome} d'obtenir une activité intense avec peu de phases de relâchement durant la réalisation de la tâche.
					La vitesse moyenne est ainsi augmentée de manière significative chez les \myglos*{glo-Quadrinome}.

					Dans l'étude précédente, nous avons également mis en évidence la présence de \myglos*{glo-ConflitDeCoordination} chez les \myglos*{glo-Binome}.
					Ces \myglos*{glo-ConflitDeCoordination} entravent la progression de la tâche.
					Cependant, nous avions constaté que les sujets parviennent à résoudre ces conflits grâce à la communication verbale.
					Dans cette troisième expérimentation, la \myref{fig-exp3-NombreDEchangeVerbauxSurLesScenariosEnFonctionDuNombreDeParticipants} montre que le nombre d'échanges verbaux en \myglos{glo-Quadrinome} est inférieur à celui en \myglos{glo-Binome}.
					Ce résultat est surprenant étant donné que le nombre d'interactions possibles entre les sujets (et donc les \myglos*{glo-ConflitDeCoordination}) sont plus nombreux chez les \myglos*{glo-Quadrinome}.
					En effet, un \myglos{glo-ConflitDeCoordination} intervient lorsqu'au moins deux collaborateurs manipulent sur la même zone de travail.
					Les combinaisons de conflits dans un \myglos{glo-Quadrinome} sont plus nombreuses que dans un \myglos{glo-Binome}.

					Il semble que la différence entre un \myglos{glo-Binome} et un groupe restreint influe sur la manière de communiquer.
					À partir d'observations effectuées durant la phase expérimentale, nous avons pu constater que certains sujets se montraient relativement silencieux, même en situation de \myglos{glo-ConflitDeCoordination}.
					Nous verrons dans la \myref{sse-exp3-DefinitionDUnMeneur} que la présence d'un \myglos{glo-Meneur} dans un groupe perturbe la communication verbale au sein d'un groupe.
					En l'occurrence, le \myglos{glo-Meneur} a tendance à monopoliser la parole et à gérer les \myglos{glo-ConflitDeCoordination}.

					Dans cette section, nous n'avons constater aucune évolution des performances entre les \myglos*{glo-Binome} et les \myglos*{glo-Quadrinome}.
					Cependant, malgré un nombre potentiel de \myglos*{glo-ConflitDeCoordination} important et une communication verbale faible, les \myglos*{glo-Quadrinome} obtiennent des performances similaires aux \myglos*{glo-Binome}.
					L'augmentation de la vitesse moyenne, provoquée par le phénomène de \myglos{glo-MotivationSociale} déjà remarqué par \mycite[author]{Roethlisberger-1939}, permet d'expliquer ces performances.
					En effet, la \myglos{glo-MotivationSociale} permet de réduire les phases d'inaction en stimulant l'intérêt des sujets pour la tâche à réaliser.
					Afin d'améliorer les performances d'un \myglos{glo-Quadrinome}, il faudrait faciliter les communications verbales pour une gestion optimale des \myglos*{glo-ConflitDeCoordination}.
				\end{mysubsubsection}
			\end{mysubsection}
			\begin{mysubsection}[sse-exp3-UtiliteDuBrainstormingPourLaCollaboration]{Utilité du \mybrainstorming pour la collaboration}
				\begin{mysubsubsection}[sss-exp3-UtiliteDuBrainstormingPourLaCollaboration-DonneesEtTestsStatistiques]{Données et tests statistiques}
					\begin{myfigure}
						\psset{xunit=0.274914089\textwidth,yunit=0.005cm}
						\begin{myps}(-0.425,-220)(2,640)
							\myaxes(0,2){\mybrainstorming}(0,600)[100]{temps~(s)}
							\myboxplot{exp3-time-brainstorm.csv}
						\end{myps}
						\mycaption[fig-exp3-TempsDeRealisationAvecOuSansBrainstorming]{Temps de réalisation avec ou sans \mybrainstorming}
					\end{myfigure}

					La \myref{fig-exp3-TempsDeRealisationAvecOuSansBrainstorming} présente le temps de réalisation \myvard{1} en fonction des groupes avec ou sans \mybrainstorming \myvari{3}.
					L'analyse montre un effet significatif du \mybrainstorming \myvari{3} sur le temps de réalisation \myvard{1} (\myanova{exp3-time-brainstorm-anova.tex}).

					\begin{myfigure}
						\psset{xunit=0.444444444\textwidth,yunit=0.005cm}
						\begin{myps}(-0.25,-220)(2,720)
							\myaxes(0,2){\mybrainstorming}(0,600)[100]{temps~(s)}
							\myboxplot{exp3-time-brainstorm-group.csv}
							\mylegend{\myleg{\myglosnl{glo-Binome}}{myblue}\myand\myleg{\myglosnl{glo-Quadrinome}}{myblue!70}}
						\end{myps}
						\mycaption[fig-exp3-TempsDeRealisationDesScenariosEnFonctionDesGroupesAvecOuSansBrainstorming]{Temps de réalisation des scénarios en fonction des groupes avec ou sans \mybrainstorming}
					\end{myfigure}

					La \myref{fig-exp3-TempsDeRealisationDesScenariosEnFonctionDesGroupesAvecOuSansBrainstorming} présente le temps de réalisation \myvard{1} pour les groupes avec ou sans \mybrainstorming \myvari{3} en fonction du nombre de sujets \myvari{1}.
					L'analyse montre qu'il n'y a pas d'effet significatif du \mybrainstorming \myvari{3} sur le temps de réalisation \myvard{1} des \myglos*{glo-Binome} (\myanova{exp3-time-brainstorm-group-anova-binome.tex}).
					Cependant, l'analyse montre un effet significatif du \mybrainstorming \myvari{3} sur le temps de réalisation \myvard{1} des \myglos*{glo-Quadrinome} (\myanova{exp3-time-brainstorm-group-anova-quadrinome.tex}).

					\begin{myfigure}
						\psset{xunit=0.43956044\textwidth,yunit=10cm}
						\begin{myps}(-0.275,-0.11)(2,0.36)
							\myaxes(0,2){\mybrainstorming}(0,0.31)[0.05]{sélections~(nb/s)}
							\myboxplot{exp3-freqsel-brainstorm-group.csv}
							\mylegend{\myleg{\myglosnl{glo-Binome}}{myblue}\myand\myleg{\myglosnl{glo-Quadrinome}}{myblue!70}}
						\end{myps}
						\mycaption[fig-exp3-FréquenceDesSelectionsSurLesScenariosEnFonctionDesGroupesAvecOuSansBrainstorming]{Fréquence des sélections sur les scénarios en fonction des groupes avec ou sans \mybrainstorming}
					\end{myfigure}

					La \myref{fig-exp3-FréquenceDesSelectionsSurLesScenariosEnFonctionDesGroupesAvecOuSansBrainstorming} présente la fréquence de sélection \myvard{2} pour les groupes avec ou sans \mybrainstorming \myvari{3} en fonction du nombre de sujets \myvari{1}.
					L'analyse montre qu'il n'y a pas d'effet significatif du \mybrainstorming \myvari{3} sur la fréquence de sélection \myvard{2} des \myglos*{glo-Binome} (\myanova{exp3-freqsel-brainstorm-group-anova-binome.tex}).
					Cependant, l'analyse montre un effet significatif du \mybrainstorming \myvari{3} sur la fréquence de sélection \myvard{2} des \myglos*{glo-Quadrinome} (\myanova{exp3-freqsel-brainstorm-group-anova-quadrinome.tex}).

					\begin{myfigure}
						\psset{xunit=0.43956044\textwidth,yunit=3cm}
						\begin{myps}(-0.275,-0.35)(2,1.45)
							\myaxes(0,2){\mybrainstorming}(0,1.25)[0.25]{vitesse~(mm/s)}
							\myboxplot{exp3-speed-brainstorm-group.csv}
							\mylegend{\myleg{\myglosnl{glo-Binome}}{myblue}\myand\myleg{\myglosnl{glo-Quadrinome}}{myblue!70}}
						\end{myps}
						\mycaption[fig-exp3-VitesseMoyenneSurLesScenariosEnFonctionDesGroupesAvecOuSansBrainstorming]{Vitesse moyenne sur les scénarios en fonction des groupes avec ou sans \mybrainstorming}
					\end{myfigure}

					La \myref{fig-exp3-VitesseMoyenneSurLesScenariosEnFonctionDesGroupesAvecOuSansBrainstorming} présente la vitesse moyenne \myvard{3} pour les groupes avec ou sans \mybrainstorming \myvari{3} en fonction du nombre de sujets \myvari{1}.
					L'analyse montre qu'il n'y a pas d'effet significatif du \mybrainstorming \myvari{3} sur la vitesse moyenne \myvard{3} des \myglos*{glo-Binome} (\myanova{exp3-speed-brainstorm-group-anova-binome.tex}).
					De la même façon, l'analyse montre qu'il n'y a pas d'effet significatif du \mybrainstorming \myvari{3} sur la vitesse moyenne \myvard{3} des \myglos*{glo-Quadrinome} (\myanova{exp3-speed-brainstorm-group-anova-quadrinome.tex}).

					\begin{myfigure}
						\psset{xunit=0.43956044\textwidth,yunit=0.075cm}
						\begin{myps}(-0.275,-14)(2,58)
							\myaxes(0,2){\mybrainstorming}(0,50)[10]{communication~(nb)}
							\myboxplot{exp3-communication-brainstorm-group.csv}
							\mylegend{\myleg{\myglosnl{glo-Binome}}{myblue}\myand\myleg{\myglosnl{glo-Quadrinome}}{myblue!70}}
						\end{myps}
						\mycaption[fig-exp3-NombreDOrdresVerbauxSurLesScenariosEnFonctionDesGroupesAvecOuSansBrainstorming]{Nombre d'ordres verbaux sur les scénarios en fonction des groupes avec ou sans \mybrainstorming}
					\end{myfigure}

					La \myref{fig-exp3-NombreDOrdresVerbauxSurLesScenariosEnFonctionDesGroupesAvecOuSansBrainstorming} présente le nombre d'ordres verbaux \myvard{5} pour les groupes avec ou sans \mybrainstorming \myvari{3} en fonction du nombre de sujets \myvari{1}.
					L'analyse montre un effet significatif du \mybrainstorming \myvari{3} sur le nombre d'ordres verbaux \myvard{5} des \myglos*{glo-Binome} (\myanova{exp3-communication-brainstorm-group-anova-binome.tex}).
					De la même façon, l'analyse montre un effet significatif du \mybrainstorming \myvari{3} sur le nombre d'ordres verbaux \myvard{5} des \myglos*{glo-Quadrinome} (\myanova{exp3-communication-brainstorm-group-anova-quadrinome.tex}).
				\end{mysubsubsection}
				\begin{mysubsubsection}[sss-exp3-UtiliteDuBrainstormingPourLaCollaboration-AnalyseEtDiscussion]{Analyse et discussion}
					La \myref{fig-exp3-NombreDOrdresVerbauxSurLesScenariosEnFonctionDesGroupesAvecOuSansBrainstorming} nous permet de constater une baisse significative du nombre d'échanges verbaux pour les sujets ayant eu une période de \mybrainstorming.
					Le \mybrainstorming permet une réflexion préalable sur la tâche afin d'aboutir à une stratégie de travail concernant différents éléments :
					\begin{itemize}
						\item répartition et distribution du travail;
						\item organisation du travail dans l'espace;
						\item organisation du travail dans le temps;
						\item identification des rôles de chaque manipulateur;
						\item prévisions sur l'évolution de l'environnement.
					\end{itemize}

					Cependant, la \myref{fig-exp3-TempsDeRealisationDesScenariosEnFonctionDesGroupesAvecOuSansBrainstorming} et la \myref{fig-exp3-FréquenceDesSelectionsSurLesScenariosEnFonctionDesGroupesAvecOuSansBrainstorming} montrent que le \mybrainstorming est seulement bénéfique pour les \myglos*{glo-Quadrinome}.
					En effet, les \myglos*{glo-Binome} n'obtiennent aucune évolution significative des performances avec ou sans \mybrainstorming.
					De même, la \myref{fig-exp3-VitesseMoyenneSurLesScenariosEnFonctionDesGroupesAvecOuSansBrainstorming} montre que la vitesse moyenne de l'\myglos{glo-EffecteurTerminal} des \myglos*{glo-Binome} n'évolue pas.
					La configuration \myglos*{glo-Bimanuel} est certainement la raison de cette vitesse inférieure.

					Le peu d'intérêt que présente le \mybrainstorming pour les \myglos*{glo-Binome} s'explique par deux raisons.
					Le \mybrainstorming étant utilisé pour définir une stratégie de travail, il permet de réduire le nombre de \myglos*{glo-ConflitDeCoordination}.
					Le nombre de \myglos{glo-ConflitDeCoordination} pour les \myglos*{glo-Binome} étant faible par rapport à celui des \myglos*{glo-Quadrinome}, l'intérêt du \mybrainstorming est amoindri.
					De plus, nous avons vu que la gestion des \myglos*{glo-ConflitDeCoordination} s'effectue par une communication verbale.
					La communication en \myglos{glo-Binome} est relativement naturelle alors que la communication dans un groupe de plus de trois sujets s'avère plus compliquée : problème de prise de parole, conversation entre deux sujets monopolisant la parole, \myetc
					La gestion des \myglos*{glo-ConflitDeCoordination} est quasiment optimale pour les \myglos*{glo-Binome}, même sans \mybrainstorming, ce qui n'est pas le cas pour les \myglos*{glo-Quadrinome}.

					Cependant, la \myref{fig-exp3-TempsDeRealisationDesScenariosEnFonctionDesGroupesAvecOuSansBrainstorming} et la \myref{fig-exp3-FréquenceDesSelectionsSurLesScenariosEnFonctionDesGroupesAvecOuSansBrainstorming} mettent en évidence l'amélioration des performances pour les \myglos*{glo-Quadrinome}.
					Nous avons vu dans la \myref{sse-exp3-AmeliorationDesPerformances} que les \myglos*{glo-Quadrinome} éprouvent des difficultés dans la résolution des \myglos*{glo-ConflitDeCoordination}.
					D'après les figures observées, le \mybrainstorming permet l'élaboration d'une stratégie et la définition des rôles de chacun des sujets.
					L'élaboration d'une stratégie réduit de façon importante le nombre de \myglos*{glo-ConflitDeCoordination} pendant la réalisation de la tâche et ainsi améliore les performances.
					De plus, la mise en place de rôles pour chacun des sujets avant le début de la tâche permet de distribuer la tâche ou de l'organiser le cas échéant et ainsi d'éviter le phénomène de \myglos{glo-ParesseSociale} ce qui rejoint les conclusions faites par \mycite[author]{Latane-1979} sur l'identification des rôles.

					Dans le cas de la molécule \myPrion, la tâche comporte un faible niveau d'interaction entre les zones à déformer; elle peut aisément être divisée en quatre tâches élémentaires.
					D'ailleurs, l'analyse des communications verbales a montré que c'était la stratégie choisie par tous les groupes ayant effectué un \mybrainstorming.
					La molécule \myUbiquitin comportant un fort niveau d'interaction, nécessite plus de coordination mais peut être divisée en deux tâches élémentaires.
					Dans ce cas, le \mybrainstorming aboutit à une scission du groupe en deux \myglos*{glo-Binome} qui réaliseront chacun une partie de la déformation.
					Ceci permet d'avoir des gestions de \myglos*{glo-ConflitDeCoordination} locaux et restreint ainsi son effet au \myglos{glo-Binome} concerné.

					De plus, la période de \mybrainstorming permet de partitionner le temps de réflexion et le temps de manipulation.
					En effet, l'analyse des communications verbales permet de constater que les groupes n'ayant pas eu de période de \mybrainstorming (\mycondition{1} et \mycondition{3}) tentent tout de même d'élaborer une stratégie de travail.
					Cependant, la manipulation créé une charge de travail cognitive importante.
					Les capacités cognitives des sujets sont alors partagées entre la manipulation et l'élaboration d'une stratégie.
					Les sujets ne sont pas pleinement attentifs à l'élaboration de la stratégie et peuvent en plus ne pas être attentifs en même temps que leurs collègues.
					La réflexion sur la meilleure stratégie à choisir n'est donc pas optimale.

					Cette section nous a permis de confirmer l'intérêt d'un \mybrainstorming pour structurer les groupes : cette période n'est bénéfique que pour les \myglos*{glo-Quadrinome}.
					En effet, elle permet d'éviter les problèmes de \myglos{glo-ParesseSociale} évoqués par \mycite[author]{Latane-1979} en stimulant la création de rôles pour chaque manipulateur.
					Nous verrons dans la section suivante que le groupe va s'organiser autour d'un \myglos{glo-Meneur} et que les autres manipulateurs se placeront plutôt dans un rôle de \myglos*{glo-Suiveur}.
					L'émergence rapide d'un \myglos{glo-Meneur} va permettre au groupe de se structurer et d'avoir une répartition des tâches plus rapide.
				\end{mysubsubsection}
			\end{mysubsection}
			\begin{mysubsection}[sse-exp3-DefinitionDUnMeneur]{Définition d'un \myglosnl{glo-Meneur}}
				Cette section va définir les caractéristiques d'un \myglos{glo-Meneur}.
				Nous utiliserons les données d'un groupe représentatif pour alimenter notre propos.
				Cependant, étant donné le peu de données d'un seul groupe, aucune analyse de la variance ne sera présentée.
				\begin{mysubsubsection}[sss-exp3-DefinitionDUnMeneur-DonneesEtStatistiques]{Données et statistiques}
					\begin{myfigure}
						\psset{xunit=0.227272727\textwidth,yunit=0.25cm}
						\begin{myps}(-0.4,-4.5)(4,13)
							\myaxes(0,4){sujets de \mygroup{1}}(0,12)[4]{ordres~(nb)}
							\mybarplot{exp3-g1-talk-subject.csv}
						\end{myps}
						\mycaption[fig-exp3-NombreDOrdresDonnesParChacunDesSujetsDeG1]{Nombre d'ordres donnés par chacun des sujets de \mygroup{1}}
					\end{myfigure}

					La \myref{fig-exp3-NombreDOrdresDonnesParChacunDesSujetsDeG1} présente le nombre d'ordres donnés \myvard{5} en fonction des sujets du groupe \mygroup{1}.
					On observe que le sujet \mysubject{1} donne beaucoup plus d'ordres que la moyenne (\myanalysis{exp3-g1-talk-subject-analysis.tex} de plus que la moyenne).

					\begin{myfigure}
						\psset{xunit=0.222222222\textwidth,yunit=3cm}
						\begin{myps}(-0.5,-0.4)(4,1.1)
							\myaxes(0,4){sujets de \mygroup{1}}(0,1.00)[0.25]{vitesse~(mm/s)}
							\mybarplot{exp3-g1-speed-subject.csv}
						\end{myps}
						\mycaption[fig-exp3-VitesseMoyenneDesEffecteursTerminauxPourChacunDesSujetsDeG1]{Vitesse moyenne des \myglosnl*{glo-EffecteurTerminal} pour chacun des sujets de \mygroup{1}}
					\end{myfigure}

					La \myref{fig-exp3-VitesseMoyenneDesEffecteursTerminauxPourChacunDesSujetsDeG1} présente la vitesse moyenne des \myglos*{glo-EffecteurTerminal} \myvard{3} en fonction des sujets du groupe \mygroup{1}.
					On observe que le sujet \mysubject{1} donne plus d'ordres que la moyenne (\myanalysis{exp3-g1-speed-subject-analysis.tex} de plus que la moyenne).

					\begin{myfigure}
						\psset{xunit=0.007017544\textwidth,yunit=1cm}
						\begin{mysubfigure}[\textwidth]
							\begin{myps}(-12.5,-1.15)(130,5.25)
								\myaxes[labels=all,ticks=all,Dx=25](0,125){temps~{(s)}}(0,5)[1]{force~(N)}
								\psscalebox{2.75 1}{\psfileplot[linecolor=myred,linewidth=0.1pt]{files/exp3-g1-force-prion-S1.csv}}
							\end{myps}
							\mysubcaption[fig-exp3-ProfilDeForceDuGroupeG1SurLaMoleculePrion-ProfilDeForceDeS1]{Profil de force de \mysubject{1}}
						\end{mysubfigure}
						\begin{mysubfigure}[\textwidth]
							\begin{myps}(-12.5,-1.15)(130,5.25)
								\myaxes[labels=all,ticks=all,Dx=25](0,125){temps~{(s)}}(0,5)[1]{force~(N)}
								\psscalebox{2.75 1}{\psfileplot[linecolor=myred,linewidth=0.1pt]{files/exp3-g1-force-prion-S2.csv}}
								\pnode(49.341,2){begin}
								\pnode(86.291,2){end}
								\ncline[linecolor=myblue,linewidth=1pt]{<->}{begin}{end}
								\ncput*{\textcolor{myblue}{\small selection}}
							\end{myps}
							\mysubcaption[fig-exp3-ProfilDeForceDuGroupeG1SurLaMoleculePrion-ProfilDeForceDeS2]{Profil de force de \mysubject{2}}
						\end{mysubfigure}
						\mycaption[fig-exp3-ProfilDeForceDuGroupeG1SurLaMoleculePrion]{Profil de force du groupe \mygroup{1} sur la molécule \myPrion}
					\end{myfigure}

					La \myref{fig-exp3-ProfilDeForceDuGroupeG1SurLaMoleculePrion} présente les profils de force \myvard{4} des sujets \mysubject{1} et \mysubject{2} du groupe \mygroup{1}.
					Chaque période où la force est maintenue représente une sélection \myref*{fig-exp3-ProfilDeForceDuGroupeG1SurLaMoleculePrion-ProfilDeForceDeS2}.
					On constate un profil très chaotique pour le sujet \mysubject{1} avec un grand nombre de sélections (\mynum{11}~sélections).
					Par opposition, le profil du sujet \mysubject{2} est très peu chaotique avec un petit nombre de sélections (\mynum{4}~sélections $> \mynum[s]{10}$).
					De plus, les efforts maximaux produits par le sujet \mysubject{2} sont plus importants que ceux du \mysubject{1} (\mynum[N]{5} pour \mysubject{2} contre \mynum[N]{4} pour \mysubject{1}).
				\end{mysubsubsection}
				\begin{mysubsubsection}[sss-exp3-DefinitionDUnMeneur-AnalyseEtDiscussion]{Analyse et discussion}
					Le \myglos{glo-Meneur} est celui qui dirige les opérations.
					Cependant, les groupes de notre expérimentation sont des groupes sans hiérarchie : aucun chef ou meneur n'est prédéterminé.
					En effet, nos groupes sont des \myglos*{glo-StructureInformelle} dans lesquelles aucun rôle n'est prédéfini.
					Dans la précédente section, nous avons identifié l'émergence de rôles, en particulier au cours du \mybrainstorming.
					Parmi les rôles, on distingue le rôle du \myglos{glo-Meneur}, déjà identifié dans les travaux de \mycite[author]{Bales-1950}.
					Nous allons à présent définir le rôle du \myglos{glo-Meneur} ainsi que les rôles de \myglos*{glo-Suiveur}.

					La \myref{fig-exp3-VitesseMoyenneDesEffecteursTerminauxPourChacunDesSujetsDeG1} et la \myref{fig-exp3-ProfilDeForceDuGroupeG1SurLaMoleculePrion-ProfilDeForceDeS1} nous permet de déterminer la stratégie de travail du \myglos{glo-Meneur}.
					En effet, on constate un grand nombre de sélections ainsi qu'une vitesse élevée.
					Le \myglos{glo-Meneur} explore l'environnement pour prendre des décisions.
					Son exploration est constituée de sélections de courte période avec une force appliquée relativement faible.
					Ces nombreuses sélections ont pour objectif de consulter différentes zones de la molécule pour évaluer le travail restant.
					Il proposera à un autre sujet d'effectuer à sa place, les déformations qu'il aura jugé nécessaire.

					Par opposition, les \myglos*{glo-Suiveur} n'explorent pas l'environnement.
					En effet, la \myref{fig-exp3-ProfilDeForceDuGroupeG1SurLaMoleculePrion-ProfilDeForceDeS2} montre un nombre de sélections peu élevées mais des sélections maintenues sur une longue période de temps.
					Le \myglos{glo-Suiveur} accepte un ordre du \myglos{glo-Meneur} et effectue la déformation jusqu'à atteindre l'objectif fixé.
					Les manipulations des suiveurs sont plutôt lentes car les déformations nécessite de la précision dans la manipulation.
					De plus, l'effort déployé est plus important car toute l'attention du \myglos{glo-Suiveur} est portée sur la déformation.
					Le \myglos{glo-Meneur} ne déploit pas autant d'effort car il n'effectue pas les déformations.

					Pour conclure cette section, le \myglos{glo-Meneur} a un rôle crucial dans la dynamique du groupe.
					C'est lui qui va définir et répartir les tâches : il élabore la stratégie du groupe.
					Cette répartition permet à chaque sujet de se faire attribuer une tâche bien identifiée.
					L'identification des rôles est nécessaire pour obtenir de bonne performances et éviter le phénomène de \myglos{glo-ParesseSociale} \myref*{UtiliteDuBrainstormingPourLaCollaboration-exp3-AvantageDuBrainstormingPourLesQuadrinomes}.
					Cependant, le \myglos{glo-Meneur} doit être accepté par les autres membres afin d'obtenir une bonne harmonie dans le groupe.
				\end{mysubsubsection}
			\end{mysubsection}
		\end{mysection}
		\begin{mysection}[sec-exp3-Synthese]{Synthèse}
			\begin{mysubsection}[sse-exp3-ResumeDesResultats]{Résumé des résultats}
				Cette expérience a permis d'étudier et de comparer des \myglos*{glo-Binome} en configuration \myglos*{glo-Bimanuel} avec des \myglos*{glo-Quadrinome} en configuration \myglos*{glo-Monomanuel}.
				L'objectif était d'observer l'évolution des performances en fonction du nombre de participants ainsi que les nouvelles contraintes liées aux dynamiques de groupe.

				Les résultats ont montré que l'augmentation du nombre de sujets ne permettait pas systématiquement d'améliorer les performances du groupe.
				En effet, les \myglos*{glo-Quadrinome}, bien que plus rapides dans leurs mouvements grâce au phénomène de \myglos{glo-MotivationSociale}, obtiennent des performances identiques aux \myglos*{glo-Binome}.
				Les \myglos*{glo-Quadrinome} perdent du temps dans la résolution des \myglos*{glo-ConflitDeCoordination} qui sont plus nombreux que chez les \myglos*{glo-Binome}.

				Cependant, une analyse basée sur la possibilité d'établir une stratégie de travail au préalable a permis d'approfondir cette conclusion.
				Le \mybrainstorming permet une organisation préalable du groupe permettant de meilleures performances tout en réduisant le nombre de \myglos*{glo-ConflitDeCoordination}.
				L'élaboration d'une stratégie de travail est surtout bénéfique pour les \myglos*{glo-Quadrinome} qui sont confrontés à de nombreux de \myglos*{glo-ConflitDeCoordination} en temps normal.

				De plus, ce \mybrainstorming permet de faire émerger les rôles rapidement au sein de cette \myglos{glo-StructureInformelle}.
				L'émergence d'un \myglos{glo-Meneur} est nécessaire pour organiser le groupe, diviser le travail et répartir les tâches.
				D'un autre côté, les \myglos*{glo-Suiveur} acceptent la présence de ce \myglos{glo-Meneur} et l'assistent dans la réalisation de la tâche.
				Le \myglos{glo-Meneur} va se distinguer par une exploration plus large et plus rapide de l'environnement afin d'avoir une vision globale de la tâche à réaliser.
				Les \myglos*{glo-Suiveur} effectuent plutôt des déformations longues et locales.

				Cette expérimentation montre que l'augmentation du nombre d'utilisateurs est bénéfique sous réserve d'une certaine organisation.
				Un \mybrainstorming préalable à la réalisation de la tâche permet de structurer un groupe.
				De plus, cette structure est obtenue avec l'accord de tous les participants ce qui rend légitime le \myglos{glo-Meneur}.
				Dans le cas contraire, les \myglos*{glo-Suiveur} pourraient ne pas vouloir suivre les indications du \myglos{glo-Meneur} ce qui serait contre-productif.
			\end{mysubsection}
			\begin{mysubsection}[sse-exp3-Conclusion]{Conclusion}
				Nous venons de montrer l'intérêt d'avoir un groupe structuré lorsque le nombre de participants excède deux.
				Notre prochaine et dernière expérimentation aura pour objectif de tester la plate-forme avec des experts de la déformation moléculaire.
				Pour cela, nous allons leur fournir des outils haptiques permettant de faciliter le travail collaboratif.

				Pour commencer, nous avons mis en avant la nécessité de faire émerger rapidement le \myglos{glo-Meneur} et les \myglos*{glo-Suiveur}.
				Ceci permet de coordonner le groupe derrière un seul utilisateur et éviter les \myglos*{glo-ConflitDeCoordination}.

				Cependant, la manière de travailler du \myglos{glo-Meneur} et très différente de celle d'un \myglos{glo-Suiveur}.
				Des outils haptiques adaptés aux besoins de chacun seront donc proposés afin d'améliorer leur possibilités d'interactions.
				En l'occurence, le \myglos{glo-Meneur} n'effectue pas réellement de déformation, il semble donc peu nécessaire de lui donner la possibilité de le faire.
				Ainsi, on le libère d'un partie de sa charge cognitive pour le focaliser sur son rôle de \myglos{glo-Meneur}.

				En ce qui concerne le \myglos{glo-Suiveur}, il est affecté à la déformation.
				Il est particulièrement occupé à effectuer des déformations de façon locale.
				Il faut donc lui laisser la possibilité d'effectuer des déformations locales et précises.
				Cependant, il faut également lui faciliter la communication avec le \myglos{glo-Meneur} et lui rendant accessible les consignes rapidement.
				Le \myglos{glo-Meneur} ayant une vision plus globale de la tâche à réaliser, il peut être justifié de donner ponctuellement des outils de déformation plus grossier.

				La majorité de ces outils seront implémentés dans la dernière version de la plate-forme afin d'effectuer une expérimentation avec des experts de la déformation moléculaire.
				Ces outils seront évalués à la fois en terme d'amélioration sur les performances mais également en terme d'utilisabilité.
				Le \myref{cha-TravailCollaboratifAssisteParHaptique} décrit ces nouveaux outils et l'ensemble de l'expérimentation.
			\end{mysubsection}
		\end{mysection}
	\end{mychapter}
	\begin{mychapter}[cha-TravailCollaboratifAssisteParHaptique]{Travail collaboratif assisté par haptique}
		\begin{mysection}[sec-exp4-Introduction]{Introduction}
			Cette dernière expérimentation aura pour objectif d'introduire et de valider des outils de communication haptique dans le cadre d'une tâche d'\myglos{glo-AmarrageMoleculaire}.
			Sur la base des précédentes expérimentations, des outils haptiques censés améliorer les interactions et les communications entre les manipulateurs sont proposés.
			L'expérimentation testera l'intérêt et l'apport de ces outils sur la collaboration de groupe.

			Le principal facteur observé sera les performances du groupe.
			Les performances regardées seront le temps mis pour achever la tâche mais également la qualité de la solution trouvée.
			En effet, la qualité de la solution est une variable non-négligeable dans le cadre d'une tâche d'\myglos{glo-AmarrageMoleculaire}.

			Le second facteur concernera l'évaluation qualitative du système par les utilisateurs.
			Il est primordial de recueillir l'avis des utilisateurs en ce qui concerne une plate-forme de travail.
			Des outils haptiques inconfortables, des détails visuels incohérents, des interactions peu intuitives sont autant de paramètres qui peuvent rendre un système inefficace.
		\end{mysection}
		\begin{mysection}[sec-exp4-AssistanceHaptiquePourLaCommunication]{Assistance haptique pour la communication}
		\end{mysection}
		\begin{mysection}[sec-exp4-PresentationDeLExperimentation]{Présentation de l'expérimentation}
			\begin{mysubsection}[sse-exp4-DescriptionDeLaTache]{Description de la tâche}
				La tâche proposée est la déformation de molécules et de complexes de molécules dans un \myacro{acr-EVC}.
				L'objectif est de la rendre le plus conforme possible au modèle.
				Cinq molécules sont utilisés dans le cadre de cette expérimentation dont trois exclusivement réservées pour l'entraînement :
				\begin{description}
					\item[\myTRPCAGE]
						La molécule nommée \myTRPCAGE \mycite{Neidigh-2002} a pour identifiant \myPDB \myPDBlink{http://www.rcsb.org/pdb/explore/explore.do?structureId=1L2Y}{1L2Y} sur la \myPDBbase\footnote{\url{http://www.pdb.org/}}.
						Cette molécule contient \mynum{304}~atomes dont \mynum{20}~\myglos*{glo-Residu}.
					\item[\myPrion]
						La molécule nommée \myPrion \mycite{Christen-2009} avec l'identifiant \myPDB \myPDBlink{http://www.rcsb.org/pdb/explore/explore.do?structureId=2KFL}{2KFL} sur la \myPDBbase\footnotemark[\value{footnote}].
						Cette molécule contient \mynum{1779}~atomes dont \mynum{112}~\myglos*{glo-Residu}.
					\item[\myUbiquitin]
						La molécule nommée \myUbiquitin \mycite{Vijay-Kumar-1987} avec l'identifiant \myPDB \myPDBlink{http://www.rcsb.org/pdb/explore/explore.do?structureId=1UBQ}{1UBQ} sur la \myPDBbase\footnotemark[\value{footnote}].
						Cette molécule contient \mynum{1231}~atomes dont \mynum{76}~\myglos*{glo-Residu}.
					\item[\myTRPZIPPER]
						La molécule \myTRPZIPPER \mycite{Cochran-2001} a pour identifiant \myPDB \myPDBlink{http://www.rcsb.org/pdb/explore/explore.do?structureId=1LE1}{1LE1} sur la \myPDBbase\footnotemark[\value{footnote}].
						Cette molécule contient \mynum{218}~atomes dont \mynum{12}~\myglos*{glo-Residu}.
					\item[\myNusENusG]
						Le complexe de molécules \myNusENusG \mycite{Burmann-2010} a pour identifiant \myPDB \myPDBlink{http://www.rcsb.org/pdb/explore/explore.do?structureId=2KVQ}{2KVQ} sur la \myPDBbase\footnotemark[\value{footnote}].
						Il est constitué de deux molécules \textsc{NusE} et \textsc{NusG} possédant respectivement \mynum{1294}~atomes pour \mynum{80}~\myglos*{glo-Residu} et \mynum{929}~atomes pour \mynum{59}~\myglos*{glo-Residu}.
				\end{description}

				La tâche est proposée à des groupes de trois sujets : les \myglos*{glo-Trinome}.
				Dans ces \myglos*{glo-Trinome}, un \og coordinateur \fg et deux \og opérateurs \fg ont à leur disposition différents outils.
				Ils ont la possibilité de communiquer sans restriction de façon orale, gestuelle ou même virtuelle.
				\begin{mysubsubsection}[sss-exp4-DescriptionDesOutils]{Description des outils}
					Pour cette expérimentation, des modifications ont été apportés aux différents outils.
					En effet, nous souhaitons apporter une assistance haptique afin d'augmenter la communication sensorielle entre les sujets.
					Les outils modifiés sont l'outil de désignation, l'outil de déformation par atome et l'outil de déformation par molécule que nous nommerons outil de manipulation\footnote{L'outil de déformation par molécule applique une force à l'ensemble des atomes de la molécule et produit ainsi un déplacement complet de la molécule; cette opération s'apparente plus à une manipulation qu'à une déformation.}
					\begin{myparagraph}[par-exp4-OutilDeDesignation]{Outil de désignation}
						Le coordinateur est en charge d'effectuer les désignations envers les opérateurs.
						Nous souhaitons fournir au coordinateur un moyen de connaître l'état de la désignation à chaque instant.
						Une vibration est donc générée sur l'outil de désignation lorsque le coordinateur désigne une cible.
						Le coordinateur est renseigné sur l'acceptation de cette désignation par l'arrêt de cette vibration.
						De plus, tant que la cible désignée par le coordinateur n'aura pas été acceptée, le coordinateur ne pourra pas désigner une autre cible.
					\end{myparagraph}
					\begin{myparagraph}[par-exp4-OutilDeDeformation]{Outil de déformation}
						Un outil de déformation au niveau atomique est fourni aux deux opérateurs présents lors de l'expérimentation.
						Nous souhaitons donné la possibilité au coordinateur d'indiqué rapidement qu'une désignation a été effectuée.
						Lorsque le coordinateur désigne une cible, tous les outils des opérateurs sont soumis à une vibration.
						Il est à noter que si les opérateurs sont en train de déformer la molécule, ils ne ressentent pas la vibration mais dès qu'ils relâchent leur sélection, leur outil leur indique qu'une désignation est en cours.
						Les opérateurs ont la possibilité d'accepter ou non la désignation.
						À l'instant où un des deux opérateur accepte la désignation, les vibrations s'arrêtent.
						L'opérateur qui a accepté la désignation est attiré vers la cible à déformer.
						De plus, il se voit offrir le pouvoir de déformer non plus au niveau atomique mais au niveau résiduel ce qui lui donne une capacité étendue.
						L'objectif de cette étendue des capacités est de stimuler l'envie d'interagir avec le coordinateur.
					\end{myparagraph}
					\begin{myparagraph}[par-exp4-OutilDeManipulation]{Outil de manipulation}
						Un dernier outil, détenu par le coordinateur, permet la manipulation de la molécule (outil de déformation au niveau moléculaire).
						Cet outil va permettre au coordinateur de déplacer la molécule comme un bloc pour la rapprocher de sa cible finale.
						Afin d'aider le coordinateur dans cette tâche, nous avons souhaiter prendre en compte les actions des opérateurs pour assister la manipulation de la molécule.
						Ainsi, lorsque les opérateurs effectuent une déformation, une infime partie de l'effort déployé est reporté sur l'ensemble de la molécule afin de la déplacer dans cette même intention.
						Les efforts reportés sont relativement faibles pour ne pas perturber la manipulation du coordinateur mais devrait sensiblement modifier le déplacement la molécule vers une position optimale.
					\end{myparagraph}

					Parmi les outils présentés précédemment, les interfaces haptiques sont répartis de la façon suivante :
					\begin{itemize}
						\item \mynum{1}~\myOmni est l'outil de désignation destiné au coordinateur;
						\item \mynum{2}~\myOmni sont les outils de déformation destinés aux opérateurs;
						\item \mynum{1}~\myDesktop est l'outil de manipulation destiné au coordinateur.
					\end{itemize}

					Pour finir, la souris permettant de modifier l'orientation de la scène est assignée au coordinateur.
					La souris est donc le troisième périphérique destiné au coordinateur.
					Ce choix est désiré afin de limiter au maximum les manipulations à la souris.
					En effet, la manipulation à la souris perturbe complètement l'environnement virtuel et modifie la position des curseurs par rapport à la molécule.
					Il est donc inutile de déformer la molécule lorsque l'orientation de la scène est modifié.
					Cependant, bien que dans la plupart des cas, les sujets n'éprouvent pas le besoin de modifier l'orientation de la scène, certains groupes en ont fait la demande explicite.
					En plaçant la souris comme troisième outil, l'objectif est qu'elle soit utilisée seulement lorsque c'est réellement nécessaire.
				\end{mysubsubsection}
				\begin{mysubsubsection}[sss-exp4-DescriptionDeLaTache]{Description de la tâche}
					Le visuel est identique à la tâche présentée pour la troisième expérimentation \myref*{sss-exp3-Visualisation}.
					En effet, la molécule est représentée en \myCPK et \myNewRibbon.
					La molécule cible est représentée en \myNewRibbon transparent.
					\begin{description}
						\item[Scénario~\myscenario{1}]
							La première tâche à réaliser est la déformation de la molécule \myUbiquitin.
							La déformation proposée est identique à la déformation proposée dans la troisième expérimentation.
							En effet, cette tâche s'est révélée très intéressante pour stimuler une collaboration étroite durant la précédente expérimentation.
							Dans cette tâche, seuls les outils de désignation et de déformation sont proposés; la molécule \myUbiquitin possède des atomes fixes afin de ne pas dériver.
						\item[Scénario~\myscenario{2}]
							La seconde tâche consiste à reconstituer le complexe de molécules \myNusENusG.
							En effet, la molécule \myNusG est laissé complétement libre et doit être amarrer à la molécule \myNusE : c'est une tâche d'\myglos{glo-AmarrageMoleculaire} simplifié.
							On distingue deux phases dans cette tâche; il faut approcher la molécule \myNusG puis il faut affiner l'amarrage par une déformation interne de \myNusG.
							Tous les outils (désignation, déformation et manipulation) sont proposés dans ce scénario; la molécule \myNusG est attaché à l'outil de manipulation et l'ensembl des atomes de la chaîne carbonée principale de la molécule \myNusE sont fixes afin d'éviter que la molécule ne dérive.
					\end{description}

					Les molécules \myTRPCAGE et \myPrion sont utilisés pour l'entraînement des outils de désignation et de déformation.
					Pour ce propos, la tâche proposée dans la seconde expérimentation pour \myTRPCAGE et la tâche proposée dans la troisième expérimentation pour \myPrion sont reprises.

					La molécule \myTRPZIPPER est utilisée pour l'entraînement de l'outil de manipulation.
					Pour ce propos, la molécule \myTRPZIPPER a été rendue complétement libre (aucun atome n'est fixé) et elle est attaché à l'outil de manipulation.

					Afin de réaliser la tâche, différentes mesures sont disponibles en temps-réel pour les sujets.
					La première de ces mesures est le score \myacro{acr-RMSD} qui est décrit dans la \myref{sse-exp2-DescriptionDeLaTache}.
					La seconde mesure est l'énergie totale du système, valeur calculée par \myacro{acr-NAMD} et représentant une synthèse des énergies électriques et des énergies de \myname{van der Waals}.
				\end{mysubsubsection}
			\end{mysubsection}
			\begin{mysubsection}[sse-exp4-SpecificitesDuProtocoleExperimental]{Spécificités du protocole expérimental}
				Les sections suivantes décrivent l'ensemble des modification apportées à la plate-forme de base \myref*{cha-pro-DispositifExperimental} et principalement aux outils.
				La méthode expérimentale est exposée dans la \myref{sec-met-exp4-QuatriemeExperimentation}.
				Un résumé de cette méthode se trouve dans la \myref{tab-exp4-SyntheseDeLaProcedureExperimentale}.
				\begin{mysubsubsection}[sss-exp4-Materiel]{Matériel}
					Dans cet quatrième et dernière expérimentation, nous introduisons de nouveaux outils destinés à améliorer les interactions entre les membres d'un \myglos{glo-Trinome}.
					Deux sujets auront toujours à leur disposition deux outils de déformation adaptés pour le processus de désignation \myref*{sss-exp4-OutilsDeManipulation}.
					Le troisième sujet aura à sa disposition trois outils dont deux basés sur une interface haptique.
					Les deux interfaces haptiques utilisées sont un \myOmni et un \myDesktop, tous deux associés à une machine de faible puissance pour le serveur \myacro{acr-VRPN}.
					De plus, ce troisième sujet aura accès à un outil de manipulation.
					En effet, bien que les outils d'orientation soient générateurs de \myglos*{glo-ConflitDeCoordination}, il s'est avéré au cours de la troisième expérimentation que certaines situations requièrent un tel outil.
					Durant la première expérimentation, une interface haptique avait été utilisé mais n'était pas vraiment adaptée pour cette fonctionnalité.
					Au cours de la seconde expérimentation, c'est une souris~\myThreeD qui a remplacé l'interface haptique.
					Bien que cet outil soit très adapté pour cette fonctionnalité, il est relativement complexe à prendre en main et nécessite un apprentissage.
					Cet apprentissage n'est pas acceptable durant un processus expérimental d'environ une heure.
					C'est donc une simple souris \myUSB qui a été choisi comme outil de manipulation : il est un peu moins adapté pour l'orientation d'une molécule qu'une souris~\myThreeD mais est nettement plus intuitif pour la plupart des sujets qui sont habitués à manipuler ce type de périphérique.

					De la même manière que dans la troisième expérimentation, une caméra vidéo \mySony (\textsc{hdr-cx550}) a été installée derrière les sujets afin de filmer à la fois les sujets et l'écran de vidéoprojection.
					Le son est également capter par la caméra.
					Là encore, les vidéos sont exportées et séquencées \myafortiori à l'aide du logiciel \myiMovie.

					La \myref{fig-exp4-SchemaDuDispositifExperimental} illustre le dispositif expérimental par un schéma.
					La \myref{fig-exp4-PhotographieDuDispositifExperimental} est une photographie de la salle d'expérimentation.

					\begin{myfigure}
						\myimage{exp4-schema}
						\mycaption[fig-exp4-SchemaDuDispositifExperimental]{Schéma du dispositif expérimental}
					\end{myfigure}
					\begin{myTodo}{Images à compléter}{Il va falloir faire des photos du dispositif expérimental}
						\begin{myfigure}
							%\myimage{exp4-photo}
							\mycaption[fig-exp4-PhotographieDuDispositifExperimental]{Photographie du dispositif expérimental}
						\end{myfigure}
					\end{myTodo}
				\end{mysubsubsection}
				\begin{mysubsubsection}[sss-exp4-Visualisation]{Visualisation}
					Dans cette quatrième et dernière expérimentation, les molécules étant très importantes, surtout pour le complexe de molécules \myNusENusG, nous avons décidé de donner un rendu secondaire aux atomes.
					En effet, ils sont à présents rendus de manière transparente.
					Cependant, nous verrons dans la section suivante que le coordinateur aura la possibilité de colorer les \myglos*{glo-Residu} au besoin.
					Pour le reste, le rendu visuel est similaire à celui des expérimentations trois et quatre.
					La \myref{fig-exp4-RepresentationDeLaMoleculeUbiquitinPourLeScenario1} représente la molécule \myUbiquitin et la \myref{fig-exp4-RepresentationDeLaMoleculeNusENusGPourLeScenario2} représente le complexe de molécules \myNusENusG.

					\begin{myfigure}
						\myimage{exp4-scenario1}
						\mycaption[fig-exp4-RepresentationDeLaMoleculeUbiquitinPourLeScenario1]{Représentation de la molécule \myUbiquitin pour le scénario~\myscenario{1}}
					\end{myfigure}
					\begin{myfigure}
						\myimage{exp4-scenario2}
						\mycaption[fig-exp4-RepresentationDeLaMoleculeNusENusGPourLeScenario2]{Représentation de la molécule \myNusENusG pour le scénario~\myscenario{2}}
					\end{myfigure}
				\end{mysubsubsection}
				\begin{mysubsubsection}[sss-exp4-OutilsDeManipulation]{Outils de manipulation}
					Pour cette expérimentation, des modifications ont été apportés aux différents outils.
					En effet, nous souhaitons apporter une assistance haptique afin d'augmenter la communication sensorielle entre les sujets.
					Les outils modifiés sont l'outil de désignation, l'outil de déformation par atome et l'outil de déformation par molécule que nous nommerons outil de manipulation\footnote{L'outil de déformation par molécule applique une force à l'ensemble des atomes de la molécule et produit ainsi un déplacement de la molécule en bloc; cette opération s'apparente plus à une translation de la molécule qu'à une déformation.}.
					Ces outils sont décrits dans la \myref{sse-Shaddock-LesNouveauxOutilsDInteraction}.
					\begin{myparagraph}[par-exp4-OutilDeDesignation]{Outil de désignation}
						Le coordinateur est en charge d'effectuer les désignations envers les opérateurs.
						Nous souhaitons fournir au coordinateur un moyen de connaître l'état de la désignation à chaque instant.
						Une vibration est donc générée sur l'outil de désignation lorsque le coordinateur désigne une cible.
						Le coordinateur est renseigné sur l'acceptation de cette désignation par l'arrêt de cette vibration.
						De plus, tant que la cible désignée par le coordinateur n'aura pas été acceptée, le coordinateur ne pourra pas désigner une autre cible.
					\end{myparagraph}
					\begin{myparagraph}[par-exp4-OutilDeDeformation]{Outil de déformation}
						Un outil de déformation au niveau atomique est fourni aux deux opérateurs présents lors de l'expérimentation.
						Nous souhaitons donner la possibilité au coordinateur d'indiquer rapidement qu'une désignation a été effectuée.
						Lorsque le coordinateur désigne une cible, tous les outils des opérateurs sont soumis à une vibration.
						Il est à noter que si les opérateurs sont en train de déformer la molécule, ils ne ressentent pas la vibration mais dès qu'ils relâchent leur sélection, la vibration leur indique qu'une désignation est en cours.
						Les opérateurs ont la possibilité d'accepter ou non la désignation.
						À l'instant où un des deux opérateur accepte la désignation, les vibrations s'arrêtent pour tous les opérateurs.
						L'opérateur qui a accepté la désignation est alors attiré vers la cible à déformer.
						De plus, il se voit offrir le pouvoir de déformer non plus au niveau atomique mais au niveau résiduel ce qui lui donne une capacité étendue.
						L'objectif de cette augmentation des capacités est de stimuler l'envie d'interagir avec le coordinateur.
					\end{myparagraph}
					\begin{myparagraph}[par-exp4-OutilDeManipulation]{Outil de manipulation}
						Un dernier outil, détenu par le coordinateur, permet le déplacement de la molécule (outil de déformation au niveau moléculaire).
						Cet outil va permettre au coordinateur de déplacer la molécule comme un bloc pour la rapprocher de sa cible finale.
						Afin d'aider le coordinateur dans cette tâche, nous avons souhaiter prendre en compte les actions des opérateurs pour assister le déplacement de la molécule.
						Ainsi, lorsque les opérateurs effectuent une déformation, une infime partie de l'effort déployé est reporté sur l'ensemble de la molécule afin de la déplacer avec cette même intention.
						Les efforts reportés sont relativement faibles pour ne pas perturber la manipulation du coordinateur mais modifient sensiblement le déplacement de la molécule.
					\end{myparagraph}
					\begin{myparagraph}[par-exp4-OutilDOrientation]{Outil d'orientation}
						Pour finir, la souris permettant de modifier l'orientation de la scène est assignée au coordinateur.
						La souris est donc le troisième périphérique destiné au coordinateur.
						Le choix de surcharger le coordinateur est délibéré afin de limiter au maximum les manipulations à la souris.
						En effet, la modification de l'orientation de la molécule ne modifie pas la position des curseurs qui se trouvent dans un référentiel différent.
						En d'autres termes, la rotation de la molécule va perturber les opérateurs qui sont en train d'effectuer des déformations en modifiant la position des atomes.
						En mettant à disposition la souris comme troisième outil pour le coordinateur, l'objectif est qu'elle soit peu utilisée, seulement lorsque c'est réellement nécessaire.
					\end{myparagraph}
				\end{mysubsubsection}
				\begin{mytable}
					\mycaption[tab-exp4-SyntheseDeLaProcedureExperimentale]{Synthèse de la procédure expérimentale}
					\newcommand{\mytitlecolumn}[2]{%
						\multirow{#1}*{%
							\begin{minipage}{6em}%
								\raggedleft #2%
							\end{minipage}%
						}
					}
					\newlength{\expfourfirstcolumn}
					\newlength{\expfoursecondcolumn}
					\setlength{\expfourfirstcolumn}{7em}
					\setlength{\expfoursecondcolumn}{\textwidth}
					\addtolength{\expfoursecondcolumn}{-\expfourfirstcolumn}
					\addtolength{\expfoursecondcolumn}{-4\tabcolsep}
					\begin{mytabular}{>{\bfseries}p{\expfourfirstcolumn}p{\expfoursecondcolumn}}
						\mytoprule
						\mytitlecolumn{1}{Tâche}                   & Déformation d'une molécule ou d'un complexe de molécule                   \\
						\mymiddlerule[\heavyrulewidth]
						\mytitlecolumn{3}{Hypothèses}              & \myhypothesis{1} Performances améliorées par l'assistance haptique        \\
						                                           & \myhypothesis{2} L'assistance haptique améliore la communication          \\
						                                           & \myhypothesis{3} La plate-forme est appréciée des experts                 \\
						\mymiddlerule
						\mytitlecolumn{2}{Variables indépendantes} & \myvari{1} Présence de l'assistance                                       \\
						                                           & \myvari{2} Molécules à déformer                                           \\
						\mymiddlerule
						\mytitlecolumn{5}{Variables dépendantes}   & \myvard{1} Score de ressemblance minimum                                  \\
						                                           & \myvard{2} Temps du score \myacronl-{acr-RMSD}                            \\
						                                           & \myvard{3} Nombre de sélections                                           \\
						                                           & \myvard{4} Communication verbales et gestuelles                           \\
						                                           & \myvard{5} Test d'utilisabilité de la plate-forme                         \\
						\mymiddlerule[\heavyrulewidth]
						\multicolumn{2}{c}{%
							\small%
							\begin{tabular}{^C-C-C-C}
								\myrowstyle{\bfseries}
								Condition \mycondition{1} & Condition \mycondition{2} & Condition \mycondition{3} & Condition \mycondition{4} \\
								\mymiddlerule
								Sans assistance           & Avec assistance           & Sans assistance           & Avec assistance           \\
								\mymiddlerule
								\myUbiquitin              & \myUbiquitin              & \myNusENusG               & \myNusENusG               \\
							\end{tabular}
						} \\
						\mybottomrule
					\end{mytabular}
				\end{mytable}
			\end{mysubsection}
		\end{mysection}
		\begin{mysection}[sec-exp4-Resultats]{Résultats}
		\end{mysection}
		\begin{mysection}[sec-exp4-Synthese]{Synthèse}
			\begin{mysubsection}[sse-exp4-ResumeDesResultats]{Résumé des résultats}
			\end{mysubsection}
			\begin{mysubsection}[sse-exp4-Conclusion]{Conclusion}
			\end{mysubsection}
		\end{mysection}
	\end{mychapter}
	\begin{mychapter+}{Conclusion et perspectives}
	\end{mychapter+}
	\myglossary
	\myappendix
	\begin{mychapter}[cha-pro-DispositifExperimental]{Dispositif expérimental}
		\begin{mysection}[sec-pro-MaterielExperimental]{Matériel expérimental}
			Les expérimentations se basent sur l'\myacro{acr-EVC} présenté dans le \myref{cha-Shaddock-ManipulationCollaborativeDeMolecules}.
			Dans cette section, nous allons présenter le matériel utilisé et sa disposition.

			Tout d'abord, voici le matériel de base utilisé pour les différentes expérimentations :
			\begin{itemize}
				\item \mynum{2}~ordinateur quatre cœurs \myIntelCore avec \myRAM[Go]{4};
				\item \mynum{2}~interfaces haptiques \myOmni;
				\item \mynum{1}~vidéoprojecteur \myACER (\textsc{p5}~series)\footnote{Pour la première expérimentation, c'est un vidéoprojecteur \myCasioXJ qui a été utilisé.};
				\item \mynum{1}~grand écran de vidéoprojection.
			\end{itemize}

			Un premier ordinateur~\mycomputer{A} est celui d'où l'expérimentateur va commander l'ensemble de l'expérimentation.
			Cet ordinateur est destiné à l'application cliente \myacro{acr-VMD} : c'est donc cette machine qui s'occupe du calcul pour les rendus visuels.
			La seconde machine~\mycomputer{B} est dédiée au moteur de simulation \myacro{acr-NAMD} : elle communique avec la machine~\mycomputer{A} par une connexion \myTCPIP.

			L'affichage de l'environnement virtuel est assuré par un vidéoprojecteur connecté à l'ordinateur~\mycomputer{A}.
			Le vidéo projecteur est placé derrière les sujets et projette la scène virtuelle sur un grand écran de \mynum[m]{2.2} par \mynum[m]{2}.
			L'écran est placé face aux sujets et tous les sujets percoivent la même scène virtuelle.
			Afin que la communication entre les sujets soit optimales, aucune contrainte de communication ne leur est donnée et ils sont libres d'utiliser tous les moyens de communication possibles (verbaux, gestuels, virtuels\myetc).

			Les ordinateurs~\mycomputer{A} et~\mycomputer{B} sont également utilisés en tant que serveur \myacro{acr-VRPN}.
			Un \myOmni est connecté sur chacune des deux machines.
			Ces interfaces haptiques sont placées sur une table devant les sujets.
			Les sujets ont la possibilité de déplacer les interfaces haptiques (avec l'aide de l'expérimentateur) afin de s'installer confortablement et d'utiliser la main qu'ils désirent pour la manipulation du périphérique.

			Ce qui vient d'être décrit est la plate-forme de base qui est utilisée au cours des différentes expérimentations.
			Cependant, des spécificités liées aux tâches proposées durant les différentes expérimentations sont détaillées au-fur-et-à-mesure.
		\end{mysection}
		\begin{mysection}[sec-pro-PresentationsDesMolecules]{Présentation des molécules}
			Durant les différentes expérimentations, plusieurs molécules ou complexe de molécules ont été utilisées.
			À partir de ces molécules, différents scénarios ont été conçus et les difficultés sont décrites au-fur-et-à-mesure de la présentation des différentes expérimentation.
			Tout d'abord, nous présenterons la liste des molécules utilisées.
			Puis nous expliquerons le rendu visuel utilisé dans tous les expérimentations.
			\begin{mysubsection}[sse-pro-ListeDesMolecules]{Liste des molécules}
				Chaque molécule utilisée est référencée sur la \myPDBbase\footnote{\url{http://www.pdb.org/}} par un identifiant \myPDB.
				Voici la liste des molécules utilisées :
				\begin{description}
					\item[\myTRPZIPPER]
						La molécule \myTRPZIPPER \mycite{Cochran-2001} a pour identifiant \myPDB \myPDBlink{http://www.rcsb.org/pdb/explore/explore.do?structureId=1LE1}{1LE1}.
						Cette molécule contient \mynum{218}~atomes dont \mynum{12}~\myglos*{glo-Residu}.
					\item[\myTRPCAGE]
						La molécule nommée \myTRPCAGE \mycite{Neidigh-2002} a pour identifiant \myPDB \myPDBlink{http://www.rcsb.org/pdb/explore/explore.do?structureId=1L2Y}{1L2Y}.
						Cette molécule contient \mynum{304}~atomes dont \mynum{20}~\myglos*{glo-Residu}.
					\item[\myPrion]
						La molécule nommée \myPrion \mycite{Christen-2009} avec l'identifiant \myPDB \myPDBlink{http://www.rcsb.org/pdb/explore/explore.do?structureId=2KFL}{2KFL}.
						Cette molécule contient \mynum{1779}~atomes dont \mynum{112}~\myglos*{glo-Residu}.
					\item[\myUbiquitin]
						La molécule nommée \myUbiquitin \mycite{Vijay-Kumar-1987} avec l'identifiant \myPDB \myPDBlink{http://www.rcsb.org/pdb/explore/explore.do?structureId=1UBQ}{1UBQ}.
						Cette molécule contient \mynum{1231}~atomes dont \mynum{76}~\myglos*{glo-Residu}.
					\item[\myNusENusG]
						Le complexe de molécules \myNusENusG \mycite{Burmann-2010} a pour identifiant \myPDB \myPDBlink{http://www.rcsb.org/pdb/explore/explore.do?structureId=2KVQ}{2KVQ}.
						Il est constitué de deux molécules \textsc{NusE} et \textsc{NusG} possédant respectivement \mynum{1294}~atomes pour \mynum{80}~\myglos*{glo-Residu} et \mynum{929}~atomes pour \mynum{59}~\myglos*{glo-Residu}.
				\end{description}

				On notera la présence de molécule de taille relativement petite comme \myTRPZIPPER et \myTRPCAGE.
				On trouve également des molécules de taille assez importante comme \myPrion et \myUbiquitin.
				Enfin, pour la dernière expérimentation, un complexe de molécules a été utilisé avec \myNusENusG.
			\end{mysubsection}
			\begin{mysubsection}[sse-pro-RepresentationDesMolecules]{Représentation des molécules}
				La représentation des molécules est un domaine de recherche à part entière.
				En effet, la complexité et l'abondance d'informations à visualiser nécessite des rendus graphiques avancés et complémentaires.
				De plus, la quantité importante d'informations à représenter peut nécessiter une machine puissante afin de générer un rendu en temps-réel.
				Heureusement, \myacro{acr-VMD} possède un moteur de rendu graphique avancé \myref*{sss-Shaddock-LesRendusGraphiques}, aussi bien en terme de choix de rendu qu'en terme d'accélération graphique.

				Afin d'obtenir un rendu de molécule pertinent, nous avons bénéficié des conseils d'un biologiste.
				Ensuite, nous avons pu adapter les rendus de molécules en fonction de nos besoins pour les différents scénarios proposés.
				Cependant, une base commune a été utilisée.

				Tout d'abord, les atomes étant l'élément constituant de la molécule, il est nécessaire de les représenter en intégralité.
				Cependant, ils sont très nombreux et produisent rapidement une surcharge de la scène donc le choix de leur taille est primordial.
				Une première solution est de s'affranchir, partiellement, des atomes d'hydrogène.
				En effet, ces derniers ne constituent pas une information importante et peuvent être déduits à partir du reste de la structure de la molécule.
				Les atomes d'hydrogènes peuvent donc être représentés avec une taille réduite par rapport aux autres atomes.
				Le rendu \myCPK est utilisé pour effectuer un rendu des atomes \myref*{fig-pro-RepresentationDesAtomesAvecCPK}.

				\begin{myfigure}
					\myimage{protocole-render-CPK}
					\mycaption[fig-pro-RepresentationDesAtomesAvecCPK]{Représentation des atomes avec \myCPK}
				\end{myfigure}

				Cependant, la représentation de la molécule exclusivement avec les atomes et les liaisons entre les atomes ne permet pas d'appréhender la structure globale.
				En effet, on peut voir une molécule comme un long brin qui se replie sur lui-même avec des feuilles tout le long du brin.
				Il est donc pertinent de représenter cette structure principale.
				C'est la représentation \myNewRibbon qui tient ce rôle \myref*{fig-pro-RepresentationDeLaStructurePrincipaleDeLaMoleculeAvecNewRibbon}.

				\begin{myfigure}
					\myimage{protocole-render-NewRibbon}
					\mycaption[fig-pro-RepresentationDeLaStructurePrincipaleDeLaMoleculeAvecNewRibbon]{Représentation de la structure principale de la molécule avec \myNewRibbon}
				\end{myfigure}

				Pour finir, pour des raisons physiques d'interaction, certains atomes sont fixés au niveau de la simulation afin d'éviter des dérives de la molécule.
				Ces atomes sont signalés visuellement par une représentation en gris \myref*{fig-pro-RepresentationDesAtomesFixesEnGris}.

				\begin{myfigure}
					\myimage{protocole-render-fixed}
					\mycaption[fig-pro-RepresentationDesAtomesFixesEnGris]{Représentation des atomes fixés en gris}
				\end{myfigure}
			\end{mysubsection}
		\end{mysection}
		\begin{mysection}[sec-pro-OutilsDeManipulation]{Outils de manipulation}
			La plate-forme de base propose deux interfaces haptiques.
			Ces deux interfaces haptiques sont utilisées comme interfaces de déformation de la molécule : des outils \mytool{tug}.
			Pour comprendre ce que sont des outils de déformation, on peut se reporter à la \myref{sse-Shaddock-LesOutilsExistants}.
			Au cours des trois premières expérimentations, seules quelques modifications du rendu visuel associés à ces outils sont effectués.
			Cependant, la quatrième expérimentation apporte des modifications plus lourdes de cet outil que ce soit au niveau visuel ou au niveau haptique.
			On pourra se reporter aux chapitres respectifs pour plus de détails.

			De plus, un outil de manipulation et d'orientation de la molécule sera proposé sous différentes formes au cours des différentes expérimentations.
			Ce sera par l'intermédiaire d'un outil \mytool{grab} dans la première expérimentation \myref*{sss-exp1-Materiel}, par une souris~\myThreeD dans la seconde \myref*{sss-exp2-Materiel} puis par une simple souris \myUSB dans la dernière expérimentation \myref*{sss-exp4-Materiel}.
		\end{mysection}
	\end{mychapter}
	\begin{mychapter}[cha-met-MethodeExperimentale]{Méthode expérimentale}
		\begin{mysection}[sec-met-exp1-PremiereExperimentation]{Première expérimentation}
			\begin{mysubsection}[sse-met-exp1-Hypotheses]{Hypothèses}
				Nous émettons plusieurs hypothèses concernant cette première expérimentation.
				Les hypothèses concernent les performances des \myglos*{glo-Binome} ainsi que leurs stratégies de travail.
				Deuxièmement, une évaluation de la plate-forme est nécessaire.
				Des hypothèses sont formulées pour noter l'utilisabilité de la plate-forme ainsi que la sensation de collaboration des utilisateurs.
				\begin{myparagraph}[par-met-exp1-AmeliorationDesPerformancesEnBinome]{\myhypothesis{1} Amélioration des performances en \myglosnl{glo-Binome}}
					Nous émettons l'hypothèse que les performances des \myglos*{glo-Binome} seront meilleures que les performances des \myglos*{glo-Monome}.
					Les performances seront évaluées en terme de temps de réalisation de la tâche mais aussi en terme de ressources utilisées comme le nombre de sélections.
				\end{myparagraph}
				\begin{myparagraph}[par-met-exp1-StrategiesVariablesEnFonctionDesBinomes]{\myhypothesis{2} Stratégies variables en fonction des \myglosnl*{glo-Binome}}
					Nous émettons l'hypothèse que les \myglos*{glo-Binome} adopteront des stratégies de collaboration différentes en fonction des affinités des sujets et de leurs espaces de travail respectifs.
					L'identification des différentes stratégies permettra de les évaluer et de trouver la plus performante.
				\end{myparagraph}
				\begin{myparagraph}[par-met-exp1-LesSujetsPreferentLeTravailEnBinome]{\myhypothesis{3} Les sujets préfèrent le travail en \myglosnl{glo-Binome}}
					Notre troisième hypothèse est de nature qualitative et suppose que les utilisateurs auront une préférence pour le travail en \myglos{glo-Binome} comparé au travail en \myglos{glo-Monome}.
					Le travail en \myglos{glo-Binome} créé une collaboration sociale qui est préférée en général.
				\end{myparagraph}
				\begin{myparagraph}[par-met-exp1-LaPlateFormeEstApprecieeDesUtilisateurs]{\myhypothesis{4} La plate-forme est appréciée des utilisateurs}
					Notre dernière hypothèse concerne la validation de notre plate-forme en terme d'utilisabilité (intuitivité, ergonomie, \myetc).
					Elle est nécessaire pour la poursuite des études de cette thèse.
				\end{myparagraph}
			\end{mysubsection}
			\begin{mysubsection}[sse-met-exp1-Sujets]{Sujets}
				\mysummary{exp1-subjects.tex} avec une distribution d'âge de \mysummary{exp1-age.tex} ont participé à cette expérimentation.
				Ils ont tous été recrutés au sein du \myacro{acr-LIMSI} et sont chercheurs, assistants de recherche, étudiants en thèse ou stagiaires dans les domaines suivants~:
				\begin{itemize}
					\item linguistique et traitement automatique de la parole;
					\item réalité virtuelle et système immersifs;
					\item audio-acoustique.
				\end{itemize}

				Tous les sujets sont francophones.
				Aucun participant n'a de déficience visuelle (ou corrigée le cas échéant), de déficience audio ou de déficience moteur du haut du corps.
				Les sujets ne sont pas rémunérés pour l'expérimentation.

				Chaque participant est complètement naïf concernant les détails de l'expérimentation.
				Une explication détaillée de la procédure expérimentale leur est donnée au commencement de l'expérimentation.
				Cependant, l'objectif de l'expérimentation n'est pas révélé.
			\end{mysubsection}
			\begin{mysubsection}[sse-met-exp1-Variables]{Variables}
				\begin{mysubsubsection}[sss-met-exp1-VariablesIndependantes]{Variables indépendantes}
					\begin{myparagraph}[par-met-exp1-NombreDeSujets]{\myvari{1} Nombre de sujets}
						La première \myglos{glo-VariableIndependante} est une \myglos{glo-VariableIntraSujets}.
						\myvari{1} possède deux valeurs possibles : \og un sujet \fg (\mycf \myemph{\myglos{glo-Monome}}) ou \og deux sujets \fg (\mycf \myemph{\myglos{glo-Binome}}).
						\mynum{24}~\myglos*{glo-Monome} et \mynum{12}~\myglos*{glo-Binome} ont été testés.
					\end{myparagraph}
					\begin{myparagraph}[par-met-exp1-ResiduRecherche]{\myvari{2} \myGlosnl{glo-Residu} recherché}
						La seconde \myglos{glo-VariableIndependante} est une \myglos{glo-VariableIntraSujets}.
						\myvari{2} concerne les \myglos*{glo-Residu} recherchés qui sont au nombre de \mynum{10} répartis à part égale dans deux molécules \myref*{tab-exp1-ListeDesResidusRecherches}.
						Différents niveaux de complexité caractérisent chaque \myglos{glo-Residu} \myref*{tab-exp1-ParametresDeComplexiteDesResidus}.
					\end{myparagraph}
				\end{mysubsubsection}
				\begin{mysubsubsection}[sss-met-exp1-VariablesDependantes]{Variables dépendantes}
					\begin{myparagraph}[par-met-exp1-TempsDeRealisation]{\myvard{1} Temps de réalisation}
						Ce temps est le temps total pour réaliser la tâche demandée, c'est-à-dire trouver le \myglos{glo-Residu} et l'extraire de la molécule.
						Il n'y a pas de limite de temps pour réaliser la tâche.
						Ce temps est divisé en deux phases bien distinctes :
						\begin{description}
							\item[La recherche] C'est la phase pendant laquelle les sujets cherchent le \myglos{glo-Residu}.
								Cette recherche peut être visuelle en orientant et en déplaçant la molécule.
								Elle peut aussi amener les sujets à déformer la molécule afin d'explorer les \myglos{glo-Residu} inaccessibles du centre de la molécule.
							\item[La sélection] La phase de sélection débute dès l'instant où un des deux sujets a identifié visuellement le \myglos{glo-Residu}.
								Elle est constituée d'une phase de sélection puis d'une phase d'extraction hors de la molécule.
						\end{description}
					\end{myparagraph}
					\begin{myparagraph}[par-met-exp1-LaDistanceEntreLesEspacesDeTravail]{\myvard{2} La distance entre les espaces de travail}
						Cette mesure est la distance moyenne entre les deux \myglos*{glo-EffecteurTerminal} correspondant aux outils \mytool{tug}.
						Elle est mesurée dans le monde réel mais peut être convertie dans l'environnement virtuel (à l'échelle de la molécule).
						L'ordre de grandeur de cette mesure est le centimètre.
					\end{myparagraph}
					\begin{myparagraph}[par-met-exp1-CommunicationsVerbales]{\myvard{3} Communications verbales}
						L'enregistrement des communications verbales permet de mesurer la durée de parole de chaque sujets pour chaque étape de l'expérimentation.
						Ces mesures différencie la phase de recherche et la phase de sélection (voir \myvard{1}) comme indiqué plus précisément sur la \myref{fig-exp1-EtapesDeLaCommunicationVerbalePourLaRechercheDUnResidu}.

						\begin{myfigure}
							\psset{unit=0.1\textwidth} % Fill entirely the page width
							\begin{myps}(0,-1.75)(10,1.5)
								\psset{linewidth=1pt,linecolor=black}%
								\psset{fillstyle=solid}%
								\psframe[fillcolor=mylightblue](0,-0.5)(6,0.5)%
								\pspolygon[fillcolor=mylightred](6,-0.5)(6,0.5)(9,0.5)(10,0)(9,-0.5)%
								\uput{16pt}[180](10,0){\LARGE\sl\textcolor{white!33}{temps}}
								\psbrace[ref=lC,rot=-90,nodesepA=-3,nodesepB=-0.25](6,0.5)(0,0.5){%
									\parbox{6\psxunit}{%
										\centering\textcolor{myblue}{Temps de recherche}%
									}%
								}%
								\psbrace[ref=lC,rot=-90,nodesepA=-2,nodesepB=-0.25](10,0.5)(6,0.5){%
									\parbox{4\psxunit}{%
										\centering\textcolor{myred}{Temps de sélection}%
									}%
								}%
								\psframe[fillcolor=myblue](1,-0.5)(1.5,0.5)
								\psframe[fillcolor=myblue](3,-0.5)(4.5,0.5)
								\psframe[fillcolor=myblue](4.8,-0.5)(5,0.5)
								\psframe[fillcolor=myred](6.5,-0.5)(7.5,0.5)
								\psframe[fillcolor=myred](8,-0.5)(8.25,0.5)
								\pnode(1.25,-0.5){verbal1}
								\pnode(3.75,-0.5){verbal2}
								\pnode(4.9,-0.5){verbal3}
								\pnode(7,-0.5){verbal4}
								\pnode(8.125,-0.5){verbal5}
								\rput(5,-1.5){%
									\Rnode{verbal}{%
										\psframebox[linestyle=none]{\centering Communication verbale}%
									}%
								}%
								\psset{linearc=0.1,angleA=-90}
								\ncdiagg{<-}{verbal1}{verbal}
								\ncdiagg{<-}{verbal2}{verbal}
								\ncdiagg{<-}{verbal3}{verbal}
								\ncdiagg{<-}{verbal4}{verbal}
								\ncdiagg{<-}{verbal5}{verbal}
							\end{myps}
							\mycaption[fig-exp1-EtapesDeLaCommunicationVerbalePourLaRechercheDUnResidu]{Étapes de la communication verbale pour la recherche d'un \myglosnl{glo-Residu}}
						\end{myfigure}
					\end{myparagraph}
					\begin{myparagraph}[par-met-exp1-AffiniteEntreLesSujets]{\myvard{4} Affinité entre les sujets}
						Le degré d'affinité -- concernant uniquement les \myglos*{glo-Binome} -- est compris entre \mynum{1} et \mynum{5} selon les critères suivants :
						\begin{enumerate}
							\item Les sujets ne se connaissent pas;
							\item Les sujets travaillent dans la même entreprise, le même laboratoire;
							\item Les sujets travaillent dans la même équipe, sur les mêmes projets;
							\item Les sujets travaillent ensemble, sont dans le même bureau;
							\item Les sujets sont amis proches.
						\end{enumerate}
					\end{myparagraph}
					\begin{myparagraph}[par-met-exp1-ForceMoyenneAppliqueeParLesSujets]{\myvard{5} Force moyenne appliquée par les sujets}
						Le force appliquée par chaque sujet sur les atomes pendant la simulation est mesurée.
						Une valeur moyenne de cette force est calculée pour être analysée.
					\end{myparagraph}
					\begin{myparagraph}[par-met-exp1-ReponsesQualitatives]{\myvard{6} Réponses qualitatives}
						Un questionnaire est proposé à tous les sujets.
						Il est constitué de trois ou quatre parties respectivement destinés aux \myglos*{glo-Monome} et \myglos*{glo-Binome}.
						Le questionnaire fourni aux sujets est disponible dans la \myref{sec-Questionnaires-PremiereExperimentation}.
					\end{myparagraph}
				\end{mysubsubsection}
			\end{mysubsection}
			\begin{mysubsection}[sse-met-exp1-Procedure]{Procédure}
				L'expérimentation débute par une phase d'apprentissage sur la molécule \myTRPZIPPER.
				L'apprentissage est destiné à familiariser les sujets avec la plate-forme, les outils de manipulation et la tâche à réaliser.
				Cette phase dure maximum \mynum[mn]{5}.
				L'expérimentateur est disponible pour répondre aux questions des sujets.

				Lorsque l'étape d'apprentissage est terminée, nous présentons aux sujets la série de \mynum{10}~\myglos*{glo-Residu} selon la procédure suivante.
				Le premier \myglos{glo-Residu} est affiché sur l'écran \myLCD et les sujets débutent la phase de recherche.
				Lorsque le \myglos{glo-Residu} est identifié, sélectionné puis extrait hors de la molécule, l'application est arrêtée.
				Ensuite, un second \myglos{glo-Residu} est affiché, l'application est de nouveau démarrée et ainsi de suite pour les \mynum{10}~\myglos*{glo-Residu} à indentifier.
				L'enregistrement audio est démarré à la fin de l'étape d'apprentissage.

				L'ensemble des \myglos*{glo-Residu} est proposé dans un ordre aléatoire afin d'éviter un biais lié à l'apprentissage de la plate-forme et de la tâche.
				Les sujets sont tenus de trouver et extraire dix \myglos*{glo-Residu} en \myglos{glo-Monome} et dix \myglos*{glo-Residu} en \myglos{glo-Binome}.
				Toujours pour éviter un biais lié à l'apprentissage, les sujets sont soumis aux tâches en \myglos{glo-Monome} et en \myglos{glo-Binome} de façon alternée selon les trois combinaisons suivantes :
				\begin{enumerate}
					\item Le \myglos{glo-Monome} \myuser{A}, puis le \myglos{glo-Monome} \myuser{B}, puis le \myglos{glo-Binome} \myuser{AB};
					\item Le \myglos{glo-Monome} \myuser{A}, puis le \myglos{glo-Binome} \myuser{AB}, puis le \myglos{glo-Monome} \myuser{B};
					\item Le \myglos{glo-Binome} \myuser{AB}, puis le \myglos{glo-Monome} \myuser{A}, puis le \myglos{glo-Monome} \myuser{B}.
				\end{enumerate}

				Lorsque les sujets ont réalisé toutes les tâches dans les deux configurations possibles (\myglos{glo-Monome} et \myglos{glo-Binome}), un questionnaire leur est proposé.
				Chaque sujet répond au questionnaire de manière autonome, sans communiquer avec son partenaire.
			\end{mysubsection}
		\end{mysection}
		\begin{mysection}[sec-met-exp2-SecondeExperimentation]{Seconde expérimentation}
			\begin{mysubsection}[sse-met-exp2-Hypotheses]{Hypothèses}
				Les hypothèses de cette nouvelle étude sont en grande partie basée sur l'étude précédente.
				Nous souhaitons confirmer l'intérêt du travail collaboratif dans la tâche élémentaire de \myemph{manipulation}, notamment sur les tâches de complexité importante.
				De plus, cette expérimentation propose d'étudier l'apprentissage de la tâche et d'en observer l'évolution dans le cadre du travail collaboratif.
				\begin{myparagraph}[par-met-exp2-AmeliorationDesPerformancesEnBinome]{\myhypothesis{1} Amélioration des performances en \myglosnl{glo-Binome}}
					Nous émettons l'hypothèse que les performances des \myglos*{glo-Binome} seront meilleures que les performances des \myglos*{glo-Monome}.
					Cette hypothèse a pour objectif de confirmer les conclusions obtenues dans la première étude dans un contexte de \myemph{manipulation}.
					La première hypothèse est une amélioration des performances pour les \myglos*{glo-Binome} en collaboratif comparés aux \myglos*{glo-Monome} en \myglos{glo-Bimanuel}.
				\end{myparagraph}
				\begin{myparagraph}[par-met-exp2-MeilleurGainDePerformancesSurLesTachesComplexes]{\myhypothesis{2} Meilleur gain de performances sur les tâches complexes}
					Nous émettons l'hypothèse que plus la tâche est complexe et plus une configuration de travail collaboratif produira un gain significatif de performances comparé à un \myglos{glo-Monome}.
				\end{myparagraph}
				\begin{myparagraph}[par-met-exp2-LApprentissageEstPlusPerformantPourLesBinomes]{\myhypothesis{3} L'apprentissage est plus performant pour les \myglosnl*{glo-Binome}}
					Nous émettons l'hypothèse que le travail en collaboration augmente la vitesse d'apprentissage de la tâche.
					En effet, nous supposons que l'interaction entre les partenaires va stimuler l'apprentissage et permettre l'échange des connaissances.
				\end{myparagraph}
				\begin{myparagraph}[par-met-exp2-LesSujetsPreferentLeTravailEnCollaboration]{\myhypothesis{4} Les sujets préfèrent le travail en collaboration}
					Nous souhaitons évaluer auprès des utilisateurs l'intérêt vis-à-vis du travail collaboratif.
					Notre hypothèse est que les utilisateurs préfèrent le travail collaboratif.
					En effet, le contact social et la possibilité de communiquer sont des apports appréciés dans le milieu du travail.
				\end{myparagraph}
			\end{mysubsection}
			\begin{mysubsection}[sse-met-exp2-Sujets]{Sujets}
				\mysummary{exp2-subjects.tex} avec une moyenne d'âge de \mysummary{exp2-age.tex} ont participé à cette expérimentation.
				Ils ont tous été recrutés au sein du laboratoire \myacro{acr-LIMSI} et sont chercheurs ou assistants de recherche dans les domaines suivants~:
				\begin{itemize}
					\item linguistique et traitement automatique de la parole;
					\item réalité virtuelle et système immersifs;
					\item audio-acoustique.
				\end{itemize}
				Ils ont tous le français comme langue principale.
				Aucun participant n'a de déficience visuelle (ou corrigée le cas échéant) ni de déficience audio.

				Chaque participant est complètement naïf concernant les détails de l'expérimentation.
				Une explication détaillée de la procédure expérimentale leur est donnée au commencement de l'expérimentation mais en omettant l'objectif de l'étude.
			\end{mysubsection}
			\begin{mysubsection}[sec-met-exp2-Variables]{Variables}
				\begin{mysubsubsection}[sss-met-exp2-VariablesIndependantes]{Variables indépendantes}
					\begin{myparagraph}[par-met-exp2-NombreDeSujets]{\myvari{1} Nombre de sujets}
						La première \myglos{glo-VariableIndependante} est une \myglos{glo-VariableInterSujets}.
						\myvari{1} possède deux valeurs possibles : \og un sujet (\mycf \myemph{\myglos{glo-Monome}}) \fg ou \og deux sujets (\mycf \myemph{\myglos{glo-Binome}}) \fg.
						\mynum{12}~\myglos*{glo-Monome} et \mynum{12}~\myglos*{glo-Binome} sont testés.
					\end{myparagraph}
					\begin{myparagraph}[par-met-exp2-ComplexiteDeLaTache]{\myvari{2} Complexité de la tâche}
						La seconde \myglos{glo-VariableIndependante} est une \myglos{glo-VariableIntraSujets}.
						Deux tâches de déformation sont proposées sur chacune des deux molécules : une déformation au niveau inter-moléculaire et une déformation au niveau intra-moléculaire.
					\end{myparagraph}
					\begin{myparagraph}[par-met-exp2-LeNiveauDApprentissage]{\myvari{3} Le niveau d'apprentissage}
						La troisième \myglos{glo-VariableIndependante} est une \myglos{glo-VariableIntraSujets}.
						Tous les sujets sont confrontés trois fois à la même série de tâches (quatre scénarios) sur trois jours successifs afin d'observer l'effet de l'apprentissage en \myglos{glo-Monome} et en \myglos{glo-Binome}.
					\end{myparagraph}
				\end{mysubsubsection}
				\begin{mysubsubsection}[sec-met-exp2-VariablesDependantes]{Variables dépendantes}
					\begin{myparagraph}[par-met-exp2-TempsDeRealisation]{\myvard{1} Temps de réalisation}
						C'est le temps total pour réaliser la tâche demandée, c'est-à-dire manipuler et déformer la molécule afin d'atteindre la conformation finale.
						Le temps est limité à \mynum[mn]{10}.
						Au-delà de cette limite, l'application est arrêtée
					\end{myparagraph}
					\begin{myparagraph}[par-met-exp2-NombreDeSelections]{\myvard{2} Nombre de sélections}
						C'est le nombre de sélections réalisées durant chaque tâche.
						Une sélection est comptabilisée lorsqu'un atome ou un \myglos{glo-Residu} est sélectionné par chacun des deux \myglos{glo-EffecteurTerminal}.
					\end{myparagraph}
					\begin{myparagraph}[par-met-exp2-DistancePassiveEntreLesEspacesDeTravail]{\myvard{3} Distance passive entre les espaces de travail}
						C'est la distance moyenne entre les deux \myglos*{glo-EffecteurTerminal} pendant toute la durée de chaque tâche.
						Mesurée dans l'espace physique de l'utilisateur, elle est de l'ordre du centimètre.
					\end{myparagraph}
					\begin{myparagraph}[par-met-exp2-DistanceActiveEntreLesEspacesDeTravail]{\myvard{4} Distance active entre les espaces de travail}
						Basée sur le même principe que la précédente, elle mesure la distance entre les deux \myglos*{glo-EffecteurTerminal}.
						Cependant, la moyenne est calculée seulement sur les distances lorsque les deux \myglos*{glo-EffecteurTerminal} sont en phase de sélection, lorsqu'ils sont actifs.
						Les distances ne sont pas prises en compte lorsque les deux \myglos*{glo-EffecteurTerminal} sont inactifs ou que seulement un des deux est en phase de sélection.
						Cette moyenne est également de l'ordre du centimètre.
					\end{myparagraph}
					\begin{myparagraph}[par-met-exp2-VitesseMoyenne]{\myvard{5} Vitesse moyenne}
						Elle mesure la vitesse moyenne de chaque \myglos{glo-EffecteurTerminal}.
						Elle est calculée par intégration numérique des positions successives en fonction du temps.
					\end{myparagraph}
					\begin{myparagraph}[par-met-exp2-ReponsesQualitatives]{\myvard{6} Réponses qualitatives}
						Un questionnaire est proposé à tous les sujets (variable en fonction des \myglos*{glo-Monome} et des \myglos*{glo-Binome}).
						Il se décline en deux versions destinées soit aux \myglos*{glo-Monome}, soit aux \myglos*{glo-Binome}.
						Le questionnaire soumis aux sujets est exposé dans la \myref{sec-Questionnaires-SecondeExperimentation}.
					\end{myparagraph}
				\end{mysubsubsection}
			\end{mysubsection}
			\begin{mysubsection}[sse-met-exp2-Procedure]{Procédure}
				L'expérimentation débute par une étape d'entraînement avec la molécule \myPrion.
				Pendant cette phase, les outils sont introduits et expliqués un par un.
				Cette phase dure entre \mynum[mn]{5} et \mynum[mn]{10}.
				Chaque sujet a la possibilité de tester les outils et peut questionner l'expérimentateur.

				Lorsque la phase d'entraînement est terminée, les sujets sont confrontées aux scénarios \myscenario{1a} et \myscenario{1b}.
				Les scénarios sont alternés entre les groupes de sujets afin d'éviter les biais d'apprentissage.
				L'application s'arrête automatiquement lorsque le seuil \myacro{acr-RMSD} désiré est atteint.

				Dès que les scénarios \myscenario{1a} et \myscenario{1b} ont été achevés, les sujets sont confrontés aux scénarios \myscenario{2a} et \myscenario{2b} également de façon alternée.
				De la même façon, l'application s'arrête automatiquement lorsque le seuil \myacro{acr-RMSD} désiré est atteint ou lorsque \mynum[mn]{10} de déformation sont dépassées.

				Tous les sujets sont confrontés trois fois à l'ensemble des quatre scénarios avec un jour d'intervalle entre chaque confrontation.
				L'objectif de cette multiple confrontation est l'étude de l'apprentissage en configuration collaborative.
			\end{mysubsection}
		\end{mysection}
		\begin{mysection}[sec-met-exp3-TroisiemeExperimentation]{Troisième expérimentation}
			\begin{mysubsection}[sse-met-exp3-Hypotheses]{Hypothèses}
				Lors de cette nouvelle étude, nous souhaitons observer les dynamiques de groupe.
				Nos hypothèses concerneront principalement l'évolution des groupes durant la réalisation de la tâche.
				\begin{myparagraph}[par-met-exp3-AmeliorationDesPerformancesEnQuadrinome]{\myhypothesis{1} Amélioration des performances en \myglosnl{glo-Quadrinome}}
					Nous émettons l'hypothèse que les performances des \myglos*{glo-Quadrinome} seront meilleures que les performances des \myglos*{glo-Binome}.
					Cette hypothèse vient contredire les conclusions obtenue par \mycite[author]{Zajonc-1965} concernant les tâches complexes.
					Cependant, notre contexte est différent puisqu'il concerne la collaboration étroite et nous pensons que dans ce contexte, il est nécessaire d'augmenter le nombre de sujets pour améliorer les performances, même sur une tâche complexe.
				\end{myparagraph}
				\begin{myparagraph}[par-met-exp3-EmergenceDeMeneursDansLeQuadrinome]{\myhypothesis{2} Émergence de \myglos{glo-Meneur} dans le \myglos{glo-Quadrinome}}
					D'après \mycite[author]{Bales-1950}, les groupes restreints voient émerger un voire deux \myglos*{glo-Meneur}, quelque soit la taille du groupe.
					Nous émettons l'hypothèse que l'émergence d'un \myglos{glo-Meneur} aura également lieu dans notre contexte de collaboration étroitement couplée.
				\end{myparagraph}
				\begin{myparagraph}[par-met-exp3-LeBrainstormingStructureLeQuadrinome]{\myhypothesis{3} Le \mybrainstorming structure le \myglos{glo-Quadrinome}}
					Dans cette nouvelle expérimentation, nous allons étudier la mise en place d'une période de réflexion, également appelée \mybrainstorming, avant le début de la tâche.
					Nous émettons l'hypothèse que cette période de réflexion sera principalement utile pour les \myglos*{glo-Quadrinome}.
				\end{myparagraph}
			\end{mysubsection}
			\begin{mysubsection}[sse-met-exp3-Sujets]{Sujets}
				\mysummary{exp3-subjects.tex} avec une moyenne d'âge de \mysummary{exp3-age.tex} ont participé à cette expérimentation.
				Ils ont tous été recrutés au sein du laboratoire \myacro{acr-LIMSI} et sont étudiants, chercheurs ou assistants de recherche dans les domaines suivants~:
				\begin{itemize}
					\item linguistique et traitement automatique de la parole;
					\item réalité virtuelle et système immersifs;
					\item audio-acoustique.
				\end{itemize}
				Ils ont tous le français comme langue principale.
				Aucun participant n'a de déficience visuelle (ou corrigée le cas échéant) ni de déficience audio.
				Tous les participants de cette expérimentation ont été choisis car ils ont déjà une expérience sur la plate-forme : les participants connaissent déjà les outils de déformation et l'environnement virtuel.
				Ceci doit permettre d'observer les évolutions de la dynamique de groupe tout en limitant les effets de l'apprentissage.

				Chaque participant est complètement naïf concernant les détails de l'expérimentation.
				Une explication détaillée de la procédure expérimentale leur est donnée au commencement de l'expérimentation mais en omettant l'objectif de l'étude.
			\end{mysubsection}
			\begin{mysubsection}[sse-met-exp3-Variables]{Variables}
				\begin{mysubsubsection}[sss-met-exp3-VariablesIndependantes]{Variables indépendantes}
					\begin{myparagraph}[par-met-exp3-NombreDeSujets]{\myvari{1} Nombre de sujets}
						Cette \myglos{glo-VariableIndependante} est une \myglos{glo-VariableIntraSujets}.
						\myvari{1} possède deux valeurs possibles : \og deux sujet (\mycf \myemph{\myglos{glo-Binome}}) \fg ou \og quatre sujets (\mycf \myemph{\myglos{glo-Quadrinome}}) \fg.
						\mynum{8}~\myglos*{glo-Binome} et \mynum{4}~\myglos*{glo-Quadrinome} sont testés.
					\end{myparagraph}
					\begin{myparagraph}[par-met-exp3-ComplexiteDeLaTache]{\myvari{2} Complexité de la tâche}
						La seconde \myglos{glo-VariableIndependante} est une \myglos{glo-VariableIntraSujets}.
						Deux tâches de déformation sont proposées et décrites dans la \myref{sse-exp3-DescriptionDeLaTache}.
					\end{myparagraph}
					\begin{myparagraph}[par-met-exp3-TempsAllouePourLeBrainstorming]{\myvari{3} Temps alloué pour le \mybrainstorming}
						La troisième \myglos{glo-VariableIndependante} est une \myglos{glo-VariableInterSujets}.
						\myvari{3} possède deux valeurs possibles : \og pas de \mybrainstorming \fg ou \og \mynum[mn]{1} de \mybrainstorming \fg.
						Cette période de \mybrainstorming est allouée avant le début de chaque tâche et permet une réflexion préalable sur la tâche.
					\end{myparagraph}
				\end{mysubsubsection}
				\begin{mysubsubsection}[sec-met-exp3-VariablesDependantes]{Variables dépendantes}
					\begin{myparagraph}[par-met-exp3-TempsDeRealisation]{\myvard{1} Temps de réalisation}
						C'est le temps total que les sujets ont mis pour réaliser la tâche demandée, c'est-à-dire manipuler et déformer la molécule afin d'atteindre l'objectif fixé.
						Le temps est limité à \mynum[mn]{10}.
					\end{myparagraph}
					\begin{myparagraph}[par-met-exp3-FrequenceDeSelections]{\myvard{2} Fréquence des sélections}
						\myvard{2} représente la fréquence des sélections réalisées durant chaque tâche à réaliser.
						Une sélection est comptabilisée lorsqu'un atome est sélectionné par un des \myglos{glo-EffecteurTerminal}.
						Un compteur est affecté pour chacun des \myglos*{glo-EffecteurTerminal} qui lui-même est associé à un sujet.
						C'est l'information de fréquence qui est conservée puisqu'elle ne dépend pas du temps total de réalisation de la tâche.
					\end{myparagraph}
					\begin{myparagraph}[par-met-exp3-VitesseMoyenne]{\myvard{3} Vitesse moyenne}
						Cette variable est une mesure de la vitesse moyenne de chaque \myglos{glo-EffecteurTerminal}.
						Elle est calculée par intégration numérique des positions successives en fonction du temps.
					\end{myparagraph}
					\begin{myparagraph}[par-met-exp3-ForceMoyenneAppliqueeParLesSujets]{\myvard{4} Force moyenne appliquée par les sujets}
						La force appliquée sur les atomes durant la simulation par les sujets est mesurée.
						C'est la force appliquée lorsqu'un atome est sélectionné.
						La mesure conservée est la valeur moyenne sur l'ensemble de la tâche réalisée.
					\end{myparagraph}
					\begin{myparagraph}[par-met-exp3-CommunicationsVerbales]{\myvard{5} Communications verbales}
						L'enregistrement des communications verbales permet de mesurer le nombre d'interventions verbales de chacun des sujets.
						Deux catégories d'interventions sont distinguées :
						\begin{description}
							\item[Les observations] pour indiquer aux autres sujets une intention d'action ou pour informer sur l'état actuel de l'environnement;
							\item[Les ordres] sont donnés aux autres sujets afin qu'ils réalisent une action déterminée.
						\end{description}
					\end{myparagraph}
				\end{mysubsubsection}
			\end{mysubsection}
			\begin{mysubsection}[sse-met-exp3-Procedure]{Procédure}
				L'expérimentation débute par une étape d'entraînement avec la molécule \myTRPCAGE.
				Pendant cette phase, les outils sont introduits et expliqués un par un.
				Les sujets ayant déjà réalisé une expérience sur la plate-forme, cette phase est effectuée pour se remémorer l'environnement et les outils.
				Cette phase dure entre \mynum[mn]{5} et \mynum[mn]{10}.
				Chaque sujet a la possibilité de tester les outils et peut questionner l'expérimentateur.

				Lorsque la phase d'entraînement est terminée, les sujets sont confrontées au scénario \myscenario{1}.
				Puis dans un second temps, le scénario \myscenario{2} leur est proposé.
				Pour chaque scénario, l'application s'arrête automatiquement lorsque le seuil \myacro{acr-RMSD} \myref*{sse-exp2-DescriptionDeLaTache} désiré est atteint.
				L'ordre de ces deux scénarios n'est pas contre-balancé sur les différents groupes de sujets.

				Tous les sujets sont confrontés aux deux scénarios deux fois.
				Une première fois en \myglos{glo-Binome} et une seconde fois en \myglos{glo-Quadrinome}.
				L'ordre de passage en \myglos{glo-Binome} et en \myglos{glo-Quadrinome} est alterné selon les groupes afin d'éviter les biais d'apprentissage.

				L'enregistrement vidéo est démarré au début de la phase d'apprentissage pour chaque groupe.
				Il est arrêté à la fin du second scénario.
				La phase d'apprentissage est filmée pour des questions de simplicité logistique mais n'est pas utilisée dans les analyses.
			\end{mysubsection}
		\end{mysection}
		\begin{mysection}[sec-met-exp4-QuatriemeExperimentation]{Quatrième expérimentation}
			\begin{mysubsection}[sse-met-exp4-Hypotheses]{Hypothèses}
				\begin{myparagraph}[par-met-exp4-PerformancesAmelioreesParLAssitanceHaptique]{\myhypothesis{1} Performances améliorées par l'assistance haptique}
					Nous émettons l'hypothèse que les performances de groupe seront meilleures lorsque l'assistance haptique sera mise à disposition des utilisateurs.
					Les performances principalement basées sur la qualité de la solution.
					En effet, dans un cadre de déformation moléculaire avec des experts, le résultat final prend une place plus importante que le temps mis pour l'atteindre.
				\end{myparagraph}
				\begin{myparagraph}[par-met-exp4-LAssistanceHaptiqueAmelioreLaCommunication]{\myhypothesis{2} L'assistance haptique améliore la communication}
					Dans cette dernière expérimentation, nous introduisons de nouveaux outils pour aider la communication entre les utilisateurs en utilisant la modalité haptique.
					Nous émettons l'hypothèse que la communication sera améliorée grâce à ces outils.
				\end{myparagraph}
				\begin{myparagraph}[par-met-exp4-LaPlateFormeEstApprecieeDesExperts]{\myhypothesis{3} La plate-forme est appréciée des experts}
					Lors de cette expérimentation, nous effectuons une analyse de l'utilisabilité du système.
					Nous émettons l'hypothèse que cette plate-forme répondra à des critères minimum d'utilisabilité.
					Le test d'utilisabilité est basé sur l'échelle de notation proposée par \mycite[author]{Brooke-1996}.
				\end{myparagraph}
			\end{mysubsection}
			\begin{mysubsection}[sse-met-exp4-Sujets]{Sujets}
				\begin{myTodo}{Nombre de sujets}{Remplir toutes les informations statistiques concernant les sujets}
					\textcolor{myred}{<\mynum{000}~sujets (\mynum{000}~femmes et \mynum{000}~hommes)>} avec une moyenne d'âge de \textcolor{myred}{<$\mu = 00.0$ ($\sigma = 0.00$)>} ont participés à cette expérimentation.
					Ils ont été recrutés au sein \textcolor{myred}{<laboratoire>} et sont \textcolor{myred}{<statuts, métier>}.
					Ils ont tous le français comme langue principale.
					Aucun participant n'a de déficience visuelle (ou corrigée le cas échéant) ni de déficience audio.
				\end{myTodo}

				Chaque participant est complètement naïf concernant les détails de l'expérimentation.
				Une explication détaillée de la procédure expérimentale leur est donnée au commencement de l'expérimentation mais l'objectif de l'étude n'est pas révélé.
			\end{mysubsection}
			\begin{mysubsection}[sec-met-exp4-Variables]{Variables}
				\begin{mysubsubsection}[sss-met-exp4-VariablesIndependantes]{Variables indépendantes}
					\begin{myparagraph}[par-met-exp4-PresenceDeLAssistance]{\myvari{1} Présence de l'assistance}
						La première \myglos{glo-VariableIndependante} est une \myglos{glo-VariableIntraSujets}.
						\myvari{1} possède deux valeurs possibles : \og sans assistance \fg ou \og avec assistance \fg.
						L'assistance haptique est ajoutée aux différents outils de manipulation, de désignation et de déformation afin d'améliorer l'intéraction et la communication entre les sujets pendant la tâche.
					\end{myparagraph}
					\begin{myparagraph}[par-met-exp4-MoleculesADeformer]{\myvari{2} Molécules à déformer}
						La seconde \myglos{glo-VariableIndependante} est une \myglos{glo-VariableIntraSujets}.
						\myvari{2} concerne les cinq molécules ou complexes de molécules à assembler :, \og \myTRPCAGE \fg, \og \myPrion \fg, \og \myUbiquitin \fg, \og \myTRPZIPPER \fg et \og \myNusENusG \fg.
						Parmi ces molécules, seules \myUbiquitin et \myNusENusG sont utilisées pour les tâches expérimentales.
						Les autres molécules sont simplement utilisé au cours de l'entraînement sur la plate-forme.
					\end{myparagraph}
				\end{mysubsubsection}
				\begin{mysubsubsection}[sec-met-exp4-VariablesDependantes]{Variables dépendantes}
					\begin{myparagraph}[par-met-exp4-ScoreDeRessemblanceMinimum]{\myvard{1} Score de ressemblance minimum}
						Un score \myacro{acr-RMSD} est calculé en temps-réel de la même façon que dans la seconde et la troisième expérimentation.
						Le score minimum atteint est enregistré : il représente la meilleure solution trouvé au cours de la manipulation.
					\end{myparagraph}
					\begin{myparagraph}[par-met-exp4-TempsDuScoreRMSDMinimum]{\myvard{2} Temps du score \myacronl{acr-RMSD} minimum}
						Les sujets ont \mynum[mn]{8} pour réaliser le meilleur score \myacro{acr-RMSD} possible.
						Cependant, c'est le temps mis pour atteindre ce score minimum qui est enregistré.
					\end{myparagraph}
					\begin{myparagraph}[par-met-exp4-NombreDeSelections]{\myvard{3} Nombre de sélections}
						\myvard{2} représente le nombre de sélections réalisées durant chaque tâche à réaliser.
						Une sélection est comptabilisée lorsque un atome est sélectionné par un des deux \myglos{glo-EffecteurTerminal}.
						Un compteur est affecté pour chacun des \myglos*{glo-EffecteurTerminal}.
					\end{myparagraph}
					\begin{myparagraph}[par-met-exp4-CommunicationsVerbalesEtGestuelles]{\myvard{4} Communications verbales et gestuelles}
						L'enregistrement audio permet de mesurer la quantité de temps de parole pendant chaque tâche de l'expérimentation.
						De plus, la vidéo permet de mettre en relation les différentes phases de l'expérimentation (déformation, désignation, modification du point de vue de la scène, \myetc) avec la quantité de temps de parole.
					\end{myparagraph}
					\begin{myparagraph}[par-met-exp4-TestDUtilisabiliteDeLaPlateForme]{\myvard{5} Test d'utilisabilité de la plate-forme}
						Un questionnaire est proposé à tous les sujets.
						Ce questionnaire est une traduction en français du questionnaire \myacro{acr-SUS} proposé par \mycite[author]{Brooke-1996}.
						La traduction soumise aux sujets est disponible dans la \myref{sec-Questionnaires-QuatriemeExperimentation}.
						Il nous permet d'obtenir un score d'utilisabilité de la plate-forme compris entre \mynum{0} et \mynum{100}.
					\end{myparagraph}
				\end{mysubsubsection}
			\end{mysubsection}
			\begin{mysubsection}[sse-met-exp4-Procedure]{Procédure}
				La procédure expérimentale se déroule en neuf phases bien distinctes.
				\begin{myparagraph}[par-met-exp4-Phase1-RepartitionDesRoles]{Phase~\mynum{1} : répartition des rôles}
					Pour commencer, avant de pénétrer dans la salle d'expérimentation, il va être demandé aux sujets de choisir leurs rôles.
					Nous nous plaçons dans le cadre d'une structure informelle dans laquelle un des sujets sera le coordinateur et les deux autres seront les opérateurs.
					Chaque rôle est important et il est nécessaire de l'expliquer aux sujets pour qu'aucun des rôles ne soit choisi par dépit.
					Durant cette première phase, l'expérimentateur explique de façon claire et concise les deux rôles possibles.
					Puis les sujets sont amenés à se répartir les rôles entre eux.
					Une fois cette phase terminée, les sujets sont invités à pénétrer dans la salle d'expérimentation et à s'installer : le coordinateur se trouve au milieu et les opérateurs se trouvent de part et d'autre du coordinateur.
				\end{myparagraph}
				\begin{myparagraph}[par-met-exp4-Phase2-PresentationDesOutils]{Phase~\mynum{2} : présentation des outils}
					Avant de commencer cette phase, l'enregistrement vidéo est activé.
					La seconde phase est une phase d'entraînement sur la molécule \myTRPCAGE.
					Elle a pour objectif premier de présenter les outils de désignation et de déformation.
					De plus, les sujets sont amenés à se familiariser avec l'interface, la tâche à effectuer, les différentes informations disponibles ainsi que le moyen d'évaluation.
					La tâche peut être recommencée autant de fois que nécessaire pour un apprentissage correct des outils de désignation et de déformation.
					L'enregistrement vidéo est mis en marche au début de cette phase.
				\end{myparagraph}
				\begin{myparagraph}[par-met-exp4-Phase3-IntroductionDeLHaptique]{Phase~\mynum{3} : introduction de l'haptique}
					Cette troisième phase est également une phase d'entraînement sur la molécule \myPrion.
					L'entraînement porte sur l'introduction des assistances haptiques (présentées dans la \myref{sss-exp4-DescriptionDesOutils}) sur les outils de désignation et de déformation.
					De plus, cette seconde molécule d'entraînement permet de familiariser les sujets avec une molécule de taille importante.
					La tâche peut être recommencée autant de fois que nécessaire pour une bonne compréhension des assistances haptiques proposées.
				\end{myparagraph}
				\begin{myparagraph}[par-met-exp4-Phase4-OutilDeManipulation]{Phase~\mynum{4} : outil de manipulation}
					Cette nouvelle phase d'entraînement sur la molécule \myTRPZIPPER est destinée à introduire l'outil de manipulation.
					La tâche peut être recommencée autant de fois que nécessaire afin que le coordinateur assimile correctement ce nouvel outil.
				\end{myparagraph}
				\begin{myparagraph}[par-met-exp4-Phase5-PremiereEtapeDEvaluation]{Phase~\mynum{5} : première étape d'évaluation}
					Cette première étape d'évaluation concerne les deux scénarios à réaliser (scénario~\myscenario{1} et scénario~\myscenario{2}) sur la molécule \myUbiquitin et le complexe de molécules \myNusENusG.
					L'évaluation s'effectue en deux étapes, avec et sans assistance haptique.
					En fonction des groupes et afin de contrebalancer la variable \myvari{1}, la première étape d'évaluation s'effectue avec ou sans haptique.

					On présente le scénario~\myscenario{1} puis le scénario~\myscenario{2} toujours dans cet ordre.
					Pour le scénario~\myscenario{1}, seuls les outils de désignation et de déformation sont présents.
					Tous les outils disponibles sont proposés pour le scénario~\myscenario{2}.

					Au début de chaque scénario, une période de \mynum[mn]{1} de \mybrainstorming est laissée aux sujets pendant laquelle ils peuvent visualiser et explorer la molécule non soumise à la simulation.
					Ensuite, la phase de déformation est proposée.
					L'objectif est d'atteindre le score \myacro{acr-RMSD} le plus petit possible dans un temps limité à \mynum[mn]{8}.
					Les sujets peuvent décider de s'arrêter avant les \mynum[mn]{8} s'ils estiment ne pas pouvoir obtenir un meilleur score.
				\end{myparagraph}
				\begin{myparagraph}[par-met-exp4-Phase6-PremierePartieDuQuestionnaire]{Phase~\mynum{6} : première partie du questionnaire}
					Lorsque la première étape d'évaluation est terminée, une première partie du questionnaire est proposée aux sujets \myref*{sec-Questionnaires-QuatriemeExperimentation}.
					La section à remplir dépend du premier passage : avec ou sans assistance haptique.
					Durant cette phase, il est demandé aux sujets de ne pas communiquer entre eux.
				\end{myparagraph}
				\begin{myparagraph}[par-met-exp4-Phase7-DeuxiemeEtapeDEvaluation]{Phase~\mynum{7} : Deuxième étape d'évaluation}
					La deuxième étape d'évaluation est identique à la première excepté pour la variable \myvari{1}.
					Si les sujets ont été confrontés à une assistance haptique dans la première étape, alors la seconde étape s'effectuera sans assistance haptique et réciproquement.

					Durant cette deuxième étape, il n'y a pas de phase exploratoire étant donné que les sujets connaissent déjà la molécule.
					Seules les deux phases de déformations de \mynum[mn]{8}, \myUbiquitin puis \myNusENusG, sont proposées.
				\end{myparagraph}
				\begin{myparagraph}[par-met-exp4-Phase8-LeQuestionnaire]{Phase~\mynum{8} : deuxième partie du questionnaire}
					La seconde partie du questionnaire est complémentaire à la première \myref*{sec-Questionnaires-QuatriemeExperimentation}.
					Les mêmes questions sont abordées mais pour cette deuxième étape de l'évaluation donc avec une condition différente concernant l'assistance haptique.
					Durant cette phase, il est demandé aux sujets de ne pas communiquer entre eux.
				\end{myparagraph}
				\begin{myparagraph}[par-met-exp4-Phase9-LeQuestionnaire]{Phase~\mynum{9} : questionnaire d'utilisabilité}
					Pour terminer l'expérimentation, les sujets sont invités à remplir un questionnaire d'utilisabilité \myref*{sec-Questionnaires-QuatriemeExperimentation}.
					Durant cette phase, il est demandé aux sujets de ne pas communiquer entre eux.
					Des informations concernant les caractéristiques du sujet sont également demandée à la fin du questionnaire.
					L'enregistrement vidéo est arrêté à la fin de cette phase.
				\end{myparagraph}
			\end{mysubsection}
		\end{mysection}
	\end{mychapter}
	\begin{mychapter}[cha-Questionnaires]{Questionnaires}
		\begin{mysection}[sec-Questionnaires-PremiereExperimentation]{Première expérimentation}
			Le questionnaire proposé durant cette expérimentation est constitué de deux parties.
			La deuxième partie est exclusivement réservée aux \myglos*{glo-Binome} et n'était pas proposée au \myglos*{glo-Monome}.
			Ce questionnaire contient \mynum[pages]{5} (\mynum[pages]{3} pour les \myglos*{glo-Monome}).
			Les questions sont évaluées selon une échelle de \mycite[author]{Likert-1932} à cinq niveaux.
			\myinsertpdf[pdfpages={1-5}]{exp1-questionnary}
		\end{mysection}
		\begin{mysection}[sec-Questionnaires-SecondeExperimentation]{Seconde expérimentation}
			Le questionnaire proposé durant la seconde expérimentation est décliné en deux versions : une version pour les \myglos*{glo-Monome} et une version pour les \myglos*{glo-Binome}.
			Le questionnaire est soumis aux sujets oralement par l'expérimentateur et les réponses sont directement reportées dans une tableau.
			Il est constitué de plusieurs questions notées sur échelle de \mycite[author]{Likert-1932} à cinq niveaux.
			\begin{mysubsection}[sse-Questionnaires-SecondeExperimentation-QuestionnairePourLesMonomes]{Questionnaire pour les \myglosnl*{glo-Monome}}
				Pour les \myglos*{glo-Monome}, le questionnaire est le suivant :
				\begin{enumerate}
					\item Vous êtes-vous senti efficace ?
					\item Pensez-vous que vous auriez été plus à l'aise seul avec un seul outil de déformation ?
					\item Pensez-vous que vous auriez été plus à l'aise avec un partenaire ?
					\item Quelle solution choisiriez-vous entre les trois configurations ?
				\end{enumerate}
			\end{mysubsection}
			\begin{mysubsection}[sse-Questionnaires-SecondeExperimentation-QuestionnairePourLesBinomes]{Questionnaire pour les \myglosnl*{glo-Binome}}
				Chaque sujet dans un \myglos{glo-Binome} est interrogé séparement pour éviter que les réponses de l'un influence les réponses de l'autre.
				Pour les \myglos*{glo-Binome}, le questionnaire est le suivant :
				\begin{enumerate}
					\item Vous êtes-vous senti efficace ?
					\item Comment évalueriez-vous votre taux de communication\dots{}
						\begin{itemize}
							\item verbale ?
							\item gestuelle ?
							\item virtuelle ?
						\end{itemize}
					\item Vous sentez-vous utile dans le groupe (par opposition à pénalisant) ?
					\item Pensez-vous avoir une position de \myglos{glo-Meneur} dans le groupe ?
					\item Pensez-vous que vous auriez été plus à l'aise seul avec votre outil de déformation ?
					\item Pensez-vous que vous auriez été plus à l'aise seul avec deux outils de déformation ?
					\item Quelle solution choisiriez-vous entre les trois configurations ?
				\end{enumerate}

				Concernant les taux de communication, les communications verbales concernent tous les échanges, dialogues exposés par la voix.
				La communication gestuelle représente les gestes que les sujets peuvent effectuer dans le monde réel pour expliquer, désigner ou pour tout autre explication à son partenaire.
				Enfin, la communication virtuelle concerne les informations données au partenaire par l'intermédiaire de l'environnement virtuel (par exemple, une désignation avec le curseur).
			\end{mysubsection}
		\end{mysection}
		\begin{mysection}[sec-Questionnaires-QuatriemeExperimentation]{Quatrième expérimentation}
			Le questionnaire proposé durant la quatrième et dernière expérimentation est une traduction en français du questionnaire \myacro{acr-SUS} proposé par \mycite[author]{Brooke-1996}.
			Il est soumis sous un format papier et chaque utilisateur est invité à y répondre seul, sans l'aide de ces partenaires.
			Il est constitué de plusieurs questions notées sur échelle de \mycite[author]{Likert-1932} à cinq niveaux.
			\begin{mysubsection}[sse-Questionnaires-QuatriemeExperimentation-LeQuestionnaireSUS]{Le questionnaire \myacronl-{acr-SUS}}
				Le questionnaire \myacro{acr-SUS} est constitué de \mynum{10}~questions.
				Chaque question donne lieu à une réponse sur une échelle de \mycite[author]{Likert-1932} à cinq niveaux allant de \og Fortement en désaccord (score de \mynum{1}) \fg à \og Fortement en accord (score de \mynum{5}) \fg.
				Les questions sont les suivantes :
				\begin{enumerate}[label={Q\arabic*.},ref={Q\arabic*}]
					\item Je pense que j'utiliserai ce système fréquemment
					\item J'ai trouvé ce système inutilement complexe
					\item J'ai pensé que ce système était facile à utiliser
					\item Je pense qu'il me faudrait l'aide d'un technicien pour être capable d'utiliser ce système
					\item J'ai trouvé que les différentes fonctions de la plate-forme étaient bien intégrées
					\item J'ai trouvé qu'il y avait trop d'incohérences dans cette plate-forme
					\item Je pense que la plupart des gens apprendraient rapidement à utiliser cette plate-forme
					\item J'ai trouvé le système très lourd à utiliser
					\item Je me sentais très confiant en utilisant cette plate-forme
					\item J'aurai besoin d'apprendre beaucoup de choses avant de pouvoir utiliser cette plate-forme
				\end{enumerate}
			\end{mysubsection}
			\begin{mysubsection}[sse-Questionnaires-QuatriemeExperimentation-EvaluationDuScoreSUS]{Évaluation du score \myacronl-{acr-SUS}}
				Pour évaluer le score \myacro-{acr-SUS} à partir du questionnaire, il faut des score entre \mynum{0} et \mynum{4} pour chacune des questions.
				Concernant les questions \mynum{1}, \mynum{3}, \mynum{5}, \mynum{7} et \mynum{9}, on prend le score compris en \mynum{1} et \mynum{5} auquel on enlève \mynum{1}.
				Concernant les questions \mynum{2}, \mynum{4}, \mynum{6}, \mynum{8} et \mynum{10}, on soustrait de \mynum{5} le score compris en \mynum{1} et \mynum{5}.
				Pour terminer, on multiplie par \mynum{2.5} la somme de l'ensemble des scores.
				Le score final obtenu est une note comprise entre \mynum{0} et \mynum{100}.
				\begin{mysubsubsection}[sss-Questionnaires-QuatriemeExperimentation-ExempleDeScoreSUS]{Exemple de score \myacronl-{acr-SUS}}
					Imaginons un questionnaire rempli de la façon suivante :
					\newcommand{\mySUSplus}[1]{%
						\fpRegSet{mySUSplus}{1}
						\fpRegSet{myscore}{#1}%
						\fpRegSub{myscore}{mySUSplus}%
						\fpRegRound{myscore}{0}%
						\fpRegGet{myscore}{\myscore}%
						réponse \textcolor{mygreen}{\mynum{#1}} $\Rightarrow$ score $\textcolor{mygreen}{\mynum{#1}} - 1 = \textcolor{myred}{\mynum{\myscore}}$%
					}
					\newcommand{\mySUSminus}[1]{%
						\fpRegSet{mylevel}{5}
						\fpRegSet{myscore}{#1}
						\fpRegSub{mylevel}{myscore}%
						\fpRegRound{mylevel}{0}%
						\fpRegGet{mylevel}{\myscore}%
						réponse \textcolor{mygreen}{\mynum{#1}} $\Rightarrow$ score $5 - \textcolor{mygreen}{\mynum{#1}} = \textcolor{myred}{\mynum{\myscore}}$%
					}
					\begin{enumerate}[label={Q\arabic*.},ref={Q\arabic*}]
						\item \mySUSplus{5}
						\item \mySUSminus{4}
						\item \mySUSplus{2}
						\item \mySUSminus{1}
						\item \mySUSplus{2}
						\item \mySUSminus{3}
						\item \mySUSplus{2}
						\item \mySUSminus{4}
						\item \mySUSplus{5}
						\item \mySUSminus{2}
					\end{enumerate}
					Le score total peut maintenant être calculé.
					\begin{displaymath}
						\left(\textcolor{myred}{\mynum{4}} + \textcolor{myred}{\mynum{1}} + \textcolor{myred}{\mynum{1}} + \textcolor{myred}{\mynum{4}} + \textcolor{myred}{\mynum{1}} + \textcolor{myred}{\mynum{2}} + \textcolor{myred}{\mynum{1}} + \textcolor{myred}{\mynum{1}} + \textcolor{myred}{\mynum{4}} + \textcolor{myred}{\mynum{3}}\right) \times \mynum{2.5} = \mynum{22} \times \mynum{2.5} = \textcolor{myred}{\mathbf{\mynum{55}}}
					\end{displaymath}
				\end{mysubsubsection}
			\end{mysubsection}
		\end{mysection}
	\end{mychapter}
\end{document}
