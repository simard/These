\documentclass[myfrancais,ngerman,english,french]{mythesis}
\usepackage{mydate}
\usepackage{mytodo}
\usepackage{mycolor}
\usepackage{myps}
\usepackage{myuml}
\usepackage{mymacro}
\usepackage{mypdf}
\usepackage{mysource}

\makeatletter
% Modify the bibliography style
\newcounter{mymaxcitenames}
\AtBeginDocument{%
	\setcounter{mymaxcitenames}{\value{maxnames}}%
}
\renewbibmacro{begentry}{%
	\printtext[brackets]{%
		\defcounter{maxnames}{\value{mymaxcitenames}}%
		\printnames{labelname}~\usebibmacro{cite:labelyear+extrayear}%
	}%
	\newline%
}
% AAA
\newcommand{\myACER}{\textsc{acer}\xspace}
\newcommand{\myAlanine}{Alanine\xspace}
\newcommand{\myanalysis}[1]{\input{files/#1}\%}
\newcommand{\myangstrom}{\AA ngström\xspace}
\newcommand{\myanova}[1]{\input{files/#1}}
\newcommand{\myatom}[2][]{%
	{%
		\ifstrempty{#1}%
			{\makefirstuc{\textsf{#2}}}%
			{\textcolor{#1}{\makefirstuc{\textsf{#2}}}}%
		\xspace%
	}%
}
\newcommand{\myAudacity}{\textsc{audacity}\myregistered}% No '\xspace' because of already one in '\myregistered'
% CCC
\newcommand{\mycarbon}{\myatom[mycarboncolor]{C}}
\newcommand{\myCasioXJ}{\textsc{Casio xj}\xspace}
\newcommand{\myCHARMM}{\textsc{charmm}\xspace}
\newcommand{\myChimera}{\textsc{chimera}\xspace}
\newcommand{\myClayWorks}{\textsc{Clayworks}\xspace}
\newcommand{\mycondition}[1]{$\left(\mathcal{C}_{#1}\right)$\xspace}
\newcommand{\myCPK}{\textsc{cpk}\xspace}
% DDD
\newcommand{\myDesktop}{\myPHANToM Desktop\myregistered}% No '\xspace' because of already one in '\myregistered'
% FFF
\newcommand{\myfeuillet}{feuillet-$\beta$\xspace}
\WithSuffix\newcommand\myfeuillet*{feuillets-$\beta$\xspace}
\newcommand{\myform}[1]{\textbf{\sffamily\MakeUppercase{#1}}}
% GGG
\newcommand{\myGhost}{\textsc{Ghost}\xspace}
\newcommand{\myGromacs}{\textsc{Gromacs}\xspace}
\newcommand{\mygroup}[1]{$\left(\mathcal{G}_{#1}\right)$\xspace}
% HHH
\newcommand{\myHaption}{\textsc{Haption}\xspace}
\newcommand{\myHawthorne}{\myemph{Hawthorne Works}\xspace}
\newcommand{\myHBonds}{\textit{HBonds}\xspace}
\newcommand{\myhelice}{hélice-$\alpha$\xspace}
\WithSuffix\newcommand\myhelice*{hélices-$\alpha$\xspace}
\newcommand{\myhypothesis}[1]{$\left(\mathcal{H}_{#1}\right)$\xspace}
% III
\newcommand{\myIntelCore}{Intel\myregistered Core\mytrademark~2 \textsc{q9450} (\mynum[GHz]{2.66})\xspace}
% JJJ
\newcommand{\myJmol}{\textsc{Jmol}\xspace}
% LLL
\newcommand{\myLCD}{\textsc{lcd}\xspace}
\newcommand{\myLicorice}{\textit{Licorice}\xspace}
\newcommand{\myLinux}{\textsc{Linux}\xspace}
% MMM
\newcommand{\myMacOS}{Mac~\textsc{OS}\xspace}
\newcommand{\myMDDriver}{\textsc{MDDriver}\xspace}
% NNN
\newcommand{\myNewRibbon}{\textit{NewRibbon}\xspace}
\def\mynode{%
	\@ifnextchar[{\mynode@i}{\mynode@i[style=nodestyle]}%
}
\def\mynode@i[#1](#2,#3)[#4]#5{%
	\rput(#2,#3){\Rnode{#4}{\psframebox[style=nodestyle,#1]{\vphantom{pÉ}#5}}}%
}
\newcommand{\mynytrogen}{\myatom[mynytrogencolor]{A}}
\newcommand{\myNusE}{\textsc{NusE}\xspace}
\newcommand{\myNusENusG}{\textsc{NusE:NusG}\xspace}
\newcommand{\myNusG}{\textsc{NusG}\xspace}
% OOO
\newcommand{\myOmni}{\myPHANToM Omni\myregistered}% No '\xspace' because of already one in '\myregistered'
\newcommand{\myOpenHaptics}{\textsc{OpenHaptics}\mytrademark}% No '\xspace' because of already one in '\mytrademark'
\newcommand{\myoxygen}{\myatom[myoxygencolor]{O}}
% PPP
\newcommand{\myPC}{\textsc{pc}\xspace}
\newcommand{\myPDB}{\textsc{pdb}\xspace}
\newcommand{\myPDBbase}{\emph{Protein~DataBase}\xspace}
\newcommand{\myPDBlink}[2]{\href{#1}{\textsc{\MakeLowercase{#2}}}}
\newcommand{\myPHANToM}{\textsc{phant}o\textsc{m}\xspace}
\newcommand{\myPremium}{\myPHANToM Premium\myregistered}% No '\xspace' because of already one in '\myregistered'
\newcommand{\myPrion}{Prion\xspace}
\newcommand{\myPSF}{\textsc{psf}\xspace}
\newcommand{\mypvalue}{$p$-value\xspace}
\newcommand{\myPyMOL}{\textsc{p}y\textsc{mol}\xspace}
% RRR
\newcommand{\myRAM}[2][Go]{\mynum[#1]{#2} de \textsc{ram}}
\newcommand{\myRasmol}{\textsc{RasMol}\xspace}
\newcommand{\myresidue}[1]{$\left(\mathcal{R}_{#1}\right)$\xspace}
% SSS
\newcommand{\myscenario}[1]{\textsc{#1}}
\newcommand{\mySensAble}{\textsc{SensAble}\xspace}
\newcommand{\myShaddock}{\textsc{Shaddock}\xspace}
\newcommand{\mySony}{\textsc{sony}\myregistered}% No '\xspace' because of already one in '\myregistered'
\newcommand{\mySpaceNavigator}{SpaceNavigator\myregistered}% No '\xspace' because of already one in '\myregistered'
\newcommand{\mysubject}[1]{$\mathcal{S}_#1$}
\newcommand{\mysulfur}{\myatom[mysulfurcolor]{S}}
\newcommand{\mysummary}[1]{\input{files/#1}}
% TTT
\newcommand{\myTCPIP}{\textsc{tcp/ip}\xspace}
\newcommand{\myThreeD}{\textsc{3d}\xspace}
\newcommand{\mytool}[1]{\myemph{#1}}
\newcommand{\myTRPCAGE}{\textsc{trp-cage}\xspace}
\newcommand{\myTRPZIPPER}{\textsc{trp-zipper}\xspace}
% UUU
\newcommand{\myUbiquitin}{Ubiquitin\xspace}
\newcommand{\myUbuntu}{\textsc{Ubuntu}~v$10.04$\xspace}
\newcommand{\myUSB}{\textsc{usb}\xspace}
\newcommand{\myuser}[1]{$\mathcal{#1}$}
% VVV
\newcommand{\myvar}[2]{$\left(\mathcal{V}_{\mathrm{#1}#2}\right)$\xspace}
\newcommand{\myvard}[1]{\myvar{d}{#1}}
\newcommand{\myvari}[1]{\myvar{i}{#1}}
\newcommand{\myVGA}{\textsc{vga}\xspace}
\newcommand{\myVirtuose}{\textsc{Virtuose}\mytrademark~\textsc{6d}\mynum{35}--\mynum{45}\xspace}
% WWW
\newcommand{\myWindows}{\textsc{Windows}\xspace}

% Needed lengths
\newlength{\mywidth}
\newlength{\myheight}

% PSTricks style
\newpsstyle{nodestyle}{framearc=0.25,shadow=true,shadowcolor=myblue,blur=true}
\makeatother

\NeedsTeXFormat{LaTeX2e}[1999/01/01]
\ProvidesPackage{mycolor}[2011/04/29]

%%%%%%%%%%%%%%%%%%%%%%%%%%%%%
%% Declare package options %%
%%%%%%%%%%%%%%%%%%%%%%%%%%%%%
% In case of unknown options
\DeclareOption*{%
	\PackageWarning{mycolor}{Unknown option `\CurrentOption'}%
}

\ProcessOptions

%% Options to pass to packages

%% Packages to call
\RequirePackage{xcolor}

%%%%%%%%%%%%%%%%%%%%%%%
%% New configuration %%
%%%%%%%%%%%%%%%%%%%%%%%
\definecolor{mydarkred}{rgb}{0.5265625 0.25 0.25}
\definecolor{myred}{rgb}{0.7265625 0 0}
\definecolor{mylightred}{rgb}{0.9265625 0.5 0.5}
\definecolor{mylightestred}{rgb}{1 0.66 0.66}
\definecolor{mydarkblue}{rgb}{0 0 0.2265625}
\definecolor{myblue}{rgb}{0 0 0.7265625}
\definecolor{mylightblue}{rgb}{0.500 0.500 0.9265625}
\definecolor{mylightestblue}{rgb}{0.7500 0.7500 0.9265625}
\definecolor{mygreen}{rgb}{0 0.7265625 0}
\definecolor{mylightgreen}{rgb}{0.500 0.9265625 0.500}
\definecolor{mylightestgreen}{rgb}{0.7500 0.9265625 0.7500}
\definecolor{mygray}{gray}{0.6666667}

%%%%%%%%%%%%%%%%%%
%% New commands %%
%%%%%%%%%%%%%%%%%%

% End of package
\endinput

% AAA
\mynewacro{acr-AFM}%
{%
	name={\textsc{afm}},
	first={microscope à force atomique (\textsc{afm} pour \myemph{Atomic Force Microscope})},%
	plural={\textsc{afm}s},%
	firstplural={microscopes à force atomique (\textsc{afm} pour \myemph{Atomic Force Microscope})},%
	description={Microscope permettant l'observation de la topologie de la surface d'un échantillon au niveau atomique}
}
\mynewacro{acr-API}%
{%
	name={\textsc{api}},%
	first={interface de programmation (\textsc{api})},%
	plural={\textsc{api}s},%
	firstplural={interfaces de programmation (\textsc{api}s)},%
	description={\textsc{api} vient de l'anglais \myemph{Application Programming Interface} et désigne une interface avec un programme informatique}%
}
% BBB
% Be careful, because this word has no plural form, but the femina word in plural form
\mynewglos{glo-Bimanuel}%
{%
	name={bimanuel},%
	description={Qui se fait avec les deux mains},%
	plural={bimanuelle}%
}
\mynewglos{glo-Binome}%
{%
	name={binôme},%
	description={Groupe constitué de \mynum{2}~personnes},%
	plural={binômes}%
}
% CCC
\mynewacro{acr-CAO}%
{%
	name={\textsc{cao}},%
	first={conception assistée par ordinateur (\textsc{cao})},%
	description={La \textsc{cao} permet de concevoir et de tester virtuellement, à l'aide d'outils informatique, des produits manufacturés}%
}
\mynewglos{glo-ConflitDeCoordination}%
{%
	name={conflit de coordination},%
	description={Conflit entre deux sujets qui peut survenir lorsque les deux sujets tente d'accéder ou de déformer un objet au même instant},%
	plural={conflits de coordination}%
}
\mynewacro{acr-CUDA}%
{%
	name={\textsc{cuda}},%
	first={\textsc{cuda} (\myemph{Compute Unified Device Architecture})},%
	description={Technologie permettant d'utiliser l'unité graphique d'un ordinateur pour effectuer des calculs à hautes performances}%
}
\mynewglos{glo-Curseur}%
{%
	name={curseur},%
	description={Élément virtuel associé à un élément physique que le sujet manipule; il est lié à l'\myglos{glo-EffecteurTerminal}},%
	plural={curseurs}%
}
% DDD
\mynewacro{acr-DDL}%
{%
	name={\textsc{ddl}},%
	first={degré de liberté (\textsc{ddl})},%
	plural={\textsc{ddl}s},%
	firstplural={degrés de liberté (\textsc{ddl}s)},%
	description={Mouvements relatifs indépendants d'un solide par rapport à un autre}%
}
\mynewglos{glo-DockingMoleculaire}%
{%
	name={\myemph{docking} moléculaire},%
	description={Méthode permettant de déterminer l'orientation et la déformation optimale de \mynum{2}~molécules afin qu'elle s'assemble pour former un complexe de molécules stable},%
	plural={\myemph{docking} moléculaires}%
}
% EEE
\mynewglos{glo-EffecteurTerminal}%
{%
	name={effecteur terminal},%
	description={Élément physique que le sujet manipule; il est lié au \myglos{glo-Curseur} du monde virtuel},%
	plural={effecteurs terminaux}%
}
\mynewacro{acr-EVC}%
{%
	name={\textsc{evc}},%
	first={Environnement Virtuel Collaboratif (\textsc{evc})},%
	firstplural={Environnements Virtuels Collaboratifs (\textsc{evc})},%
	description={Ensemble logiciel et matériel permettant de faire interagir plusieurs utilisateurs au sein d'un même environnement; ils jouent un rôle important dans le développement de nouvelles méthodes de travail collaboratives}%
}
% HHH
\mynewglos{glo-Homoscedasticite}%
{%
	name={homoscedasticité},%
	description={Équivalent à homogénéité des variances; permet de comparer des variables aléatoires possédant des variances similaires},%
	plural={homoscedasticités}%
}
% III
\mynewacro{acr-IBPC}%
{%
	name={\textsc{ibpc}},%
	first={Institut de Biologie Physico-Chimie (\textsc{ibpc})},%
	description={Institut de recherche, géré par la fédération de recherche \textsc{frc}~\mynum{550}, étudiant les bases structurales, génétiques et physico-chimiques à leur différents niveaux d'intégration}%
}
\mynewacro{acr-IMD}%
{%
	name={\textsc{imd}},%
	first={\textsc{imd} (\myemph{Interactive Molecular Dynamics})},%
	description={Programme permettant de connecter le logiciel de visualisation moléculaire \myacro-{acr-VMD} avec le logiciel de simulation \myacro-{acr-NAMD} pour une simulation interactive en temps-réel \mycite{Stadler-1997}}%
}
\mynewacro{acr-ITAP}%
{%
	name={\textsc{itap}},%
	first={\myemph{Institut für Theoretische und Angewandte Physik} (\textsc{itap})},%
	description={Institut de Physique Théorique et Appliquée de \myname{Stuttgart} à l'origine du développement du logiciel \myacro{acr-IMD}}%
}
% LLL
\mynewacro{acr-LIMSI}%
{%
	name={\textsc{cnrs--limsi}},%
	first={Laboratoire pour l'Informatique, la Mécanique et les Sciences de l'Ingénieur (\textsc{cnrs--limsi})},%
	description={Unité Propre de Recherche du \textsc{cnrs} (\textsc{upr}~3251) associé aux universités \textsc{Paris} Sud et Pierre et Marie \textsc{Curie}}%
}
% MMM
\mynewglos{glo-Meneur}%
{%
	name={meneur},%
	description={En anglais \myemph{leader}, personne qui dirige un groupe afin d'atteindre des objectifs communs à ce groupe; c'est celui qui prend les décisions (voir aussi \myglos{glo-Suiveur})},%
	plural={meneurs}%
}
\mynewglos{glo-Monomanuel}%
{%
	name={monomanuel},%
	description={Qui se fait avec une main},%
	plural={monomanuelle}%
}
\mynewglos{glo-Monome}%
{%
	name={monôme},%
	description={\myemph{Groupe} constitué d'une unique personne},%
	plural={monômes}%
}
\mynewglos{glo-MotivationSociale}%
{%
	name={motivation sociale},%
	description={En anglais \myemph{social facilitation} \mycite{Triplett-1900}, phénomène de groupe où les personnes fournissent plus d'efforts grâce à la présence de partenaires},%
	plural={motivation sociale}%
}
\mynewacro{acr-TRM}%
{%
	name={\textsc{trm}},
	first={Théorie des Ressources Multiples (\textsc{trm})},%
	description={Cette théorie, élaborée par \mycite[author]{Wickens-1984} (\textsc{mrt} pour \myemph{Multiple Resource Theory}), propose un modèle pour la gestion des charges de travail pour un humain}
}
% NNN
\mynewacro{acr-NAMD}%
{%
	name={\textsc{namd}},%
	first={\textsc{namd} (\myemph{Scalable Molecular Dynamics})},%
	description={Programme de simulation pour la dynamique moléculaire \mycite{Phillips-2005}}%
}
% PPP
\mynewglos{glo-ParesseSociale}%
{%
	name={paresse sociale},%
	description={En anglais \myemph{social loafing} \mycite{Ringelmann-1913}, phénomène de groupe où les personnes fournissent moins d'effort pour la réalisation d'une tâche que s'ils effectuaient la tâche seuls},%
	plural={paresse sociale}%
}
\mynewacro{acr-PCV}%
{%
	name={\textsc{pcv}},
	first={Primitive Comportementale Virtuelle (\textsc{pcv})},%
	plural={\textsc{pcv}s},%
	firstplural={Primitives Comportementales Virtuelles (\textsc{pcv}s)},%
	description={Dans une application de réalité virtuelle, les activités d'un sujet peuvent toujours être décomposées en quatre comportements de base, appelés \myacro+{acr-PCV}, qui sont : observer, se déplacer, agir et communiquer \mycite{Fuchs-2006}}
}
% QQQ
\mynewglos{glo-Quadrinome}%
{%
	name={quadrinôme},%
	description={Groupe constitué de \mynum{4}~personnes},%
	plural={quadrinômes}%
}
% RRR
\mynewglos{glo-Residu}%
{%
	name={résidu},%
	description={Groupe d'atomes constituant un des blocs élémentaires d'une molécule},%
	plural={résidus}%
}
\mynewacro{acr-RMSD}%
{%
	name={\textsc{rmsd}},%
	first={\myemph{Root Mean Square Deviation} (\textsc{rmsd})},%
	description={Appelé Écart Quadratique Moyen en français, il permet -- dans le cadre de la biologie moléculaire -- de mesurer la différence entre deux déformations d'une même molécule}%
}
% SSS
\mynewglos{glo-StructureInformelle}%
{%
	name={structure informelle},%
	description={Groupe de personnes sans structures ni hiérarchie},%
	plural={structures informelles}%
}
\mynewglos{glo-Suiveur}%
{%
	name={suiveur},%
	description={En anglais \myemph{follower}, personne qui se laisse diriger dans un groupe afin d'atteindre des objectifs communs à ce groupe; c'est une personne qui ne prend pas de décision (voir aussi \myglos{glo-Meneur})},%
	plural={suiveurs}%
}
\mynewacro{acr-SUS}%
{%
	name={\textsc{sus}},%
	first={\textsc{sus} (\myemph{System Usability Scale})},%
	description={Échelle de notation entre \mynum{0} et \mynum{100} proposée par \mycite[author]{Brooke-1996} permettant d'évaluer l'utilisabilité d'un système}%
}
% TTT
\mynewglos{glo-Tetranome}%
{%
	name={tetranôme},%
	description={Groupe constitué de \mynum{4}~personnes},%
	plural={tetranômes}%
}
\mynewglos{glo-Trinome}%
{%
	name={trinôme},%
	description={Groupe constitué de \mynum{3}~personnes},%
	plural={trinômes}%
}
% UUU
\mynewacro{acr-UDP}%
{%
	name={\textsc{udp}},
	first={\textsc{udp} (\myemph{User Datagram Protocol} pour protocole de datagramme utilisateur)},%
	plural={\textsc{afm}s},%
	firstplural={\textsc{udp} (\myemph{User Datagram Protocol} pour protocole de datagramme utilisateur)},%
	description={c'est un des principaux protocole de télécommunication sur internet ; il a pour distinction de ne pas vérifier l'intégrité des données transmises}
}
\mynewacro{acr-UML}%
{%
	name={\textsc{uml}},%
	first={\textsc{uml} (\myemph{Unified Modeling Language})},%
	description={C'est un langage graphique de modélisation utilisé principalement en génie logiciel}%
}
% VVV
\mynewglos{glo-VariableDependante}%
{%
	name={variable dépendante},%
	description={Facteur mesuré sur une expérimentation (nombre de sélections, trajectoire, \myetc); ces variables sont influencées par les \myglos*{glo-VariableIndependante}},%
	plural={variables dépendantes}%
}
\mynewglos{glo-VariableIndependante}%
{%
	name={variable indépendante},%
	description={Facteur pouvant varier et être manipuler sur une expérimentation (nombre de participants, tâche, \myetc); ces variables vont avoir une incidence sur les \myglos*{glo-VariableDependante}},%
	plural={variables indépendantes}%
}
\mynewglos{glo-VariableInterSujets}%
{%
	name={variable inter-sujets},%
	description={Variables pour lesquelles les sujets sont confrontés à une et une seule des modalités de la variable},%
	plural={variables inter-sujets}%
}
\mynewglos{glo-VariableIntraSujets}%
{%
	name={variable intra-sujets},%
	description={Variables pour lesquelles les sujets sont confrontés à toutes les modalités de la variable},%
	plural={variables intra-sujets}%
}
\mynewacro{acr-VMD}%
{%
	name={\textsc{vmd}},%
	first={\textsc{vmd} (\myemph{Visual Molecular Dynamics})},%
	description={Programme de visualisation moléculaire \mycite{Humphrey-1996}}%
}
\mynewacro{acr-VRPN}%
{%
	name={\textsc{vrpn}},%
	first={\textsc{vrpn} (\myemph{Virtual Reality Protocol Network})},%
	description={Logiciel permettant de connecter différents périphériques de réalité virtuelle à une même application sous forme d'une architecture client/serveur \mycite{Taylor-II-2001}}%
}


\hypersetup{%
	pdftitle={Collaboration haptique étroitement couplée pour la déformation moléculaire interactive},%
	pdfauthor={Jean SIMARD},%
	pdfkeywords={collaboration,haptique,environnement virtuel,simulation moléculaire},%
	pdflang={FR-fr},%
	pdfsubject={Mémoire de thèse en informatique}%
}
\addglobalbib[datatype=bibtex]{biblio.bib}

\title{Collaboration haptique étroitement couplée pour la déformation moléculaire interactive}
\author{Jean~\myname{Simard}}
\university{École Doctorale d'Informatique de \myname{Paris}-Sud (\myEDIPS)}
\laboratory{Laboratoire d'Informatique, de Mécanique et de Sciences de l'Ingénieur (\myCNRSLIMSI)}
\grade{Docteur en informatique de l'Université \myname{Paris}-Sud}
\date{\mydate[datestyle=long]{01/02/2012}}
\date{\myemph{<à définir>}}
\jury{%
	\jurymember{Professeur}{Indira}{Thouvenin}{rapporteur}{Université de technologie de \myname{Compiègne} -- Heuristique et Diagnostic des Systèmes Complexes (\myHeuDiaSyC)}
	\jurymember{Directeur de recherche}{Jean-Marie}{Burkhardt}{rapporteur}{Institut Français des Sciences et Technologies des Transports, de l'Aménagement et des Réseaux (\myIFSTTAR)}
	\jurymember{Professeur}{Abdulmotaleb}[el]{Saddik}{examinateur}{Université d'\myname{Ottawa} -- Multimedia Communications Research Laboratory (\myMCRLab)}
	\jurymember{Chargé de recherche}{Marc}{Baaden}{examinateur}{Laboratoire de Biochimie Théorique (\myLBT)}
	%\jurymember{Maître de Conférence}{Paul}{Richard}{examinateur}{Université d'\myname{Angers} -- Laboratoire d'Ingénierie des Systèmes Automatisés (\myLISA)}
	%\jurymember{Professeur}{Michel}{Beaudouin-Lafon}{examinateur}{Université de \myname{Paris}-Sud -- Pôle Commun de Recherche en Informatique (\myPCRI)}
	\jurymember{Professeur}{Philippe}{Tarroux}{directeur}{École Normale Supérieure d'\myname{Ulm} -- Laboratoire d'Informatique, de Mécanique et de Sciences de l'Ingénieur (\myCNRSLIMSI)}
	\jurymember{Maître de Conférence}{Mehdi}{Ammi}{encadrant}{Université de \myname{Paris}-Sud -- Laboratoire d'Informatique, de Mécanique et de Sciences de l'Ingénieur (\myCNRSLIMSI)}
}

\begin{document}
	\frontmatter
	\maketitle
	\mytoc
	\mylof
	\mylot
	\begin{myabstract}[french]
		Le \myglos{glo-DockingMoleculaire} consiste à comprendre comment deux molécules s'assemblent pour former un complexe de molécules.
		C'est un problème complexe pour lequel les algorithmes existants permettent seulement d'obtenir des solutions approximatives.

		Récemment, des systèmes collaboratifs ont émergé pour tenter de résoudre ce problème de manière robuste.
		Dans ce mémoire, nous allons plus loin dans la collaboration avec un environnement de manipulation moléculaire synchrone et colocalisé.
		L'objectif est de favoriser la communication pour profiter de la distribution des charges de travail et du partage de connaissances.

		Dans cette optique, la plateforme \myShaddock a été développée à l'aide de modules de visualisation et de simulation moléculaire largement utilisés par les biologistes.
		Elle immerge les utilisateurs dans une simulation moléculaire en temps-réel et permet l'interaction avec des interfaces haptiques.

		Dans un premier temps, nous étudions différentes configurations de travail collaboratif étroitement couplé à travers trois expérimentations.
		L'objectif est d'identifier les contraintes d'une telle configuration de travail et de caractériser les stratégies adoptées par les utilisateurs.
		Il en ressort que la communication, génératrice de conflits, est un point essentiel de la collaboration.

		Dans un second temps, des outils basés sur le retour haptique ont été développés pour améliorer la communication.
		Une dernière expérimentation, faisant intervenir des biologistes, montre la pertinence de la communication haptique pour améliorer la collaboration.

		Ce travail de thèse propose de nouvelles méthodes de travail pour le \myglos{glo-DockingMoleculaire}.
		De plus, nous montrons la pertinence de la modalité haptique dans l'amélioration de la communication entre utilisateurs.
		\begin{mykeywords}
			\mykeyword \myglosnl{glo-DockingMoleculaire}
			\mykeyword communication haptique
			\mykeyword collaboration étroite
			\mykeyword interaction haptique
		\end{mykeywords}
	\end{myabstract}
	\begin{myabstract}[english]
		Molecular docking tries to predicts the structure when binding two or more molecules to form a stable complex.
		It is a complex problem for which the existing algorithms only make it possible to obtain approximate solutions.

		Recently, collaborative systems have emerged to address this problem more robust.
		In this memoir, we will go further in the collaboration with an environment of synchronous and colocated molecular manipulation.
		The aim is to foster communication to take advantage of the workloads distribution and knowledge sharing.

		Accordingly, the Shaddock platform was developed by using molecular simulation and visualization modules, which are largely used by the biologists.
		It immerses users in a real time molecular simulation and allows the interaction with haptic interfaces.

		First of all, we study different closely-coupled collaborative configurations through three experiments.
		The objective is to identify the constraints of such a configuration and then characterize the strategies adopted by users.
		It shows that communication, source of conflicts, takes an essential place in the collaboration.

		At the second step, tools based on haptic feedback have been developed to improve the communication.
		The last experiment, involving biologists, shows the relevance of haptic communication to enhance collaboration.

		This thesis proposes new working methods for molecular docking.
		Moreover, we can find the relevance of the haptic modality in the improvement of users communication.
		\begin{mykeywords}
			\mykeyword molecular docking
			\mykeyword haptic communication
			\mykeyword closely collaboration
			\mykeyword haptic interaction
		\end{mykeywords}
	\end{myabstract}
	\mainmatter
	\begin{mychapter+}{Introduction générale}
		En biologie moléculaire, il est un problème vieux de plus d'un siècle qui n'a toujours pas trouvé de solution optimale.
		Ce problème, le \myglos{glo-DockingMoleculaire}, consiste à comprendre comment des molécules peuvent s'assembler afin d'en former une nouvelle.
		L'avènement de l'informatique a permis à de nombreux chercheurs de proposer des algorithmes toujours plus évolués les uns que les autres mais aucun ne permet d'obtenir une solution robuste et fiable dans un temps raisonnable.

		Afin de donner un nouveau souffle à ce domaine de recherche, l'aptitude des humains à décider de manière intelligente a été introduit dans le processus de recherche.
		Cependant, un biologiste seul peut difficilement appréhender un problème d'une telle complexité.
		C'est pourquoi, des systèmes collaboratifs permettant le partage de connaissances ont commencé à émerger.

		Parallèlement à ces nouvelles approches, les biologistes souhaitent avoir la possibilité de manipuler les molécules.
		Les contraintes physiques et techniques d'une telle manipulation ont amenés certains d'entre eux à s'orienter vers une modélisation puis une numérisation du comportement des molécules.
		Des moteurs de simulation sont alors développés pour permettre d'explorer virtuellement les comportements de ces molécules.

		Puis, afin de pouvoir interagir avec ces simulations, certains biologistes se sont orientés vers les technologies utilisées dans la réalité virtuelle, notamment les interfaces haptiques.
		Ces interfaces permettent de percevoir et d'interagir avec les forces d'interactions entre les structures moléculaires.

		C'est dans ce contexte que les travaux de cette thèse prennent place.
		Nous proposons aux biologistes de manipuler et d'interagir de manière collaborative et en temps-réel avec des structures moléculaires à l'aide d'interfaces haptiques.
		Nous souhaitons identifier les contraintes liées à cette collaboration étroitement couplée (synchrone et colocalisée) puis nous proposerons des solutions exploitant le canal haptique pour améliorer la communication et la coordination entre les partenaires.

		Le mémoire qui va suivre s'organise de la manière suivante.
		Dans un premier temps, nous présenterons de manière précise dans le \myref{cha-sota-EtudeBibliographique}, le contexte dans lequel s'inscrit notre travail.
		Nous commencerons par y présenter le \myglos{glo-DockingMoleculaire} et la complexité que représente ce problème.
		Puis nous nous intéresserons à la psychologie sociale du travail collaboratif en soulignant les avantages et les limites d'une telle approche.
		Enfin, nous reviendrons sur les particularités de la collaboration en environnement virtuel.

		Dans le \myref{cha-Shaddock-ManipulationCollaborativeDeMolecules}, nous allons présenter la plateforme \myShaddock qui a été développée dans le cadre de cette thèse pour répondre aux problématiques de collaboration et de manipulation moléculaire en temps-réel.
		Nous justifierons les choix architecturaux et les modules utilisés.
		La dernière section sera consacrée à la présentation des outils haptiques développés pour répondre aux différentes contraintes de manipulation.

		Sur la base de la plateforme \myShaddock, nous présentons ensuite trois études de cas afin d'étudier la collaboration étroitement couplée.
		Le \myref{cha-RechercheCollaborativeDeResiduSurUneMolecule} s'intéresse aux processus d'exploration et de sélection collaborative.
		Puis, le \myref{cha-DeformationCollaborativeDeMolecule} propose de comparer différentes répartitions des ressources entre les utilisateurs sur une tâche de déformation.
		Ces deux études ne proposant des manipulations que de manière individuelle ou en couple, le \myref{cha-LaDynamiqueDeGroupe} abordera les contraintes liées à la déformation de molécules sur des groupes de quatre utilisateurs.

		Sur la base des résultats de ces trois études de cas, nous proposons, dans le \myref{cha-TravailCollaboratifAssisteParHaptique}, des outils haptiques permettant d'améliorer le processus de collaboration.
		Ces outils seront testés, entre autre par des biologistes, afin d'évaluer leur pertinence et leur utilité dans ce contexte de déformation moléculaire.

		Pour finir, nous présenterons la synthèse des travaux effectués dans le cadre de cette thèse.
		Puis nous proposerons des perspectives à plus ou moins long terme concernant les poursuites possibles de ce travail.
	\end{mychapter+}
	\begin{mychapter}[cha-sota-EtudeBibliographique]{Étude bibliographique}
		\begin{mysection}[sec-sota-Introduction]{Introduction}
			Ce travail de thèse aborde de nombreux concepts, de la biologie moléculaire au système interactifs temps-réel en passant par la psychologie sociale.
			Il est nécessaire de fixer le contexte dans lequel se place nos travaux : le \myglos{glo-DockingMoleculaire}.
			Ce dernier offre un contexte de travail extrêmement complexe, idéal pour justifier une approche par le travail collaboratif.

			Cette étude bibliographique permettra dans un premier temps de comprendre les contraintes de complexité qui s'appliquent au \myglos{glo-DockingMoleculaire} ainsi que les solutions actuellement proposées.
			Dans un second temps, nous verrons les apports de la collaboration pour la résolution de problèmes complexes en soulignant les avantages et les inconvénients d'une telle approche.
			Pour terminer, nous nous intéresserons plus précisément à la collaboration au sein d'un environnement virtuel, la manière dont elle s'organise et les concepts essentiels à la mise en place d'une plateforme.
		\end{mysection}
		\begin{mysection}[sec-sota-ContexteDeTravail-LaModelisationMoleculaire]{Contexte de travail : la modélisation moléculaire}
			Dans le cadre de ce travail de thèse, nous allons nous intéresser à la modélisation moléculaire et plus précisément au \myglos{glo-DockingMoleculaire}\footnote{Pour la suite des développements, l'expression \og \myglos{glo-DockingMoleculaire} \fg sera utilisée plutôt que sa traduction \og amarrage moléculaire \fg \mycite{Nurisso-2010} qui est peu citée dans la littérature française.}.
			Cette section a pour objectif de présenter le \myglos{glo-DockingMoleculaire} puis expose les différentes solutions existantes pour traiter ce problème.
			\begin{mysubsection}[sse-sota-LeDockingMoleculaire]{Le \myglosnl{glo-DockingMoleculaire}}
				Dans le domaine de la modélisation moléculaire, le \myglos{glo-DockingMoleculaire} consiste à prédire la conformation optimale entre deux molécules afin de créer un complexe de molécules stable \myref*{fig-sota-ComplexeDeMoleculesAssembleAPartirDeDeuxMolecules}.
				Le \myglos{glo-DockingMoleculaire} permet soit de découvrir de nouvelles molécules (par assemblage de deux ou plusieurs molécules), soit de comprendre la nature d'un complexe de molécules obtenu par cristallographie\footnote{La cristallographie permet de déterminer les molécules présentes dans un complexe de molécules mais ne permet pas de déterminer avec précisions comment elles sont assemblées.}.
				\mycite[author]{Fischer-1894} illustrent le \myglos{glo-DockingMoleculaire} avec le modèle \og clef-serrure \fg décrit de la façon suivante.
				\begin{myquote}[ngerman]
					\it Um ein Bild zu gebrauchen, will ich sagen, da\ss{} Enzym und Glycosid wie Schlo\ss{} und Schlüssel zueinander passen müssen, um eine chemische Wirkung aufeinander ausüben zu können.
				\end{myquote}
				On trouve une traduction en français de cette citation dans \mycite{Hasenknopf-2005}.
				\begin{myquote}[french]
					Pour utiliser une image, je dirais que l'enzyme et le glucoside doivent être ajustés comme la serrure et la clef, pour exercer une action chimique l'un sur l'autre.
				\end{myquote}

				\begin{myfigure}
					\psset{xunit=0.066666667\textwidth,yunit=0.05\textwidth}
					\def\mycircleletter(#1,#2)#3{%
						\rput(#1,#2){\pscirclebox*[fillcolor=myblue!70]{\white #3}}%
					}
					\begin{myps}(-7.5,-9.5)(7.5,7.5)
						\rput(-5,4){\myimage[width=0.25\textwidth]{sota-docking-molecule-A}}
						\mycircleletter(-5,0){A}
						\rput(0,4){\Huge\bfseries +}
						\rput(5,4){\myimage[width=0.207437276\textwidth]{sota-docking-molecule-B}}
						\mycircleletter(5,0){B}
						\rput(0,-5){\myimage[width=0.386200717\textwidth]{sota-docking-complex-AB}}
						\psline[linewidth=10pt,linecolor=myblue!70]{->}(0,1)(0,-2)
						\mycircleletter(-1,-9){A}
						\rput(0,-9){\Large\bfseries +}
						\mycircleletter(1,-9){B}
					\end{myps}
					\mycaption[fig-sota-ComplexeDeMoleculesAssembleAPartirDeDeuxMolecules]{Complexe de molécules assemblé à partir de deux molécules}
				\end{myfigure}

				La métaphore s'arrête ici.
				En effet, si pour vérifier la concordance d'une clef avec une serrure, il suffit de tester l'ouverture de la serrure, l'évaluation d'un complexe de molécules est moins évidente.
				La stabilité d'un complexe de molécules est principalement évaluée selon deux critères : la complémentarité géométrique et la complémentarité chimique \myref*{fig-sota-MesuresPourLEvaluationDuDockingMoleculaire}.
				La complémentarité géométrique, parfois nommée complémentarité structurelle \mycite{Church-1977}, consiste à trouver les parties de chaque molécule qui s'imbriquent le mieux l'une avec l'autre, comme un puzzle en \myThreeD.
				\mycite[author]{Jiang-2003} montre l'importance de la complémentarité géométrique dans le \myglos{glo-DockingMoleculaire} dans l'évaluation d'un complexe de molécules.
				La \myref{fig-sota-MesuresPourLEvaluationDuDockingMoleculaire-ComplementariteGeometrique} illustre la complémentarité géométrique dans une représentation simplifiée en \myTwoD.

				Cependant, la stabilité d'un complexe de molécule s'accompagne également d'une évaluation de la complémentarité chimique comme décrit par \mycite[author]{Kessler-1999}.
				Cette complémentarité tient compte des interactions chimiques entre les molécules comme les charges électrostatiques \mycite{McCoy-1997}, les ponts hydrogènes \mycite{Arunan-2011} ou encore les régions hydrophiles et hydrophobes \mycite{Blalock-1984}.
				La \myref{fig-sota-MesuresPourLEvaluationDuDockingMoleculaire-ComplementariteElectrostatique} illustre la complémentarité électrostatique.

				\begin{myfigure}
					\newcommand{\mymolA}{%
						\psclip{%
							\pscustom[linestyle=none]{%
								\psline[linestyle=none](5,2)(5,3)(-5,3)(-5,2)%
								\pscurve[linestyle=none](-5,2)(-5,1.5)(-3,0)(0,0.5)(1,-0.5)(2,1)(3,0)(5,1.5)(5,2)%
							}%
						}%
						\psframe[linestyle=none,fillstyle=gradient,gradmidpoint=0.75,gradbegin=white,gradend=myred,GradientCircle=true,GradientScale=7.5,GradientPos={(0.001,-10)}](-5,-0.5)(5,3)%
						\endpsclip%
						\psecurve[linestyle=solid,linewidth=1pt,linecolor=black](-5,2)(-5,1.5)(-3,0)(0,0.5)(1,-0.5)(2,1)(3,0)(5,1.5)(5,2)%
					}
					\newcommand{\mymolB}{%
						\psclip{%
							\pscustom[linestyle=none]{%
								\psline[linestyle=none](5,-2)(5,-3)(-5,-3)(-5,-2)%
								\pscurve[linestyle=none](-5,-2)(-5,-1.5)(-3,-0.5)(0,0)(1,-1)(2,0.5)(3,-0.5)(5,-1.5)(5,-2)
							}%
						}
						\psframe[linestyle=none,fillstyle=gradient,gradmidpoint=0.75,gradbegin=white,gradend=myblue,GradientCircle=true,GradientScale=7.5,GradientPos={(0.001,10)}](-5,0.5)(5,-3)
						\endpsclip
						\psecurve[linestyle=solid,linewidth=1pt,linecolor=black](-5,-2)(-5,-1.5)(-3,-0.5)(0,0)(1,-1)(2,0.5)(3,-0.5)(5,-1.5)(5,-2)
					}
					\def\mysign(#1,#2)#3{\rput(#1,#2){\Large\bfseries\sffamily\textcolor{white}{#3}}}
					\begin{mysubfigure}
						\psset{unit=0.049\textwidth}
						\begin{myps}(-5,-3)(5,3)
							\mymolA
							\mymolB
						\end{myps}
						\mysubcaption[fig-sota-MesuresPourLEvaluationDuDockingMoleculaire-ComplementariteGeometrique]{Complémentarité géométrique}
					\end{mysubfigure}
					\begin{mysubfigure}
						\psset{unit=0.049\textwidth}
						\begin{myps}(-5,-3)(5,3)
							\mymolA
							\mymolB
							\mysign(-3,0.5){+}
							\mysign(-3,-1){--}
							\mysign(-0.5,-0.5){+}
							\mysign(-0.5,0.75){--}
							\mysign(2,1.5){+}
							\mysign(2,-0.25){--}
						\end{myps}
						\mysubcaption[fig-sota-MesuresPourLEvaluationDuDockingMoleculaire-ComplementariteElectrostatique]{Complémentarité électrostatique}
					\end{mysubfigure}
					\mycaption[fig-sota-MesuresPourLEvaluationDuDockingMoleculaire]{Mesures pour l'évaluation du \myglosnl{glo-DockingMoleculaire}}
				\end{myfigure}

				Le nombre de combinaisons géométriques et le nombre de contraintes chimiques font du \myglos{glo-DockingMoleculaire} une tâche de recherche très complexe.
				Une exploration exhaustive de l'espace de recherche est impossible.
				La section suivante présente les différentes solutions logicielles existante pour tenter de trouver des solutions de \myglos{glo-DockingMoleculaire}.
			\end{mysubsection}
			\begin{mysubsection}[sse-sota-RechercheDeSolutionsDeDockingMoleculaire]{Recherche de solutions de \myglosnl{glo-DockingMoleculaire}}
				La recherche de solutions de \myglos{glo-DockingMoleculaire} consiste à trouver les zones de liaisons entre les molécules.
				Les algorithmes de recherche se basent principalement sur deux éléments : l'évaluation et l'optimisation.
				L'évaluation consiste à calculer un score pour la conformation trouvée.
				L'optimisation s'intéresse à l'amélioration et à l'affinage des conformations.

				\mycite[author]{Schulz-Gasch-2004} ou encore \mycite[author]{Leach-2006} proposent un état de l'art sur les moyens d'évaluer un \myglos{glo-DockingMoleculaire}.
				Les algorithmes de recherche commencent par identifier les différents sites de liaisons potentiels par complémentarité géométrique.
				Cette première évaluation permet de filtrer l'espace des solutions.
				Puis une évaluation chimique partielle ou complète est éventuellement effectuée.

				Parmi les algorithmes d'optimisation les plus utilisés, on peut citer les algorithmes génétiques, les \myacro*{acr-ICM}, la méthode de \myname{Monte-Carlo} ou encore la reconstruction incrémentale autour d'une base protéinique.
				À chaque algorithme est associé une ou plusieurs solutions logicielles dont les plus référencées selon \mycite[author]{Grosdidier-2007} sont \myAutoDock (\mynum[\%]{27}), \myGOLD (\mynum[\%]{15}), \myFlexX (\mynum[\%]{11}), \myDOCK (\mynum[\%]{6}) ou encore \myICMDocking (\mynum[\%]{6}).
				La \myref{tab-sota-ListeNonExhaustiveDeSolutionsLogiciellesDeDockingMoleculaire} propose une liste de solutions logicielles de \myglos{glo-DockingMoleculaire}.

				Cependant, les solutions présentées ci-dessus se basent sur une approximation importante de l'environnement moléculaire : les molécules sont considérées comme des corps rigides.
				En effet, une molécule est constituée d'un ensemble d'atomes possédant chacun une mobilité par rapport à ces voisins; une molécule s'apparente plutôt à un corps flexible.
				La flexibilité d'une molécule peut être vue à différents niveaux de granularité, de l'atome aux molécules en passant par les structures secondaires (\myhelice* et \myfeuillet*).
				On peut distinguer trois niveaux de flexibilité différentes :
				\begin{description}
					\item[Niveau inter-moléculaire] Cette déformation au niveau macro-moléculaire concerne des transformations de grande amplitude sur chaque molécule.
						Elle permet de trouver la meilleure concordance entre les molécules en terme de position et d'orientation.
					\item[Niveau intra-moléculaire] Cette déformation est au niveau moléculaire.
						L'amarrage de deux molécules (ou plus) permet d'obtenir de nombreux sites de liaison qui doivent être optimisées en fonction de critères variés (la complémentarité géométrique, les forces électrostatiques, \myetc).
						La flexibilité s'organise alors autour de macro-structures telles que les structures secondaires qui font de la molécule une sorte de chaîne articulée \myref*{fig-sota-IllustrationDUneMoleculeAvecCesHelicesAlphaEtCesFeuilletsBeta}.
					\item[Niveau atomique] Cette déformation très fine va optimiser la position des atomes au niveau du site de liaison en modifiant l'état des \myglos*{glo-Residu} (groupement d'atomes).
						L'intérêt de cette étape sera portée sur plusieurs types d'interactions chimiques à échelle réduite (les ponts hydrogènes, les zones hydrophobiques et hydrophylliques, les ponts salins, les forces de \myname[van der]{Waals} \mycite{Muller-1994}, \myetc).
				\end{description}

				\begin{myfigure}
					\myimage{sota-docking-alphabeta}
					\mycaption[fig-sota-IllustrationDUneMoleculeAvecCesHelicesAlphaEtCesFeuilletsBeta]{Illustration d'une molécule avec ces \myhelice* et ces \myfeuillet*}
				\end{myfigure}

				La complexité induite par la flexibilité rend l'exploration de l'espace de recherche encore plus complexe par rapport aux corps rigides considérés précédemment.
				Afin de répondre à cette problématique supplémentaire, différentes approches, basées sur des plateforme de \myglos{glo-DockingMoleculaire} \myref*{tab-sota-ListeNonExhaustiveDeSolutionsLogiciellesDeDockingMoleculaire}, ont été proposées afin de réduire l'espace de recherche :
				\begin{itemize}
					\item Une partie du complexe de molécules étudié est rendue rigide;
					\item Utiliser plusieurs conformations rigides d'une molécule \mycite{Meagher-2004};
					\item Découper l'espace de recherche avec une granularité plus grossière pour améliorer les performances d'évaluation \mycite{Osterberg-2002}.
				\end{itemize}

				\begin{mytable}
					\begin{mytabular}{^>{\hsize=0.2\textwidth}C->{\hsize=0.3\textwidth}C->{\hsize=0.4\textwidth}C}
						\mytoprule
						\myrowstyle{\bfseries}
						Logiciel & Algorithme & Références \\
						\mymiddlerule[\heavyrulewidth]
						\myAutoDock & Algorithme génétique & \mycite{Morris-1998}\newline\mycite{Osterberg-2002} \\
						\mymiddlerule
						\myDOCK & Reconstruction incrémentale & \mycite{Ewing-2001} \\
						\mymiddlerule
						\myICMDocking & Méthode \myacro{acr-ICM} & \mycite{Abagyan-1994a}\newline\mycite{Abagyan-1994b} \\
						\mymiddlerule
						\myGOLD & Algorithme génétique & \mycite{Jones-1997} \\
						\mymiddlerule
						\myFlexX & Reconstruction incrémentale & \mycite{Rarey-1997}\newline\mycite{Rarey-1999} \\
						\mymiddlerule
						\myGlide & Méthode de \myname{Monte-Carlo} & \mycite{Friesner-2004}\newline\mycite{Halgren-2004} \\
						\mymiddlerule
						\myBoxSearch & Méthode de \myname{Monte-Carlo} & \mycite{Hart-1992}\newline\mycite{Cummings-1995} \\
						\mybottomrule
					\end{mytabular}
					\mycaption[tab-sota-ListeNonExhaustiveDeSolutionsLogiciellesDeDockingMoleculaire]{Liste non-exhaustive de solutions logicielles de \myglosnl{glo-DockingMoleculaire}}
				\end{mytable}

				La flexibilité introduit une complexité importante dans la recherche de solutions en \myglos{glo-DockingMoleculaire}.
				Cependant, nous allons voir dans la section suivante que l'utilisation des capacités combinées de l'humain et de la machine permet une approche différente face à ce problème complexe.
			\end{mysubsection}
			\begin{mysubsection}[sse-sota-DockingMoleculaireEnEnvironnementVirtuel]{\myGlosnl{glo-DockingMoleculaire} en environnement virtuel}
				Malgré une communauté scientifique très active pour améliorer les solutions de \myglos{glo-DockingMoleculaire} existantes et se rapprocher toujours plus près de conditions biologiques réalistes, la complexité du problème rend difficile la découverte de solutions pertinentes.
				Devant cette complexité, une approche alternative basée sur l'introduction de l'humain et de ses capacités de décisions au sein du processus de recherche.
				En effet, bien que moins rapide pour traiter un grand nombre de données, un expert est capable de classer plus intelligemment les solutions pertinentes et les solutions aberrantes.

				L'idée d'immerger un humain au sein du processus de \myglos{glo-DockingMoleculaire} date de $1967$ avec le projet \myGROPE comme l'explique \mycite[author]{Grunwald-2008}.
				L'intervention d'un expert durant le processus de recherche est effectuée par l'immersion dans des environnements de \myglos{glo-RealiteVirtuelle}.
				\mycite[author]{Batter-1972} proposent les premières solutions d'immersion visuelle grâce à ce projet \myGROPE.

				Avec l'immersion dans un environnement virtuel, le problème de la représentation des molécules se pose rapidement.
				\mycite[author]{Bergman-1993} proposent la plateforme \myVIEW avec différentes possibilités de rendus graphiques pour représenter une molécule \myref*{fig-sota-RepresentationAvanceesDeMoleculesProposeeParBergman1993}.
				Les environnement virtuels offrent une certaine souplesse dans la représentation des molécules ce qui a permis d'imaginer différentes manières d'afficher une molécule en fonction des informations à mettre en évidence.
				Des logiciels dédiés à ce genre de tâche ont alors pu voir le jour comme \myacro{acr-VMD} \mycite{Humphrey-1996} ou \myPyMOL \mycite{DeLano-2002}.

				\begin{myfigure}
					\myimage{sota-Bergman-1993}
					\mycaption[fig-sota-RepresentationAvanceesDeMoleculesProposeeParBergman1993]{Représentation avancée proposée par \mycite[author]{Bergman-1993}}
				\end{myfigure}

				Cependant, la nature statique de ces visualisations ne permet pas de comprendre la dynamique chimique au sein d'une molécule.
				\mycite[author]{Huitema-2000} proposent un outil pour visualiser des trajectoires d'atomes afin d'appréhender la dynamique des protéines.
				Puis \mycite[author]{Klosowski-2002} franchissent une étape supplémentaire en proposant une visualisation temps-réel de la dynamique d'une molécule à l'aide du moteur de simulation \myGromacs \mycite{Berendsen-1995,Hess-2008}.
				Enfin, \mycite[author]{Krenek-1999} puis \mycite[author]{Davies-2005} apporte une visualisation multimodale avec la combinaison des retours visuels et haptiques pour percevoir les champs de force électriques d'une molécule \myref*{fig-sota-VisualisationMultimodaleProposeeParDavies2005}.

				\begin{myfigure}
					\myimage{sota-Davies-2005}
					\mycaption[fig-sota-VisualisationMultimodaleProposeeParDavies2005]{Visualisation multimodale proposée par \mycite[author]{Davies-2005}}
				\end{myfigure}

				L'utilisation de l'haptique dans ce contexte n'est d'ailleurs pas nouveau.
				En effet, les biologistes ont également cherché à interagir virtuellement avec les molécules afin de pouvoir s'affranchir de la manipulation dans le monde réel nécessitant une procédure plus longue et plus complexe.
				Le projet \myGROPEHaptic\footnote{Le projet \myGROPEHaptic a été réaliser dans le prolongement du projet \myGROPE.} se propose d'utiliser la modalité haptique pour interagir avec les molécules \mycite{Ouh-Young-1988,Brooks-1990}.
				Puis, une plateforme pour la modélisation moléculaire assistée par ordinateur est proposée avec \myHIMM (\textit{Highly Immersive Molecular Modeling}) \mycite{Drees-1996,Drees-1998}; l'objectif est de proposer une plateforme immersive et modulable afin de pouvoir facilement changer les différents outils d'interaction.
				Certains proposent même des interfaces tangibles se substituant aux molécules à manipuler comme \mycite[author]{Weghorst-2003} et \mycite[author]{Kim-2004b}.

				En tenant compte de la flexibilité des molécules, les besoins en interaction se sont affinés; il n'est plus question de modifier la position d'une molécule, on cherche à présent à modifier la position d'un atome ou d'un groupe d'atomes.
				Le modèle d'interaction proposé est la traction des atomes par un simple modèle masse-ressort \mycite{Haan-2002,Koutek-2002} ce qui permet une déformation des atomes de manière locale.
				Puis \mycite[author]{Lee-2004} propose de baser les modèles d'interactions haptiques sur l'approximation du champ de force décrit dans \mycite{Lennard-Jones-1924a,Lennard-Jones-1924b} \myref*{eq-sota-PotentielDeLennardJones}; ce modèle de force a l'avantage de permettre une relative stabilité du retour haptique \myref*{fig-sota-PotentielDeLennardJones1924}.
				\begin{equation}\label{eq-sota-PotentielDeLennardJones}
					V_{LJ} = 4\varepsilon \left[\left(\frac{\sigma}{r_m}\right)^{12} - \left(\frac{\sigma}{r_m}\right)^6\right]
				\end{equation}

				\begin{myfigure}
					\psset{xunit=0.181818182\textwidth,yunit=0.125\textwidth}
					\begin{myps}(-0.5,-1)(5,2.25)
						\newcommand{\myEzero}{0.8 }
						\newcommand{\mysigma}{1.5 }
						\newcommand{\myrm}{1.683693072}
						\psaxes{->}(0,0)(0,-1)(5,2.25)
						\makeatletter
						\setlength{\my@ps@xoffset}{1\pslabelsep}%
						\settowidth{\my@ps@yoffset}{-1}
						\addtolength{\my@ps@yoffset}{2\pslabelsep}%
						\uput{\my@ps@xoffset}[90](4.5,0){\textit{distance}}
						\uput{\my@ps@yoffset}[180](0,0.5){\rotateleft{\textit{énergie potentielle}}}
						\makeatother
						\psplot[linewidth=1.5pt,linecolor=myred,plotpoints=1000]{1.412304381}{4.75}{4 \myEzero \mysigma x div \mysigma x div mul \mysigma x div mul \mysigma x div mul \mysigma x div mul \mysigma x div mul \mysigma x div mul \mysigma x div mul \mysigma x div mul \mysigma x div mul \mysigma x div mul \mysigma x div mul \mysigma x div \mysigma x div mul \mysigma x div mul \mysigma x div mul \mysigma x div mul \mysigma x div mul sub mul mul}
						\psline[linestyle=dashed,linecolor=myblue](\myrm,0)(\myrm,-\myEzero)
						\uput[45](\myrm,0){\textcolor{myblue}{$r_m = 2^{\frac{1}{6}} \sigma$}}
						\psline[linestyle=dashed,linecolor=myblue](0,-\myEzero)(\myrm,-\myEzero)
						\uput[180](0,-\myEzero){\textcolor{myblue}{$\varepsilon$}}
					\end{myps}
					\mycaption[fig-sota-PotentielDeLennardJones1924]{Potentiel de \mycite[author]{Lennard-Jones-1924a}}
				\end{myfigure}

				Les différents outils de manipulation moléculaire se mettant en place, les chercheurs se sont de nouveau orientés vers l'une de leur première problématique : le \myglos{glo-DockingMoleculaire}.
				Après les projets comme \myGROPE, d'autres projets de \myglos{glo-DockingMoleculaire} comme \mySTALK \mycite{Levine-1997} voient le jour.
				La communauté haptique commence également à s'intéresser à cette problématique \mycite{Subasi-2006a,Subasi-2006b,Subasi-2008}.
				Par exemple, \mycite[author]{Lai-Yuen-2005} proposent une interface haptique avec cinq \myacro*{acr-DDL} \myref*{fig-sota-DockingMoleculaireALAideDUneInterfaceACinqDDLsLaiYuen2006} afin d'effectuer du \myglos{glo-DockingMoleculaire} \mycite{Lai-Yuen-2006}.
				Puis, à l'aide d'interfaces haptiques comme le \myVirtuose (six \myacro*{acr-DDL}), la manipulation moléculaire \mycite{Daunay-2007} puis le \myglos{glo-DockingMoleculaire} \mycite{Daunay-2009} accompagné d'une évaluation temps-réel de l'énergie du complexe moléculaire devient possible.
				Le \myglos{glo-DockingMoleculaire} faisant intervenir des champs de force spécifiques, \mycite[author]{Hou-2010} proposent des modèles de retours haptiques permettant de ressentir les moments\footnote{Utilisé ici dans le sens mécanique, force autour d'un pivot (moment ou couple).} interviennent au niveau de structures intra-moléculaires ou inter-moléculaires.

				Plus récemment, deux projets ont permis d'aboutir à des plateformes de manipulation haptiques de molécules avec un moteur de simulation en temps-réel.
				Tout d'abord, \mycite[author]{Redon-2005} développent des algorithmes pour la simulation en temps-réel de corps articulés avec un grand nombre de \myacro*{acr-DDL}.
				Puis, il adapte son travail pour créer un moteur de simulation moléculaire ce qui lui permet d'obtenir des simulations temps-réel \mycite{Rossi-2007}, utilisable pour la manipulation haptique \mycite{Bolopion-2009}.
				En suivant un cheminement similaire, \mycite[author]{Delalande-2009} proposent un outil permettant d'obtenir des simulations moléculaires en temps-réel.
				Puis dans un second temps, \mycite[author]{Delalande-2010} rendent possible la manipulation haptique temps-réel de molécules en proposant la déformation au niveau atomique.

				\begin{myfigure}
					\myimage{sota-Lai-Yuen-2006}
					\mycaption[fig-sota-DockingMoleculaireALAideDUneInterfaceACinqDDLsLaiYuen2006]{\myGlosnl{glo-DockingMoleculaire} à l'aide d'une interface à cinq \myacronl*{acr-DDL} \mycite{Lai-Yuen-2006}}
				\end{myfigure}

				Les différentes briques techniques et logicielles permettant d'effectuer du \myglos{glo-DockingMoleculaire} flexible interactif existent.
				\mycite[author]{Bolopion-2009} puis \mycite[author]{Delalande-2010} nous montrent la faisabilité des simulations moléculaires réalistes et interactives.
				Puis \mycite[author]{Daunay-2009} offrent un moyen d'évaluer en temps-réel ces simulations.
				Le logiciel \myacro{acr-VMD} offre une plateforme de visualisation moléculaire permettant de connecter des outils haptiques \mycite{Stone-2010}.
				Des modèles de forces haptiques spécifiques au \myglos{glo-DockingMoleculaire} sont développés \mycite{Hou-2010}.

				En fournissant tous ces outils de visualisation et d'interaction avec des environnement moléculaires virtuels, les possibilités des biologistes sont augmentées.
				Cependant, la complexité du problème de \myglos{glo-DockingMoleculaire} est tellement importante que nous allons avoir recours à une aide supplémentaire : le travail collaboratif.

				\begin{myfigure}
					\myimage{sota-Daunay-2009}
					\mycaption[fig-sota-DockingMoleculaireRigideAvecLeVirtuoseDaunay2009]{\myGlosnl{glo-DockingMoleculaire} rigide avec le \myVirtuose \mycite{Daunay-2009}}
				\end{myfigure}
			\end{mysubsection}
			\begin{mysubsection}[sse-sota-PlateformesCollaborativesPourLaBiologieMoleculaire]{Plateformes collaboratives pour la biologie moléculaire}
				Les applications collaboratives en biologie moléculaire ont commencé à être développées pour répondre au besoin de travailler à plusieurs.
				L'une des premières application a été proposée par \mycite[author]{Casher-1995} sous le nom de \myEyeChem.
				Cette application constituée d'une nœud serveur et de nœuds clients sur un réseau interne, permet à plusieurs utilisateurs d'éditer et de manipuler une molécule de manière synchrone.

				Sur le même principe, \mycite[author]{Bourne-1998} développent \myMICE (\textit{Molecular Interactive Collaborative Environment}), permettant de visualiser des molécules en collaboration distante.
				Cette application se distingue par l'utilisation d'\myInternet comme réseau ce qui permet une plus grande souplesse dans le déploiement.
				\mycite[author]{Tate-2001} étendent les possibilités de \myMICE en permettant l'édition, la manipulation et l'interaction d'une même molécule par plusieurs utilisateurs.
				Un projet similaire appelé \myChimera, développé par \mycite[author]{Pettersen-2004}, existe également et propose le même type de fonctionnalités.

				Puis, basé sur la plateforme de \mycite[author]{Kim-2004b} permettant de faire de la modélisation de molécules par gestes, \mycite[author]{Park-2006} proposent un système distribué de collaboration distante.
				C'est l'une des premières applications qui aborde le problème du \myglos{glo-DockingMoleculaire}.
				Les gestes sont effectuées avec des \textit{haptic eggs}, deux petites interfaces tangibles placées dans chaque main de l'utilisateur, chaque \textit{haptic egg} étant l'abstraction d'une molécule.
				Cependant, le système de communication est déporté dans le temps puisque les utilisateurs ont la possibilité de commenter et de donner des opinions sur la manière de réaliser le \myglos{glo-DockingMoleculaire} à l'aide d'un système de messagerie instantanée.
				Les décisions sont alors transmises à l'utilisateur qui s'occupe des manipulations en environnement virtuel.
				Plus récemment, le projet \myeMinerals \mycite{Dove-2005}, fournit une solution pour la collaboration entre biologistes (messagerie instantanée, vidéo-conférence, serveur centralisé, \myetc) comme on peut le voir sur la \myref{fig-sota-PlateformeDeCollaborationInterUniversiteProposeeParDove2005}.
				Cependant, il s'agit surtout de collaboration distante et permet un travail synchrone ou asynchrone.

				\begin{myfigure}
					\myimage{sota-Dove-2005}
					\mycaption[fig-sota-PlateformeDeCollaborationInterUniversiteProposeeParDove2005]{Plateforme de collaboration inter-universités proposée par \mycite{Dove-2005}}
				\end{myfigure}

				Parmi les applications collaboratives de biologie moléculaire permettant une interaction synchrone, \mycite[author]{Chastine-2005} sont les premiers à s'intéresser aux problèmes de communication entre les sujets \myref*{sss-sota-LaConscienceDeGroupe}.
				À travers l'application \myAMMPVis, il propose des outils pour désigner des zones de l'environnement virtuel.
				Afin d'améliorer la communication, il propose également de modéliser les mains de chaque utilisateur dans l'environnement virtuel pour que chacun ait conscience des agissement des partenaires \myref*{fig-sota-ManipulationCollaborativeDeMoleculeAvecRepresentationDesMainsDesParticipantsChastine2005}.
				Cette plateforme a ensuite été réutilisée sous la forme de \myAMMPEXTN afin d'introduire des procédures de contrôle d'accès.
				Pour éviter les conflits entre les utilisateurs, les différentes strates d'informations (paramètres de visualisation, modélisation, modification des données) ont été soumises à différents droits de visualisation et de modification \mycite{Ma-2007,Ma-2007a}.

				\begin{myfigure}
					\myimage{sota-Chastine-2005}
					\mycaption[fig-sota-ManipulationCollaborativeDeMoleculeAvecRepresentationDesMainsDesParticipantsChastine2005]{Manipulation collaborative de molécule avec représentation des mains des participants \mycite{Chastine-2005}}
				\end{myfigure}

				Cependant, les applications présentées ne concerne que des application de collaboration distante.
				La manipulation colocalisée permet de créer un environnement de travail social qui permet une meilleure communication.
				L'application \myPaulingWorld \mycite{Su-2000,Su-2001} propose une des premières applications de collaboration colocalisée pour la visualisation et l'interaction.
				La visualisation s'effectue sur un \myWorkbench ce qui permet aux utilisateurs d'avoir un niveau d'interaction similaire dans l'application.
				Pour améliorer l'immersion, \mycite[author]{Kriz-2003} proposent aux utilisateurs la visualisation de molécule au travers différents équipements dont un \myCAVE à l'aide de l'application \myDIVERSE.

				Nous venons de présenter plusieurs solutions logicielles et matérielles permettant de créer un espace de collaboration pour la modélisation moléculaire et le \myglos{glo-DockingMoleculaire}.
				Cependant, la présentation de ces travaux abordent surtout les contraintes techniques nécessaires à la mise en place de tels systèmes sans évoquer les contraintes humaines de la collaboration.
				Seul \mycite[author]{Chastine-2005} semblent avoir aborder les aspects de communication dans ces applications.
				La prochaine section va nous permettre d'aborder cette problématique.
			\end{mysubsection}
		\end{mysection}
		\begin{mysection}[sec-sota-ApprocheCollaborativePourLesProblemesComplexes]{Approche collaborative pour les problèmes complexes}
			C'est grâce à la communauté en psychologie sociale que nous allons à présent définir et caractériser les différents aspects d'une collaboration entre humains.
			Quels sont les avantages d'un travail en collaboration ?
			Quels en sont les inconvénients ?
			En quoi le travail collaboratif est-il un choix pertinent pour appréhender des tâches complexes ?

			Avant d'aborder ces différents points, il est nécessaire de préciser la distinction entre travail coopératif et travail collaboratif.
			Pour comprendre cette différence, nous nous appuierons sur la définition relativement claire proposée par \mycite[author]{Roschelle-1995}.
			\begin{myquote}[english]
				\it Cooperative work is accomplished by the division of labour among participants, as an activity where each person is responsible for a portion of the problem solving.
				We focus on collaboration as the mutual engagement of participants in a coordinated effort to solve the problem together.
			\end{myquote}
			dont on trouve une traduction en français dans les travaux de \mycite[author]{Knauf-2010}
			\begin{myquote}[french]
				Le travail coopératif implique une division du travail entre les participants, chaque participant étant responsable d’une partie du problème à résoudre.
				Dans la collaboration, les participants s’engagent tous dans les mêmes tâches, en se coordonnant, afin de résoudre le problème ensemble.
			\end{myquote}
			Les travaux présentés ici concernent le travail collaboratif et ne traiterons pas de travail coopératif, sauf mention contraire.
			\begin{mysubsection}[sse-sota-LaDistributionCognitiveDesChargesDeTravail]{La distribution cognitive des charges de travail}
				\mycite[author]{Bandura-1986} puis \mycite[author]{Foushee-1987} sont parmi les premiers à avoir l'intuition que le travail collaboratif apporte plus qu'une simple multiplication des ressources : les interactions entre les individus font émerger des compétences propre au groupe.
				Par exemple, \mycite[author]{Wegner-1987} s'intéresse à la distribution de mémoire dans un groupe : chaque individu ne détient qu'une partie de l'information mais la capacité mémorielle du groupe dans son ensemble est plus importante.
				\mycite[author]{George-1990} souligne également que l'état émotionnel de chacun affecte l'état émotionnel du groupe\footnote{\mycite[author]{Yammarino-1992} ont remis en cause les conclusions obtenues mais le travail de \mycite[author]{George-1990} a été de nouveau confirmé par \mycite[author]{George-1993}.}.

				\mycite[author]{Hollan-2000} proposent de définir la distribution cognitive de la façon suivante
				\begin{myquote}[english]
					\it Unlike traditional theories, however, [the theory of distributed cognition] extends the reach of what is considered cognitive beyond the individual to encompass interactions between people and with resources and materials in the environment.
				\end{myquote}
				pour laquelle \mycite[author]{Conein-2004} propose une traduction
				\begin{myquote}[french]
					On peut déplacer la frontière de l'unité cognitive d'analyse au-delà de l'enveloppe corporelle de l'individu de façon à inclure le matériel et l'environnement social comme composant d'un système cognitif plus étendu.
				\end{myquote}

				C'est \mycite[author]{Hutchins-1995} qui met en évidence la notion de distribution cognitive des charges de travail avec une étude des interactions qui ont lieu dans un cockpit d'avion.
				Ces conclusions sont confirmées et explicités grâce à ses observations en psychologie sociale et ses connaissances en anthropologie \mycite{Hutchins-1996} : il fait une distinction entre les propriétés cognitives d'un individu et les propriétés cognitives d'un groupe.
				Ces travaux donnent lieu à la création de deux communautés; ceux qui considèrent que le travail cognitif d'un groupe d'individus est une somme des propriétés cognitives de chaque individu; et ceux qui considèrent que certains aspects cognitifs d'un groupe d'individus sont propres à la collaboration.

				Pour justifier le rejet de l'approche individualiste, \mycite[author]{Clark-1998} explique que lorsque notre cognition s'appuie sur une aide externe, elle devient interactive et relationnelle, c'est-à-dire non détachable d'un composant externe présent dans l'environnement.
				De plus, \mycite[author]{Clark-2001} évoque le lien étroit entre la notion d'extension cognitive et les processus cognitifs complexes : les problèmes de nature complexe stimule cette extension cognitive.

				Plus récemment, \mycite[author]{Zhang-2006} synthétisent la distribution cognitive comme des systèmes ayant des interactions externes (avec des matériels) \mycite{Zhang-1994} et des interactions internes (avec des collaborateurs) comme illustré sur la \myref{fig-sota-RepresentationDunSystemeCognitivementDistribue}.
				Ces éléments externes sont inclus dans le processus de distribution cognitive puisqu'il permettent également de soulager la charge cognitive des individus.
				D'ailleurs, une liste des propriétés cognitives auxquelles peuvent répondre ces éléments externes est dressées par \mycite[author]{Zhang-1997} :
				\begin{itemize}
					\item fournir une aide mémorielle à court ou long terme afin de réduire la charge cognitive;
					\item améliorer et simplifier la perception de l'information pour la rendre rapidement accessible et appréhendable;
					\item fournir des connaissances et des compétences qui ne sont pas disponibles en interne;
					\item aider les opérateurs pour la perception afin qu'ils identifient facilement les caractéristiques et puisse effectuer des déductions;
					\item structurer et fixer les comportements cognitifs de manière inconsciente;
					\item changer la nature de la tâche en générant des séquences d'actions plus efficaces;
					\item arrêter le temps ou permettre des répétitions afin de rendre visible et durables des informations qui ne le sont pas;
					\item limiter l'abstraction;
					\item maximiser la précision et minimiser l'effort dans la prise de décisions pour déterminer une stratégie.
				\end{itemize}

				\begin{myfigure}
					\psset{unit=0.1\textwidth}
					\definecolor{mylightestredblue}{rgb}{0.875 0.705 0.79328125}
					\begin{myps}(-5,-2.5)(5,4)
						\pspolygon*[linearc=0.5,linecolor=mylightestred](-0.5,4)(-4.5,4)(-4.5,0)(-0.5,-1.7)(1.15,-1.6)
						\pspolygon*[linearc=0.5,linecolor=mylightestblue](0.5,4)(4.5,4)(4.5,0)(0.5,-1.7)(-1.15,-1.6)
						\psclip{%
							\pscustom[linestyle=none]{%
								\pspolygon*[linearc=0.5](-0.5,4)(-4.5,4)(-4.5,0)(-0.5,-1.7)(1.15,-1.6)
							}
							\pscustom[linestyle=none]{%
								\pspolygon*[linearc=0.5](0.5,4)(4.5,4)(4.5,0)(0.5,-1.7)(-1.15,-1.6)
							}
						}
						\psframe*[linecolor=mylightestredblue](-5,-2.5)(5,4)
						\endpsclip
						\psset{fillcolor=white}
						\psframe[fillstyle=solid,linecolor=black!70,linestyle=dashed,framearc=0.25](-4.5,0)(-0.5,4)
						\rput(-2.5,3.75){\textcolor{black!70}{Espace interne}}
						\psframe[fillstyle=solid,linecolor=black!70,linestyle=dashed,framearc=0.25](4.5,0)(0.5,4)
						\rput(2.5,3.75){\textcolor{black!70}{Espace externe}}
						\psframe[linecolor=black!70,linestyle=dashed,framearc=0.25](-4.5,-2.5)(4.5,-0.5)
						\mynode(-2,3)[individu1]{\textcolor{myblue}{Individu~\mynum{1}}}
						\mynode(-3,2)[individu2]{\textcolor{myred}{Individu~\mynum{2}}}
						\mynode(-2,1)[individu3]{\textcolor{mygreen}{Individu~\mynum{3}}}
						\mynode(2.5,2.5)[artefact1]{Artefact~\mynum{1}}
						\mynode(2.5,1.5)[artefact2]{Artefact~\mynum{2}}
						\psellipse[fillstyle=solid,fillcolor=white](0,-1.25)(1,0.5)
						\rput(0,-1.25){Tâche}
						\rput(0,-2.25){\textcolor{black!70}{Abstraction de l'espace de travail}}
						\psset{arrows={<->}}
						\ncline{individu1}{individu2}
						\ncline{individu2}{individu3}
						\ncline[offset=8pt]{individu1}{individu3}
						\psset{arrows={-}}
						\ncline{artefact1}{artefact2}
						\psset{arrows={<->}}
						\psset{linecolor=myblue}
						\ncline{individu1}{artefact1}
						\ncline{individu1}{artefact2}
						\psset{linecolor=myred}
						\ncline{individu2}{artefact1}
						\ncline{individu2}{artefact2}
						\psset{linecolor=mygreen}
						\ncline{individu3}{artefact1}
						\ncline{individu3}{artefact2}
					\end{myps}
					\mycaption[fig-sota-RepresentationDunSystemeCognitivementDistribue]{Représentation d'un système cognitivement distribué}
				\end{myfigure}

				Les éléments externes de la distribution cognitive, que \mycite[author]{Norman-1991} et \mycite[author]{Kirsh-1999} appellent artefact, répondent au besoin d'affordance.
				Le concept d'affordance a été identifié par \mycite[author]{Gibson-1977}.
				L'affordance est l'ensemble des possibilités d'interaction d'un acteur sur et avec un artefact \mycite{Gibson-1979} mais cette définition s'est par la suite réduite aux seules possibilités dont l'acteur est conscient.
				C'est \mycite[author]{Norman-1988} qui utilise ce terme pour la première fois dans le contexte de la distribution cognitive.
				Il en fait une description très précise lorsqu'il se rend compte que ce terme est parfois mal utilisé par la communauté \mycite{Norman-1999}.
				Le terme est ensuite utilisé par les fondateurs de la psychologie cognitive sociale \mycite{Patel-2000}.
				Il est donc nécessaire de prendre en compte l'affordance au sein de la conception de plateformes collaboratives.

				Nous venons de voir que la collaboration permet de créer une intelligence de groupe mais cette intelligence permet-elle d'augmenter les performances d'un groupe ?
				\mycite[author]{Foushee-1987} montrent qu'une configuration de travail collaboratif peut amener un gain en efficacité mais peut également amener une perte d'efficacité.
				Par exemple, \mycite[author]{Patel-1999} montrent une perte d'efficacité sur une application permettant la collaboration distante.
				\mycite[author]{Zhang-1998} esquisse une explication dépendant de la répartition des connaissances dans un groupe : si les connaissances sont complémentaires, il y a un gain en efficacité.
				Nous allons voir dans les sections suivantes que différents phénomènes influent sur les performances d'un groupe.
			\end{mysubsection}
			\begin{mysubsection}[sse-sota-LaFacilitationSociale]{La facilitation sociale}
				La \myglos{glo-FacilitationSociale} est un phénomène qui a été mis en évidence par \mycite[author]{Triplett-1898}.
				Il s'est intéressé aux résultats de coureurs cyclistes ayant concouru dans trois conditions différentes :
				\begin{enumerate}
					\item Course seul;
					\item Course avec un meneur (également appelé \myemph{lièvre});
					\item Course dans des conditions de compétition.
				\end{enumerate}
				Les résultats \myref*{fig-sota-ResultatsObtenusParTriplettAvecDesCyclistes}, largement étudiés par \mycite[author]{Seashore-1899}, montrent que les coureurs en présence d'autres individus sont plus rapides que les coureurs effectuant l'épreuve seuls.

				\begin{myfigure}
					\psset{xunit=0.298507463\textwidth,yunit=0.05cm}
					\begin{myps}(-0.35,-25)(3,65)
						\myaxes(0,3){condition}(0,60)[10]{vitesse~(km/h)}
						\mybarplot{sota-triplett.csv}
						\uput{1pt}[-90](0.5,38.650022682){\bfseries\textcolor{white}{\mynum[km/h]{39}}}
						\uput{1pt}[-90](1.5,50.161371429){\bfseries\textcolor{white}{\mynum[km/h]{50}}}
						\uput{1pt}[-90](2.5,52.502386951){\bfseries\textcolor{white}{\mynum[km/h]{52}}}
					\end{myps}
					\mycaption[fig-sota-ResultatsObtenusParTriplettAvecDesCyclistes]{Résultats obtenus par \mycite[author]{Triplett-1898} avec des cyclistes}
				\end{myfigure}

				Suite à ces résultats, des expérimentations ont été menées sur des étudiants par \mycite[author]{Mayer-1903} puis par \mycite[author]{Meumann-1904}.
				\mycite[author]{Mayer-1903} confronte les étudiants à la réalisation d'une tâche (rédaction de dictée ou résolution de problème arithmétique) sous le regard d'un observateur.
				Il constate que les étudiants sont plus performants lorsqu'ils sont en présence d'un autre individu.
				\mycite[author]{Meumann-1904} ajoute une pierre à l'édifice en confirmant avec des tests plus poussés (tests de mémoire, ergographe et dynamomètre) que les étudiants sont toujours moins performants lorsqu'ils sont seuls.

				D'après \mycite[author]{Strauss-2002} qui propose un état de l'art sur le sujet, le terme \myglos{glo-FacilitationSociale} est utilisé pour la première fois par \mycite[author]{Allport-1924}.
				Il en donne la définition suivante.
				\begin{myquote}[english]
					\it The action prepared or in progress is some response participated in by all, and the social stimuli releasing or augmenting such response are the sight and sound of others doing the same thing.
				\end{myquote}
				qui peut-être traduite par
				\begin{myquote}[french]
					Une action collaborative préparée ou en progression possède une réponse et la stimulation sociale provoque une augmentation de cette réponse uniquement à la vue et au son provoqué par d’autres effectuant les mêmes mouvements.
				\end{myquote}

				\mycite[author]{Allport-1924} aborde ce phénomène par une collaboration où chaque individu effectue la même tâche.
				Dans ce même contexte, \mycite[author]{Roethlisberger-1939}\footnote{\mycite[author]{Roethlisberger-1939} basent leurs conclusions sur les travaux non-publiés de Elton \myname{Mayo} connus sous le nom de \og l'étude \textit{Hawthorne} \fg, du nom de l'entreprise \myHawthorne dans laquelle l'étude a été réalisée.} constate que le travail en groupe génère une stimulation qui augmente les performances du groupe : les performances du groupe sont meilleures que la somme des performances individuelles de chacun des membres.
				La \myglos{glo-FacilitationSociale} a également été observée sur des animaux comme les cafards \mycite{Zajonc-1969} ou les singes \mycite{Dindo-2009}.

				Cependant, des nuances de ce phénomène commencent à être observées.
				Par exemple, des différences de performances sont observées en fonction de la nature de la tâche.
				Déjà, \mycite[author]{Yerkes-1908} avait constaté, dans un contexte non-collaboratif, que les performances d'un individu pouvait dépendre de la complexité de la tâche et du niveau de stimulation.
				Une trop faible ou une trop forte stimulation diminue les performances de l'individu lors de la réalisation d'une tâche complexe \myref*{fig-sota-LoiDeYerkesDodsonSurLeLienEntreLaStimulationDesIndividusEtLesPerformances}.
				Une trop faible stimulation désintéresse l'individu de la tâche alors qu'une stimulation trop importante génère un stress diminuant les capacités de l'individu.

				\mycite[author]{Zajonc-1965} fait le lien entre la loi de \mycite[author]{Yerkes-1908} et la \myglos{glo-FacilitationSociale}.
				En effet, la présence de partenaires dans la réalisation d'une tâche permet une stimulation globale du groupe.
				Dans le cadre d'une tâche simple, chacun des membres est confiant dans ses propres capacités à réaliser la tâche et la présence d'observateurs va stimuler son besoin de bien réussir la tâche.
				Les individus ne craignent pas d'être évalué par les collaborateurs sur le travail réalisé.
				Cependant, l'évaluation par les collaborateurs dans le cadre d'une tâche complexe est différente : la tâche n'étant ni habituelle, ni facile, les membres du groupe perdent leur confiance et craignent un jugement dépréciatif.

				\begin{myfigure}
					\psset{xunit=0.090909091\textwidth,yunit=0.06\textwidth}
					\begin{myps}(-1,-1.5)(10,5)
						\psaxes[ticks=none,labels=none]{->}(0,0)(0,0)(10,5)
						\psplot[plotpoints=1000,linecolor=myblue]{1}{9}[%
							/expconst 2.71828182845904523536028747135266249775724709369995 def%
							/lambda 2 def%
							/offsetx 4 def
						]{5 1 1 expconst lambda x offsetx sub mul neg exp add div mul}
						\psplot[plotpoints=1000,linecolor=myred]{1}{9}[%
							/expconst 2.71828182845904523536028747135266249775724709369995 def%
							/piconst 3.141592653589793238462643383279502884197169399375105 def%
							/mu 0 def%
							/sigma 1 def%
							/ampconst 10 sigma 2 piconst mul sqrt mul div def%
						]{ampconst expconst x 5 sub mu sub 2 exp 2 sigma sigma mul mul div neg exp mul}
						\uput[-90](1,0){faible}
						\uput[-90](9,0){élevée}
						\uput{4ex}[-90](5,0){\textit{stimulation}}
						\uput[180](0,4){\rotateleft{bonne}}
						\uput[180](0,1){\rotateleft{mauvaise}}
						\uput{2em}[180](0,2.5){\rotateleft{\textit{performance}}}
						\uput[-90](8.5,5){\textcolor{myblue}{Tâche simple}}
						\uput[90](8.5,0){\textcolor{myred}{Tâche complexe}}
					\end{myps}
					\mycaption[fig-sota-LoiDeYerkesDodsonSurLeLienEntreLaStimulationDesIndividusEtLesPerformances]{Loi de \mycite[author]{Yerkes-1908} sur le lien entre la stimulation des individus et les performances}
				\end{myfigure}

				Malgré cela, \mycite[author]{Castro-1994} montre que la stimulation sociale est meilleure lorsque les participants se connaissent entre eux.
				En effet, les participants qui se connaissent déjà vont s'affranchir de la peur d'être évalué.

				Dans cette section, nous avons présenté la \myglos{glo-FacilitationSociale}.
				Ce phénomène permet, par la simple présence ou la participation de plusieurs collaborateurs, de stimuler un groupe et d'en augmenter les performances.
				Cependant, nous allons voir dans la section suivante qu'une autre théorie vient compléter celle de la \myglos{glo-FacilitationSociale} : la \myglos{glo-ParesseSociale}.
			\end{mysubsection}
			\begin{mysubsection}[sse-sota-LaParesseSociale]{La paresse sociale}
				\mycite[author]{Ringelmann-1913} est le premier à constater le phénomène de \myglos{glo-ParesseSociale}\footnote{Ce phénomène de \myglos{glo-ParesseSociale} est d'ailleurs parfois nommé \og effet de \mycite[author]{Ringelmann-1913} \fg.} dans un rapport technique qui sera signalé quelques années plus tard par \mycite[author]{Moede-1927} puis repris de manière détaillée dans la littérature scientifique par \mycite[author]{Kravitz-1986}.
				Il propose une expérience de traction de corde à plusieurs individus et observe la traction totale exercée par le groupe en faisant varier la taille des groupes.
				La traction totale observée pour un groupe est inférieure à la somme des efforts individuels \myref*{fig-sota-ResultatsObtenusParRingelmannEtPresentesParKravitz}.
				\mycite[author]{Schermerhorn-2009} définit la \myglos{glo-ParesseSociale} de la manière suivante
				\begin{myquote}[english]
					\it The tendency of group members to do less that they are capable of as individuals.
				\end{myquote}
				qui peut être traduite par
				\begin{myquote}[french]
					Tendance à fournir un effort moindre lorsqu'une tâche est effectuée en groupe plutôt qu'individuellement.
				\end{myquote}

				\begin{myfigure}
					\psset{xunit=0.108108108\textwidth,yunit=0.25\textwidth}
					\begin{myps}(-1,-0.3)(8.25,1.1)
						\myaxes(0,8){taille du groupe}(0,1.0)[0.2]{performance}
						\mybarplot{sota-ringelmann.csv}
						\uput{1pt}[-90](0.5,1.00){\bfseries\textcolor{white}{\mynum[\%]{100}}}
						\uput{1pt}[-90](1.5,0.93){\bfseries\textcolor{white}{\mynum[\%]{93}}}
						\uput{1pt}[-90](2.5,0.85){\bfseries\textcolor{white}{\mynum[\%]{85}}}
						\uput{1pt}[-90](3.5,0.77){\bfseries\textcolor{white}{\mynum[\%]{77}}}
						\uput{1pt}[-90](4.5,0.70){\bfseries\textcolor{white}{\mynum[\%]{70}}}
						\uput{1pt}[-90](5.5,0.63){\bfseries\textcolor{white}{\mynum[\%]{63}}}
						\uput{1pt}[-90](6.5,0.56){\bfseries\textcolor{white}{\mynum[\%]{56}}}
						\uput{1pt}[-90](7.5,0.49){\bfseries\textcolor{white}{\mynum[\%]{49}}}
					\end{myps}
					\mycaption[fig-sota-ResultatsObtenusParRingelmannEtPresentesParKravitz]{Résultats obtenus par \mycite[author]{Ringelmann-1913} et présentés par \mycite[author]{Kravitz-1986}}
				\end{myfigure}

				Cependant, les résultats de \mycite[author]{Ringelmann-1913} ne permettent pas de déterminer si la perte d'efficacité est liée à un effort individuel plus faible ou à un manque de coordination entre les membres du groupe \mycite{Steiner-1972}.
				Entre temps, \mycite[author]{Latane-1979} ont recréé l'expérience proposée par \mycite[author]{Ringelmann-1913} en modifiant le protocole expérimental afin de réduire les erreurs de mesure liées aux \myglos*{glo-ConflitDeCoordination}.
				Bien que les résultats varient légèrement, les conclusions sont les mêmes que celles avancées précédemment.
				Les travaux de \mycite[author]{Kerr-1981} permettent d'apporter des précisions pour définir les raisons de cette \myglos{glo-ParesseSociale}.
				Chaque membre effectuant strictement la même tâche que ses partenaires, il considère que le travail sera effectué par les autres et qu'il n'a pas besoin de s'investir autant que s'il était seul.

				Par la suite, les travaux de recherche ont tenté de modéliser l'évolution de la \myglos{glo-ParesseSociale} en fonction du nombre de participants.
				\mycite[author]{Ingham-1974} puis \mycite[author]{Karau-1993} montrent que l'ajout d'un premier puis d'un deuxième collaborateur a des conséquences importantes sur la \myglos{glo-ParesseSociale} mais que l'ajout de collaborateurs supplémentaires provoque une baisse plus modérée.
				\mycite{Suleiman-2008} proposent une synthèse de la collaboration, illustrée sur la \myref{fig-sota-SyntheseDesEffetsDeLaCollaborationSelonSuleiman}.

				\begin{myfigure}
					\psset{xunit=0.125\textwidth,yunit=0.004\textwidth}
					\begin{myps}(-0.75,-25)(7.25,110)
						\pspolygon*[linecolor=myblue!25](1,0)(1,100)(3,65)(6,50)(6,0)
						\pspolygon*[linecolor=mygreen!25](1,100)(3,65)(6,50)(6,75)(3,85)
						\pspolygon*[linecolor=myred!25](1,100)(3,85)(6,75)(6,100)
						\pspolygon[linewidth=0pt,linestyle=none,fillstyle=gradient,gradbegin=white,gradend=myblue!25,gradangle=-90,gradmidpoint=1](6,0)(6,50)(6.2,49)(6.2,0)
						\pspolygon[linewidth=0pt,linestyle=none,fillstyle=gradient,gradbegin=white,gradend=mygreen!25,gradangle=-90,gradmidpoint=1](6,50)(6,75)(6.2,74.33333)(6.2,49)
						\pspolygon[linewidth=0pt,linestyle=none,fillstyle=gradient,gradbegin=white,gradend=myred!25,gradangle=-90,gradmidpoint=1](6,75)(6,100)(6.2,100)(6.2,74.33333)
						\psline(1,100)(3,65)(6,50)
						\psdots(1,100)(3,65)(6,50)
						\psline(1,100)(3,85)(6,75)
						\psdots(1,100)(3,85)(6,75)
						\psline(1,100)(3,100)(6,100)
						\psdots(1,100)(3,100)(6,100)
						\psset{framearc=0.1,framesep=2pt}
						\newcommand{\myframebox}[3]{\psrotate(6,#1){#2}{\uput[-135](6,#1){\psframebox*[linecolor=black!10,framearc=0.1]{\footnotesize\vphantom{pÉ}#3}}}}
						\myframebox{50}{0}{Réponse obtenue}
						\myframebox{75}{0}{Problème de coordination}
						\myframebox{100}{0}{Effort individuel réduit}
						\psaxes[Dx=1,Dy=25]{->}(0,0)(0,0)(6.5,110)
						\uput{5ex}[-90](3,0){\textit{taille du groupe}}
						\uput{2.25em}[180](0,50){\rotateleft{\textit{réponse}}}
						\psset{braceWidthInner=5\pslinewidth,braceWidthOuter=5\pslinewidth,nodesepA=5pt}
						\psbrace[ref=lC](6.2,74.16667)(6.2,100){\footnotesize\parbox{1.25cm}{\myglosnl{glo-ParesseSociale}}}
					\end{myps}
					\mycaption[fig-sota-SyntheseDesEffetsDeLaCollaborationSelonSuleiman]{Synthèse des effets de la collaboration selon \mycite[author]{Suleiman-2008}}
				\end{myfigure}

				Cependant, différentes études proposent des solutions pour contrer les effets de la \myglos{glo-ParesseSociale}.
				Deux propositions ressortent particulièrement : l'identification et l'auto-évaluation.
				Selon \mycite[author]{Kerr-1981}, l'identification a pour objectif de donner un rôle défini et unique à chaque participants.
				Cette identification permet de responsabiliser chaque participant.
				\mycite[author]{Latane-1979} en avaient déjà eu l'intuition puisqu'ils proposent une telle solution dans leurs perspectives.
				\mycite[author]{Karau-1993} repris par \mycite[author]{Kraut-2003} mettent en avant cette idée d'identification à la suite d'expérimentations.

				D'ailleurs, \mycite[author]{Karau-1993} ont également proposé le principe d'auto-évaluation pour neutraliser les effets de la \myglos{glo-ParesseSociale}.
				De cette manière, les utilisateurs peuvent s'auto-évaluer et évaluer leurs collaborateurs ce qui génère une pression sociale pour favoriser la \myglos{glo-FacilitationSociale} \mycite{Szymanski-1987,Harkins-1988}.
				Dans une étude dans laquelle il compare entre autre une configuration de travail collaboratif distribuée avec une configuration colocalisée, \mycite[author]{Chidambaram-2005} montrent qu'un travail colocalisé permet de faciliter l'auto-évaluation et augmente cette pression sociale.
				En effet, le travail colocalisé permet d'avoir une meilleur perception des actions de chacun des membres et ainsi d'évaluer de façon plus précise le travail des collaborateurs.

				Cependant, \mycite[author]{Chidambaram-2005} constatent un effet parallèle : certains membres du groupe donne simplement l'impression qu'ils travaillent sans être réellement effectif.
				On trouve également les récents travaux de \mycite[author]{Buisine-2011} qui montrent que l'occupation spatiale de l'environnement est importante : un accès limité à l'espace de travail provoque de la \myglos{glo-ParesseSociale}.
				En effet, si l'accès à une ressource externe ne permet que la présence d'un seul membre du groupe, les autres membres ne pouvant pas l'utiliser peuvent générer le phénomène de \myglos{glo-ParesseSociale}.
				Il est donc préférable d'avoir un accès équitable à l'espace de travail pour tous les membres du groupe (une table ronde par exemple).

				La \myglos{glo-ParesseSociale} est un phénomène qui apporte des contraintes importantes au travail de groupe.
				Cependant, fournir une identification des rôles de chacun est une première réponse à ce problème, l'auto-évaluation étant la seconde.
				Comme l'a montré \mycite[author]{Chidambaram-2005}, la collaboration colocalisée est une solution appropriée pour survenir aux besoins de l'auto-évaluation.
				Cependant, la mise en place d'une plateforme collaborative doit également permettre à chaque utilisateur de s'identifier au sein du processus collaboratif.
			\end{mysubsection}
		\end{mysection}
		\begin{mysection}[sec-sota-CollaborationEnEnvironnementVirtuel]{Collaboration en environnement virtuel}
			Dans la section précédente, nous avons identifié deux types d'éléments qui vont interagir au sein d'un système collaboratif :
			\begin{description}
				\item[Les participants] qui sont les individus qui vont contribuer directement ou indirectement à la réalisation d'une tâche en collaboration;
				\item[Les artefacts] qui sont les différents composants ou matériels avec lesquels ou sur lesquels la collaboration peut s'appuyer.
			\end{description}
			Dans cette section, nous allons décrire les interactions qui ont lieu entre les différents acteurs d'un système collaboratif.
			\begin{mysubsection}[sse-sota-MecanismesDeLaCommunicationDeGroupe]{Mécanismes de la communication de groupe}
				Certains mécanismes de la communication sont relativement implicites et difficilement observables.
				D'autres sont plus concrets mais apportent également leurs contraintes et leurs limites.
				Cette section a pour but de décrire les mécanismes que sont la conscience périphérique et le \mygrounding.
				\begin{mysubsubsection}[sss-sota-LaConscienceDeGroupe]{La conscience de groupe}
					La conscience de groupe, également nommée \myawareness, a été définie par \mycite[author]{Dourish-1992}.
					\begin{myquote}[english]
						\it Awareness is an understanding of the activities of others, which provides a context for your own activity.
					\end{myquote}
					traduit en français par \mycite[author]{Betbeder-2004}
					\begin{myquote}[french]
						[La conscience de groupe est une] compréhension des activités des autres, qui permet de donner un contexte à sa propre activité.
					\end{myquote}

					La conscience des autres partenaires peut concerner aussi bien la position spatiale (réelle ou virtuelle), des informations durables sur un collaborateur (âge, sexe, spécialité, culture, \myetc) ou encore l'action réalisée par celui-ci \mycite{Cockburn-1999}.
					Cette conscience est bien souvent inconsciente et de nombreux évènements peuvent survenir autour d'un participant sans qu'il n'y prête attention.
					Parfois, ce manque d'attention permet un gain de temps dans l'achèvement de la tâche (par exemple, si le collaborateur éternue, cela n'a pas d'influence sur la réalisation de la tâche).
					Cependant, le fait de savoir qu'un collaborateur s'est absenté (pour aller aux toilettes par exemple) est beaucoup plus pertinent car il n'est alors plus question de compter sur son aide pendant quelques minutes.

					\mycite[author]{Cahour-2005} identifient trois avantages d'avoir une bonne conscience du groupe en étudiant la coopération entre serveurs et cuisiniers d'un café-restaurant :
					\begin{itemize}
						\item l'économie collective de déplacements et actions, grâce à une vision périphérique des mouvements des collègues;
						\item le besoin de communications rapides et non intrusives en écoutant sur un mode périphérique les messages oraux adressés dans le brouhaha du café;
						\item le besoin d'éviter des collisions dans un petit espace partagé grâce à des modalités visuelles et kinesthésiques.
					\end{itemize}

					Cependant, la conscience de groupe est constituée de plusieurs aspects que les sociologues ont cherché à segmenter.
					\mycite[author]{Gutwin-1996} propose les quatre catégories de conscience suivantes\footnote{Les catégories en anglais sont les suivantes : \textit{informal awareness}, \textit{social awareness}, \textit{group-structural awareness} et \textit{workspace awareness}.} :
					\begin{description}
						\item[Conscience informelle] Ce sont les informations générales sur les collègues, le type d'informations que les gens connaissent lorsqu'ils travaillent dans le même bureau (âge, origines culturelles, situation familiale, \myetc);
						\item[Conscience sociale] Ce sont les informations que chacun établi dans n'importe quelle relation sociale comme l'état émotionnel de l'interlocuteur, son niveau d'attention ou encore son niveau d'intérêt;
						\item[Conscience de la structure du groupe] C'est la connaissance de la hiérarchie du groupe, les rôles et responsabilités de chacun, leurs assignation sur une tâche ou leur statut;
						\item[Conscience de l'espace de travail] C'est la conscience des actions et interactions des autre membres du groupe sur et avec l'espace de travail et les artefacts.
					\end{description}

					Parmi tous ces types de conscience, seule la conscience de l'espace de travail nécessite une mise à jour en temps-réel des informations environnantes par les membres du groupe, surtout dans le cadre d'une collaboration synchrone.
					Chaque collaborateur émet des informations (action en cours, position du curseur, \myetc) de manière consciente ou inconsciente.
					Pour les informations émises de manière inconsciente, elles seront plus ou moins visibles des autres collaborateurs en fonction de leur attention.
					Cependant, lorsqu'un utilisateur souhaite émettre une information, il doit également être sûr que l'information a été comprise correctement.

					Cette notion d'inter-référencement, récemment mise en évidence à travers la plateforme \myAMMPVis \mycite{Chastine-2005,Chastine-2006}, a été définie par \mycite[author]{Chastine-2007} de la manière suivante
					\begin{myquote}[english]
						\it The ability for one participant to refer to a set of artifacts in the environment, and for that reference to be correctly interpreted by others.
					\end{myquote}
					qu'on traduira par
					\begin{myquote}[french]
						La capacité pour un des participants de désigner un ensemble d'artefacts dans l'environnement et que cette désignation soit correctement comprise par les autres collaborateurs.
					\end{myquote}

					Au travers de ces travaux de thèse, \mycite[author]{Chastine-2007} aborde en détail la notion d'inter-référencement.
					Un utilisateur initie une désignation d'un élément de l'environnement puis un (ou plusieurs) utilisateur(s) reçoit(reçoivent) cette désignation.
					Cette désignation est caractérisée par plusieurs paramètres :
					\begin{itemize}
						\item une technique de sélection;
						\item un groupe d'éléments sélectionnés;
						\item une technique de représentation;
						\item une relation entre l'initiateur et l'élément sélectionné;
						\item une relation entre le (ou les) receveur(s) et l'élément sélectionné;
						\item un contexte entre l'initiateur et le (ou les) receveur(s);
						\item un moyen d'accuser réception du référencement (optionnel).
					\end{itemize}

					Afin de résoudre cette problématique, \mycite[author]{Chastine-2007a} se proposent, dans un contexte de collaboration distante, d'utiliser des techniques de réalité augmentée pour désigner les éléments \myref*{fig-sota-InterReferencementVisuelProposeParChastineALAideDeTechniquesDeRealiteAugmentee}.
					Dans une étude l'année suivante, \mycite[author]{Chastine-2008} montrent que les techniques qu'ils proposent pour améliorer l'inter-référencement permettent de limiter les ambiguïtés et d'améliorer les performances globales du groupe.

					\begin{myfigure}
						\myimage{sota-Chastine}
						\mycaption[fig-sota-InterReferencementVisuelProposeParChastineALAideDeTechniquesDeRealiteAugmentee]{Inter-référencement visuel proposé par \mycite[author]{Chastine-2007a} à l'aide de techniques de réalité augmentée}
					\end{myfigure}

					L'inter-référencement fait partie du processus d'échange et de communication au sein d'une collaboration synchrone (distante ou colocalisée).
					Afin d'effectuer des désignations à un collaborateur, il est nécessaire de savoir si le collaborateur est attentif.
					Il est également nécessaire d'adapter la manière de procéder en fonction des connaissances et des compétences de ce collaborateur.
					La conscience de groupe prend donc une place importante dans ce processus d'inter-référencement.
				\end{mysubsubsection}
				\begin{mysubsubsection}[sss-sota-LeGrounding]{Le \mygrounding}
					Cette notion, mise en évidence par \mycite[author]{Clark-1989}, est nécessaire à la collaboration.
					\mycite[author]{Clark-1991} expliquent ce besoin de \mygrounding par la définition suivante.
					\begin{myquote}[english]
						\it [A group] cannot even begin to coordinate on content without assuming a vast amount of shared information or common ground -- that is, mutual knowledge, mutual beliefs, and mutual assumptions.
					\end{myquote}
					qu'on pourrait traduire par
					\begin{myquote}[french]
						Un groupe ne peut pas commencer à se coordonner sur une tâche sans supposer une quantité importante d'informations partagées ou d'une base commune -- c'est-à-dire des connaissances mutuelles, des convictions communes et des hypothèses communes.
					\end{myquote}

					Le \mygrounding est nécessaire à la transmission d'informations.
					En effet, un locuteur qui souhaite transmettre une information doit savoir quel jargon il peut utiliser avec son interlocuteur; il faut également que les deux interlocuteurs aient accès aux mêmes données concernant l'environnement sur lequel ils travaillent.
					\mycite[author]{Dillenbourg-1996} étudient ce partage d'informations sur une enquête fictive à propos d'un meurtre.
					Il constate que chaque membre du groupe se construit sa propre représentation de l'environnement dans lequel il se trouve et évolue en fonction des informations que chacun trouve et de sa culture existante \mycite{Baker-1999}.

					Lors d'un travail de collaboration, chaque membre possède déjà sa propre base de connaissances qui peut être complémentaire ou recouvrir partiellement celle des autres membres.
					Cependant, durant la réalisation de la tâche, une base de connaissance commune va se créer.
					\mycite[author]{Hertzum-2008} lie les processus de recherche d'informations et de \mygrounding.
					En effet, la recherche d'informations mène les membres du groupe à découvrir l'environnement dans lequel ils évoluent ensemble pour étoffer leur base de connaissance commune.
					Cependant, en observant le travail d'un hôpital où plusieurs médecins sont à la recherches de symptômes pour diagnostiquer un patient et fournir un traitement approprié, \mycite[author]{Hertzum-2010} s'est rendu compte que chaque médecin est capable individuellement de découvrir des symptômes mais que le manque d'échanges entre les différents médecins ne permettait pas toujours d'établir le bon diagnostic.

					Afin de constituer correctement une base de connaissance commune, les membres du groupe doivent communiquer.
					En particulier, ils doivent continuellement être sûr que les éléments dont ils parlent sont bien les mêmes afin de transmettre l'information.
					Fournir des outils appropriés pour l'inter-référencement devrait permettre d'améliorer la constitution de cette base commune de connaissance.
				\end{mysubsubsection}
			\end{mysubsection}
			\begin{mysubsection}[sse-sota-CommunicationEnEnvironnementVirtuel]{Communication en environnement virtuel}
				La communication en environnement virtuel est constituée de différentes composantes.
				Elle inclut la communication entre les collaborateurs et la communication avec ou à travers l'environnement.
				\mycite[author]{Dix-1997} a proposé de classifier la communication en quatre catégories différentes.
				\begin{mysubsubsection}[sss-sota-CommunicationEnEnvironnementVirtuel-CommunicationDirecte]{Communication directe}
					C'est le moyen de communication le plus naturel.
					La communication se fait de manière orale ou gestuelle principalement.
					Elle est conscient la plupart du temps mais peut contenir une part de communication inconsciente d'après \mycite[author]{Gutwin-2000}.
				\end{mysubsubsection}
				\begin{mysubsubsection}[sss-sota-CommunicationEnEnvironnementVirtuel-ControleEtRetourSensoriel]{Contrôle et retour sensoriel}
					C'est l'interaction entre un participant et un artefact.
					Cette interaction est bidirectionnelle car le participant agit sur l'artefact qui produit un retour sur un ou plusieurs des modalités du participants (vue, ouïe, toucher, \myetc).
				\end{mysubsubsection}
				\begin{mysubsubsection}[sss-sota-CommunicationEnEnvironnementVirtuel-Feedthrough]{\myFeedthrough}
					C'est un canal de communication indirect où les participants communiquent entre eux par l'intermédiaire des artefacts \mycite{Dix-2004}.
					Les actions effectuées par un participant à l'aide et sur l'artefact modifient l'environnement ce qui donne des informations sur les actions et les intentions au collaborateur.
					\mycite[author]{Junuzovic-2009} utilisent même le terme \myfeedthrough pour une communication indirecte avec un participant en intermédiaire.
				\end{mysubsubsection}
				\begin{mysubsubsection}[sss-sota-CommunicationEnEnvironnementVirtuel-Grounding]{\myGrounding}
					Les participants doivent posséder au moins un niveau de compréhension commun (même langue parlée, même base culturelle, même jargon, \myetc) afin de se comprendre \mycite{Dix-2004}.
					Cette aspect de la communication est inconscient mais est nécessaire pour la collaboration.
				\end{mysubsubsection}
			\end{mysubsection}
			\begin{mysubsection}[sse-sota-StrategiesDeCollaboration]{Stratégies de collaboration}
				Avec les différents canaux de communication décrits dans la \myref{sse-sota-CommunicationEnEnvironnementVirtuel}, conscients pour certains (communication directe, contrôle et retour sensoriel) et inconscients pour d'autres (\myfeedthrough et \mygrounding), il est possible de définir trois scénarios de collaboration différents basés sur les travaux de \mycite[author]{Dix-2004} et de \mycite[author]{Grasset-2004}.
				\begin{mysubsubsection}[sss-sota-CollaborationAPlusieursUtilisateurs]{Collaboration à plusieurs utilisateurs}
					Illustré sur la \myref{fig-sota-CollaborationAPlusieursUtilisateurs}, c'est l'aspect le plus naturel de la distribution cognitive des charges que nous avons déjà découvert dans la \myref{sse-sota-LaDistributionCognitiveDesChargesDeTravail} : on souhaite diviser la tâche à réaliser entre plusieurs utilisateurs car elle trop complexe ou trop fastidieuse.
					Chaque participant peut accéder librement aux artefacts et possède la possibilité d'analyser l'environnement par l'observation ou par les retours sensoriels des artefacts.

					\begin{myfigure}
						\let\oldarraystretch\arraystretch
						\def\arraystretch{0.75}
						\psset{unit=0.083333333\textwidth,linewidth=1pt}
						\begin{myps}(-6,-1.5)(6,5.75)
							\mynode(-4,4)[alice]{\Large Sujet~\mynum{1}}
							\mynode(4,4)[bob]{\Large Sujet~\mynum{2}}
							\rput(0,0){\circlenode[style=nodestyle]{artefact}{\Large Artefact}}
							\ncdiag[angleA=-45,angleB=-135,armA=4.25,armB=4.25,linestyle=dashed,linearc=.25]{<->}{alice}{bob}
							\ncput*[nrot=:U,npos=0.5]{\myFeedthrough}
							\ncput*[nrot=:U,npos=2.5]{\myFeedthrough}
							\psclip{%
								\rput(0,0){\circlenode[style=nodestyle,shadow=false]{artefact}{\Large Artefact}}
							}
							\ncdiag[angleA=-45,angleB=-135,armA=4.25,armB=4.25,linestyle=dashed,linearc=.25,linecolor=black!25]{<->}{alice}{bob}
							\endpsclip
							\rput(0,0){\circlenode[style=nodestyle,fillstyle=none,shadow=false]{artefact}{\Large Artefact}}
							\ncangle[angleA=-90,angleB=180,linearc=.25]{<->}{alice}{artefact}
							\ncput*[npos=0.67]{\begin{tabular}{@{}c@{}}Contrôle\\et retour\\sensoriel\end{tabular}}
							\ncangle[angleA=-90,angleB=0,linearc=.25]{<->}{bob}{artefact}
							\ncput*[npos=0.67]{\begin{tabular}{@{}c@{}}Contrôle\\et retour\\sensoriel\end{tabular}}
							\ncline[angleA=0,angleB=180]{<->}{alice}{bob}
							\ncput*{\begin{tabular}{@{}c@{}}Communication\\directe\end{tabular}}
							\ncangle[angleA=90,angleB=90,arm=1,linearc=.25,linestyle=dashed]{<->}{alice}{bob}
							\ncput*[npos=1.5]{\myGrounding}
						\end{myps}
						\mycaption[fig-sota-CollaborationAPlusieursUtilisateurs]{Collaboration à plusieurs utilisateurs}
						\let\arraystretch\oldarraystretch
					\end{myfigure}

					Ce scénario permet une collaboration avec un même niveau d'interaction et d'accès aux ressources entre les collaborateurs ce qui peut faciliter la communication.
					Cependant, nous avons vu dans la \myref{sse-sota-LaParesseSociale} sur la \myglos{glo-ParesseSociale} qu'une identification des rôles était préférable.
					Ce type de scénario aura tendance à favoriser la \myglos{glo-ParesseSociale} en proposant des rôles strictement identiques aux utilisateurs.

					De plus, on notera que tous les participants accèdent aux mêmes artefacts.
					Si deux collaborateurs accèdent, modifient ou manipulent en même temps un même artefact, cela peut poser des problèmes de cohérence de l'environnement virtuel ou de sécurité de l'information comme l'explique \mycite[author]{Dewan-1998}.
					\mycite[author]{Dewan-1998} énoncent une liste de conditions à respecter (automatisation, généralisation, médiation totale, privilège minimum, facilité d'utilisation, efficacité) pour limiter les problèmes de concurrences d'accès aux codes sources dans le développement de programmes informatiques.
					Quelques années plus tard, l'outil de développement informatique collaboratif \mycite[author]{SVN-2011} voit le jour.
					\mycite[author]{Grasset-2002} sont également confrontés à ce problème d'accès aux ressources sur sa plateforme multi-utilisateurs \myMARE et propose un système de contrôle d'accès aux informations.
					Toutefois, \mycite[author]{Buisine-2011} soulignent qu'une possibilité d'accès aux artefacts équitablement distribuée entre les utilisateurs permet de réduire la \myglos{glo-ParesseSociale}.
					En effet, si un utilisateur ne parvient pas à accéder aux artefacts, il va s'isoler.
				\end{mysubsubsection}
				\begin{mysubsubsection}[sss-sota-CollaborationAPlusieursExperts]{Collaboration à plusieurs experts}
					Le scénario à plusieurs experts se distingue du scénario à plusieurs utilisateurs par la compétence des membres du groupe : les experts proviennent de spécialités différentes.
					La médecine est un domaine qui s'adapte très bien à ce type de scénario étant donné le nombre important de spécialités différentes \mycite{Althoff-2007a,Althoff-2007b}.
					Les experts possédant des connaissances différentes, il est nécessaire pour eux d'avoir des artefacts adaptés aux besoins de chacun \myref*{fig-sota-CollaborationAPlusieursExperts}.
					Si on prend l'exemple du \myglos{glo-DockingMoleculaire}, un biologiste n'aura pas les mêmes besoins en information qu'un chimiste et leurs actions sur la tâche à réaliser seront également de natures différentes.
					Les artefacts se synchronisent entre eux afin de conserver un environnement virtuel cohérent pour l'ensemble des experts.

					\begin{myfigure}
						\let\oldarraystretch\arraystretch
						\def\arraystretch{0.75}
						\psset{unit=0.083333333\textwidth,linewidth=1pt}
						\begin{myps}(-6,-1.5)(6,5.75)
							\mynode(-4,4)[alice]{\begin{tabular}{c}\Large Expert~\mynum{1}\\Spécialité~\textsc{a}\end{tabular}}
							\mynode(4,4)[bob]{\begin{tabular}{c}\Large Expert~\mynum{2}\\Spécialité~\textsc{b}\end{tabular}}
							\rput(-4,0){\circlenode[style=nodestyle]{artefact1}{\Large Artefact~\mynum{1}}}
							\rput(4,0){\circlenode[style=nodestyle]{artefact2}{\Large Artefact~\mynum{2}}}
							\ncdiag[offset=0.5,angleA=-90,angleB=-90,armA=3,armB=3,linestyle=dashed,linearc=.25]{<->}{alice}{bob}
							\rput(-4,0){\circlenode[style=nodestyle,shadow=false]{artefact1}{\Large Artefact~\mynum{1}}}
							\rput(4,0){\circlenode[style=nodestyle,shadow=false]{artefact2}{\Large Artefact~\mynum{2}}}
							\ncput*[nrot=:U,npos=1.5]{\myFeedthrough}
							\psclip{%
								\rput(-4,0){\circlenode[linestyle=none,shadow=false]{artefact}{\Large Artefact~\mynum{1}}}
							}
							\ncdiag[offset=0.5,angleA=-90,angleB=-90,armA=3,armB=3,linestyle=dashed,linearc=.25,linecolor=black!25]{<->}{alice}{bob}
							\endpsclip
							\psclip{%
								\rput(4,0){\circlenode[linestyle=none,shadow=false]{artefact}{\Large Artefact~\mynum{2}}}
							}
							\ncdiag[offset=0.5,angleA=-90,angleB=-90,armA=3,armB=3,linestyle=dashed,linearc=.25,linecolor=black!25]{<->}{alice}{bob}
							\endpsclip
							\rput(-4,0){\circlenode[style=nodestyle,fillstyle=none,shadow=false]{artefact1}{\Large Artefact~\mynum{1}}}
							\rput(4,0){\circlenode[style=nodestyle,fillstyle=none,shadow=false]{artefact2}{\Large Artefact~\mynum{2}}}
							\ncline[offset=-0.5,angleA=-90,angleB=90,linearc=.25,nodesepB=-3pt]{<->}{alice}{artefact1}
							\ncput*[npos=0.5]{\begin{tabular}{@{}c@{}}Contrôle\\et retour\\sensoriel\end{tabular}}
							\ncline[offset=0.5,angleA=-90,angleB=90,linearc=.25,nodesepB=-3pt]{<->}{bob}{artefact2}
							\ncput*[npos=0.5]{\begin{tabular}{@{}c@{}}Contrôle\\et retour\\sensoriel\end{tabular}}
							\ncline[angleA=0,angleB=180]{<->}{alice}{bob}
							\ncput*{\begin{tabular}{@{}c@{}}Communication\\directe\end{tabular}}
							\ncangle[angleA=90,angleB=90,arm=1,linearc=.25,linestyle=dashed]{<->}{alice}{bob}
							\ncput*[npos=1.5]{\myGrounding}
							\ncline[angleA=0,angleB=180]{<->}{artefact1}{artefact2}
							\ncput*[npos=0.5]{Synchronization}
						\end{myps}
						\mycaption[fig-sota-CollaborationAPlusieursExperts]{Collaboration à plusieurs experts}
						\let\arraystretch\oldarraystretch
					\end{myfigure}

					Certaines tâches sont très adaptées à ce type de collaboration.
					Elle permet une identification des rôles de chacun et limite ainsi la \myglos{glo-ParesseSociale} : chaque utilisateur ne peut se fier qu'à lui-même pour réaliser la tâche.
					De plus, l'accès aux artefacts devient simplifié puisqu'il n'y a plus d'accès concurrent à un même artefact par deux utilisateurs.
					Cependant, une synchronisation entre les artefacts est nécessaire afin de conserver une cohérence de l'environnement virtuel pour l'ensemble des utilisateurs ce qui peut poser quelques difficultés techniques.
					\mycite[author]{Barbic-2007} ainsi que \mycite[author]{Gautier-2008} abordent ces problèmes de synchronisation sur des simulations d'objets déformables.

					Dans les scénarios à plusieurs experts, les bases de connaissances sont plutôt complémentaires.
					Il est nécessaire d'identifier les connaissances communes selon \mycite[author]{Bach-2010}, notion déjà identifiée comme le \mygrounding par \mycite[author]{Clark-1991} \myref*{sss-sota-LeGrounding}.
					Par exemple, le jargon utilisé entre deux experts de spécialités différentes peut devenir une importante contrainte à la communication directe.
				\end{mysubsubsection}
				\begin{mysubsubsection}[sss-sota-CollaborationParManipulateurEtObservateur]{Collaboration par manipulateur et observateur}
					Dans les deux scénarios précédents, il est difficile de synchroniser les actions des différents utilisateurs.
					Dans le scénario à plusieurs utilisateurs, c'est l'accès concurrent aux artefacts qui pose problème; dans le scénario à plusieurs experts, c'est la synchronisation entre les artefacts.
					Afin de passer outre ces contraintes, le scénario par manipulateur et observateur propose un accès aux artefacts exclusif à un seul utilisateur du groupe, les autres n'ayant que la possibilité d'interagir directement avec cet utilisateur \myref*{fig-sota-CollaborationParManipulateurEtObservateur}.
					L'observateur agit alors sur les artefacts par l'intermédiaire du manipulateur.

					\begin{myfigure}
						\let\oldarraystretch\arraystretch
						\def\arraystretch{0.75}
						\psset{unit=0.083333333\textwidth,linewidth=1pt}
						\begin{myps}(-6,-1.5)(6,5.75)
							\mynode(-4,4)[alice]{\Large Sujet~\mynum{1}}
							\mynode(4,4)[bob]{\Large Sujet~\mynum{2}}
							\rput(-4,0){\circlenode[style=nodestyle]{artefact}{\Large Artefact}}
							\ncdiag[offset=0.5,angleA=-90,angleB=-90,armA=3.5,armB=0,linestyle=dashed,linearc=.25]{->}{alice}{bob}
							\ncput*[nrot=:U,npos=1.5]{\emph{Feedthrough}}
							\psclip{%
								\rput(-4,0){\circlenode[style=nodestyle,shadow=false]{artefact}{\Large Artefact}}
							}
							\ncdiag[offset=0.5,angleA=-90,angleB=-90,armA=3.5,armB=0,linestyle=dashed,linearc=.25,linecolor=black!25]{->}{alice}{bob}
							\endpsclip
							\rput(-4,0){\circlenode[style=nodestyle,fillstyle=none,shadow=false]{artefact}{\Large Artefact}}
							\ncline[offset=-0.5,angleA=-90,angleB=90,linearc=.25,nodesepB=-3pt]{<->}{alice}{artefact}
							\ncput*[npos=0.5]{\begin{tabular}{@{}c@{}}Contrôle\\et retour\\sensoriel\end{tabular}}
							\ncangle[angleA=-90,angleB=0,linearc=.25]{<-}{bob}{artefact}
							\ncput*[npos=0.5]{\begin{tabular}{@{}c@{}}Retour\\sensoriel\end{tabular}}
							\ncline[angleA=0,angleB=180]{<->}{alice}{bob}
							\ncput*{\begin{tabular}{@{}c@{}}Communication\\directe\end{tabular}}
							\ncangle[angleA=90,angleB=90,arm=1,linearc=.25,linestyle=dashed]{<->}{alice}{bob}
							\ncput*[npos=1.5]{\myGrounding}
						\end{myps}
						\mycaption[fig-sota-CollaborationParManipulateurEtObservateur]{Collaboration par manipulateur et observateur}
						\let\arraystretch\oldarraystretch
					\end{myfigure}

					Ce scénario permet de s'affranchir des contraintes d'accès concurrent aux artefacts ou de synchronisation entre les artefacts.
					Les observateurs possèdent également une charge de travail cognitive beaucoup plus faible et peuvent se consacrer de manière plus importante à l'analyse des données disponibles et à la proposition de solutions pertinentes.

					Cependant, l'activité de l'observateur le mène à désigner des objectifs au manipulateur ce qui augmente le besoin de communication.
					Cette action renvoie aux travaux de \mycite[author]{Chastine-2007} sur l'inter-référencement \myref*{sss-sota-LaConscienceDeGroupe}.
					De plus, un travail récent cherchant à caractériser \og l'intelligence d'un groupe \fg a conclu que l'un des facteurs d'intelligence d'un groupe était la répartition équitable des charges de travails entre les membres du groupe \mycite{Woolley-2010}.
					Dans ce scénario, l'asymétrie entre les différents rôles (manipulateur et observateur) semble favoriser un déséquilibre important des charges de travail ce qui n'est pas idéal pour la collaboration.
				\end{mysubsubsection}
			\end{mysubsection}
		\end{mysection}
		\begin{mysection}[sec-sota-Conclusion]{Conclusion}
			À travers ce chapitre, nous avons découvert la complexité que représente la tâche de \myglos{glo-DockingMoleculaire}.
			En proposant des plateformes collaboratives, les scientifiques ont permis d'améliorer les résultats obtenus mais la collaboration reste relativement faiblement couplée (messages textuels ou verbaux, vidéo-conférence, \myetc).
			Pourtant, de récents développements nous permettent enfin d'envisager d'immerger les scientifiques au sein d'une simulation moléculaire en temps-réel.
			C'est donc afin de répondre à la complexité du \myglos{glo-DockingMoleculaire} que nous souhaitons proposer une plateforme collaborative synchrone et colocalisée pour la déformation moléculaire interactive et temps-réel.

			Dans ce chapitre, nous avons également mis en évidence les avantages et les contraintes de la collaboration.
			La distribution cognitive des charges de travail permet d'apporter une réponse pertinente à la réalisation d'un problème extrêmement complexe tel que le \myglos{glo-DockingMoleculaire}.
			Cependant, de nombreux facteurs (\myglos{glo-FacilitationSociale}, \myglos{glo-ParesseSociale}, \mygrounding, conscience de groupe, \myetc) interviennent et influent sur la communication et les interactions entre les acteurs de la collaboration.
			C'est la raison qui nous encourage à envisager la modalité haptique comme un support supplémentaire adéquat pour améliorer les communications et les interactions entre les collaborateurs.

			Cependant, un tel contexte de travail (collaboration synchrone et colocalisée, manipulation moléculaire temps-réel, interaction avec retour haptique) n'a encore jamais été étudié.
			Aucune donnée ne permet d'infirmer ou de confirmer les théories sur la collaboration dans ce contexte bien que des hypothèses puisse être formulées.

			Afin d'étudier ce contexte de travail, nous avons souhaiter segmenter les étapes de la déformation moléculaire en nous basant sur les travaux de \mycite[author]{Bowman-1999}.
			Il identifie quatre tâches élémentaires pour segmenter une interaction en environnement virtuel, nommées \myacro*{acr-PCV} par \mycite[author]{Fuchs-2006a}.
			Les quatre \myacro*{acr-PCV} pour la manipulation moléculaire, illustrée dans la \myref{fig-sota-ProcessusDeDeformationMoleculaireEnQuatreEtapes}, sont expliquées ci-dessous :
			\begin{description}
				\item[Exploration] Cette tâche concerne la recherche et l'identification d'une cible (atome, \myglos{glo-Residu}, structure secondaire, \myetc) en fonction de critères multiples (articulations, bilan énergétique, régions hydrophobique, \myetc).
				\item[Sélection] Une fois la cible trouvée, la tâche consiste à accéder puis à sélectionner la cible par l'intermédiaire d'un périphérique d'entrée (une souris, une interface haptique, \myetc).
				\item[Déformation] Cette étape consiste à manipuler la cible précédemment sélectionnée pour déformer la structure moléculaire au niveau atomique, intra-moléculaire ou inter-moléculaire.
					L'objectif inhérent à cette tâche et d'atteindre l'objectif fixé (par exemple, minimiser l'énergie totale du système).
				\item[Évaluation] Cette dernière partie concerne l'évaluation du travail précédemment réalisé en observant différents indicateurs (énergie potentielle, énergie électrostatique, complémentarité des surfaces, \myetc).
					En fonction de la synthèse des résultats de cette dernière phase, un nouveau cycle pourra recommencer (recherche, sélection, déformation, évaluation, \myetc).
			\end{description}

			\begin{myfigure}
				\psset{xunit=0.2\textwidth}
				\def\mycirclenum(#1,#2)#3{%
					\uput{5em}[0](#1,#2){\pscirclebox*[fillcolor=myblue!70]{\white #3}}%
				}
				\begin{myps}(-2.5,-1)(2.5,5)
					\mynode(0,4)[Exploration]{Exploration}
					\mycirclenum(0,4){1}
					\mynode(0,3)[Selection]{Sélection}
					\mycirclenum(0,3){2}
					\mynode(0,2)[Manipulation]{Manipulation}
					\mycirclenum(0,2){3}
					\mynode(0,1)[Evaluation]{Évaluation}
					\mycirclenum(0,1){4}
					\mynode[fillstyle=solid,fillcolor=myblue!25](0,-0.5)[Objective]{Objectif atteint}
					\ncline{->}{Exploration}{Selection}
					\ncline{->}{Selection}{Manipulation}
					\ncline{->}{Manipulation}{Evaluation}
					\ncline{->}{Evaluation}{Objective}
					\ncloop[loopsize=4em,angleA=-90,angleB=90,linearc=0.05]{->}{Evaluation}{Exploration}
				\end{myps}
				\mycaption[fig-sota-ProcessusDeDeformationMoleculaireEnQuatreEtapes]{Processus de déformation moléculaire en quatre étapes}
			\end{myfigure}

			Sur la base de ces \myacro*{acr-PCV}, nous avons segmenté notre étude en deux parties distinctes; une centrée sur le processus d'exploration et de sélection \myref*{cha-RechercheCollaborativeDeResiduSurUneMolecule}; la seconde centrée sur les processus de manipulation et d'évaluation \myref*{cha-DeformationCollaborativeDeMolecule}.
			Dans un troisième temps, nous avons souhaiter augmenter la taille des groupes étudiés \myref*{cha-LaDynamiqueDeGroupe}.
		\end{mysection}
	\end{mychapter}
	\begin{mychapter}[cha-Shaddock-ManipulationCollaborativeDeMolecules]{\myShaddock\ -- Plateforme de manipulation collaborative de molécules}
		\begin{mysection}[sec-Shaddock-Introduction]{Introduction}
			Le \myref{cha-sota-EtudeBibliographique} nous a permis d'identifier des problématiques de recherche : nombre important d'atomes, multiple connaissances nécessaires, flexibilité des molécules, \myetc
			C'est autour de ces problématiques que la plateforme \myShaddock a été élaborée.

			Nous commencerons par présenter les choix de matériels et d'architecture logicielle \myref*{sec-Shaddock-ArchitectureMaterielleEtLogicielle}.
			L'ensemble des éléments de la plateforme sont organisés selon une architecture client/serveur; les raisons de ce choix sont expliquées dans la \myref{sse-Shaddock-ArchitectureLogicielleAdoptee}.

			Ensuite, la plateforme de simulation moléculaire en temps-réel est présentée \myref*{sec-Shaddock-PlateformeDeSimulationEtDeVisualisation}.
			Tout d'abord, un module de visualisation est nécessaire pour obtenir des visualisations détaillées et complètes de molécules; le module est présenté dans la \myref{sse-Shaddock-ModuleDeVisualisationMoleculaire}.
			Puis un module de simulation moléculaire est ajouté au module visualisation; plusieurs solutions sont possible et le module retenu est présenté dans la \myref{sse-Shaddock-ModuleDeSimulationMoleculaire}.
			Cependant, afin d'obtenir une simulation moléculaire interactive temps-réel, un module spécifique doit être ajouté; il sera présenté dans la \myref{sss-Shaddock-SimulationMoleculaireInteractiveEtTempsReel}.

			Le module de visualisation moléculaire utilisé propose déjà des outils permettant d'interagir avec les molécules.
			Ces outils sont présentés dans la \myref{sse-Shaddock-OutilsExistants}.
			Cependant, notre étude du travail collaboratif nous a amené à proposer des outils de manipulation avancés qui seront présentés dans la \myref{sse-Shaddock-OutilsDeManipulationAvances}.

			Les différents éléments de cette plateforme sont résumés dans deux diagrammes \myacro{acr-UML}.
			Un diagramme de déploiement \myacro{acr-UML} de la plateforme \myShaddock est présenté sur la \myref{fig-Shaddock-DiagrammeDeDeploiementUMLDeLaPlateformeShaddock}.
			L'application \myacro{acr-VMD} est détaillée dans un diagramme de composant \myacro{acr-UML} sur la \myref{fig-Shaddock-DiagrammeDeComposantUMLDuNoeudVMD}.

			\begin{myfigure}
				\psset{xunit=0.1\textwidth,yunit=0.0475\textheight}
				\psset{framearc=.1,shadow=true,blur=true}
				\begin{myps}(-5,-10)(5,10)
					\rotateleft{%
						\psframe*[linecolor=mygreen!2,shadow=false](-7.4,-7)(-2.6,7)
						\psframe*[linecolor=myblue!2,shadow=false](-2.4,-7)(2.4,7)
						\psframe*[linecolor=myred!2,shadow=false](2.6,-7)(7.4,7)
						\uput[-90](-5,7){\LARGE\textcolor{mygreen!25}{Simulation}}
						\uput[-90](0,7){\LARGE\textcolor{myblue!25}{Visualisation}}
						\uput[-90](5,7){\LARGE\textcolor{myred!25}{Interaction}}
						\rput(0,0){%
							\myumlnode*<PCUtilisateur>[\myLinux et \myacronl-{acr-CUDA}]{Client : Nœud principal}{%
								\myumlcomponent<VMD>[application]{\myacronl-{acr-VMD}}%
							}%
						}
						\rput(-5,0){%
							\myumlnode*<ServeurNAMD>[\myLinux et \myacronl-{acr-CUDA}]{Serveur : Nœud \myacronl-{acr-NAMD}}{%
								\begin{psmatrix}[rowsep=1]%
									\myumlcomponent<NAMD>[programme]{{\mysource{namd2}}} \\%
									\myumlcomponent<FichierSimulation>[fichier]{%
										\\[-1ex]%
											\begin{psmatrix}[rowsep=0]%
												Données de\\ simulation%
											\end{psmatrix}%
										}
									\end{psmatrix}%
								}%
							}
							\rput(5,4){%
								\myumlnode*<ServeurVRPN1>[\myLinux, \myMacOS ou \myWindows]{Serveur : Nœud \myacronl-{acr-VRPN}~\mynum{1}}{%
									\myumlcomponent<VRPN1>[programme]{{\mysource{vrpn_server}}}%
								}%
							}
							\rput(5,1){%
								\myumlnode<PHANToM1>[\myOmni]{Interface : Interface haptique~\mynum{1}}%
							}
							\rput(5,-0.5){\Huge$\vdots$}
							\rput(5,-3){%
								\myumlnode*<ServeurVRPNn>[\myLinux, \myMacOS ou \myWindows]{Serveur : Nœud \myacronl-{acr-VRPN}~$n$}{%
									\myumlcomponent<VRPNn>[programme]{{\mysource{vrpn_server}}}%
								}%
							}
							\rput(5,-6){%
								\myumlnode<PHANToMn>[\myOmni]{Interface : Interface haptique~$n$}%
							}
							\rput(0,-5){%
								\myumlnode<VideoProjecteur>[vue partagée]{Affichage : Vidéoprojecteur}
							}
							\psset{shadow=false}

							\myumlrealization[angleA=-90,angleB=90]{NAMD}{FichierSimulation}[nccurve]%
							\myumlrealization[angleA=-90,angleB=90]{VRPN1}{PHANToM1}[nccurve]%
							\myumlrealization[angleA=-90,angleB=90]{VRPNn}{PHANToMn}[nccurve]%
							\myumlinterface[angleA=-90,angleB=90,ArrowInsidePos=0.666667]{VMD}{VideoProjecteur}[nccurve]
							\nbput[npos=0.666667]{\tiny\begin{psmatrix}[rowsep=0]Transmission\\ des données\\ d'affichage\end{psmatrix}}
							\ncput*[fillcolor=myblue!2,framesep=1pt,nrot=:D,npos=0.333333]{\textcolor{black!30}{\scriptsize \myVGA}}
							\myumlinterface[angleA=0,angleB=180,offsetB=-8pt,ArrowInsidePos=0.3]{NAMD}{VMD}[nccurve]
							\naput[npos=0.25]{\tiny\begin{psmatrix}[rowsep=0]Transmission\\ des données\\ de simulation\end{psmatrix}}
							\ncput*[fillcolor=myblue!2,framesep=1pt,nrot=:U,npos=0.635]{\textcolor{black!30}{\scriptsize \myTCPIP}}
							\psframe*[linecolor=white](-2.6,-2)(-2.4,-0.06) % To hide the background of the label
							\myumlinterface[angleA=180,angleB=0,ncurvA=1.5,offsetA=8pt,ArrowInsidePos=0.5]{VRPN1}{VMD}[nccurve]
							\nbput[npos=0.4]{\tiny\begin{psmatrix}[rowsep=0]Transmission\\ des données\\ d'interaction\end{psmatrix}}
							\ncput*[fillcolor=myblue!2,framesep=1pt,nrot=:D,npos=0.6]{\textcolor{black!30}{\scriptsize \myTCPIP}}
							\myumlinterface[angleA=180,angleB=0,ncurvA=1.5,offsetA=8pt,ArrowInsidePos=0.5]{VRPNn}{VMD}[nccurve]
							\naput[npos=0.4]{\tiny\begin{psmatrix}[rowsep=0]Transmission\\ des données\\ d'interaction\end{psmatrix}}
							\ncput*[fillcolor=myblue!2,framesep=1pt,nrot=:D,npos=0.6]{\textcolor{black!30}{\scriptsize \myTCPIP}}
						}
					\end{myps}
					\mycaption[fig-Shaddock-DiagrammeDeDeploiementUMLDeLaPlateformeShaddock]{Diagramme de déploiement \myacronl-{acr-UML} de la plateforme \myShaddock}
			\end{myfigure}
			\begin{myfigure}
					\psset{unit=0.05\textwidth}
					\begin{myps}(-10,-5)(10,4)
						\rput(0,0){%
							\myumlcomponent*[framesep=10pt,framearc=0,shadow=false]<VMD>[application]{\myacronl-{acr-VMD}}{%
								\psset{framesep=5pt,framearc=.1,shadow=true,blur=true}%
								\psframebox[linestyle=none,fillstyle=none,shadow=false]{%
									\begin{psmatrix}[rowsep=1]%
										\myumlcomponent<IMD>[extension]{\myacronl-{acr-IMD}}%
										\hspace{3em}%
										\myumlcomponent<VRPNclient>[fonction]{Client \myacronl-{acr-VRPN}} \\%
										\myumlcomponent<Renderer>[fonction]{Moteur de rendus}%
									\end{psmatrix}%
								}%
							}%
						}

						\psset{fillstyle=none,shadow=false}
						\myumlrealization[angleA=90,angleB=-90]{Renderer}{IMD}[nccurve]%
						\myumlrelation[angleA=180,angleB=0,offsetA=8pt]{VRPNclient}[-135]<*>{IMD}[-45]<*>[nccurve]%
						\myumlinterface[angleA=180,angleB=180,outAngleB=0,offsetA=8pt,ArrowInside={}]{IMD}{VMD}[nccurve]
						\ncput[npos=1]{\uput[180](0,0){\tiny\begin{psmatrix}[rowsep=0]Chargement\\ des données\\ de simulation\end{psmatrix}}}
						\myumlinterface[angleA=0,angleB=0,outAngleB=180,ArrowInside={}]{VRPNclient}{VMD}[nccurve]
						\ncput[npos=1]{\uput[0](0,0){\tiny\begin{psmatrix}[rowsep=0]Communication\\ avec les outils\\ de manipulation\end{psmatrix}}}
						\myumlinterface[angleA=-90,angleB=-90,outAngleB=90,offsetA=8pt,ArrowInside={}]{Renderer}{VMD}[nccurve]
						\ncput[npos=1]{\uput[-90](0,0){\tiny\begin{psmatrix}[rowsep=0]Affichage\\ de la scène\end{psmatrix}}}
					\end{myps}
					\mycaption[fig-Shaddock-DiagrammeDeComposantUMLDuNoeudVMD]{Diagramme de composant \myacronl-{acr-UML} du nœud \myacronl-{acr-VMD}}
			\end{myfigure}
		\end{mysection}
		\begin{mysection}[sec-Shaddock-ArchitectureMaterielleEtLogicielle]{Architecture matérielle et logicielle}
			\begin{mysubsection}[sse-Shaddock-TravailCollaboratifSynchroneEtColocalise]{Travail collaboratif synchrone et colocalisé}
				\mycite[author]{Ellis-1991} a proposé une classification des différentes configurations de travail collaboratif dans le temps et dans l'espace.
				Il distingue quatre classes principales qu'on peut représenter sur un graphique \myref*{fig-Shaddock-ClassificationDesTachesCollaborativesSelonEllis}.

				\begin{myfigure}
					\psset{unit=0.083333333\textwidth}
					\begin{myps}(-1,-1)(8,4)
						\psclip{%
							\pscustom[linestyle=none]{%
								\pszigzag[coilwidth=0.5cm,coilheight=5,linearc=.5]{-}(0,2)(8,2)
								\lineto(0,2)
								\lineto(8,2)
							}
							\pscustom[linestyle=none]{%
								\pszigzag[coilwidth=0.5cm,coilheight=5,linearc=.5]{-}(4,0)(4,4)
								\lineto(0,-4)
								\lineto(4,-4)
							}
						}
						\psframe*[linecolor=myblue!30](0,0)(8,4)
						\endpsclip
						\myaxes[ticks=none,labels=none,arrows={->}](0,8){distance}(0,4){temps}
						\pszigzag[coilwidth=0.5cm,coilheight=5,linestyle=dashed,linearc=.5,linewidth=.5pt]{-}(0,2)(8,2)
						\pszigzag[coilwidth=0.5cm,coilheight=5,linestyle=dashed,linearc=.5,linewidth=.5pt]{-}(4,0)(4,4)
						\rput(2,1){\begin{tabular}{c}Collaboration\\face-à-face\end{tabular}}
						\rput(2,3){\begin{tabular}{c}Collaboration\\asynchrone\end{tabular}}
						\rput(6,1){\begin{tabular}{c}Collaboration\\synchrone\\distribuée\end{tabular}}
						\rput(6,3){\begin{tabular}{c}Collaboration\\asynchrone\\distribuée\end{tabular}}
						\uput[-90](2,0){\small colocalisé}
						\uput[-90](6,0){\small distant}
						\uput[180](0,1){\rotateleft{\small synchrone}}
						\uput[180](0,3){\rotateleft{\small asynchrone}}
					\end{myps}
					\mycaption[fig-Shaddock-ClassificationDesTachesCollaborativesSelonEllis]{Classification des tâches collaboratives selon \mycite[author]{Ellis-1991}}
				\end{myfigure}

				Nous avons choisi de nous placer dans une configuration de collaboration synchrone et colocalisée, ou collaboration face-à-face.
				En effet, afin de pouvoir faire interagir en temps-réel les utilisateurs les uns avec les autres, la synchronisation dans le temps est nécessaire.
				En ce qui concerne la collaboration dans l'espace, une collaboration pourrait être envisageable.
				Cependant, cette solution de collaboration distante n'a pas été retenue pour différentes raisons.

				Tout d'abord, la \myref{sss-sota-LaConscienceDeGroupe} évoque les problématiques liées à la conscience périphérique des partenaires dans la réalisation d'une tâche collaborative.
				Étant donné que plusieurs études ont montré que les performances collaboratives sont meilleures lorsque la conscience périphérique est bonne, il nous paraît nécessaire de se placer dans des conditions optimales pour augmenter cette conscience périphérique.
				La présence physique au même endroit de tous les collaborateurs est la façon la plus naturelle de créer cette conscience de manière optimale.
				Dans une solution de collaboration distante, de nombreuses informations doivent être enregistrées puis transmises numériquement avec les problématiques de représentations de l'information qui peut se poser.

				De plus, la collaboration distante introduit des problématiques techniques.
				Par exemple, la latence d'un réseau peut poser problème pour la synchronisation de la simulation entre les différents postes de travail.

				Pour finir, la configuration de collaboration colocalisée permet d'avoir un environnement virtuel commun par l'intermédiaire d'un affichage partagé.
				L'affichage est effectué par l'intermédiaire d'un vidéo projecteur et projeté sur un grand écran.
			\end{mysubsection}
			\begin{mysubsection}[sse-Shaddock-ArchitectureLogicielleAdoptee]{Architecture logicielle adoptée}
				Deux types d'architectures ont été explorés pour les \myacro*{acr-EVC} : client/serveur ou pair-à-pair\footnote{\textit{Peer-to-peer} en anglais, parfois abrégé en \myPtwoP.}.
				\begin{mysubsubsection}[sss-Shaddock-ArchitecturePairAPair]{Architecture pair-à-pair}
					L'architecture pair-à-pair consiste à faire communiquer chaque nœud du réseau avec chacun des autres nœuds : le réseau forme un graphe complet.
					L'avantage d'une telle architecture et la rapidité de transfert de l'information puisqu'il n'y a pas d'intermédiaire.
					Cependant, plus le réseau est grand et plus ce type de solutions est difficile à mettre en place.

					Actuellement, les travaux concernant l'utilisation de ce type d'architecture pour la synchronisation d'une manipulation virtuelle à l'aide de l'haptique sont peu nombreux.
					Par exemple, \mycite[author]{Kim-2004} constatent que la latence des réseaux est une problématique importante pour les retours haptiques générés.
					En effet, la fréquence de traitement nécessaire pour l'haptique est nettement plus élevée que pour le retour visuel ($\approx \mynum[Hz]{1000}$ contre $\approx \mynum[Hz]{60}$).
					C'est pourquoi il utilise une architecture pair-à-pair.
					En effet, en dehors d'être une communication directe entre les nœuds, c'est surtout une architecture qui nécessite d'avoir une instance du programme exécutée sur chaque nœud : ainsi, il n'est pas nécessaire d'avoir un réseau pour supporter une fréquence de \mynum[Hz]{1000}, il suffit d'avoir des nœuds qui le supportent.

					Cependant, \mycite[author]{Iglesias-2008} explique, à travers une tâche d'assemblage collaboratif, qu'il est nécessaire que les simulations entre les nœuds du réseau se synchronisent pour rester cohérentes entre elles.
					C'est l'inconvénient majeur de l'architecture pair-à-pair : les environnements virtuels présentés aux différents collaborateurs peuvent diverger et ne pas être cohérents les uns avec les autres.
				\end{mysubsubsection}
				\begin{mysubsubsection}[sss-Shaddock-ArchitectureClientServeur]{Architecture client/serveur}
					Pour contrer les incohérences de simulation, l'architecture client/serveur est nettement plus adaptée.
					La simulation principale n'est exécutée qu'une seule fois sur un nœud du réseau appelé serveur.
					Les collaborateurs utilisent des nœuds clients qui communiquent avec le serveur pour récupérer les informations.

					Bien évidemment, dans ce cas, la latence du réseau a beaucoup plus d'influence sur la qualité des retours haptiques.
					Cependant, nous nous plaçons dans un contexte de collaboration colocalisée ce qui veut dire que des réseaux internes peuvent être utilisés ce qui devrait limiter grandement les problèmes de latence.

					\mycite[author]{Huang-2010} montrent la faisabilité d'une telle configuration en proposant la manipulation d'un jeu de construction par blocs.
					La simulation est centralisée sur un serveur et les interactions haptiques sont produites par l'intermédiaire de clients.
					Il ne souligne aucune instabilité dans les interactions haptiques.
					\mycite[author]{Marsh-2006} ou encore \mycite[author]{Norman-2010} s'intéressent particulièrement aux influences du réseau sur les interactions visuo-haptiques.
					D'après eux, l'architecture client/serveur est la plus adaptée pour la gestion de simulation.
					Cependant, il conclue sur la nécessité d'avoir une information qui transite rapidement afin d'obtenir un rendu haptique le plus fidèle possible.

					Étant donné que nous nous plaçons dans un contexte de simulation moléculaire, il nous est nécessaire de maintenir un environnement virtuel cohérent tout au long de la simulation.
					La plateforme \myShaddock est pourvue d'une architecture client/serveur comme on peut le voir sur la \myref{fig-Shaddock-DiagrammeDeDeploiementUMLDeLaPlateformeShaddock}.
				\end{mysubsubsection}
				\begin{mysubsubsection}[sss-Shaddock-ServeurDePeripheriques]{Serveur de périphériques}
					Afin de gérer les connexions client/serveur pour les interfaces haptiques, nous utilisons le logiciel \myacro{acr-VRPN} développé par \mycite[author]{Taylor-II-2001}.
					La connexion avec le moteur de simulation est gérée par un autre module qui sera détaillé plus tard dans la \myref{sss-Shaddock-SimulationMoleculaireInteractiveEtTempsReel}.

					\myacro{acr-VRPN} offre un moyen simple et relativement universel de connecter des périphériques principalement utilisés en réalité virtuelle.
					En effet, il fournit un serveur pour chaque périphérique.
					Ensuite, l'application cliente peut envoyer et recevoir les informations nécessaires à la communication avec chacun des périphériques.

					Dans notre cas, l'interface haptique est connectée physiquement à un ordinateur\footnote{Cela suppose d'avoir un ordinateur par interface haptique ce qui peut considérablement complexifier la logistique avec un grand nombre de participants.} et un serveur \myacro{acr-VRPN} commande cette interface haptique.
					C'est seulement par l'intermédiaire de ce serveur \myacro{acr-VRPN} et à travers le réseau que le client (\myacro{acr-VMD} dans notre cas) va interagir avec l'interface haptique.

					La compilation de \myacro{acr-VRPN} en tant que serveur de \myOmni sous le système d'exploitation \myLinux (\myUbuntu) a nécessité quelques modifications dans le code source qui n'avait pas été prévu pour ce cas d'utilisation.
					Ces modifications ont été soumises aux développeurs de \myacro{acr-VRPN} qui les ont intégrées dans les versions les plus récentes.
				\end{mysubsubsection}
			\end{mysubsection}
		\end{mysection}
		\begin{mysection}[sec-Shaddock-PlateformeDeSimulationEtDeVisualisation]{Plateforme de simulation et de visualisation}
			\myShaddock permet d'effectuer la visualisation de l'environnement moléculaire manipulé.
			La visualisation est un processus complexe qui nécessite des rendus détaillés et complets des éléments virtuels de l'environnement.
			En effet, devant le nombre important d'informations que possède une molécule (atomes, flexibilité, chimie, champs de forces, \myetc), il est primordial d'avoir des rendus graphiques permettant d'afficher un maximum d'informations pour un minimum de charge visuelle.
			Cette tâche est effectuée par un module de visualisation présenté dans la \myref{sse-Shaddock-ModuleDeVisualisationMoleculaire}.

			Ensuite, \myShaddock simule une dynamique moléculaire.
			Un module de simulation est nécessaire pour réaliser cette tâche.
			Il faut que ce module puisse interagir avec le module de visualisation.
			De plus, il est nécessaire de pouvoir paramétrer finement la simulation.
			Le module de simulation choisi est présenté dans la \myref{sse-Shaddock-ModuleDeSimulationMoleculaire}.

			Cependant, les moteurs de simulation ne sont pas conçus pour effectuer des simulations en temps-réel, et encore moins des simulations interactives en temps-réel.
			Pourtant, afin de proposer une dynamique moléculaire interactive et temps-réel aux utilisateurs, un module présenté dans la \myref{sss-Shaddock-SimulationMoleculaireInteractiveEtTempsReel} permet de faire communiquer le module de visualisation avec le module de simulation.

			\begin{mysubsection}[sse-Shaddock-ModuleDeVisualisationMoleculaire]{Module de visualisation moléculaire}
				Les outils de visualisation moléculaire disponibles sont relativement nombreux.
				Parmi les plus populaires, on peut citer \myPyMOL \mycite{PyMOL-2010}, \myacro{acr-VMD} \mycite{Humphrey-1996}, \myChimera \mycite{Pettersen-2004}, \myRasmol \mycite{Sayle-1995} sans compter les nombreux dérivés permettant un affichage en ligne tel que \myJmol \mycite{Jmol-2006} pour ne citer que le plus connu.
				\myPyMOL et \myacro{acr-VMD} se distinguent particulièrement par leurs nombreuses fonctionnalités et leur large utilisation dans le milieu spécialisé.

				\myPyMOL est probablement le logiciel de visualisation le plus utilisé par les experts du domaine car c'est le plus abouti en ce qui concerne les performances d'affichage.
				Cependant, \myPyMOL ne permet pas l'affichage de simulations temps-réel\footnote{Les travaux récents de \mycite[author]{Baugh-2011} semblent proposer la visualisation en temps-réel d'une simulation moléculaire avec \myPyMOL.}, ni la manipulation interactive de molécules.

				\myacro{acr-VMD} possède également une large gamme de rendus graphiques.
				De plus, depuis maintenant quelques mois, \myacro{acr-VMD} utilise également des optimisations \myGPU pour l'affichage à l'aide de la bibliothèque \myCUDA.
				Contrairement à \myPyMOL, \myacro{acr-VMD} permet le rendu graphique en temps-réel de données de simulation.

				La possibilité d'avoir accès à des rendus graphiques détaillés et complets de la molécule est primordiale pour la visualisation.
				La complexité des molécules, le nombre important d'atomes, les nombreuses meta-informations, les structures particulières nécessitent d'avoir à sa disposition des rendus graphiques spécifiques.

				Le rendu graphique \myCPK a été retenu pour un affichage complet de la molécule avec tous ces atomes et ces liaisons covalentes.
				Cependant, dans certains cas, l'affichage explicite des atomes (une sphère) surcharge l'environnement visuel.
				C'est pourquoi, le rendu complémentaire \myLicorice a également été retenu pour un simple affichage des liaisons covalentes.
				On notera toutefois que les couleurs informent tout de même sur la natures des atomes.
				Ensuite, la chaîne carbonée principale de la molécule est une structure tout à fait particulière mais primordiale dans la compréhension des déformations et de la dynamique de la molécule.
				C'est le rendu \myNewRibbon qui permet de représenter cette chaîne carbonée.
				Enfin, les liaisons hydrogènes ajoutent des informations supplémentaires pertinentes sur la dynamique des atomes et sont représentés par le rendu \myHBonds.

				Les quatre représentations différentes \myref*{fig-simulation-IllustrationsDesRendusGraphiquesDeMoleculesSurVMD} sélectionnées sur la plateforme \myShaddock sont présentées ici :
				\begin{description}
					\item[\myCPK] affiche tous les atomes de la molécule sous forme de sphères en les reliant par des cylindres; c'est un affichage très chargé lorsque le nombre d'atomes est important mais la taille des sphères et des cylindres peut être modifiée \myref*{fig-simulation-CPK};
					\item[\myLicorice] représente tous les liens entre les atomes par des cylindres, sans représenter les atomes; la taille des cylindres peut être modifiée \myref*{fig-simulation-Licorice};
					\item[\myNewRibbon] produit une courbe spline sur les atomes $C_{\alpha}$ représentant l'armature principale de la molécule; la courbe est représentée sous forme de ruban \myref*{fig-simulation-NewRibbon};
					\item[\myHBonds] affiche les potentielles liaisons hydrogène sous forme de traits en pointillés; les seuils physiques ainsi que les paramètres d'affichage de la ligne (couleur, largeur, \myetc) sont modifiables \myref*{fig-simulation-HBonds}.
				\end{description}

				\begin{myfigure}
					\begin{mysubfigure}
						\myimage[width=0.49\textwidth]{simulation-CPK}
						\mysubcaption[fig-simulation-CPK]{\myCPK}
					\end{mysubfigure}
					\begin{mysubfigure}
						\myimage[width=0.49\textwidth]{simulation-Licorice}
						\mysubcaption[fig-simulation-Licorice]{\myLicorice}
					\end{mysubfigure}
					\begin{mysubfigure}
						\myimage[width=0.49\textwidth]{simulation-NewRibbon}
						\mysubcaption[fig-simulation-NewRibbon]{\myNewRibbon}
					\end{mysubfigure}
					\begin{mysubfigure}
						\myimage[width=0.49\textwidth]{simulation-HBonds}
						\mysubcaption[fig-simulation-HBonds]{\myHBonds}
					\end{mysubfigure}
					\mycaption[fig-simulation-IllustrationsDesRendusGraphiquesDeMoleculesSurVMD]{Illustration des rendus graphiques de molécules sur \myacronl{acr-VMD}}
				\end{myfigure}

				Chacune de ces représentations visuelles peut être affectée à tout ou partie de la molécule comme par exemple \og le \myglos{glo-Residu} \mynum{13} \fg, \og seulement les atomes de carbone \fg ou \og tous les \myglos*{glo-Residu} entre \mynum{1} et \mynum{16} sauf les atomes d'hydrogène \fg.
				De plus, pour chacune des représentations précédentes, différentes colorations sont possibles permettant de mettre en évidence ou au contraire de rendre discrètes certaines parties de la molécule :
				\begin{description}
					\item[Couleur fixe] donne une couleur unie prédéfinie (la couleur du curseur par exemple);
					\item[Couleur des atomes] donne une couleur différente à chaque atome selon un code couleur standard dépendant de sa nature (rouge pour oxygène, blanc pour hydrogène, \myetc);
					\item[Couleur des \myglosnl*{glo-Residu}] donne une couleur différente pour chaque atome selon une palette de couleurs prédéfinie par \myacro{acr-VMD};
					\item[Transparence] rend transparent les objets tout en conservant la teinte;
					\item[\textit{GoodSell}] accentue les contours des objets sous le principe du \textit{cell shading}.
				\end{description}
			\end{mysubsection}
			\begin{mysubsection}[sse-Shaddock-ModuleDeSimulationMoleculaire]{Module de simulation moléculaire}
				Les trois principaux modules de simulation moléculaire existants sont \myGromacs \mycite{Berendsen-1995}, \myacro{acr-NAMD} \mycite{Phillips-2005} et \myAMBER \mycite{Case-2005}.
				\mycite[author]{Hess-2008} montre que \myGromacs est plus performant que \myacro{acr-NAMD}.
				Un rapport technique de \mycite[author]{Loeffler-2009a} a également permis de montrer que \myGromacs domine en terme de performances de calculs.
				Une comparaison plus détaillée montre que chaque moteur de simulation offre de bons résultats en fonction des problèmes auquel il est confronté et du matériel utilisé \mycite{Loeffler-2009b}.

				Cependant, \myacro{acr-NAMD} est développé par la même université que \myacro{acr-VMD} ce qui fait que la connexion entre les deux logiciels est facilitée.
				Enfin, l'interaction en temps-réel avec le moteur de simulation est possible avec \myacro{acr-NAMD} mais pas avec \myGromacs.
				C'est pourquoi le moteur de simulation moléculaire \myacro{acr-NAMD} a été retenu pour notre plateforme.

				Une des fonctionnalités de \myacro{acr-NAMD} utilisée est la possibilité de \og fixer \fg des atomes.
				En effet, la restriction de la simulation moléculaire à certaines régions de la molécule permet d'exclure certains atomes de la déformation.
				Ces atomes interviennent dans le calcul des forces de la simulation mais eux-mêmes ne sont pas soumis aux forces de l'environnement.
				Cette fonctionnalité nous permet de créer un point d'ancrage pour la molécule avant d'éviter toute dérive.
				\begin{mysubsubsection}[sss-Shaddock-SimulationMoleculaireInteractiveEtTempsReel]{Simulation moléculaire interactive et temps-réel}
					Les logiciels de simulation moléculaire existants ne sont pas prévus pour des simulations interactives en temps-réel.
					Cependant, l'\myacro{acr-ITAP} a développé le protocole \myacro{acr-IMD} permettant d'utiliser \myacro{acr-NAMD} couplé à \myacro{acr-VMD} pour des simulations en temps-réel \mycite{Stadler-1997}.
					L'extension \textit{\myacronl{acr-IMD} connect} permet de connecter rapidement le logiciel \myacro{acr-VMD} avec la simulation offerte par \myacro{acr-NAMD}.

					Cependant, le protocole \myacro{acr-IMD} a récemment été porté par l'\myacro{acr-IBPC} pour une utilisation avec \myGromacs.
					En effet, \myMDDriver \mycite{Delalande-2009} est une interface permettant d'utiliser le protocole \myacro{acr-IMD} avec d'autre module de simulation comme \myGromacs.
					Cette nouvelle solution pourrait permettre d'améliorer les capacités de simulation de notre plateforme mais elle n'a pas encore été implémentée.
				\end{mysubsubsection}
				\begin{mysubsubsection}[sss-Shaddock-LaGenerationAutomatiqueDeFichierDeSimulation]{La génération automatique de fichier de simulation}
					La configuration de la simulation nécessite un grand nombre d'informations.
					Une partie de ces informations découle directement de la molécule à l'état d'équilibre; ces informations sont les suivantes :
					\begin{itemize}
						\item l'ensemble des liaisons entre atomes;
						\item des angles simples;
						\item des angles dihédraux;
						\item des angles de torsion.
					\end{itemize}
					La simple description des atomes et de leurs positions à l'état d'équilibre (fichier \myPDB) couplée aux données générées par \myCHARMM \mycite{Brooks-1983} permet de générer les fichiers nécessaires au module de simulation.
					\myacro{acr-VMD} fournit l'outil permettant de générer le fichier nécessaire à la simulation (fichier \myPSF) par l'intermédiaire d'une extension : \textit{Automatic \textsc{psf} builder}.
				\end{mysubsubsection}
			\end{mysubsection}
		\end{mysection}
		\begin{mysection}[sec-Shaddock-LesOutilsDInteraction]{Les outils d'interaction}
			\begin{mysubsection}[sse-Shaddock-OutilDInteractionUtilise]{Outil d'interaction utilisé}
				Pour cette plateforme de manipulation de molécules interactive et temps-réel, l'outil d'interaction doit répondre à plusieurs contraintes.

				Tout d'abord, la manipulation de molécules intervient dans un environnement virtuel en \myThreeD.
				Bien que la simple souris soit déjà utilisée dans de nombreuses applications pour interagir avec des environnement virtuels en \myThreeD (modélisation, jeux vidéos, \myCAO, \myetc), ce n'est pas l'outil le plus adapté, notamment pour la sélection en profondeur.
				L'outil doit permettre la manipulation et la sélection dans cet environnement virtuel \myThreeD avec six~\myacro*{acr-DDL}.

				Nous souhaitons proposer une plateforme pouvant être facilement déployée.
				En effet, les biologistes, public potentiel de ce type de plateforme, sont plus habitués à travailler sur des ordinateurs de bureau que dans les environnement immersifs utilisés en réalité virtuelle.
				C'est pourquoi nous orientons notre choix de matériel sur des outils de bureau.
				Cependant, \mycite[author]{Sallnas-2000} a montré l'intérêt d'un retour haptique pour l'amélioration des performances dans les configurations de travail collaboratif; c'est donc naturellement que nous nous sommes orientées vers des interfaces haptiques de bureau.

				Parmi les interfaces haptiques disponibles sur le marché, le \myOmni répond à l'ensemble de ces contraintes avec un rapport qualité/prix correct.
				Le \myDesktop, qui répond également aux contraintes, aurait permis des retours haptiques plus fidèles mais son prix élevé ne convient à notre plateforme collaborative où il faudra en déployer plusieurs.
				De plus, sa connectique (port parallèle) quasiment obsolète, est de plus en plus difficile à trouver sur les ordinateurs récents\footnote{Un convertisseur vers une connectique \myFireWire vendu par \mySensAble existe mais nécessite une dépense supplémentaire significative.}.
			\end{mysubsection}
			\begin{mysubsection}[sse-Shaddock-OutilsExistants]{Outils existants}
				\myAcro{acr-VMD} dispose de différents outils permettant d'effectuer diverses manipulations sur les molécules (sélection, orientation, déplacement, \myetc).

				Une des fonctions élémentaires proposée par \myacro{acr-VMD} est la possibilité d'orienter la scène sur trois \myacro*{acr-DDL} afin d'observer la molécule sous différents angles.
				C'est la souris qui tient ce rôle et elle peut également être configurée pour translater la molécule ou obtenir diverses informations sur la molécule et sur les atomes.
				Il est également possible d'utiliser une souris \myThreeD permettant de regrouper les fonctions de translation et d'orientation de la scène.
				La souris \myThreeD \mySpaceNavigator est utilisée dans le cadre de notre seconde expérimentation \myref*{cha-DeformationCollaborativeDeMolecule}.

				Ces fonctionnalités de navigation sont complétées par des fonctionnalités de manipulations élémentaires.
				Il est possible de déplacer des atomes ou des groupes d'atomes à l'aide des périphériques précédemment cités.
				Cependant, la fonctionnalité qui nous intéresse est la possibilité d'effectuer ces manipulations à l'aide d'interfaces haptiques par l'intermédiaire d'une connexion \myacro{acr-VRPN} \myref*{sss-Shaddock-ServeurDePeripheriques}.

				La connexion à de telles interfaces permet à \myacro{acr-VMD} de proposer deux outils : un outil de navigation et un outil de manipulation.
				Ils sont les suivants :
				\begin{description}
					\item[\mytool{grab}] qui permet de sélectionner une molécule dans son intégralité et de la déplacer dans la scène;
					\item[\mytool{tug}] qui permet de sélectionner un atome de la molécule et de lui appliquer une force (qui sera transmise au moteur de simulation) pour déformer la molécule.
				\end{description}
				Ces outils ont été utilisés dans la première expérimentation \myref*{cha-RechercheCollaborativeDeResiduSurUneMolecule}.
				Cependant, de nombreux outils supplémentaires ont été développés au-fur-et-à-mesure des besoins identifiés durant les expérimentations.
				Ces nouveaux outils sont détaillés dans la \myref{sse-Shaddock-OutilsDeManipulationAvances}.
			\end{mysubsection}
			\begin{mysubsection}[sse-Shaddock-OutilsDeManipulationAvances]{Outils de manipulation avancés}
				Durant les différentes études menées dans la suite de ce document, les analyses et les remarques d'utilisateurs ont permis de mettre en évidence les limites et les contraintes des outils existants.
				De nouveaux outils ont été développés pour répondre à ces besoins, notamment en terme de collaboration.
				Le développement de ces nouveaux outils a nécessité la modification de \myacro{acr-VMD} par extension des outils déjà existants.
				Les fonctionnalités ajoutées sont présentées dans les sections suivantes.
				\begin{mysubsubsection}[sss-Shaddock-OutilDeSelectionCollaboratif]{Outil de sélection collaboratif}
					Afin d'effectuer correctement une sélection, il est préférable de connaître \myapriori quel cible sera sélectionnée en fonction de la position actuelle du curseur.
					C'est pourquoi, cet outil de sélection amélioré met continuellement en surbrillance la cible la plus proche du curseur.
					Afin de distinguer une cible sélectionnée d'une cible par encore sélectionnée, les cibles non-sélectionnées sont simplement surligné de manière transparente alors que les cibles sélectionnées sont de couleurs opaques \myref*{fig-selection-improvement-DifferenceVisuelleEntreLesElementsPointesEtSelectionnes}.

					Cependant, dans un contexte de travail collaboratif, il doit être possible pour chacun des collaborateurs de distinguer sa sélection de la sélections des autres.
					Les curseurs de chacun des participants étant de couleur différente, le surlignage est effectué en utilisant cette couleur.

					Il est également possible de sélectionner des groupes d'atomes (dispositif mis en place sur certaines des expérimentations).
					Cependant, plutôt que de considérer un centre virtuel de ce groupe d'atomes (barycentre par exemple) qui ne serait pas affiché, les interactions haptiques sont effectuées par rapport à un atome.
					Pour que les utilisateurs aient une indication précise concernant les interactions haptiques, cet atome est agrandi par rapport aux reste du groupe d'atome \myref*{fig-selection-improvement-DifferenceVisuelleEntreLesElementsPointesEtSelectionnes}.

					\begin{myfigure}
						\begin{mysubfigure}
							\myimage[width=0.49\textwidth]{selection-improvement-pointed}
							\mysubcaption[fig-selection-improvement-ElementPointe]{Élément pointé}
						\end{mysubfigure}
						\begin{mysubfigure}
							\myimage[width=0.49\textwidth]{selection-improvement-targeted}
							\mysubcaption[fig-selection-improvement-ElementSelectionne]{Élément sélectionné}
						\end{mysubfigure}
						\mycaption[fig-selection-improvement-DifferenceVisuelleEntreLesElementsPointesEtSelectionnes]{Différence visuelle entre les éléments pointés et sélectionnés}
					\end{myfigure}
				\end{mysubsubsection}
				\begin{mysubsubsection}[sss-Shaddock-DeformationParGroupeDAtomes]{Déformation par groupe d'atomes}
					L'outil \mytool{tug} permet de déformer la molécule en appliquant un effort à l'atome sélectionné.
					Cependant, la déformation par l'intermédiaire d'un seul atome possède deux désavantages :
					\begin{itemize}
						\item la déformation d'une molécule atome par atome est un processus très fastidieux;
							il serait plus efficace de déplacer un groupe d'atomes d'un bloc.
						\item la molécule se trouve la plupart du temps dans un état relativement stable et le déplacement d'un atome perturbe cet état de stabilité par un étirement des structures;
							le déplacement d'un groupe d'atomes en un bloc permettrait d'éviter cet étirement et de conserver une meilleure stabilité.
					\end{itemize}

					C'est pourquoi nous proposons un outil appliquant une force à un groupe d'atomes pour le déplacer tout en conservant la stabilité intrinsèque de ce groupe.
					Les groupes d'atomes dignes d'intérêt sont les \myglos*{glo-Residu} (quelques dizaines d'atomes), les \myhelice* ou \myfeuillet* (quelques dizaines de \myglos*{glo-Residu}) et les molécules (enchaînement de \myhelice* et de \myfeuillet*).
					L'outil de déformation a été généralisé aux \myglos*{glo-Residu} et aux molécules.
					Cependant, la généralisation aux \myhelice* et aux \myfeuillet* a été laissée de côté pour deux raisons :
					\begin{itemize}
						\item la forme allongée de ces structures nécessite de repenser partiellement voire complétement l'outil de déformation car la translation seule ne peut plus être utilisée;
						\item les informations concernant ces structures ne sont pas toujours bien renseignées dans les bases de données de molécules.
					\end{itemize}

					Cependant, appliquer le même effort à l'ensemble des atomes d'un \myglos{glo-Residu} ou d'une molécule produit un effort total très important.
					Si l'effort total est trop élevé, les perturbations envoyées à la simulation sont trop importantes et peuvent produire des incohérences dans la simulation voire même un arrêt de la simulation\footnote{\myacro-{acr-NAMD} s'arrête automatiquement lorsqu'il estime que les perturbations produisent trop d'incohérences.}.
					Afin de conserver un effort total de déformation constant quelque soit le nombre d'atomes sélectionnés, la force appliquée à chaque atome est divisée par le nombre totale d'atomes sélectionnés.
				\end{mysubsubsection}
				\begin{mysubsubsection}[sss-Shaddock-OutilDeDesignation]{Outil de désignation}
					L'outil de désignation a été créé pour répondre au besoin d'indiquer des régions d'intérêt aux partenaires.
					Il se découpe en quatre étapes élémentaires :
					\begin{description}
						\item[Recherche d'une cible] Cette étape consiste pour un utilisateur~\myuser{A} à identifier une cible à désigner digne d'intérêt;
							cette cible est choisie en fonction des objectifs de la tâche à réaliser \myref*{fig-designation-RechercheDUneCible}.
						\item[Désignation d'une cible] Une fois la cible trouvée, l'utilisateur~\myuser{A} la désigne à son partenaire~\myuser{B};
							la cible est alors mise en surbrillance de façon à être vue des autres utilisateurs \myref*{fig-designation-CibleDesignee}.
						\item[Acceptation d'une cible] L'utilisateur~\myuser{B} peut alors accepter ou non cette désignation;
							s'il accepte la désignation, la cible est alors colorée de la couleur du curseur de l'utilisateur~\myuser{B} qui a accepté \myref*{fig-designation-CibleAcceptee};
							tant qu'elle n'est pas acceptée, la \myglos{glo-Residu} reste en surbrillance jusqu'à ce que la requête soit accepté ou modifiée par l'utilisateur~\myuser{A}.
						\item[Sélection d'une cible] L'utilisateur~\myuser{B} ayant accepté doit maintenant sélectionner la cible pour achever le processus de désignation;
							tant que l'utilisateur~\myuser{B} n'a pas sélectionné le \myglos{glo-Residu} ciblé, le processus ne peut pas être considéré comme terminé et l'effet de surbrillance reste actif;
							lorsque la cible est sélectionnée, le processus de manipulation reprend normalement \myref*{fig-designation-CibleSelectionnee}.
					\end{description}

					\begin{myfigure}
						\psset{xunit=0.0245\textwidth,yunit=0.013033172\textwidth}
						\begin{mysubfigure}
							\begin{myps}(-10,-10)(10,10)
								\rput(0,0){\myimage[width=0.49\textwidth]{designation-normal}}
								\uput{0pt}[45](2,7){\myimage[width=25pt]{designation-red-cursor}}
								\uput{0pt}[45](-7,6){\myimage[width=25pt]{designation-yellow-cursor}}
							\end{myps}
							\mysubcaption[fig-designation-RechercheDUneCible]{Recherche d'une cible}
						\end{mysubfigure}
						\begin{mysubfigure}
							\begin{myps}(-10,-10)(10,10)
								\rput(0,0){\myimage[width=0.49\textwidth]{designation-called}}
								\uput{0pt}[45](7.6,4.2){\myimage[width=25pt]{designation-red-cursor}}
								\mypsvibration(7.6,4.2)
								\uput{0pt}[45](-7,6){\myimage[width=25pt]{designation-yellow-cursor}}
								\mypsvibration(-7,6)
							\end{myps}
							\mysubcaption[fig-designation-CibleDesignee]{Cible désignée}
						\end{mysubfigure}
						\begin{mysubfigure}
							\begin{myps}(-10,-10)(10,10)
								\rput(0,0){\myimage[width=0.49\textwidth]{designation-accepted}}
								\uput{0pt}[45](5,-8){\myimage[width=25pt]{designation-red-cursor}}
								\uput{0pt}[45](-7,6){\myimage[width=25pt]{designation-yellow-cursor}}
							\end{myps}
							\mysubcaption[fig-designation-CibleAcceptee]{Cible acceptée}
						\end{mysubfigure}
						\begin{mysubfigure}
							\begin{myps}(-10,-10)(10,10)
								\rput(0,0){\myimage[width=0.49\textwidth]{designation-selected}}
								\uput{0pt}[45](5,-8){\myimage[width=25pt]{designation-red-cursor}}
								\uput{0pt}[45](7.9,8.1){\myimage[width=25pt]{designation-yellow-cursor}}
								\pscurve[linewidth=3pt,linecolor=myyellowcolor]{->}(-7,6)(-1,10)(6,10)(7.5,8.9)
							\end{myps}
							\mysubcaption[fig-designation-CibleSelectionnee]{Cible sélectionnée}
						\end{mysubfigure}
						\mycaption[fig-designation-LesQuatreEtapesDeLaDesignation]{Les quatre étapes de la désignation (utilisateur~\myuser{A} en rouge et utilisateur~\myuser{B} en jaune)}
					\end{myfigure}
				\end{mysubsubsection}
			\end{mysubsection}
		\end{mysection}
	\end{mychapter}
	\begin{mychapter}[cha-RechercheCollaborativeDeResiduSurUneMolecule]{Recherche collaborative de résidu sur une molécule}
		\begin{mysection}[sec-exp1-Introduction]{Introduction}
			L'état de l'art du premier chapitre nous a permis d'identifier les principales tâches élémentaires concernant l'interaction en environnement virtuel : les \myacro*{acr-PCV}.
			Dans cette première étude, nous souhaitons étudier les premières \myacro*{acr-PCV} qui sont l'exploration et la sélection \myref*{fig-sota-ProcessusDeDeformationMoleculaireEnQuatreEtapes} dans un contexte de manipulation moléculaire collaborative.
			Le processus d'exploration et de sélection est primordial à toute manipulation ultérieure.

			Dans ce chapitre, nous commencerons par présenter les travaux existants en matière d'exploration et de sélection en environnement virtuel.
			Puis nous présenterons les objectifs de cette étude pour aboutir sur un protocole expérimental.
			Les résultats de l'expérimentation seront analysés et discutés dans la dernière partie.
		\end{mysection}
		\begin{mysection}[sec-exp1-ExplorationEtSelectionCollaborative]{Exploration et sélection collaborative}
			\begin{mysubsection}[sse-exp1-TravauxExistants]{Travaux existants}
				L'exploration et la sélection en environnement virtuel sont des tâches élémentaires relativement peu explorées en environnement virtuel, notamment dans le domaine de la biologie moléculaire.
				Pourtant, des travaux en biologie moléculaire existent dans les environnement réels mais les contraintes techniques sont très différentes de celles en environnement virtuel; nous ne nous étendrons pas sur ce sujet.

				Actuellement, les tâches d'exploration et de recherche collaborative sont principalement destinées aux applications pour les moteurs de recherche.
				\mycite[author]{Morris-2007} souligne que la coopération dans ce domaine existe déjà sur \myInternet mais que les échanges entre les utilisateurs sont encore trop peu exploités.
				Sur cette base, \mycite[author]{Pickens-2007} propose un système pour la recherche collaborative où des utilisateurs à la recherche d'informations peuvent échanger de manière synchrone les résultats intermédiaires afin de progresser plus rapidement.

				En environnement virtuel, \mycite[author]{Wernert-1999} propose simplement une exploration collaborative de l'environnement virtuel; un guide permet à l'utilisateur d'explorer plus efficacement un environnement qu'il ne connait pas.
				À partir de ce travail, \mycite[author]{Hughes-2002} propose une tâche de recherche basée sur le principe du guidage mais en supprimant la collaboration; le guide est un algorithme qui suggére à l'utilisateur les zones d'intérêt.

				L'objectif premier de l'exploration dans notre cas est de pouvoir sélectionner une zone d'intérêt après l'avoir identifiée.
				En biologie moléculaire, les logiciels les plus utilisés tel que \myPyMOL ou \myacro{acr-VMD} proposent des moteurs de sélection à base de chaîne de caractères.
				Par exemple, pour sélectionner tous les atomes de type \myatom{C}, \myatom{O}, \myatom{N} ou \myatom{CA}, on utilisera les commandes suivantes dans \myPyMOL
				\begin{mySource*}[language={}]
pymol> select mysel, name c+o+n+ca
				\end{mySource*}
				ou dans \myacro{acr-VMD}
				\begin{mySource*}[language={}]
vmd> set mysel [atomselect "top" "name C or name O or name N or name CA"]
				\end{mySource*}

				Cependant, ces méthodes de sélection ne s'intègrent dans aucun processus d'exploration : les zones d'intérêt doivent déjà être identifiées avant de procéder à ce type de sélection.
				\myacro{acr-VMD} propose ce type d'exploration avec une souris et permet la sélection à l'aide d'une méthode de \textit{picking}\footnote{Le \textit{picking} consiste à identifier l'élément pointé par la souris (\myTwoD) dans l'environnement virtuel (\myThreeD).}.
				Cependant, la sélection par \textit{picking} est limitée par les problèmes de perception de la profondeur.

				Différentes approches non collaboratives ont été étudiées pour la sélection en environnement moléculaire virtuel.
				Par exemple, \mycite[author]{Pavlovic-1996} propose une plateforme permettant d'interagir avec les molécules en utilisant la voix et les gestes.
				Cependant, les techniques de segmentations de la parole et des gestes sont encore imprécises.
				On trouve également les travaux de \mycite[author]{Polys-2004} qui proposent une interaction avec une \textit{wand}\footnote{Dispositif d'interaction en réalité virtuelle permettant la manipulation sur \mynum{6}~\myacro*{acr-DDL} et disposant de boutons comme une souris.} ou encore les travaux de \mycite[author]{Obeysekare-1996} permettant l'interaction gestuelle avec un gant sur un \myWorkbench\footnote{Dispositif d'affichage permettant l'affichage en \myThreeD stéréoscopique sur deux écrans.}.
				Ces dispositifs sont relativement difficiles à déployer pour des applications collaboratives (multiplication de matériel, accès aux ressources matérielles, \myetc).
				De plus, les tâches de manipulation étudiées sont extrêmement simplifiées et n'ont pas d'application réelle en biologie moléculaire.

				Cependant, certains travaux se sont consacrés à la mise en place d'une plateforme pour des tâches pertinentes.
				Par exemple, \mycite[author]{Levine-1997} proposent une plateforme d'interaction avec un environnement moléculaire virtuel afin d'explorer un complexe ligand-protéine pour réaliser un \myglos{glo-DockingMoleculaire}.
				Cependant, le \myglos{glo-DockingMoleculaire} s'effectue ici sur des corps rigides pour simplifier la complexité de l'exploration.
				On trouve également les travaux de \mycite[author]{Ferey-2008a} mais là encore, il s'agit de corps rigides.
				Cependant, \mycite[author]{Delalande-2010} propose une plateforme en utilisant des périphériques haptiques pour aider à la localisation de ponts ioniques au sein d'une simulation moléculaire en temps-réel sur des corps flexibles.
				Cependant, les travaux ont évolué vers des corps flexibles avec \mycite[author]{Delalande-2010} qui utilisent les périphériques haptiques pour aider à la localisation de ponts ioniques au sein d'une simulation moléculaire en temps-réel.
				L'interface haptique utilisée permet de ressentir les forces en action dans la simulation et ainsi améliorer le processus d'exploration et de sélection.

				Cette brève bibliographie montre que les études manquent concernant les processus d'exploration et de sélection en \myglos{acr-EVC}.
				Ce chapitre a pour but de proposer une tâche de recherche collaborative dans un environnement moléculaire virtuel afin d'identifier les avantages et les limites de cette approche.
			\end{mysubsection}
			\begin{mysubsection}[sse-exp1-Objectifs]{Objectifs}
				Dans ce chapitre, nous abordons les tâches d'exploration et de sélection dans un \myacro{acr-EVC}.
				La recherche au sein d'une simulation moléculaire est une tâche très complexe en raison du grand nombre d'atomes et de la flexibilité de la molécule.
				Nous proposons d'étudier la distribution des charges de travail pour la réalisation de cette tâche.

				Les objectifs de cette première étude sont multiples.
				Tout d'abord, nous souhaitons observer les performances comparées d'un travail autonome face à une configuration de travail collaborative.
				Notre hypothèse va dans le sens d'une amélioration des performances pour les configurations de travail collaboratives sur des tâches de nature complexe.

				De plus, nous souhaitons observer les stratégies de travail qui pourraient émerger de cette collaboration.
				Nous supposons que les stratégies vont varier d'un groupe à l'autre en fonction des affinités et des connaissances intrinsèques du groupe.

				Ensuite, nous nous intéressons plus précisément aux avis des utilisateurs.
				Nous supposons que la configuration collaborative est plus apprécié des utilisateurs grâce à l'émulsion sociale qui naît des interactions et de la communication entre les manipulateurs.

				Enfin, nous souhaitons valider l'utilisabilité de la plateforme de manipulation proposée afin d'améliorer l'ergonomie des outils proposés.
				L'évaluation sera confiée aux sujets.
				L'objectif est d'identifier les faiblesses pour proposer des solutions dans les versions ultérieures \myShaddock.

				Les objectifs sont résumés sous forme d'hypothèses dans la \myref{sse-met-exp1-Hypotheses}.
			\end{mysubsection}
		\end{mysection}
		\begin{mysection}[sec-exp1-PresentationDeLExperimentation]{Présentation de l'expérimentation}
			Afin de répondre à nos hypothèses de travail résumées dans la \myref{sse-met-exp1-Hypotheses}, nous décrivons ici la tâche proposée aux sujets pour répondre aux objectifs fixés.
			\begin{mysubsection}[sse-exp1-DescriptionDeLaTache]{Description de la tâche}
				La tâche proposée consiste à trouver des \myglos*{glo-Residu} au sein d'une molécule puis à les extraire hors de la molécule.
				Les \myglos*{glo-Residu} sont des groupes d'atomes s'associant les uns aux autres le long d'une chaîne carbonée pour former une molécule.
				Trois molécules sont proposées dans le cadre de cette expérimentation.
				La molécule \myTRPZIPPER sera utilisée pour la procédure d'apprentissage.
				Les molécules \myTRPCAGE et \myPrion sont utilisées pour la tâche de recherche et d'extraction de \myglos*{glo-Residu} : \mynum{5}~\myglos*{glo-Residu} à rechercher et à extraire sont proposés sur chaque molécule.
				Les molécules \myTRPCAGE et \myPrion sont utilisées pour la tâche de recherche et d'extraction de \myglos*{glo-Residu} : chaque molécule possède \mynum{5}~\myglos*{glo-Residu} à extraire.
				Tous les \myglos*{glo-Residu} à rechercher sont affichés dans la \myref{tab-exp1-ListeDesResidusRecherches}.
				Pour une description précise des molécules, se reporter à la \myref{sec-pro-PresentationsDesMolecules}.

				\begin{mytable}
					\mycaption[tab-exp1-ListeDesResidusRecherches]{Liste des \myglosnl*{glo-Residu} recherchés}
					\setlength{\myheight}{10ex}
					\newcommand{\mypatternpicture}[1]{\myimage[width=\myheight]{exp1-#1}}
					\begin{mysubtable}
						\mysubcaption[tab-exp1-ListeDesResidusRecherches-ResidusSurLaMoleculeTRPCAGE]{Residus sur la molécule \myTRPCAGE}
						\begin{mytabular}[0.49\textwidth]{^C-C}
							\mytoprule
							\myrowstyle{\bfseries}
							\myGlosnl{glo-Residu} & Image \\
							\mymiddlerule
							\myresidue{1} & \mypatternpicture{pattern1} \\
							\myresidue{2} & \mypatternpicture{pattern2} \\
							\myresidue{3} & \mypatternpicture{pattern3} \\
							\myresidue{4} & \mypatternpicture{pattern4} \\
							\myresidue{5} & \mypatternpicture{pattern5} \\
							\mybottomrule
						\end{mytabular}
					\end{mysubtable}
					\begin{mysubtable}
						\mysubcaption[tab-exp1-ListeDesResidusRecherches-ResidusSurLaMoleculePrion]{Residus sur la molécule \myPrion}
						\begin{mytabular}[0.49\textwidth]{^C-C}
							\mytoprule
							\myrowstyle{\bfseries}
							\myGlosnl{glo-Residu} & Image \\
							\mymiddlerule
							\myresidue{6}  & \mypatternpicture{pattern6}  \\
							\myresidue{7}  & \mypatternpicture{pattern7}  \\
							\myresidue{8}  & \mypatternpicture{pattern8}  \\
							\myresidue{9}  & \mypatternpicture{pattern9}  \\
							\myresidue{10} & \mypatternpicture{pattern10} \\
							\mybottomrule
						\end{mytabular}
					\end{mysubtable}
				\end{mytable}

				La \myref{fig-exp1-RepartitionDesResidusSurLesMolecules} montre la répartition des \myglos*{glo-Residu} sur les deux molécules.

				\begin{myfigure}
					\newcommand{\schemafactor}{0.20}
					\newlength{\schemaunit}\setlength{\schemaunit}{\schemafactor\textwidth}
					\psset{unit=\schemaunit}
					\mycaption[fig-exp1-RepartitionDesResidusSurLesMolecules]{Répartition des \myglosnl*{glo-Residu} sur les molécules}
					\begin{myps}(-2.5,-3)(2.5,3)
						\rput(0,1.75){%
							\myimage[height=2\schemaunit,angle=90]{exp1-trp-cage}}
						\rput(0,2.9){%
							\textcolor{black!25}{\Huge\bfseries\myTRPCAGE}}
						\rput(0,-1.25){%
							\myimage[height=2\schemaunit,angle=90]{exp1-prion}}
						\rput(0,0){%
							\textcolor{black!25}{\Huge\bfseries\myPrion}}
						\rput(-1.5,2){%
							\myimage[height=\schemaunit]{exp1-pattern1}}
						\rput(1.5,2){%
							\myimage[width=\schemaunit]{exp1-pattern3}}
						\rput(1.5,-0){%
							\myimage[width=\schemaunit]{exp1-pattern2}}
						\rput(-1.5,-0){%
							\myimage[width=\schemaunit]{exp1-pattern4}}
						\rput(-1.5,-2){%
							\myimage[width=\schemaunit]{exp1-pattern5}}
						\rput(1.5,-2){%
							\myimage[height=\schemaunit]{exp1-pattern6}}

						\psset{framesize=1 1}
						\fnode(-1.5,2){P1}
						\uput[90](-1.5,2.5){\myresidue{1}}
						\fnode(1.5,2){P38}
						\uput[90](1.5,2.5){\myresidue{3} et \myresidue{8}}
						\fnode(1.5,-0){P27}
						\uput[90](1.5,0.5){\myresidue{2} et \myresidue{7}}
						\fnode(-1.5,-0){P49}
						\uput[90](-1.5,0.5){\myresidue{4} et \myresidue{9}}
						\fnode(-1.5,-2){P510}
						\uput[90](-1.5,-1.5){\myresidue{5} et \myresidue{10}}
						\fnode(1.5,-2){P6}
						\uput[90](1.5,-1.5){\myresidue{6}}

						\psset{linecolor=myred}
						\cnode(0.3,1.5){0.2}{TRPP1}
						\cnode(0.15,2){0.2}{TRPP38}
						\cnode(-0.1,1.25){0.2}{TRPP27}
						\cnode(-0.5,2.2){0.2}{TRPP49}
						\cnode(-0.65,1.25){0.2}{TRPP510}
						\ncline{-}{P1}{TRPP1}
						\ncline{-}{P38}{TRPP38}
						\ncline{-}{P27}{TRPP27}
						\ncline{-}{P49}{TRPP49}
						\ncline{-}{P510}{TRPP510}

						\psset{linecolor=myblue}
						\cnode(0.4,0.2){0.2}{PrionP38}
						\cnode(0.6,-2.8){0.2}{PrionP27}
						\cnode(0.2,-0.8){0.2}{PrionP49}
						\cnode(-0.7,-1.7){0.2}{PrionP510}
						\cnode(0.0,-1.4){0.2}{PrionP6}
						\ncline{-}{P38}{PrionP38}
						\ncline{-}{P27}{PrionP27}
						\ncline{-}{P49}{PrionP49}
						\ncline{-}{P510}{PrionP510}
						\ncline{-}{P6}{PrionP6}
					\end{myps}
				\end{myfigure}

				Chaque \myglos{glo-Residu} possède ses propres caractéristiques (position, couleur, \myetc) offrant des niveaux de complexité différents comme on peut le voir dans la \myref{tab-exp1-ParametresDeComplexiteDesResidus}.
				Les critères de complexité utilisés sont les suivants :
				\begin{description}
					\item[Nombre de \myglosnl*{glo-Residu}] Le nombre total de \myglos*{glo-Residu} présents dans la molécule.
						Un nombre important de \myglos*{glo-Residu} surcharge visuellement l'environnement virtuel et augmente le nombre de cibles potentielles.
					\item[Position] Le \myglos{glo-Residu} peut se trouver soit à la périphérie de la molécule (position \myemph{externe}) ou au centre de la molécule (position \myemph{interne}).
						Un \myglos{glo-Residu} en position externe ne nécessite pas de déformer la molécule pour l'identifier et l'atteindre contrairement à un \myglos{glo-Residu} en position interne dont l'accès sera plus complexe.
					\item[Forme] La forme du \myglos{glo-Residu} est un motif graphique plus ou moins complexe à identifier.
						On distingue trois formes différentes :
						\begin{description}
							\item[Chaîne] Une chaîne d'atomes (la plupart du temps carbonés) avec des atomes d'hydrogène de chaque côté; difficile à identifier par sa forme relativement neutre.
							\item[Cycle] Une chaîne fermée d'atomes de carbone ou d'azote; facile à identifier par sa forme très spécifique.
							\item[Étoile] Séries de chaînes d'atomes toutes reliées sur un atome central (la plupart du temps, un atome de carbone); assez facile à identifier.
						\end{description}
					\item[Couleurs] Les atomes sont colorés en fonction de leur nature (rouge pour l'oxygène, blanc pour l'hydrogène, \myetc).
						Les atomes \myemph{rares} sont donc facilement identifiables grâce à leur couleur singulière.
						Par contre, les atomes nombreux (comme les hydrogènes ou les carbones) seront plus difficiles à identifier à cause de leur fréquence d'apparition élevée.
					\item[Similarité] Certains \myglos*{glo-Residu} à chercher sont très similaires à d'autres \myglos*{glo-Residu} également présents sur la molécule.
						Les \myglos*{glo-Residu} similaires possèdent un atome de moins ou de plus par rapport au \myglos{glo-Residu} recherché.
						À cause de cette similarité, les sujets vont mobiliser une partie du temps à identifier des \myglos*{glo-Residu} incorrects.
				\end{description}
				\begin{mytable}
					\newcommand{\myatomincolor}[3]{\csname my#1\endcsname{}{}#2 en \myemph{#3}}
					\mycaption[tab-exp1-ParametresDeComplexiteDesResidus]{Paramètres de complexité des \myglosnl*{glo-Residu} -- \myatomincolor{carbon}{arbone}{cyan}, \myatomincolor{nytrogen}{zote}{bleu}, \myatomincolor{oxygen}{xygène}{rouge} et \myatomincolor{sulfur}{oufre}{jaune}}
					\begin{mytabular}{^C-C-C-C-C-C}
						\mytoprule
						\myrowstyle{\bfseries}
						\myGlosnl{glo-Residu} & Nombre de \myglosnl*{glo-Residu} & Position & Forme  & Couleurs                                   & Similarité présente \\
						\mymiddlerule[\heavyrulewidth]
						\myresidue{1}         & \mynum{20}                       & Interne  & Cycle  & \mynum{8}~\mycarbon, \mynum{1}~\mynytrogen & Non                 \\
						\mymiddlerule
						\myresidue{2}         & \mynum{20}                       & Interne  & Étoile & \mynum{1}~\mycarbon, \mynum{3}~\mynytrogen & Non                 \\
						\mymiddlerule
						\myresidue{3}         & \mynum{20}                       & Interne  & Cycle  & \mynum{6}~\mycarbon, \mynum{1}~\myoxygen   & Non                 \\
						\mymiddlerule
						\myresidue{4}         & \mynum{20}                       & Externe  & Chaîne & \mynum{4}~\mycarbon                        & Non                 \\
						\mymiddlerule
						\myresidue{5}         & \mynum{20}                       & Externe  & Chaîne & \mynum{4}~\mycarbon, \mynum{1}~\mynytrogen & Non                 \\
						\mymiddlerule[\heavyrulewidth]
						\myresidue{6}         & \mynum{112}                      & Interne  & Chaîne & \mynum{2}~\mycarbon, \mynum{2}~\mysulfur   & Non                 \\
						\mymiddlerule
						\myresidue{7}         & \mynum{112}                      & Externe  & Étoile & \mynum{1}~\mycarbon, \mynum{3}~\mynytrogen & Non                 \\
						\mymiddlerule
						\myresidue{8}         & \mynum{112}                      & Externe  & Cycle  & \mynum{6}~\mycarbon, \mynum{1}~\myoxygen   & Non                 \\
						\mymiddlerule
						\myresidue{9}         & \mynum{112}                      & Interne  & Chaîne & \mynum{4}~\mycarbon                        & Oui                 \\
						\mymiddlerule
						\myresidue{10}        & \mynum{112}                      & Interne  & Chaîne & \mynum{4}~\mycarbon, \mynum{1}~\mynytrogen & Oui                 \\
						\mybottomrule
					\end{mytabular}
				\end{mytable}

				La tâche proposée nécessite deux étapes.
				Selon \mycite[author]{Bowman-1999}, on distingue tout d'abord l'étape d'exploration.
				Pour explorer la molécule afin d'identifier la cible recherchée, les sujets disposent de l'outil \mytool{grab}.
				Lorsque la cible recherchée est identifiée, les sujets commencent la seconde étape : la sélection.
				Pour effectuer cette étape de sélection, les sujets disposent de l'outil \mytool{tug}.
				Les outils \mytool{grab} et \mytool{tug} sont décrits dans la \myref{sse-Shaddock-OutilsExistants}.
			\end{mysubsection}
			\begin{mysubsection}[sse-exp1-SpecificitesDuProtocoleExperimental]{Spécificités du protocole expérimental}
				L'expérimentation est basée sur le dispositif expérimental décrit dans le \myref{cha-pro-DispositifExperimental}.
				Cependant, certains choix expérimentaux concernant cette expérimentation sont détaillés dans les sections suivantes.
				La méthode expérimentale, présentée dans la \myref{sec-met-exp1-PremiereExperimentation}, est résumée dans la \myref{tab-exp1-SyntheseDeLaMethodeExperimentale}.
				\begin{mysubsubsection}[sss-exp1-MaterielUtilise]{Matériel utilisé}
					Cette première expérimentation propose aux sujets d'effectuer une recherche de \myglos*{glo-Residu} au sein d'une molécule de taille importante.
					Les sujets disposent de deux outils de déformation \mytool{tug}.
					Cependant, un outil d'orientation de la molécule est mis à disposition pour des raisons détaillés dans la \myref{sss-exp1-OutilsDeManipulation}.
					Un outil supplémentaire nécessite quelques modifications logistiques.

					L'outil d'orientation de la molécule est assuré par un \myOmni associé à l'outil \mytool{grab} \myref*{sse-Shaddock-OutilsExistants}.
					L'ajout d'un outil nécessite également l'ajout d'un ordinateur comme serveur \myacro{acr-VRPN}.
					L'interface est placée devant le sujet en charge de cet outil.

					Durant l'expérimentation, le \myglos{glo-Residu} à rechercher est affiché aux sujets pendant toute la durée de la tâche.
					Afin de ne pas perturber la scène virtuelle, le \myglos{glo-Residu} est affiché sur un écran \myLCD \mynum[pouces]{17} placé sur la table devant les sujets.

					Pour terminer, cette expérimentation doit nous permettre également d'identifier les contraintes en communication d'une tâche collaborative.
					C'est pourquoi nous enregistrons tous les échanges oraux entre les sujets à l'aide d'un microphone installé sur la table, face aux sujets.
					L'enregistrement est assuré par le logiciel \myAudacity.
					Un filtrage du bruit de fond est effectué \myafortiori afin de rendre les enregistrements plus audibles.

					La \myref{fig-exp1-SchemaDuDispositifExperimental} est un schéma récapitulatif de la disposition des tous les éléments dans la salle d'expérimentation.
					La \myref{fig-exp1-PhotographieDuDispositifExperimental} est une photographie de la salle d'expérimentation.

					\begin{myfigure}
						\myimage[width=0.9\textwidth]{exp1-schema}
						\mycaption[fig-exp1-SchemaDuDispositifExperimental]{Schéma du dispositif expérimental}
					\end{myfigure}
					\begin{myfigure}
						\myimage[width=0.9\textwidth]{exp1-photo}
						\mycaption[fig-exp1-PhotographieDuDispositifExperimental]{Photographie du dispositif expérimental}
					\end{myfigure}
				\end{mysubsubsection}
				\begin{mysubsubsection}[sss-exp1-VisualisationEtRepresentation]{Visualisation et représentation}
					La représentation des molécules utilisées dans cette expérimentation est la représentation classique décrite dans la \myref{sse-pro-RepresentationDesMolecules}.
					L'illustration des molécules \myTRPCAGE et \myPrion telles qu'elles étaient représentées durant l'expérimentation sont affichées sur la \myref{fig-exp1-RepartitionDesResidusSurLesMolecules}.
				\end{mysubsubsection}
				\begin{mysubsubsection}[sss-exp1-OutilsDeManipulation]{Outils de manipulation}
					Durant cette tâche de recherche, nous donnons aux utilisateurs la possibilité de déformer la molécule à l'aide de l'outil \mytool{tug}.
					Cependant, afin de fournir un moyen d'explorer la molécule sous tous les angles, nous proposons également un outil d'orientation de la molécule.
					Sans un tel outil, l'accès à tout \myglos{glo-Residu} qui se trouverait derrière la molécule nécessiterait une longue et fastidieuse déformation.

					Cet outil, concrétisé par une interface haptique associé à l'outil \mytool{grab}, permet de sélectionner la molécule puis de la déplacer et de l'orienter.
					Aucune modification de l'outil proposé par \myacro{acr-VMD} n'a été apportée.
					Cependant, l'outil n'est pas partagé entre les utilisateurs; au début de l'expérimentation, il est demandé aux sujets de choisir celui qui sera en charge de cet outil de manipulation et ceci, tout au long de l'expérimentation.
					Ce choix a été fait pour limiter les conflits entre les deux utilisateurs pendant la phase de recherche et de sélection.
					Il est à noter que pour les \myglos*{glo-Monome}, le sujet n'a accès qu'à un seul outil de déformation et un outil de manipulation.
				\end{mysubsubsection}
				\begin{mytable}
					\mycaption[tab-exp1-SyntheseDeLaMethodeExperimentale]{Synthèse de la méthode expérimentale}
					\newcommand{\mytitlecolumn}[2]{%
						\multirow{#1}*{%
							\begin{minipage}{6em}%
								\raggedleft #2%
							\end{minipage}%
						}
					}
					\newlength{\exponefirstcolumn}
					\newlength{\exponesecondcolumn}
					\setlength{\exponefirstcolumn}{7em}
					\setlength{\exponesecondcolumn}{\textwidth}
					\addtolength{\exponesecondcolumn}{-\exponefirstcolumn}
					\addtolength{\exponesecondcolumn}{-4\tabcolsep}
					\begin{mytabular}{>{\bfseries}p{\exponefirstcolumn}p{\exponesecondcolumn}}
						\mytoprule
						\mytitlecolumn{1}{Tâche}                   & Recherche et sélection de motifs                                             \\
						\mymiddlerule[\heavyrulewidth]
						\mytitlecolumn{4}{Hypothèses}              & \myhypothesis{1} Amélioration des performances en \myglosnl{glo-Binome}      \\
						                                           & \myhypothesis{2} Stratégies variables en fonction des \myglosnl*{glo-Binome} \\
						                                           & \myhypothesis{3} Les sujets préfèrent le travail en \myglosnl{glo-Binome}    \\
						                                           & \myhypothesis{4} Bonne utilisabilité de la plateforme                        \\
						\mymiddlerule
						\mytitlecolumn{2}{Variables indépendantes} & \myvari{1} Nombre de sujets                                                  \\
						                                           & \myvari{2} \myGlosnl{glo-Residu} à chercher                                  \\
						\mymiddlerule
						\mytitlecolumn{6}{Variables dépendantes}   & \myvard{1} Temps de réalisation                                              \\
						                                           & \myvard{2} Distance entre les espaces de travail                             \\
						                                           & \myvard{3} Communication verbales                                            \\
						                                           & \myvard{4} Affinités entre les sujets                                        \\
						                                           & \myvard{5} Force moyenne appliquée par le sujet                              \\
						                                           & \myvard{6} Réponses qualitatives                                             \\
						\mymiddlerule[\heavyrulewidth]
						\multicolumn{2}{c}{%
							\small%
							\begin{tabular}{^C-C-C}
								\myrowstyle{\bfseries}
								\myconditions{1}{4}               & \myconditions{5}{8}               & \myconditions{9}{10}              \\
								\mymiddlerule
								Sujet~\myuser{A}                  & Sujet~\myuser{A}                  & \myGlosnl{glo-Binome}~\myuser{AB} \\
								\mynum{10}~\myglosnl*{glo-Residu} & \mynum{10}~\myglosnl*{glo-Residu} & \mynum{10}~\myglosnl*{glo-Residu} \\
								\mymiddlerule
								Sujet~\myuser{B}                  & \myGlosnl{glo-Binome}~\myuser{AB} & Sujet~\myuser{A}                  \\
								\mynum{10}~\myglosnl*{glo-Residu} & \mynum{10}~\myglosnl*{glo-Residu} & \mynum{10}~\myglosnl*{glo-Residu} \\
								\mymiddlerule
								\myGlosnl{glo-Binome}~\myuser{AB} & Sujet~\myuser{B}                  & Sujet~\myuser{B}                  \\
								\mynum{10}~\myglosnl*{glo-Residu} & \mynum{10}~\myglosnl*{glo-Residu} & \mynum{10}~\myglosnl*{glo-Residu} \\
							\end{tabular}
						} \\
						\mybottomrule
					\end{mytabular}
				\end{mytable}
			\end{mysubsection}
		\end{mysection}
		\begin{mysection}[sec-exp1-Resultats]{Résultats}
			Cette section présente et analyse l'ensemble des mesures expérimentales de cette première étude concernant la recherche et la sélection sur une tâche complexe de collaboration.
			Les données, confrontées à un test de \mycite[author]{Shapiro-1965}, ne sont pas distribuées selon une loi normale.
			Cependant, un test de \mycite[author]{Brown-1974} permet de confirmer l'\myglos{glo-Homoscedasticite}.
			L'analyse de la variance est alors pratiquée à l'aide d'un test de \mycite[author]{Friedman-1940} adapté pour les \myglos*{glo-VariableIntraSujets} non-paramètriques.

			Il est à noter que les données comparées entre les \myglos*{glo-Monome} et les \myglos*{glo-Binome} ne sont pas du même ordre de grandeur (\mynum{24}~\myglos*{glo-Monome} face à \mynum{12}~\myglos*{glo-Binome}).
			Afin de pouvoir effectuer une comparaison du même ordre de grandeur, les données des sujets ayant fait partie d'un même \myglos{glo-Binome} ont été moyennées.
			Ainsi, pour chaque variable correspond une donnée en \myglos{glo-Monome} et une donnée en \myglos{glo-Binome}.
			\begin{mysubsection}[sse-exp1-AmeliorationDesPerformancesEnBinome]{Amélioration des performances en \myglosnl{glo-Binome}}
				Dans cette section, nous étudions l'évolution des performances en présentant les données et les analyses statistiques dans un premier temps, puis une discussion critique de ces résultats dans un second temps.
				\begin{mysubsubsection}[sss-exp1-AmeliorationDesPerformancesEnBinome-DonneesEtStatistiques]{Données et statistiques}
					\begin{myfigure}
						\psset{xunit=0.0889\textwidth,yunit=0.01cm}
						\begin{myps}(-1.25,-115)(10,425)
							\myaxes(0,10){\myglosnl*{glo-Residu}}(0,400)[100]{temps~(s)}
							\myboxplot{exp1-time-residue.csv}
						\end{myps}
						\mycaption[fig-exp1-TempsDeRealisationParResidu]{Temps de réalisation par \myglosnl{glo-Residu}}
					\end{myfigure}

					La \myref{fig-exp1-TempsDeRealisationParResidu} présente le temps de réalisation \myvard{1} pour l'identification et l'extraction de chaque \myglos{glo-Residu} \myvari{2}.
					L'analyse montre qu'il y a un effet significatif des \myglos*{glo-Residu} \myvari{2} sur le temps de réalisation \myvard{1} (\myanova{exp1-time-residue-anova.tex}).
					Un test post-hoc de \mycite[author]{Mann-1947} avec une correction de \mycite[author]{Holm-1979} permet de déterminer que les \myglos*{glo-Residu} \myresidue{6}, \myresidue{9} et \myresidue{10} obtiennent des temps de réalisation significativement plus longs de \myratio{exp1-time-residue-ratio.tex} que les autres \myglos*{glo-Residu}.

					\begin{myfigure}
						\psset{xunit=0.0889\textwidth,yunit=0.01cm}
						\begin{myps}(-1.25,-115)(10,460)
							\myaxes(0,10){\myglosnl*{glo-Residu}}(0,400)[100]{temps~(s)}
							\myboxplot{exp1-time-residue-group.csv}
							\mylegend{\myleg{\myGlosnl{glo-Monome}}{myblue}\myand\myleg{\myGlosnl{glo-Binome}}{myblue!70}}
						\end{myps}
						\mycaption[fig-exp1-TempsDeRealisationComparesMonomeOuBinomeParResidu]{Temps de réalisation comparés (\myglosnl{glo-Monome} ou \myglosnl{glo-Binome}) par \myglosnl{glo-Residu}}
					\end{myfigure}

					La \myref{fig-exp1-TempsDeRealisationComparesMonomeOuBinomeParResidu} présente les temps de réalisation \myvard{1} de chaque \myglos{glo-Residu} \myvari{2} en fonction du nombre de participants \myvari{1}.
					L'analyse ne montre pas d'effet significatif du nombre de participants \myvari{1} sur le temps de réalisation \myvard{1} (\myanova{exp1-time-residue-group-anova.tex}).
					Cependant, en se limitant aux groupes des trois \myglos*{glo-Residu} \myresidue{6}, \myresidue{9} et \myresidue{10} identifiés précédemment comme significativement plus longs à trouver et extraire, l'analyse montre un effet significatif du nombre de participants \myvari{1} sur le temps de réalisation \myvard{1} (\myanova{exp1-time-residue-group-anova-restricted.tex}); le temps de réalisation pour ces résidus est inférieur de \myratio{exp1-time-residue-group-ratio-restricted.tex}.

					\begin{myfigure}
						\psset{xunit=0.0889\textwidth,yunit=0.01cm}
						\begin{myps}(-1.25,-115)(10,460)
							\myaxes(0,10){\myglosnl*{glo-Residu}}(0,400)[100]{temps~(s)}
							\myboxplot{exp1-timeaudio-residue-searchselection.csv}
							\mylegend{\myleg{Recherche}{myblue}\myand\myleg{Sélection}{myblue!70}}
						\end{myps}
						\mycaption[fig-exp1-TempsDeRechercheEtDeSelectionComparesParResidu]{Temps de recherche et de sélection comparés par \myglosnl{glo-Residu}}
					\end{myfigure}

					La \myref{fig-exp1-TempsDeRechercheEtDeSelectionComparesParResidu} présente les temps de recherche et de sélection par \myglos{glo-Residu} \myvari{2}.
					L'analyse montre un effet significatif des \myglos*{glo-Residu} \myvari{2} sur les temps de recherche (\myanova{exp1-timeaudio-residue-searchselection-anova-search.tex}).
					Un test post-hoc de \mycite[author]{Mann-1947} avec une correction de \mycite[author]{Holm-1979} permet de déterminer que les \myglos*{glo-Residu} \myresidue{9} et \myresidue{10} obtiennent des temps de recherche significativement plus longs de \myratio{exp1-timeaudio-residue-searchselection-ratio-search.tex} que les autres \myglos*{glo-Residu}.
					L'analyse montre également un effet significatif des \myglos*{glo-Residu} \myvari{2} sur les temps de sélection (\myanova{exp1-timeaudio-residue-searchselection-anova-selection.tex}).
					Un test post-hoc de \mycite[author]{Mann-1947} avec une correction de \mycite[author]{Holm-1979} permet de déterminer que le \myglos{glo-Residu} \myresidue{6} obtient un temps de sélection significativement plus long de \myratio{exp1-timeaudio-residue-searchselection-ratio-selection.tex} que les autres \myglos*{glo-Residu}.
				\end{mysubsubsection}
				\begin{mysubsubsection}[sss-exp1-AmeliorationDesPerformancesEnBinome-AnalyseEtDiscussion]{Analyse et discussion}
					Les cinq \myglos*{glo-Residu} \myresidue{1}, \myresidue{2}, \myresidue{3}, \myresidue{4} et \myresidue{5} sont au sein de la molécule \myTRPCAGE qui en compte un nombre total relativement limité (\mynum{20}~\myglos*{glo-Residu}).
					Durant la phase d'exploration, les sujets construisent rapidement une carte mentale de la molécule ce qui leur permet de d'identifier rapidement les \myglos*{glo-Residu} recherchés.
					De plus, les faibles contraintes physiques de la molécule (énergie totale du système peu élevée à cause du faible nombre d'atomes) la rende facile à déformer et permet un accès rapide aux structures internes.
					Cela facilite la recherche des \myglos*{glo-Residu} qui sont dans une position interne à la molécule et qui nécessitent une déformation plus importante afin de pouvoir l'extraire.
					Tous ces facteurs rendent les tâches de recherche et de sélection peu complexes sur la molécule \myTRPCAGE ce qui explique des temps de réalisation de la tâche très courts, aussi bien pour les \myglos*{glo-Monome} que pour les \myglos*{glo-Binome}.

					Les cinq \myglos*{glo-Residu} \myresidue{6}, \myresidue{7}, \myresidue{8}, \myresidue{9} et \myresidue{10} sont au sein de la molécule \myPrion qui en compte un nombre total relativement important (\mynum{112}~\myglos*{glo-Residu}).
					La construction complète d'une carte mentale est très complexe à cause du nombre importants d'atomes qui sont continuellement en mouvement (dû à la simulation en temps-réel).
					Les sujets n'étant jamais confronté plus de deux fois à la même tâche (une fois en \myglos{glo-Monome} et une fois en \myglos{glo-Binome}), le phénomène d'apprentissage ne peut pas être effectué.
					En effet, les sujets ne se souviennent pas de la position d'un \myglos{glo-Residu} d'une confrontation à l'autre (contrairement à la molécule \myTRPCAGE pour certains cas).
					Les sujets adoptent une stratégie en plusieurs étapes en fonction de la caractéristique de la tâche et du \myglos{glo-Residu} à trouver.
					Tout d'abord, une recherche exploratoire permet d'identifier les \myglos*{glo-Residu} \myresidue{7} et \myresidue{8} qui se trouvent en position externe.
					Ensuite, lorsque cette première étape exploratoire ne permet pas d'identifier le \myglos{glo-Residu} recherché, les sujets déforment la molécule afin d'accéder aux \myglos*{glo-Residu} \myresidue{6}, \myresidue{9} et \myresidue{10} qui se trouvent en position interne.

					Le travail en \myglos{glo-Binome} comparé au travail en \myglos{glo-Monome} ne montre pas d'amélioration significative bien que la \mypvalue soit très proche du seuil de significativité.
					Cependant, un test post-hoc a permis de d'identifier les \myglos*{glo-Residu} \myresidue{6}, \myresidue{9} et \myresidue{10} comme ayant un temps de réalisation significativement plus long.
					Sur ce groupe de \myglos*{glo-Residu} plus complexes, les \myglos*{glo-Binome} obtiennent une amélioration significative des performances par rapport aux \myglos*{glo-Monome}.
					Ce résultat confirme notre hypothèse \myhypothesis{1} exclusivement sur des tâches de fortes complexité.

					Comme développé dans la procédure expérimentale, le temps de réalisation de la tâche peut être séparé en deux parties : le temps de recherche et le temps de sélection \myref*{fig-exp1-EtapesDeLaCommunicationVerbalePourLaRechercheDUnResidu}.
					Les \myglos*{glo-Residu} \myresidue{9} et \myresidue{10} se distinguent par un temps de recherche significativement plus long que les autres \myglos*{glo-Residu} (excepté \myresidue{6}).
					En effet, ces deux \myglos*{glo-Residu} sont en présence d'autres \myglos*{glo-Residu} similaires au sein de la même molécule \myref*{tab-exp1-ParametresDeComplexiteDesResidus}.
					Ces similarités ont pour effet de monopoliser l'attention des sujets ce qui provoque une hausse significative du temps de recherche du \myglos{glo-Residu} au sein de la molécule.

					De la même façon, le \myglos{glo-Residu} \myresidue{6} se distingue par un temps de sélection significativement plus long que les autres \myglos*{glo-Residu} (excepté \myresidue{9} et \myresidue{10}).
					Ce \myglos{glo-Residu} possède deux atomes de \myatom{Soufre} de couleur jaune.
					Cette particularité aisément identifiable malgré le nombre importants d'atomes de la molécules.
					Le temps de recherche est alors extrêmement court.
					Cependant, ce \myglos{glo-Residu} est positionné au centre de la molécule.
					L'accès au \myglos{glo-Residu} nécessite de \myemph{déplier} en grande partie la molécule afin de pouvoir le sélectionner et l'extraire.

					L'analyse du rapport entre les temps de recherche et de sélection met en évidence trois configurations en fonction des différents \myglos*{glo-Residu} :
					\begin{description}
						\item[Temps de recherche et de sélection égaux]
							Les sujets ont un temps similaire alloué à l'étape de recherche et de sélection.
							Les \myglos*{glo-Residu} concernés ne présentent pas de forte complexité (tous les \myglos*{glo-Residu} de la molécule \myTRPCAGE et les \myglos*{glo-Residu} \myresidue{7} et \myresidue{8} de la molécule \myPrion) et sur lesquels, le travail collaboratif n'améliore pas les performances.
						\item[Temps de recherche prédominant]
							Les sujets ont un temps important alloué à l'identification du \myglos{glo-Residu} recherché.
							Une fois identifié, le \myglos{glo-Residu} est facile à sélectionner puis à extraire.
							Les \myglos*{glo-Residu} \myresidue{9} et \myresidue{10} sont concernés.
							Dans cette configuration, le travail collaboratif améliore significativement les performances.
							En effet, l'étape de recherche est fortement parallélisable : l'espace de recherche est séparé entre les sujets (stratégie \myemph{diviser pour régner}).
						\item[Temps de sélection prédominant]
							Les sujets ont un temps important alloué à la sélection et à l'extraction du \myglos{glo-Residu} recherché.
							Le \myglos{glo-Residu} est rapidement identifié mais il est difficile d'y accéder directement.
							Une phase de déformation est nécessaire pour le sélectionner.
							Le \myglos{glo-Residu} \myresidue{6} est concerné.
							Dans cette configuration, le travail collaboratif améliore significativement les performances.
							En effet, l'étape de déformation bénéficie d'une action coordonnée entre les sujets : l'effort déployé est alors plus important et le contrôle sur la déformation meilleur ce qui permet une réalisation de la tâche plus rapide.
					\end{description}
				\end{mysubsubsection}
			\end{mysubsection}
			\begin{mysubsection}[sse-exp1-StrategiesDeTravail]{Stratégies de travail}
				Dans cette section, nous mettons en avant les différentes stratégies adoptées par les sujets à l'aide d'une analyse statistiques des données.
				\begin{mysubsubsection}[sss-exp1-StrategiesDeTravail-DonneesEtAnalyses]{Données et analyses}
					Dans cette section, les données concernent exclusivement les \myglos*{glo-Binome}.
					Une numérotation des \myglos*{glo-Binome} a été effectuée afin de pouvoir comparer les mesures effectuées et ainsi, étudier les différentes stratégies.

					\begin{myfigure}
						\psset{xunit=0.074\textwidth,yunit=0.15cm}
						\begin{myps}(-1.5,-7)(12,22)
							\myaxes(0,12){\myglosnl*{glo-Binome}}(0,20)[4]{distance~(mm)}
							% Once header are readed, they are defined for other barplot
							% That's why barplots without headers are in first position
							\mybarplot[header=false,barstyle=third-barstyle]{exp1-diff-groups3.csv}
							\mybarplot[header=false,barstyle=second-barstyle]{exp1-diff-groups2.csv}
							\mybarplot[barstyle=first-barstyle]{exp1-diff-groups1.csv}
							\psset{linecolor=myred,linewidth=1pt,linestyle=solid}
							\psline(0,14)(12,14)
							\psline(0,8)(12,8)
							\psset{linewidth=0.1pt,linecolor=white,fillstyle=solid,fillcolor=myred}
							\uput[180](12,5){\pscharpath{\LARGE\bf\sffamily Champ proche}}
							\uput[180](12,11){\pscharpath{\LARGE\bf\sffamily Champ voisin}}
							\uput[180](12,17){\pscharpath{\LARGE\bf\sffamily Champ distant}}
						\end{myps}
						\mycaption[fig-exp1-DistanceMoyenneEntreLesSujetsPourChaqueBinomeSurLesResidusSixNeufEtDix]{Distance moyenne entre les sujets pour chaque \myglosnl{glo-Binome} sur les \myglosnl*{glo-Residu} \myresidue{6}, \myresidue{9} et \myresidue{10}}
					\end{myfigure}

					La \myref{fig-exp1-DistanceMoyenneEntreLesSujetsPourChaqueBinomeSurLesResidusSixNeufEtDix} présente la distance moyenne entre les espaces de travail \myvard{2} de chaque \myglos{glo-Binome}.
					Les \myglos*{glo-Binome} peuvent être classés en trois catégories : \myemph{espace distant}, \myemph{espace voisin} et \myemph{espace proche}.

					\begin{myfigure}
						\psset{xunit=0.074\textwidth,yunit=0.5cm}
						\begin{myps}(-1.5,-2)(12,5.5)
							\myaxes(0,12){\myglosnl*{glo-Binome}}(0,5)[1]{affinité~(\mynum{1}--\mynum{5})}
							\mybarplot{exp1-affinity-groups.csv}
						\end{myps}
						\mycaption[fig-exp1-AffiniteEntreLesSujetsPourChaqueBinome]{Affinité entre les sujets pour chaque \myglosnl{glo-Binome}}
					\end{myfigure}

					La \myref{fig-exp1-AffiniteEntreLesSujetsPourChaqueBinome} présente les affinités \myvard{4} de chaque \myglos{glo-Binome}.
					Les notes, comprises entre un et cinq, montre que les \myglos*{glo-Binome} choisis ont des affinités relativement variées.
					L'affinité entre les sujets du \myglos*{glo-Binome} \mygroup{1} est très basse contrairement aux \myglos*{glo-Binome} \mygroup{8} et \mygroup{12} pour lesquelles l'affinité est très élevée.

					\begin{myfigure}
						\psset{xunit=0.074\textwidth,yunit=0.01cm}
						\begin{myps}(-1.5,-110)(12,325)
							\myaxes(0,12){\myglosnl*{glo-Binome}}(0,300)[50]{temps~(s)}
							\mybarplot{exp1-time-groups.csv}
						\end{myps}
						\mycaption[fig-exp1-TempsDeRealisationEntreLesSujetsPourChaqueBinome]{Temps de réalisation entre les sujets pour chaque \myglosnl{glo-Binome}}
					\end{myfigure}

					La \myref{fig-exp1-TempsDeRealisationEntreLesSujetsPourChaqueBinome} présente les temps de réalisation \myvard{1} de chaque \myglos{glo-Binome}.
					Le temps de réalisation de \mygroup{1} est particulièrement important (plus d'une fois et demi les \myglos*{glo-Binome} les plus longs).
					À l'opposé, on note que \mygroup{2}, \mygroup{3} et \mygroup{4} obtiennent des temps de réalisation extrêmement bas.

					\begin{myfigure}
						\psset{xunit=0.074\textwidth,yunit=0.05cm}
						\begin{myps}(-1.5,-22)(12,75)
							\myaxes(0,12){\myglosnl*{glo-Binome}}(0,70)[10]{temps~(s)}
							\mybarplot{exp1-timeaudio-groups.csv}
						\end{myps}
						\mycaption[fig-exp1-TempsDeCommunicationVerbaleEntreLesSujetsPourChaqueBinome]{Temps de communication verbale entre les sujets pour chaque \myglosnl{glo-Binome}}
					\end{myfigure}

					La \myref{fig-exp1-TempsDeCommunicationVerbaleEntreLesSujetsPourChaqueBinome} présente les temps de communication verbale \myvard{3} de chaque \myglos{glo-Binome}.
					\mygroup{2}, \mygroup{3} et \mygroup{4} ont des temps de communication verbale inférieurs à \mynum[s]{20}.
					À l'opposé, \mygroup{1}, \mygroup{5} et \mygroup{11} ont des temps de communication verbale qui approche les \mynum[s]{60}.

					\begin{myfigure}
						\psset{xunit=0.074\textwidth,yunit=0.03cm}
						\begin{myps}(-1.5,-35)(12,120)
							\myaxes(0,12){\myglosnl*{glo-Binome}}(0,100)[25]{temps~(\%)}
							\mybarplot{exp1-timeaudio-groups-searchselection.csv}
							\mylegend{\myleg{Recherche}{myblue}\myand\myleg{Sélection}{myblue!70}}
						\end{myps}
						\mycaption[fig-exp1-PourcentageDeTempsDeCommunicationVerbalePendantLaRechercheEtLaSelectionEntreLesSujetsPourChaqueBinome]{Pourcentage de temps de communication verbale pendant la recherche et la sélection entre les sujets pour chaque \myglosnl{glo-Binome}}
					\end{myfigure}

					La \myref{fig-exp1-PourcentageDeTempsDeCommunicationVerbalePendantLaRechercheEtLaSelectionEntreLesSujetsPourChaqueBinome} présente les pourcentages de temps de communication verbale durant la phase de recherche et durant la phase de sélection de chaque \myglos{glo-Binome} par rapport au temps total de réalisation de la tâche.
					Le pourcentage représente le rapport du temps de communication verbale durant la phase recherche ou de sélection rapporté respectivement au temps total de la phase de recherche ou de sélection.
					Les \myglos*{glo-Binome} \mygroup{1} à \mygroup{4} ainsi que \mygroup{9} communiquent plus durant la phase de sélection.
					Les \myglos*{glo-Binome} \mygroup{5} à \mygroup{8} et \mygroup{10} à \mygroup{12} communiquent plus durant la phase de recherche.
					Notons également que \mygroup{1} communique assez peu par rapport aux autres \myglos*{glo-Binome}.

					\begin{myfigure}
						\psset{xunit=0.074\textwidth,yunit=1cm}
						\begin{myps}(-1.5,-1)(12,3.6)
							\myaxes(0,12){\myglosnl*{glo-Binome}}(0,3)[1]{force~(N)}
							\mybarplot{exp1-force-groups-meandiff.csv}
							\mylegend{\myleg{Force moyenne}{myblue}\myand\myleg{Différence de force}{myblue!70}}
						\end{myps}
						\mycaption[fig-exp1-ForceMoyenneEtDifferenceDeForceEntreLesSujetsPourChaqueBinome]{Force moyenne et différence de force entre les sujets pour chaque \myglosnl{glo-Binome}}
					\end{myfigure}

					La \myref{fig-exp1-ForceMoyenneEtDifferenceDeForceEntreLesSujetsPourChaqueBinome} représente la force moyenne appliquée par les sujets \myvard{5} et la différence de force entre les sujets.
					La différence de force est la différence entre les forces moyennes de chaque sujet.
					\mygroup{9} et \mygroup{11} apporte un effort moyen très important par rapport aux autres \myglos*{glo-Binome}.
					\mygroup{2}, \mygroup{3} et \mygroup{4} apporte un effort moyen important également tout en ayant une différence de force quasiment nulle entre les deux membres du \myglos{glo-Binome}.

					L'ensemble des résultats et analyses précédentes permet de différencier les \myglos*{glo-Binome} ce qui confirme notre hypothèse \myhypothesis{2}.
					Les \myglos*{glo-Binome} se différencient pas des stratégies de travail variables.
					Les sections suivantes caractérisent les différentes stratégies de travail en fonction de plusieurs paramètres (distance entre les espaces de travail, affinités, temps de réalisation de la tâche, communication verbale, forces moyennes appliquées).
					Trois stratégies sont décrites, distinguées en fonction des distances entre les espaces de travail.
					\begin{description}
						\item[Collaboration en champ proche] pour les distances inférieures à \mynum[mm]{8};
						\item[Collaboration en champ voisin] pour les distances comprises entre \mynum[mm]{8} et \mynum[mm]{14};
						\item[Collaboration en champ distant] pour les distances supérieures à \mynum[mm]{14}.
					\end{description}
					Les mesures de distances sont données dans le référentiel du monde réel.
				\end{mysubsubsection}
				\begin{mysubsubsection}[sss-exp1-CollaborationEnChampProche]{Collaboration en champ proche}
					\begin{myparagraph}[par-exp1-CollaborationEnChampProche-Caracteristiques]{Caractéristiques}
						La collaboration en champ proche, inférieures à \mynum[mm]{8}, correspond, dans l'environnement virtuel, à des distances inférieures à \mynum[\AA]{10} ce qui est environ l'envergure d'un \myglos{glo-Residu}\footnote{\og \AA \fg désigne l'\myangstrom qui est une unité de mesure telle que $\mynum[\AA]{1} = \mynum[m]{e-10}$}.
						\mynum{8}~\myglos*{glo-Binome} sur \mynum{12} sont concernés par cette catégorie (\myglos*{glo-Binome} \mygroup{5}, \mygroup{6}, \mygroup{7}, \mygroup{8}, \mygroup{9}, \mygroup{10}, \mygroup{11} et \mygroup{12}).
						Étant donné la distance inférieure à \mynum[\AA]{10}, les \myglos*{glo-Binome} concernés manipulent en collaboration étroite sur les mêmes \myglos*{glo-Residu}.
						Ces \myglos*{glo-Binome} se caractérisent par une forte affinité ($\mu = 4$) : ce sont des collègues ou des amis \myref*{fig-exp1-AffiniteEntreLesSujetsPourChaqueBinome}.
						D'après la \myref{fig-exp1-TempsDeRealisationEntreLesSujetsPourChaqueBinome}, ces \myglos*{glo-Binome} obtiennent des temps de réalisation de la tâche moyens.
					\end{myparagraph}
					\begin{myparagraph}[par-exp1-CollaborationEnChampProche-PartageDeLaTache]{Partage de la tâche}
						La \myref{fig-exp1-ForceMoyenneEtDifferenceDeForceEntreLesSujetsPourChaqueBinome} montre de fortes disparités entre les \myglos*{glo-Binome} concernant la force moyenne appliquée pendant la manipulation.
						Des observations pendant l'expérimentation ont permis d'identifier deux stratégies adoptées par les sujets : \og par contrôle \fg où les deux sujets effectuent la même action pour obtenir un meilleur contrôle sur les structures manipulées; \og par guidage \fg où un des deux sujets indique à son partenaire la déformation à effectuer ou la position à atteindre.
						Le partage des tâches est donc très différent selon les initiatives de chacun des sujets.
						Cependant, les \myglos*{glo-Binome} se distribuent mal la charge de travail comme le montre les différences importantes entre les forces appliquées par les deux sujets \myref*{fig-exp1-ForceMoyenneEtDifferenceDeForceEntreLesSujetsPourChaqueBinome}.
						En effet, seul un des deux sujets réalise une grande partie de la tâche à réaliser.
						Le second sujet joue plutôt le rôle du suiveur.
					\end{myparagraph}
					\begin{myparagraph}[par-exp1-CollaborationEnChampProche-Communication]{Communication}
						Les temps de communication verbale sur la \myref{fig-exp1-PourcentageDeTempsDeCommunicationVerbalePendantLaRechercheEtLaSelectionEntreLesSujetsPourChaqueBinome} montrent une disparité entre les sujets.
						Les \myglos*{glo-Binome} passent plus de temps à communiquer pendant la phase de recherche que pendant la phase de sélection (excepté pour \mygroup{9}) ce qui met en évidence les difficultés du travail en champ proche liées aux nombreux \myglos*{glo-ConflitDeCoordination} pendant la phase de recherche.
						En effet, les \myglos*{glo-Binome} doivent coordonner leurs mouvements de manipulation pour déplacer un \myglos{glo-Residu} et cette coordination nécessite une communication verbale importante.
						La collaboration est alors étroitement couplée mais il en résulte une perte de temps à cause du temps alloué à la communication.
						D'ailleurs, l'analyse des communications verbales a permis de mettre en évidence de nombreuses incompréhensions dans l'inter-référencement (\og Pas dans cette direction \fg, \og Pas ici mais ici \fg, \og C'est juste derrière \fg, \myetc).
						En effet, la grande complexité des tâches ainsi qu'une conscience incomplète de l'environnement et de l'état de son partenaire provoque des inter-référencements imprécis entrainant une mauvaise coordination.
						Ces \myglos*{glo-ConflitDeCoordination} et ces incompréhensions diminuent les performances globales du \myglos{glo-Binome}.
					\end{myparagraph}
				\end{mysubsubsection}
				\begin{mysubsubsection}[sss-exp1-CollaborationEnChampVoisin]{Collaboration en champ voisin}
					\begin{myparagraph}[par-exp1-CollaborationEnChampVoisin-Caracteristiques]{Caractéristiques}
						La collaboration en champ voisin, comprises entre \mynum[mm]{8} et \mynum[mm]{14}, correspond, dans l'environnement virtuel, à des distances de l'ordre de \myglos*{glo-Residu} voisins (entre \mynum[\AA]{10} et \mynum[\AA]{20}).
						\mynum{3}~\myglos*{glo-Binome} sur \mynum{12} se trouvent dans cette catégorie (\myglos*{glo-Binome} \mygroup{2}, \mygroup{3} et \mygroup{4}).
						Ces \myglos*{glo-Binome} travaillent en collaboration relativement étroite sur des \myglos*{glo-Residu} voisins.
						La \myref{fig-exp1-CouplagePhysiqueEtStructurelEntreLesResidus} montre la dépendance physique ou structurelle entre les \myglos*{glo-Residu} voisins.
						En effet, les \myglos*{glo-Residu} interagissent entre eux à travers diverses forces physiques : plus les distances sont courtes, plus les contraintes physiques sont fortes.
						La \myref{fig-exp1-AffiniteEntreLesSujetsPourChaqueBinome} montre que les \myglos*{glo-Binome} concernés ont des affinités moyennes ($\mu = 3$) : ce sont des collègues de bureau ou d'équipe ne travaillant pas forcément sur les mêmes projets.
						Ces \myglos*{glo-Binome} obtiennent de très bonnes performances sur les temps de réalisation de la tâche \myref{fig-exp1-TempsDeRealisationEntreLesSujetsPourChaqueBinome}.

						\begin{myfigure}
							\setlength{\mywidth}{10pc}
							\setlength{\myheight}{10pc}
							\psset{xunit=\mywidth,yunit=\myheight}
							\begin{myps}(-1,-0.2)(1.15,1)
								\psset{ref=c}
								\rput(0.5,0.5){\myimage[width=\mywidth,angle=90]{exp1-trp-zipper}}
								\rput(0.35,-0.1){%
									\rnode{connected-residues-label}{%
										\begin{tabular}{c}%
											Connected residues\\[-1ex]%
											\textcolor{black!70}{\scriptsize Structural dependencies}%
										\end{tabular}%
									}%
								}
								\rput(0.2,0.3){\pnode{connected-residues1}}
								\rput(0.5,0.3){\pnode{connected-residues2}}
								\rput(-0.5,0.7){%
									\rnode{close-macro-label}{%
										\begin{tabular}{c}%
											Close macrostructures\\[-1ex]%
											\textcolor{black!70}{\scriptsize Physical dependencies}%
										\end{tabular}%
									}%
								}
								\rput(0,0.9){\pnode{close-macro1}}
								\rput(-0.1,0.5){\pnode{close-macro2}}
								\nccurve[angleA=135,angleB=-90]{->}{connected-residues-label}{connected-residues1}
								\nccurve[angleA=45,angleB=-90]{->}{connected-residues-label}{connected-residues2}
								\nccurve[angleA=90,angleB=180]{->}{close-macro-label}{close-macro1}
								\nccurve[angleA=-90,angleB=180]{->}{close-macro-label}{close-macro2}
							\end{myps}
							\mycaption[fig-exp1-CouplagePhysiqueEtStructurelEntreLesResidus]{Couplage physique et structure entre les \myglosnl*{glo-Residu}}
						\end{myfigure}
					\end{myparagraph}
					\begin{myparagraph}[par-exp1-CollaborationEnChampVoisin-PartageDeLaTache]{Partage de la tâche}
						La \myref{fig-exp1-ForceMoyenneEtDifferenceDeForceEntreLesSujetsPourChaqueBinome} illustre une bonne répartition des efforts entre les deux membres du \myglos{glo-Binome}.
						En effet, la force moyenne est assez élevée par rapport à la plupart des autres \myglos*{glo-Binome} ce qui montre qu'aucun des deux sujets n'est moins actif (ce qui entraînerait une force moyenne moins élevée).
						La différence des forces moyennes quasi-nulle entre les deux sujets confirme ce résultat.
						Ceci s'explique par une bonne coordination entre les sujets pendant laquelle les deux membres du \myglos{glo-Binome} vont effectuer des actions complémentaires mais de même intensité.
						La stratégie adoptée peut être définie comme une stratégie \myemph{par manipulation complémentaire} : les deux sujets sont attentifs aux actions de leur partenaire afin d'avoir un meilleur contrôle du processus de déformation par une coordination améliorée.
					\end{myparagraph}
					\begin{myparagraph}[par-exp1-CollaborationEnChampVoisin-Communication]{Communication}
						La communication verbale est faible comme le montre la \myref{fig-exp1-TempsDeCommunicationVerbaleEntreLesSujetsPourChaqueBinome}.
						La manipulation en champ voisin permet d'être continuellement conscient des actions du partenaire (grâce à la vision périphérique) ce qui limite les communications verbales.
						Cependant, les sujets manipulent des \myglos*{glo-Residu} différents restreignant ainsi les \myglos*{glo-ConflitDeCoordination} par rapport à la collaboration en champ proche.
						De plus, la \myref{fig-exp1-PourcentageDeTempsDeCommunicationVerbalePendantLaRechercheEtLaSelectionEntreLesSujetsPourChaqueBinome} montre un nombre de \myglos*{glo-ConflitDeCoordination} plus faible pendant la phase de recherche.
						En effet, la communication verbale est nettement moins importante pendant la phase de recherche que pendant la phase de sélection.
						L'analyse des communication verbales met en évidence des phases de communication de coordination (\og Maintenant, prends ça \fg, \og peux-tu m'aider ici ? \fg, \og Bien ! \fg, \myetc).
						Les performances des \myglos*{glo-Binome} travaillant en champ voisin sont relativement élevées bien que quelques \myglos*{glo-ConflitDeCoordination} similaires à ceux rencontrés dans une collaboration en champ proche soient présents.
						Cependant, le nombre de \myglos*{glo-ConflitDeCoordination} est plus limité.
					\end{myparagraph}
				\end{mysubsubsection}
				\begin{mysubsubsection}[sss-exp1-CollaborationEnChampDistant]{Collaboration en champ distant}
					\begin{myparagraph}[par-exp1-CollaborationEnChampDistant-Caracteristiques]{Caractéristiques}
						La collaboration en champ voisin, supérieures à \mynum[mm]{14}, correspond, dans l'environnement virtuel, à des \myglos*{glo-Residu} sans interaction physique (supérieur à \mynum[\AA]{20}).
						\mynum{1}~\myglos{glo-Binome} sur \mynum{12} est concerné par cette catégorie (\myglos{glo-Binome} \mygroup{1}).
						Ce \myglos{glo-Binome} travaille de façon faiblement couplée.
						En effet, les membres de ce \myglos{glo-Binome} travaillent de façon complétement indépendante, en limitant au maximum le nombre d'interactions.
						Les affinités des membres de ce \myglos{glo-Binome} sont très faibles \myref*{fig-exp1-AffiniteEntreLesSujetsPourChaqueBinome} : les membres ne se connaissent presque pas.
						Le \myglos{glo-Binome} obtient de très mauvaises performances en ce qui concerne le temps de réalisation de la tâche comme le montre la \myref{fig-exp1-TempsDeRealisationEntreLesSujetsPourChaqueBinome}.
					\end{myparagraph}
					\begin{myparagraph}[par-exp1-CollaborationEnChampDistant-PartageDeLaTache]{Partage de la tâche}
						La \myref{fig-exp1-ForceMoyenneEtDifferenceDeForceEntreLesSujetsPourChaqueBinome} montre un effort moyen appliqué par les \myglos*{glo-Binome} peu élevé (comparé aux stratégies de collaboration en champ voisin).
						De plus, les forces moyennes appliquées par chacun des deux sujets sont très inégales.
						Il y a une mauvaise répartition de la charge de travail au sein du \myglos{glo-Binome}.
					\end{myparagraph}
					\begin{myparagraph}[par-exp1-CollaborationEnChampDistant-Communication]{Communication}
						La \myref{fig-exp1-TempsDeCommunicationVerbaleEntreLesSujetsPourChaqueBinome} montre que le temps de communication verbale est assez important.
						Cependant, le temps de réalisation étant nettement plus important, le taux de communication verbale est beaucoup plus faible que les autres \myglos*{glo-Binome} \myref*{fig-exp1-PourcentageDeTempsDeCommunicationVerbalePendantLaRechercheEtLaSelectionEntreLesSujetsPourChaqueBinome}.
						En effet, les membres du \myglos{glo-Binome} travaillent à distance et ont peu d'interactions entre eux.
						Le peu d'interaction permet de limiter le nombre de \myglos*{glo-ConflitDeCoordination} ce qui implique le peu de communication verbale comme on peut le voir sur la \myref{fig-exp1-PourcentageDeTempsDeCommunicationVerbalePendantLaRechercheEtLaSelectionEntreLesSujetsPourChaqueBinome}.
						Cette figure montre également que ce \myglos{glo-Binome} communique plus dans les phases de sélection que dans les phases de recherche.
						En effet, les phases de sélection forcent une collaboration étroite (spécificité de la tâche proposée) et favorisent les \myglos*{glo-ConflitDeCoordination}.
						Cependant, les phases de recherche permettent aux sujets de manipuler de manière distante.
						Ainsi, ils se définissent leur propre espace de travail mais également leur propre stratégie en fonction des évènement locaux à leur espace de travail.
						Pourtant, la phase de sélection nécessite une collaboration étroite et si les stratégies sont différentes, il en résulte de mauvaises performances dû au temps important pour se coordonner à nouveau.
					\end{myparagraph}
				\end{mysubsubsection}
				\begin{mysubsubsection}[sss-exp1-SyntheseDesStrategiesDeTravail]{Synthèse des stratégies de travail}
					Les \myglos*{glo-Binome} sont susceptibles d'adopter une des trois stratégies de travail vues dans les sections précédentes.
					Pour certaines, les interactions en champ distants semblent convenir mais au détriment des performances : la collaboration est quasiment inexistante.
					D'autres \myglos*{glo-Binome} interagissent en champ proches et obtiennent des performances moyennes : la collaboration est étroitement couplée mais souffre des nombreux \myglos*{glo-ConflitDeCoordination}.

					Cependant, ce sont les interactions en champ voisins qui produisent les meilleures performances.
					En effet, les \myglos*{glo-ConflitDeCoordination} sont plus limités que pour des interactions en champ proche mais la collaboration est tout de même couplée.
					Les résultats montrent à la fois de bonnes performances en terme de temps de réalisation mais aussi en terme de répartition des charges de travail tout en limitant les communication verbales.
					La plupart du temps, les communications verbales sont destinées à la résolution de \myglos*{glo-ConflitDeCoordination} : elles sont très chronophages et peuvent être évitées.
					C'est pour cette raison que nous proposerons des outils haptiques pour améliorer cette gestion des \myglos*{glo-ConflitDeCoordination} \myref*{cha-TravailCollaboratifAssisteParHaptique}.
				\end{mysubsubsection}
			\end{mysubsection}
			\begin{mysubsection}[sse-exp1-ResultatsQualitatifs]{Résultats qualitatifs}
				Les résultats qualitatifs sont constitués de deux parties.
				La première permet de déterminer les impressions des sujets concernant la collaboration, les rôles et efficacité de chacun pendant la tâche.
				La seconde partie a pour but d'évaluer la plateforme\footnote{L'échelle de notation est comprise entre \mynum{1} et \mynum{5} mais les moyennes ont été normalisées entre \mynum{0} et \mynum{4}.}.
				\begin{mysubsubsection}[sss-exp1-EvaluationDuTravailEnCollaboration]{Évaluation du travail en collaboration}
					Les résultats du questionnaire montre qu'une majorité des sujets de cette expérimentation ont apprécié et préféré la réalisation de la tâche en configuration collaborative (\mysummary{exp1-evaluation-group.tex}).
					De plus, le sentiment d'effectuer une tâche en collaboration est fort.
					L'hypothèse \myhypothesis{3} est confirmée par les sujets qui préfèrent le travail en collaboration que le travail en \myglos{glo-Monome}.

					Durant les tâches collaboratives, les sujets considèrent qu'ils ont effectivement contribués à la réalisation de la tâche (\mysummary{exp1-evaluation-help.tex}).
					Cependant, les sujets considèrent qu'ils ne se sont imposés ni en \myglos{glo-Meneur} ou ni en \myglos{glo-Suiveur} (\mysummary{exp1-evaluation-leader.tex}).
					En effet, des questions supplémentaires ont permis de mettre en évidence que chaque sujet a tendance à surestimer le rôle du partenaire ($\approx\mynum[\%]{70}$).

					Concernant la communication, les participants estiment qu'ils exploitent principalement la communication verbale (\mysummary{exp1-evaluation-verbal.tex}) et, dans une proportion plus faible mais tout de même importante, virtuelle (\mysummary{exp1-evaluation-virtual.tex}).
					En ce qui concerne la communication gestuelle, ils la considèrent quasiment inexistante (\mysummary{exp1-evaluation-gestural.tex}).

					La communication gestuelle n'est pas ou peu utilisée.
					La principale raison est la difficulté de communiquer avec des gestes lorsque les mains sont occupées par la manipulation.
					Les sujets ont rapidement adopté la désignation virtuelle qui est plus précise et plus adaptée dans les phases de désignation qui constituent la plupart des besoins de communication.
					La communication verbale reste le principal moyen de communication : c'est la manière la plus naturelle de communiquer.
					Cependant, il vient aussi en soutien de la désignation virtuelle.
					En effet, aucun outil visuel ou haptique n'a été fourni pour effectuer des désignations et le curseur ne suffit pas toujours à remplir cette mission.
				\end{mysubsubsection}
				\begin{mysubsubsection}[sss-exp1-EvaluationDuSysteme]{Évaluation du système}
					L'évaluation du système en terme d'utilisabilité est relativement satisfaisante.
					En effet, en ce qui concerne les graphismes et les effets visuels, les participants les ont trouvé accessibles (\mysummary{exp1-platform-visual-intuitive.tex}).
					De la même façon, l'utilisabilité des moyens d'interaction avec le système sont bien notés (\mysummary{exp1-platform-interaction-intuitive.tex}).
					En terme de confort d'utilisation, les effets visuels (\mysummary{exp1-platform-visual-confortable.tex}) et les interactions (\mysummary{exp1-platform-interaction-confortable.tex}) sont bien évalués également.

					Là encore, les résultats permettent de valider l'hypothèse \myhypothesis{4}.
					La plateforme est relativement bien évaluée.
					Il semble cependant nécessaire d'apporter encore des améliorations afin de répondre au mieux aux attentes des utilisateurs.

					Ces résultats sont cependant à nuancer.
					Les écart-types sont relativement élevés ce qui veut dire qu'il y a de fortes disparités dans ces notations entre les différents sujets : certains sujets se sont déclarés plutôt insatisfaits concernant le confort (visuel : \mynum{2}, interaction : \mynum{2}).
					De plus, les outils proposés pendant cette expérimentation sont relativement simples et peu envahissants.
					Des outils plus complexes, plus informatifs seraient peut-être moins intuitifs au premier abord et pourrait mener à un inconfort.
				\end{mysubsubsection}
			\end{mysubsection}
		\end{mysection}
		\begin{mysection}[sec-exp1-Conclusion]{Conclusion}
			\begin{mysubsection}[sse-exp1-ResumeDesResultats]{Résumé des résultats}
				Dans ce chapitre, nous avons observé et comparé les performances de \myglos*{glo-Monome} et de \myglos*{glo-Binome} pendant une tâche d'exploration et de sélection sur une simulation moléculaire en temps-réel.
				L'objectif était de montrer l'intérêt de la distribution des charges de travail pour l'amélioration des performances puis d'identifier les différentes stratégies de travail.
				De plus, il fallait valider la pertinence de la plateforme mise en place.

				Les approches collaboratives ont prouvé leur intérêt, notamment sur les tâches les plus complexes.
				Cependant, la complexité d'une tâche est relativement difficile à établir.
				Au-delà des facteur de position, de couleur ou de forme, le nombre d'atomes de la molécule (et donc le nombre de \myglos*{glo-Residu}) semble jouer un rôle important dans cette complexité.
				Un grand nombre d'atomes surcharge l'environnement virtuel qui difficile à appréhender.
				Un deuxième facteur de complexité constaté durant l'expérimentation, est l'amplitude des contraintes physiques de la molécule.
				Certaines parties de la molécule sont dans un état de stabilité suffisamment dense pour qu'il soit difficile d'en déformer les \myglos*{glo-Residu}.

				En observant et en analysant les différentes stratégies de travail, il ressort que les interactions en champ proche et les interactions en champ distant ne sont pas des stratégies très efficaces.
				En effet, le nombre de \myglos*{glo-ConflitDeCoordination} durant les interactions en champ proche est trop important alors que le potentiel de la collaboration est perdu dans les interactions en champ distant.
				Ce sont les interactions en champ voisin qui offre les meilleures performances, générant un bon compromis en terme de communication et de gestion des \myglos*{glo-ConflitDeCoordination}.

				Enfin, les relations sociales entre les différents membres du \myglos{glo-Binome} tiennent une place importante.
				Les résultats montrent que tout déséquilibre dans le \myglos{glo-Binome} mène à des performances dégradées ce qui rejoint les conclusions de \mycite[author]{Woolley-2010}.
			\end{mysubsection}
			\begin{mysubsection}[sse-exp1-SyntheseEtPerspectives]{Synthèse et perspectives}
				Basés sur les résultats précédents, certaines perspectives assez évidentes s'imposent et ont guidé les expérimentations qui suivent.
				Tout d'abord, pour observer en détail le travail collaboratif et ses avantages, il semble nécessaire de proposer des tâches suffisamment complexes; soit des tâches à fortes zones de contraintes \myref*{cha-DeformationCollaborativeDeMolecule}; soit la manipulation de molécules de taille importante \myref*{cha-LaDynamiqueDeGroupe}.

				Les différentes stratégies observées ont permis de mettre en évidence l'intérêt de la collaboration en champ voisin.
				Il semble nécessaire de favoriser ce type de collaboration par des tâches stimulantes et des outils d'interaction adaptés.

				L'évaluation qualitative par questionnaire apporte également de nombreuses réponses intéressantes.
				Tout d'abord, les sujets ont mis en avant la communication visuelle dans l'\myacro{acr-EVC} au détriment de la communication gestuelle.
				Des observations durant les phases expérimentales nous ont permis de constater que ce moyen de communication est principalement utilisé pour effectuer des désignations.
				Proposer des outils adaptés aux contraintes de la désignation en environnement complexe semble être une réponse pertinente.

				Enfin, ces évaluations qualitatives ont permis de confirmer une utilisabilité acceptable de l'\myacro{acr-EVC} \myShaddock.
				Des améliorations sont cependant nécessaires en ce qui concerne le rendu visuel et les interactions.
				De nombreux sujets ont par exemple demandé une mise en surbrillance du \myglos{glo-Residu} survolé avec le curseur.
				Une assistance haptique pour la sélection (en percevant les cibles de manière haptique) est également une des améliorations possibles.
			\end{mysubsection}
		\end{mysection}
	\end{mychapter}
	\begin{mychapter}[cha-DeformationCollaborativeDeMolecule]{Déformation collaborative de molécule}
		\begin{mysection}[sec-exp2-Introduction]{Introduction}
			La précédente expérimentation nous a permis d'étudier les premières \myacro*{acr-PCV} que sont l'exploration et la sélection.
			Afin de compléter notre étude, nous souhaitons à présent nous intéresser au processus de déformation.
			En effet, la déformation est une tâche permettant de stimuler les actions coordonnées pour une collaboration étroitement couplée.
			Actuellement, des environnements virtuels existent pour manipuler des molécules rigides pour effectuer un \myglos{glo-DockingMoleculaire} comme les travaux de \mycite[author]{Levine-1997} ou encore de \mycite[author]{Ferey-2008a}.
			Cependant, afin d'effectuer un \myglos{glo-DockingMoleculaire} avancé, il est nécessaire de pouvoir déformer la molécule.
			Ceci est rendu possible par l'avénement des simulations moléculaires interactives en temps-réel, notamment avec \myacro{acr-IMD} développé par \mycite[author]{Stadler-1997} ou encore \mycite[author]{Rossi-2007}.
			Plus récemment, \mycite[author]{Delalande-2009} ont également amené une pierre à l'édifice avec \myMDDriver pour permettre une simulation moléculaire en temps-réel basée sur différents moteurs de simulation (\myacro{acr-NAMD} ou \myGromacs).
			Puis, \mycite[author]{Delalande-2010} améliorent la manipulation et la déformation interactive par l'utilisation d'une interface haptique.

			Dans ce chapitre, nous souhaitons étudier la pertinence d'une configuration collaborative pour appréhender la déformation d'une molécule.
			De plus, nous aborderons la question de l'apprentissage au sein d'un \myglos{glo-Binome}.
			En effet, certains éléments de la première expérimentation semble indiquer qu'une configuration collaborative stimule l'apprentissage, que ce soit pour l'utilisation des outils, de la plateforme ou encore pour la tâche à réaliser.
		\end{mysection}
		\begin{mysection}[sec-exp2-DeformationCollaborativeEnEnvironnementVirtuel]{Déformation collaborative en environnement virtuel}
			\begin{mysubsection}[sse-exp2-TravauxExistants]{Travaux existants}
				L'utilisation de retours haptiques pour la déformation d'objets flexibles n'est pas une idée nouvelle.
				Par exemple, \mycite[author]{Shen-2006} proposent une solution pour déformer des objets non-rigides à l'aide de retour haptique.
				Les objets concernés sont de faible complexité, comme des sphères par exemple.
				Puis, les travaux de thèse \mycite[author]{Peterlik-2009} abordent les déformations de tissus cellulaires.
				Là encore, les tâche proposées sont de faible complexité et n'ont pas d'application concrète.

				Cependant, afin d'effectuer des déformations complexes, certains chercheurs se sont intéressés aux processus de déformation collaboratifs dans les \myacro*{acr-EVC}.
				\mycite[author]{Sumengen-2007} proposent une plateforme permettant la déformation de maillages destinés à des simulations d'objets déformables (tissus, organes, \myetc) dans un \myacro{acr-EVC}; la simulation de l'objet déformable est partitionnée en fonction des zones de travail des différents utilisateurs.
				De son côté, \mycite[author]{Tang-2010a} proposent une plateforme client/serveur de déformation collaborative de maillages.
				La présentation de ces deux plateformes de collaboration distante se focalisent principalement sur les contraintes techniques sans aborder les contraintes de la collaboration.
				\mycite[author]{Muller-2006} développent le logiciel \myClayWorks, complété plus tard par \mycite[author]{Gorlatch-2009}, permettant la sculpture virtuelle sur glaise.
				Dans cette étude, les problématiques d'accès exclusif à certains objets ou à certaines parties d'un objet sont brièvement évoquées afin de faciliter la coordination des différents acteurs.

				Tous les travaux présentés ci-dessus proposent une déformation collaborative distante où chaque utilisateur effectue une déformation dans une région restreinte où lui seul peut agir.
				De cette manière, les contraintes liées à la collaboration entre les acteurs sont en grande partie évitées.
				Pour cette raison, ce sont principalement les contraintes techniques de la collaboration qui sont abordées mais ni les conflits entre utilisateurs, ni les aspects de communication ne sont évoqués.
			\end{mysubsection}
			\begin{mysubsection}[sse-exp2-Objectifs]{Objectifs}
				Ce chapitre nous permettra d'aborder les problématiques de déformation collaborative.
				La déformation d'une molécule nécessite de positionner certains éléments de manière précise.
				Cette tâche nécessite plus de précision que l'exploration et la sélection car les cibles doivent être déplacées vers un endroit défini.
				Nous souhaitons comparer les performances sur une tâche nécessitant de la coordination pour les étapes de déformation.

				L'étude met en jeu un nombre de ressources fixe pour la tâche de déformation pour comparer une distribution des ressources (configuration \myglos*{glo-Monomanuel} en \myglos{glo-Binome}) à une mutualisation des ressources (configuration \myglos*{glo-Bimanuel} en \myglos{glo-Monome}).
				En effet, la première étude nous a permis de mettre en avant les contraintes d'une configuration collaborative en terme de temps de communication.
				Paradoxalement, les utilisateurs qui manipulent seuls sont confrontés à une charge de travail importante.
				En fournissant un nombre de ressources fixe (deux outils de déformation), nous souhaitons comparer la capacité de coordination d'un \myglos{glo-Binome}, aux capacités de traitement d'un \myglos{glo-Monome} face à une importante charge de travail.
				Dans notre hypothèse, nous pensons que les \myglos*{glo-Binome} en configuration \myglos*{glo-Monomanuel} sont plus performants que les \myglos*{glo-Monome} en configuration \myglos*{glo-Bimanuel}.

				Dans un second temps, nous souhaitons définir un lien entre la complexité de la tâche et le nombre de sujets impliqués.
				En effet, les tâches complexes fournissent une charge de travail très importante; plus cette charge de travail est importante et plus les \myglos*{glo-Monome} devraient éprouver des difficultés à traiter l'ensemble de cette charge.
				Nous émettons l'hypothèse que les tâches les plus complexes, représentant la plus grande charge de travail, obtiendront de meilleures performances avec la configuration collaborative.

				Enfin, cette seconde étude va nous permettre d'observer l'effet du travail collaboratif sur l'apprentissage.
				Nous comparons les performances des \myglos*{glo-Monome} et des \myglos*{glo-Binome} concernant la réalisation d'une même tâche répétée plusieurs fois.
				Nous supposons que la \myglos{glo-FacilitationSociale} \myref*{sse-sota-LaFacilitationSociale} et la communication qui a lieu lors d'un travail collaboratif va permettre aux \myglos*{glo-Binome} d'appréhender plus rapidement la plateforme, les outils ou encore la tâche.

				Les objectifs sont résumés sous forme d'hypothèses dans la \myref{sse-met-exp2-Hypotheses}.
			\end{mysubsection}
		\end{mysection}
		\begin{mysection}[sec-exp2-PresentationDeLExperimentation]{Présentation de l'expérimentation}
			\begin{mysubsection}[sse-exp2-DescriptionDeLaTache]{Description de la tâche}
				La tâche proposée est la déformation de molécules complexes dans un \myacro{acr-EVC}.
				L'objectif est de modifier la conformation initiale d'une molécule pour atteindre une conformation stable, une tâche relativement proche du \myglos{glo-DockingMoleculaire}.

				Trois molécules sont utilisées dans le cadre de cette expérimentation.
				\myPrion est une molécule très complexe et sera simplement utilisé dans la phase d'entraînement.
				\myTRPZIPPER et \myTRPCAGE seront chacune utilisée dans deux scénarios distincts.
				Ces molécules sont détaillées dans la \myref{sse-pro-ListeDesMolecules}.

				Afin de pouvoir évaluer la déformation effectuée, un score est affiché en temps-réel en haut de l'écran \myref*{fig-exp2-AffichageDeLaMoleculeADeformerEtDeLaMoleculeCible}.
				Le score affiché est le \myacro{acr-RMSD} \myref*{equ-RMSD} qui mesure l'écart géométrique entre deux conformations d'une même molécule.
				\begin{equation}\label{equ-RMSD}
					\mathrm{RMSD}\left(\mathbf{c},\mathbf{m}\right) = \sqrt{\frac{1}{N}\sum_{i=1}^{N}\mynorm{c_i - m_i}^2}
				\end{equation}
				où $N$~est le nombre total d'atomes et~$c_i$, $m_i$~sont respectivement les atomes~$i$ de la molécule à comparer $\mathbf{c}$ et de la molécule modèle $\mathbf{m}$.

				\begin{myfigure}
					\psset{unit=0.08\textwidth}
					\begin{myps}(0,0)(12,9)
						\rput[bl](1,0){\myimage[width=0.6\textwidth]{exp2-trp-zipper}}
						\rput[bl](6.2,5){\myimage[width=5cm,angle=0]{exp2-red-cursor}}
						\rput[bl](8.5,0.5){\myimage[width=3.5cm,angle=-20]{exp2-green-cursor}}
						\psframe*[linecolor=red](0,8)(12,9)
						\psframe*[linecolor=green](0,8)(2,9)
						\rput(6,8.5){\textcolor{white}{\bfseries\sffamily\LARGE Score RMSD}}
						\psframe[linewidth=1pt,linecolor=black](0,0)(12,9)
					\end{myps}
					\mycaption[fig-exp2-AffichageDeLaMoleculeADeformerEtDeLaMoleculeCible]{Affichage de la molécule à déformer et de la molécule cible}
				\end{myfigure}

				Les scénarios de distinguent par différents critères de complexité :
				\begin{description}
					\item[Niveau de déformation] Deux niveaux de déformation sont proposés : inter-moléculaire et intra-moléculaire \myref*{sse-sota-RechercheDeSolutionsDeDockingMoleculaire};
					\item[Nombre d'atomes] C'est le nombre total d'atomes que contient la molécule à manipuler;
					\item[\myGlosnl{glo-Residu} libre] C'est le nombre de \myglos*{glo-Residu} de la molécules non fixés dans la simulation qui sont déformables;
					\item[Cassure] Elles représentent les jointures entre les structures secondaires, formant des courbures prononcées difficiles à maintenir en place;
					\item[Champ de force] C'est l'intensité des forces dans les zones de déformation; il exprime l'énergie minimum nécessaire à déployer pour atteindre l'objectif et se traduit par trois niveaux (\myemph{faible}, \myemph{moyen} et \myemph{fort}).
				\end{description}

				Basé sur ces définitions, quatre scénarios sont proposés dont les critères de complexité sont résumés dans la \myref{tab-exp2-ParametresDeComplexiteDesTaches} :
				\begin{description}
					\item[Scénario~\myscenario{1a}]
						Cette tâche concerne la manipulation de la molécule \myTRPZIPPER à l'échelle inter-moléculaire.
						Un \myglos{glo-Residu} à l'extrémité de la chaîne carbonée\footnote{La molécule forme une chaîne carbonée; il s'agit ici d'une des extrémités de cette chaîne carbonée.} est fixé afin d'\myemph{ancrer} la molécule dans la scène virtuelle et éviter que la molécule sorte du champ de vision.
						Les onze autres \myglos*{glo-Residu} de cette molécule sont libres de mouvement ce qui en fait une molécule assez flexible avec un champ de force à contrainte moyenne.
						La forme général de la molécule peut être comparée à un \myform{V} : la chaîne de \myglos*{glo-Residu} de la molécule contient une cassure.
						La difficulté de ce scénario réside dans la nécessité de maintenir les \myglos*{glo-Residu} déjà placés pendant que le reste de la molécule est déformée.
					\item[Scénario~\myscenario{1b}]
						Cette tâche concerne la manipulation de la molécule \myTRPCAGE à l'échelle inter-moléculaire.
						Comme le scénario~\myscenario{1a}, elle contient un \myglos{glo-Residu} fixe à une extrémité.
						Les dix-neuf autres \myglos*{glo-Residu} sont libres de mouvement ce qui en fait une molécule assez flexible avec un champ de force moyennement contraint.
						La forme général de la molécule peut être comparée à un \myform{W} : la chaîne de \myglos*{glo-Residu} de la molécule contient deux cassures.
						Ce scénario est plus difficile que le scénario~\myscenario{1a} car le nombre d'atomes à placer est plus élevé et qu'il est nécessaire de maintenir en place deux cassures.
					\item[Scénario~\myscenario{2a}]
						Cette tâche concerne la manipulation de la molécule \myTRPZIPPER à l'échelle intra-moléculaire.
						Seulement trois \myglos*{glo-Residu} sont laissés libres et les autres \myglos*{glo-Residu} sont fixés.
						Le champ de force au sein de la zone de déformation pour cette molécule est très faible et aucune cassure n'est à former.
						Cependant, la difficulté de ce scénario réside dans la précision de la déformation nécessaire.
						En effet, plutôt que de modifier la position des \myglos*{glo-Residu}, ce scénario nécessite la modification de l'orientation d'un \myglos{glo-Residu} avec une précision accrue dans la sélection et la déformation des atomes.
					\item[Scénario~\myscenario{2b}]
						Cette tâche concerne la manipulation de la molécule \myTRPCAGE à l'échelle intra-moléculaire.
						Seulement six \myglos*{glo-Residu} sont laissés libres et les autres \myglos*{glo-Residu} sont fixés.
						Le champ de force au sein de la zone de déformation est très important et l'énergie qu'il est nécessaire de déployer est importante.
						Cette déformation ne peut être réalisée qu'avec la manipulation simultanée et coordonnée de deux \myglos*{glo-Residu} afin de recréer la cassure.
				\end{description}

				\begin{mytable}
					\mycaption[tab-exp2-ParametresDeComplexiteDesTaches]{Paramètres de complexité des tâches}
					\begin{mytabular}{^>{\bfseries}p{12em}-C-C-C-C}
						\mytoprule
						\myrowstyle{\bfseries}
						Scénario                      & \myscenario{1a} & \myscenario{1b} & \myscenario{2a} & \myscenario{2b} \\
						\mymiddlerule[\heavyrulewidth]
						Niveau de déformation         & inter           & inter           & intra           & intra           \\
						\mymiddlerule
						Nombre d'atomes               & \mynum{218}     & \mynum{304}     & \mynum{218}     & \mynum{304}     \\
						\mymiddlerule
						\myGlosnl*{glo-Residu} libres & \mynum{11}      & \mynum{19}      & \mynum{3}       & \mynum{7}       \\
						\mymiddlerule
						Cassure                       & \mynum{1}       & \mynum{2}       & \mynum{0}       & \mynum{1}       \\
						\mymiddlerule
						Champ de force                & Moyen           & Moyen           & Faible          & Fort            \\
						\mybottomrule
					\end{mytabular}
				\end{mytable}
			\end{mysubsection}
			\begin{mysubsection}[sse-exp2-SpecificitesDuProtocoleExperimental]{Spécificités du protocole expérimental}
				L'expérimentation, basée sur le dispositif expérimental présenté dans le \myref{cha-pro-DispositifExperimental}, a subi quelques modifications qui seront détaillées dans les sections suivantes.
				Un résumé de la methode expérimentale se trouve dans la \myref{tab-exp2-SyntheseDeLaProcedureExperimentale} qu'on pourra retrouver de manière détaillée dans la \myref{sec-met-exp2-SecondeExperimentation}.
				\begin{mysubsubsection}[sss-exp2-MaterielUtilise]{Matériel utilisé}
					Pour cette seconde expérimentation, une unique modification a été effectuée par rapport à la plateforme de base présentée dans la \myref{sec-pro-MaterielExperimental}.
					L'outil \mytool{grab} qui était assuré par un \myOmni et permettant d'orienter la molécule a été remplacer par une souris~\myThreeD \mySpaceNavigator.
					Cette souris ne nécessite pas d'ordinateur supplémentaire et peut être connectée directement à la machine principale sur laquelle \myacro{acr-VMD} est exécuté.
					Elle est placée sur la table entre les deux sujets et chaque sujet peut l'utiliser comme il le souhaite : nous créons ainsi artificiellement un point de conflit pour l'accès à cet outil.
					L'objectif est de stimuler les interactions.

					Les \myref{fig-exp2-SchemaDuDispositifExperimental} et \myref{fig-exp2-PhotographieDuDispositifExperimental} illustrent par un schéma et une photographie le dispositif expérimental.

					\begin{myfigure}
						\myimage[width=0.9\textwidth]{exp2-schema}
						\mycaption[fig-exp2-SchemaDuDispositifExperimental]{Schéma du dispositif expérimental}
					\end{myfigure}
					\begin{myfigure}
						\myimage[width=0.9\textwidth]{exp2-photo}
						\mycaption[fig-exp2-PhotographieDuDispositifExperimental]{Photographie du dispositif expérimental}
					\end{myfigure}
				\end{mysubsubsection}
				\begin{mysubsubsection}[sss-exp2-VisualisationEtRepresentation]{Visualisation et représentation}
					Dans cette seconde expérimentation, quatre scénarios sont proposés et présentés dans la \myref{sse-exp2-DescriptionDeLaTache}.
					Les rendus graphiques de base sont utilisés pour afficher les molécules correspondantes \myref*{sse-pro-RepresentationDesMolecules}.

					Cependant, les sujets doivent également avoir accès aux informations concernant l'objectif : la molécule dans son état stable.
					Cette molécule est affichée avec un rendu \myNewRibbon transparent et les atomes ne sont pas affichés.
					Cet affichage très synthétique permet de ne pas surcharger la scène.
					La \myref{fig-exp2-RepresentationDeLaMoleculeTRPZIPPERPourLeScenario1A}, la \myref{fig-exp2-RepresentationDeLaMoleculeTRPCAGEPourLeScenario1B}, la \myref{fig-exp2-RepresentationDeLaMoleculeTRPZIPPERPourLeScenario2A} et la \myref{fig-exp2-RepresentationDeLaMoleculeTRPCAGEPourLeScenario2B} représentent respectivement les scénarios \myscenario{1a}, \myscenario{1b}, \myscenario{2a} et \myscenario{2b} tels qu'ils sont affichés pour la réalisation de la tâche.

					\begin{myfigure}
						\myimage[width=0.5\textwidth]{exp2-scenario1A}
						\mycaption[fig-exp2-RepresentationDeLaMoleculeTRPZIPPERPourLeScenario1A]{Représentation de la molécule \myTRPZIPPER pour le scénario~\myscenario{1A}}
					\end{myfigure}
					\begin{myfigure}
						\myimage[width=0.5\textwidth]{exp2-scenario1B}
						\mycaption[fig-exp2-RepresentationDeLaMoleculeTRPCAGEPourLeScenario1B]{Représentation de la molécule \myTRPCAGE pour le scénario~\myscenario{1B}}
					\end{myfigure}
					\begin{myfigure}
						\myimage{exp2-scenario2A}
						\mycaption[fig-exp2-RepresentationDeLaMoleculeTRPZIPPERPourLeScenario2A]{Représentation de la molécule \myTRPZIPPER pour le scénario~\myscenario{2A}}
					\end{myfigure}
					\begin{myfigure}
						\myimage{exp2-scenario2B}
						\mycaption[fig-exp2-RepresentationDeLaMoleculeTRPCAGEPourLeScenario2B]{Représentation de la molécule \myTRPCAGE pour le scénario~\myscenario{2B}}
					\end{myfigure}

					Pour finir, afin d'aider les sujets à trouver l'emplacement final du \myglos{glo-Residu} sélectionné, un affichage du \myglos{glo-Residu} correspondant est effectué sur la molécule stable.
					Ce \myglos{glo-Residu} est représenté par un rendu \myCPK coloré de la couleur du curseur du sujet concerné comme on peut le voir sur la \myref{fig-exp2-IllustrationDesRendusPourLAffichageDeLaMolecule}.
				\end{mysubsubsection}
				\begin{mysubsubsection}[sss-exp2-OutilsDeManipulation]{Outils de manipulation}
					Suite à la première expérimentation, nous avons remis en cause la présence de l'outil d'orientation de la molécule.
					En effet, bien que nécessaire dans certains cas, certains sujets n'ont pas réussi à s'approprier rapidement l'outil \mytool{grab}.

					Un bio-informaticien nous a suggéré l'utilisation d'une souris~\myThreeD, outil plus approprié pour l'orientation d'une molécule.
					En effet, les contraintes mécaniques de l'interface haptique provoquait une problématique d'interaction connue sous le nom de débrayage \mycite{Dominjon-2006}.
					La souris~\myThreeD ne souffre pas d'une telle contrainte mécanique.

					Cet outil permettant de différencier facilement les translations et les rotations, nous avons choisi de ne conserver que les \myacro*{acr-DDL} en rotation.
					Cela évite que les sujets puisse sortir la molécule de l'écran tout en conservant la possibilité de l'observer sous tous les points de vue.

					Les \myglos*{glo-Binome} étant dans une configuration \myglos*{glo-Monomanuel}, ils peuvent accéder à l'outil d'orientation de la molécule sans relâcher l'outil de déformation; ce n'est pas le cas des \myglos*{glo-Monome} en configuration \myglos*{glo-Bimanuel}.
					Afin de compenser ce désavantage et de conserver une équité entre les \myglos*{glo-Monome} et les \myglos*{glo-Binome}, une déformation ne peut pas être effectuée lorsque l'outil d'orientation est utilisé; toutes les sélections en cours sont désactivées.

					\begin{myfigure}
						\psset{unit=0.08\textwidth}
						\def\myexptwolabel(#1,#2)[#3]#4#5{\rput(#1,#2){\rnode{#3}{\textcolor{#4}{\sffamily #5}}}}
						\begin{myps}(0,0)(11,8.5)
							\rput[bl](1,0){\myimage{exp2-trp-zipper}}
							\myexptwolabel(9.4,2.6)[deformed-label]{myred}{Molécule à déformer}
							\myexptwolabel(1,5.5)[ghost-label]{myred}{Molécule cible}
							\myexptwolabel(7.1,7.3)[deformed-residue-label]{myblue}{\myGlosnl{glo-Residu} sélectionné}
							\myexptwolabel(1,3)[ghost-residue-label]{myblue}{\myGlosnl{glo-Residu} cible}
							\myexptwolabel(4.0,8.15)[fixed-residue-label]{mygray}{\myGlosnl{glo-Residu} fixe}
							\pnode(7.4,3.6){deformed}
							\pnode(1.8,4){ghost}
							\psset{linecolor=myblue}
							\cnode(6.2,5.2){1.0}{deformed-residue}
							\cnode(2.3,1.5){0.8}{ghost-residue}
							\psset{linecolor=mygray}
							\cnode(2.0,7){0.8}{fixed-residue}
							\psset{linewidth=1pt,linecolor=myred,linearc=.1,arrowsize=1pt 3,arrowinset=.2,nodesepA=3pt}
							\ncangle[angleA=90,angleB=0]{c->}{deformed-label}{deformed}
							\ncangle[angleA=-90,angleB=180,offsetA=-0.5]{c->}{ghost-label}{ghost}
							\psset{linecolor=myblue,nodesepB=0pt}
							\ncdiagg[angleA=-90,offsetA=0.5]{c->}{deformed-residue-label}{deformed-residue}
							\ncdiagg[angleA=-90,offsetA=-0.5]{c->}{ghost-residue-label}{ghost-residue}
							\ncdiagg[angleA=180,linecolor=mygray]{c->}{fixed-residue-label}{fixed-residue}
							\ncline[linewidth=10pt,linecolor=myblue,arrowsize=2pt 2,nodesepA=4pt]{C->}{deformed-residue}{ghost-residue}
						\end{myps}
						\mycaption[fig-exp2-IllustrationDesRendusPourLAffichageDeLaMolecule]{Illustration des rendus pour l'affichage de la molécule}
					\end{myfigure}
				\end{mysubsubsection}
				\begin{mytable}
					\mycaption[tab-exp2-SyntheseDeLaProcedureExperimentale]{Synthèse de la procédure expérimentale}
					\newcommand{\mytitlecolumn}[2]{%
						\multirow{#1}*{%
							\begin{minipage}{6em}%
								\raggedleft #2%
							\end{minipage}%
						}
					}
					\newlength{\exptwofirstcolumn}
					\newlength{\exptwosecondcolumn}
					\setlength{\exptwofirstcolumn}{7em}
					\setlength{\exptwosecondcolumn}{\textwidth}
					\addtolength{\exptwosecondcolumn}{-\exptwofirstcolumn}
					\addtolength{\exptwosecondcolumn}{-4\tabcolsep}
					\begin{mytabular}{>{\bfseries}p{\exptwofirstcolumn}p{\exptwosecondcolumn}}
						\mytoprule
						\mytitlecolumn{1}{Tâche}                   & Déformation d'une molécule                                                        \\
						\mymiddlerule[\heavyrulewidth]
						\mytitlecolumn{4}{Hypothèses}              & \myhypothesis{1} Amélioration des performances en \myglosnl{glo-Binome}           \\
						                                           & \myhypothesis{2} \myglosnl*{glo-Binome} plus performants sur les tâches complexes \\
						                                           & \myhypothesis{3} Apprentissage plus performant en \myglosnl{glo-Binome}           \\
						                                           & \myhypothesis{4} Les sujets préfèrent le travail en collaboration                 \\
						\mymiddlerule
						\mytitlecolumn{3}{Variables indépendantes} & \myvari{1} Nombre de sujets                                                       \\
						                                           & \myvari{2} Complexité de la tâche                                                 \\
						                                           & \myvari{3} Niveau d'apprentissage                                                 \\
						\mymiddlerule
						\mytitlecolumn{6}{Variables dépendantes}   & \myvard{1} Temps de réalisation                                                   \\
						                                           & \myvard{2} Nombre de sélections                                                   \\
						                                           & \myvard{3} Distance passive entre les espaces de travail                          \\
						                                           & \myvard{4} Distance active entre les espaces de travail                           \\
						                                           & \myvard{5} Vitesse moyenne                                                        \\
						                                           & \myvard{6} Réponses qualitatives                                                  \\
						\mymiddlerule[\heavyrulewidth]
						\multicolumn{2}{c}{%
							\small%
							\begin{tabular}{^C-C-C-C}
								\myrowstyle{\bfseries}
								\myconditions{1}{3}      & \myconditions{4}{6}      & \myconditions{7}{9}       & \myconditions{10}{12}     \\
								\mymiddlerule
								\mynum{1}~sujet          & \mynum{1}~sujet          & \mynum{2}~sujets          & \mynum{2}~sujets          \\
								\myGlosnl{glo-Bimanuel}  & \myGlosnl{glo-Bimanuel}  & \myGlosnl{glo-Monomanuel} & \myGlosnl{glo-Monomanuel} \\
								\mymiddlerule
								Scénario~\myscenario{1a} & Scénario~\myscenario{1b} & Scénario~\myscenario{1a}  & Scénario~\myscenario{1b}  \\
								Scénario~\myscenario{1b} & Scénario~\myscenario{1a} & Scénario~\myscenario{1b}  & Scénario~\myscenario{1a}  \\
								Scénario~\myscenario{2a} & Scénario~\myscenario{2b} & Scénario~\myscenario{2a}  & Scénario~\myscenario{2b}  \\
								Scénario~\myscenario{2b} & Scénario~\myscenario{2a} & Scénario~\myscenario{2b}  & Scénario~\myscenario{2a}  \\
							\end{tabular}
						} \\
						\mybottomrule
					\end{mytabular}
				\end{mytable}
			\end{mysubsection}
		\end{mysection}
		\begin{mysection}[sec-exp2-Resultats]{Résultats}
			Cette section présente et analyse l'ensemble des mesures expérimentales de cette seconde étude concernant la déformation de molécules complexes en configuration collaborative.
			Les données, confrontées à un test de \mycite[author]{Shapiro-1965}, ne sont pas distribuées selon une loi normale.
			Cependant, un test de \mycite[author]{Brown-1974} permet de confirmer l'\myglos{glo-Homoscedasticite}.
			L'analyse de la variance est alors pratiquée avec différents tests statistiques suivant les cas :
			\begin{itemize}
				\item test de \mycite[author]{Friedman-1940} pour les \myglos*{glo-VariableIntraSujets} non-paramètriques;
				\item test de \mycite[author]{Kruskal-1952} pour les \myglos*{glo-VariableInterSujets} non-paramètriques.
			\end{itemize}
			\begin{mysubsection}[sse-exp2-AmeliorationDesPerformancesEnBinome]{Amélioration des performances en \myglosnl{glo-Binome}}
				Tout d'abord, nous présentons les données accompagnées des analyses statistiques.
				Puis nous discuterons l'évolution des performances entre les \myglos*{glo-Monome} et les \myglos*{glo-Binome} pour la réalisation de cette tâche.
				\begin{mysubsubsection}[sss-exp2-AmeliorationDesPerformancesEnBinome-DonneesEtStatistiques]{Données et statistiques}
					\begin{myfigure}
						\psset{xunit=0.272108844\textwidth,yunit=0.02cm}
						\begin{myps}(-0.45,-55)(2,210)
							\myaxes(0,2){nombre de sujets}(0,200)[50]{temps~(s)}
							\myboxplot{exp2-time-group.csv}
						\end{myps}
						\mycaption[fig-exp2-TempsDeRealisationEnFonctionDuNombreDeSujets]{Temps de réalisation en fonction du nombre de sujets}
					\end{myfigure}

					La \myref{fig-exp2-TempsDeRealisationEnFonctionDuNombreDeSujets} présente le temps de réalisation \myvard{1} en fonction du nombre de sujets \myvari{1}.
					L'analyse montre qu'il y a un effet significatif du nombre de sujets \myvari{1} sur le temps de réalisation \myvard{1} (\myanova{exp2-time-group-anova.tex}) avec une diminution de \myratio{exp2-time-group-ratio.tex}.

					\begin{myfigure}
						\psset{xunit=0.272108844\textwidth,yunit=1.25cm}
						\begin{myps}(-0.45,-0.85)(2,3.5)
							\myaxes(0,2){distance}(0,3)[1]{distance~(mm)}
							\myboxplot{exp2-diff-activepassive-group.csv}
							\mylegend{\myleg{\myglosnl{glo-Monome}}{myblue}\myand\myleg{\myglosnl{glo-Binome}}{myblue!70}}
						\end{myps}
						\mycaption[fig-exp2-DistancePassiveEtActiveEntreLesEffecteursTerminauxEnFonctionDuNombreDeSujets]{Distance passive et active entre les \myglosnl*{glo-EffecteurTerminal} en fonction du nombre de sujets}
					\end{myfigure}

					La \myref{fig-exp2-DistancePassiveEtActiveEntreLesEffecteursTerminauxEnFonctionDuNombreDeSujets} présente la distance passive \myvard{3} et active \myvard{4} entre les \myglos*{glo-EffecteurTerminal} en fonction du nombre de sujets \myvari{1}.
					L'analyse montre qu'il n'y a pas d'effet significatif du nombre de sujets \myvari{1} sur la distance passive \myvard{3} (\myanova{exp2-diff-activepassive-group-anova-passive.tex}).
					Cependant, l'analyse montre qu'il y a un effet significatif du nombre de sujets \myvari{1} sur la distance active \myvard{4} (\myanova{exp2-diff-activepassive-group-anova-active.tex}); la distance active est supérieure de \myratio{exp2-diff-activepassive-group-ratio-active.tex} pour les \myglos*{glo-Binome}.

					On peut également comparer les distances passive et active en fonction du nombre de sujets \myvari{1}.
					L'analyse montre qu'il y a un effet significatif de la nature de la distance (passive ou active) au sein d'un \myglos{glo-Monome} (\myanova{exp2-diff-activepassive-group-anova-monome.tex}); la distance active est inférieure de \myratio{exp2-diff-activepassive-group-ratio-monome.tex}.
					Par contre, l'analyse ne montre pas d'effet significatif de la nature de la distance (passive ou active) au sein d'un \myglos{glo-Binome} (\myanova{exp2-diff-activepassive-group-anova-binome.tex}).

					\begin{myfigure}
						\psset{xunit=0.272108844\textwidth,yunit=0.075cm}
						\begin{myps}(-0.45,-15)(2,58)
							\myaxes(0,2){nombre de sujets}(0,50)[10]{nombre de sélections~(nb)}
							\myboxplot{exp2-numsel-group-dominant.csv}
							\mylegend{\myleg{main dominante}{myblue}\myand\myleg{main dominée}{myblue!70}}
						\end{myps}
						\mycaption[fig-exp2-NombreDeSelectionsParMainDominanteDomineeEnFonctionDuNombreDeSujets]{Nombre de sélections par main dominante/dominée en fonction du nombre de sujets}
					\end{myfigure}

					La \myref{fig-exp2-NombreDeSelectionsParMainDominanteDomineeEnFonctionDuNombreDeSujets} présente le nombre de sélections par main dominante/dominée \myvard{2} en fonction du nombre de sujets \myvari{1}.
					Les \myglos*{glo-Binome} n'utilisant que leur main dominante, il n'y a pas de résultat pour la main dominée; le résultat utilisé est donc la somme des sélections par main dominante des membres du \myglos{glo-Binome}.
					On constate un déséquilibre du nombre de sélections entre la main dominante et la main dominée pour les \myglos*{glo-Monome}.
					En comparant la somme des mains dominée et dominante des \myglos*{glo-Monome} avec le nombre de sélections total des \myglos*{glo-Binome}, l'analyse montre qu'il y a un effet significatif du nombre de sujets \myvari{1} sur le nombre total de sélections \myvard{2} (\myanova{exp2-numsel-group-dominant-anova-cumulative.tex}); le nombre de sélections est supérieur de \myratio{exp2-numsel-group-dominant-ratio-cumulative.tex} pour les \myglos*{glo-Binome}.

					Le nombre de sélections pour la main dominante comptabilise les sélections des deux sujets du \myglos{glo-Binome} contrairement aux \myglos*{glo-Monome} : ceci explique le nombre plus élevé de sélections en \myglos*{glo-Binome}.
					Cependant, si on compare le nombre moyen de sélections par sujet seulement pour la main dominante, l'analyse montre qu'il n'y a pas d'effet significatif du nombre de sujets \myvari{1} sur le nombre de sélections \myvard{2} (\myanova{exp2-numsel-group-dominant-anova-dominant.tex}).

					\begin{myfigure}
						\psset{xunit=0.272108844\textwidth,yunit=2.5cm}
						\begin{myps}(-0.45,-0.45)(2,1.75)
							\myaxes(0,2){nombre de sujets}(0,1.5)[0.5]{speed~(mm/s)}
							\myboxplot{exp2-speed-group-dominant.csv}
							\mylegend{\myleg{main dominante}{myblue}\myand\myleg{main dominée}{myblue!70}}
						\end{myps}
						\mycaption[fig-exp2-VitesseMoyenneDeLaMainDominanteEtDomineeEnFonctionDuNombreDeSujets]{Vitesse moyenne de la main dominante et dominée en fonction du nombre de sujets}
					\end{myfigure}

					La \myref{fig-exp2-VitesseMoyenneDeLaMainDominanteEtDomineeEnFonctionDuNombreDeSujets} présente la vitesse moyenne des \myglos*{glo-EffecteurTerminal} \myvard{5} en fonction du nombre de sujets \myvari{1}.
					L'analyse montre un effet significatif du nombre de sujets \myvari{1} sur la vitesse moyenne \myvard{5} (\myanova{exp2-speed-group-dominant-anova.tex}) avec une augmentation de \myratio{exp2-speed-group-dominant-ratio.tex}.
					L'analyse montre un déséquilibre de vitesse entre la main dominante et dominée des \myglos*{glo-Monome} avec un effet significatif (\myanova{exp2-speed-group-dominant-anova-monome.tex}); la vitesse moyenne de la main dominante est supérieure de \myratio{exp2-speed-group-dominant-ratio-monome.tex} à celle de la main dominée.
					L'analyse montre également un effet significatif du nombre de sujets \myvari{1} sur la vitesse moyenne \myvard{5} de la main dominante (\myanova{exp2-speed-group-dominant-anova-dominant.tex}); la vitesse moyenne de la main dominante est supérieure de \myratio{exp2-speed-group-dominant-ratio-dominant.tex} chez les \myglos*{glo-Binome}.
				\end{mysubsubsection}
				\begin{mysubsubsection}[sss-exp2-AmeliorationDesPerformancesEnBinome-AnalyseEtDiscussion]{Analyse et discussion}
					Le premier résultat sur la \myref{fig-exp2-TempsDeRealisationEnFonctionDuNombreDeSujets} nous permet de confirmer notre hypothèse \myhypothesis{1} : les \myglos*{glo-Binome} sont plus performants que les \myglos*{glo-Monome}.
					La suite de l'analyse va mettre en avant les paramètres qui amènent un gain de performances ainsi que les scénarios les plus adaptés à la configuration de travail collaborative.

					Pour commencer, les distances moyennes entre les \myglos*{glo-EffecteurTerminal} nous permettent de constater un désequilibre de performances entre les \myglos*{glo-Monome} et les \myglos*{glo-Binome} \myref*{fig-exp2-DistancePassiveEtActiveEntreLesEffecteursTerminauxEnFonctionDuNombreDeSujets}.
					En effet, la distance passive entre les \myglos*{glo-EffecteurTerminal} est plus importante pour les \myglos*{glo-Monome} que pour les \myglos*{glo-Binome}.
					Cependant, la distance active montre un effet inverse.
					En effet, la manipulation \myglos*{glo-Bimanuel} (pour les \myglos*{glo-Monome}) constitue une charge de travail importante.
					Le sujet doit alors être capable de gérer deux \myglos*{glo-EffecteurTerminal} simultanément.
					Cette configuration a amené la plupart des sujets en \myglos{glo-Monome} à utiliser un seul \myglos{glo-EffecteurTerminal} en laissant le second sur le côté afin que le curseur ne gêne pas à l'écran.
					La main dominée n'est utilisée que dans les cas où le sujet estime que c'est absolument nécessaire pour achever la tâche.
					Le second outil de déformation étant placé sur le côté de l'écran, la mesure de distance passive mesurée n'est pas représentative de l'espace de travail où le sujet est actif.

					La distance active permet d'éviter ce biais de mesure.
					En effet, cette mesure ne prend en compte que les phases d'activité d'un \myglos{glo-EffecteurTerminal}.
					On constate alors que les \myglos*{glo-Binome} couvrent un plus grand espace de travail \myref*{fig-exp2-DistancePassiveEtActiveEntreLesEffecteursTerminauxEnFonctionDuNombreDeSujets}.
					Les \myglos*{glo-Monome} couvrent un espace de travail plus restreint car ils ne peuvent pas focaliser visuellement sur plusieurs zone de travail simultanément.
					Par conséquent, les deux \myglos*{glo-EffecteurTerminal} se trouvent toujours proche de la zone de manipulation, dans la zone de focus visuel du sujet.

					La \myref{fig-exp2-NombreDeSelectionsParMainDominanteDomineeEnFonctionDuNombreDeSujets} confirme ce déséquilibre.
					En effet, on constate un nombre total de sélections plus grand pour les \myglos*{glo-Binome} (\myanova{exp2-numsel-group-dominant-mean-binome.tex}~sélections) que pour les \myglos*{glo-Monome} (\myanova{exp2-numsel-group-dominant-mean-monome.tex}~sélections).
					Là encore, le sujet effectuant la tâche en \myglos{glo-Monome} n'exploite pas pleinement les deux outils en sa possession : la charge de travail est trop importante.
					En effet, la \myacro{acr-TRM}, proposée par \mycite[author]{Wickens-1984}, considère que la gestion de plusieurs ressources pour la même modalité est impossible.
					Cependant, les analyses statistiques montre que l'outil utilisé par la main dominante obtient un taux d'utilisation identique entre les \myglos*{glo-Monome} et les \myglos*{glo-Binome}.
					Les \myglos*{glo-Binome} en configuration \myglos*{glo-Monomanuel} répartissent correctement la charge de travail entre les deux ressources disponibles ce qui n'est pas le cas des \myglos*{glo-Monome}.

					Cependant, on constate tout de même que l'outil utilisé par la main dominante n'est pas tout à fait aussi performant chez les \myglos*{glo-Monome} que chez les \myglos*{glo-Binome}.
					En effet, l'analyse montre une différence significative sur la mesure de la vitesse.
					La configuration \myglos*{glo-Bimanuel} provoque une séquentialité dans les actions du sujet : il manipule avec un outil, puis avec l'autre mais rarement les deux en même temps.
					Cette séquentialité a pour effet des pauses régulières pour chaque outil ce qui explique une vitesse moyenne moins élevée.

					Cette section nous a permis de constater que le travail en \myglos{glo-Binome} permet de meilleures performances que le travail en \myglos{glo-Monome}.
					Une analyse plus détaillé a mis en avant la difficulté du travail en configuration \myglos*{glo-Bimanuel} : la charge de travail à assumer avec deux outils est trop importante.
					Cette difficulté a pour effet de fortement dégrader le taux d'utilisation de l'outil associé à la main dominée.
					On constate également une légère baisse de l'utilisation de l'outil associé à la main dominante.
					Pour résumer, il est préférable de distribuer les ressources disponibles (outil de déformation dans notre cas) entre plusieurs sujets.
				\end{mysubsubsection}
			\end{mysubsection}
			\begin{mysubsection}[sse-exp2-EvolutionDesPerformancesEnFonctionDeLaComplexiteDeLaTache]{Évolution des performances en fonction de la complexité de la tâche}
				Dans cette section, nous nous intéressons plus précisément à la complexité de la tâche et à son influence sur le nombre de sujets.
				Les données et les analyses statistiques, présentées en premier, sont discutées par la suite.
				\begin{mysubsubsection}[sss-exp2-EvolutionDesPerformancesEnFonctionDeLaComplexiteDeLaTache-DonneesEtStatistiques]{Données et statistiques}
					\begin{myfigure}
						\psset{xunit=0.222222222\textwidth,yunit=0.015cm}
						\begin{myps}(-0.5,-75)(4,290)
							\myaxes(0,4){scénario}(0,250)[50]{temps~(s)}
							\myboxplot{exp2-time-task.csv}
						\end{myps}
						\mycaption[fig-exp2-TempsDeRealisationDesScenarios]{Temps de réalisation des scénarios}
					\end{myfigure}

					La \myref{fig-exp2-TempsDeRealisationDesScenarios} présente le temps de réalisation \myvard{1} (temps cumulé des \myglos*{glo-Monome} et des \myglos*{glo-Binome}) en fonction de la complexité de la tâche \myvari{2}.
					L'analyse montre un effet significatif de la complexité de la tâche \myvari{2} sur le temps de réalisation \myvard{1} (\myanova{exp2-time-task-anova.tex}).
					Un test post-hoc de \mycite[author]{Mann-1947} avec une correction de \mycite[author]{Holm-1979} permet de trier les scénarios en deux classes de complexité : $\left\{\myscenario{1a}, \myscenario{2a}\right\}$ et $\left\{\myscenario{1b}, \myscenario{2b}\right\}$.

					\begin{myfigure}
						\psset{xunit=0.222222222\textwidth,yunit=0.0125cm}
						\begin{myps}(-0.5,-90)(4,300)
							\myaxes(0,4){scénario}(0,250)[50]{temps~(s)}
							\myboxplot{exp2-time-task-group.csv}
							\mylegend{\myleg{\myglosnl{glo-Monome}}{myblue}\myand\myleg{\myglosnl{glo-Binome}}{myblue!70}}
						\end{myps}
						\mycaption[fig-exp2-TempsDeRealisationDesScenariosEnFonctionDuNombreDeSujets]{Temps de réalisation des scénarios en fonction du nombre de sujets}
					\end{myfigure}

					La \myref{fig-exp2-TempsDeRealisationDesScenariosEnFonctionDuNombreDeSujets} présente le temps de réalisation \myvard{1} des différents scénarios \myvari{2} en fonction du nombre de sujets \myvari{1}.
					En regroupant les scénarios par classe de complexité, l'analyse montre qu'il n'y a pas d'effet significatif du nombre de sujets \myvari{1} sur le temps de réalisation \myvard{1} pour les scénarios \myscenario{1a} et \myscenario{2a} (\myanova{exp2-time-task-group-anova-trpzipper.tex}).
					Cependant, l'analyse montre un effet significatif du nombre de sujets \myvari{1} sur le temps de réalisation \myvard{1} pour les scénarios \myscenario{1b} et \myscenario{2b} (\myanova{exp2-time-task-group-anova-trpcage.tex}); le temps de réalisation des \myglos*{glo-Binome} est inférieur de \myratio{exp2-time-task-group-ratio-trpcage.tex}.

					\begin{myfigure}
						\psset{xunit=0.222222222\textwidth,yunit=0.04cm}
						\begin{myps}(-0.5,-30)(4,115)
							\myaxes(0,4){scénario}(0,100)[25]{nombre de sélection~(nb)}
							\myboxplot{exp2-numsel-task-group.csv}
							\mylegend{\myleg{\myglosnl{glo-Monome}}{myblue}\myand\myleg{\myglosnl{glo-Binome}}{myblue!70}}
						\end{myps}
						\mycaption[fig-exp2-NombreDeSelectionsDeChaqueScenarioEnFonctionDuNombreDeSujets]{Nombre de sélections de chaque scénario en fonction du nombre de sujets}
					\end{myfigure}

					La \myref{fig-exp2-NombreDeSelectionsDeChaqueScenarioEnFonctionDuNombreDeSujets} présente le nombre de sélections \myvard{2} des différents scénarios \myvari{2} en fonction du nombre de sujets \myvari{1}.
					En regroupant les scénarios par classe de complexité, l'analyse montre un effet significatif du nombre de sujets \myvari{1} sur le nombre de sélections \myvard{2} pour les scénarios \myscenario{1a} et \myscenario{2a} (\myanova{exp2-numsel-task-group-anova-trpzipper.tex}); le nombre de sélections est supérieur de \myratio{exp2-numsel-task-group-ratio-trpzipper.tex} pour les \myglos*{glo-Binome}.
					Cependant, l'analyse montre qu'il n'y a pas d'effet significatif du nombre de sujets \myvari{1} sur le nombre de sélections \myvard{2} pour les scénarios \myscenario{1b} et \myscenario{2b} (\myanova{exp2-numsel-task-group-anova-trpcage.tex}).

					\begin{myfigure}
						\psset{xunit=0.222222222\textwidth,yunit=1.25cm}
						\begin{mysubfigure}[\textwidth]
							\begin{myps}(-0.5,-0.9)(4,3)
								\myaxes(0,4){scénario}(0,2.5)[0.5]{distance~(mm)}
								\myboxplot{exp2-passive-task-group.csv}
								\mylegend{\myleg{\myglosnl{glo-Monome}}{myblue}\myand\myleg{\myglosnl{glo-Binome}}{myblue!70}}
							\end{myps}
							\mysubcaption[fig-exp2-DistancePassiveEtActiveEntreLesEffecteursTerminauxSurChaqueScenarioEnFonctionDuNombreDeSujets-DistancePassive]{Distance passive}
						\end{mysubfigure}
						\begin{mysubfigure}[\textwidth]
							\begin{myps}(-0.5,-0.9)(4,3)
								\myaxes(0,4){scénario}(0,2.5)[0.5]{distance~(mm)}
								\myboxplot{exp2-active-task-group.csv}
								\mylegend{\myleg{\myglosnl{glo-Monome}}{myblue}\myand\myleg{\myglosnl{glo-Binome}}{myblue!70}}
							\end{myps}
							\mysubcaption[fig-exp2-DistancePassiveEtActiveEntreLesEffecteursTerminauxSurChaqueScenarioEnFonctionDuNombreDeSujets-DistanceActive]{Distance active}
						\end{mysubfigure}
						\mycaption[fig-exp2-DistancePassiveEtActiveEntreLesEffecteursTerminauxSurChaqueScenarioEnFonctionDuNombreDeSujets]{Distance passive et active entre les \myglosnl*{glo-EffecteurTerminal} sur chaque scénario en fonction du nombre de sujets}
					\end{myfigure}

					La \myref{fig-exp2-DistancePassiveEtActiveEntreLesEffecteursTerminauxSurChaqueScenarioEnFonctionDuNombreDeSujets} présente les distances passives \myvard{3} et actives \myvard{4} des différents scénarios \myvari{2} en fonction du nombre de sujets \myvari{1}.
					En regroupant les scénarios par classe de complexité, l'analyse montre un effet significatif du nombre de sujets \myvari{1} sur la distance passive \myvard{3} pour les scénarios \myscenario{1a} et \myscenario{2a} (\myanova{exp2-passive-task-group-anova-trpzipper.tex}); la distance passive des \myglos*{glo-Binome} est inférieure de \myratio{exp2-passive-task-group-ratio-trpzipper.tex}
					On n'observe pas d'effet significatif sur les scénarios \myscenario{1b} et \myscenario{2b} (\myanova{exp2-passive-task-group-anova-trpcage.tex}).
					Cependant, on constate un effet significatif du nombre de sujets \myvari{1} sur la distance active \myvard{4} pour les scénarios \myscenario{1a} et \myscenario{2a} (\myanova{exp2-active-task-group-anova-trpzipper.tex}), avec une augmentation de la distance active pour les \myglos*{glo-Binome} de \myratio{exp2-active-task-group-ratio-trpzipper.tex}, ainsi que sur les scénarios \myscenario{1b} et \myscenario{2b} (\myanova{exp2-active-task-group-anova-trpcage.tex}) avec une augmentation de \myratio{exp2-active-task-group-ratio-trpcage.tex}.

					\begin{myfigure}
						\psset{xunit=0.222222222\textwidth,yunit=1.25cm}
						\begin{myps}(-0.5,-0.9)(4,3)
							\myaxes(0,4){scénario}(0,2.5)[0.5]{vitesse~(mm/s)}
							\myboxplot{exp2-speed-task-group.csv}
							\mylegend{\myleg{\myglosnl{glo-Monome}}{myblue}\myand\myleg{\myglosnl{glo-Binome}}{myblue!70}}
						\end{myps}
						\mycaption[fig-exp2-VitesseMoyenneSurChaqueScenarioEnFonctionDuNombreDeSujets]{Vitesse moyenne sur chaque scénario en fonction du nombre de sujets}
					\end{myfigure}

					La \myref{fig-exp2-VitesseMoyenneSurChaqueScenarioEnFonctionDuNombreDeSujets} présente la vitesse moyenne \myvard{5} des différents scénarios \myvari{2} en fonction du nombre de sujets \myvari{1}.
					En regroupant les scénarios par classe de complexité, l'analyse montre un effet significatif du nombre de sujets \myvari{1} sur la vitesse moyenne \myvard{5} pour les scénarios \myscenario{1a} et \myscenario{2a} (\myanova{exp2-speed-task-group-anova-trpzipper.tex}); la vitesse moyenne des \myglos*{glo-Binome} est supérieure de \myratio{exp2-speed-task-group-ratio-trpzipper.tex}.
					De même, l'analyse montre un effet significatif du nombre de sujets \myvari{1} sur la vitesse moyenne \myvard{5} pour les scénarios \myscenario{1b} et \myscenario{2b} (\myanova{exp2-speed-task-group-anova-trpcage.tex}); la vitesse moyenne des \myglos*{glo-Binome} est supérieure de \myratio{exp2-speed-task-group-ratio-trpcage.tex}.
				\end{mysubsubsection}
				\begin{mysubsubsection}[sss-exp2-EvolutionDesPerformancesEnFonctionDeLaComplexiteDeLaTache-AnalyseEtDiscussion]{Analyse et discussion}
					L'analyse du temps de réalisation des différentes tâches ainsi que la \myref{tab-exp2-ParametresDeComplexiteDesTaches} nous permet de classifier ces tâches par niveau de complexité : les scénarios \myscenario{1a} et \myscenario{2a} sont relativement simples alors que les scénarios \myscenario{1b} et \myscenario{2b} sont complexes.
					En effet, les scénarios \myscenario{1a} et \myscenario{2a} concernent la molécule \myTRPZIPPER contenant peu d'atomes et de \myglos*{glo-Residu} à déformer.
					Par contre, les scénarios \myscenario{1b} et \myscenario{2b}, dont le nombre d'atomes et de \myglos*{glo-Residu} libres est plus important, est constitué de champ de force à fortes contraintes physiques et nécessité la formation de plusieurs cassures.

					En observant les différences de performances entre les \myglos*{glo-Monome} et les \myglos*{glo-Binome} sur la \myref{fig-exp2-TempsDeRealisationDesScenariosEnFonctionDuNombreDeSujets}, on constate que la configuration collaborative n'amélioré ses performances que dans le cas des tâches complexes.
					Ces tâches ont la particularité de nécessiter une coordination entre les deux outils de déformation.
					En effet, en observant la \myref{fig-exp2-DistancePassiveEtActiveEntreLesEffecteursTerminauxSurChaqueScenarioEnFonctionDuNombreDeSujets-DistancePassive}, l'analyse de la distance active montre une différence significative entre les \myglos*{glo-Monome} et les \myglos*{glo-Binome} pour les scénarios simples.

					Selon les résultats de la section précédente \myref*{sse-exp2-AmeliorationDesPerformancesEnBinome}, les \myglos*{glo-Monome} ont tendance à délaisser le deuxième outil.
					L'outil délaissé augmente ainsi la valeur de la distance passive mesurée en étant mis à l'écart.
					En observant seulement les scénarios simples \myscenario{1a} et \myscenario{2a}, on constate que la distance passive des \myglos*{glo-Monome} est plus importante que celle des \myglos*{glo-Binome}.
					On en conclue que la complexité de ces scénarios n'oblige pas à effectuer une manipulation \myglos*{glo-Bimanuel} et que la tâche peut être achevée avec un seul outil.
					Il y a donc peu d'intérêt à effectuer ces tâches peu complexe en configuration collaborative puisqu'aucune amélioration significative des performances n'est constatée.

					Pour les scénarios complexes, l'analyse ne montre pas de différence significative de la distance passive entre les \myglos*{glo-Monome} et les \myglos*{glo-Binome}.
					Pour ces scénarios, l'utilisation du deuxième outil est nécessaire et malgré la charge de travail importante que cela représente pour les \myglos*{glo-Monome}, la tâche est réalisée à l'aide des deux outils (configuration \myglos*{glo-Bimanuel}).
					La configuration \myglos*{glo-Monomanuel} adoptée dans ce cas par les \myglos*{glo-Binome} permet de meilleures performances \myref*{fig-exp2-TempsDeRealisationDesScenariosEnFonctionDuNombreDeSujets} pour une distance active similaire \myref*{fig-exp2-DistancePassiveEtActiveEntreLesEffecteursTerminauxSurChaqueScenarioEnFonctionDuNombreDeSujets-DistanceActive}.
					En effet, l'espace de travail couvert par les \myglos*{glo-Monome} est identique à celui des \myglos*{glo-Binome} mais leur incapacité à traiter cette charge de travail supplémentaire les rend moins performants.

					L'analyse du nombre de sélections vient appuyer ces conclusions.
					En effet, les \myglos*{glo-Monome} effectuent moins de sélections que les \myglos*{glo-Binome} dans la réalisation des scénarios simples.
					Cependant, on comptabilise un nombre de sélections similaire entre les \myglos*{glo-Monome} et les \myglos*{glo-Binome} sur les scénarios complexes avec des performances moins élevées pour les \myglos*{glo-Monome}.

					Dans cette section, nous avons montré que les améliorations de performances des \myglos*{glo-Binome} par rapport aux \myglos*{glo-Monome} étaient très liées à la complexité de la tâche.
					En effet, sur des tâches de faible complexité, les \myglos*{glo-Monome} obtiennent de bonnes performances (malgré une manipulation \myglos*{glo-Monomanuel}) pendant que les \myglos*{glo-Binome} souffrent de \myglos*{glo-ConflitDeCoordination} : les performances sont similaires.
					Cependant, dans le cas de tâches complexes, les \myglos*{glo-ConflitDeCoordination} ne sont pas suffisamment préjudiciables et la collaboration permet d'obtenir de meilleures performances que le travail individuel.
					Dans la section précédente, nous avons montré que la configuration \myglos*{glo-Bimanuel} ne permet pas d'égaler les performances d'un travail en collaboration.
					Complétons cette conclusion par le fait qu'elle est surtout vraie pour les scénarios complexes.
				\end{mysubsubsection}
			\end{mysubsection}
			\begin{mysubsection}[sse-exp2-AmeliorationDeLApprentissagePourLesBinomes]{Amélioration de l'apprentissage pour les \myglosnl*{glo-Binome}}
				Nous étudions l'effet du nombre de sujets sur l'apprentissage dans cette section.
				Les graphiques accompagnés d'analyses statistiques sont présentés puis discutés dans un second temps.
				\begin{mysubsubsection}[sss-exp2-AmeliorationDeLApprentissagePourLesBinomes-DonneesEtStatistiques]{Données et statistiques}
					\begin{myfigure}
						\psset{xunit=0.285714286\textwidth,yunit=0.0125cm}
						\begin{myps}(-0.45,-90)(3,250)
							\myaxes(0,3){essai}(0,200)[50]{temps~(s)}
							\myboxplot{exp2-time-try.csv}
						\end{myps}
						\mycaption[fig-exp2-TempsDeRealisationDeChaqueEssai]{Temps de réalisation de chaque essai}
					\end{myfigure}

					La \myref{fig-exp2-TempsDeRealisationDeChaqueEssai} présente le temps de réalisation \myvard{1} des différents essais \myvari{3}.
					L'analyse montre un effet significatif du numéro de l'essai \myvari{3} sur le temps de réalisation \myvard{1} (\myanova{exp2-time-try-anova.tex}).
					Un test post-hoc de \mycite[author]{Mann-1947} avec une correction de \mycite[author]{Holm-1979} montre une évolution significative entre le premier essai et le deuxième essai (diminution de \myratio{exp2-time-try-ratio-12.tex}) ainsi qu'entre le deuxième essai et le troisième (diminution de \myratio{exp2-time-try-ratio-23.tex}).

					\begin{myfigure}
						\psset{xunit=0.285714286\textwidth,yunit=0.0125cm}
						\begin{myps}(-0.45,-90)(3,250)
							\myaxes(0,3){essai}(0,200)[50]{temps~(s)}
							\myboxplot{exp2-time-try-group.csv}
							\mylegend{\myleg{\myglosnl{glo-Monome}}{myblue}\myand\myleg{\myglosnl{glo-Binome}}{myblue!70}}
						\end{myps}
						\mycaption[fig-exp2-TempsDeRealisationDeChaqueEssaiEnFonctionDuNombreDeSujets]{Temps de réalisation de chaque essai en fonction du nombre de sujets}
					\end{myfigure}

					La \myref{fig-exp2-TempsDeRealisationDeChaqueEssaiEnFonctionDuNombreDeSujets} présente le temps de réalisation \myvard{1} des différents essais \myvari{3} en fonction du nombre de sujets \myvari{1}.
					L'analyse montre qu'il n'y a pas d'effet significatif du nombre de sujets \myvari{1} sur le temps de réalisation \myvard{1} pour le premier essai (\myanova{exp2-time-try-group-anova-try1.tex}), le deuxième essai (\myanova{exp2-time-try-group-anova-try2.tex}) ou le troisième essai (\myanova{exp2-time-try-group-anova-try3.tex}).

					Cependant, l'analyse montre un effet significatif du numéro de l'essai \myvari{3} sur le temps de réalisation \myvard{1} pour les \myglos*{glo-Monome} (\myanova{exp2-time-try-group-anova-monome.tex}) et pour les \myglos*{glo-Binome} (\myanova{exp2-time-try-group-anova-binome.tex}).
					Un test post-hoc de \mycite[author]{Mann-1947} avec une correction de \mycite[author]{Holm-1979} montre une évolution significative pour les \myglos*{glo-Monome} à partir de dernier essai avec une diminution de \myratio{exp2-time-try-group-ratio-monome.tex} alors que l'évolution est significative dès le deuxième essai pour les \myglos*{glo-Binome} avec une diminution de \myratio{exp2-time-try-group-ratio-binome.tex}.

					\begin{myfigure}
						\psset{xunit=0.285714286\textwidth,yunit=0.04cm}
						\begin{myps}(-0.5,-30)(3,115)
							\myaxes(0,3){essai}(0,100)[25]{nombre de sélections~(nb)}
							\myboxplot{exp2-numsel-try-group.csv}
							\mylegend{\myleg{\myglosnl{glo-Monome}}{myblue}\myand\myleg{\myglosnl{glo-Binome}}{myblue!70}}
						\end{myps}
						\mycaption[fig-exp2-NombreDeSelectionsDeChaqueEssaiEnFonctionDuNombreDeSujets]{Nombre de sélections de chaque essai en fonction du nombre de sujets}
					\end{myfigure}

					La \myref{fig-exp2-NombreDeSelectionsDeChaqueEssaiEnFonctionDuNombreDeSujets} présente le nombre de sélections \myvard{2} des différents essais \myvari{3} en fonction du nombre de sujets \myvari{1}.
					L'analyse montre qu'il n'y a pas d'effet significatif du nombre de sujets \myvari{1} sur le nombre de sélections \myvard{2} pour le premier essai (\myanova{exp2-numsel-try-group-anova-try1.tex}) ou le troisième essai (\myanova{exp2-numsel-try-group-anova-try3.tex}).
					Cependant, l'analyse montre un effet significatif du nombre de sujets \myvari{1} sur le nombre de sélections \myvard{2} pour le deuxième essai (\myanova{exp2-numsel-try-group-anova-try2.tex}) supérieur de \myratio{exp2-numsel-try-group-ratio-try2.tex}.

					De plus, l'analyse montre qu'il n'y a pas d'effet significatif du numéro de l'essai \myvari{3} sur le nombre de sélections \myvard{2} pour les \myglos*{glo-Monome} (\myanova{exp2-numsel-try-group-anova-monome.tex}).
					Cependant, l'analyse montre un effet significatif du numéro de l'essai \myvari{3} sur le nombre de sélections \myvard{2} pour les \myglos*{glo-Binome} (\myanova{exp2-numsel-try-group-anova-binome.tex}).
					Le test post-hoc de \mycite[author]{Mann-1947} avec une correction de \mycite[author]{Holm-1979} montre une diminution significative du nombre de sélections pour les \myglos*{glo-Binome} entre le premier et le dernier essai avec une diminution de \myratio{exp2-numsel-try-group-ratio-binome.tex}.

					\begin{myfigure}
						\psset{xunit=0.285714286\textwidth,yunit=1.25cm}
						\begin{myps}(-0.5,-0.9)(3,3)
							\myaxes(0,3){essai}(0,2.5)[0.5]{distance~(mm)}
							\myboxplot{exp2-active-try-group.csv}
							\mylegend{\myleg{\myglosnl{glo-Monome}}{myblue}\myand\myleg{\myglosnl{glo-Binome}}{myblue!70}}
						\end{myps}
						\mycaption[fig-exp2-DistanceActiveEntreLesEffecteursTerminauxPourChaqueEssaiEnFonctionDuNombreDeSujets]{Distance active entre les \myglosnl*{glo-EffecteurTerminal} pour chaque essai en fonction du nombre de sujets}
					\end{myfigure}

					La \myref{fig-exp2-DistanceActiveEntreLesEffecteursTerminauxPourChaqueEssaiEnFonctionDuNombreDeSujets} présente la distance active \myvard{4} des différents essais \myvari{3} en fonction du nombre de sujets \myvari{1}.
					Étant donné le biais de mesure décrit dans la \myref{sse-exp2-AmeliorationDesPerformancesEnBinome}, la distance passive n'a pas été prise en considération.
					L'analyse montre un effet significatif du nombre de sujets \myvari{1} sur la distance active \myvard{4} pour le premier essai (\myanova{exp2-active-try-group-anova-try1.tex}), supérieur de \myratio{exp2-active-try-group-ratio-try1.tex} pour les \myglos*{glo-Binome} et pour le deuxième essai (\myanova{exp2-active-try-group-anova-try2.tex}) supérieur de \myratio{exp2-active-try-group-ratio-try2.tex} pour les \myglos*{glo-Binome} mais pas pour le troisième essai (\myanova{exp2-active-try-group-anova-try3.tex}).

					De plus, l'analyse montre qu'il n'y a pas d'effet significatif du numéro de l'essai \myvari{3} sur la distance active \myvard{4} pour les \myglos*{glo-Binome} (\myanova{exp2-active-try-group-anova-binome.tex}).
					Cependant, l'analyse montre un effet significatif du numéro de l'essai \myvari{3} sur la distance active \myvard{4} pour les \myglos*{glo-Monome} (\myanova{exp2-active-try-group-anova-monome.tex}).
					Un test post-hoc de \mycite[author]{Mann-1947} avec une correction de \mycite[author]{Holm-1979} montre une augmentation significative de \myratio{exp2-active-try-group-ratio-monome.tex} entre le premier essai et le troisième essai.

					\begin{myfigure}
						\psset{xunit=0.285714286\textwidth,yunit=1.25cm}
						\begin{myps}(-0.5,-0.9)(3,3)
							\myaxes(0,3){essai}(0,2.5)[0.5]{vitesse~(mm/s)}
							\myboxplot{exp2-speed-try-group.csv}
							\mylegend{\myleg{\myglosnl{glo-Monome}}{myblue}\myand\myleg{\myglosnl{glo-Binome}}{myblue!70}}
						\end{myps}
						\mycaption[fig-exp2-VitesseMoyennePourChaqueEssaiEnFonctionDuNombreDeSujets]{Vitesse moyenne pour chaque essai en fonction du nombre de sujets}
					\end{myfigure}

					La \myref{fig-exp2-VitesseMoyennePourChaqueEssaiEnFonctionDuNombreDeSujets} présente la vitesse moyenne \myvard{5} des différents essais \myvari{3} en fonction du nombre de sujets \myvari{1}.
					L'analyse montre un effet significatif du nombre de sujets \myvari{1} sur la vitesse moyenne \myvard{5} pour le premier essai (\myanova{exp2-speed-try-group-anova-try1.tex}) supérieur de \myratio{exp2-speed-try-group-ratio-try1.tex} pour les \myglos*{glo-Binome}, le second essai (\myanova{exp2-speed-try-group-anova-try2.tex}) supérieur de \myratio{exp2-speed-try-group-ratio-try2.tex} pour les \myglos*{glo-Binome} et le troisième essai (\myanova{exp2-speed-try-group-anova-try3.tex}) supérieur de \myratio{exp2-speed-try-group-ratio-try3.tex} pour les \myglos*{glo-Binome}.

					De plus, l'analyse montre un effet significatif du numéro de l'essai \myvari{3} sur la vitesse moyenne \myvard{5} pour les \myglos*{glo-Monome} (\myanova{exp2-speed-try-group-anova-monome.tex}) et les \myglos*{glo-Binome} (\myanova{exp2-speed-try-group-anova-binome.tex}).
					Le test post-hoc de \mycite[author]{Mann-1947} avec une correction de \mycite[author]{Holm-1979} montre dans chaque cas (\myglos{glo-Monome} et \myglos{glo-Binome}) une augmentation significative après le premier essai (augmentation de \myratio{exp2-speed-try-group-ratio-monome.tex} pour les \myglos*{glo-Monome} et de \myratio{exp2-speed-try-group-ratio-binome.tex} pour les \myglos*{glo-Binome}).
				\end{mysubsubsection}
				\begin{mysubsubsection}[sss-exp2-AmeliorationDeLApprentissagePourLesBinomes-AnalyseEtDiscussion]{Analyse et discussion}
					L'observation des temps de réalisation de la tâche \myref*{fig-exp2-TempsDeRealisationDeChaqueEssai} nous permet de caractériser un apprentissage réel sur l'ensemble des trois réalisations de la tâche.
					Le détail de l'apprentissage en fonction du nombre de sujets sur la \myref{fig-exp2-TempsDeRealisationDeChaqueEssaiEnFonctionDuNombreDeSujets} apporte cependant un point important : les \myglos*{glo-Binome} améliorent plus rapidement leurs performances que les \myglos*{glo-Monome}.
					En effet, on constate une amélioration significative des performances dès le second essai dans le cas des \myglos*{glo-Binome} alors que ce n'est que sur le dernier essai que les \myglos*{glo-Monome} montrent une évolution.
					L'amélioration plus rapide des performances chez les \myglos*{glo-Binome} suggère un apprentissage plus efficace de la tâche, des outils ou de la molécule\footnote{On observe une amélioration des performances par apprentissage mais rien ne permet de distinguer sur quels notions de l'expérimentation l'apprentissage a lieu.}.

					En observant l'évolution des variables \myvard{1} (temps de réalisation) et \myvard{2} (nombre de sélections), on constate que les \myglos*{glo-Binome} ont un apprentissage rapide.
					Le temps de réalisation décroît ainsi que le nombre de sélections ce qui n'est pas le cas des \myglos*{glo-Monome}.
					En effet, le temps de réalisation des \myglos*{glo-Monome} décroît alors que le nombre de sélections ne décroît pas.
					Au-fur-et-à-mesure des essais, les \myglos*{glo-Monome} apprennent et intègre la manipulation en configuration \myglos*{glo-Bimanuel} : ils augmentent ainsi leurs performances (diminution du temps de réalisation et du nombre de sélections de la main dominante) tout en conservant un nombre de sélections relativement constant (par une augmentation du nombre de sélections de la main dominée).

					On observe clairement l'apprentissage progressif du deuxième outil mis à disposition des \myglos*{glo-Monome} dans la \myref{fig-exp2-DistanceActiveEntreLesEffecteursTerminauxPourChaqueEssaiEnFonctionDuNombreDeSujets}.
					Alors que l'espace de travail des \myglos*{glo-Binome} reste stable sur l'ensemble des essais, celui des \myglos*{glo-Monome} s'étend au-fur-et-à-mesure des essais jusqu'à atteindre une valeur similaire à celle des \myglos*{glo-Binome}.
					En effet, seul l'apprentissage permet de s'affranchir en partie de la charge de travail importante que représente la manipulation \myglos*{glo-Bimanuel} \mycite{Wickens-1984} : avec l'apprentissage, les \myglos*{glo-Monome} sont capables de gérer un espace de travail de plus en plus grand.
					Le potentiel du deuxième outil n'est pas ignoré et est utilisé (avec la main dominée) comme un moyen de fixer un \myglos{glo-Residu} déjà placé pendant que l'autre outil déforme.
					Ceci permet de déformer une partie de la molécule tout en conservant la stabilité de la partie déjà déformée.

					En ce qui concerne les vitesses moyennes, les \myglos*{glo-Monome} comme les \myglos*{glo-Binome} s'améliorent en manipulant de plus en plus rapidement.
					Cependant, les \myglos*{glo-Binome} restent nettement plus rapides que les \myglos*{glo-Monome}.
					Cette donnée est à mettre en corrélation avec l'évolution des temps de réalisation : plus les sujets déforment rapidement, plus la tâche est réalisée rapidement..

					Dans cette section, nous avons mis en évidence les améliorations en terme d'apprentissage pour les configurations collaboratives sans distinguer les notions sur lesquelles s'appliquait l'apprentissage (molécule, outils, tâche, \myetc).
					En effet, les \myglos*{glo-Binome} atteignent des performances optimales dès le second essau tandis que les \myglos*{glo-Monome} ont besoin de plus d'apprentissage pour converger vers de bonnes performances.
					La capacité des \myglos*{glo-Binome} à communiquer, échanger et conseiller permet de mutualiser l'apprentissage et de l'accélérer.
					De plus, un \myglos{glo-Binome} peut bénéficier des connaissances spécifiques ou de l'expérience d'un des membres du \myglos{glo-Binome} et ainsi, la mutualisation des aptitudes de chacun crée une dynamique de groupe.
					La configuration \myglos*{glo-Bimanuel} offre une alternative de manipulation aux \myglos*{glo-Monome} avec une surcharge de travail importante : l'apprentissage est plus long.
				\end{mysubsubsection}
			\end{mysubsection}
			\begin{mysubsection}[sse-exp2-ResultatsQualitatifs]{Résultats qualitatifs}
				Le questionnaire est destiné à évaluer la collaboration du point du vue de l'utilisateur.

				La majorité des sujets travaillant en \myglos{glo-Binome} se sont trouvés utiles dans cette tâche de collaboration (\mysummary{exp2-evaluation-help.tex})\footnote{L'échelle de notation est comprise entre \mynum{1} et \mynum{5} mais les moyennes ont été normalisées entre \mynum{0} et \mynum{4}.}.
				Ce résultat élevé permet de vérifier que les sujets ne se sentent pas écartés et participent activement à la réalisation de la tâche.
				La collaboration peut se traduire soit par une participation active à la déformation, soit par une participation plutôt passive (échanges verbaux, conseils, remarques, \myetc).
				Dans un cas comme dans l'autre, les sujets ne sont pas isolés ce qui permet d'éviter les phénomènes de \myglos{glo-ParesseSociale}.

				Le sentiment d'avoir été \myglos{glo-Meneur} durant la réalisation de la tâche est relativement neutre (\mysummary{exp2-evaluation-leader.tex}).
				Cependant, cette question semble avoir été posée de manière incorrecte.
				En effet, les sujets ne souhaitent pas prétendre avoir été \myglos{glo-Meneur} ou chef des opérations par modestie.
				Paradoxalement, ils ne souhaitent pas non plus avouer avoir été dirigé par quelqu'un d'autre par fierté.
				D'ailleurs, on observe un écart-type relativement bas concernant cette note ce qui signifie que la majorité des sujets ont répondu de façon neutre.

				L'évaluation de la communication confirme ce qui a été observé dans la précédente expérimentation \myref*{sss-exp1-EvaluationDuTravailEnCollaboration}.
				Tout d'abord, l'importance de la communication verbale a été mise en avant (\mysummary{exp2-evaluation-verbal}).
				Par opposition, les sujets ont considéré qu'ils ne communiquaient pas par le virtuel (\mysummary{exp2-evaluation-virtual.tex}) et encore moins par la gestuelle (\mysummary{exp2-evaluation-gestural.tex}).

				La communication verbale étant la plus naturelle, il n'est pas étonnant d'obtenir de tels scores.
				La communication gestuelle est difficile étant donné que les sujets sont déjà en train de manipuler avec leurs mains.
				De plus, leur vision se focalise principalement sur le déroulement de la tâche à l'écran mais pas sur le partenaire ce qui laisse peu de place à la communication gestuelle.

				Pour finir, l'expérimentation ne proposant aucun outil permettant d'exploiter la modalité de communication virtuelle, cela explique probablement le faible taux d'investissement dans les communications par le virtuel.
				La dernière expérimentation \myref*{cha-TravailCollaboratifAssisteParHaptique} propose des outils de désignation qui vont permettre d'exploiter le potentiel de ce canal de communication.

				Pour finir, les sujets ont été interrogés sur leur configuration de travail préférée.
				Le questionnaire propose aux sujets de donner leur avis sur une configuration pour laquelle ils n'ont pas été testés.
				La configuration \myglos*{glo-Monomanuel} en \myglos{glo-Monome} (qui n'a été testée par aucun sujet) a été relativement peu choisie (\mysummary{exp2-evaluation-monome-monomanual.tex}).
				Les sujets évalués en \myglos{glo-Monome} sont mitigés sur l'intérêt d'une configuration \myglos*{glo-Monomanuel} en \myglos{glo-Binome} (\mysummary{exp2-evaluation-binome-monomanual.tex}).
				De la même façon, les sujets évalués en \myglos{glo-Binome} sont mitigés sur l'intérêt d'une configuration \myglos*{glo-Bimanuel} en \myglos{glo-Monome} (\mysummary{exp2-evaluation-monome-bimanual.tex}).
				Quoiqu'il en soit, ils ont été seulement \myratio{exp2-evaluation-preference-monome-bimanual.tex} à préférer la configuration \myglos*{glo-Bimanuel} en \myglos{glo-Monome} alors qu'ils ont été \myratio{exp2-evaluation-preference-binome-monomanual.tex} à opter pour la configuration \myglos*{glo-Monomanuel} en \myglos{glo-Binome}.
				Une petite majorité des sujets semble préférer la configuration collaborative.
			\end{mysubsection}
		\end{mysection}
		\begin{mysection}[sec-exp2-Conclusion]{Conclusion}
			\begin{mysubsection}[sse-exp2-ResumeDesResultats]{Résumé des résultats}
				Dans cette seconde expérimentation, nous avons comparé et étudié les performances de \myglos*{glo-Monome} et de \myglos*{glo-Binome}, possédant un nombre de ressources identiques, sur une tâche de déformation.
				De plus, nous avons cherché à observer l'apport de la configuration collaborative sur l'apprentissage.

				Il a été montré qu'avec un nombre de ressources déterminé (un outil d'orientation et deux outils de déformation dans notre cas), il est préférable de répartir les ressources sur plusieurs sujets.
				Cette répartition des ressources permet une meilleure distribution des charges de travail.
				En effet, la charge de travail est trop importante pour un utilisateur seul.
				La configuration collaborative, bien que souffrant de \myglos*{glo-ConflitDeCoordination}, obtient tout de même des meilleures performances.

				Deuxièmement, nous avons montré que la configuration collaborative est particulièrement performante pour les scénarios à forte complexité.
				En ce qui concerne les scénarios à faible complexité, les performances d'une configuration collaborative ne sont ni meilleures, ni moins bonnes que celle d'un manipulateur seul en configuration \myglos*{glo-Bimanuel}.
				On notera tout de même que les sujets semblent légèrement préférer la configuration collaborative.

				Le troisième résultat important concerne l'apprentissage.
				Nous avons montré que le travail en collaboration a une influence sur l'évolution de l'apprentissage.
				L'apprentissage semble être catalysé par la communication et les échanges entre les sujets.
			\end{mysubsection}
			\begin{mysubsection}[sse-exp2-SyntheseEtPerspectives]{Synthèse et perspectives}
				Nous avons vu les avantages d'une configuration collaborative sur les \myglos*{glo-Binome}.
				L'étape suivante sera l'étude du travail collaboratif sur des groupes (plus de deux sujets).
				Ceci devrait permettre d'augmenter encore le potentiel du groupe pour la gestion de charges de travail importantes.

				Pour mener une telle étude, il va falloir proposer des scénarios plus complexes.
				Cette deuxième expérimentation a montré une nouvelle fois le rôle prépondérant de la taille de la molécule dans la complexité de la tâche.
				Nous verrons que les molécules proposées dans la prochaine étude sont plus importantes en taille que celle utilisées jusqu'à présent.

				Cette deuxième expérimentation a également permis de remettre en cause la pertinence d'une manipulation en configuration \myglos*{glo-Bimanuel}.
				D'après les analyses, la charge de travail qu'apporte la gestion d'un deuxième outil de déformation est trop importante; l'outil de déformation est relativement complexe à appréhender.
				Il ne faut donc pas exclure la possibilité d'obtenir des performances acceptables en fournissant un outil simple et un outil complexe pour une manipulation \myglos*{glo-Bimanuel}.
				Nous verrons que la configuration de la dernière étude \myref*{cha-TravailCollaboratifAssisteParHaptique} propose une configuration \myglos*{glo-Bimanuel} avec un outil simple de déplacement et un outil plus complexe de désignation.

				Le questionnaire nous a également permis de mettre en avant les lacunes en ce qui concerne l'utilisation de l'environnement virtuel pour la communication.
				La dernière expérimentation sera l'occasion d'introduire des outils adaptés afin de permettre l'utilisation de cet environnement pour assister la communication.
			\end{mysubsection}
		\end{mysection}
	\end{mychapter}
	\begin{mychapter}[cha-LaDynamiqueDeGroupe]{La dynamique de groupe}
		\begin{mysection}[sec-exp3-Introduction]{Introduction}
			Après avoir étudié les différentes \myacro*{acr-PCV} sur des \myglos*{glo-Binome}, nous souhaitons élargir notre étude à des groupes.
			Le terme \myemph{groupe} est défini par \mycite[author]{Bales-1950} comme un \og ensemble d'au moins trois individus \fg.
			Ce travail en groupe implique des processus de communication différents que nous souhaitons étudier.
			L'augmentation du nombres de manipulateur nous permet également de confronter les utilisateurs à des tâches plus complexes que celles proposées dans les études précédentes.

			Selon la littérature en psychologie sociale, le travail en groupe provoque des modifications sur l'organisation et la structure \myref*{sec-sota-ApprocheCollaborativePourLesProblemesComplexes} qu'on nomme \og dynamique de groupe \fg.
			Cependant, les études sur ce sujet manquent en ce qui concerne la collaboration étroitement couplée.
			Le but de l'expérimentation exposée dans ce chapitre est d'identifier les contraintes et les stratégies de travail liées à ce type de collaboration.

			De plus, \mycite[author]{Osborn-1963} montre l'importance du \mybrainstorming\footnote{Pour la suite des développements, le mot \mybrainstorming sera utilisé à la place de sa traduction \myemph{remue-méninges} qui est peu utilisée dans la littérature scientifique.} pour améliorer les performances d'un groupe.
			Nous souhaitons évaluer la mise en place d'un dispositif de \mybrainstorming dans le cadre d'une collaboration étroitement couplée afin d'observer l'évolution des performances de groupe.
		\end{mysection}
		\begin{mysection}[sec-exp3-CollaborationDeGroupe]{Collaboration de groupe}
			\begin{mysubsection}[sse-exp3-TravauxExistants]{Travaux existants}
				Nous avons déjà vu, dans la \myref{sec-sota-ApprocheCollaborativePourLesProblemesComplexes}, les dynamiques de groupe qui se mettent en place lors d'un travail en collaboration.
				Pour mémoire, nous avons abordé les phénomènes de \myglos{glo-FacilitationSociale} \mycite{Ringelmann-1913} et de \myglos{glo-ParesseSociale} \mycite{Roethlisberger-1939}.
				La \myglos{glo-FacilitationSociale} est la tendance des collaborateurs à améliorer les performances lors d'un travail de groupe (motivation, échange de connaissances, partage de mémoire, \myetc).
				La \myglos{glo-ParesseSociale} est la tendance des collaborateurs ne pas s'impliquer activement dans le travail de groupe (isolation, peur de l'évaluation, \myetc).
				\mycite[author]{Mugny-1995} propose un état de l'art intéressant concernant la dynamique des groupes.

				Cependant, \mycite[author]{Bales-1950} identifie une différence entre les groupes de taille importante et les groupes d'une vingtaine d'individus ou moins\footnote{Selon les différentes sources, la définition d'un groupe restreint varie d'une douzaine à une vingtaine d'individus.}, qu'il nomme \og groupe restreint \fg.
				L'étude qu'il mène sur ces groupes restreints montre que quelque soit la taille du groupe, le groupe sera toujours dominé par un voire deux \myglos*{glo-Meneur}.

				Afin d'améliorer les performances de ces groupes restreints, \mycite[author]{Osborn-1963} propose de mettre en place une séance de \mybrainstorming.
				L'idée est créer une période de temps où tous les collaborateurs peuvent élaborer une stratégie sur la tâche à réaliser.
				Selon \mycite[author]{Tuckman-1965}, le \mybrainstorming permet de renforcer la cohésion sociale et d'améliorer les performances du groupe à long terme.

				Cependant, \mycite[author]{Diehl-1987} tempère les avantages du \mybrainstorming en groupe.
				En effet, il montre que cette réflexion pré-opératoire peut également être effectuée de manière individuelle et que les individus seuls profite plus de cette réflexion.
				De plus, \mycite[author]{Poole-2005} explique que le \mybrainstorming n'est pas toujours utilisé à bon escient.
				Selon l'étude qu'il propose, les groupes ont tendance à se focaliser sur les connaissances communes à tous les membres du groupe et n'aborde pas les connaissances spécifiques de chacun des membres; de cette manière, le partage des connaissances ne se réalise pas correctement.

				Comme le montre l'état de l'art, les études sur la dynamique de groupe concernant les tâches faiblement couplées sont nombreuses.
				Cependant, la généralisation de ces résultats à des tâches étroitement couplées doit donner lieu à des études plus approfondies.
				C'est l'objectif de ce chapitre.
			\end{mysubsection}
			\begin{mysubsection}[sse-exp3-Objectifs]{Objectifs}
				Dans cette troisième étude, nous souhaitons étudier le travail collaboratif étroitement couplé au sein de groupes restreints.
				L'objectif est d'identifier les nouvelles contraintes liées à ces groupes et de caractériser les stratégies qui se mettent en place.

				Dans un premier temps, nous allons étudier l'évolution des performances en fonction du nombre de sujets.
				Selon le principe de la distribution des charges de travail, nous supposons que les performances d'un groupe seront meilleures que celle d'un \myglos{glo-Binome}, malgré une augmentation du nombre de \myglos*{glo-ConflitDeCoordination}.

				De plus, en se basant sur les conclusions de \mycite[author]{Bales-1950}, nous supposons qu'au moins un \myglos{glo-Meneur} va émerger durant la réalisation des tâches de l'expérimentation.
				Nous allons observer ce phénomène en essayant d'identifier des stratégies de travail différentes entre les membres du groupes.

				\mycite[author]{Bales-1950} avait également noté que les participants éprouvent le besoin de discuter pour se connaître avant de commencer la tâche.
				C'est pourquoi, nous souhaitons mettre en place une période de \mybrainstorming pour améliorer les performances du groupe, en accord avec les conclusions obtenues par \mycite[author]{Osborn-1963}.
				Nous supposons que cette période de temps sera mise à contribution pour faire émerger les \myglos*{glo-Meneur} rapidement.

				Les objectifs sont résumés sous forme d'hypothèses dans la \myref{sse-met-exp3-Hypotheses}.
			\end{mysubsection}
		\end{mysection}
		\begin{mysection}[sec-exp3-PresentationDeLExperimentation]{Présentation de l'expérimentation}
			\begin{mysubsection}[sse-exp3-DescriptionDeLaTache]{Description de la tâche}
				La tâche proposée est de nature similaire à celle exposée dans la seconde expérimentation, il faut modifier la conformation initiale d'une molécule pour atteindre une conformation stable \myref*{sse-exp2-DescriptionDeLaTache}.
				Cependant, les scénarios proposés dans cette expérimentation sont plus complexes afin de générer plus de difficultés dans la réalisation.

				Dans cette expérimentation, la molécule \myTRPCAGE est utilisée pour la phase d'entraînement.
				Des molécules de taille importante (\myPrion et \myUbiquitin) sont utilisées pour les scénarios de déformation collaborative.
				Ces molécules sont détaillées dans la \myref{sse-pro-ListeDesMolecules}.

				Le mécanisme de sélection et d'affichage est strictement identique à la seconde expérimentation \myref*{sse-exp2-SpecificitesDuProtocoleExperimental}.
				De la même façon, le système d'évaluation basé sur le score \myacro{acr-RMSD} est identique.
				On pourra trouver la description de ces éléments dans la \myref{sse-exp2-DescriptionDeLaTache}.

				Deux scénarios sont proposés :
				\begin{description}
					\item[Scénario~\myscenario{1}]
						Basé sur la molécule \myPrion, il nécessite de replacer correctement une chaîne de \mynum{16}~\myglos*{glo-Residu} par rapport à un état stable.
						Cette chaîne se trouve en périphérie de la molécule et est soumise à de faibles contraintes physiques.
						Ce scénario permet une division en sous-tâches élémentaires présentant de faibles interactions.
						L'objectif est de proposer un niveau de collaboration faiblement couplée.
					\item[Scénario~\myscenario{2}]
						Basé sur la molécule \myUbiquitin, il nécessite de replacer correctement une chaîne de \mynum{19}~\myglos*{glo-Residu} par rapport à un modèle.
						Cette chaîne se trouve au centre de la molécule où elle est soumise à de fortes contraintes physiques, notamment le milieu de la chaîne; le contrôle précis de cette déformation est complexe.
						La réalisation de ce scénario nécessite plusieurs points de contrôle et une coordination de l'ensemble des sujets.
						L'objectif est de proposer un niveau de collaboration étroitement couplée.
				\end{description}
			\end{mysubsection}
			\begin{mysubsection}[sse-exp3-SpecificitesDuProtocoleExperimental]{Spécificités du protocole expérimental}
				Le dispositif expérimental utilisé, basé sur celui présenté dans le \myref{cha-pro-DispositifExperimental}, a été adapté pour les besoins de l'expérimentation.
				Les modifications sont présentées dans les sections qui vont suivre.
				Le protocole expérimental est détaillé dans la \myref{sec-met-exp3-TroisiemeExperimentation} avec un résumé dans la \myref{tab-exp3-SyntheseDeLaProcedureExperimentale}.
				\begin{mysubsubsection}[sss-exp3-MaterielUtilise]{Matériel utilisé}
					Dans cette expérimentation, des \myglos*{glo-Quadrinome} vont être amenés à manipuler les molécules.
					Étant donné que la plateforme de base n'est constituée que de deux outils de déformation, nous avons ajouté deux outils de déformations supplémentaires.
					Ces deux outils sont matérialisés par des \myOmni connectés à des serveurs \myacro{acr-VRPN}.
					Les outils de déformations sont placés sur la table face aux sujets.

					Dans le cas des \myglos*{glo-Binome}, les sujets seront amenés à manipuler deux outils chacun (configuration \myglos*{glo-Bimanuel}).
					Pour les \myglos*{glo-Quadrinome}, chaque sujet aura un outil à sa disposition (configuration \myglos*{glo-Monomanuel}).
					Les outils peuvent être déplacés pour que la position de travail des sujets soit la meilleure possible.

					Afin d'enregistrer les communications, une caméra vidéo \mySony (\textsc{pj50v hd}) a été placée derrière les sujets afin de filmer les sujets de dos et l'écran de vidéoprojection dans un même plan.
					La caméra enregistre également toutes les communications orales.
					Ces vidéos sont exportées et séquencées \myafortiori à l'aide du logiciel \myiMovie.

					La \myref{fig-exp3-SchemaDuDispositifExperimental} illustre le dispositif expérimental par un schéma.
					La \myref{fig-exp3-PhotographieDuDispositifExperimental} est une photographie de la salle d'expérimentation.

					\begin{myfigure}
						\myimage[width=0.9\textwidth]{exp3-schema}
						\mycaption[fig-exp3-SchemaDuDispositifExperimental]{Schéma du dispositif expérimental}
					\end{myfigure}
					\begin{myfigure}
						\myimage[width=0.9\textwidth]{exp3-photo}
						\mycaption[fig-exp3-PhotographieDuDispositifExperimental]{Photographie du dispositif expérimental}
					\end{myfigure}
				\end{mysubsubsection}
				\begin{mysubsubsection}[sss-exp3-VisualisationEtRepresentation]{Visualisation et représentation}
					La visualisation et la représentation des molécules sont identiques à l'expérimentation précédente \myref*{sss-exp2-VisualisationEtRepresentation}.
					La molécule \myPrion est utilisée pour le scénario~\myscenario{1} \myref*{fig-exp3-RepresentationDeLaMoleculePrionPourLeScenario1}; la molécule \myUbiquitin est utilisée pour le scénario~\myscenario{2} \myref*{fig-exp3-RepresentationDeLaMoleculeUbiquitinPourLeScenario2}.

					\begin{myfigure}
						\myimage{exp3-scenario1}
						\mycaption[fig-exp3-RepresentationDeLaMoleculePrionPourLeScenario1]{Représentation de la molécule \myPrion pour le scénario~\myscenario{1}}
					\end{myfigure}
					\begin{myfigure}
						\myimage{exp3-scenario2}
						\mycaption[fig-exp3-RepresentationDeLaMoleculeUbiquitinPourLeScenario2]{Représentation de la molécule \myUbiquitin pour le scénario~\myscenario{2}}
					\end{myfigure}
				\end{mysubsubsection}
				\begin{mysubsubsection}[sss-exp3-OutilsDeManipulation]{Outils de manipulation}
					Cette expérimentation faisant intervenir des \myglos*{glo-Quadrinome}, nous avons décidé d'enlever l'outil d'orientation qui était proposé jusqu'à présent.
					En effet, étant donné la configuration de la salle d'expérimentation, il est physiquement impossible de placer un outil d'orientation qui soit accessible par tous les sujets comme le préconise \mycite[author]{Buisine-2011}.
					Les scénarios ont été conçus en tenant compte de cette contrainte et sont réalisables sans avoir besoin de modifier le point de vue.

					Les outils de déformation sont identiques à ceux présentés dans la seconde expérimentation \myref*{sss-exp2-OutilsDeManipulation}.
					Chaque \myglos{glo-Residu} qu'un sujet sélectionne est mis en surbrillance sur la molécule déformable et sur la molécule stable.
				\end{mysubsubsection}
				\begin{mytable}
					\mycaption[tab-exp3-SyntheseDeLaProcedureExperimentale]{Synthèse de la procédure expérimentale}
					\newcommand{\mytitlecolumn}[2]{%
						\multirow{#1}*{%
							\begin{minipage}{6em}%
								\raggedleft #2%
							\end{minipage}%
						}
					}
					\newlength{\expthreefirstcolumn}
					\newlength{\expthreesecondcolumn}
					\setlength{\expthreefirstcolumn}{7em}
					\setlength{\expthreesecondcolumn}{\textwidth}
					\addtolength{\expthreesecondcolumn}{-\expthreefirstcolumn}
					\addtolength{\expthreesecondcolumn}{-4\tabcolsep}
					\begin{mytabular}{>{\bfseries}p{\expthreefirstcolumn}p{\expthreesecondcolumn}}
						\mytoprule
						\mytitlecolumn{1}{Tâche}                   & Déformation d'une molécule en groupe                                                   \\
						\mymiddlerule[\heavyrulewidth]
						\mytitlecolumn{3}{Hypothèses}              & \myhypothesis{1} Amélioration des performances en \myglosnl{glo-Quadrinome}            \\
						                                           & \myhypothesis{2} Un \myglosnl{glo-Meneur} émerge dans les \myglosnl*{glo-Quadrinome}   \\
						                                           & \myhypothesis{3} Les \myglosnl*{glo-Quadrinome} se structurent par le \mybrainstorming \\
						\mymiddlerule
						\mytitlecolumn{3}{Variables indépendantes} & \myvari{1} Nombre de sujets                                                            \\
						                                           & \myvari{2} Complexité de la tâche                                                      \\
						                                           & \myvari{3} Temps alloué pour le \mybrainstorming                                       \\
						\mymiddlerule
						\mytitlecolumn{5}{Variables dépendantes}   & \myvard{1} Temps de réalisation                                                        \\
						                                           & \myvard{2} Fréquence des sélections                                                    \\
						                                           & \myvard{3} Vitesse moyenne                                                             \\
						                                           & \myvard{4} Force moyenne                                                               \\
						                                           & \myvard{5} Communications verbales                                                     \\
						\mymiddlerule[\heavyrulewidth]
						\multicolumn{2}{c}{%
							\small%
							\begin{tabular}{^C-C-C-C}
								\myrowstyle{\bfseries}
								Condition \mycondition{1} & Condition \mycondition{2}         & Condition \mycondition{3} & Condition \mycondition{4}         \\
								\mymiddlerule
								\mynum{2}~sujets          & \mynum{2}~sujets                  & \mynum{4}~sujets          & \mynum{4}~sujets                  \\
								\myGlosnl{glo-Bimanuel}   & \myGlosnl{glo-Bimanuel}           & \myGlosnl{glo-Monomanuel} & \myGlosnl{glo-Monomanuel}         \\
								\mymiddlerule
								Pas de \mybrainstorming   & \mynum[mn]{1} de \mybrainstorming & Pas de \mybrainstorming   & \mynum[mn]{1} de \mybrainstorming \\
								\mymiddlerule
								Scénario~\myscenario{1}   & Scénario~\myscenario{1}           & Scénario~\myscenario{1}   & Scénario~\myscenario{1}           \\
							\end{tabular}
						} \\
						\mymiddlerule[\heavyrulewidth]
						\multicolumn{2}{c}{%
							\small%
							\begin{tabular}{^C-C-C-C}
								\myrowstyle{\bfseries}
								Condition \mycondition{5} & Condition \mycondition{6}         & Condition \mycondition{7} & Condition \mycondition{8}         \\
								\mymiddlerule
								\mynum{2}~sujets          & \mynum{2}~sujets                  & \mynum{4}~sujets          & \mynum{4}~sujets                  \\
								\myGlosnl{glo-Bimanuel}   & \myGlosnl{glo-Bimanuel}           & \myGlosnl{glo-Monomanuel} & \myGlosnl{glo-Monomanuel}         \\
								\mymiddlerule
								Pas de \mybrainstorming   & \mynum[mn]{1} de \mybrainstorming & Pas de \mybrainstorming   & \mynum[mn]{1} de \mybrainstorming \\
								\mymiddlerule
								Scénario~\myscenario{2}   & Scénario~\myscenario{2}           & Scénario~\myscenario{2}   & Scénario~\myscenario{2}           \\
							\end{tabular}
						} \\
						\mybottomrule
					\end{mytabular}
				\end{mytable}
			\end{mysubsection}
		\end{mysection}
		\begin{mysection}[sec-exp3-Resultats]{Résultats}
			\begin{mysubsection}[sse-exp3-AmeliorationDesPerformances]{Amélioration des performances}
				Dans cette section, nous cherchons à étudier l'évolution des performances en fonction du nombre de sujets impliqués dans la collaboration en commençant par présenter les graphiques et les analyses statistiques.
				\begin{mysubsubsection}[sss-exp3-AmeliorationDesPerformances-DonneesEtTestsStatistiques]{Données et tests statistiques}
					\begin{myfigure}
						\psset{xunit=0.274914089\textwidth,yunit=0.005cm}
						\begin{myps}(-0.425,-220)(2,640)
							\myaxes(0,2){scénario}(0,600)[100]{temps~(s)}
							\myboxplot{exp3-time-molecule.csv}
						\end{myps}
						\mycaption[fig-exp3-TempsDeRealisationDesScenarios]{Temps de réalisation des scénarios}
					\end{myfigure}

					La \myref{fig-exp3-TempsDeRealisationDesScenarios} présente le temps de réalisation \myvard{1} de chaque scénario \myvari{2}.
					L'analyse montre un effet significatif des scénarios \myvari{2} sur le temps de réalisation \myvard{1} (\myanova{exp3-time-molecule-anova.tex}) supérieur de \myratio{exp3-time-molecule-ratio.tex}.

					\begin{myfigure}
						\psset{xunit=0.444444444\textwidth,yunit=0.005cm}
						\begin{myps}(-0.25,-220)(2,720)
							\myaxes(0,2){scénario}(0,600)[100]{temps~(s)}
							\myboxplot{exp3-time-molecule-group.csv}
							\mylegend{\myleg{\myglosnl{glo-Binome}}{myblue}\myand\myleg{\myglosnl{glo-Quadrinome}}{myblue!70}}
						\end{myps}
						\mycaption[fig-exp3-TempsDeRealisationDesScenariosEnFonctionDuNombreDeParticipants]{Temps de réalisation des scénarios en fonction du nombre de participants}
					\end{myfigure}

					La \myref{fig-exp3-TempsDeRealisationDesScenariosEnFonctionDuNombreDeParticipants} présente le temps de réalisation \myvard{1} de chaque scénario \myvari{2} en fonction du nombre de sujets \myvari{1}.
					L'analyse montre qu'il n'y a pas d'effet significatif du nombre de sujets \myvari{1} sur le temps de réalisation \myvard{1} du scénario~\myscenario{1} (\myanova{exp3-time-molecule-group-anova-prion.tex}).
					De la même façon, l'analyse montre qu'il n'y a pas d'effet significatif du nombre de sujets \myvari{1} sur le temps de réalisation \myvard{1} du scénario~\myscenario{2} (\myanova{exp3-time-molecule-group-anova-ubiquitin.tex}).

					\begin{myfigure}
						\psset{xunit=0.43956044\textwidth,yunit=10cm}
						\begin{myps}(-0.275,-0.11)(2,0.36)
							\myaxes(0,2){scénario}(0,0.31)[0.05]{sélections~(nb/s)}
							\myboxplot{exp3-freqsel-molecule-group.csv}
							\mylegend{\myleg{\myglosnl{glo-Binome}}{myblue}\myand\myleg{\myglosnl{glo-Quadrinome}}{myblue!70}}
						\end{myps}
						\mycaption[fig-exp3-FréquenceDesSelectionsSurLesScenariosEnFonctionDuNombreDeParticipants]{Fréquence des sélections sur les scénarios en fonction du nombre de participants}
					\end{myfigure}

					La \myref{fig-exp3-FréquenceDesSelectionsSurLesScenariosEnFonctionDuNombreDeParticipants} présente la fréquence de sélection \myvard{2} de chaque scénario \myvari{2} en fonction du nombre de sujets \myvari{1}.
					L'analyse montre qu'il n'y a pas d'effet significatif du nombre de sujets \myvari{1} sur la fréquence de sélection \myvard{2} du scénario~\myscenario{1} (\myanova{exp3-freqsel-molecule-group-anova-prion.tex}).
					De la même façon, l'analyse montre qu'il n'y a pas d'effet significatif du nombre de sujets \myvari{1} sur la fréquence de sélection \myvard{2} du scénario~\myscenario{2} (\myanova{exp3-freqsel-molecule-group-anova-ubiquitin.tex}).

					\begin{myfigure}
						\psset{xunit=0.43956044\textwidth,yunit=3cm}
						\begin{myps}(-0.275,-0.35)(2,1.45)
							\myaxes(0,2){scénario}(0,1.25)[0.25]{vitesse~(mm/s)}
							\myboxplot{exp3-speed-molecule-group.csv}
							\mylegend{\myleg{\myglosnl{glo-Binome}}{myblue}\myand\myleg{\myglosnl{glo-Quadrinome}}{myblue!70}}
						\end{myps}
						\mycaption[fig-exp3-VitesseMoyenneSurLesScenariosEnFonctionDuNombreDeParticipants]{Vitesse moyenne sur les scénarios en fonction du nombre de participants}
					\end{myfigure}

					La \myref{fig-exp3-VitesseMoyenneSurLesScenariosEnFonctionDuNombreDeParticipants} présente la vitesse moyenne \myvard{3} de chaque scénario \myvari{2} en fonction du nombre de sujets \myvari{1}.
					L'analyse montre un effet significatif du nombre de sujets \myvari{1} sur la vitesse moyenne \myvard{3} du scénario~\myscenario{1} (\myanova{exp3-speed-molecule-group-anova-prion.tex}); la vitesse des \myglos*{glo-Quadrinome} est supérieure de \myratio{exp3-speed-molecule-group-ratio-prion.tex}.
					De la même façon, l'analyse montre un effet significatif du nombre de sujets \myvari{1} sur la vitesse moyenne \myvard{3} du scénario~\myscenario{2} (\myanova{exp3-speed-molecule-group-anova-ubiquitin.tex}); la vitesse des \myglos*{glo-Quadrinome} est supérieure de \myratio{exp3-speed-molecule-group-ratio-ubiquitin.tex}.

					\begin{myfigure}
						\psset{xunit=0.449438202\textwidth,yunit=0.1cm}
						\begin{myps}(-0.225,-11)(2,36)
							\myaxes(0,2){scénario}(0,30)[5]{échanges verbaux~(nb)}
							\myboxplot{exp3-talk-molecule-group.csv}
							\mylegend{\myleg{\myglosnl{glo-Binome}}{myblue}\myand\myleg{\myglosnl{glo-Quadrinome}}{myblue!70}}
						\end{myps}
						\mycaption[fig-exp3-NombreDEchangeVerbauxSurLesScenariosEnFonctionDuNombreDeParticipants]{Nombre d'échanges verbaux sur les scénarios en fonction du nombre de participants}
					\end{myfigure}

					La \myref{fig-exp3-NombreDEchangeVerbauxSurLesScenariosEnFonctionDuNombreDeParticipants} présente le nombre d'échanges verbaux \myvard{5} de chaque scénario \myvari{2} en fonction du nombre de sujets \myvari{1}.
					L'analyse montre un effet significatif du nombre de sujets \myvari{1} sur le nombre d'échanges verbaux \myvard{5} du scénario~\myscenario{1} (\myanova{exp3-talk-molecule-group-anova-prion.tex}); le nombres d'échanges est inférieur de \myratio{exp3-talk-molecule-group-ratio-prion.tex} pour les \myglos*{glo-Quadrinome}.
					De la même façon, l'analyse montre un effet significatif du nombre de sujets \myvari{1} sur le nombre d'échanges verbaux \myvard{5} du scénario~\myscenario{2} (\myanova{exp3-talk-molecule-group-anova-ubiquitin.tex}); le nombres d'échanges est inférieur de \myratio{exp3-talk-molecule-group-ratio-ubiquitin.tex} pour les \myglos*{glo-Quadrinome}.
				\end{mysubsubsection}
				\begin{mysubsubsection}[sss-exp3-AmeliorationDesPerformances-AnalyseEtDiscussion]{Analyse et discussion}
					Les deux tâches proposées sont de natures très différentes.
					Malgré l'apprentissage, la \myref{fig-exp3-TempsDeRealisationDesScenarios} montre que la molécule \myUbiquitin a été la plus longue à réaliser malgré l'apprentissage : la tâche collaborative étroitement couplée (scénario~\myscenario{2}) est plus complexe que la tâche faiblement couplée (scénario~\myscenario{1}).
					De plus, de nombreux groupes ont atteint la limite de \mynum[mn]{10} lors de la réalisation du scénario \myscenario{2} (\myUbiquitin).
					Nous pouvons en déduire que la collaboration étroitement couplée est plus complexe à appréhender.

					L'étude précédente présentée dans le \myref{cha-RechercheCollaborativeDeResiduSurUneMolecule} a montré que les performances sont meilleures lorsque les ressources disponibles (outils de déformation) sont distribuées entre plusieurs utilisateurs et qu'il est préférable que les utilisateurs manipulent de manière \myglos*{glo-Monomanuel}.
					Pourtant, on constate sur la \myref{fig-exp3-TempsDeRealisationDesScenariosEnFonctionDuNombreDeParticipants} que les \myglos*{glo-Quadrinome} obtiennent des performances similaires aux \myglos*{glo-Binome}, indépendamment du scénario.
					Les \myglos*{glo-Binome} et les \myglos*{glo-Quadrinome} ont également effectué des fréquences de sélections similaires ce qui appuie ce résultat \myref*{fig-exp3-FréquenceDesSelectionsSurLesScenariosEnFonctionDuNombreDeParticipants}.

					Pourtant, la \myref{fig-exp3-VitesseMoyenneSurLesScenariosEnFonctionDuNombreDeParticipants} montre des différences significatives, en terme de vitesse, entre les \myglos*{glo-Binome} et les \myglos*{glo-Quadrinome}.
					L'étude exposée par \mycite[author]{Roethlisberger-1939} a mis en évidence ce phénomène de \myglos{glo-FacilitationSociale} : les utilisateurs se motivent entre eux pour réaliser la tâche.
					Il permet aux \myglos*{glo-Quadrinome} de générer une activité intense avec peu de phases de relâchement.
					La vitesse moyenne est ainsi augmentée de manière significative sur toute la durée de la tâche.

					Dans l'étude précédente, nous avons également mis en évidence la présence de \myglos*{glo-ConflitDeCoordination} chez les \myglos*{glo-Binome}.
					Ces \myglos*{glo-ConflitDeCoordination} entravent la progression de la tâche.
					Cependant, nous avions constaté que les sujets parviennent à résoudre ces conflits grâce à la communication verbale.
					Dans cette troisième expérimentation, la \myref{fig-exp3-NombreDEchangeVerbauxSurLesScenariosEnFonctionDuNombreDeParticipants} montre que le nombre d'échanges verbaux en \myglos{glo-Quadrinome} est inférieur à celui en \myglos{glo-Binome}.
					Ce résultat est surprenant étant donné que le nombre d'interactions possibles entre les sujets (et donc les \myglos*{glo-ConflitDeCoordination} potentiels) sont plus nombreux chez les \myglos*{glo-Quadrinome}.
					En effet, un \myglos{glo-ConflitDeCoordination} intervient lorsqu'au moins deux collaborateurs manipulent sur la même zone de travail.
					Les combinaisons possibles de conflits dans un \myglos{glo-Quadrinome} sont donc plus nombreuses que dans un \myglos{glo-Binome}.

					À partir d'observations effectuées durant la phase expérimentale, nous avons pu constater que certains sujets se montrent relativement silencieux, même en situation de \myglos{glo-ConflitDeCoordination}.
					La \myref{sse-exp3-DefinitionDUnMeneur} montre que la présence d'un \myglos{glo-Meneur} modifie les processus de communication verbale au sein d'un groupe.

					Dans cette section, nous n'avons constaté aucune évolution des performances entre les \myglos*{glo-Binome} et les \myglos*{glo-Quadrinome}.
					Cette conservation des performances est présente malgré un nombre potentiel de \myglos*{glo-ConflitDeCoordination} important et une communication verbale faible chez les \myglos*{glo-Quadrinome}.
					L'augmentation de la vitesse moyenne, provoquée par le phénomène de \myglos{glo-FacilitationSociale} déjà remarqué par \mycite[author]{Roethlisberger-1939}, permet d'expliquer ces performances.
					En effet, la \myglos{glo-FacilitationSociale} permet de réduire les phases d'inaction en stimulant l'intérêt des sujets pour la tâche à réaliser.
					Afin d'améliorer les performances d'un \myglos{glo-Quadrinome}, il faudrait faciliter les communications pour une gestion optimale des \myglos*{glo-ConflitDeCoordination}.
				\end{mysubsubsection}
			\end{mysubsection}
			\begin{mysubsection}[sse-exp3-UtiliteDuBrainstormingPourLaCollaboration]{Utilité du \mybrainstorming pour la collaboration}
				Cette section nous permettra d'étudier l'effet du \mybrainstorming sur les performances des sujets.
				Nous présentons tout d'abord les analyses statistiques accompagnées de graphiques puis nous discuterons ces résultats.
				\begin{mysubsubsection}[sss-exp3-UtiliteDuBrainstormingPourLaCollaboration-DonneesEtTestsStatistiques]{Données et tests statistiques}
					\begin{myfigure}
						\psset{xunit=0.274914089\textwidth,yunit=0.005cm}
						\begin{myps}(-0.425,-220)(2,640)
							\myaxes(0,2){\mybrainstorming}(0,600)[100]{temps~(s)}
							\myboxplot{exp3-time-brainstorm.csv}
						\end{myps}
						\mycaption[fig-exp3-TempsDeRealisationAvecOuSansBrainstorming]{Temps de réalisation avec ou sans \mybrainstorming}
					\end{myfigure}

					La \myref{fig-exp3-TempsDeRealisationAvecOuSansBrainstorming} présente le temps de réalisation \myvard{1} en fonction des groupes avec ou sans \mybrainstorming \myvari{3}.
					L'analyse montre un effet significatif du \mybrainstorming \myvari{3} sur le temps de réalisation \myvard{1} (\myanova{exp3-time-brainstorm-anova.tex}) inférieur de \myratio{exp3-time-brainstorm-ratio.tex}.

					\begin{myfigure}
						\psset{xunit=0.444444444\textwidth,yunit=0.005cm}
						\begin{myps}(-0.25,-220)(2,720)
							\myaxes(0,2){\mybrainstorming}(0,600)[100]{temps~(s)}
							\myboxplot{exp3-time-brainstorm-group.csv}
							\mylegend{\myleg{\myglosnl{glo-Binome}}{myblue}\myand\myleg{\myglosnl{glo-Quadrinome}}{myblue!70}}
						\end{myps}
						\mycaption[fig-exp3-TempsDeRealisationDesScenariosEnFonctionDesGroupesAvecOuSansBrainstorming]{Temps de réalisation des scénarios en fonction des groupes avec ou sans \mybrainstorming}
					\end{myfigure}

					La \myref{fig-exp3-TempsDeRealisationDesScenariosEnFonctionDesGroupesAvecOuSansBrainstorming} présente le temps de réalisation \myvard{1} pour les groupes avec ou sans \mybrainstorming \myvari{3} en fonction du nombre de sujets \myvari{1}.
					L'analyse montre qu'il n'y a pas d'effet significatif du \mybrainstorming \myvari{3} sur le temps de réalisation \myvard{1} des \myglos*{glo-Binome} (\myanova{exp3-time-brainstorm-group-anova-binome.tex}).
					Cependant, l'analyse montre un effet significatif du \mybrainstorming \myvari{3} sur le temps de réalisation \myvard{1} des \myglos*{glo-Quadrinome} (\myanova{exp3-time-brainstorm-group-anova-quadrinome.tex}); le temps de réalisation est diminué de \myratio{exp3-time-brainstorm-group-ratio-quadrinome.tex} avec le \mybrainstorming.

					\begin{myfigure}
						\psset{xunit=0.43956044\textwidth,yunit=10cm}
						\begin{myps}(-0.275,-0.11)(2,0.36)
							\myaxes(0,2){\mybrainstorming}(0,0.31)[0.05]{sélections~(nb/s)}
							\myboxplot{exp3-freqsel-brainstorm-group.csv}
							\mylegend{\myleg{\myglosnl{glo-Binome}}{myblue}\myand\myleg{\myglosnl{glo-Quadrinome}}{myblue!70}}
						\end{myps}
						\mycaption[fig-exp3-FréquenceDesSelectionsSurLesScenariosEnFonctionDesGroupesAvecOuSansBrainstorming]{Fréquence des sélections sur les scénarios en fonction des groupes avec ou sans \mybrainstorming}
					\end{myfigure}

					La \myref{fig-exp3-FréquenceDesSelectionsSurLesScenariosEnFonctionDesGroupesAvecOuSansBrainstorming} présente la fréquence de sélection \myvard{2} pour les groupes avec ou sans \mybrainstorming \myvari{3} en fonction du nombre de sujets \myvari{1}.
					L'analyse montre qu'il n'y a pas d'effet significatif du \mybrainstorming \myvari{3} sur la fréquence de sélection \myvard{2} des \myglos*{glo-Binome} (\myanova{exp3-freqsel-brainstorm-group-anova-binome.tex}).
					Cependant, l'analyse montre un effet significatif du \mybrainstorming \myvari{3} sur la fréquence de sélection \myvard{2} des \myglos*{glo-Quadrinome} (\myanova{exp3-freqsel-brainstorm-group-anova-quadrinome.tex}); la fréquence des sélections est diminuée de \myratio{exp3-freqsel-brainstorm-group-ratio-quadrinome.tex} avec le \mybrainstorming.

					\begin{myfigure}
						\psset{xunit=0.43956044\textwidth,yunit=3cm}
						\begin{myps}(-0.275,-0.35)(2,1.45)
							\myaxes(0,2){\mybrainstorming}(0,1.25)[0.25]{vitesse~(mm/s)}
							\myboxplot{exp3-speed-brainstorm-group.csv}
							\mylegend{\myleg{\myglosnl{glo-Binome}}{myblue}\myand\myleg{\myglosnl{glo-Quadrinome}}{myblue!70}}
						\end{myps}
						\mycaption[fig-exp3-VitesseMoyenneSurLesScenariosEnFonctionDesGroupesAvecOuSansBrainstorming]{Vitesse moyenne sur les scénarios en fonction des groupes avec ou sans \mybrainstorming}
					\end{myfigure}

					La \myref{fig-exp3-VitesseMoyenneSurLesScenariosEnFonctionDesGroupesAvecOuSansBrainstorming} présente la vitesse moyenne \myvard{3} pour les groupes avec ou sans \mybrainstorming \myvari{3} en fonction du nombre de sujets \myvari{1}.
					L'analyse montre qu'il n'y a pas d'effet significatif du \mybrainstorming \myvari{3} sur la vitesse moyenne \myvard{3} des \myglos*{glo-Binome} (\myanova{exp3-speed-brainstorm-group-anova-binome.tex}).
					De la même façon, l'analyse montre qu'il n'y a pas d'effet significatif du \mybrainstorming \myvari{3} sur la vitesse moyenne \myvard{3} des \myglos*{glo-Quadrinome} (\myanova{exp3-speed-brainstorm-group-anova-quadrinome.tex}).

					\begin{myfigure}
						\psset{xunit=0.43956044\textwidth,yunit=0.075cm}
						\begin{myps}(-0.275,-14)(2,58)
							\myaxes(0,2){\mybrainstorming}(0,50)[10]{communication~(nb)}
							\myboxplot{exp3-communication-brainstorm-group.csv}
							\mylegend{\myleg{\myglosnl{glo-Binome}}{myblue}\myand\myleg{\myglosnl{glo-Quadrinome}}{myblue!70}}
						\end{myps}
						\mycaption[fig-exp3-NombreDOrdresVerbauxSurLesScenariosEnFonctionDesGroupesAvecOuSansBrainstorming]{Nombre d'ordres verbaux sur les scénarios en fonction des groupes avec ou sans \mybrainstorming}
					\end{myfigure}

					La \myref{fig-exp3-NombreDOrdresVerbauxSurLesScenariosEnFonctionDesGroupesAvecOuSansBrainstorming} présente le nombre d'ordres verbaux \myvard{5} pour les groupes avec ou sans \mybrainstorming \myvari{3} en fonction du nombre de sujets \myvari{1}.
					L'analyse montre un effet significatif du \mybrainstorming \myvari{3} sur le nombre d'ordres verbaux \myvard{5} des \myglos*{glo-Binome} (\myanova{exp3-communication-brainstorm-group-anova-binome.tex}); le nombre d'ordres est diminué de \myratio{exp3-communication-brainstorm-group-ratio-binome.tex} avec le \mybrainstorming.
					De la même façon, l'analyse montre un effet significatif du \mybrainstorming \myvari{3} sur le nombre d'ordres verbaux \myvard{5} des \myglos*{glo-Quadrinome} (\myanova{exp3-communication-brainstorm-group-anova-quadrinome.tex}); le nombre d'ordres est diminué de \myratio{exp3-communication-brainstorm-group-ratio-quadrinome.tex} avec le \mybrainstorming.
				\end{mysubsubsection}
				\begin{mysubsubsection}[sss-exp3-UtiliteDuBrainstormingPourLaCollaboration-AnalyseEtDiscussion]{Analyse et discussion}
					La \myref{fig-exp3-NombreDOrdresVerbauxSurLesScenariosEnFonctionDesGroupesAvecOuSansBrainstorming} nous permet de constater une baisse significative du nombre d'échanges verbaux pour les sujets ayant eu une période de \mybrainstorming.
					Le \mybrainstorming permet une réflexion préalable sur la tâche afin d'aboutir à une stratégie de travail concernant différents éléments :
					\begin{itemize}
						\item répartition et distribution du travail;
						\item organisation du travail dans l'espace;
						\item organisation du travail dans le temps;
						\item identification des rôles de chaque manipulateur;
						\item prévisions sur l'évolution de l'environnement.
					\end{itemize}

					Cependant, la \myref{fig-exp3-TempsDeRealisationDesScenariosEnFonctionDesGroupesAvecOuSansBrainstorming} et la \myref{fig-exp3-FréquenceDesSelectionsSurLesScenariosEnFonctionDesGroupesAvecOuSansBrainstorming} montrent que le \mybrainstorming est surtout profitable aux \myglos*{glo-Quadrinome}.
					En effet, les \myglos*{glo-Binome} n'obtiennent aucune évolution significative des performances avec \mybrainstorming.
					De même, la \myref{fig-exp3-VitesseMoyenneSurLesScenariosEnFonctionDesGroupesAvecOuSansBrainstorming} montre que la vitesse moyenne de l'\myglos{glo-EffecteurTerminal} des \myglos*{glo-Binome} n'évolue pas.

					Deux raisons expliquent pourquoi le \mybrainstorming ne permet pas aux \myglos*{glo-Binome} d'améliorer leurs performances.
					Le \mybrainstorming étant utilisé pour définir une stratégie de travail, il permet de réduire le nombre de \myglos*{glo-ConflitDeCoordination}.
					Le nombre de \myglos{glo-ConflitDeCoordination} pour les \myglos*{glo-Binome} étant potentiellement plus faible que celui des \myglos*{glo-Quadrinome}, l'intérêt du \mybrainstorming est amoindri.
					De plus, nous avons vu que la gestion des \myglos*{glo-ConflitDeCoordination} s'effectue par une communication verbale.
					La communication en \myglos{glo-Binome} est relativement naturelle (simple conversation) alors que la communication dans un groupe de plus de trois sujets s'avère plus complexe : problème de prise de parole, conversation entre deux sujets monopolisant la parole, \myetc
					Sur ce point, les \myglos*{glo-Binome} ont peu de marge d'amélioration possible liée au \mybrainstorming contrairement aux \myglos*{glo-Quadrinome}.

					La \myref{fig-exp3-TempsDeRealisationDesScenariosEnFonctionDesGroupesAvecOuSansBrainstorming} et la \myref{fig-exp3-FréquenceDesSelectionsSurLesScenariosEnFonctionDesGroupesAvecOuSansBrainstorming} mettent en évidence l'amélioration des performances pour les \myglos*{glo-Quadrinome}.
					Nous avons vu dans la \myref{sse-exp3-AmeliorationDesPerformances} que les \myglos*{glo-Quadrinome} éprouvent des difficultés dans la résolution des \myglos*{glo-ConflitDeCoordination}.
					D'après les figures observées, le \mybrainstorming permet l'élaboration d'une stratégie et la définition des rôles de chacun des sujets.
					L'élaboration d'une stratégie réduit de façon importante le nombre de \myglos*{glo-ConflitDeCoordination} pendant la réalisation de la tâche et ainsi améliore les performances.
					De plus, la définition des rôles de chacun avant le début de la tâche permet de partager la tâche afin d'éviter le phénomène de \myglos{glo-ParesseSociale}.
					Cette conclusion rejoint la suggestion effectuée par \mycite[author]{Latane-1979} concernant l'identification des rôles.

					Dans le cas du scénario~\myscenario{1} (\myPrion), la tâche comporte un faible niveau d'interaction entre les zones à déformer; elle peut aisément être subdivisée en quatre tâches élémentaires.
					D'ailleurs, l'analyse des communications verbales lors des \mybrainstorming a montré que c'était la stratégie choisie par tous les groupes.

					Le scénario~\myscenario{2} (\myUbiquitin) comportant un fort niveau d'interaction, nécessite plus de coordination mais peut être subdivisée en deux tâches élémentaires.
					Dans ce cas, le \mybrainstorming aboutit à une scission du groupe en deux \myglos*{glo-Binome} qui réaliseront chacun une partie de la déformation.
					Ceci permet d'avoir des gestions de \myglos*{glo-ConflitDeCoordination} locaux et restreints à deux \myglos*{glo-Binome}.

					La période de \mybrainstorming permet de partitionner le temps de réflexion et le temps de manipulation.
					En effet, l'analyse des communications verbales permet de constater que les groupes n'ayant pas eu de période de \mybrainstorming tentent d'élaborer une stratégie de travail pendant la manipulation.
					Cependant, la manipulation créé une charge de travail importante.
					Les capacités de travail des sujets sont alors partagées entre la manipulation et l'élaboration d'une stratégie.
					Les sujets ne sont pas pleinement attentifs à l'élaboration de la stratégie et peuvent ne pas être attentifs au même instant que leurs collègues.
					La stratégie élaborée est alors de qualité inférieure.

					Cette section montre l'intérêt d'un \mybrainstorming pour structurer les groupes.
					En effet, elle permet d'éviter les problèmes de \myglos{glo-ParesseSociale} en stimulant l'identification de rôles pour chaque utilisateur.
					Nous verrons dans la section suivante que le groupe va s'organiser autour d'un \myglos{glo-Meneur} et que les collaborateurs seront plutôt dans un rôle de \myglos*{glo-Suiveur}.
				\end{mysubsubsection}
			\end{mysubsection}
			\begin{mysubsection}[sse-exp3-DefinitionDUnMeneur]{Définition d'un \myglosnl{glo-Meneur}}
				Cette section va définir les caractéristiques d'un \myglos{glo-Meneur}.
				Nous utiliserons les données d'un groupe représentatif pour alimenter notre propos.
				Cependant, étant donné le peu de données que représente un seul groupe, aucune analyse de la variance n'est présentée.
				\begin{mysubsubsection}[sss-exp3-DefinitionDUnMeneur-DonneesEtStatistiques]{Données et statistiques}
					\begin{myfigure}
						\psset{xunit=0.227272727\textwidth,yunit=0.25cm}
						\begin{myps}(-0.4,-4.5)(4,13)
							\myaxes(0,4){sujets de \mygroup{1}}(0,12)[4]{ordres~(nb)}
							\mybarplot{exp3-g1-talk-subject.csv}
						\end{myps}
						\mycaption[fig-exp3-NombreDOrdresDonnesParChacunDesSujetsDeG1]{Nombre d'ordres donnés par chacun des sujets de \mygroup{1}}
					\end{myfigure}

					La \myref{fig-exp3-NombreDOrdresDonnesParChacunDesSujetsDeG1} présente le nombre d'ordres donnés \myvard{5} en fonction des sujets du groupe \mygroup{1}.
					On observe que le sujet \mysubject{1} donne plus d'ordres que la moyenne (\myanalysis{exp3-g1-talk-subject-analysis.tex} de plus que la moyenne).

					\begin{myfigure}
						\psset{xunit=0.222222222\textwidth,yunit=3cm}
						\begin{myps}(-0.5,-0.4)(4,1.1)
							\myaxes(0,4){sujets de \mygroup{1}}(0,1.00)[0.25]{vitesse~(mm/s)}
							\mybarplot{exp3-g1-speed-subject.csv}
						\end{myps}
						\mycaption[fig-exp3-VitesseMoyenneDesEffecteursTerminauxPourChacunDesSujetsDeG1]{Vitesse moyenne des \myglosnl*{glo-EffecteurTerminal} pour chacun des sujets de \mygroup{1}}
					\end{myfigure}

					La \myref{fig-exp3-VitesseMoyenneDesEffecteursTerminauxPourChacunDesSujetsDeG1} présente la vitesse moyenne des \myglos*{glo-EffecteurTerminal} \myvard{3} en fonction des sujets du groupe \mygroup{1}.
					On observe que le vitesse du sujet \mysubject{1} est plus élevée que la moyenne (\myanalysis{exp3-g1-speed-subject-analysis.tex} de plus que la moyenne).

					\begin{myfigure}
						\psset{xunit=0.007017544\textwidth,yunit=1cm}
						\begin{mysubfigure}[\textwidth]
							\begin{myps}(-12.5,-1.15)(130,5.25)
								\myaxes[labels=all,ticks=all,Dx=25](0,125){temps~{(s)}}(0,5)[1]{force~(N)}
								\psscalebox{2.75 1}{\psfileplot[linecolor=myred,linewidth=0.1pt]{files/exp3-g1-force-prion-S1.csv}}
							\end{myps}
							\mysubcaption[fig-exp3-ProfilDeForceDuGroupeG1SurLaMoleculePrion-ProfilDeForceDeS1]{Profil de force de \mysubject{1}}
						\end{mysubfigure}
						\begin{mysubfigure}[\textwidth]
							\begin{myps}(-12.5,-1.15)(130,5.25)
								\myaxes[labels=all,ticks=all,Dx=25](0,125){temps~{(s)}}(0,5)[1]{force~(N)}
								\psscalebox{2.75 1}{\psfileplot[linecolor=myred,linewidth=0.1pt]{files/exp3-g1-force-prion-S2.csv}}
								\pnode(49.341,2){begin}
								\pnode(86.291,2){end}
								\ncline[linecolor=myblue,linewidth=1pt]{<->}{begin}{end}
								\ncput*{\textcolor{myblue}{\small selection}}
							\end{myps}
							\mysubcaption[fig-exp3-ProfilDeForceDuGroupeG1SurLaMoleculePrion-ProfilDeForceDeS2]{Profil de force de \mysubject{2}}
						\end{mysubfigure}
						\mycaption[fig-exp3-ProfilDeForceDuGroupeG1SurLaMoleculePrion]{Profil de force du groupe \mygroup{1} sur la molécule \myPrion}
					\end{myfigure}

					La \myref{fig-exp3-ProfilDeForceDuGroupeG1SurLaMoleculePrion} présente les profils de force \myvard{4} des sujets \mysubject{1} et \mysubject{2} du groupe \mygroup{1}.
					Chaque période où la force est maintenue constante représente une sélection \myref*{fig-exp3-ProfilDeForceDuGroupeG1SurLaMoleculePrion-ProfilDeForceDeS2}.
					On constate un profil très chaotique pour le sujet \mysubject{1} avec un grand nombre de sélections (\mynum{11}~sélections).
					Par opposition, le profil du sujet \mysubject{2} est très peu chaotique avec un petit nombre de sélections (\mynum{4}~sélections d'un temps supérieur à $\mynum[s]{10}$).
					De plus, les efforts maximaux produits par le sujet \mysubject{2} sont plus importants que ceux du \mysubject{1} (\mynum[N]{5} pour \mysubject{2} contre \mynum[N]{4} pour \mysubject{1}).
				\end{mysubsubsection}
				\begin{mysubsubsection}[sss-exp3-DefinitionDUnMeneur-AnalyseEtDiscussion]{Analyse et discussion}
					Le \myglos{glo-Meneur} est celui qui dirige les opérations.
					Dans notre population, les groupes ne possèdent \myapriori pas de \myglos{glo-Meneur} : ils sont appelés \og \myglos*{glo-StructureInformelle} \fg \mycite{Roethlisberger-1939}.
					Dans la précédente section, nous avons identifié l'émergence de rôles, en particulier grâce au \mybrainstorming.
					Parmi les rôles, on distingue le rôle du \myglos{glo-Meneur}, déjà identifié dans les travaux de \mycite[author]{Bales-1950} et les rôles de \myglos*{glo-Suiveur}.

					La \myref{fig-exp3-VitesseMoyenneDesEffecteursTerminauxPourChacunDesSujetsDeG1} et la \myref{fig-exp3-ProfilDeForceDuGroupeG1SurLaMoleculePrion-ProfilDeForceDeS1} nous permet de déterminer la stratégie de travail du \myglos{glo-Meneur}.
					En effet, on constate un grand nombre de sélections ainsi qu'une vitesse élevée.
					Le \myglos{glo-Meneur} explore l'environnement pour prendre des décisions.
					Il effectue des sélections de courte durée avec une force relativement faible.
					Ces nombreuses sélections ont pour objectif d'explorer différentes zones de la molécule pour évaluer le travail à effectuer.
					Ainsi, il distribue des tâches élémentaires de déformation aux autres sujets en fonction de son analyse.

					Par opposition, les \myglos*{glo-Suiveur} n'explorent pas l'environnement.
					En effet, la \myref{fig-exp3-ProfilDeForceDuGroupeG1SurLaMoleculePrion-ProfilDeForceDeS2} montre un nombre de sélections peu élevées mais des sélections maintenues sur une longue période de temps.
					Cette longue période de temps correspond à une déformation d'une cible vers un objectif fixe.
					La déformation a pu être proposée par le \myglos{glo-Meneur}.
					Les manipulations des suiveurs sont plutôt longues et lentes \myref*{fig-exp3-VitesseMoyenneDesEffecteursTerminauxPourChacunDesSujetsDeG1} car les déformations nécessitent une manipulation précise.
					De plus, l'effort déployé est plus important car toute l'attention du \myglos{glo-Suiveur} est portée sur la déformation.
					Le \myglos{glo-Meneur} de déploie pas autant de force car son objectif est d'explorer l'environnement, pas de déformer.

					Pour conclure cette section, le \myglos{glo-Meneur} a un rôle crucial dans la dynamique du groupe.
					C'est lui qui va définir et répartir les tâches : il élabore la stratégie du groupe.
					Cette répartition permet à chaque sujet de se faire attribuer une tâche bien identifiée.
					Cette identification est nécessaire pour éviter le phénomène de \myglos{glo-ParesseSociale} \myref*{sse-exp3-UtiliteDuBrainstormingPourLaCollaboration} et permet ainsi d'améliorer les performances.
				\end{mysubsubsection}
			\end{mysubsection}
		\end{mysection}
		\begin{mysection}[sec-exp3-Conclusion]{Conclusion}
			\begin{mysubsection}[sse-exp3-ResumeDesResultats]{Résumé des résultats}
				Cette expérience a permis d'étudier et de comparer des \myglos*{glo-Binome} en configuration \myglos*{glo-Bimanuel} avec des \myglos*{glo-Quadrinome} en configuration \myglos*{glo-Monomanuel}.
				L'objectif était d'observer l'évolution des performances en fonction du nombre de participants ainsi que les nouvelles contraintes liées aux dynamiques de groupe.

				Les résultats ont montré que l'augmentation du nombre de sujets ne permettait pas systématiquement d'améliorer les performances du groupe.
				En effet, les \myglos*{glo-Quadrinome}, bien que plus rapides dans leurs mouvements grâce au phénomène de \myglos{glo-FacilitationSociale}, obtiennent des performances identiques aux \myglos*{glo-Binome}.
				Les \myglos*{glo-Quadrinome} perdent du temps dans la résolution des \myglos*{glo-ConflitDeCoordination} qui sont plus nombreux que chez les \myglos*{glo-Binome}.

				Cependant, le \mybrainstorming permet une organisation préalable du groupe permettant de meilleures performances tout en réduisant le nombre de \myglos*{glo-ConflitDeCoordination}.
				L'élaboration d'une stratégie de travail est surtout profitable aux \myglos*{glo-Quadrinome} qui sont potentiellement confrontés à de nombreux de \myglos*{glo-ConflitDeCoordination}.

				De plus, ce \mybrainstorming permet de faire rapidement émerger les rôles au sein de cette \myglos{glo-StructureInformelle}.
				L'émergence d'un \myglos{glo-Meneur} est nécessaire pour structurer le groupe, diviser le travail et répartir les tâches.
				D'un autre côté, les \myglos*{glo-Suiveur} acceptent la présence de ce \myglos{glo-Meneur} et l'assistent dans la réalisation de la tâche.
				Le \myglos{glo-Meneur} cherche à établir une stratégie globale en explorant continuellement l'environnement.
				Les \myglos*{glo-Suiveur} effectuent plutôt des déformations locales sur des durées plus importantes.

				Cette expérimentation montre que l'augmentation du nombre d'utilisateurs est bénéfique sous certaines réserves concernant la structure du groupe.
				Par exemple, un \mybrainstorming, préalable à la réalisation de la tâche, permet de structurer un groupe.
			\end{mysubsection}
			\begin{mysubsection}[sse-exp3-SyntheseEtPerspectives]{Synthèse et perspectives}
				Nous venons de montrer l'intérêt de structurer les groupes pour effectuer des tâches collaboratives étroitement couplée.
				Sur la base des résultats obtenus, nous allons pouvoir proposer une configuration de travail intégrant l'ensemble des facteurs identifiés.

				Pour commencer, nous avons mis en avant la nécessité de faire émerger rapidement les rôles de \myglos*{glo-Meneur} et de \myglos*{glo-Suiveur}.
				Le groupe est ainsi coordonné par le \myglos{glo-Meneur} ce qui permet de limiter les \myglos*{glo-ConflitDeCoordination}.

				Cependant, la manière de travailler du \myglos{glo-Meneur} et très différente de celle d'un \myglos{glo-Suiveur}.
				Des outils haptiques adaptés aux besoins de chaque rôle seront proposés afin d'améliorer les moyens d'interaction de chacun.
				En l'occurence, le \myglos{glo-Meneur} n'a pas réellement besoin d'un outil de déformation mais plutôt d'un outil d'exploration.

				En ce qui concerne le \myglos{glo-Suiveur}, il est principalement affecté à la réalisation de déformations locales.
				Il faut lui fournir un outil permettant d'effectuer des déformations locales et précises.
				Cependant, il faut également lui faciliter la communication avec le \myglos{glo-Meneur} et lui permettant d'accéder rapidement aux consignes.

				La majorité de ces fonctionnalités seront implémentées dans la dernière version de la plateforme puis testées, notamment pas des biologistes.
				Ces outils seront évalués à la fois en terme d'amélioration sur les performances mais également en terme d'utilisabilité.
			\end{mysubsection}
		\end{mysection}
	\end{mychapter}
	\begin{mychapter}[cha-TravailCollaboratifAssisteParHaptique]{Travail collaboratif assisté par haptique}
		\begin{mysection}[sec-exp4-Introduction]{Introduction}
			Sur la base des résultat obtenus dans les trois études précédentes, nous avons développé des outils haptiques afin d'améliorer la communication et la coordination des actions entre les utilisateurs durant la réalisation d'un \myglos{glo-DockingMoleculaire} collaboratif.
			Nous avons également identifié un certains nombres de facteurs permettant d'améliorer les performances d'un groupe (identification des rôles, \mybrainstorming, présence d'un \myglos{glo-Meneur}, \myetc).
			Ce chapitre a pour but de proposer et d'évaluer une configuration de travail ainsi que des outils permettant d'améliorer la communication et la coordination durant une tâche de déformation moléculaire.

			Ce dernier chapitre sera également l'occasion de confronter la plateforme \myShaddock à des experts en biologie moléculaire.
			En effet, il est nécessaire pour cette dernière étape d'avoir l'avis de spécialistes afin d'évaluer l'utilisabilité de la plateforme et de confirmer la validité de ce processus de travail.
		\end{mysection}
		\begin{mysection}[sec-exp4-AssistanceHaptiquePourLaCommunication]{Assistance haptique pour la communication}
			Dans cette section, nous commencerons par présenter les travaux existants en terme de communication haptique.
			Puis nous exposerons les choix qui ont guidés le développement des outils haptiques proposés.
			Enfin, nous présenterons les objectifs de cette quatrième et dernière expérimentation.
			\begin{mysubsection}[sse-exp4-TravauxExistants]{Travaux existants}
				Le concept de communication haptique a été initié par les travaux de \mycite[author]{Sallnas-2000} avec une étude sur la manipulation synchrone d'objets \myThreeD dans environnement virtuel.
				Les utilisateurs disposent d'un outil haptique leur permettant d'exercer une pression sur les objets \myThreeD virtuels; il faut alors deux utilisateurs pour soulever un objet en exerçant une pression de chaque côté de l'objet.
				Sur le même principe de manipulation synchrone, \mycite[author]{Basdogan-2000} proposent le déplacement d'un anneau par une manipulation en \myglos{glo-Binome} coordonnée; l'anneau doit passer autour d'un fil sans le toucher.
				Les résultats de ces deux études montrent clairement que l'utilisation de la modalité haptique permet des améliorations significatives pour la coordination entre les partenaires.

				Sur la base de ces résultats, \mycite[author]{Oakley-2001} propose de créer des outils haptiques pour la création de diagrammes \myacro{acr-UML} en environnement \myTwoD.
				Il fournit aux utilisateurs différents outils haptiques permettant d'interagir avec le partenaire :
				\begin{itemize}
					\item ressentir le curseur du partenaire avec la possibilité de le pousser;
					\item attraper et guider le curseur du partenaire vers une position choisie;
					\item activer à distance un guidage du partenaire vers son propre curseur;
					\item activer à distance un guidage de son propre curseur vers celui de son partenaire;
					\item ressentir une viscosité plus importante par la proximité du curseur du partenaire.
				\end{itemize}

				Dans la continuité de \mycite[author]{Oakley-2001}, \mycite[author]{Moll-2009} expérimentent des outils haptiques similaires dans un environnement virtuel de construction par bloc en \myThreeD.
				Ces deux dernières études prouvent que la communication entre deux partenaires par l'intermédiaire de la modalité haptique est possible.
				D'ailleurs, \mycite[author]{Moll-2009} soulignent la pertinence des retours haptiques pour la désignation de cibles en environnement virtuel.

				Le terme \myemph{communication haptique} a pris tout son sens avec les premières tentatives de \myemph{langages haptiques}.
				Tout d'abord, \mycite[author]{Chang-2002} proposent d'augmenter l'information d'une communication par \textit{talkies-walkies} avec différents types de vibrations.
				Puis, \mycite[author]{Enriquez-2006} créé une liste de phonèmes haptiques permettant la communication.
				Il constate cependant qu'une telle solution nécessite un fort apprentissage ($\approx \mynum[mn]{45}$) pour une utilisation persistente.
				\mycite[author]{Chan-2008} utilise également des mots haptiques pour aider et assister la prise de contrôle d'un système lors de la réalisation d'une tâche collaborative.

				Récemment, \mycite[author]{Ullah-2011} s'est intéressé à l'influence de la communication haptique sur le sentiment de présence et sur la conscience périphérique \myref*{sec-sota-ApprocheCollaborativePourLesProblemesComplexes}.
				Tout d'abord avec une étude sur la manipulation collaborative d'un bras articulé en environnement virtuel, il montre une amélioration de la conscience périphérique \mycite{Naud-2009}.
				Puis, en proposant une tâche de déplacement d'objet virtuel coordonné (une cheville à déplacer dans un trou), \mycite[author]{Ullah-2010} montre une amélioration significative du sentiment de présence et de la conscience périphérique.
			\end{mysubsection}
			\begin{mysubsection}[sse-exp4-OutilsHaptiquesPourLaCollaboration]{Outils haptiques pour la collaboration}
				Sur la base de cet état de l'art et des conclusions obtenues au fil des trois études précédentes, plusieurs propositions s'imposent pour le développement d'outils haptiques de collaboration.
				\begin{mysubsubsection}[sss-ConfigurationCollaborative]{Configuration collaborative}
					Pour commencer, la présence d'un coordinateur répond à deux contraintes identifiées lors de la troisième expérimentation : l'identification des rôles et la coordination des tâches \myref*{sse-exp3-DefinitionDUnMeneur}.
					En effet, nous avons identifié le rôle du \myglos{glo-Meneur} dans la précédente expérimentation qui permettait de coordonner les actions du groupe.
					Afin de répondre à ce besoin, nous allons proposer des outils spécifiques aux différents membres du groupe qui seront soit coordinateur, soit opérateur afin de créer naturellement une identification de rôles distincts.

					Nous avons vu que le \myglos{glo-Meneur} élabore surtout une stratégie grâce à une exploration de l'environnement virtuel puis qu'il expose sa stratégie aux \myglos*{glo-Suiveur}.
					Dans cette expérimentation, le coordinateur est doté d'outils d'exploration et de désignation alors que les opérateurs sont dotés d'outils de déformation.

					De plus, la précédente expérimentation nous a permi de constater l'utilité d'un \mybrainstorming pour améliorer les performances.
					Nous introduisons une période de \mybrainstorming dans cette expérimentation durant laquelle nous donnons la possibilité d'explorer, mais pas de déformer, les structures moléculaires; c'est le coordinateur qui est en charge de cette tâche.
				\end{mysubsubsection}
				\begin{mysubsubsection}[sss-OutilPourLeCoordinateur]{Outils pour le coordinateur}
					Durant le \mybrainstorming, un outil d'orientation est mis à la disposition du coordinateur.
					Cet outil, décrit dans la  \myref{par-exp4-OutilDOrientation}, lui permet d'observer la molécule avec différents points de vue afin d'élaborer une stratégie avec l'aide des opérateurs.

					Puis, le coordinateur va désigner des cibles durant la réalisation de la tâche afin d'appliquer la stratégie élaborée.
					Nous avions souligné ce point lors de la première expérimentation et \mycite[author]{Moll-2009} met en avant la pertinence de la modalité haptique pour effectuer une désignation.
					L'outil de désignation, présenté dans la \myref{par-exp4-OutilDeDesignation}, a été développé afin de permettre au coordinateur d'indiquer des cibles aux opérateurs.
				\end{mysubsubsection}
				\begin{mysubsubsection}[sss-OutilPourLesOperateurs]{Outils pour les opérateurs}
					La première expérimentation avait également permis d'identifier une stratégie offrant un bon compromis entre la collaboration et les \myglos*{glo-ConflitDeCoordination} : l'interaction en champs voisins.
					Pour mémoire, les interactions en champs voisins correspondent à la manipulation de deux zones distantes d'environ un \myglos{glo-Residu}.
					Afin de favoriser ce type de stratégie, les outils proposés agiront de préférence à l'échelle des \myglos*{glo-Residu} en laissant la possibilité le cas échéant, de manipuler au niveau atomique afin d'avoir accès à un niveau de déformation fin.

					Pour finir, la seconde expérimentation a mis en avant la difficulté d'utiliser deux outils de déformation en configuration \myglos*{glo-Bimanuel}.
					C'est pourquoi, les opérateurs qui seront chargés d'effectuer la déformation seront en configuration \myglos*{glo-Monomanuel} avec à leur disposition un unique outil de déformation.
				\end{mysubsubsection}
			\end{mysubsection}
			\begin{mysubsection}[sse-exp4-Objectifs]{Objectifs}
				Dans cette expérimentation, nous proposons des outils haptiques pour assister la communication entre les partenaires et souhaitons les évaluer dans le cadre d'une tâche de déformation collaborative de molécule.

				Sur la base de la configuration collaborative détaillée précédemment \myref*{sse-exp4-OutilsHaptiquesPourLaCollaboration}, nous souhaitons étudier la contribution des outils haptiques proposés pour l'amélioration des performances puis de la communication dans le groupe.
				Notre hypothèse liée à ces outils est double : nous pensons que l'assistance haptique permettra d'améliorer les performances et que cette amélioration sera due à une amélioration de la communication.

				De plus, cette dernière expérimentation est l'occasion de confronter la plateforme et les méthodes de travail proposées à des experts susceptibles d'utiliser un tel environnement de travail.
				Nous supposons que la plateforme sera évaluée positivement par les utilisateurs.

				Les objectifs sont résumés sous forme d'hypothèses dans la \myref{sse-met-exp4-Hypotheses}.
			\end{mysubsection}
		\end{mysection}
		\begin{mysection}[sec-exp4-PresentationDeLExperimentation]{Présentation de l'expérimentation}
			\begin{mysubsection}[sse-exp4-DescriptionDeLaTache]{Description de la tâche}
				La tâche étudiée est constituée d'une tâche de déformation, telle que celle abordée dans les deux expérimentations précédentes \myref*{sse-exp2-DescriptionDeLaTache}, et d'une tâche d'assemblage; les deux tâches abordées permettent de créer un scénario de \myglos{glo-DockingMoleculaire} complet.

				La tâche est proposée à des groupes de trois sujets appelés \myglos*{glo-Trinome}.
				Dans ces \myglos*{glo-Trinome}, un \og coordinateur \fg et deux \og opérateurs \fg ont à leur disposition différents outils pour l'orientation, la déformation, la désignation ou la manipulation qui seront décrits dans la \myref{sss-exp4-OutilsDInteraction}.
				Les sujets ont la possibilité de communiquer sans restriction par différents canaux : oral, gestuel ou virtuel (à travers l'utilisation des curseurs et de l'environnement virtuel).

				Durant cette expérimentation, nous présentons aux sujets cinq molécules; trois d'entre elles sont utilisées pour l'entraînement.
				Chaque molécule est présentée dans la \myref{sse-pro-ListeDesMolecules}.
				Les cinq molécules sont utilisées pour les différents scénarios décrits ci-dessous.
				\begin{description}
					\item[Étape d'entraînement~\myscenario{1}]
						Le premier entraînement est destiné à familiariser les sujets avec l'outil de désignation ainsi qu'avec la structure du groupe.
						Cet entraînement se déroule sur la molécule \myTRPCAGE.
						La tâche est volontairement peu complexe.
						Elle est destinée à enseigner aux sujets les étapes du processus de désignation \myref*{sss-Shaddock-OutilDeDesignation}.
					\item[Étape d'entraînement~\myscenario{2}]
						La seconde phase d'entraînement s'effectue sur la molécule \myPrion qui est de taille plus importante.
						C'est dans cette phase que sont introduits les outils de communication haptique pour les phases de désignation.
						Cet entraînement permet de renforcer l'apprentissage du processus de désignation sur une tâche plus complexe en ajoutant la modalité haptique.
					\item[Étape d'entraînement~\myscenario{3}]
						La dernière phase d'entraînement s'adresse au coordinateur puisqu'elle introduit l'outil de manipulation qui jusqu'à présent n'a pas été utilisé.
						La molécule \myTRPZIPPER, de petite taille, a été choisie.
						La molécule est rendue solidaire de l'outil de manipulation pour que le coordinateur puisse découvrir ce nouvel outil.
					\item[Scénario~\myscenario{1}]
						La première tâche à réaliser est la déformation de la molécule \myUbiquitin.
						La déformation proposée est identique à la déformation proposée dans la troisième expérimentation.
						En effet, cette tâche s'est révélée très intéressante pour stimuler une collaboration étroite.
						Dans cette tâche, seuls les outils de désignation, de déformation et d'orientation sont activés; la molécule \myUbiquitin est ancrée dans l'environnement virtuel à l'aide d'atomes fixes.
					\item[Scénario~\myscenario{2}]
						La seconde tâche consiste à reconstituer le complexe de molécules \myNusENusG.
						La molécule \myNusG est laissée libre de mouvement (pas d'atome fixes) et doit être amarrée à la molécule \myNusE : c'est une tâche de \myglos{glo-DockingMoleculaire} simplifiée.
						On distingue deux phases dans cette tâche; il faut approcher la molécule \myNusG; puis il faut affiner l'amarrage par une déformation interne de \myNusG.
						Tous les outils (désignation, déformation, orientation et manipulation) sont activés dans ce scénario; la molécule \myNusG est solidaire de l'outil de manipulation.
						La molécule \myNusE est ancrée dans l'environnement virtuel à l'aide d'atome fixes (sa chaîne carbonée).
				\end{description}

				Durant l'expérimentation, les sujets disposent de deux informations calculées en temps-réel.
				La première mesure est le score \myacro{acr-RMSD}, déjà décrit dans la \myref{sse-exp2-DescriptionDeLaTache}.
				La seconde mesure est l'énergie totale du système, valeur calculée par \myacro{acr-NAMD} et représentant la synthèse des énergies électriques et des énergies de \myname[van der]{Waals}.
			\end{mysubsection}
			\begin{mysubsection}[sse-exp4-SpecificitesDuProtocoleExperimental]{Spécificités du protocole expérimental}
				Les sections suivantes décrivent l'ensemble des modification apportées à la plateforme de base \myref*{cha-pro-DispositifExperimental} et principalement aux outils d'interaction.
				La méthode expérimentale est exposée dans la \myref{sec-met-exp4-QuatriemeExperimentation}.
				Un résumé de cette méthode se trouve dans la \myref{tab-exp4-SyntheseDeLaProcedureExperimentale}.
				\begin{mysubsubsection}[sss-exp4-MaterielUtilise]{Matériel utilisé}
					Dans cette expérimentation, nous introduisons de nouveaux outils destinés à améliorer les interactions entre les membres d'un \myglos{glo-Trinome}.
					Les deux opérateurs auront à leur disposition deux outils de déformation matérialisés par des \myOmni.
					Le coordinateur aura à sa disposition trois outils :
					\begin{itemize}
						\item une souris \myUSB pour l'outil d'orientation;
						\item un \myOmni pour l'outil de désignation (lié à un serveur \myacro{acr-VRPN});
						\item un \myDesktop pour l'outil de manipulation (lié à un serveur \myacro{acr-VRPN}).
					\end{itemize}

					De la même manière que dans la troisième expérimentation, une caméra vidéo \mySony (\textsc{hdr-cx550}) a été installée derrière les sujets afin de filmer à la fois les sujets et l'écran de vidéoprojection.
					Le son est également enregistré.
					Là encore, les vidéos sont exportées et séquencées \myafortiori à l'aide des logiciels \myiMovie et \texttt{ffmpeg}.

					La \myref{fig-exp4-SchemaDuDispositifExperimental} illustre le dispositif expérimental par un schéma.
					La \myref{fig-exp4-PhotographieDuDispositifExperimental} est une photographie de la salle d'expérimentation.

					\begin{myfigure}
						\myimage[width=0.9\textwidth]{exp4-schema}
						\mycaption[fig-exp4-SchemaDuDispositifExperimental]{Schéma du dispositif expérimental}
					\end{myfigure}
					\begin{myfigure}
						\myimage[width=0.9\textwidth]{exp4-photo}
						\mycaption[fig-exp4-PhotographieDuDispositifExperimental]{Photographie du dispositif expérimental}
					\end{myfigure}
				\end{mysubsubsection}
				\begin{mysubsubsection}[sss-exp4-VisualisationEtRepresentation]{Visualisation et représentation}
					L'affichage des molécules est similaire à celui de la troisième expérimentation présenté dans la \myref{sss-exp3-VisualisationEtRepresentation} : la molécule déformable est représentée en \myCPK et \myNewRibbon alors que la molécule cible est en \myNewRibbon transparent.
					Cependant, les molécules étant de taille importante, nous avons décidé de représenter les atomes en transparence et sans couleur afin d'éviter une surcharge visuelle, surtout dans le cas du complexe de molécules \myNusENusG.
					Cependant, nous verrons dans la section suivante que l'outil de désignation permet de mettre en évidence les atomes des \myglos*{glo-Residu} désignés par une mise en couleur.

					La \myref{fig-exp4-RepresentationDeLaMoleculeUbiquitinPourLeScenario1} représente la molécule \myUbiquitin et la \myref{fig-exp4-RepresentationDeLaMoleculeNusENusGPourLeScenario2} représente le complexe de molécules \myNusENusG.

					\begin{myfigure}
						\myimage{exp4-scenario1}
						\mycaption[fig-exp4-RepresentationDeLaMoleculeUbiquitinPourLeScenario1]{Représentation de la molécule \myUbiquitin pour le scénario~\myscenario{1}}
					\end{myfigure}
					\begin{myfigure}
						\myimage{exp4-scenario2}
						\mycaption[fig-exp4-RepresentationDeLaMoleculeNusENusGPourLeScenario2]{Représentation de la molécule \myNusENusG pour le scénario~\myscenario{2}}
					\end{myfigure}
				\end{mysubsubsection}
				\begin{mysubsubsection}[sss-exp4-OutilsDInteraction]{Outils d'interaction}
					Pour cette expérimentation, nous avons développé de nouveaux outils héritant des outils de base proposés par \myacro{acr-VMD}.
					En effet, nous souhaitons apporter une assistance haptique afin d'augmenter la communication sensorielle entre les sujets.
					Quatre outils sont proposés dont les fonctionnalités sont résumées ci-après.
					Pour une description technique détaillée de ces outils, se reporter à la \myref{sse-Shaddock-OutilsDeManipulationAvances}.
					\begin{myparagraph}[par-exp4-OutilDeDesignation]{Outil de désignation}
						\begin{description}
							\item[Utilisateur] Coordinateur.
							\item[Matériel] \myOmni.
							\item[Fonction] Permet de désigner un \myglos{glo-Residu} aux opérateurs.
							\item[Fonctionnement]
								Un des rôles du coordinateur est de désigner des structures (des \myglos*{glo-Residu}) à déformer aux opérateurs.
								L'opérateur pourra accepter ou non cette désignation.
								Tant que la cible désignée par le coordinateur n'aura pas été acceptée, le coordinateur pourra modifier la cible courante.
							\item[Rendu haptique]
								Lorsque le coordinateur effectue une désignation, une vibration est générée sur son outil.
								Cette vibration sinusoïdale s'arrête dès qu'un opérateur a accepté la désignation ce qui permet au coordinateur d'être informé que la désignation a été acceptée.
						\end{description}
					\end{myparagraph}
					\begin{myparagraph}[par-exp4-OutilDeDeformation]{Outil de déformation}
						\begin{description}
							\item[Utilisateur] Opérateur.
							\item[Matériel] \myOmni.
							\item[Fonction] Permet de déformer un ou plusieurs atomes.
							\item[Fonctionnement]
								Le rôle principal de l'opérateur est de déformer les structures moléculaire en manipulant un ou plusieurs atomes de la molécule selon deux niveaux de déformation.
								Soit il choisi de déformer de manière autonome; dans ce cas, il ne peut manipuler qu'un atome à la fois.
								Soit il accepte une désignation proposée par le coordinateur; dans ce cas, il peut manipuler un \myglos{glo-Residu}.
								Dans ce dernier cas, une force de déformation plus importante est fournie à l'opérateur, l'objectif étant de favoriser l'utilisation de cet outil.
							\item[Rendu haptique]
								Lorsque le coordinateur effectue une désignation, l'opérateur est soumis à une vibration sinusoïdale sauf si ce dernier est déjà en train de déformer.
								S'il décide d'accepter la désignation, la vibration s'arrête et l'opérateur est alors attiré directement vers la cible par une attraction de type masse-ressort.
						\end{description}
					\end{myparagraph}
					\begin{myparagraph}[par-exp4-OutilDeManipulation]{Outil de manipulation}
						\begin{description}
							\item[Utilisateur] Coordinateur.
							\item[Matériel] \myDesktop.
							\item[Fonction] Permet de déplacer une molécule.
							\item[Fonctionnement]
								Cet outil est attaché à une molécule (le ligand ou le récepteur) prédéfinie pour chaque scénario.
								Le coordinateur peut ainsi translater la molécule afin de l'amarrer correctement avec la seconde molécule.
								Tout mouvement de cet outil permet de déplacer l'ensemble des atomes de la molécule attachée.
							\item[Rendu haptique]
								Le retour haptique est un système masse-ressort contraignant le coordinateur à rester lié à l'inertie de la molécule attachée.
								De plus, afin d'améliorer la coordination entre les opérateurs et le coordinateur, nous proposons de communiquer les déformations effectuées par les opérateurs au mouvement global de la molécule, dans une proportion relativement faible.
								Concrétement, lorsqu'un opérateur effectue une déformation, la molécule aura tendance à se déplacer faiblement dans la même direction; cette fonctionnalité est additionnée au mouvement du coordinateur.
						\end{description}
					\end{myparagraph}
					\begin{myparagraph}[par-exp4-OutilDOrientation]{Outil d'orientation}
						\begin{description}
							\item[Utilisateur] Coordinateur.
							\item[Matériel] Souris \myUSB.
							\item[Fonction] Permet de modifier le point de vue de la scène.
							\item[Fonctionnement]
								Il permet de modifier l'orientation de la scène.
								Cependant, les curseurs ne sont pas soumis à cette transformation ce qui peu créer des incohérences visuelles, surtout pour les opérateurs qui ne sont pas en charge de cet outil.
								C'est pourquoi ce troisième outil est présenté comme secondaire au coordinateur afin qu'il ne soit utilisé que lorsque c'est vraiment nécessaire.
							\item[Rendu haptique]
								Cet outil ne permet de retour haptique.
						\end{description}
					\end{myparagraph}
				\end{mysubsubsection}
				\begin{mytable}
					\mycaption[tab-exp4-SyntheseDeLaProcedureExperimentale]{Synthèse de la procédure expérimentale}
					\newcommand{\mytitlecolumn}[2]{%
						\multirow{#1}*{%
							\begin{minipage}{6em}%
								\raggedleft #2%
							\end{minipage}%
						}
					}
					\newlength{\expfourfirstcolumn}
					\newlength{\expfoursecondcolumn}
					\setlength{\expfourfirstcolumn}{7em}
					\setlength{\expfoursecondcolumn}{\textwidth}
					\addtolength{\expfoursecondcolumn}{-\expfourfirstcolumn}
					\addtolength{\expfoursecondcolumn}{-4\tabcolsep}
					\begin{mytabular}{>{\bfseries}p{\expfourfirstcolumn}p{\expfoursecondcolumn}}
						\mytoprule
						\mytitlecolumn{1}{Tâche}                   & Déformation de molécule ou de complexe de molécule                 \\
						\mymiddlerule[\heavyrulewidth]
						\mytitlecolumn{3}{Hypothèses}              & \myhypothesis{1} Performances améliorées par l'assistance haptique \\
						                                           & \myhypothesis{2} L'assistance haptique améliore la communication   \\
						                                           & \myhypothesis{3} La plateforme est appréciée des experts          \\
						\mymiddlerule
						\mytitlecolumn{2}{Variables indépendantes} & \myvari{1} Présence de l'assistance                                \\
						                                           & \myvari{2} Molécules à déformer                                    \\
						\mymiddlerule
						\mytitlecolumn{10}{Variables dépendantes}  & \myvard{1} Score \myacronl-{acr-RMSD} minimum                      \\
						                                           & \myvard{2} Temps du score \myacronl-{acr-RMSD} minimum             \\
						                                           & \myvard{3} Nombre de sélections                                    \\
						                                           & \myvard{4} Temps moyen d'une sélection                             \\
						                                           & \myvard{5} Temps moyen pour accepter une cible                     \\
						                                           & \myvard{6} Temps moyen pour atteindre une cible acceptée           \\
						                                           & \myvard{7} Nombre de désignations acceptées                        \\
						                                           & \myvard{8} Vitesse moyenne                                         \\
						                                           & \myvard{9} Temps des communication verbale par sujet               \\
						                                           & \myvard{10} Questionnaire d'utilisabilité et sur la conscience     \\
						\mymiddlerule[\heavyrulewidth]
						\multicolumn{2}{c}{%
							\small%
							\begin{tabular}{^C-C-C-C}
								\myrowstyle{\bfseries}
								Condition \mycondition{1} & Condition \mycondition{2} & Condition \mycondition{3} & Condition \mycondition{4} \\
								\mymiddlerule
								Sans assistance           & Avec assistance           & Sans assistance           & Avec assistance           \\
								\mymiddlerule
								\myUbiquitin              & \myUbiquitin              & \myNusENusG               & \myNusENusG               \\
							\end{tabular}
						} \\
						\mybottomrule
					\end{mytabular}
				\end{mytable}
			\end{mysubsection}
		\end{mysection}
		\begin{mysection}[sec-exp4-Resultats]{Résultats}
			Cette section présente et analyse l'ensemble des mesures expérimentales de cette quatrième étude.
			Les données, qui sont appareillées et en faible nombre, ont été analysées par le test des rangs signés de \mycite[author]{Wilcoxon-1945}.
			\begin{mysubsection}[sse-exp4-AmeliorationDesPerformances]{Amélioration des performances}
				Nous abordons ici les performances des groupes afin d'observer les évolutions liées à la présence de l'assistance haptique.
				Nous commençons par présenter les graphiques accompagnés de leurs analyses statistiques puis nous discuterons ces résultats.
				\begin{mysubsubsection}[sss-exp4-AmeliorationDesPerformances-DonneesEtTestsStatistiques]{Données et tests statistiques}
					\begin{myfigure}
						\psset{xunit=0.274914089\textwidth,yunit=0.75cm}
						\begin{myps}(-0.425,-1.5)(2,6.25)
							\myaxes(0,2){haptique}(0,6)[1]{score \myacronl{acr-RMSD}}
							\myboxplot{exp4-rmsd-haptic.csv}
						\end{myps}
						\mycaption[fig-exp4-ScoreRMSDMinimumAtteintAvecEtSansHaptique]{Score \myacronl{acr-RMSD} minimum atteint avec et sans haptique}
					\end{myfigure}

					La \myref{fig-exp4-ScoreRMSDMinimumAtteintAvecEtSansHaptique} présente le score \myglos{acr-RMSD} minimum atteint \myvard{1} en fonction de la présence de l'assistance haptique \myvari{1}.
					L'analyse montre qu'il n'y a pas d'effet significatif de l'assistance haptique \myvari{1} sur le score \myacro{acr-RMSD} minimum atteint \myvard{1} (\myanova{exp4-rmsd-haptic-anova.tex}).

					\begin{myfigure}
						\psset{xunit=0.274914089\textwidth,yunit=0.0075cm}
						\begin{myps}(-0.425,-150)(2,525)
							\myaxes(0,2){haptique}(0,500)[100]{temps~(s)}
							\myboxplot{exp4-rmsd-time-haptic.csv}
						\end{myps}
						\mycaption[fig-exp4-TempsPourAtteindreLeScoreRMSDMinimumAvecEtSansHaptique]{Temps pour atteindre le score \myacronl{acr-RMSD} minimum avec et sans haptique}
					\end{myfigure}

					La \myref{fig-exp4-TempsPourAtteindreLeScoreRMSDMinimumAvecEtSansHaptique} présente le temps du score \myglos{acr-RMSD} minimum atteint \myvard{2} en fonction de la présence de l'assistance haptique \myvari{1}.
					L'analyse montre qu'il n'y a pas d'effet significatif de l'assistance haptique \myvari{1} sur le temps du score \myacro{acr-RMSD} minimum atteint \myvard{2} (\myanova{exp4-rmsd-time-haptic-anova.tex}).

					\begin{myfigure}
						\psset{xunit=0.274914089\textwidth,yunit=0.0075cm}
						\begin{myps}(-0.425,-150)(2,575)
							\myaxes(0,2){scénario}(0,500)[100]{temps~(s)}
							\myboxplot{exp4-rmsd-time-molecule-haptic.csv}
							\mylegend{\myleg{Sans assistance}{myblue}\myand\myleg{Avec assistance}{myblue!70}}
						\end{myps}
						\mycaption[fig-exp4-TempsPourAtteindreLeScoreRMSDMinimumAvecEtSansHaptiquePourChaqueScenario]{Temps pour atteindre le score \myacronl{acr-RMSD} minimum avec et sans haptique pour chaque scénario}
					\end{myfigure}

					La \myref{fig-exp4-TempsPourAtteindreLeScoreRMSDMinimumAvecEtSansHaptiquePourChaqueScenario} présente le temps du score \myglos{acr-RMSD} minimum atteint \myvard{2} en fonction de la présence de l'assistance haptique \myvari{1} et des scénarios \myvari{2}.
					L'analyse montre qu'il n'y a pas d'effet significatif de l'assistance haptique \myvari{1} sur le temps du score \myacro{acr-RMSD} minimum atteint \myvard{2} pour la molécule \myUbiquitin (\myanova{exp4-rmsd-time-molecule-haptic-ubiquitin-anova.tex}).
					L'analyse montre un effet significatif de l'assistance haptique \myvari{1} sur le temps du score \myacro{acr-RMSD} minimum atteint \myvard{2} pour le complexe de molécules \myNusENusG (\myanova{exp4-rmsd-time-molecule-haptic-nuse-anova.tex}); le temps est inférieur de \myratio{exp4-rmsd-time-molecule-haptic-nuse-ratio.tex} avec l'assistance haptique.

					\begin{myfigure}
						\psset{xunit=0.256410256\textwidth,yunit=30cm}
						\begin{myps}(-0.6,-0.04)(2,0.155)
							\myaxes(0,2){haptique}(0,0.150)[0.025]{fréquence~(nb/s)}
							\myboxplot{exp4-freq-button1-haptic.csv}
						\end{myps}
						\mycaption[fig-exp4-NombreDeSelectionsParSecondeEffectueesParUnOperateurPourLaDeformationAvecEtSansHaptique]{Nombre de sélections par seconde effectuées par un opérateur pour la déformation avec et sans haptique}
					\end{myfigure}

					La \myref{fig-exp4-NombreDeSelectionsParSecondeEffectueesParUnOperateurPourLaDeformationAvecEtSansHaptique} présente le nombre de sélections par seconde effectuées par les opérateurs pour une déformation \myvard{3} en fonction de la présence de l'assistance haptique \myvari{1}.
					L'analyse montre un effet significatif de l'assistance haptique \myvari{1} sur le nombre de sélections \myvard{3} (\myanova{exp4-freq-button1-haptic-anova.tex}); le nombre de sélections est inférieur de \myratio{exp4-freq-button1-haptic-ratio.tex} avec l'assistance haptique.

					\begin{myfigure}
						\psset{xunit=0.274914089\textwidth,yunit=0.175cm}
						\begin{myps}(-0.425,-7)(2,26)
							\myaxes(0,2){haptique}(0,25)[5]{temps~(s)}
							\myboxplot{exp4-meantime-button1-haptic.csv}
						\end{myps}
						\mycaption[fig-exp4-TempsMoyenDUneSelectionParSecondeEffectueeParUnOperateurPourLaDeformationAvecEtSansHaptique]{Temps moyen d'une sélection effectuée par un opérateur pour la déformation avec et sans haptique}
					\end{myfigure}

					La \myref{fig-exp4-TempsMoyenDUneSelectionParSecondeEffectueeParUnOperateurPourLaDeformationAvecEtSansHaptique} présente le temps moyen d'une sélection effectuée par un opérateur pour une déformation \myvard{4} en fonction de la présence de l'assistance haptique \myvari{1}.
					L'analyse montre un effet significatif de l'assistance haptique \myvari{1} sur le temps moyen d'une sélection \myvard{4} (\myanova{exp4-meantime-button1-haptic-anova.tex}); le temps de sélection est supérieur de \myratio{exp4-meantime-button1-haptic-ratio.tex} avec l'assistance haptique.

					\begin{myfigure}
						\psset{xunit=0.274914089\textwidth,yunit=0.1cm}
						\begin{myps}(-0.425,-12.5)(2,42.5)
							\myaxes(0,2){haptique}(0,40)[10]{temps~(s)}
							\myboxplot{exp4-attract-time-haptic.csv}
						\end{myps}
						\mycaption[fig-exp4-TempsMoyenPourAtteindreUneCibleDesigneeLorsDUneDesignationAvecEtSansHaptique]{Temps moyen pour atteindre une cible désignée lors d'une désignation avec et sans haptique}
					\end{myfigure}

					La \myref{fig-exp4-TempsMoyenPourAtteindreUneCibleDesigneeLorsDUneDesignationAvecEtSansHaptique} présente le temps moyen mis par un opérateur pour atteindre une cible acceptée lors du processus de désignation \myvard{6} en fonction de la présence de l'assistance haptique \myvari{1}.
					L'analyse montre un effet significatif de l'assistance haptique \myvari{1} sur ce temps moyen \myvard{6} (\myanova{exp4-attract-time-haptic-anova.tex}); le temps moyen pour atteindre une cible est inférieur de \myratio{exp4-attract-time-haptic-ratio.tex} avec l'assistance haptique.
				\end{mysubsubsection}
				\begin{mysubsubsection}[sss-exp4-AmeliorationDesPerformances-AnalyseEtDiscussion]{Analyse et discussion}
					En observant le meilleur score \myacro{acr-RMSD} obtenu par les participants avec et sans assistance haptique, on constate qu'il n'y a pas d'amélioration significative des performances \myref*{fig-exp4-ScoreRMSDMinimumAtteintAvecEtSansHaptique}.
					L'assistance haptique proposée ne semble pas permettre d'améliorer les performances et ne confirme pas l'hypothèse~\myhypothesis{1}.
					D'ailleurs, la \myref{fig-exp4-TempsPourAtteindreLeScoreRMSDMinimumAvecEtSansHaptique} montre que le temps mis pour atteindre ce score n'est également pas amélioré.

					Cependant, une analyse en fonction des scénarios montre que les outils d'assistance haptique permettent une amélioration sur le complexe de molécules \myNusENusG (scénario~\myscenario{2}) mais pas sur la molécule \myUbiquitin (scénario~\myscenario{1}).
					Le complexe de molécules \myNusENusG est la tâche la plus complexe à réaliser notamment à cause du nombre importants de \myglos*{glo-Residu} à déformer.
					D'ailleurs, les annotations vidéos ont permis de constater à plusieurs reprises que les groupes abandonnaient avant la fin des \mynum[mn]{8} face à la complexité de cette tâche : \og On ne fera pas mieux \fg, \og On n'arrivera jamais à améliorer le score \fg, \og Cette molécule est trop difficile \fg, \myetc
					Confrontés à ces difficultés, les sujets ont été plus performants lorsqu'ils étaient assistés par les outils d'assistance haptique.

					Le coordinateur est le seul sujet qui ne peut pas effectuer de déformation.
					Il se consacre à la planification des tâches à réaliser.
					Grâce aux outils d'assistance haptique mis à sa disposition, il peut communiquer plus facilement et plus rapidement ces directives comme expliqué dans la \myref{sse-exp4-AmeliorationDeLaCommunication}.
					Ces outils permettent d'augmenter l'attention des opérateurs qui vont effectuer moins d'actions autonomes.
					Une action autonome est une action que l'opérateur décide de réaliser sans concertation avec les autres membres du groupe et potentiellement, en désaccord avec la stratégie globale adoptée par ses partenaires.
					Elles sont en général relativement brèves dans le temps par rapport aux actions proposées et désignées par le coordinateur car elle peut être interrompue à tout moment par une requête du coordinateur.
					Ces actions peuvent ainsi être relativement improductives.

					La \myref{fig-exp4-NombreDeSelectionsParSecondeEffectueesParUnOperateurPourLaDeformationAvecEtSansHaptique} et la \myref{fig-exp4-TempsMoyenDUneSelectionParSecondeEffectueeParUnOperateurPourLaDeformationAvecEtSansHaptique} montrent que ces actions autonomes diminuent avec la présence des outils d'assistance haptique.
					En effet, on constate sur la \myref{fig-exp4-NombreDeSelectionsParSecondeEffectueesParUnOperateurPourLaDeformationAvecEtSansHaptique} que la fréquence des sélections effectuées par les opérateurs diminue avec la présence d'une assistance haptique.
					Cependant, cette diminution peut être due à deux raisons différentes.
					Soit les opérateurs ont une tendance à travailler moins ce qui diminue le nombre total de sélections (et donc la fréquence); soit ils effectuent des déformations plus longues (ce qui diminue la fréquence).
					La \myref{fig-exp4-TempsMoyenDUneSelectionParSecondeEffectueeParUnOperateurPourLaDeformationAvecEtSansHaptique} montre que la présence des outils d'assistance haptique allonge la durée des sélections ce qui nous permet de déduire que les opérateurs produisent des déformations plus longues dans le temps.
					Les outils haptiques proposés ont eu pour effet de diminuer le nombre d'actions autonomes.

					Si les actions autonomes sont moins nombreuses, alors les opérateurs sont plus disponibles pour respecter la stratégie globale mise en place par le coordinateur.
					Les séquences de sélection et donc de déformation sont alors plus longues et moins nombreuses.
					D'ailleurs, la \myref{fig-exp4-TempsMoyenPourAtteindreUneCibleDesigneeLorsDUneDesignationAvecEtSansHaptique} montre que les opérateurs sont plus rapides à atteindre la cible désignée grâce à l'assistance haptique ce qui leur permet de passer plus de temps à déformer mais moins de temps à effectuer la sélection.

					Nous avons vu que les outils d'assistance haptique pour la désignation aident le coordinateur à imposer une stratégie globale ce qui a pour effet d'améliorer les performances du groupe, surtout sur les tâches les plus complexes.
					L'hypothèse~\myhypothesis{1} est donc partiellement confirmée pour la tâche la plus complexe.
				\end{mysubsubsection}
			\end{mysubsection}
			\begin{mysubsection}[sse-exp4-AmeliorationDeLaCommunication]{Amélioration de la communication}
				Dans cette section, nous nous intéressons plus précisément aux évolutions de la communication liées à la présence de l'assistance haptique.
				Les analyses statistiques et les graphiques sont suivis d'une discussion.
				\begin{mysubsubsection}[sss-exp4-AmeliorationDeLaCommunication-DonneesEtTestsStatistiques]{Données et tests statistiques}
					\begin{myfigure}
						\psset{xunit=0.274914089\textwidth,yunit=0.5cm}
						\begin{myps}(-0.425,-2.25)(2,10.25)
							\myaxes(0,2){haptique}(0,10)[2]{temps~(s)}
							\myboxplot{exp4-shake-time-haptic.csv}
						\end{myps}
						\mycaption[fig-exp4-TempsMoyenDAcceptationDUneDesignationAvecEtSansHaptique]{Temps moyen d'acceptation d'une désignation avec et sans haptique}
					\end{myfigure}

					La \myref{fig-exp4-TempsMoyenDAcceptationDUneDesignationAvecEtSansHaptique} présente le temps moyen mis par un opérateur pour accepter une désignation \myvard{5} en fonction de la présence de l'assistance haptique \myvari{1}.
					L'analyse montre un effet significatif de l'assistance haptique \myvari{1} sur ce temps moyen \myvard{5} (\myanova{exp4-shake-time-haptic-anova.tex}); le temps moyen pour accepter une désignation est inférieur de \myratio{exp4-shake-time-haptic-ratio.tex} avec l'assistance haptique.

					\begin{myfigure}
						\psset{xunit=0.274914089\textwidth,yunit=0.3cm}
						\begin{myps}(-0.425,-4)(2,16)
							\myaxes(0,2){haptique}(0,15)[5]{sélections~(nb)}
							\myboxplot{exp4-accept-haptic.csv}
						\end{myps}
						\mycaption[fig-exp4-NombreDeDesignationsAccepteesAuCoursDuProcessusDeDesignationAvecEtSansHaptique]{Nombre de désignations acceptées au cours du processus de désignation avec et sans haptique}
					\end{myfigure}

					La \myref{fig-exp4-NombreDeDesignationsAccepteesAuCoursDuProcessusDeDesignationAvecEtSansHaptique} présente le nombre de désignations acceptées au cours du processus de désignation \myvard{7} en fonction de la présence de l'assistance haptique \myvari{1}.
					L'analyse montre un effet significatif de l'assistance haptique \myvari{1} sur ce nombre de sélections \myvard{7} (\myanova{exp4-accept-haptic-anova.tex}); le nombre de sélections est supérieur de \myratio{exp4-accept-haptic-ratio.tex} avec l'assistance haptique.

					\begin{myfigure}
						\psset{xunit=0.274914089\textwidth,yunit=4cm}
						\begin{myps}(-0.425,-0.3)(2,1.05)
							\myaxes(0,2){haptique}(0,1.0)[0.2]{vitesse~(mm/s)}
							\myboxplot{exp4-speed-haptic.csv}
						\end{myps}
						\mycaption[fig-exp4-VitesseMoyenneDuCoordinateurAvecEtSansHaptique]{Vitesse moyenne du coordinateur avec et sans haptique}
					\end{myfigure}

					La \myref{fig-exp4-VitesseMoyenneDuCoordinateurAvecEtSansHaptique} présente la vitesse moyenne du coordinateur \myvard{8} en fonction de la présence de l'assistance haptique \myvari{1}.
					L'analyse montre un effet significatif de l'assistance haptique \myvari{1} sur cette vitesse \myvard{8} (\myanova{exp4-speed-haptic-anova.tex}); la vitesse moyenne est supérieur de \myratio{exp4-speed-haptic-ratio.tex} avec l'assistance haptique.

					\begin{myfigure}
						\psset{xunit=0.274914089\textwidth,yunit=0.016cm}
						\begin{myps}(-0.425,-75)(2,260)
							\myaxes(0,2){haptique}(0,250)[50]{temps~(s)}
							\myboxplot{exp4-annotation-time-haptic.csv}
						\end{myps}
						\mycaption[fig-exp4-TempsDeParoleDesSujetsAvecEtSansHaptique]{Temps de parole des sujets avec et sans haptique}
					\end{myfigure}

					La \myref{fig-exp4-TempsDeParoleDesSujetsAvecEtSansHaptique} présente le temps de parole des sujets \myvard{10} en fonction de la présence de l'assistance haptique \myvari{1}.
					L'analyse montre qu'il n'y a pas d'effet significatif de l'assistance haptique \myvari{1} sur le temps de parole \myvard{10} (\myanova{exp4-annotation-time-haptic-anova.tex}).

					\begin{myfigure}
						\psset{xunit=0.274914089\textwidth,yunit=0.016cm}
						\begin{myps}(-0.425,-75)(2,310)
							\myaxes(0,2){haptique}(0,300)[50]{temps~(s)}
							\myboxplot{exp4-annotation-time-haptic-first.csv}
						\end{myps}
						\mycaption[fig-exp4-TempsDeParoleDesSujetsEnFonctionDeLOrdreDePassageDeLAssistanceHaptique]{Temps de parole des sujets en fonction de l'ordre de passage de l'assistance haptique}
					\end{myfigure}

					La \myref{fig-exp4-TempsDeParoleDesSujetsEnFonctionDeLOrdreDePassageDeLAssistanceHaptique} présente le temps de parole des sujets \myvard{4} en fonction de l'ordre de passage de l'assistance haptique.
					L'analyse montre un effet significatif de l'ordre de passage de l'assistance haptique sur le temps de parole \myvard{4} (\myanova{exp4-annotation-time-haptic-first-anova.tex}) avec une augmentation de \myratio{exp4-annotation-time-haptic-first-ratio.tex} lorsque les sujets commencent avec l'assistance haptique.
				\end{mysubsubsection}
				\begin{mysubsubsection}[sss-exp4-AmeliorationDeLaCommunication-AnalyseEtDiscussion]{Analyse et discussion}
					Dans le cas sans assistance haptique, les opérateurs réagissent à une désignation pour deux raisons : soit ils ont vu une nouvelle cible désignée, soit le coordinateur a indiqué oralement la présence d'une nouvelle cible.
					Dans le premier cas, le temps de réponse de l'opérateur pour accepter la désignation dépend de son degré d'attention et de la visibilité de la cible (cible masquée par des atomes ou cible trop éloignée du champ de vision).
					Dans le second cas, c'est le temps utilisé pour les échanges verbaux et la qualité des explications qui va déterminer le temps de réponse.
					Dans le second cas, le temps de réponse dépend de la quantité et de la qualité des échanges verbaux échangés.
					Ces deux configurations nécessitent que l'opérateur soit attentif concernant toute nouvelle désignation ce qui réduit son attention pour les autres évènements (conscience périphérique).

					La \myref{fig-exp4-TempsMoyenDAcceptationDUneDesignationAvecEtSansHaptique} nous montre que l'utilisation d'un retour haptique au sein de ce processus de désignation améliore considérablement le temps de réponse.
					L'outil d'assistance haptique signale aux opérateurs qu'une nouvelle cible a été désignée par le coordinateur.

					Ce moyen de communication permet d'obtenir une réactivité qui ne dépend plus ni du degré d'attention de l'opérateur, ni des interventions verbales du coordinateur.
					Cela permet de réduire la charge de travail des opérateurs et du coordinateur qu'ils peuvent consacrer à la réalisation d'autre tâches.

					L'utilisation du canal haptique permet d'informer seulement le ou les sujets concernés par la désignation.
					Notre expérimentation utilisant un écran partagé, le canal visuel ne permet pas d'adresser l'information de manière aussi ciblée.

					De plus, on constate sur la \myref{fig-exp4-NombreDeDesignationsAccepteesAuCoursDuProcessusDeDesignationAvecEtSansHaptique} que le nombre de requêtes acceptées par les opérateurs est significativement supérieur avec l'assistance haptique.
					Les désignations sont mieux perçues avec l'assistance haptique ce qui permet aux opérateurs d'être plus réactifs.
					Ainsi, la stratégie proposée par le coordinateur est mieux adoptée par les opérateurs.

					Les opérateurs étant plus rapides pour accepter une désignation avec l'assistance haptique, le temps d'attente du coordinateur entre chaque désignation est réduit.
					De cette manière, la fréquence à laquelle il effectue des désignations est augmentée et sa vitesse moyenne de travail est augmentée \myref*{fig-exp4-VitesseMoyenneDuCoordinateurAvecEtSansHaptique}.
					Les étapes du processus de désignation étant plus rapides, les opérateurs attendent moins que le coordinateur leur donne une nouvelle tâche à effectuer.

					La \myref{fig-exp4-TempsDeParoleDesSujetsAvecEtSansHaptique} ne montre pas d'évolution du temps de parole lié à la présence ou à l'absence d'une assistance haptique.
					Les canaux visuels et haptiques ne semblent pas substituables mais plutôt complémentaires.
					Cependant, la \myref{fig-exp4-TempsDeParoleDesSujetsEnFonctionDeLOrdreDePassageDeLAssistanceHaptique} montre que l'ordre de passage a une influence forte sur la communication verbale.
					En effet, les sujets qui commencent par l'utilisation des outils haptiques produisent significativement plus d'échanges verbaux que les sujets qui commencent sans assistance haptique.

					En ce qui concerne les sujets qui commencent sans assistance haptique, ils éprouvent dès le début le besoin de communiquer verbalement puisque qu'aucun autre moyen de communication ou d'assistance à la communication n'est mis à leur disposition.
					Lorsqu'on ajoute l'assistance haptique dans la seconde étape de l'expérimentation, les sujets ont déjà pu identifier les problématiques d'interaction et utilisent l'assistance haptique à bon escient qui se substitue aux communications verbales.

					Par opposition, les sujets commençant par l'utilisation de l'assistance haptique apprennent à utiliser les outils et découvrent les tâches à réaliser dans le même temps.
					Lorsqu'on supprime l'assistance haptique dans la seconde étape de l'expérimentation, ils doivent redéfinir les méthodes de travail et compensent ce manque par une augmentation des communications verbales.

					On constate dans cette section que les outils d'assistance à la communication remplissent leur rôle en améliorant l'efficacité et la qualité de la communication.
					Ceci nous permet de valider l'hypothèse~\myhypothesis{2}.
					En effet, en améliorant la qualité de l'information -- signaler directement aux opérateurs qu'une désignation est en cours -- le temps mis par un opérateur pour être opérationnel sur cette désignation est réduit et l'efficacité globale du processus de désignation est augmentée.
					De plus, l'utilisation de ces outils haptiques se substitue en partie aux échanges verbaux ce qui rend les canaux de communication verbaux et haptiques complémentaires.
				\end{mysubsubsection}
			\end{mysubsection}
			\begin{mysubsection}[sse-exp4-EvaluationQualitative]{Évaluation qualitative}
				Lors de cette expérimentation, un questionnaire a été proposé aux sujets abordant deux points essentiels : la conscience et l'utilisabilité.
				Les résultats de ces questionnaires sont discutés dans cette section.
				\begin{mysubsubsection}[sss-exp4-QuestionnaireSurLaCommunicationEtLaConsciencePeripherique]{Questionnaire sur la communication et la conscience périphérique}
					Les utilisateurs pensent que la plateforme collaborative leur permet d'être relativement conscients des actions de leurs partenaires puisqu'ils estiment comprendre les requêtes et consignes de leurs collègues rapidement (\mysummary{exp4-evaluation-resume-2.tex}) et qu'ils considèrent, à chaque instant, être conscients de la position spatiale virtuelle des autres collaborateurs (\mysummary{exp4-evaluation-resume-1.tex})\footnote{L'échelle de notation est comprise entre \mynum{1} et \mynum{5} mais les moyennes ont été normalisées entre \mynum{0} et \mynum{4}.}.
					D'ailleurs, les sujets n'estiment pas nécessaire de signaler leurs actions aux partenaires (\mysummary{exp4-evaluation-resume-3.tex}).
					Avec cette communication à sens unique, c'est la plateforme et ses outils (système de visualisation, assistance haptique) qui permet aux sujets d'avoir une conscience de groupe.

					Pourtant, les sujets participent inconsciemment au processus de communication pour supporter la conscience de groupe (\mysummary{exp4-evaluation-resume-4.tex}).
					En effet, les annotations vidéos ont permis de relever, à plusieurs reprises, le besoin d'informer les partenaires sur l'état actuel de l'environnement virtuel (les désignations en cours, les déformations à effectuer, l'évolution du score \myacro{acr-RMSD}, \myetc).
					Nous pouvons citer par exemple les cas où un coordinateur indique oralement une cible désignée à un opérateur qui ne l'aurait pas vu ou encore un opérateur qui indique au coordinateur un \myglos{glo-Residu} nécessitant une déformation.

					Les sujets participent et soutiennent, par leurs actions, la conscience de groupe.
					Cependant, ce soutien est souvent inconscient car les sujets ne font pas souvent l'effort de signaler explicitement leurs propres actions ou leur propre situation dans l'environnement virtuel.
					Seuls les cas d'incompréhension ou de litiges donnent lieu à des précisions verbales des sujets concernés.
					Il semble que l'assistance haptique ne soit pas utilisée dans ce cas.
					En effet, un test de \mycite[author]{Wilcoxon-1945} montre qu'il n'y a aucun effet significatif de l'assistance haptique \myvari{1} sur :
					\begin{itemize}
						\item la compréhension rapide des actions effectuées par les collègues (\myanova{exp4-evaluation-resume-2-anova.tex});
						\item le besoin de signaler ses propres actions (\myanova{exp4-evaluation-resume-3-anova.tex});
						\item le besoin d'informer les partenaires lors de l'acceptation d'une requête du coordinateur (\myanova{exp4-evaluation-resume-4-anova.tex}).
					\end{itemize}

					Cependant, les sujets estiment que les outils d'assistance haptique permettent souvent de se passer de communication verbale (\mysummary{exp4-evaluation-resume-5.tex}).
					Un test de \mycite[author]{Wilcoxon-1945} montre un effet significatif de l'assistance haptique sur le besoin des sujets à utiliser la communication verbale (\myanova{exp4-evaluation-resume-5-anova.tex}).
					Les outils d'assistance haptique ont été développés pour améliorer la communication pour le processus de désignation.
					À travers le questionnaire, les sujets pensent, de manière modérée, avoir une meilleure communication entre eux avec l'assistance haptique (\mysummary{exp4-evaluation-resume-5-true.tex}) que sans assistance haptique (\mysummary{exp4-evaluation-resume-5-false.tex}).
					On peut expliquer ce résultat mitigé par le fait que le processus de désignation n'occupe qu'une portion du temps total de réalisation de la tâche; le reste du temps, les moyens habituels de communication sont utilisés par les sujets.

					De plus, le rôle de l'assistance haptique s'est montré particulièrement utile dans la conscience de la position spatiale virtuelle des partenaires (\myanova{exp4-evaluation-resume-1-anova.tex}).
					Les sujets estiment mieux connaître la position spatiale virtuelle de leurs partenaires avec l'assistance haptique (\mysummary{exp4-evaluation-resume-1-true.tex}) que sans l'assistance haptique (\mysummary{exp4-evaluation-resume-1-false.tex}).
					Pourtant, les outils d'assistance haptique n'indique en rien la position mais plutôt le statut des partenaires (en cours d'acceptation, en cours de déformation, \myetc).
					On peut supposer que les sujets souhaitent parfois connaître le statut (en cours de déformation, en recherche d'une tâche à effectuer, \myetc) de leur partenaire et que cette information ne nécessite pas d'être accompagnée de la position spatiale virtuelle des partenaires concernés; les sujets cherchent seulement les informations dont ils ont besoin pour obtenir une conscience de groupe adaptée à la situation.
				\end{mysubsubsection}
				\begin{mysubsubsection}[sss-exp4-QuestionnaireDUtilisabilite]{Questionnaire d'utilisabilité}
					Cette dernière expérimentation a été l'occasion d'évaluer l'utilisabilité de la plateforme \myShaddock à l'aide du score \myacro{acr-SUS} décrit dans la \myref{sse-q-QuatriemeExperimentation-LeQuestionnaireSUS}.
					Cette évaluation, menée sur des sujets dont un quart de biologistes, permet d'obtenir un score d'utilisabilité de l'application (\mysummary{exp4-evaluation-sus.tex})\footnote{Pour rappel, les tests d'utilisabilité \myacronl-{acr-SUS} fournissent des scores compris entre \mynum{0} et \mynum{100}.}.
					\mycite[author]{Bangor-2009} a conduit une étude sur \mynum{3500}~enquêtes menées avec ce test pour obtenir une moyenne : le score moyen est de \mynum{68}.
					Cependant, si on restreint notre étude de l'utilisabilité aux sujets biologistes, le score augmente (\mysummary{exp4-evaluation-sus-biologists.tex}).

					D'après les utilisateurs, notre plateforme est encore considérée comme incomplète.
					\mycite[author]{Bangor-2009} propose également de noter les applications évaluées avec le score \myacro{acr-SUS} sur une échelle à sept niveaux : \textit{Worst imaginable}, \textit{Awful}, \textit{Poor}, \textit{Ok}, \textit{Good}, \textit{Excellent} et \textit{Best imaginable}.
					Selon ce score, \myShaddock se place dans la catégorie \textit{Ok} ce qui en fait un application utilisable en l'état mais pour laquelle des améliorations sont nécessaires.
					L'hypothèse~\myhypothesis{3} n'est pas validée avec ces résultats mitigés.

					Une limite importante de la plateforme concerne le système de visualisation.
					En effet, une proportion importante des utilisateurs se sont plaint d'une grande difficulté à percevoir la dimension de profondeur dans l'environnement virtuel.
					Plusieurs utilisateurs ont évoqué le besoin d'avoir un système de visualisation en \myThreeD stéréoscopique.

					La présence de biologistes dans le panel de sujets a également permis de mettre en évidence différents points critiques à prendre en compte pour les développements futurs.
					En particulier, ils auraient souhaité pouvoir stabiliser un \myglos{glo-Residu} dans sa position finale ou encore pouvoir déformer des structures moléculaires de taille plus importante que les \myglos*{glo-Residu} tels que les structures secondaires (\myhelice* et \myfeuillet*).
					Ce dernier point a été abordé dans la \myref{sss-Shaddock-DeformationParGroupeDAtomes}.
				\end{mysubsubsection}
			\end{mysubsection}
		\end{mysection}
		\begin{mysection}[sec-exp4-Conclusion]{Conclusion}
			\begin{mysubsection}[sse-exp4-ResumeDesResultats]{Résumé des résultats}
				Cette expérimentation avait pour objectif d'évaluer de nouveaux outils de communication entre les membres d'un groupe sur une tâche collaborative étroitement couplée.
				De plus, nous avons pu confronter la plateforme \myShaddock à des bio-informaticiens, ayant une expertise dans le domaine du \myglos{glo-DockingMoleculaire}.

				Il ressort de cette expérimentation que les outils d'assistance à la communication permettent d'améliorer significativement la communication entre les membres du groupe avec pour conséquence une amélioration des performances, notamment dans les tâches les plus complexes.
				L'utilisation de l'haptique à bon escient permet de diminuer les temps de réponse lors des communications tout en adressant directement les informations à transmettre aux membres concernés.
				De plus, la répartition de la communication sur différentes canaux sensoriels permet de soulager de la charge cognitive de travail pour chaque utilisateur.

				Avec le questionnaire portant sur la conscience de groupe, on constate que la conscience repose principalement sur les outils fournis par la plateforme, à savoir les retours visuels et les retours haptiques, lorsqu'ils sont disponibles.
				En ce sens, la communication haptique est pertinente puisqu'elle permet d'améliorer les performances en augmentant la conscience périphérique.
				Par exemple, elle permet d'indiquer rapidement l'action d'un partenaire ou elle guide rapidement les utilisateurs les uns vers les autres.

				Les questionnaires d'utilisabilité ont également permis de montrer que la plateforme souffre d'un défaut important : la visualisation, notamment les problèmes liés à la perception de la profondeur.
				Cependant, un système de visualisation stéréoscopique devrait permettre de résoudre en grande partie ce problème.

				En ce qui concerne les autres aspects de la plateforme \myShaddock, les biologistes ont commentés la plateforme de manière relativement positive excepté pour les problèmes de visualisation.
				Ils ont tout de même suggéré des outils qui, selon leur propre expérience, seraient nécessaires à une utilisation plus pertinente de cette plateforme.
				En particulier, ils ont émis le désir de pouvoir stabiliser la position des atomes après les avoir placés dans une position finale.
				Le second outil dont ils auraient aimé disposer est la déformation de blocs tels que les structures secondaires.
				Cependant, les contraintes de manipulation d'un tel ensemble d'atomes, exposée dans la \myref{sss-Shaddock-DeformationParGroupeDAtomes}, entraînerait des modifications importantes sur les métaphores de manipulation.
				Globalement, la plateforme a séduit les utilisateurs mais a montré quelques faiblesses importantes qui la rende encore trop instable dans l'état actuel.
			\end{mysubsection}
			\begin{mysubsection}[sse-exp4-SyntheseEtPerspectives]{Synthèse et perspectives}
				Cette dernière expérimentation nous a permis de mettre en avant la pertinence de la communication haptique dans le cadre d'une tâche de collaboration étroitement couplée.
				L'outil haptique proposé ne concerne que le processus de désignation mais d'autres aspects de la communication peuvent être explorés.
				On peut se reporter aux travaux de \mycite[author]{Oakley-2001} qui propose plusieurs métaphores sur le sujet qui devrait permettre d'améliorer la conscience de groupe comme l'ajout de la perception haptique entre les utilisateurs au sein de l'environnement virtuel ou encore la possibilité d'emmener le curseur d'un partenaire pour le guider vers une cible.

				En ce qui concerne les améliorations de la plateforme \myShaddock, de nombreuses propositions ont été suggérées par les bio-informaticiens.
				Quelques-unes seront intégrées dans un futur proche, d'autres nécessitent une réflexion plus approfondie sur la pertinence et sur la faisabilité.
				Quoiqu'il en soit, devant l'enthousiasme des sujets pour cette expérimentation relativement longue ($\approx \mynum[mn]{75}$), \myShaddock semble fournir les bases suffisantes pour une plateforme pertinente de \myglos{glo-DockingMoleculaire} en temps-réel.
			\end{mysubsection}
		\end{mysection}
	\end{mychapter}
	\begin{mychapter+}{Conclusion et perspectives}
		Tout au long de ce travail de thèse, nous avons étudié le travail collaboratif dans le contexte particulier de la collaboration étroitement couplée pour la résolution de tâches complexes, en l'occurence, le \myglos{glo-DockingMoleculaire}.
		Cette conclusion va nous permettre de mettre en avant les similitudes entre les travaux existants sur la collaboration et nos résultats, mais également certaines distinctions importantes.

		Tout d'abord, nous avons pu constater le gain en performances lié à la distribution cognitive des charges de travail, notamment dans la seconde étude \myref*{cha-DeformationCollaborativeDeMolecule} dans laquelle nous comparons deux distributions différentes d'un même nombre de ressources; ces conclusions vont dans le même sens que les études actuelles sur la collaboration.
		Cependant, une différence importante par rapport aux travaux existants a été relevée durant les différentes études.
		En effet, la collaboration étroitement couplée provoque d'avantage d'interactions entre les utilisateurs (accès à une zone, besoin de coordination, \myetc) ce qui entraîne d'avantage de \myglos*{glo-ConflitDeCoordination}.
		La troisième étude nous a permis de constater que ces \myglos*{glo-ConflitDeCoordination} avaient pour effet de réduire considérablement voire d'annuler complétement les gains en performances.

		Nous avons également pu constater les effets de la \myglos{glo-FacilitationSociale} semblent particulièrement présents dans ce type de configuration synchrone et colocalisée.
		Cet effet semble également présent dans le cas de tâches complexes, contrairement aux conclusions de \mycite[author]{Zajonc-1965} obtenues pour des collaborations faiblement couplées.
		Les tâches complexes générant une collaboration étroitement couplée et les interactions fortes entre utilisateurs qui en découle semble les stimuler.
		Cependant, modérons ce propos car les tâches proposées sont toujours, dans l'esprit du sujet, des tâches réalisables.
		Il n'est pas certain que les sujets soient autant motivés s'ils étaient confrontés à des tâches dont on ne connaît pas la faisabilité \myapriori.

		Au cours de la troisième étude \myref*{cha-LaDynamiqueDeGroupe}, nous avons également pu arriver à des conclusions similaires à celles obtenues par \mycite[author]{Bales-1950} concernant l'apparition de membres dominants au sein d'un groupe.
		Ce point avait besoin d'être validé dans le contexte particulier de la collaboration étroitement couplée car les mécanismes de communication et les interactions entre les utilisateurs sont différents : cette forme de structuration automatique du groupe aurait pu ne pas se réaliser de la même manière.

		L'ensemble de ces conclusions nous a mené à la réalisation d'un outil haptique pour assister la communication et plus particulièrement, la tâche élémentaire de désignation.
		L'utilité de cet outil a été confirmée lors de la dernière expérimentation \myref*{cha-TravailCollaboratifAssisteParHaptique} dans laquelle une évolution significative en terme de communication a été observée.
		Cette expérimentation \myref*{cha-TravailCollaboratifAssisteParHaptique} nous a permis d'apporter une réponse à la proposition de \mycite[author]{Moll-2009} d'utiliser l'haptique comme moyen de désignation : la modalité haptique est pertinente pour la désignation.
		Plus généralement, cette expérimentation permet de montrer l'utilité de la modalité haptique dans le processus de communication pour les collaborations étroitement couplées en environnements complexes.

		L'étude et l'amélioration des interactions en environnement collaboratifs a déjà donné lieu à quelques travaux.
		Par exemple, les travaux de thèse de \mycite[author]{Ullah-2011} s'intéressent à l'utilisation d'une assistance multi-modale en environnement collaboratif.
		De mes travaux découle également le projet de thèse, actuellement mené par Adrien \myname{Girard}, concernant la proposition d'outils haptiques pour assister la collaboration, particulièrement en ce qui concerne les aspects de conscience périphérique et de présence.

		Pour aller plus loin, une première piste concerne la mise en collaboration de plusieurs experts provenant de spécialités différentes.
		En effet, nou avons pu voir que le \myglos{glo-DockingMoleculaire} fait appel à des spécialités diverses.
		La collaboration multi-experts amènera de nouvelles contraintes de collaboration.
		Notre dernière expérimentation n'a fait qu'effleurer le sujet en proposant des niveaux de collaboration différents (coordinateur ou opérateur) mais l'intervention de différents experts pourrait nécessiter des modifications de l'environnement plus profondes (environnements de visualisation privés, outils adaptés à la spécialité, \myetc).

		Le besoin de travailler à distance est également très présent à travers les différentes applications collaboratives comme on a pu le voir tout au long de l'étude bibliographique \myref*{cha-sota-EtudeBibliographique}.
		La collaboration à distance entraîne de nouvelles contraintes, en ce qui concerne les contraintes techniques mais surtout en ce qui concerne la communication.
		Nous avons délibéremment restreints nos travaux de thèse à un environnement colocalisé afin de bénéficier de moyens de communication optimaux : audio, visuel, gestuels, tactiles, \myetc

		Un aspect plus théorique de notre travail doit également être abordé.
		Tout au long de notre travail, nous avons pu mettre en évidence différents \myglos*{glo-ConflitDeCoordination}.
		Cependant, nous avons eu des difficultés pour mesurer, quantifier ou même qualifier ces \myglos*{glo-ConflitDeCoordination}.
		Il est nécessaire de définir des mesures objectives permettant de répondre à cette problématique.

		De nombreuses pistes restent à explorer.
		Cependant, ce travail de thèse permet de fournir les premières pistes pour la communication haptique en environnement complexe.
	\end{mychapter+}
	\mybiblio%
	\myglossary
	\myappendix
	\begin{mychapter}[cha-pro-DispositifExperimental]{Dispositif expérimental}
		\begin{mysection}[sec-pro-MaterielExperimental]{Matériel expérimental}
			Les expérimentations se basent sur l'\myacro{acr-EVC} présenté dans le \myref{cha-Shaddock-ManipulationCollaborativeDeMolecules}.
			Dans cette section, nous allons présenter le matériel utilisé et sa disposition.

			Tout d'abord, voici le matériel de base utilisé pour les différentes expérimentations :
			\begin{itemize}
				\item \mynum{2}~ordinateur quatre cœurs \myIntelCore avec \myRAM[Go]{4};
				\item \mynum{2}~interfaces haptiques \myOmni;
				\item \mynum{1}~vidéoprojecteur \myACER (\textsc{p5}~series)\footnote{Pour la première expérimentation, c'est un vidéoprojecteur \myCasioXJ qui a été utilisé.};
				\item \mynum{1}~grand écran de vidéoprojection.
			\end{itemize}

			Un premier ordinateur~\mycomputer{A} est celui d'où l'expérimentateur va commander l'ensemble de l'expérimentation.
			Cet ordinateur est destiné à l'application cliente \myacro{acr-VMD} : c'est donc cette machine qui s'occupe du calcul pour les rendus visuels.
			La seconde machine~\mycomputer{B} est dédiée au moteur de simulation \myacro{acr-NAMD} : elle communique avec la machine~\mycomputer{A} par une connexion \myTCPIP.

			L'affichage de l'environnement virtuel est assuré par un vidéoprojecteur connecté à l'ordinateur~\mycomputer{A}.
			Le vidéo projecteur est placé derrière les sujets et projette la scène virtuelle sur un grand écran de \mynum[m]{2.2} par \mynum[m]{2}.
			L'écran est placé face aux sujets et tous les sujets percoivent la même scène virtuelle.
			Afin que la communication entre les sujets soit optimales, aucune contrainte de communication ne leur est donnée et ils sont libres d'utiliser tous les moyens de communication possibles (verbaux, gestuels, virtuels, \myetc).

			Les ordinateurs~\mycomputer{A} et~\mycomputer{B} sont également utilisés en tant que serveur \myacro{acr-VRPN}.
			Un \myOmni est connecté sur chacune des deux machines.
			Ces interfaces haptiques sont placées sur une table devant les sujets.
			Les sujets ont la possibilité de déplacer les interfaces haptiques (avec l'aide de l'expérimentateur) afin de s'installer confortablement et d'utiliser la main qu'ils désirent pour la manipulation du périphérique.

			Ce qui vient d'être décrit est la plateforme de base qui est utilisée au cours des différentes expérimentations.
			Cependant, des spécificités liées aux tâches proposées durant les différentes expérimentations sont détaillées au-fur-et-à-mesure.
		\end{mysection}
		\begin{mysection}[sec-pro-PresentationsDesMolecules]{Présentation des molécules}
			Durant les différentes expérimentations, plusieurs molécules ou complexe de molécules ont été utilisées.
			À partir de ces molécules, différents scénarios ont été conçus et les difficultés sont décrites au-fur-et-à-mesure de la présentation des différentes expérimentation.
			Tout d'abord, nous présenterons la liste des molécules utilisées.
			Puis nous expliquerons le rendu visuel utilisé dans tous les expérimentations.
			\begin{mysubsection}[sse-pro-ListeDesMolecules]{Liste des molécules}
				Chaque molécule utilisée est référencée sur la \myPDBbase\footnote{\url{http://www.pdb.org/}} par un identifiant \myPDB.
				Voici la liste des molécules utilisées :
				\begin{description}
					\item[\myTRPZIPPER]
						La molécule \myTRPZIPPER \mycite{Cochran-2001} a pour identifiant \myPDB \myPDBlink{http://www.rcsb.org/pdb/explore/explore.do?structureId=1LE1}{1LE1}.
						Cette molécule contient \mynum{218}~atomes dont \mynum{12}~\myglos*{glo-Residu}.
					\item[\myTRPCAGE]
						La molécule nommée \myTRPCAGE \mycite{Neidigh-2002} a pour identifiant \myPDB \myPDBlink{http://www.rcsb.org/pdb/explore/explore.do?structureId=1L2Y}{1L2Y}.
						Cette molécule contient \mynum{304}~atomes dont \mynum{20}~\myglos*{glo-Residu}.
					\item[\myPrion]
						La molécule nommée \myPrion \mycite{Christen-2009} avec l'identifiant \myPDB \myPDBlink{http://www.rcsb.org/pdb/explore/explore.do?structureId=2KFL}{2KFL}.
						Cette molécule contient \mynum{1779}~atomes dont \mynum{112}~\myglos*{glo-Residu}.
					\item[\myUbiquitin]
						La molécule nommée \myUbiquitin \mycite{Vijay-Kumar-1987} avec l'identifiant \myPDB \myPDBlink{http://www.rcsb.org/pdb/explore/explore.do?structureId=1UBQ}{1UBQ}.
						Cette molécule contient \mynum{1231}~atomes dont \mynum{76}~\myglos*{glo-Residu}.
					\item[\myNusENusG]
						Le complexe de molécules \myNusENusG \mycite{Burmann-2010} a pour identifiant \myPDB \myPDBlink{http://www.rcsb.org/pdb/explore/explore.do?structureId=2KVQ}{2KVQ}.
						Il est constitué de deux molécules \textsc{NusE} et \textsc{NusG} possédant respectivement \mynum{1294}~atomes pour \mynum{80}~\myglos*{glo-Residu} et \mynum{929}~atomes pour \mynum{59}~\myglos*{glo-Residu}.
				\end{description}

				On notera la présence de molécule de taille relativement petite comme \myTRPZIPPER et \myTRPCAGE.
				On trouve également des molécules de taille assez importante comme \myPrion et \myUbiquitin.
				Enfin, pour la dernière expérimentation, un complexe de molécules a été utilisé avec \myNusENusG.
			\end{mysubsection}
			\begin{mysubsection}[sse-pro-RepresentationDesMolecules]{Représentation des molécules}
				La représentation des molécules est un domaine de recherche à part entière toujours d'actualité \mycite{Chavent-2011}.
				En effet, la complexité et l'abondance d'informations à visualiser nécessite des rendus graphiques avancés et complémentaires.
				De plus, la quantité importante d'informations à représenter peut nécessiter une machine puissante afin de générer un rendu en temps-réel.
				Heureusement, \myacro{acr-VMD} possède un moteur de rendu graphique avancé \myref*{sse-Shaddock-ModuleDeVisualisationMoleculaire}, aussi bien en terme de choix de rendu qu'en terme d'accélération graphique.

				Afin d'obtenir un rendu de molécule pertinent, nous avons bénéficié des conseils d'un biologiste.
				Ensuite, nous avons pu adapter les rendus de molécules en fonction de nos besoins pour les différents scénarios proposés.
				Cependant, une base commune a été utilisée.

				Tout d'abord, les atomes étant l'élément constituant de la molécule, il est nécessaire de les représenter en intégralité.
				Cependant, ils sont très nombreux et produisent rapidement une surcharge de la scène donc le choix de leur taille est primordial.
				Une première solution est de s'affranchir, partiellement, des atomes d'hydrogène.
				En effet, ces derniers ne constituent pas une information importante et peuvent être déduits à partir du reste de la structure de la molécule.
				Les atomes d'hydrogènes peuvent donc être représentés avec une taille réduite par rapport aux autres atomes.
				Le rendu \myCPK est utilisé pour effectuer un rendu des atomes \myref*{fig-pro-RepresentationDesAtomesAvecCPK}.

				\begin{myfigure}
					\myimage{protocole-render-CPK}
					\mycaption[fig-pro-RepresentationDesAtomesAvecCPK]{Représentation des atomes avec \myCPK}
				\end{myfigure}

				Cependant, la représentation de la molécule exclusivement avec les atomes et les liaisons entre les atomes ne permet pas d'appréhender la structure globale.
				En effet, on peut voir une molécule comme un long brin qui se replie sur lui-même avec des feuilles tout le long du brin.
				Il est donc pertinent de représenter cette structure principale.
				C'est la représentation \myNewRibbon qui tient ce rôle \myref*{fig-pro-RepresentationDeLaStructurePrincipaleDeLaMoleculeAvecNewRibbon}.

				\begin{myfigure}
					\myimage{protocole-render-NewRibbon}
					\mycaption[fig-pro-RepresentationDeLaStructurePrincipaleDeLaMoleculeAvecNewRibbon]{Représentation de la structure principale de la molécule avec \myNewRibbon}
				\end{myfigure}

				Pour finir, pour des raisons physiques d'interaction, certains atomes sont fixés au niveau de la simulation afin d'éviter des dérives de la molécule.
				Ces atomes sont signalés visuellement par une représentation en gris \myref*{fig-pro-RepresentationDesAtomesFixesEnGris}.

				\begin{myfigure}
					\myimage{protocole-render-fixed}
					\mycaption[fig-pro-RepresentationDesAtomesFixesEnGris]{Représentation des atomes fixés en gris}
				\end{myfigure}
			\end{mysubsection}
		\end{mysection}
		\begin{mysection}[sec-pro-OutilsDeManipulation]{Outils de manipulation}
			La plateforme de base propose deux interfaces haptiques.
			Ces deux interfaces haptiques sont utilisées comme interfaces de déformation de la molécule : des outils \mytool{tug}.
			Pour comprendre ce que sont des outils de déformation, on peut se reporter à la \myref{sse-Shaddock-OutilsExistants}.
			Au cours des trois premières expérimentations, seules quelques modifications du rendu visuel associés à ces outils sont effectués.
			Cependant, la quatrième expérimentation apporte des modifications plus lourdes de cet outil que ce soit au niveau visuel ou au niveau haptique.
			On pourra se reporter aux chapitres respectifs pour plus de détails.

			De plus, un outil de manipulation et d'orientation de la molécule sera proposé sous différentes formes au cours des différentes expérimentations.
			Ce sera par l'intermédiaire d'un outil \mytool{grab} dans la première expérimentation \myref*{sss-exp1-MaterielUtilise}, par une souris~\myThreeD dans la seconde \myref*{sss-exp2-MaterielUtilise} puis par une simple souris \myUSB dans la dernière expérimentation \myref*{sss-exp4-MaterielUtilise}.
		\end{mysection}
	\end{mychapter}
	\begin{mychapter}[cha-met-MethodeExperimentale]{Méthode expérimentale}
		\begin{mysection}[sec-met-exp1-PremiereExperimentation]{Première expérimentation}
			\begin{mysubsection}[sse-met-exp1-Hypotheses]{Hypothèses}
				Nous émettons plusieurs hypothèses concernant cette première expérimentation.
				Les hypothèses concernent les performances des \myglos*{glo-Binome} ainsi que leurs stratégies de travail.
				Nous souhaitons également recueillir les avis des sujets sur l'intérêt qu'ils portent à la collaboration ainsi que sur l'utilisabilité de la plateforme \myShaddock.
				\begin{myparagraph}[par-met-exp1-AmeliorationDesPerformancesEnBinome]{\myhypothesis{1} Amélioration des performances en \myglosnl{glo-Binome}}
					Nous émettons l'hypothèse que les performances des \myglos*{glo-Binome} seront meilleures que les performances des \myglos*{glo-Monome} grâce à la distribution des charges de travail.
					Les performances seront évaluées en terme de temps de réalisation de la tâche mais aussi en terme de ressources utilisées comme le nombre de sélections effectuées.
				\end{myparagraph}
				\begin{myparagraph}[par-met-exp1-StrategiesVariablesEnFonctionDesBinomes]{\myhypothesis{2} Stratégies variables en fonction des \myglosnl*{glo-Binome}}
					Nous émettons l'hypothèse que les \myglos*{glo-Binome} adopteront des stratégies de collaboration différentes en fonction des affinités entre les sujets et de leurs connaissances respectives.
					L'identification des différentes stratégies permettra d'identifier celles qui obtiennent les meilleures performances.
				\end{myparagraph}
				\begin{myparagraph}[par-met-exp1-LesSujetsPreferentLeTravailEnBinome]{\myhypothesis{3} Les sujets préfèrent le travail en \myglosnl{glo-Binome}}
					Nous émettons l'hypothèse que les sujets préfèrent le travail en collaboration grâce à l'aspect social que celui-ci implique.
				\end{myparagraph}
				\begin{myparagraph}[par-met-exp1-BonneUtilisabiliteDeLaPlateforme]{\myhypothesis{4} Bonne utilisabilité de la plateforme}
					La dernière hypothèse que nous émettons est que notre plateforme est utilisable dans la forme sous laquelle elle est proposée (intuitivité, ergonomie, \myetc).
					Cependant, nous souhaitons identifier les faiblesses à l'aide d'un questionnaire soumis aux sujets.
				\end{myparagraph}
			\end{mysubsection}
			\begin{mysubsection}[sse-met-exp1-Sujets]{Sujets}
				\mysummary{exp1-subjects.tex} avec une distribution d'âge de \mysummary{exp1-age.tex} ont participé à cette expérimentation.
				Ils ont tous été recrutés au sein du \myacro{acr-LIMSI} et sont chercheurs, assistants de recherche, étudiants en thèse ou stagiaires dans les domaines suivants~:
				\begin{itemize}
					\item linguistique et traitement automatique de la parole;
					\item réalité virtuelle et système immersifs;
					\item audio-acoustique.
				\end{itemize}

				Tous les sujets sont francophones.
				Aucun participant n'a de déficience visuelle (ou corrigée le cas échéant), de déficience audio ou de déficience moteur du haut du corps.
				Les sujets ne sont pas rémunérés pour l'expérimentation.

				Chaque participant est complètement naïf concernant les détails de l'expérimentation.
				Une explication détaillée de la procédure expérimentale leur est donnée au commencement de l'expérimentation.
				Cependant, l'objectif de l'expérimentation n'est pas révélé.
			\end{mysubsection}
			\begin{mysubsection}[sse-met-exp1-Variables]{Variables}
				\begin{mysubsubsection}[sss-met-exp1-VariablesIndependantes]{Variables indépendantes}
					\begin{myparagraph}[par-met-exp1-NombreDeSujets]{\myvari{1} Nombre de sujets}
						C'est une \myglos{glo-VariableIntraSujets}.
						\myvari{1} possède deux modalités : \og un sujet \fg (\mycf \myemph{\myglos{glo-Monome}}) ou \og deux sujets \fg (\mycf \myemph{\myglos{glo-Binome}}).
						\mynum{24}~\myglos*{glo-Monome} et \mynum{12}~\myglos*{glo-Binome} ont été testés.
					\end{myparagraph}
					\begin{myparagraph}[par-met-exp1-ResiduRecherche]{\myvari{2} \myGlosnl{glo-Residu} recherché}
						C'est une \myglos{glo-VariableIntraSujets}.
						\myvari{2} concerne les \myglos*{glo-Residu} recherchés qui sont au nombre de \mynum{10} répartis à part égale dans deux molécules \myref*{tab-exp1-ListeDesResidusRecherches}.
						Différents niveaux de complexité caractérisent chaque \myglos{glo-Residu} \myref*{tab-exp1-ParametresDeComplexiteDesResidus}.
					\end{myparagraph}
				\end{mysubsubsection}
				\begin{mysubsubsection}[sss-met-exp1-VariablesDependantes]{Variables dépendantes}
					\begin{myparagraph}[par-met-exp1-TempsDeRealisation]{\myvard{1} Temps de réalisation}
						Ce temps est le temps total pour réaliser la tâche demandée, c'est-à-dire trouver le \myglos{glo-Residu} et l'extraire de la molécule.
						Il n'y a pas de limite de temps pour réaliser la tâche.
						Ce temps est divisé en deux phases bien distinctes :
						\begin{description}
							\item[L'exploration] C'est la phase pendant laquelle les sujets cherchent le \myglos{glo-Residu}.
								Cette exploration peut se limiter à une exploration visuelle en orientant et en déplaçant la molécule.
								Elle peut aussi s'effectuer par la déformation de la molécule afin d'atteindre les \myglos{glo-Residu} inaccessibles (derrière ou au centre de la molécule).
							\item[La sélection] La phase de sélection débute dès l'instant où un des deux sujets a identifié visuellement le \myglos{glo-Residu}.
								Elle est constituée d'une étape de sélection puis d'une étape d'extraction hors de la molécule.
						\end{description}
					\end{myparagraph}
					\begin{myparagraph}[par-met-exp1-LaDistanceEntreLesEspacesDeTravail]{\myvard{2} La distance entre les espaces de travail}
						Cette mesure est la distance moyenne entre les deux \myglos*{glo-EffecteurTerminal} correspondant aux outils \mytool{tug}.
						Elle est mesurée dans le monde réel mais peut être convertie dans l'environnement virtuel (à l'échelle de la molécule).
						L'ordre de grandeur de cette mesure est le centimètre.
					\end{myparagraph}
					\begin{myparagraph}[par-met-exp1-CommunicationsVerbales]{\myvard{3} Communications verbales}
						L'enregistrement des communications verbales permet de mesurer la durée de parole de chaque sujet pour chaque étape de l'expérimentation.
						Ces mesures segmentent la phase d'exploration et la phase de sélection (voir \myvard{1}) comme indiqué plus précisément sur la \myref{fig-exp1-EtapesDeLaCommunicationVerbalePourLaRechercheDUnResidu}.

						\begin{myfigure}
							\psset{unit=0.1\textwidth} % Fill entirely the page width
							\begin{myps}(0,-1.75)(10,1.5)
								\psset{linewidth=1pt,linecolor=black}%
								\psset{fillstyle=solid}%
								\psframe[fillcolor=mylightblue](0,-0.5)(6,0.5)%
								\pspolygon[fillcolor=mylightred](6,-0.5)(6,0.5)(9,0.5)(10,0)(9,-0.5)%
								\uput{16pt}[180](10,0){\LARGE\sl\textcolor{white!33}{temps}}
								\psbrace[ref=lC,rot=-90,nodesepA=-3,nodesepB=-0.25](6,0.5)(0,0.5){%
									\parbox{6\psxunit}{%
										\centering\textcolor{myblue}{Temps de recherche}%
									}%
								}%
								\psbrace[ref=lC,rot=-90,nodesepA=-2,nodesepB=-0.25](10,0.5)(6,0.5){%
									\parbox{4\psxunit}{%
										\centering\textcolor{myred}{Temps de sélection}%
									}%
								}%
								\psframe[fillcolor=myblue](1,-0.5)(1.5,0.5)
								\psframe[fillcolor=myblue](3,-0.5)(4.5,0.5)
								\psframe[fillcolor=myblue](4.8,-0.5)(5,0.5)
								\psframe[fillcolor=myred](6.5,-0.5)(7.5,0.5)
								\psframe[fillcolor=myred](8,-0.5)(8.25,0.5)
								\pnode(1.25,-0.5){verbal1}
								\pnode(3.75,-0.5){verbal2}
								\pnode(4.9,-0.5){verbal3}
								\pnode(7,-0.5){verbal4}
								\pnode(8.125,-0.5){verbal5}
								\rput(5,-1.5){%
									\Rnode{verbal}{%
										\psframebox[linestyle=none]{\centering Communication verbale}%
									}%
								}%
								\psset{linearc=0.1,angleA=-90}
								\ncdiagg{<-}{verbal1}{verbal}
								\ncdiagg{<-}{verbal2}{verbal}
								\ncdiagg{<-}{verbal3}{verbal}
								\ncdiagg{<-}{verbal4}{verbal}
								\ncdiagg{<-}{verbal5}{verbal}
							\end{myps}
							\mycaption[fig-exp1-EtapesDeLaCommunicationVerbalePourLaRechercheDUnResidu]{Étapes de la communication verbale pour la recherche d'un \myglosnl{glo-Residu}}
						\end{myfigure}
					\end{myparagraph}
					\begin{myparagraph}[par-met-exp1-AffiniteEntreLesSujets]{\myvard{4} Affinité entre les sujets}
						Le degré d'affinité -- concernant uniquement les \myglos*{glo-Binome} -- est compris entre \mynum{1} et \mynum{5} selon les critères suivants :
						\begin{enumerate}
							\item Les sujets ne se connaissent pas;
							\item Les sujets travaillent dans la même entreprise, le même laboratoire;
							\item Les sujets travaillent dans la même équipe, sur les mêmes projets;
							\item Les sujets travaillent ensemble, sont dans le même bureau;
							\item Les sujets sont amis proches.
						\end{enumerate}
					\end{myparagraph}
					\begin{myparagraph}[par-met-exp1-ForceMoyenneAppliqueeParLesSujets]{\myvard{5} Force moyenne appliquée par les sujets}
						La force appliquée par chaque sujet sur les atomes pendant la simulation est mesurée en \myname{Newton}.
						Une valeur moyenne de cette force est calculée pour être analysée.
					\end{myparagraph}
					\begin{myparagraph}[par-met-exp1-ReponsesAuQuestionnaire]{\myvard{6} Réponses au questionnaire}
						Un questionnaire est proposé à tous les sujets.
						Il est constitué de trois sections, avec une section supplémentaire pour les \myglos*{glo-Binome} \myref*{sec-q-PremiereExperimentation}.
					\end{myparagraph}
				\end{mysubsubsection}
			\end{mysubsection}
			\begin{mysubsection}[sse-met-exp1-Procedure]{Procédure}
				L'expérimentation débute par une phase d'apprentissage avec la molécule \myTRPZIPPER.
				Un \myglos{glo-Residu} est proposé et doit être identifier et extrait.
				L'apprentissage est destiné à familiariser les sujets avec la plateforme, les outils de manipulation et la tâche à réaliser.
				Cette phase dure maximum \mynum[mn]{5}.
				L'expérimentateur est disponible pour répondre aux questions des sujets.

				Lorsque l'étape d'apprentissage est terminée, nous présentons aux sujets une série de \mynum{10}~\myglos*{glo-Residu} selon la procédure suivante.
				Le premier \myglos{glo-Residu} est affiché sur l'écran \myLCD et les sujets débutent la phase de recherche.
				Lorsque le \myglos{glo-Residu} est identifié, sélectionné puis extrait hors de la molécule, l'application est arrêtée.
				Ensuite, un second \myglos{glo-Residu} est affiché, l'application est de nouveau démarrée et ainsi de suite pour les \mynum{10}~\myglos*{glo-Residu} à identifier.
				L'enregistrement audio est démarré à la fin de l'étape d'apprentissage.

				L'ensemble des \myglos*{glo-Residu} est proposé dans un ordre aléatoire afin d'éviter un biais lié à l'apprentissage de la plateforme et de la tâche.
				Les sujets sont tenus de trouver et extraire dix \myglos*{glo-Residu} en \myglos{glo-Monome} et dix \myglos*{glo-Residu} en \myglos{glo-Binome}.
				Toujours pour éviter un biais lié à l'apprentissage, les sujets sont soumis aux tâches en \myglos{glo-Monome} et en \myglos{glo-Binome} de façon alternée selon les trois combinaisons suivantes :
				\begin{enumerate}
					\item Le \myglos{glo-Monome} \myuser{A}, puis le \myglos{glo-Monome} \myuser{B}, puis le \myglos{glo-Binome} \myuser{AB};
					\item Le \myglos{glo-Monome} \myuser{A}, puis le \myglos{glo-Binome} \myuser{AB}, puis le \myglos{glo-Monome} \myuser{B};
					\item Le \myglos{glo-Binome} \myuser{AB}, puis le \myglos{glo-Monome} \myuser{A}, puis le \myglos{glo-Monome} \myuser{B}.
				\end{enumerate}

				Lorsque les sujets ont réalisé toutes les tâches dans les deux configurations possibles (\myglos{glo-Monome} et \myglos{glo-Binome}), un questionnaire leur est soumis.
				Chaque sujet répond au questionnaire de manière autonome, sans communication avec son partenaire.
			\end{mysubsection}
		\end{mysection}
		\begin{mysection}[sec-met-exp2-SecondeExperimentation]{Seconde expérimentation}
			\begin{mysubsection}[sse-met-exp2-Hypotheses]{Hypothèses}
				Les hypothèses de cette nouvelle étude sont en grande partie basée sur l'étude précédente.
				Nous souhaitons confirmer l'intérêt du travail collaboratif dans ce contexte de déformation moléculaire, notamment sur les tâches de forte complexité.
				De plus, cette expérimentation propose d'étudier l'apprentissage dans le travail collaboratif et d'en observer l'évolution.
				\begin{myparagraph}[par-met-exp2-AmeliorationDesPerformancesEnBinome]{\myhypothesis{1} Amélioration des performances en \myglosnl{glo-Binome}}
					Nous émettons l'hypothèse que les performances des \myglos*{glo-Binome} seront meilleures que les performances des \myglos*{glo-Monome} sur cette tâche de déformation.
					Cette hypothèse est basée sur les conclusions obtenues dans la première étude.
				\end{myparagraph}
				\begin{myparagraph}[par-met-exp2-BinomesPlusPerformantsSurLesTachesComplexes]{\myhypothesis{2} \myglosnl*{glo-Binome} plus performants sur les tâches complexes}
					Nous émettons l'hypothèse que plus la tâche est complexe et plus la configuration collaborative produira un gain significatif de performances par rapport à la configuration individuelle.
				\end{myparagraph}
				\begin{myparagraph}[par-met-exp2-ApprentissagePlusPerformantEnBinome]{\myhypothesis{3} Apprentissage plus performant en \myglosnl{glo-Binome}}
					Nous émettons l'hypothèse que le travail en collaboration augmente la vitesse d'apprentissage de la tâche.
					En effet, nous supposons que l'interaction sociale entre les partenaires va stimuler l'apprentissage et permettre l'échange et le partage de connaissances.
				\end{myparagraph}
				\begin{myparagraph}[par-met-exp2-LesSujetsPreferentLeTravailEnCollaboration]{\myhypothesis{4} Les sujets préfèrent le travail en collaboration}
					Nous souhaitons évaluer auprès des utilisateurs l'intérêt vis-à-vis du travail collaboratif dans ce contexte de collaboration étroite, notre hypothèse étant que les utilisateurs préfèrent le travail en collaboration.
					Rappelons que les sujets ne sont passés que dans une des conditions (\myglos{glo-Monome} ou \myglos{glo-Binome}); leur évaluation est effectuée à l'aveugle.
				\end{myparagraph}
			\end{mysubsection}
			\begin{mysubsection}[sse-met-exp2-Sujets]{Sujets}
				\mysummary{exp2-subjects.tex} avec une moyenne d'âge de \mysummary{exp2-age.tex} ont participé à cette expérimentation.
				Ils ont tous été recrutés au sein du laboratoire \myacro{acr-LIMSI} et sont chercheurs ou assistants de recherche dans les domaines suivants~:
				\begin{itemize}
					\item linguistique et traitement automatique de la parole;
					\item réalité virtuelle et système immersifs;
					\item audio-acoustique.
				\end{itemize}
				Ils ont tous le français comme langue principale.
				Aucun participant n'a de déficience visuelle (ou corrigée le cas échéant) ni de déficience audio.
				Les sujets ne sont pas rémunérés pour l'expérimentation.

				Chaque participant est complètement naïf concernant les détails de l'expérimentation.
				Une explication détaillée de la procédure expérimentale leur est donnée au commencement de l'expérimentation.
				Cependant, l'objectif de l'expérimentation n'est pas révélé.
			\end{mysubsection}
			\begin{mysubsection}[sec-met-exp2-Variables]{Variables}
				\begin{mysubsubsection}[sss-met-exp2-VariablesIndependantes]{Variables indépendantes}
					\begin{myparagraph}[par-met-exp2-NombreDeSujets]{\myvari{1} Nombre de sujets}
						C'est une \myglos{glo-VariableInterSujets}.
						\myvari{1} possède deux modalités : \og un sujet (\mycf \myemph{\myglos{glo-Monome}}) \fg ou \og deux sujets (\mycf \myemph{\myglos{glo-Binome}}) \fg.
						\mynum{12}~\myglos*{glo-Monome} et \mynum{12}~\myglos*{glo-Binome} sont testés.
					\end{myparagraph}
					\begin{myparagraph}[par-met-exp2-ComplexiteDeLaTache]{\myvari{2} Complexité de la tâche}
						C'est une \myglos{glo-VariableIntraSujets}.
						Quatre scénarios sont proposés, basés sur différents critères de complexité \myref*{sse-exp2-DescriptionDeLaTache}.
					\end{myparagraph}
					\begin{myparagraph}[par-met-exp2-LeNiveauDApprentissage]{\myvari{3} Le niveau d'apprentissage}
						C'est une \myglos{glo-VariableIntraSujets}.
						Tous les sujets sont confrontés trois fois à la même série de tâches (les scénarios \myscenario{1a}, \myscenario{1b}, \myscenario{2a} et \myscenario{2b}) sur trois jours successifs afin d'observer l'effet de l'apprentissage en \myglos{glo-Monome} et en \myglos{glo-Binome}.
					\end{myparagraph}
				\end{mysubsubsection}
				\begin{mysubsubsection}[sec-met-exp2-VariablesDependantes]{Variables dépendantes}
					\begin{myparagraph}[par-met-exp2-TempsDeRealisation]{\myvard{1} Temps de réalisation}
						C'est le temps total pour réaliser la tâche demandée, c'est-à-dire manipuler et déformer la molécule afin d'atteindre la conformation stable.
						Le temps est limité à \mynum[mn]{10}; au-delà de cette limite, l'application est arrêtée
					\end{myparagraph}
					\begin{myparagraph}[par-met-exp2-NombreDeSelections]{\myvard{2} Nombre de sélections}
						C'est le nombre de sélections réalisées durant chaque tâche.
						Une sélection est comptabilisée lorsqu'un atome ou un \myglos{glo-Residu} est sélectionné par un des deux \myglos{glo-EffecteurTerminal}.
					\end{myparagraph}
					\begin{myparagraph}[par-met-exp2-DistancePassiveEntreLesEspacesDeTravail]{\myvard{3} Distance passive entre les espaces de travail}
						Durant chaque tâche, la distance moyenne entre les deux \myglos*{glo-EffecteurTerminal} est mesurée.
						Elle est de l'ordre du centimètre.
					\end{myparagraph}
					\begin{myparagraph}[par-met-exp2-DistanceActiveEntreLesEspacesDeTravail]{\myvard{4} Distance active entre les espaces de travail}
						Basée sur la même mesure que \myvard{3}, elle n'est mesurée que lorsque les deux \myglos*{glo-EffecteurTerminal} sont en phase de sélection.
						Lorsqu'au moins un des deux \myglos*{glo-EffecteurTerminal} ne possède pas de sélection active, la distance n'est pas mesurée.
					\end{myparagraph}
					\begin{myparagraph}[par-met-exp2-VitesseMoyenne]{\myvard{5} Vitesse moyenne}
						Elle mesure la vitesse moyenne de chaque \myglos{glo-EffecteurTerminal}.
						Elle est calculée par intégration numérique des positions successives en fonction du temps.
					\end{myparagraph}
					\begin{myparagraph}[par-met-exp2-ReponsesQualitatives]{\myvard{6} Réponses qualitatives}
						Un questionnaire est proposé à tous les sujets (différent pour les \myglos*{glo-Monome} et les \myglos*{glo-Binome}).
						Le questionnaire soumis aux sujets est présenté dans la \myref{sec-q-SecondeExperimentation}.
					\end{myparagraph}
				\end{mysubsubsection}
			\end{mysubsection}
			\begin{mysubsection}[sse-met-exp2-Procedure]{Procédure}
				L'expérimentation débute par une étape d'entraînement pendant laquelle les sujets doivent déformer la molécule \myPrion vers son état stable (deux zones à déformer).
				Pendant cette phase, les outils sont introduits et expliqués un par un.
				Cette phase dure entre \mynum[mn]{5} et \mynum[mn]{10}.
				Chaque sujet a la possibilité de tester les outils et peut questionner l'expérimentateur.

				Lorsque la phase d'entraînement est terminée, les sujets sont confrontées aux scénarios \myscenario{1a} et \myscenario{1b}.
				Les scénarios sont alternés entre les groupes de sujets afin d'éviter les biais d'apprentissage.
				L'application s'arrête automatiquement lorsque le seuil \myacro{acr-RMSD} désiré est atteint.

				Dès que les scénarios \myscenario{1a} et \myscenario{1b} ont été achevés, les sujets sont confrontés aux scénarios \myscenario{2a} et \myscenario{2b} également de manière alternée.
				De la même façon, l'application s'arrête automatiquement lorsque le seuil \myacro{acr-RMSD} désiré est atteint ou lorsque les \mynum[mn]{10} sont atteintes.

				Tous les sujets sont confrontés trois fois à l'ensemble des quatre scénarios avec un jour d'intervalle entre chaque confrontation.
				L'objectif de cette multiple confrontation est l'étude de l'apprentissage en configuration collaborative.
			\end{mysubsection}
		\end{mysection}
		\begin{mysection}[sec-met-exp3-TroisiemeExperimentation]{Troisième expérimentation}
			\begin{mysubsection}[sse-met-exp3-Hypotheses]{Hypothèses}
				Cette nouvelle étude à pour objectif d'identifier les dynamiques de groupe.
				Nous allons observer l'évolution des performances en fonction du nombre de participants.
				\begin{myparagraph}[par-met-exp3-AmeliorationDesPerformancesEnQuadrinome]{\myhypothesis{1} Amélioration des performances en \myglosnl{glo-Quadrinome}}
					Nous émettons l'hypothèse que les performances des \myglos*{glo-Quadrinome} seront meilleures que les performances des \myglos*{glo-Binome}.
					Cette hypothèse est basée sur les conclusions obtenues dans la seconde étude à propos de la distribution des ressources.
				\end{myparagraph}
				\begin{myparagraph}[par-met-exp3-UnMeneurEmergeDansLesQuadrinomes]{\myhypothesis{2} Un \myglosnl{glo-Meneur} émerge dans les \myglosnl*{glo-Quadrinome}}
					D'après \mycite[author]{Bales-1950}, les groupes restreints voient émerger au moins un \myglos{glo-Meneur}, quelque soit la taille du groupe.
					Nous émettons l'hypothèse qu'un \myglos{glo-Meneur} émergera durant une collaboration étroitement couplée.
				\end{myparagraph}
				\begin{myparagraph}[par-met-exp3-LesQuadrinomesSeStructurentParLeBrainstorming]{\myhypothesis{3} Les \myglosnl*{glo-Quadrinome} se structurent par le \mybrainstorming}
					Dans cette expérimentation, nous étudions la mise en place d'une période de \mybrainstorming avant le début de la tâche.
					Nous émettons l'hypothèse que cette période de réflexion sera surtout mise à profit par les \myglos*{glo-Quadrinome}.
				\end{myparagraph}
			\end{mysubsection}
			\begin{mysubsection}[sse-met-exp3-Sujets]{Sujets}
				\mysummary{exp3-subjects.tex} avec une moyenne d'âge de \mysummary{exp3-age.tex} ont participé à cette expérimentation.
				Ils ont tous été recrutés au sein du laboratoire \myacro{acr-LIMSI} et sont étudiants, chercheurs ou assistants de recherche dans les domaines suivants~:
				\begin{itemize}
					\item linguistique et traitement automatique de la parole;
					\item réalité virtuelle et système immersifs;
					\item audio-acoustique.
				\end{itemize}
				Ils ont tous le français comme langue principale.
				Aucun participant n'a de déficience visuelle (ou corrigée le cas échéant) ni de déficience audio.
				Les sujets ne sont pas rémunérés pour l'expérimentation.

				Tous les participants de cette expérimentation ont été choisis car ils ont déjà une expérience sur la plateforme : les participants connaissent déjà les outils de déformation et l'environnement virtuel.
				L'objectif est de limiter les évolutions liées à l'apprentissage afin de pouvoir observer les évolutions de la dynamique de groupe.

				Chaque participant est complètement naïf concernant les détails de l'expérimentation.
				Une explication détaillée de la procédure expérimentale leur est donnée au commencement de l'expérimentation mais en omettant l'objectif de l'étude.
			\end{mysubsection}
			\begin{mysubsection}[sse-met-exp3-Variables]{Variables}
				\begin{mysubsubsection}[sss-met-exp3-VariablesIndependantes]{Variables indépendantes}
					\begin{myparagraph}[par-met-exp3-NombreDeSujets]{\myvari{1} Nombre de sujets}
						C'est une \myglos{glo-VariableIntraSujets}.
						\myvari{1} possède deux modalités : \og deux sujet (\mycf \myemph{\myglos{glo-Binome}}) \fg ou \og quatre sujets (\mycf \myemph{\myglos{glo-Quadrinome}}) \fg.
						\mynum{8}~\myglos*{glo-Binome} et \mynum{4}~\myglos*{glo-Quadrinome} sont testés.
					\end{myparagraph}
					\begin{myparagraph}[par-met-exp3-ComplexiteDeLaTache]{\myvari{2} Complexité de la tâche}
						C'est une \myglos{glo-VariableIntraSujets}.
						Deux tâches de déformation sont proposées et décrites dans la \myref{sse-exp3-DescriptionDeLaTache}.
					\end{myparagraph}
					\begin{myparagraph}[par-met-exp3-TempsAllouePourLeBrainstorming]{\myvari{3} Temps alloué pour le \mybrainstorming}
						C'est une \myglos{glo-VariableInterSujets}.
						\myvari{3} possède deux modalités : \og pas de \mybrainstorming \fg ou \og \mynum[mn]{1} de \mybrainstorming \fg.
						Cette période de \mybrainstorming est allouée avant le début de chaque tâche et permet une réflexion préalable au commencement de la tâche.
					\end{myparagraph}
				\end{mysubsubsection}
				\begin{mysubsubsection}[sec-met-exp3-VariablesDependantes]{Variables dépendantes}
					\begin{myparagraph}[par-met-exp3-TempsDeRealisation]{\myvard{1} Temps de réalisation}
						C'est le temps total que les sujets mettent pour réaliser la tâche demandée, c'est-à-dire manipuler et déformer la molécule afin d'atteindre l'objectif fixé.
						Le temps est limité à \mynum[mn]{10}.
					\end{myparagraph}
					\begin{myparagraph}[par-met-exp3-FrequenceDeSelections]{\myvard{2} Fréquence des sélections}
						C'est la fréquence des sélections réalisées durant chaque tâche.
						Une sélection est comptabilisée lorsqu'un atome est sélectionné par un des \myglos{glo-EffecteurTerminal}, pour chacun des sujets.
						La fréquence des sélections est le rapport du nombre de sélections par le temps de réalisation.
					\end{myparagraph}
					\begin{myparagraph}[par-met-exp3-VitesseMoyenne]{\myvard{3} Vitesse moyenne}
						Cette variable est une mesure de la vitesse moyenne de chaque \myglos{glo-EffecteurTerminal}.
						Elle est calculée par intégration numérique des positions successives en fonction du temps.
					\end{myparagraph}
					\begin{myparagraph}[par-met-exp3-ForceMoyenne]{\myvard{4} Force moyenne}
						La force appliquée sur les atomes durant la simulation est mesurée.
						C'est la force moyenne appliquée par un sujet lorsqu'un atome est sélectionné.
						Elle est exprimée en \myname{Newton}.
					\end{myparagraph}
					\begin{myparagraph}[par-met-exp3-CommunicationsVerbales]{\myvard{5} Communications verbales}
						L'enregistrement des communications verbales permet de mesurer le nombre d'interventions verbales de chacun des sujets.
						Deux catégories d'interventions sont distinguées :
						\begin{description}
							\item[Les observations] indiquent aux autres sujets une intention d'action ou informent sur l'état actuel de l'environnement;
							\item[Les ordres] sont donnés aux autres sujets afin qu'ils réalisent une action déterminée.
						\end{description}
					\end{myparagraph}
				\end{mysubsubsection}
			\end{mysubsection}
			\begin{mysubsection}[sse-met-exp3-Procedure]{Procédure}
				L'expérimentation débute par une étape d'entraînement avec pour objectif de déformer la molécule \myTRPCAGE vers un état stable.
				Pendant cette phase, les outils sont introduits et expliqués un par un.
				Les sujets ayant déjà réalisé une expérience sur la plateforme, cette phase est effectuée pour remémorer l'environnement et les outils.
				Cette phase dure entre \mynum[mn]{5} et \mynum[mn]{10}.
				Chaque sujet a la possibilité de tester les outils et peut questionner l'expérimentateur.

				Lorsque la phase d'entraînement est terminée, les sujets sont confrontées au scénario \myscenario{1}.
				Puis dans un second temps, le scénario \myscenario{2} leur est proposé.
				Pour chaque scénario, l'application s'arrête automatiquement lorsque le seuil \myacro{acr-RMSD} \myref*{sse-exp2-DescriptionDeLaTache} désiré est atteint.
				L'ordre de ces deux scénarios n'est pas contre-balancé sur les différents groupes de sujets.

				Tous les sujets sont confrontés aux deux scénarios deux fois.
				Une première fois en \myglos{glo-Binome} et une seconde fois en \myglos{glo-Quadrinome}.
				L'ordre de passage en \myglos{glo-Binome} et en \myglos{glo-Quadrinome} est alterné selon les groupes.

				L'enregistrement vidéo est démarré au début de la phase d'apprentissage pour chaque groupe.
				Il est arrêté à la fin du second scénario.
			\end{mysubsection}
		\end{mysection}
		\begin{mysection}[sec-met-exp4-QuatriemeExperimentation]{Quatrième expérimentation}
			\begin{mysubsection}[sse-met-exp4-Hypotheses]{Hypothèses}
				\begin{myparagraph}[par-met-exp4-PerformancesAmelioreesParLAssitanceHaptique]{\myhypothesis{1} Performances améliorées par l'assistance haptique}
					Nous émettons l'hypothèse que les performances de groupe seront meilleures lorsque les outils d'assistance haptique seront mis à disposition des utilisateurs.
					Les performances sont évaluées sur la qualité de la solution fournie.
				\end{myparagraph}
				\begin{myparagraph}[par-met-exp4-LAssistanceHaptiqueAmelioreLaCommunication]{\myhypothesis{2} L'assistance haptique améliore la communication}
					Nous émettons l'hypothèse que la communication et la coordination sera améliorée grâce aux nouveaux outils d'assistance haptique.
				\end{myparagraph}
				\begin{myparagraph}[par-met-exp4-LaPlateformeEstApprecieeDesExperts]{\myhypothesis{3} La plateforme est appréciée des experts}
					Lors de cette expérimentation, nous effectuons une analyse de l'utilisabilité du système.
					Nous émettons l'hypothèse que cette plateforme répondra à des critères minimum d'utilisabilité.
					Le test d'utilisabilité est basé sur l'échelle de notation \myacro{acr-SUS} proposée par \mycite[author]{Brooke-1996}.
				\end{myparagraph}
			\end{mysubsection}
			\begin{mysubsection}[sse-met-exp4-Sujets]{Sujets}
				\mysummary{exp4-subjects.tex} avec une moyenne d'âge de \mysummary{exp4-age.tex} ont participé à cette expérimentation.
				Ils ont été recrutés au sein du \myCNRSLIMSI et de l'\myacro{acr-IBPC}; ils sont étudiants, chercheurs ou assistants de recherche dans les domaines suivants :
				\begin{itemize}
					\item bio-informatique;
					\item linguistique et traitement automatique de la parole;
					\item réalité virtuelle et système immersifs;
					\item audio-acoustique.
				\end{itemize}
				Ils ont tous le français comme langue principale à l'exception d'un sujet néerlandais qui parle un français courant.
				Aucun participant n'a de déficience visuelle (ou corrigée le cas échéant) ni de déficience audio.
				Les sujets ne sont pas rémunérés pour l'expérimentation.

				Chaque participant est complètement naïf concernant les détails de l'expérimentation.
				Cependant, tous les sujets sont familiers avec la plateforme \myShaddock ou ont déjà eu l'occasion de manipuler des plateformes de manipulation interactive de molécules.
				Une explication détaillée de la procédure expérimentale leur est donnée au commencement de l'expérimentation mais l'objectif de l'étude n'est pas révélé.
			\end{mysubsection}
			\begin{mysubsection}[sec-met-exp4-Variables]{Variables}
				\begin{mysubsubsection}[sss-met-exp4-VariablesIndependantes]{Variables indépendantes}
					\begin{myparagraph}[par-met-exp4-PresenceDeLAssistance]{\myvari{1} Présence de l'assistance}
						C'est une \myglos{glo-VariableIntraSujets}.
						\myvari{1} possède deux modalités : \og sans assistance haptique \fg ou \og avec assistance haptique \fg.
						L'assistance haptique est ajoutée aux outils de manipulation, de désignation et de déformation afin d'améliorer l'interaction et la communication entre les sujets pendant la tâche.
					\end{myparagraph}
					\begin{myparagraph}[par-met-exp4-MoleculesADeformer]{\myvari{2} Molécules à déformer}
						C'est une \myglos{glo-VariableIntraSujets}.
						\myvari{2} concerne les cinq molécules ou complexes de molécules à assembler : \og \myTRPCAGE \fg, \og \myPrion \fg, \og \myUbiquitin \fg, \og \myTRPZIPPER \fg et \og \myNusENusG \fg.
						Parmi ces molécules, seules \myUbiquitin et \myNusENusG sont utilisées pour les mesures expérimentales.
						Les autres molécules sont utilisées au cours de l'entraînement.
					\end{myparagraph}
				\end{mysubsubsection}
				\begin{mysubsubsection}[sec-met-exp4-VariablesDependantes]{Variables dépendantes}
					\begin{myparagraph}[par-met-exp4-ScoreRMSDMinimum]{\myvard{1} Score \myacronl-{acr-RMSD} minimum}
						Un score \myacro{acr-RMSD} est calculé en temps-réel de la même façon que dans la seconde et la troisième expérimentation.
						Le score minimum atteint est étudié : il représente la meilleure solution trouvé au cours de la manipulation.
					\end{myparagraph}
					\begin{myparagraph}[par-met-exp4-TempsDuScoreRMSDMinimum]{\myvard{2} Temps du score \myacronl-{acr-RMSD} minimum}
						Les sujets ont \mynum[mn]{8} pour réaliser le meilleur score \myacro{acr-RMSD} possible.
						Cette variable représente le temps mis pour atteindre ce score minimum.
					\end{myparagraph}
					\begin{myparagraph}[par-met-exp4-NombreDeSelections]{\myvard{3} Nombre de sélections}
						\myvard{3} représente le nombre de sélections réalisées par les sujets.
						Une sélection est comptabilisée lorsque un atome est sélectionné par un des deux \myglos{glo-EffecteurTerminal}.
					\end{myparagraph}
					\begin{myparagraph}[par-met-exp4-TempsMoyenDUneSelection]{\myvard{4} Temps moyen d'une sélection}
						C'est le rapport entre le temps total et le nombre total de sélections réalisées durant la tâche.
						Il représente le temps moyen que dure une sélection effectuée par un opérateur.
					\end{myparagraph}
					\begin{myparagraph}[par-met-exp4-TempsMoyenPourAccepterUneCible]{\myvard{5} Temps moyen pour accepter une cible}
						C'est le temps mesuré entre le moment où le coordinateur effectue une désignation et le moment où l'opérateur accepte la désignation.
					\end{myparagraph}
					\begin{myparagraph}[par-met-exp4-TempsMoyenPourAtteindreUneCibleAcceptee]{\myvard{6} Temps moyen pour atteindre une cible acceptée}
						Lorsqu'une cible est acceptée par un opérateur, cette mesure représente le temps que met l'opérateur pour atteindre et sélectionner le \myglos{glo-Residu} désigné.
					\end{myparagraph}
					\begin{myparagraph}[par-met-exp4-NombreDeDesignationsAcceptees]{\myvard{7} Nombre de désignations acceptées}
						Parmi toutes les désignations proposées par le coordinateur, cette mesure comptabilise seulement celles qui ont donné lieu à une acceptation et donc à une déformation.
					\end{myparagraph}
					\begin{myparagraph}[par-met-exp4-VitesseMoyenne]{\myvard{8} Vitesse moyenne}
						Cette variable est une mesure de la vitesse moyenne de chaque \myglos{glo-EffecteurTerminal}.
						Elle est calculée par intégration numérique des positions successives en fonction du temps.
					\end{myparagraph}
					\begin{myparagraph}[par-met-exp4-TempsDesCommunicationsVerbalesParSujet]{\myvard{9} Temps des communications verbales par sujet}
						L'enregistrement audio permet de mesurer la quantité de temps de parole pendant chaque tâche de l'expérimentation.
					\end{myparagraph}
					\begin{myparagraph}[par-met-exp4-QuestionnaireDUtilisabiliteEtSurLaPresence]{\myvard{10} Questionnaire d'utilisabilité et sur la conscience}
						Un questionnaire \myacro{acr-SUS} est proposé à tous les sujets \myref*{sec-q-QuatriemeExperimentation}.
						Il nous permet d'obtenir un score d'utilisabilité compris entre \mynum{0} et \mynum{100}.
						De plus, un questionnaire sur la conscience périphérique est proposé après chaque modalité de la variable \myvard{1}.
					\end{myparagraph}
				\end{mysubsubsection}
			\end{mysubsection}
			\begin{mysubsection}[sse-met-exp4-Procedure]{Procédure}
				La procédure expérimentale se déroule en neuf phases bien distinctes.
				\begin{myparagraph}[par-met-exp4-Phase1-RepartitionDesRoles]{Phase~\mynum{1} : répartition des rôles}
					Pour commencer, avant de pénétrer dans la salle d'expérimentation, il est demandé aux sujets de choisir leurs rôles.
					Ce groupe qui est au début, une \myglos{glo-StructureInformelle}, doit se structurer avec un coordinateur et deux opérateurs.
					Les rôles sont expliqués de manières claires mais concise à ce stade de l'expérimentation.
					Chaque rôle est important et l'expérimentateur insiste sur ce point pour qu'aucun des rôles ne soit choisi par dépit.
					Puis les sujets sont amenés à se répartir les rôles entre eux.
					Une fois cette phase terminée, les sujets sont invités à pénétrer dans la salle d'expérimentation et à s'installer : le coordinateur se trouve au milieu et les opérateurs se trouvent de part et d'autre du coordinateur.
				\end{myparagraph}
				\begin{myparagraph}[par-met-exp4-Phase2-PresentationDesOutils]{Phase~\mynum{2} : présentation des outils}
					Avant de commencer cette phase, l'enregistrement vidéo est activé.
					La seconde phase est une phase d'entraînement sur la molécule \myTRPCAGE.
					Elle a pour objectif de présenter les outils de désignation et de déformation.
					Cette tâche relativement simple permet aux sujets de se familiariser avec les outils, la tâche à effectuer, les différentes informations visuelles ainsi que les outils d'évaluation (score \myacro{acr-RMSD}).
				\end{myparagraph}
				\begin{myparagraph}[par-met-exp4-Phase3-IntroductionDeLHaptique]{Phase~\mynum{3} : introduction de l'haptique}
					Cette troisième phase est une phase d'entraînement sur la molécule \myPrion.
					L'entraînement porte sur l'introduction des assistances haptiques (présentées dans la \myref{sss-exp4-OutilsDInteraction}) pour les outils de désignation et de déformation.
					De plus, cette seconde molécule d'entraînement permet de familiariser les sujets avec une tâche de nature plus complexe que la précédente.
				\end{myparagraph}
				\begin{myparagraph}[par-met-exp4-Phase4-OutilDeManipulation]{Phase~\mynum{4} : outil de manipulation}
					Cette nouvelle phase d'entraînement sur la molécule \myTRPZIPPER est destinée à introduire l'outil de manipulation qui sera utilisé par le coordinateur.
				\end{myparagraph}
				\begin{myparagraph}[par-met-exp4-Phase5-PremiereEtapeDEvaluation]{Phase~\mynum{5} : première étape d'évaluation}
					Cette première étape d'évaluation concerne les deux scénarios à réaliser (scénario~\myscenario{1} et scénario~\myscenario{2}) sur la molécule \myUbiquitin et le complexe de molécules \myNusENusG.
					En fonction des groupes et afin de contrebalancer la variable \myvari{1}, la première étape d'évaluation s'effectue avec ou sans assistance haptique.

					On présente le scénario~\myscenario{1} puis le scénario~\myscenario{2} toujours dans cet ordre.
					Pour le scénario~\myscenario{1}, seuls les outils d'orientation, de désignation et de déformation sont présents.
					Tous les outils sont proposés pour le scénario~\myscenario{2}.

					Au début de chaque scénario, une période de \mynum[mn]{1} de \mybrainstorming est proposée aux sujets pendant laquelle ils peuvent visualiser et explorer la molécule non soumise à la simulation.
					Ensuite, la phase de déformation avec simulation est exécutée.
					L'objectif est d'atteindre le score \myacro{acr-RMSD} le plus petit possible dans un temps limité à \mynum[mn]{8}.
					Les sujets peuvent décider de s'arrêter avant les \mynum[mn]{8} s'ils estiment ne pas pouvoir obtenir un meilleur score.
				\end{myparagraph}
				\begin{myparagraph}[par-met-exp4-Phase6-PremierePartieDuQuestionnaire]{Phase~\mynum{6} : première partie du questionnaire}
					Lorsque la première étape d'évaluation est terminée, une première partie du questionnaire est proposée aux sujets \myref*{sec-q-QuatriemeExperimentation}.
					La section à remplir dépend du premier passage : avec ou sans assistance haptique.
					Durant cette phase, il est demandé aux sujets de ne pas communiquer entre eux.
				\end{myparagraph}
				\begin{myparagraph}[par-met-exp4-Phase7-DeuxiemeEtapeDEvaluation]{Phase~\mynum{7} : deuxième étape d'évaluation}
					La deuxième étape d'évaluation est identique à la première excepté pour la variable \myvari{1}.
					Si les sujets ont été confrontés à une assistance haptique dans la première étape, alors la seconde étape s'effectuera sans assistance haptique et réciproquement.

					Durant cette deuxième étape, il n'y a pas de phase exploratoire étant donné que les sujets connaissent déjà la molécule.
					Seules les phases de déformation de \mynum[mn]{8} sont proposées (\myUbiquitin puis \myNusENusG).
				\end{myparagraph}
				\begin{myparagraph}[par-met-exp4-Phase8-LeQuestionnaire]{Phase~\mynum{8} : deuxième partie du questionnaire}
					La seconde partie du questionnaire est similaire à la première partie \myref*{sec-q-QuatriemeExperimentation}.
					Les mêmes questions sont abordées mais adaptées pour cette deuxième étape de l'évaluation.
					Durant cette phase, il est demandé aux sujets de ne pas communiquer entre eux.
				\end{myparagraph}
				\begin{myparagraph}[par-met-exp4-Phase9-LeQuestionnaire]{Phase~\mynum{9} : questionnaire d'utilisabilité}
					Pour terminer l'expérimentation, les sujets sont invités à remplir un questionnaire d'utilisabilité \myref*{sec-q-QuatriemeExperimentation}.
					Durant cette phase, il est demandé aux sujets de ne pas communiquer entre eux.
					Des informations permettant de caractériser le sujet (âge, sexe, main dominante, \myetc) sont également demandée à la fin du questionnaire.
					L'enregistrement vidéo est arrêté à la fin de cette phase.
				\end{myparagraph}
			\end{mysubsection}
		\end{mysection}
	\end{mychapter}
	\begin{mychapter}[cha-Questionnaires]{Questionnaires}
		\begin{mysection}[sec-q-PremiereExperimentation]{Première expérimentation}
			Le questionnaire proposé durant cette expérimentation est constitué de deux parties.
			La deuxième partie est exclusivement réservée aux \myglos*{glo-Binome} et n'était pas proposée au \myglos*{glo-Monome}.
			Ce questionnaire contient \mynum[pages]{5} (\mynum[pages]{3} pour les \myglos*{glo-Monome}).
			Les questions sont évaluées selon une échelle de \mycite[author]{Likert-1932} à cinq niveaux.
			\myinsertpdf[pdfpages={1-5}]{exp1-questionnaire}
		\end{mysection}
		\begin{mysection}[sec-q-SecondeExperimentation]{Seconde expérimentation}
			Le questionnaire proposé durant la seconde expérimentation est décliné en deux versions : une version pour les \myglos*{glo-Monome} et une version pour les \myglos*{glo-Binome}.
			Le questionnaire est soumis aux sujets oralement par l'expérimentateur et les réponses sont directement reportées dans une tableau.
			Il est constitué de plusieurs questions notées sur échelle de \mycite[author]{Likert-1932} à cinq niveaux.
			\begin{mysubsection}[sse-q-SecondeExperimentation-QuestionnairePourLesMonomes]{Questionnaire pour les \myglosnl*{glo-Monome}}
				Pour les \myglos*{glo-Monome}, le questionnaire est le suivant :
				\begin{enumerate}
					\item Vous êtes-vous senti efficace ?
					\item Pensez-vous que vous auriez été plus à l'aise seul avec un seul outil de déformation ?
					\item Pensez-vous que vous auriez été plus à l'aise avec un partenaire ?
					\item Quelle solution choisiriez-vous entre les trois configurations ?
				\end{enumerate}
			\end{mysubsection}
			\begin{mysubsection}[sse-q-SecondeExperimentation-QuestionnairePourLesBinomes]{Questionnaire pour les \myglosnl*{glo-Binome}}
				Chaque sujet dans un \myglos{glo-Binome} est interrogé séparement pour éviter que les réponses de l'un influence les réponses de l'autre.
				Pour les \myglos*{glo-Binome}, le questionnaire est le suivant :
				\begin{enumerate}
					\item Vous êtes-vous senti efficace ?
					\item Comment évalueriez-vous votre taux de communication\dots{}
						\begin{itemize}
							\item verbale ?
							\item gestuelle ?
							\item virtuelle ?
						\end{itemize}
					\item Vous sentez-vous utile dans le groupe (par opposition à pénalisant) ?
					\item Pensez-vous avoir une position de \myglos{glo-Meneur} dans le groupe ?
					\item Pensez-vous que vous auriez été plus à l'aise seul avec votre outil de déformation ?
					\item Pensez-vous que vous auriez été plus à l'aise seul avec deux outils de déformation ?
					\item Quelle solution choisiriez-vous entre les trois configurations ?
				\end{enumerate}

				Concernant les taux de communication, les communications verbales concernent tous les échanges, dialogues exposés par la voix.
				La communication gestuelle représente les gestes que les sujets peuvent effectuer dans le monde réel pour expliquer, désigner ou pour tout autre explication à son partenaire.
				Enfin, la communication virtuelle concerne les informations données au partenaire par l'intermédiaire de l'environnement virtuel (par exemple, une désignation avec le curseur).
			\end{mysubsection}
		\end{mysection}
		\begin{mysection}[sec-q-QuatriemeExperimentation]{Quatrième expérimentation}
			Le questionnaire proposé durant la quatrième et dernière expérimentation contient une traduction en français du questionnaire \myacro{acr-SUS} proposé par \mycite[author]{Brooke-1996}.
			Une explication détaillé de ce questionnaire se trouve dans la \myref{sse-q-QuatriemeExperimentation-LeQuestionnaireSUS}.
			Le questionnaire est soumis sous un format papier et chaque utilisateur est invité à y répondre seul, sans l'aide de ces partenaires.
			Il est constitué de plusieurs questions notées sur échelle de \mycite[author]{Likert-1932} à cinq niveaux.
			\myinsertpdf[pdfpages={1-5}]{exp4-questionnaire}
			\begin{mysubsection}[sse-q-QuatriemeExperimentation-LeQuestionnaireSUS]{Le questionnaire \myacronl-{acr-SUS}}
				\begin{mysubsubsection}[sss-q-QuatriemeExperimentation-LesQuestions]{Les questions}
					Le questionnaire \myacro{acr-SUS} est constitué de \mynum{10}~questions.
					Chaque question donne lieu à une réponse sur une échelle de \mycite[author]{Likert-1932} à cinq niveaux allant de \og Fortement en désaccord (score de \mynum{1}) \fg à \og Fortement en accord (score de \mynum{5}) \fg.
					Les questions sont les suivantes :
					\begin{enumerate}[label={Q\arabic*.},ref={Q\arabic*}]
						\item Je pense que j'utiliserai ce système fréquemment
						\item J'ai trouvé ce système inutilement complexe
						\item J'ai pensé que ce système était facile à utiliser
						\item Je pense qu'il me faudrait l'aide d'un technicien pour être capable d'utiliser ce système
						\item J'ai trouvé que les différentes fonctions de la plateforme étaient bien intégrées
						\item J'ai trouvé qu'il y avait trop d'incohérences dans cette plateforme
						\item Je pense que la plupart des gens apprendraient rapidement à utiliser cette plateforme
						\item J'ai trouvé le système très lourd à utiliser
						\item Je me sentais très confiant en utilisant cette plateforme
						\item J'aurai besoin d'apprendre beaucoup de choses avant de pouvoir utiliser cette plateforme
					\end{enumerate}
				\end{mysubsubsection}
				\begin{mysubsubsection}[sss-q-QuatriemeExperimentation-EvaluationDuScoreSUS]{Évaluation du score \myacronl-{acr-SUS}}
					Afin d'évaluer le score \myacro-{acr-SUS} à partir du questionnaire, chaque question doit avoir une note comprise entre \mynum{0} et \mynum{4}.
					Concernant les questions \mynum{1}, \mynum{3}, \mynum{5}, \mynum{7} et \mynum{9}, on prend le score compris en \mynum{1} et \mynum{5} auquel on enlève \mynum{1}.
					Concernant les questions \mynum{2}, \mynum{4}, \mynum{6}, \mynum{8} et \mynum{10}, on soustrait de \mynum{5} le score compris en \mynum{1} et \mynum{5}.
					Pour terminer, on multiplie par \mynum{2.5} la somme de l'ensemble des scores.
					Le score final obtenu est une note comprise entre \mynum{0} et \mynum{100}.
				\end{mysubsubsection}
				\begin{mysubsubsection}[sss-q-QuatriemeExperimentation-ExempleDeScoreSUS]{Exemple de score \myacronl-{acr-SUS}}
					Imaginons un questionnaire rempli de la façon suivante :
					\newcommand{\mySUSplus}[1]{%
						\fpRegSet{mySUSplus}{1}
						\fpRegSet{myscore}{#1}%
						\fpRegSub{myscore}{mySUSplus}%
						\fpRegRound{myscore}{0}%
						\fpRegGet{myscore}{\myscore}%
						réponse \textcolor{mygreen}{\mynum{#1}} $\Rightarrow$ score $\textcolor{mygreen}{\mynum{#1}} - 1 = \textcolor{myred}{\mynum{\myscore}}$%
					}
					\newcommand{\mySUSminus}[1]{%
						\fpRegSet{mylevel}{5}
						\fpRegSet{myscore}{#1}
						\fpRegSub{mylevel}{myscore}%
						\fpRegRound{mylevel}{0}%
						\fpRegGet{mylevel}{\myscore}%
						réponse \textcolor{mygreen}{\mynum{#1}} $\Rightarrow$ score $5 - \textcolor{mygreen}{\mynum{#1}} = \textcolor{myred}{\mynum{\myscore}}$%
					}
					\begin{enumerate}[label={Q\arabic*.},ref={Q\arabic*}]
						\item \mySUSplus{5}
						\item \mySUSminus{4}
						\item \mySUSplus{2}
						\item \mySUSminus{1}
						\item \mySUSplus{2}
						\item \mySUSminus{3}
						\item \mySUSplus{2}
						\item \mySUSminus{4}
						\item \mySUSplus{5}
						\item \mySUSminus{2}
					\end{enumerate}
					Le score total peut maintenant être calculé.
					\begin{displaymath}
						\left(\textcolor{myred}{\mynum{4}} + \textcolor{myred}{\mynum{1}} + \textcolor{myred}{\mynum{1}} + \textcolor{myred}{\mynum{4}} + \textcolor{myred}{\mynum{1}} + \textcolor{myred}{\mynum{2}} + \textcolor{myred}{\mynum{1}} + \textcolor{myred}{\mynum{1}} + \textcolor{myred}{\mynum{4}} + \textcolor{myred}{\mynum{3}}\right) \times \mynum{2.5} = \mynum{22} \times \mynum{2.5} = \textcolor{myred}{\mathbf{\mynum{55}}}
					\end{displaymath}
				\end{mysubsubsection}
			\end{mysubsection}
		\end{mysection}
	\end{mychapter}
\end{document}
