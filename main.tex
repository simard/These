\documentclass[myfrancais]{mythesis}
\usepackage{mydate}
\usepackage{mytodo}
\usepackage{graphicx}
\usepackage{caption}
\usepackage{pstricks}
\usepackage{pstricks-add}
\newcommand{\name}[1]{\textsc{#1}}
\newcommand{\mynum}[1]{\nombre{#1}}
\newcommand{\myhyp}[1]{$\left(\mathcal{H}_#1\right)$\xspace}
\newcommand{\mygroup}[1]{$\left(\mathcal{G}_#1\right)$\xspace}
\newcommand{\myvar}[2]{$\left(\mathcal{V}_{\mathrm{#1}#2}\right)$\xspace}
\newcommand{\mysubject}[1]{\textsc{p}\nombre{#1}}
\newcommand{\myvari}[1]{\myvar{i}{#1}}
\newcommand{\myvard}[1]{\myvar{d}{#1}}
\newcommand{\myresidue}[1]{$\left(\mathcal{R}_{#1}\right)$\xspace}
\newcommand{\myPDB}{\textsc{pdb}\xspace}
\newcommand{\myPDBbase}{\emph{Protein~DataBase}\xspace}
\newcommand{\myPDBlink}[2]{\href{#1}{\textsc{\MakeLowercase{#2}}}}
\newcommand{\TRPCAGE}{TRP-CAGE\xspace}
\newcommand{\TRPZIPPER}{TRP-ZIPPER\xspace}
\newcommand{\Prion}{Prion\xspace}
\newcommand{\Ubiquitin}{Ubiquitin\xspace}
\newcommand{\myemph}[1]{\emph{#1}}
\newcommand{\myThreeD}{\textsc{3d}\xspace}
\newcommand{\myregistered}{\textsuperscript{\textregistered}}
\newcommand{\mySpaceNavigator}{SpaceNavigator\myregistered\xspace}
\newcommand{\mycf}{\textit{c.f.}\xspace}
\newcommand{\myetc}{\textit{etc.}\xspace}
\newrgbcolor{myred}{0.7265625 0 0}
\newrgbcolor{mylightred}{0.9265625 0.500 0.500}
\newrgbcolor{mydarkblue}{0 0 0.2265625}
\newrgbcolor{myblue}{0 0 0.7265625}
\newrgbcolor{mylightblue}{0.500 0.500 0.9265625}
\newrgbcolor{mylightestblue}{0.7500 0.7500 0.9265625}
\newrgbcolor{mygreen}{0 0.7265625 0}
\newrgbcolor{mylightgreen}{0.500 0.9265625 0.500}
\newlength{\mywidth}
\newlength{\myheight}

\hypersetup{%
	pdftitle={Interactions haptiques collaboratives pour la manipulation moléculaire},%
	pdfauthor={Jean SIMARD},%
	pdfsubject={Mémoire de thèse en informatique},%
	pdfdisplaydoctitle=true,%
	pdflang={FR-fr}%
}

\title{Interactions haptiques collaboratives pour la manipulation moléculaire}
\author{Jean~\name{Simard}}
\documenttype{Thèse en Informatique}
\university{École Doctorale d'Informatique de Paris Sud}
\date{\mydate[datestyle=long]{01/12/2011}}
\jury{%
	Martin & \name{DUPONT} & (rapporteur) & Directeur de recherche au \name{LIMSI} \\
	Martin & \name{DUPOND} & (examinateur) & Directeur de recherche au \name{LIMSI}}

\addglobalbib[datatype=bibtex]{biblio.bib}

% AAA
\mynewacro{acr-AFM}%
{%
	name={\textsc{afm}},
	first={microscope à force atomique (\textsc{afm} pour \myemph{Atomic Force Microscope})},%
	plural={\textsc{afm}s},%
	firstplural={microscopes à force atomique (\textsc{afm} pour \myemph{Atomic Force Microscope})},%
	description={Microscope permettant l'observation de la topologie de la surface d'un échantillon au niveau atomique}
}
\mynewacro{acr-API}%
{%
	name={\textsc{api}},%
	first={interface de programmation (\textsc{api})},%
	plural={\textsc{api}s},%
	firstplural={interfaces de programmation (\textsc{api}s)},%
	description={\textsc{api} vient de l'anglais \myemph{Application Programming Interface} et désigne une interface avec un programme informatique}%
}
% BBB
% Be careful, because this word has no plural form, but the femina word in plural form
\mynewglos{glo-Bimanuel}%
{%
	name={bimanuel},%
	description={Qui se fait avec les deux mains},%
	plural={bimanuelle}%
}
\mynewglos{glo-Binome}%
{%
	name={binôme},%
	description={Groupe constitué de \mynum{2}~personnes},%
	plural={binômes}%
}
% CCC
\mynewacro{acr-CAO}%
{%
	name={\textsc{cao}},%
	first={conception assistée par ordinateur (\textsc{cao})},%
	description={La \textsc{cao} permet de concevoir et de tester virtuellement, à l'aide d'outils informatique, des produits manufacturés}%
}
\mynewglos{glo-ConflitDeCoordination}%
{%
	name={conflit de coordination},%
	description={Conflit entre deux sujets qui peut survenir lorsque les deux sujets tente d'accéder ou de déformer un objet au même instant},%
	plural={conflits de coordination}%
}
\mynewacro{acr-CUDA}%
{%
	name={\textsc{cuda}},%
	first={\textsc{cuda} (\myemph{Compute Unified Device Architecture})},%
	description={Technologie permettant d'utiliser l'unité graphique d'un ordinateur pour effectuer des calculs à hautes performances}%
}
\mynewglos{glo-Curseur}%
{%
	name={curseur},%
	description={Élément virtuel associé à un élément physique que le sujet manipule; il est lié à l'\myglos{glo-EffecteurTerminal}},%
	plural={curseurs}%
}
% DDD
\mynewacro{acr-DDL}%
{%
	name={\textsc{ddl}},%
	first={degré de liberté (\textsc{ddl})},%
	plural={\textsc{ddl}s},%
	firstplural={degrés de liberté (\textsc{ddl}s)},%
	description={Mouvements relatifs indépendants d'un solide par rapport à un autre}%
}
\mynewglos{glo-DockingMoleculaire}%
{%
	name={\myemph{docking} moléculaire},%
	description={Méthode permettant de déterminer l'orientation et la déformation optimale de \mynum{2}~molécules afin qu'elle s'assemble pour former un complexe de molécules stable},%
	plural={\myemph{docking} moléculaires}%
}
% EEE
\mynewglos{glo-EffecteurTerminal}%
{%
	name={effecteur terminal},%
	description={Élément physique que le sujet manipule; il est lié au \myglos{glo-Curseur} du monde virtuel},%
	plural={effecteurs terminaux}%
}
\mynewacro{acr-EVC}%
{%
	name={\textsc{evc}},%
	first={Environnement Virtuel Collaboratif (\textsc{evc})},%
	firstplural={Environnements Virtuels Collaboratifs (\textsc{evc})},%
	description={Ensemble logiciel et matériel permettant de faire interagir plusieurs utilisateurs au sein d'un même environnement; ils jouent un rôle important dans le développement de nouvelles méthodes de travail collaboratives}%
}
% HHH
\mynewglos{glo-Homoscedasticite}%
{%
	name={homoscedasticité},%
	description={Équivalent à homogénéité des variances; permet de comparer des variables aléatoires possédant des variances similaires},%
	plural={homoscedasticités}%
}
% III
\mynewacro{acr-IBPC}%
{%
	name={\textsc{ibpc}},%
	first={Institut de Biologie Physico-Chimie (\textsc{ibpc})},%
	description={Institut de recherche, géré par la fédération de recherche \textsc{frc}~\mynum{550}, étudiant les bases structurales, génétiques et physico-chimiques à leur différents niveaux d'intégration}%
}
\mynewacro{acr-IMD}%
{%
	name={\textsc{imd}},%
	first={\textsc{imd} (\myemph{Interactive Molecular Dynamics})},%
	description={Programme permettant de connecter le logiciel de visualisation moléculaire \myacro-{acr-VMD} avec le logiciel de simulation \myacro-{acr-NAMD} pour une simulation interactive en temps-réel \mycite{Stadler-1997}}%
}
\mynewacro{acr-ITAP}%
{%
	name={\textsc{itap}},%
	first={\myemph{Institut für Theoretische und Angewandte Physik} (\textsc{itap})},%
	description={Institut de Physique Théorique et Appliquée de \myname{Stuttgart} à l'origine du développement du logiciel \myacro{acr-IMD}}%
}
% LLL
\mynewacro{acr-LIMSI}%
{%
	name={\textsc{cnrs--limsi}},%
	first={Laboratoire pour l'Informatique, la Mécanique et les Sciences de l'Ingénieur (\textsc{cnrs--limsi})},%
	description={Unité Propre de Recherche du \textsc{cnrs} (\textsc{upr}~3251) associé aux universités \textsc{Paris} Sud et Pierre et Marie \textsc{Curie}}%
}
% MMM
\mynewglos{glo-Meneur}%
{%
	name={meneur},%
	description={En anglais \myemph{leader}, personne qui dirige un groupe afin d'atteindre des objectifs communs à ce groupe; c'est celui qui prend les décisions (voir aussi \myglos{glo-Suiveur})},%
	plural={meneurs}%
}
\mynewglos{glo-Monomanuel}%
{%
	name={monomanuel},%
	description={Qui se fait avec une main},%
	plural={monomanuelle}%
}
\mynewglos{glo-Monome}%
{%
	name={monôme},%
	description={\myemph{Groupe} constitué d'une unique personne},%
	plural={monômes}%
}
\mynewglos{glo-MotivationSociale}%
{%
	name={motivation sociale},%
	description={En anglais \myemph{social facilitation} \mycite{Triplett-1900}, phénomène de groupe où les personnes fournissent plus d'efforts grâce à la présence de partenaires},%
	plural={motivation sociale}%
}
\mynewacro{acr-TRM}%
{%
	name={\textsc{trm}},
	first={Théorie des Ressources Multiples (\textsc{trm})},%
	description={Cette théorie, élaborée par \mycite[author]{Wickens-1984} (\textsc{mrt} pour \myemph{Multiple Resource Theory}), propose un modèle pour la gestion des charges de travail pour un humain}
}
% NNN
\mynewacro{acr-NAMD}%
{%
	name={\textsc{namd}},%
	first={\textsc{namd} (\myemph{Scalable Molecular Dynamics})},%
	description={Programme de simulation pour la dynamique moléculaire \mycite{Phillips-2005}}%
}
% PPP
\mynewglos{glo-ParesseSociale}%
{%
	name={paresse sociale},%
	description={En anglais \myemph{social loafing} \mycite{Ringelmann-1913}, phénomène de groupe où les personnes fournissent moins d'effort pour la réalisation d'une tâche que s'ils effectuaient la tâche seuls},%
	plural={paresse sociale}%
}
\mynewacro{acr-PCV}%
{%
	name={\textsc{pcv}},
	first={Primitive Comportementale Virtuelle (\textsc{pcv})},%
	plural={\textsc{pcv}s},%
	firstplural={Primitives Comportementales Virtuelles (\textsc{pcv}s)},%
	description={Dans une application de réalité virtuelle, les activités d'un sujet peuvent toujours être décomposées en quatre comportements de base, appelés \myacro+{acr-PCV}, qui sont : observer, se déplacer, agir et communiquer \mycite{Fuchs-2006}}
}
% QQQ
\mynewglos{glo-Quadrinome}%
{%
	name={quadrinôme},%
	description={Groupe constitué de \mynum{4}~personnes},%
	plural={quadrinômes}%
}
% RRR
\mynewglos{glo-Residu}%
{%
	name={résidu},%
	description={Groupe d'atomes constituant un des blocs élémentaires d'une molécule},%
	plural={résidus}%
}
\mynewacro{acr-RMSD}%
{%
	name={\textsc{rmsd}},%
	first={\myemph{Root Mean Square Deviation} (\textsc{rmsd})},%
	description={Appelé Écart Quadratique Moyen en français, il permet -- dans le cadre de la biologie moléculaire -- de mesurer la différence entre deux déformations d'une même molécule}%
}
% SSS
\mynewglos{glo-StructureInformelle}%
{%
	name={structure informelle},%
	description={Groupe de personnes sans structures ni hiérarchie},%
	plural={structures informelles}%
}
\mynewglos{glo-Suiveur}%
{%
	name={suiveur},%
	description={En anglais \myemph{follower}, personne qui se laisse diriger dans un groupe afin d'atteindre des objectifs communs à ce groupe; c'est une personne qui ne prend pas de décision (voir aussi \myglos{glo-Meneur})},%
	plural={suiveurs}%
}
\mynewacro{acr-SUS}%
{%
	name={\textsc{sus}},%
	first={\textsc{sus} (\myemph{System Usability Scale})},%
	description={Échelle de notation entre \mynum{0} et \mynum{100} proposée par \mycite[author]{Brooke-1996} permettant d'évaluer l'utilisabilité d'un système}%
}
% TTT
\mynewglos{glo-Tetranome}%
{%
	name={tetranôme},%
	description={Groupe constitué de \mynum{4}~personnes},%
	plural={tetranômes}%
}
\mynewglos{glo-Trinome}%
{%
	name={trinôme},%
	description={Groupe constitué de \mynum{3}~personnes},%
	plural={trinômes}%
}
% UUU
\mynewacro{acr-UDP}%
{%
	name={\textsc{udp}},
	first={\textsc{udp} (\myemph{User Datagram Protocol} pour protocole de datagramme utilisateur)},%
	plural={\textsc{afm}s},%
	firstplural={\textsc{udp} (\myemph{User Datagram Protocol} pour protocole de datagramme utilisateur)},%
	description={c'est un des principaux protocole de télécommunication sur internet ; il a pour distinction de ne pas vérifier l'intégrité des données transmises}
}
\mynewacro{acr-UML}%
{%
	name={\textsc{uml}},%
	first={\textsc{uml} (\myemph{Unified Modeling Language})},%
	description={C'est un langage graphique de modélisation utilisé principalement en génie logiciel}%
}
% VVV
\mynewglos{glo-VariableDependante}%
{%
	name={variable dépendante},%
	description={Facteur mesuré sur une expérimentation (nombre de sélections, trajectoire, \myetc); ces variables sont influencées par les \myglos*{glo-VariableIndependante}},%
	plural={variables dépendantes}%
}
\mynewglos{glo-VariableIndependante}%
{%
	name={variable indépendante},%
	description={Facteur pouvant varier et être manipuler sur une expérimentation (nombre de participants, tâche, \myetc); ces variables vont avoir une incidence sur les \myglos*{glo-VariableDependante}},%
	plural={variables indépendantes}%
}
\mynewglos{glo-VariableInterSujets}%
{%
	name={variable inter-sujets},%
	description={Variables pour lesquelles les sujets sont confrontés à une et une seule des modalités de la variable},%
	plural={variables inter-sujets}%
}
\mynewglos{glo-VariableIntraSujets}%
{%
	name={variable intra-sujets},%
	description={Variables pour lesquelles les sujets sont confrontés à toutes les modalités de la variable},%
	plural={variables intra-sujets}%
}
\mynewacro{acr-VMD}%
{%
	name={\textsc{vmd}},%
	first={\textsc{vmd} (\myemph{Visual Molecular Dynamics})},%
	description={Programme de visualisation moléculaire \mycite{Humphrey-1996}}%
}
\mynewacro{acr-VRPN}%
{%
	name={\textsc{vrpn}},%
	first={\textsc{vrpn} (\myemph{Virtual Reality Protocol Network})},%
	description={Logiciel permettant de connecter différents périphériques de réalité virtuelle à une même application sous forme d'une architecture client/serveur \mycite{Taylor-II-2001}}%
}

\begin{document}
	\frontmatter
	\maketitle
	\mytoc
	\mylof
	\mylot
	\mylotodo
	\mainmatter
	\begin{mypart}{Le sujet}
		\begin{mychapter}{Introduction}
		\end{mychapter}
	\end{mypart}
	\begin{mypart}{Étude du travail collaboratif}
		\begin{mychapter}{La recherche collaborative}
			\begin{mysection}{Présentation}
				\begin{mysubsection}{Objectifs}
				\end{mysubsection}
				\begin{mysubsection}{Hypothèses}
				\end{mysubsection}
			\end{mysection}
			\begin{mysection}{Dispositif expérimental}
			\end{mysection}
			\begin{mysection}{Méthode}
				\begin{mysubsection}{Sujets}
					\mynum{24}~sujets (\mynum{4}~femmes et \mynum{20}~hommes) avec une moyenne d'âge de $\mu = 27.8$ ($\sigma = 7.19$) ont participés à cette expérimentation.
					Ils ont tous été recrutés au sein du laboratoire \myacro{acr-LIMSI} et sont chercheurs ou assistants de recherche dans les domaines suivants~:
					\begin{itemize}
						\item linguistique et traitement automatique de la parole;
						\item réalité virtuelle et système immersifs;
						\item audio-acoustique.
					\end{itemize}
					Ils ont tous le français comme langue principale.
					Aucun participant n'a de déficience visuelle (ou corrigée le cas échéant) ni déficience audio.

					Chaque participants est complètement naïf concernant les détails de l'expérimentation.
					Une explication détaillée de la procédure expérimentale leur est donnée au commencement de l'expérimentation mais en omettant l'objectif de l'étude.
				\end{mysubsection}
				\begin{mysubsection}{Variables}
					\begin{mysubsubsection}{Variables indépendantes}
						\begin{myparagraph}{\myvari{1} Nombre de sujets}
							La première \myglos{glo-VariableIndependante} est une \myglos{glo-VariableIntraPopulation}, c'est-à-dire que tous les sujets seront expérimentés dans toutes les conditions de cette variable.
							\myvari{1} possède \mynum{2}~valeurs possibles: \og \mynum{1}~sujet (\mycf \myemph{\myglos{glo-Monome}}) \fg ou \og \mynum{2}~sujets (\mycf \myemph{\myglos{glo-Binome}}) \fg.
							Le sujets seuls et les sujets en couples ont à leur disposition \mynum{2}~interfaces haptiques et une souris~\myThreeD (\mySpaceNavigator).
							Pour les \myglos*{glo-Binome}, seulement un des deux sujets est désigné pour l'utilisation exclusive de la souris~\myThreeD.
							\mynum{24}~\myglos*{glo-Monome} et \mynum{12}~\myglos*{glo-Binome} ont été testés ce qui fait deux fois plus de \myglos*{glo-Monome} que de \myglos*{glo-Binome}.
						\end{myparagraph}
						\begin{myparagraph}{\myvari{2} Résidu recherché}
							La seconde \myglos{glo-VariableIndependante} est une \myglos{glo-VariableIntraPopulation}.
							\myvari{2} concerne les \myglos*{glo-Residu} recherchés qui sont au nombre de \mynum{10} répartis à part égale dans \mynum{2}~molécules \mytableref*{tab-exp1-ListeDesResidusRecherches}.
							La première molécule est couramment nommée \TRPCAGE \mycite{Neidigh-2002} et a pour identifiant \myPDB \myPDBlink{http://www.rcsb.org/pdb/explore/explore.do?structureId=1L2Y}{1L2Y} sur la \myPDBbase\footnote{\url{http://www.pdb.org/}}.
							La seconde molécule nommée \Prion \mycite{Christen-2009} avec l'identifiant \myPDB \myPDBlink{http://www.rcsb.org/pdb/explore/explore.do?structureId=2KFL}{2KFL}.
							\mynum{5}~\myglos*{glo-Residu} sont présents sur chaque molécule \myfigureref*{fig-exp1-RepartitionDesResidusSurLesMolecules} et chacun présente différents niveaux de complexité \mytableref*{tab-exp1-ParametresDeComplexiteDesResidus} :
							\begin{description}
								\item[Position] La position du \myglos{glo-Residu} peut se trouver sur le pourtour de la molécule, en position \myemph{externe} ou à l'intérieur, au milieu de l'amas d'atome (position \myemph{interne}).
									Un \myglos{glo-Residu} en position externe ne nécessite pas de déformer la molécule pour le trouver et l'atteindre contrairement à un \myglos{glo-Residu} en position interne qui sera plus complexe d'accès.
								\item[Forme] La forme du \myglos{glo-Residu} influe énormément sur la complexité de la recherche.
									On distingue \mynum{3}~formes différentes :
									\begin{description}
										\item[Chaîne] Un enchaînement d'atomes seuls les atomes d'hydrogène sont de part et d'autres de cet enchaînement.
										\item[Cercle] Une chaîne d'atomes de carbone ou d'azote qui boucle sur elle-même.
										\item[Étoile] Séries de chaînes d'atomes toutes reliées sur un atome central (la plupart du temps, un atome de carbone).
									\end{description}
								\item[Couleurs] Les atomes sont colorés en fonction de leur nature (rouge pour l'oxygène, blanc pour l'hydrogène, \myetc).
									Les atomes qui sont rares seront donc rapidement trouvés grâce à leur couleur différente.
									Par contre, les atomes nombreux (comme les hydrogènes ou les carbones) seront plus difficiles à filtrer à cause de leur nombre important.
								\item[Similarité] Certains \myglos*{glo-Residu} à chercher sont très similaires à d'autres \myglos*{glo-Residu} également présents sur la molécule.
									De par leur similarité, ils vont mobilier la recherche sur des \myglos*{glo-Residu} incorrects.
							\end{description}

							\begin{mytable}
								\mycaption[tab-exp1-ListeDesResidusRecherches]{Liste des résidus recherchés}
								\setlength{\myheight}{10ex}
								\newcommand{\mypatternpicture}[1]{\myimage[width=\myheight]{exp1-#1}}
								\begin{mysubtable}
									\mysubcaption[tab-exp1-ListeDesResidusRecherches-ResidusSurLaMoleculeTRPCAGE]{Residus sur la molécule \TRPCAGE}
									\begin{mytabular}[0.49\textwidth]{^C-C}
										\mytoprule
										\myrowstyle{\bfseries}
										Résidu & Image \\
										\mymiddlerule
										\myresidue{1} & \mypatternpicture{pattern1} \\
										\myresidue{2} & \mypatternpicture{pattern2} \\
										\myresidue{3} & \mypatternpicture{pattern3} \\
										\myresidue{4} & \mypatternpicture{pattern4} \\
										\myresidue{5} & \mypatternpicture{pattern5} \\
										\mybottomrule
									\end{mytabular}
								\end{mysubtable}
								\begin{mysubtable}
									\mysubcaption[tab-exp1-ListeDesResidusRecherches-ResidusSurLaMoleculePrion]{Residus sur la molécule \Prion}
									\begin{mytabular}[0.49\textwidth]{^C-C}
										\mytoprule
										\myrowstyle{\bfseries}
										Résidu & Image \\
										\mymiddlerule
										\myresidue{6}  & \mypatternpicture{pattern6}  \\
										\myresidue{7}  & \mypatternpicture{pattern7}  \\
										\myresidue{8}  & \mypatternpicture{pattern8}  \\
										\myresidue{9}  & \mypatternpicture{pattern9}  \\
										\myresidue{10} & \mypatternpicture{pattern10} \\
										\mybottomrule
									\end{mytabular}
								\end{mysubtable}
							\end{mytable}

							\begin{myfigure}
								\newcommand{\schemafactor}{0.20}
								\newlength{\schemaunit}\setlength{\schemaunit}{\schemafactor\textwidth}
								\psset{unit=\schemaunit}
								\mycaption[fig-exp1-RepartitionDesResidusSurLesMolecules]{Répartition des résidus sur les molécules}
								\begin{pspicture}(-2.5,-3)(2.5,3)
									\rput(0,1.75){%
										\myimage[height=2\schemaunit,angle=90]{exp1-trp-cage}}
									\rput(0,-1.25){%
										\myimage[height=2\schemaunit,angle=90]{exp1-prion}}
									\rput(-1.5,2){%
										\myimage[height=\schemaunit]{exp1-pattern1}}
									\rput(1.5,2){%
										\myimage[width=\schemaunit]{exp1-pattern3}}
									\rput(1.5,-0){%
										\myimage[width=\schemaunit]{exp1-pattern2}}
									\rput(-1.5,-0){%
										\myimage[width=\schemaunit]{exp1-pattern4}}
									\rput(-1.5,-2){%
										\myimage[width=\schemaunit]{exp1-pattern5}}
									\rput(1.5,-2){%
										\myimage[height=\schemaunit]{exp1-pattern6}}

									\psset{framesize=1 1}
									\fnode(-1.5,2){P1}
									\uput[90](-1.5,2.5){\myresidue{1}}
									\fnode(1.5,2){P38}
									\uput[90](1.5,2.5){\myresidue{3} et \myresidue{8}}
									\fnode(1.5,-0){P27}
									\uput[90](1.5,0.5){\myresidue{2} et \myresidue{7}}
									\fnode(-1.5,-0){P49}
									\uput[90](-1.5,0.5){\myresidue{4} et \myresidue{9}}
									\fnode(-1.5,-2){P510}
									\uput[90](-1.5,-1.5){\myresidue{5} et \myresidue{10}}
									\fnode(1.5,-2){P6}
									\uput[90](1.5,-1.5){\myresidue{6}}

									\psset{linecolor=myred}
									\cnode(0.3,1.5){0.2}{TRPP1}
									\cnode(0.15,2){0.2}{TRPP38}
									\cnode(-0.1,1.25){0.2}{TRPP27}
									\cnode(-0.5,2.2){0.2}{TRPP49}
									\cnode(-0.65,1.25){0.2}{TRPP510}
									\ncline{-}{P1}{TRPP1}
									\ncline{-}{P38}{TRPP38}
									\ncline{-}{P27}{TRPP27}
									\ncline{-}{P49}{TRPP49}
									\ncline{-}{P510}{TRPP510}

									\psset{linecolor=myblue}
									\cnode(0.4,0.2){0.2}{PrionP38}
									\cnode(0.6,-2.8){0.2}{PrionP27}
									\cnode(0.2,-0.8){0.2}{PrionP49}
									\cnode(-0.7,-1.7){0.2}{PrionP510}
									\cnode(0.0,-1.4){0.2}{PrionP6}
									\ncline{-}{P38}{PrionP38}
									\ncline{-}{P27}{PrionP27}
									\ncline{-}{P49}{PrionP49}
									\ncline{-}{P510}{PrionP510}
									\ncline{-}{P6}{PrionP6}
								\end{pspicture}
							\end{myfigure}

							\begin{mytable}
								\mycaption[tab-exp1-ParametresDeComplexiteDesResidus]{Paramètres de complexité des résidus}
								\begin{mytabular}{^C-L-L-L-L}
									\mytoprule
									\myrowstyle{\bfseries}
									Résidu & Position & Forme & Couleurs & Similarité \\
									\mymiddlerule[\heavyrulewidth]
									\myresidue{1}  & Interne & Cercle & 8~C, 1~N & Non \\
									\mymiddlerule
									\myresidue{2}  & Interne & Étoile & 1~C, 3~N & Non \\
									\mymiddlerule
									\myresidue{3}  & Interne & Cercle & 6~C, 1~O & Non \\
									\mymiddlerule
									\myresidue{4}  & Externe & Chaîne & 4~C      & Non \\
									\mymiddlerule
									\myresidue{5}  & Externe & Chaîne & 4~C, 1~N & Non \\
									\mymiddlerule[\heavyrulewidth]
									\myresidue{6}  & Interne & Chaîne & 2~C, 2~S & Non \\
									\mymiddlerule
									\myresidue{7}  & Externe & Étoile & 1~C, 3~N & Non \\
									\mymiddlerule
									\myresidue{8}  & Externe & Cercle & 6~C, 1~O & Non \\
									\mymiddlerule
									\myresidue{9}  & Interne & Chaîne & 4~C      & Oui \\
									\mymiddlerule
									\myresidue{10} & Interne & Chaîne & 4~C, 1~N & Oui \\
									\mybottomrule
								\end{mytabular}
							\end{mytable}
						\end{myparagraph}
					\end{mysubsubsection}
					\begin{mysubsubsection}{Variables dépendantes}
						\begin{myparagraph}{\myvard{1} Le temps de complétion}
							Ce temps est le temps total pour réaliser la tâche demandée, c'est-à-dire trouver le \myglos{glo-Residu} et l'extraire de la molécule.
							Ce temps est divisé en \mynum{2}~phases bien distinctes :
							\begin{description}
								\item[La recherche] C'est la phase pendant laquelle les sujets cherchent le \myglos{glo-Residu}.
									Cette recherche peut être simplement visuelle en orientant et en déplaçant la molécule mais elle peut aussi amener les sujets à déformer la molécule afin d'explorer les \myglos{glo-Residu} inaccessibles.
								\item[La sélection] La phase de sélection débute dès l'instant où un des \mynum{2}~sujets a trouvé le \myglos{glo-Residu}.
									Elle est constitué d'une phase de sélection puis d'une phase d'extraction.
							\end{description}
							Il n'y a pas de limite de temps pour réaliser la tâche.
						\end{myparagraph}
						\begin{myparagraph}{\myvard{2} La distance entre les espaces de travail}
							Cette distance est la distance moyenne entre les \mynum{2}~\myglos*{glo-EffecteurTerminal} présents durant l'expérimentation.
							Cette distance représente donc une distance physique du monde réel, pas une distance virtuelle.
							Elle est de l'ordre du centimètre.
						\end{myparagraph}
						\begin{myparagraph}{\myvard{3} Les communications orales}
							L'enregistrement audio permet de mesurer la quantité de temps de parole pendant chaque tâche de l'expérimentation.
							Ces mesures discrimine la phase de recherche de la phase de sélection (voir \myvard{1}) comme indiqué plus précisément sur la \myfigureref{fig-exp1-SchemaDesPhasesDeLaCommunicationVerbale}.

							\begin{myfigure}
								\psset{unit=0.1\textwidth} % Fill entirely the page width
								\begin{pspicture}(0,-1.75)(10,1.5)
									\psset{linewidth=1pt,linecolor=black}%
									\psframe(0,-0.5)(10,0.5)%
									\psset{fillstyle=solid}%
									\psframe[fillcolor=mylightblue](0,-0.5)(6,0.5)%
									\psframe[fillcolor=mylightred](6,-0.5)(10,0.5)%
									\psbrace[ref=lC,rot=-90,nodesepA=-3,nodesepB=-0.25](6,0.5)(0,0.5){%
										\parbox{6\psxunit}{%
											\centering\textcolor{myblue}{Temps de recherche}%
										}%
									}%
									\psbrace[ref=lC,rot=-90,nodesepA=-2,nodesepB=-0.25](10,0.5)(6,0.5){%
										\parbox{4\psxunit}{%
											\centering\textcolor{myred}{Temps de sélection}%
										}%
									}%
									\psframe[fillcolor=myblue](1,-0.5)(1.5,0.5)
									\psframe[fillcolor=myblue](3,-0.5)(4.5,0.5)
									\psframe[fillcolor=myblue](4.8,-0.5)(5,0.5)
									\psframe[fillcolor=myred](6.5,-0.5)(7.5,0.5)
									\psframe[fillcolor=myred](8,-0.5)(8.25,0.5)
									\pnode(1.25,-0.5){verbal1}
									\pnode(3.75,-0.5){verbal2}
									\pnode(4.9,-0.5){verbal3}
									\pnode(7,-0.5){verbal4}
									\pnode(8.125,-0.5){verbal5}
									\rput(5,-1.5){%
										\Rnode{verbal}{%
											\psframebox[linestyle=none]{\centering Communication verbale}%
										}%
									}%
									\psset{linearc=0.1,angleA=-90}
									\ncdiagg{<-}{verbal1}{verbal}
									\ncdiagg{<-}{verbal2}{verbal}
									\ncdiagg{<-}{verbal3}{verbal}
									\ncdiagg{<-}{verbal4}{verbal}
									\ncdiagg{<-}{verbal5}{verbal}
								\end{pspicture}
								\mycaption[fig-exp1-SchemaDesPhasesDeLaCommunicationVerbale]{Schéma des phases de la communication verbale}
							\end{myfigure}
						\end{myparagraph}
						\begin{myparagraph}{\myvard{4} L'affinité entre les sujets}
							Le degré d'affinité -- concernant uniquement les \myglos*{glo-Binome} -- est compris entre \mynum{1} et \mynum{5} selon les critères suivants :
							\begin{enumerate}
								\item Les sujets ne se connaissent pas;
								\item Les sujets travaillent dans la même entreprise, le même laboratoire;
								\item Les sujets travaillent dans la même équipe;
								\item Les sujets travaillent dans le même bureau;
								\item Les sujets sont amis.
							\end{enumerate}
						\end{myparagraph}
					\end{mysubsubsection}
				\end{mysubsection}
				\begin{mysubsection}{Tâche}
				\end{mysubsection}
				\begin{mysubsection}{Procédure}
				\end{mysubsection}
			\end{mysection}
			\begin{mysection}{Résultats}
			\end{mysection}
		\end{mychapter}
		\begin{mychapter}{La manipulation collaborative}
		\end{mychapter}
		\begin{mychapter}{Les dynamiques de groupe}
		\end{mychapter}
	\end{mypart}
	\begin{mypart}{Propositions pour le travail collaboratif}
		\begin{mychapter}{Travail collaboratif assisté par haptique}
		\end{mychapter}
	\end{mypart}
	\begin{mypart}{Synthèse}
		\begin{mychapter}{Conclusion et perspectives}
		\end{mychapter}
	\end{mypart}

	\myglossary
	\appendix
	\begin{mychapter}{\textsc{Shaddock} -- Collaborative Virtual Environment for Molecular Design}
	\end{mychapter}
\end{document}
