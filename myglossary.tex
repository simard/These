% AAA
\mynewglos{glo-AmarrageMoleculaire}%
{%
	name={amarrage moléculaire},%
	description={Méthode permettant de déterminer l'orientation et la déformation optimale de \mynum{2}~molécules afin qu'elle s'assemble pour former un complexe stable},%
	plural={amarrages moléculaires}%
}
% BBB
% Be careful, because this word has no plural form, put the femina word in plural form
\mynewglos{glo-Bimanuel}%
{%
	name={bimanuel},%
	description={Qui se fait avec les deux mains},%
	plural={bimanuelle}%
}
\mynewglos{glo-Binome}%
{%
	name={binôme},%
	description={Groupe constitué de \mynum{2}~personnes},%
	plural={binômes}%
}
% CCC
\mynewacro{acr-EVC}%
{%
	name={\textsc{evc}},%
	first={Environnement Virtuel Collaboratif (\textsc{evc})},%
	firstplural={Environnements Virtuels Collaboratifs},%
	description={Ensemble logiciel et matériel permettant de faire interagir plusieurs utilisateurs au sein d'un même environnement; ils jouent un rôle important dans le développement de nouvelles méthodes de travail collaboratives}%
}
\mynewglos{glo-Curseur}%
{%
	name={curseur},%
	description={Élément virtuel associé à un élément physique que le sujet manipule; il est lié à l'\myglos{glo-EffecteurTerminal}},%
	plural={curseurs}%
}
% DDD
\mynewglos{glo-DockingMoleculaire}%
{%
	name={\myemph{docking} moléculaire},%
	description={Voir \myglos{glo-AmarrageMoleculaire}},%
	plural={\myemph{dockings} moléculaires}%
}
% EEE
\mynewglos{glo-EffecteurTerminal}%
{%
	name={effecteur terminal},%
	description={Élément physique que le sujet manipule; il est lié au \myglos{glo-Curseur} du monde virtuel},%
	plural={effecteurs terminaux}%
}
% LLL
\mynewacro{acr-LIMSI}%
{%
	name={\textsc{cnrs--limsi}},%
	first={Laboratoire pour l'Informatique, la Mécanique et les Sciences de l'Ingénieur (\textsc{cnrs--limsi})},%
	description={Unité Propre de Recherche du \textsc{cnrs} (\textsc{upr}\mynum{3251}) associé aux universités \textsc{Paris} Sud et Pierre et Marie \textsc{Curie}}%
}
% MMM
\mynewglos{glo-Monome}%
{%
	name={monôme},%
	description={\myemph{Groupe} constitué d'une unique personne},%
	plural={monômes}%
}
% RRR
\mynewglos{glo-Residu}%
{%
	name={résidu},%
	description={Groupe d'atomes constituant un des blocs élémentaires d'une molécule},%
	plural={résidus}%
}
% TTT
\mynewglos{glo-Tetranome}%
{%
	name={tetranôme},%
	description={Groupe constitué de \mynum{4}~personnes},%
	plural={tetranômes}%
}
% VVV
\mynewglos{glo-VariableDependante}%
{%
	name={variable dépendante},%
	description={Facteur mesuré sur une expérimentation (nombre de sélections, trajectoire, \myetc); ces variables sont influencées par les \myglos*{glo-VariableIndependante}},%
	plural={variables dépendantes}%
}
\mynewglos{glo-VariableIndependante}%
{%
	name={variable indépendante},%
	description={Facteur pouvant varier et être manipuler sur une expérimentation (nombre de participants, tâche, \myetc); ces variables vont avoir une incidence sur les \myglos*{glo-VariableDependante}},%
	plural={variables indépendantes}%
}
\mynewglos{glo-VariableInterPopulation}%
{%
	name={variable inter-population},%
	description={Variables pour lesquelles les sujets sont confrontés à une et une seule des modalités de la variable},%
	plural={variables inter-population}%
}
\mynewglos{glo-VariableIntraPopulation}%
{%
	name={variable intra-population},%
	description={Variables pour lesquelles les sujets sont confrontés à toutes les modalités de la variable},%
	plural={variables intra-population}%
}
