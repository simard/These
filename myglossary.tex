% AAA
\mynewglos{glo-AmarrageMoleculaire}%
{%
	name={amarrage moléculaire},%
	description={Méthode permettant de déterminer l'orientation et la déformation optimale de \mynum{2}~molécules afin qu'elle s'assemble pour former un complexe stable},%
	plural={amarrages moléculaires}%
}
\mynewacro{acr-API}%
{%
	name={\textsc{api}},%
	first={interface de programmation (\textsc{api})},%
	plural={\textsc{api}s},%
	firstplural={interfaces de programmation (\textsc{api}s)},%
	description={\textsc{api} vient de l'anglais \myemph{Application Programming Interface} et désigne une interface avec un programme informatique}%
}
% BBB
% Be careful, because this word has no plural form, put the femina word in plural form
\mynewglos{glo-Bimanuel}%
{%
	name={bimanuel},%
	description={Qui se fait avec les deux mains},%
	plural={bimanuelle}%
}
\mynewglos{glo-Binome}%
{%
	name={binôme},%
	description={Groupe constitué de \mynum{2}~personnes},%
	plural={binômes}%
}
% CCC
\mynewacro{acr-CAO}%
{%
	name={\textsc{cao}},%
	first={conception assistée par ordinateur (\textsc{cao})},%
	description={La \textsc{cao} permet de concevoir et de tester virtuellement, à l'aide d'outils informatique, des produits manufacturés}%
}
\mynewglos{glo-Curseur}%
{%
	name={curseur},%
	description={Élément virtuel associé à un élément physique que le sujet manipule; il est lié à l'\myglos{glo-EffecteurTerminal}},%
	plural={curseurs}%
}
% DDD
\mynewacro{acr-DDL}%
{%
	name={\textsc{ddl}},%
	first={degré de liberté (\textsc{ddl})},%
	plural={\textsc{ddl}s},%
	firstplural={degrés de liberté (\textsc{ddl}s)},%
	description={Mouvements relatifs indépendants d'un solide par rapport à un autre}%
}
\mynewglos{glo-DockingMoleculaire}%
{%
	name={\myemph{docking} moléculaire},%
	description={Voir \myglos{glo-AmarrageMoleculaire}},%
	plural={\myemph{dockings} moléculaires}%
}
% EEE
\mynewglos{glo-EffecteurTerminal}%
{%
	name={effecteur terminal},%
	description={Élément physique que le sujet manipule; il est lié au \myglos{glo-Curseur} du monde virtuel},%
	plural={effecteurs terminaux}%
}
\mynewacro{acr-EVC}%
{%
	name={\textsc{evc}},%
	first={Environnement Virtuel Collaboratif (\textsc{evc})},%
	firstplural={Environnements Virtuels Collaboratifs},%
	description={Ensemble logiciel et matériel permettant de faire interagir plusieurs utilisateurs au sein d'un même environnement; ils jouent un rôle important dans le développement de nouvelles méthodes de travail collaboratives}%
}
% HHH
\mynewglos{glo-Homoscedasticite}%
{%
	name={homoscedasticité},%
	description={Équivalent à homogénéité des variances; permet de comparer des variables aléatoires possédant des variances similaires},%
	plural={homoscedasticités}%
}
% III
\mynewacro{acr-IBPC}%
{%
	name={\textsc{ibpc}},%
	first={Institut de Biologie Physico-Chimie (\textsc{ibpc})},%
	description={Institut de recherche, géré par la fédération de recherche \textsc{frc}~550, étudiant les bases structurales, génétiques et physico-chimiques à leur différents niveaux d'intégration}%
}
\mynewacro{acr-IMD}%
{%
	name={\textsc{imd}},%
	first={\textsc{imd} (\myemph{Interactive Molecular Dynamics})},%
	description={Programme permettant de connecter le logiciel de visualisation moléculaire \myacro-{acr-VMD} avec le logiciel de simulation \myacro-{acr-NAMD} pour une simulation interactive en temps-réel \mycite{Stadler-1997}}%
}
\mynewacro{acr-ITAP}%
{%
	name={\textsc{itap}},%
	first={\myemph{Institut für Theoretische und Angewandte Physik} (\textsc{itap})},%
	description={Institut de Physique Théorique et Appliquée de \myname{Stuttgart} à l'origine du développement du logiciel \myacro{acr-IMD}}%
}
% LLL
\mynewacro{acr-LIMSI}%
{%
	name={\textsc{cnrs--limsi}},%
	first={Laboratoire pour l'Informatique, la Mécanique et les Sciences de l'Ingénieur (\textsc{cnrs--limsi})},%
	description={Unité Propre de Recherche du \textsc{cnrs} (\textsc{upr}~3251) associé aux universités \textsc{Paris} Sud et Pierre et Marie \textsc{Curie}}%
}
% MMM
\mynewglos{glo-Monomanuel}%
{%
	name={monomanuel},%
	description={Qui se fait avec une main},%
	plural={monomanuelle}%
}
\mynewglos{glo-Monome}%
{%
	name={monôme},%
	description={\myemph{Groupe} constitué d'une unique personne},%
	plural={monômes}%
}
% NNN
\mynewacro{acr-NAMD}%
{%
	name={\textsc{namd}},%
	first={\textsc{namd} (\myemph{Scalable Molecular Dynamics})},%
	description={Programme de simulation pour la dynamique moléculaire \mycite{Phillips-2005}}%
}
% QQQ
\mynewglos{glo-Quadrinome}%
{%
	name={quadrinôme},%
	description={Groupe constitué de \mynum{4}~personnes},%
	plural={quadrinômes}%
}
% RRR
\mynewglos{glo-Residu}%
{%
	name={résidu},%
	description={Groupe d'atomes constituant un des blocs élémentaires d'une molécule},%
	plural={résidus}%
}
\mynewacro{acr-RMSD}%
{%
	name={\textsc{rmsd}},%
	first={\myemph{Root Mean Square Deviation} (\textsc{rmsd})},%
	description={Appelé Écart Quadratique Moyen en français, il permet -- dans le cadre de la biologie moléculaire -- de mesurer la différence entre deux déformations d'une même molécule}%
}
% TTT
\mynewglos{glo-Tetranome}%
{%
	name={tetranôme},%
	description={Groupe constitué de \mynum{4}~personnes},%
	plural={tetranômes}%
}
% VVV
\mynewglos{glo-VariableDependante}%
{%
	name={variable dépendante},%
	description={Facteur mesuré sur une expérimentation (nombre de sélections, trajectoire, \myetc); ces variables sont influencées par les \myglos*{glo-VariableIndependante}},%
	plural={variables dépendantes}%
}
\mynewglos{glo-VariableIndependante}%
{%
	name={variable indépendante},%
	description={Facteur pouvant varier et être manipuler sur une expérimentation (nombre de participants, tâche, \myetc); ces variables vont avoir une incidence sur les \myglos*{glo-VariableDependante}},%
	plural={variables indépendantes}%
}
\mynewglos{glo-VariableInterSujets}%
{%
	name={variable inter-sujets},%
	description={Variables pour lesquelles les sujets sont confrontés à une et une seule des modalités de la variable},%
	plural={variables inter-sujets}%
}
\mynewglos{glo-VariableIntraSujets}%
{%
	name={variable intra-sujets},%
	description={Variables pour lesquelles les sujets sont confrontés à toutes les modalités de la variable},%
	plural={variables intra-sujets}%
}
\mynewacro{acr-VMD}%
{%
	name={\textsc{vmd}},%
	first={\textsc{vmd} (\myemph{Visual Molecular Dynamics})},%
	description={Programme de visualisation moléculaire \mycite{Humphrey-1996}}%
}
\mynewacro{acr-VRPN}%
{%
	name={\textsc{vrpn}},%
	first={\textsc{vrpn} (\myemph{Virtual Reality Protocol Network})},%
	description={Logiciel permettant de connecter différents périphériques de réalité virtuelle à une même application sous forme d'une architecture client/serveur \mycite{Taylor-II-2001}}%
}
