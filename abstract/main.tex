\documentclass[10pt,myfrancais]{article}
\usepackage[utf8]{inputenc}
\usepackage[T1]{fontenc}
\usepackage[frenchb]{babel}
\usepackage[explicit]{titlesec}
\usepackage{titling}
\usepackage{myfloat}
\usepackage{mymacro}
\usepackage{mycolor}
\usepackage{myheadfoot}
\usepackage{myps}

\makeatletter
\setlength{\parindent}{0cm}
\setlength{\droptitle}{-16ex}
\posttitle{\par\end{center}}
\postauthor{\end{tabular}\par\end{center}\vspace{-8ex}}
\titleformat{\section}[block]%
	{\color{mydarkblue!70}}%
	{\gdef\sectionlabel{\thesection\ }}%
	{0pt}%
	{%
		\Large%
		\psset{xunit=\textwidth,yunit=3ex}%
		\begin{myps}(0,0)(1,1)
			\psframe[framearc=.5,linestyle=none,fillstyle=gradient,gradbegin=gray,gradend=white,gradlines=100,gradmidpoint=1,gradangle=90,GradientCircle=true,GradientScale=1.5,GradientPos={(-0.01,0)}](-0.025,0)(1,1)%
			\rput[l](0,0.5){\vphantom{Ép}\bfseries\sffamily #1}%
		\end{myps}%
	}
\titlespacing{\section}{0pt}{8pt}{2pt}
\pagestyle{empty}
\let\ps@plain\ps@empty
\makeatother

\title{\bfseries\sffamily Résumé de thèse}
\author{Jean \myname{Simard}}
\date{}

\begin{document}
	\maketitle
	\section*{Discipline}
	Informatique
	\section*{Encadrement}
	\begin{mytabular}{@{}>{\bfseries}l@{: }l}
		Directeur & Philippe \myname{Tarroux} \\
		Encadrant & Mehdi \myname{Ammi} \\
	\end{mytabular}
	\section*{École doctorale}
	\begin{mytabular}{@{}l}
		\myrowstyle{\bfseries}
		École Doctorale d'Informatique de \myname{Paris}-Sud \\
		UFR des Sciences d'\myname{Orsay} \\
		Bâtiment 650 -- Aile nord -- \mynum{417} \\
		\mynum{91400} \myname{Orsay} \\
	\end{mytabular}
	\section*{Laboratoire d'accueil}
	\begin{mytabular}{@{}l}
		\myrowstyle{\bfseries}
		\textsc{cnrs--limsi}\\
		Bâtiment \mynum{508}, \mynum{502}bis, \mynum{512} et bâtiment \textsc{s} \\
		\mynum{91400} \myname{Orsay} \\
	\end{mytabular}
	\section*{Intitulé du sujet}
	Collaboration haptique étroitement couplée pour la déformation moléculaire interactive
	\section*{Résumé du sujet}
	Le \myemph{docking} moléculaire est une tâche complexe, difficile à appréhender pour une personne seule.
	C'est pourquoi, nous nous proposons d'étudier la distribution cognitive des charges de travail à travers la collaboration.
	Une plate-forme distribuée de déformation moléculaire interactive a été mise en place afin d'étudier les avantages mais aussi les limites et les contraintes du travail collaboratif étroitement couplé.
	Cette première étude, basée sur trois expérimentations, a permis de valider l'intérêt d'une approche collaborative pour des tâches complexes à fort couplage.
	Cependant, elle a mis en évidence des conflits de coordination ainsi que des problématiques liées à la dynamique d'un groupe.
	Suite à cette première étude, des outils d'assistance et de communication haptiques ont été proposés afin d'améliorer la communication et les interactions entre les différents collaborateurs.
	Une dernière expérimentation avec des bio-informaticiens a permis de montrer l'utilité de la communication haptique pour le travail collaboratif sur des tâches complexes à fort couplage.
\end{document}
